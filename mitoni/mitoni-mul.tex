\documentclass[12pt,twoside]{book}
\usepackage[a5paper,vmargin={2cm,2cm},hmargin={2cm,2cm}]{geometry}
\usepackage{graphicx}
\usepackage{marvosym}
\usepackage{palatino}
\usepackage{fancyhdr}
\usepackage{microtype}
\usepackage{xspace}

\renewcommand{\footrulewidth}{0.5pt}
\lhead[\fancyplain{}{\thepage}]%
      {\fancyplain{}{\rightmark}}
\rhead[\fancyplain{}{\leftmark}]%
      {\fancyplain{}{\thepage}}
\pagestyle{fancy}
\lfoot[\emph{Mitoni dan pemberkatan rumah Keluarga \keluarga}]{}
\rfoot[]{\emph{Mitoni dan pemberkatan rumah Keluarga \keluarga}}
\cfoot{}

\makeatletter
\newcommand{\judul}[1]{%
  {\parindent \z@ \centering \normalfont
    \interlinepenalty\@M \large \bfseries #1\par\nobreak \vskip 20\p@ }}
\newcommand{\subjudul}[1]{%
  {\parindent \z@ \normalfont
    \interlinepenalty\@M \bfseries #1\par\nobreak \vskip 20\p@ }}
\newcommand{\lagu}[1]{%
  {\parindent \z@ \normalfont
    \interlinepenalty\@M \bfseries \emph{#1}\par\nobreak \vskip 20\p@ }}

\renewenvironment{description}
               {\list{}{\labelwidth\z@ \itemindent-\leftmargin
                        \let\makelabel\descriptionlabel}}
               {\endlist}
\renewcommand*\descriptionlabel[1]{\hspace\labelsep 
                                \normalfont\bfseries #1 }
    
\newcommand{\doa}[2]{%
  \begin{description}
  \item[Doa untuk #1] #2
   
   Kami mohon : Kabulkanlah doa kami ya Tuhan.
  \end{description}
}

\newcommand{\bait}[1]{%
  \begin{enumerate}
  \slshape
  \setcounter{enumi}{\value{urut}}
  \item #1
  \setcounter{urut}{\value{enumi}}
  \end{enumerate}	
}
  

  
\makeatother

\newcommand{\BU}[1]{\begin{itemize} \item[U:] #1 \end{itemize}}
\newcommand{\BI}[1]{\begin{itemize} \item[I:] #1 \end{itemize}}
\newcommand{\BP}[1]{\begin{itemize} \item[P:] #1 \end{itemize}}
\newcommand{\keluarga}{Lazarus Sardju\xspace}
\newcommand{\calonibu}{Heriberta Anggun Hayuningtyasmara\xspace}
\newcommand{\calonayah}{Antonius Budi Praptono\xspace}
\newcommand{\romo}{Romo?\xspace}
\usepackage[bahasa]{babel}
\selectlanguage{bahasa}

\title{Ekaristi \vspace{1cm}\\MITONI\\ 
Ibu \calonibu \\ 
}

\author{
{~}\vspace{4cm}
oleh Romo \romo
} 
\date{6 Februari 2016}


\begin{document}
\maketitle
\Large  
\thispagestyle{empty}
{~}\newpage
\thispagestyle{empty}
\judul{RITUS PEMBUKA}
\lagu{Lagu Pembuka}
\small
\begin{center}
\itshape{Sungai Mengalir}
\end{center}

\begin{verse}
\itshape{
Sungai mengalir tiada henti-hentinya,\\ 
memberi hidup di sekitarnya.\\
Tuhan melimpahkan RahmatNya \\
bagi yang percaya kepadaNya.\\
{~}\\
Bunga-bunga tiada akan mekar mewangi,\\ 
jika tanpa disegarkan air.\\
Hidup akan menjadi hampa, \\
jika tanpa Cinta Kasih Tuhan.\\
{~}\\
Ya Tuhan Allah limpahkan Kasih SayangMu,\\ 
bagaikan air sungai abadi,\\
agar segarlah hidup kami, \\
tiada akan layu selamanya.\\
}
\end{verse}
\normalsize


\subjudul{Salam pembuka}

\BI{Demi Nama Bapa dan Putera dan Roh Kudus}
\BU{Amin.}
\BI{Semoga Allah Bapa serta PuteraNya, Tuhan kita Yesus Kristus, memberikan Kurnia dan Kesejahteraan kepada kita.}
\BU{Sekarang dan selama-lamanya.}

\subjudul{Pengantar}

\BI{Saudara-saudara terkasih. 

Upacara mitoni atau saat usia kandungan menginjak tujuh bulan, mempunyai makna yang dalam, bagi mayoritas masyarakat Jawa. Angka 7 dalam bahasa Jawa disebut \textit{pitu}. Maka dari itu makna di balik upacara itu adalah, pihak keluarga, suami dan ibu yang mengandung tersebut, ingin minta \textit{pitulungan} atau pertolongan kepada Tuhan Sang pencipta segala kehidupan. Dengan harapan jika tiba saatnya melahirkan nanti, sang ibu dan bayinya diberi keselamatan dan kesehatan dari-Nya.

Kedatangan Allah ke dunia menegaskan bahwa dunia ini telah dibebaskanNya dari Kuasa Kegelapan, dan sekarang dikuasai oleh Kuasa Allah. Manusia dilepaskan dari belenggu dan kuasa Si Jahat, dan sekarang kembali kepada pemiliknya semula yaitu Penciptanya. Sekarang ini Allah-lah yang berkuasa dan meraja di setiap jengkal tanah di dunia ini, dan di setiap sudut kehidupan manusia. Maka hari ini rumah kita, kita serahkan kepadaNya agar diberkatiNya. Salib Kristus akan kita pasang di kamar tamu. Semua itu untuk menyatakan Iman kita. Sekali Allah telah membiarkan Diri dipaku di bumi ini, untuk selamanya Dia bersatu dengan kita tanpa dapat dipisahkan lagi.}

\subjudul{Tobat}
\BI{Marilah kita hening sejenak untuk mempersiapkan diri dalam perayaan syukur ini sambil menyadari bahwa kita sering melupakan kebaikan Tuhan dan enggan mewartakan dan mewujudkan kebaikan tersebut melalui pikiran, perkataan, dan perbuatan kita.}

\BI{Saya mengaku}

\BU{Kepada Allah yang Maha Kuasa dan kepada saudara sekalian bahwa saya telah berdosa dengan pikiran dan perkataan, dengan perbuatan dan kelalaian. Saya berdosa, saya berdosa, saya sungguh berdosa. Oleh sebab itu saya mohon kepada Santa Perawan Maria, kepada Para Malaikat dan orang kudus dan kepada saudara sekalian, supaya mendoakan saya kepada Allah Tuhan kita.}

\BI{Semoga Allah Yang Maha Kuasa mengasihi kita, mengampuni dosa kita dan menghantar kita ke hidup yang kekal.}

\BU{Amin}

\lagu{Tuhan Kasihanilah Kami}

\subjudul{Doa Pembuka}

\BI{Marilah berdoa

Allah Pencipta dan Pemelihara kami yang mahakuasa dan kekal. Kami memuji dan meluhurkan Dikau atas segala Kasih karunia yang telah Kau curahkan kepada kami. Terlebih kami bersyukur kepadaMu, karena Engkau telah berkenan membebaskan dunia ini dari Kuasa Kegelapan, bahkan Engkau sudi datang dan tinggal di tengah-tengah kami untuk selama-lamanya. Sebagai ungkapan iman dan ucapan syukur kami, kami menyerahkan diri dan seisi rumah kami ke dalam kekuasaanMu yang manis. Ambillah diri kami. Milikilah kami. Kuasailah. Dan simpanlah kami dalam perlindungan hatiMu yang kudus. Kami percaya : dalam persatuan dengan Dikau, kami selamat dan sejahtera. Dalam persatuan dengan RahmatMu : kami akan bertambah baik hati dan suci. Semakin hari semakin menjadi putera-puteri kesayanganMu. Menjadi alat keselamatanMu yang mampu memberikan kebahagiaan dan kesucian kepada siapa pun yang kami jumpai, dan mampu menghapus kesusahan dan dosa dari lingkungan kami. Ini kami haturkan demi keluhuran dan kemuliaanMu, dan demi keselamatan umatMu. Dengan perantaraan PuteraMu, Tuhan kami Yesus Kristus, yang hidup dan berkuasa bersama Dikau dan Roh Kudus, kini dan sepanjang segala masa.
Amin.}

\judul{LITURGI SABDA}

\subjudul{Bacaan Kitab Suci}

Bacaan dari Kitab Keluaran (32:7-14)

\BP{Berfirmanlah TUHAN kepada Musa: "Pergilah, turunlah, sebab bangsamu yang kaupimpin keluar dari tanah Mesir telah rusak lakunya.
Segera juga mereka menyimpang dari jalan yang Kuperintahkan kepada mereka; mereka telah membuat anak lembu tuangan, dan kepadanya mereka sujud menyembah dan mempersembahkan korban, sambil berkata: Hai Israel, inilah Allahmu yang telah menuntun engkau keluar dari tanah Mesir."

Lagi firman TUHAN kepada Musa: "Telah Kulihat bangsa ini dan sesungguhnya mereka adalah suatu bangsa yang tegar tengkuk.
Oleh sebab itu biarkanlah Aku, supaya murka-Ku bangkit terhadap mereka dan Aku akan membinasakan mereka, tetapi engkau akan Kubuat menjadi bangsa yang besar."

Lalu Musa mencoba melunakkan hati TUHAN, Allahnya, dengan berkata: "Mengapakah, TUHAN, murka-Mu bangkit terhadap umat-Mu, yang telah Kaubawa keluar dari tanah Mesir dengan kekuatan yang besar dan dengan tangan yang kuat?

Mengapakah orang Mesir akan berkata: Dia membawa mereka keluar dengan maksud menimpakan malapetaka kepada mereka dan membunuh mereka di gunung dan membinasakannya dari muka bumi? Berbaliklah dari murka-Mu yang bernyala-nyala itu dan menyesallah karena malapetaka yang hendak Kaudatangkan kepada umat-Mu.

Ingatlah kepada Abraham, Ishak dan Israel, hamba-hamba-Mu itu, sebab kepada mereka Engkau telah bersumpah demi diri-Mu sendiri dengan berfirman kepada mereka: Aku akan membuat keturunanmu sebanyak bintang di langit, dan seluruh negeri yang telah Kujanjikan ini akan Kuberikan kepada keturunanmu, supaya dimilikinya untuk selama-lamanya."
Dan menyesallah TUHAN karena malapetaka yang dirancangkan-Nya atas umat-Nya.

Demikian Sabda Tuhan}

\BU{Syukur kepada Allah}

\lagu{Lagu Pengantar Bacaan}
\small
\begin{center}
\itshape{Bahagia manusia - MB 214}
\end{center}
\begin{verse}
\itshape{
Bahagia manusia\\
Yang tidak tuli hatinya\\
Yang mendengarkan sabda Bapa\\
Tekun melaksanakannya\\
Sabda Tuhan penuh daya\\
Yang tersesat dipanggilNya\\
DisembuhkanNya yang luka\\
Yang mati dihidupkanNya\\
{~}\\
Bahagia manusia\\
Yang menerima Sang sabda\\
Sabda yang sudah menjelma\\
Dalam wujud manusia\\
Terpujilah oh Sang Kristus\\
Sabda kekal dan penebus\\
Kebenaran, kehidupan\\
Serta jalan keslamatan
}
\end{verse}
\normalsize


\subjudul{Injil}
\BI{Tuhan sertamu,}
\BU{Dan sertamu juga.}
\BI{Inilah Injil Yesus Kristus menurut Yohanes - 5:31--47}
\BU{Terpujilah Kristus}

\BI{Kalau Aku bersaksi tentang diri-Ku sendiri, maka kesaksian-Ku itu tidak benar;
ada yang lain yang bersaksi tentang Aku dan Aku tahu, bahwa kesaksian yang diberikan-Nya tentang Aku adalah benar.

Kamu telah mengirim utusan kepada Yohanes dan ia telah bersaksi tentang kebenaran;
tetapi Aku tidak memerlukan kesaksian dari manusia, namun Aku mengatakan hal ini, supaya kamu diselamatkan.

Ia adalah pelita yang menyala dan yang bercahaya dan kamu hanya mau menikmati seketika saja cahayanya itu.
Tetapi Aku mempunyai suatu kesaksian yang lebih penting dari pada kesaksian Yohanes, yaitu segala pekerjaan yang diserahkan Bapa kepada-Ku, supaya Aku melaksanakannya. Pekerjaan itu juga yang Kukerjakan sekarang, dan itulah yang memberi kesaksian tentang Aku, bahwa Bapa yang mengutus Aku.

Bapa yang mengutus Aku, Dialah yang bersaksi tentang Aku. Kamu tidak pernah mendengar suara-Nya, rupa-Nyapun tidak pernah kamu lihat,
dan firman-Nya tidak menetap di dalam dirimu, sebab kamu tidak percaya kepada Dia yang diutus-Nya.

Kamu menyelidiki Kitab-kitab Suci, sebab kamu menyangka bahwa oleh-Nya kamu mempunyai hidup yang kekal, tetapi walaupun Kitab-kitab Suci itu memberi kesaksian tentang Aku,
namun kamu tidak mau datang kepada-Ku untuk memperoleh hidup itu.

Aku tidak memerlukan hormat dari manusia.
Tetapi tentang kamu, memang Aku tahu bahwa di dalam hatimu kamu tidak mempunyai kasih akan Allah.

Aku datang dalam nama Bapa-Ku dan kamu tidak menerima Aku; jikalau orang lain datang atas namanya sendiri, kamu akan menerima dia.

Bagaimanakah kamu dapat percaya, kamu yang menerima hormat seorang dari yang lain dan yang tidak mencari hormat yang datang dari Allah yang Esa?
Jangan kamu menyangka, bahwa Aku akan mendakwa kamu di hadapan Bapa; yang mendakwa kamu adalah Musa, yaitu Musa, yang kepadanya kamu menaruh pengharapanmu.

Sebab jikalau kamu percaya kepada Musa, tentu kamu akan percaya juga kepada-Ku, sebab ia telah menulis tentang Aku.
Tetapi jikalau kamu tidak percaya akan apa yang ditulisnya, bagaimanakah kamu akan percaya akan apa yang Kukatakan?"

Demikianlah Injil Tuhan}

\BU{Terpujilah Kristus}


\subjudul{Homili}

\judul{PEMBERKATAN IBU YANG SEDANG MENGANDUNG}
{\itshape (Romo menumpangkan kedua tangan di atas kepala Ibu \calonibu yang sedang mengandung 7 bulan seraya berdoa:)}
\BI{Allah Bapa yang Maha Kasih, Engkau telah mengutus Putera-Mu sendiri untuk menjelma menjadi manusia dengan perantaraan Perawan Maria. Sampai saatnya tiba, Putera-Mu lahir selamat dan luput dari bahaya pembunuhan Herodes berkat pertolongan-Mu lewat bisikan para malaikat. Maka dari itu ya Bapa, dampingilah Ibu \calonibu yang telah mengandung puteranya selama 7 bulan ini, agar nanti juga selamat sampai saatnya melahirkan.}
\BU{Amin.}
\BI{Allah Bapa yang Maha Kasih, Engkaulah yang meniupkan nafas kehidupan bagi janin-janin yang tumbuh di dalam rahim seorang ibu. Dari hari ke hari karena kuasa dan kasih-Mu, makhluk lemah tak berdaya itu menjadi kuat, bertumbuh menjadi ciptaan yang sempurna secitra dengan diri-Mu sendiri. Maka dari itu ya Bapa, semoga makhluk ciptaan-Mu yang berusia 7 bulan ini semakin Engkau kuatkan agar nanti bisa menghirup udara kehidupan dalam dekapan kasih kedua orang tuanya.}
\BU{Amin.}
\BI{Ya Bapa, perkenan kami juga berdoa dengan perantaraan wanita yang dikandung tanpa dosa, wanita pilihan-Mu yang karena kuasa Roh Kudus mengandung Putera-Mu sendiri, yakni Bunda Maria.

Ya Bunda Maria, teladan para ibu yang bersahaja, engkau telah mengalami sendiri bagaimana mengandung seorang putra. Antara bahagia, was-was, takut namun penuh harap semua itu pernah Bunda rasakan. Karena itu ya Bunda Maria, dampingilah Ibu \calonibu agar hatinya mantap, tidak ada rasa takut maupun was-was saat menyongsong kelahiran anaknya nanti.} 
\BU{Amin.}

\BI{Ya Allah Bapa yang maha rahim, sumber pengharapan dan keselamatan kami, sekarang Ibu \calonibu dan bayi dalam kandungan kami serahkan ke dalam tangan-Mu. Rawatlah selalu melalui tangan-tangan-Mu agar ibu dan anaknya senatiasa sehat. Cintailah selalu agar ibu dan anaknya merasa tentram karena Engkau selalu berada di sisinya. 

Semua permohonan ini kami sampaikan kepada-Mu dengan perantaraan Putera-Mu terkasih Tuhan Yesus yang bersama dikau dan Roh Kudus hidup dan berkuasa sepanjang segala masa.}
\BU{Amin.}


\judul{PEMBERKATAN RUMAH}

\BI{Allah Pencipta, Pemelihara, dan Bapa keluarga kami. Pada hari ini kami menyerahkan diri, seluruh keluarga kami serta seisi rumah kami kepada kekuasaanMu yang manis. Sudilah memberkati kami \Cross ~Perkuatlah kuasa Yesus Kristus PuteraMu atas rumah kami, agar Dia kami-cintai dan taati, supaya di bawah perlindunganNya, kami aman, selamat dan sejahtera. Supaya Kristus menjadi Kepala keluarga kami, menolong kami mengendalikan pikiran, hati dan tingkah-laku kami. Agar kami selalu sadar akan status kami sebagai umat yang telah Kau tebus. Agar kami hidup seperti PuteraMu itu : baik hati, suci, bijaksana, sederhana, rukun, saling sayang-menyayangi, hormat-menghormati, dan tolong-menolong. Berkatilah ya Bapa \Cross ~agar jangan seorang pun dari kami menjauh dari padaMu karena dosa atau kekecewaan. Sebab Engkaulah satu-satunya sumber kehidupan dan kebahagiaan kami, sekarang dan selama-lamanya.
Amin.}


{\itshape Rumah diberkati. Sementara itu umat menyanyikan lagu \emph{DATANGLAH ROH MAHAKUDUS} dan setiap bait diselingi dengan doa.}


\newcounter{urut}
\bait{Datanglah Roh Mahakudus. \\Masuki hati umatMu. \\Sirami jiwa yang layu, \\dengan embun KurniaMu.}

\doa{kamar tamu}{%
Tuhan, berkatilah kamar tamu. Penuhilah ruangan ini dengan kesejukan sejati. Semoga para tamu yang memasuki rumah ini, membawa Berkah, damai dan cinta. Dan semoga mereka pulang dengan senang hati.}

\bait{Roh Cinta Bapa dan Putera.\\ Taburkanlah cinta mesra,\\ dalam hati manusia,\\ Cinta anak pada Bapa.}

\doa{kamar tidur}{
Tuhan, berkatilah kamar tidur. Semoga kamar ini dapat menjadi tempat istirahat yang nyaman, sehingga pulihlah segala kelelahan. Lindungilah penghuninya dari dosa dan bahaya yang mengancam mereka pada waktu mereka tidur.}

\bait{Datanglah Roh Mahakudus.\\ Bentara Cinta Sang Kristus.\\ Tolong kami jadi saksi,\\ membawa Cinta Ilahi.}

\doa{kamar kerja}{
Tuhan, berkatilah kamar kerja. Semoga segala yang direncanakan dan dikerjakan di tempat ini berkenan kepadaMu dan mendatangkan kesejahteraan bagi seluruh keluarga dan masyarakat.}

\bait{Roh Kristus, ajari kami\\ bahasa cinta ilahi. \\Satulah bangsa semua,\\ karena bahasa cinta.}

\doa{dapur}{
Tuhan, berkatilah dapur. Jauhkanlah dari bahaya kebakaran. Semoga rejeki yang disiapkan di tempat ini sungguh bermanfaat bagi perkembangan jasmani-rohani keluarga ini. Jauhkanlah keluarga ini dari kekurangan dan kelaparan, supaya mereka tetap hidup dengan tenteram dan aman sentausa.}

\bait{Cinta yang laksana api,\\ kobarkan semangat kami,\\ agar musnahlah terbasmi\\ jiwa angkuh, hati dengki.}

\doa{sumur dan kamar mandi}{
Tuhan, berkatilah sumur dan kamar mandi. Bukalah selalu sumber-sumber air, agar bagi keluarga ini selalu tersedia air bersih secukupnya. Semoga setiap mengalami kesejukan dan kesegaran daripadanya, mereka selalu ingat akan Dikau, Sumber Kehidupan yang tidak pernah surut.}

\bait{Sang Penghibur umat Allah, \\kuatkan iman yang lemah,\\ agar hati bergembira \\walau dilanda derita.}

\doa{seluruh dan sekeliling rumah}{
Tuhan, berkatilah seluruh rumah ini, karena di dalamnya berdiam putera-puteri kesayanganMu. Jauhkanlah keluarga ini dari segala yang dapat merusak dan mengganggu ketenangan dan ketenteraman. Berilah mereka rejeki melimpah. Jagalah agar rumah ini tetap kokoh berdiri, sehingga dapat menjadi naungan yang aman bagi seluruh penghuninya.}

\bait{Penggerak para rasulMu,\\ lepaskan lidah yang kelu,\\ supaya kami wartakan\\ Karya Keslamatan Tuhan.\\ AMIN.}

\subjudul{Pemberkatan Salib}
\BI{Saudara-saudara. Salib ini adalah gambaran Iman kita. Kayu vertikal (dari atas ke bawah) melambangkan hubungan mesra Allah-manusia. Kayu horisontal (dari kiri ke kanan) : hubungan mesra kita dengan sesama. Tubuh Kristus adalah kurban Diri-Allah demi keselamatan umatNya, dan kesetiaan manusia sampai mati kepada kehendak Allah. Dengan pemasangan salib ini di kamar tamu, kita memaklumkan kepada seluruh dunia, bahwa rumah ini percaya kepada kekuasaan Allah yang tidak dapat dibatalkan, dan bahwa rumah ini telah menyerahkan diri untuk dikuasai dan diatur oleh Allah. Seringlah memandangNya, memberi hormat dan berdoa dalam hati, agar kita dikuatkan, diberi ketenteraman dan semangat.

Marilah berdoa.

Tuhan, berkatilah salib ini \Cross {~}curahkanlah Roh KudusMu padanya, dan usirlah Kuasa Kegelapan dari padanya, agar salib ini menjadi tanda kehadiranMu di tengah keluarga \keluarga , supaya barangsiapa memandangnya dan diberkati dengannya, memperoleh kekuatan Iman-Pengharapan-dan-Cinta-Kasih dari padaMu dan Kau selamatkan, demi Kristus pengantara kami.}
\BU{Amin}
\BI{(\textit{Kepada Kepala Keluarga})\\
Terimalah salib ini dan pasanglah di rumahmu sebagai tanda Imanmu.}
\BU{Syukur kepada Allah}

(\textit{Lalu Kepala Keluarga memasang salib tersebut di ruang tamu}).

\subjudul{Doa Umat}
\BI{Allah Bapa yang Mahakuasa, berkat dan rahmat-Mu senantiasa Kau limpahkan kepada kami, khususnya keluarga ini. Engkaulah tumpuan dan harapan kami. Untuk itu perkenankanlah kami berdoa:}

\BP{Ya Bapa, Engkau telah mempersatukan Bernadus Budhiprayoga dan Cicilia Nony Ayuningsih sebagai suami istri dalam ikatan sakramen pernikahan yang suci. Kini Engkau anugerahi buah cinta kasih mereka yang saat ini berusia 7 bulan dalam kandungan ibunya. Limpahilah berkat kekuatan dan kesehatan bagi ibu serta bayi dalam kandungannya hingga saat proses kelahirannya nanti berjalan dengan lancar, ibu dan anaknya sehat dan selamat.

Marilah kita mohon \ldots}

\BU{Kabulkanlah doa kami, ya Tuhan}

\BP{Ya Bapa, kami berdoa bagi calon orangtua bayi, semoga mereka siap menjadi orangtua yang baik bagi anaknya sesuai dengan janji mereka di depan altar, bertanggung jawab mendidik secara iman Katolik. Berilah kekuatan dan kemampuan kepada mereka untuk mengasuh dan membesarkannya, sehingga menjadi anak yang bisa dibanggakan.

Marilah kita mohon \ldots}

\BU{Kabulkanlah doa kami, ya Tuhan}

\BP{Ya Bapa, semoga anak sebagai buah cinta kasih mereka yang akan Kau hadirkan di tengah-tengah keluarga baru ini semakin memperdalam iman akan Dikau dan kebahagiaan senantiasa mewarnai keluarganya. Berilah kesempatan kepada mereka untuk turut serta mewartakan kesaksian iman dalam karya dan pelayanan kepada sesama.

Marilah kita mohon \ldots}

\BU{Kabulkanlah doa kami, ya Tuhan}

\BP{Ya Bapa, keluarga \keluarga mengucapkan syukur atas tempat tinggal ini terlebih atas diterimanya sebagai anggota dalam keluarga besar lingkungan St. Petrus. Sucikan rumah ini ya Tuhan agar bersih dan bebas dari segala marabahaya, serta gangguan, baik yang kelihatan maupun tidak kelihatan, sehingga di tempat tinggal ini hanya Engkaulah yang meraja yang senantiasa memberi ketentraman dan kedamaian bagi penghuninya.  

Marilah kita mohon \ldots}

\BU{Kabulkanlah doa kami, ya Tuhan}

\BP{Ya Bapa, kami berdoa bagi umat lingkungan St. Petrus, serta bagi sanak saudara yang hadir dalam perayaan Ekaristi untuk memberikan dukungan pada upacara mitoni serta pemberkatan rumah ini. Berkatilah mereka ya Tuhan berserta keluarganya dengan karunia kesehatan, kedamaian dan kesejahteraan.

Marilah kita mohon \ldots}

\BU{Kabulkanlah doa kami, ya Tuhan}

\BI{Bapa Yang Maha Pemurah, itulah doa-doa yang kami panjatkan ke hadirat-Mu. Semoga rahmat dan karunia yang telah kami terima dan senantiasa akan Engkau alirkan semakin membesarkan Nama-Mu dan semakin menjadi kesaksian iman dalam hidup kami. Ya Bapa, sudilah kiranya Engkau mengabulkan doa-doa kami, demi Kristus, Tuhan dan pengantara kami.}

\BU{Amin.}

\judul{LITURGI EKARISTI}

\lagu{Lagu persembahan}
\begin{center}
\itshape{Ku Persembahkan}
\end{center}

\small
\begin{verse}
\itshape{
Ku persembahkan bagi-Mu Yesus\\
yang terbaik dan terindah.\\
Segala apa yang kumiliki\\
itu semua anugrah-Mu.\\
{~}\\
Ref:\\
Terimalah Tuhan persembahan ini \\
dengan tulus hati dan dengan hati bersyukur\\ 
terimalah Tuhan persembahan ini\\ 
segala puji hormat bagimu.\\
}
\end{verse}
\normalsize



\subjudul{Doa Syukur Agung}

\subjudul{Bapa Kami}

\lagu{Lagu Komuni}

\judul{RITUS PENUTUP}

\BI{Marilah berdoa,\\
Allah, Bapa kami. Engkau menghidupkan kami dan menghendaki agar kami hidup berbahagia untuk selama-lamanya. Kami mengucap syukur atas segala berkat yang kami terima sekeluarga. Terlebih kami bersyukur atas PuteraMu Yesus Kristus yang Kau berikan kepada kami sebagai kekuatan untuk hidup sekarang, dan jaminan untuk hidup surgawi. Semoga dengan berlindung di bawah pimpinanNya dan meneladan jalan hidupNya, kami sampai ke rumahMu dan berbahagia abadi bersama dengan Dikau, dalam persatuan dengan PuteraMu dan Roh Kudus, sepanjang segala masa,}
\BU{Amin.}

\subjudul{Berkat}

\BI{Tuhan sertamu}
\BU{Dan sertamu juga}
\BI{Semoga saudara sekalian diberkati oleh Allah yang mahakuasa :
\Cross {~}Bapa, dan Putera, dan Roh Kudus,}
\BU{Amin.}
\BI{Saudara sekalian, upacara pemberkatan rumah sudah selesai.
Marilah kita memuji Tuhan.}
\BU{Syukur kepada Allah.}
\BI{Marilah pergi, kita diutus}
\BU{Amin.}

\lagu{Penutup}

\small
\itshape{
\begin{center}
NDHEREK DEWI MARIYAH\end{center} 
\begin{verse}
Ndherek Dewi Mariyah temtu geng kang manah \\
Mboten yen kuwatosa Ibu njangkung tansah \\
Kanjeng Ratu ing swarga amba sumarah samya \\
Sang Dewi Sang Dewi Mangestonana \\
Sang Dewi Sang Dewi Mangestonana 
\end{verse}
\begin{verse}
Nadyan manah getera dipun godha setan\\ 
Nanging batos engetna wonten pitulungan\\ 
Wit Sang Putri Mariyah mangsa tega anilar\\ 
Sang Dewi Sang Dewi Mangestonana \\
Sang Dewi Sang Dewi Mangestonana 
\end{verse}
}
\normalsize

\newpage
\begin{flushright}
{\Large Ucapan terima kasih}

\noindent Dengan penuh syukur dalam kasih Tuhan, kami mengucapkan banyak
terima kasih kepada:

\textbf{Romo \romo}\\
yang telah berkenan mempimpin perayaan ekaristi pada malam hari ini.

\textbf{Umat lingkungan Santo Petrus Maguwo\\ dan tamu undangan}\\
yang telah mendukung perayaan ekaristi ini.

\textbf{Segenap keluarga dan orang-orang terkasih}\\
yang telah berkenan hadir memberikan cinta dan doa dalam perayaan
ekaristi ini.

Semoga Tuhan memberkati dan memelihara ikatan kasih\\ di antara kita semua.

Amin.

\bigskip 

Bapak dan Ibu \keluarga\\
\calonibu dan \calonayah\\
dan segenap keluarga
\end{flushright}


\end{document}