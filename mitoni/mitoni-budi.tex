\documentclass[12pt,twoside]{book}
\usepackage[a5paper,vmargin={2cm,2cm},hmargin={2cm,2cm}]{geometry}
\usepackage{graphicx}
\usepackage{marvosym}
\usepackage{palatino}
\usepackage{fancyhdr}
\usepackage{microtype}
\usepackage{xspace}

\renewcommand{\footrulewidth}{0.5pt}
\lhead[\fancyplain{}{\thepage}]%
      {\fancyplain{}{\rightmark}}
\rhead[\fancyplain{}{\leftmark}]%
      {\fancyplain{}{\thepage}}
\pagestyle{fancy}
\lfoot[\emph{\scriptsize Mitoni keluarga \calonibu \& \calonayah}]{}
\rfoot[]{\emph{\scriptsize Mitoni keluarga \calonibu \& \calonayah}}
\cfoot{}

\makeatletter
\newcommand{\judul}[1]{%
  {\parindent \z@ \centering \normalfont
    \interlinepenalty\@M \large \bfseries #1\par\nobreak \vskip 20\p@ }}
\newcommand{\subjudul}[1]{%
  {\parindent \z@ \normalfont
    \interlinepenalty\@M \bfseries #1\par\nobreak \vskip 20\p@ }}
\newcommand{\lagu}[1]{%
  {\parindent \z@ \normalfont
    \interlinepenalty\@M \bfseries \emph{#1}\par\nobreak \vskip 20\p@ }}

\renewenvironment{description}
               {\list{}{\labelwidth\z@ \itemindent-\leftmargin
                        \let\makelabel\descriptionlabel}}
               {\endlist}
\renewcommand*\descriptionlabel[1]{\hspace\labelsep 
                                \normalfont\bfseries #1 }
    
\newcommand{\doa}[2]{%
  \begin{description}
  \item[Doa untuk #1] #2
   
   Kami mohon : Kabulkanlah doa kami ya Tuhan.
  \end{description}
}

  
\makeatother

\newcommand{\BU}[1]{\begin{itemize} \item[U:] #1 \end{itemize}}
\newcommand{\BI}[1]{\begin{itemize} \item[I:] #1 \end{itemize}}
\newcommand{\BP}[1]{\begin{itemize} \item[P:] #1 \end{itemize}}
\newcommand{\keluarga}{Laurentius Sardju\xspace}
\newcommand{\calonibu}{Heriberta Anggun Hayuningtyasmara\xspace}
\newcommand{\calonayah}{Antonius Budi Praptono\xspace}
\newcommand{\romo}{Rosarius Sapto Nugroho Pr.\xspace}
\newcommand{\lingkungan}{Thomas Tegalsari\xspace}
\usepackage[bahasa]{babel}
\selectlanguage{bahasa}

\title{Ekaristi \vspace{1cm}\\MITONI\\ 
Ibu \calonibu \\ 
}

\author{
{~}\vspace{4cm}
oleh Romo \romo
} 
\date{6 Februari 2016}


\begin{document}
\maketitle
\Large  
\thispagestyle{empty}
{~}\newpage
\thispagestyle{empty}
\judul{RITUS PEMBUKA}
\lagu{Lagu Pembuka}
\small
\begin{center}
\itshape{Sungai Mengalir}
\end{center}

\begin{verse}
\itshape{
Sungai mengalir tiada henti-hentinya,\\ 
memberi hidup di sekitarnya.\\
Tuhan melimpahkan RahmatNya \\
bagi yang percaya kepadaNya.\\
{~}\\
Bunga-bunga tiada akan mekar mewangi,\\ 
jika tanpa disegarkan air.\\
Hidup akan menjadi hampa, \\
jika tanpa Cinta Kasih Tuhan.\\
{~}\\
Ya Tuhan Allah limpahkan Kasih SayangMu,\\ 
bagaikan air sungai abadi,\\
agar segarlah hidup kami, \\
tiada akan layu selamanya.\\
}
\end{verse}
\normalsize


\subjudul{Salam pembuka}

\BI{Demi Nama Bapa dan Putera dan Roh Kudus}
\BU{Amin.}
\BI{Semoga Allah Bapa serta PuteraNya, Tuhan kita Yesus Kristus, memberikan Kurnia dan Kesejahteraan kepada kita.}
\BU{Sekarang dan selama-lamanya.}

\subjudul{Pengantar}

\BI{Saudara-saudara terkasih. 

Upacara mitoni atau saat usia kandungan menginjak tujuh bulan, mempunyai makna yang dalam, bagi mayoritas masyarakat Jawa. Angka 7 dalam bahasa Jawa disebut \textit{pitu}. Maka dari itu makna di balik upacara itu adalah, pihak keluarga, suami dan ibu yang mengandung tersebut, ingin minta \textit{pitulungan} atau pertolongan kepada Tuhan Sang pencipta segala kehidupan. Dengan harapan jika tiba saatnya melahirkan nanti, sang ibu dan bayinya diberi keselamatan dan kesehatan dari-Nya.
}

\subjudul{Tobat}
\BI{Marilah kita hening sejenak untuk mempersiapkan diri dalam perayaan syukur ini sambil menyadari bahwa kita sering melupakan kebaikan Tuhan dan enggan mewartakan dan mewujudkan kebaikan tersebut melalui pikiran, perkataan, dan perbuatan kita.}

\BI{Saya mengaku}

\BU{Kepada Allah yang Maha Kuasa dan kepada saudara sekalian bahwa saya telah berdosa dengan pikiran dan perkataan, dengan perbuatan dan kelalaian. Saya berdosa, saya berdosa, saya sungguh berdosa. Oleh sebab itu saya mohon kepada Santa Perawan Maria, kepada Para Malaikat dan orang kudus dan kepada saudara sekalian, supaya mendoakan saya kepada Allah Tuhan kita.}

\BI{Semoga Allah Yang Maha Kuasa mengasihi kita, mengampuni dosa kita dan menghantar kita ke hidup yang kekal.}

\BU{Amin}

\lagu{Tuhan Kasihanilah Kami - MB 185}

\subjudul{Doa Pembuka}

\BI{Marilah berdoa

Bapa yang maha kasih, kami bersyukur kepada-Mu, karena Engkau berkenan mengikutsertakan \calonayah dan \calonibu dalam karya penciptaanMu. Makhluk baru sudah Kauciptakan dengan perantaraan mereka berdua. Maka kami mohon, dampingilah selalu suami istri ini, supaya dapat menjaga dan merawat ciptaanMu yang suci ini dengan baik, sampai kelak lahir dengan selamat, dan memberikan kegembiraan besar bagi kami semua. Demi Yesus Kristus PutraMu, Tuhan dan Pengantara kami, yang bersatu dengan Dikau dan Roh Kudus, hidup dan berkuasa, kini dan sepanjang masa.
Amin.}

\judul{LITURGI SABDA}

\subjudul{Bacaan Kitab Suci}

Bacaan dari Kitab Keluaran (32:7-14)

\BP{Berfirmanlah TUHAN kepada Musa: "Pergilah, turunlah, sebab bangsamu yang kaupimpin keluar dari tanah Mesir telah rusak lakunya.
Segera juga mereka menyimpang dari jalan yang Kuperintahkan kepada mereka; mereka telah membuat anak lembu tuangan, dan kepadanya mereka sujud menyembah dan mempersembahkan korban, sambil berkata: Hai Israel, inilah Allahmu yang telah menuntun engkau keluar dari tanah Mesir."

Lagi firman TUHAN kepada Musa: "Telah Kulihat bangsa ini dan sesungguhnya mereka adalah suatu bangsa yang tegar tengkuk.
Oleh sebab itu biarkanlah Aku, supaya murka-Ku bangkit terhadap mereka dan Aku akan membinasakan mereka, tetapi engkau akan Kubuat menjadi bangsa yang besar."

Lalu Musa mencoba melunakkan hati TUHAN, Allahnya, dengan berkata: "Mengapakah, TUHAN, murka-Mu bangkit terhadap umat-Mu, yang telah Kaubawa keluar dari tanah Mesir dengan kekuatan yang besar dan dengan tangan yang kuat?

Mengapakah orang Mesir akan berkata: Dia membawa mereka keluar dengan maksud menimpakan malapetaka kepada mereka dan membunuh mereka di gunung dan membinasakannya dari muka bumi? Berbaliklah dari murka-Mu yang bernyala-nyala itu dan menyesallah karena malapetaka yang hendak Kaudatangkan kepada umat-Mu.

Ingatlah kepada Abraham, Ishak dan Israel, hamba-hamba-Mu itu, sebab kepada mereka Engkau telah bersumpah demi diri-Mu sendiri dengan berfirman kepada mereka: Aku akan membuat keturunanmu sebanyak bintang di langit, dan seluruh negeri yang telah Kujanjikan ini akan Kuberikan kepada keturunanmu, supaya dimilikinya untuk selama-lamanya."
Dan menyesallah TUHAN karena malapetaka yang dirancangkan-Nya atas umat-Nya.

Demikian Sabda Tuhan}

\BU{Syukur kepada Allah}

\lagu{Lagu Pengantar Bacaan}
\small
\begin{center}
\itshape{Bahagia manusia - MB 214}
\end{center}
\begin{verse}
\itshape{
Bahagia manusia\\
Yang tidak tuli hatinya\\
Yang mendengarkan sabda Bapa\\
Tekun melaksanakannya\\
Sabda Tuhan penuh daya\\
Yang tersesat dipanggilNya\\
DisembuhkanNya yang luka\\
Yang mati dihidupkanNya\\
{~}\\
Bahagia manusia\\
Yang menerima Sang sabda\\
Sabda yang sudah menjelma\\
Dalam wujud manusia\\
Terpujilah oh Sang Kristus\\
Sabda kekal dan penebus\\
Kebenaran, kehidupan\\
Serta jalan keslamatan
}
\end{verse}
\normalsize


\subjudul{Injil}
\BI{Tuhan sertamu,}
\BU{Dan sertamu juga.}
\BI{Inilah Injil Yesus Kristus menurut Yohanes - 5:31--47}
\BU{Terpujilah Kristus}

\BI{Kalau Aku bersaksi tentang diri-Ku sendiri, maka kesaksian-Ku itu tidak benar;
ada yang lain yang bersaksi tentang Aku dan Aku tahu, bahwa kesaksian yang diberikan-Nya tentang Aku adalah benar.

Kamu telah mengirim utusan kepada Yohanes dan ia telah bersaksi tentang kebenaran;
tetapi Aku tidak memerlukan kesaksian dari manusia, namun Aku mengatakan hal ini, supaya kamu diselamatkan.

Ia adalah pelita yang menyala dan yang bercahaya dan kamu hanya mau menikmati seketika saja cahayanya itu.
Tetapi Aku mempunyai suatu kesaksian yang lebih penting dari pada kesaksian Yohanes, yaitu segala pekerjaan yang diserahkan Bapa kepada-Ku, supaya Aku melaksanakannya. Pekerjaan itu juga yang Kukerjakan sekarang, dan itulah yang memberi kesaksian tentang Aku, bahwa Bapa yang mengutus Aku.

Bapa yang mengutus Aku, Dialah yang bersaksi tentang Aku. Kamu tidak pernah mendengar suara-Nya, rupa-Nyapun tidak pernah kamu lihat,
dan firman-Nya tidak menetap di dalam dirimu, sebab kamu tidak percaya kepada Dia yang diutus-Nya.

Kamu menyelidiki Kitab-kitab Suci, sebab kamu menyangka bahwa oleh-Nya kamu mempunyai hidup yang kekal, tetapi walaupun Kitab-kitab Suci itu memberi kesaksian tentang Aku,
namun kamu tidak mau datang kepada-Ku untuk memperoleh hidup itu.

Aku tidak memerlukan hormat dari manusia.
Tetapi tentang kamu, memang Aku tahu bahwa di dalam hatimu kamu tidak mempunyai kasih akan Allah.

Aku datang dalam nama Bapa-Ku dan kamu tidak menerima Aku; jikalau orang lain datang atas namanya sendiri, kamu akan menerima dia.

Bagaimanakah kamu dapat percaya, kamu yang menerima hormat seorang dari yang lain dan yang tidak mencari hormat yang datang dari Allah yang Esa?
Jangan kamu menyangka, bahwa Aku akan mendakwa kamu di hadapan Bapa; yang mendakwa kamu adalah Musa, yaitu Musa, yang kepadanya kamu menaruh pengharapanmu.

Sebab jikalau kamu percaya kepada Musa, tentu kamu akan percaya juga kepada-Ku, sebab ia telah menulis tentang Aku.
Tetapi jikalau kamu tidak percaya akan apa yang ditulisnya, bagaimanakah kamu akan percaya akan apa yang Kukatakan?"

Demikianlah Injil Tuhan}

\BU{Terpujilah Kristus}


\subjudul{Homili}

\judul{PEMBERKATAN IBU YANG SEDANG MENGANDUNG}
{\itshape (Romo menumpangkan kedua tangan di atas kepala Ibu \calonibu yang sedang mengandung 7 bulan seraya berdoa:)}
\BI{Allah Bapa yang Maha Kasih, Engkau telah mengutus Putera-Mu sendiri untuk menjelma menjadi manusia dengan perantaraan Perawan Maria. Sampai saatnya tiba, Putera-Mu lahir selamat dan luput dari bahaya pembunuhan Herodes berkat pertolongan-Mu lewat bisikan para malaikat. Maka dari itu ya Bapa, dampingilah Ibu \calonibu yang telah mengandung puteranya selama 7 bulan ini, agar nanti juga selamat sampai saatnya melahirkan.}
\BU{Amin.}
\BI{Allah Bapa yang Maha Kasih, Engkaulah yang meniupkan nafas kehidupan bagi janin-janin yang tumbuh di dalam rahim seorang ibu. Dari hari ke hari karena kuasa dan kasih-Mu, makhluk lemah tak berdaya itu menjadi kuat, bertumbuh menjadi ciptaan yang sempurna secitra dengan diri-Mu sendiri. Maka dari itu ya Bapa, semoga makhluk ciptaan-Mu yang berusia 7 bulan ini semakin Engkau kuatkan agar nanti bisa menghirup udara kehidupan dalam dekapan kasih kedua orang tuanya.}
\BU{Amin.}
\BI{Ya Bapa, perkenan kami juga berdoa dengan perantaraan wanita yang dikandung tanpa dosa, wanita pilihan-Mu yang karena kuasa Roh Kudus mengandung Putera-Mu sendiri, yakni Bunda Maria.

Ya Bunda Maria, teladan para ibu yang bersahaja, engkau telah mengalami sendiri bagaimana mengandung seorang putra. Antara bahagia, was-was, takut namun penuh harap semua itu pernah Bunda rasakan. Karena itu ya Bunda Maria, dampingilah Ibu \calonibu agar hatinya mantap, tidak ada rasa takut maupun was-was saat menyongsong kelahiran anaknya nanti.} 
\BU{Amin.}

\BI{Ya Allah Bapa yang maha rahim, sumber pengharapan dan keselamatan kami, sekarang Ibu \calonibu dan bayi dalam kandungan kami serahkan ke dalam tangan-Mu. Rawatlah selalu melalui tangan-tangan-Mu agar ibu dan anaknya senatiasa sehat. Cintailah selalu agar ibu dan anaknya merasa tentram karena Engkau selalu berada di sisinya. 

Semua permohonan ini kami sampaikan kepada-Mu dengan perantaraan Putera-Mu terkasih Tuhan Yesus yang bersama dikau dan Roh Kudus hidup dan berkuasa sepanjang segala masa.}
\BU{Amin.}



\subjudul{Doa Umat}
\BI{Allah Bapa yang Mahakuasa, berkat dan rahmat-Mu senantiasa Kau limpahkan kepada kami, khususnya keluarga ini. Engkaulah tumpuan dan harapan kami. Untuk itu perkenankanlah kami berdoa:}

\BP{Ya Bapa, Engkau telah mempersatukan \calonayah dan \calonibu sebagai suami istri dalam ikatan sakramen pernikahan yang suci. Kini Engkau anugerahi buah cinta kasih mereka yang saat ini berusia 7 bulan dalam kandungan ibunya. Limpahilah berkat kekuatan dan kesehatan bagi ibu serta bayi dalam kandungannya hingga saat proses kelahirannya nanti berjalan dengan lancar, ibu dan anaknya sehat dan selamat.

Marilah kita mohon \ldots}

\BU{Kabulkanlah doa kami, ya Tuhan}

\BP{Ya Bapa, kami berdoa bagi calon orangtua bayi, semoga mereka siap menjadi orangtua yang baik bagi anaknya sesuai dengan janji mereka di depan altar, bertanggung jawab mendidik secara iman Katolik. Berilah kekuatan dan kemampuan kepada mereka untuk mengasuh dan membesarkannya, sehingga menjadi anak yang bisa dibanggakan.

Marilah kita mohon \ldots}

\BU{Kabulkanlah doa kami, ya Tuhan}

\BP{Ya Bapa, semoga anak sebagai buah cinta kasih mereka yang akan Kau hadirkan di tengah-tengah keluarga baru ini semakin memperdalam iman akan Dikau dan kebahagiaan senantiasa mewarnai keluarganya. Berilah kesempatan kepada mereka untuk turut serta mewartakan kesaksian iman dalam karya dan pelayanan kepada sesama.

Marilah kita mohon \ldots}

\BU{Kabulkanlah doa kami, ya Tuhan}

\BP{Ya Bapa, kami berdoa bagi bapak, ibu, serta sanak saudara yang hadir dalam perayaan Ekaristi untuk memberikan dukungan pada upacara mitoni serta pemberkatan rumah ini. Berkatilah mereka ya Tuhan berserta keluarganya dengan karunia kesehatan, kedamaian dan kesejahteraan.

Marilah kita mohon \ldots}

\BU{Kabulkanlah doa kami, ya Tuhan}

\BI{Bapa Yang Maha Pemurah, itulah doa-doa yang kami panjatkan ke hadirat-Mu. Semoga rahmat dan karunia yang telah kami terima dan senantiasa akan Engkau alirkan semakin membesarkan Nama-Mu dan semakin menjadi kesaksian iman dalam hidup kami. Ya Bapa, sudilah kiranya Engkau mengabulkan doa-doa kami, demi Kristus, Tuhan dan pengantara kami.}

\BU{Amin.}

\judul{LITURGI EKARISTI}

\lagu{Lagu persiapan persembahan}
\small
\begin{center}
\itshape{Hidup Cerah - MB230}
\end{center}


\begin{verse}
\itshape{
Dengan haru kita serahkan, hidup kita yang tanpa arti.\\
S'moga Tuhan nanti berkenan mewarnai hidup yang mati. \\
{~}\\
Dalam Yesus kita berbakti, hidup kita jadi berarti.\\
Tiada lagi warna yang suram, jiwa raga cerah bersinar.\\
{~}\\
Lagu syukur kita haturkan, dengan khidmat hati berbakti.\\
S'moga Tuhan nanti berkenan memberikan hidup abadi.
}
\end{verse}
\normalsize

\subjudul{Doa Persembahan}
\BI{Kami memuji Engkau, ya Bapa, Allah semesta alam, sebab dari kemurahanMu kami menerima roti dan anggur yang kami persembahkan ini. Inilah hasil dari bumi dan dari usaha manusia yang bagi kami akan menjadi santapan rohani.}
 
\BU{Terpujilah Allah selama-lamanya.}
 
\BI{Berdoalah saudara-saudara, supaya persembahan kita ini diterima oleh Allah Bapa yang maha Kuasa.}
 
\BU{Semoga persembahan ini diterima demi kemuliaan Tuhan dan keselamatan kita serta seluruh umat Allah yang kudus.}
 
\BI{Marilah berdoa,

Ya Tuhan, kami mohon sudilah menerima persembahan yang kami sampaikan kepadaMu untuk keselamatan Ibu \calonibu dan bayi yang dikandungnya beserta keluarganya. Semoga berkat perayaan suci ini, mereka makin saling mencintai dan mengasihi Engkau. Demi Kristus, Tuhan dan Pengantara kami.}
 
\BU{Amin.}

\subjudul{Doa Syukur Agung}

\lagu{Kudus - MB 256}

\subjudul{Bapa Kami}

\lagu{Anak Domba Allah - MB 277}

\lagu{Lagu Komuni}

\subjudul{Doa sesudah komuni}

\BI{Marilah berdoa,\\
Allah, Bapa kami. Engkau menghidupkan kami dan menghendaki agar kami hidup berbahagia untuk selama - lamanya. Kami mengucap syukur atas segala berkat yang kami terima sekeluarga. Terlebih kami bersyukur atas PuteraMu Yesus Kristus yang Kau berikan kepada kami sebagai kekuatan untuk hidup sekarang, dan jaminan untuk hidup surgawi. Dengan pengantaran Tuhan kami Yesus Kristus, kini dan sepanjang segala masa,}
\BU{Amin.}

\judul{RITUS PENUTUP}


\subjudul{Berkat}

\BI{Tuhan sertamu}
\BU{Dan sertamu juga}
\BI{Semoga saudara sekalian diberkati oleh Allah yang mahakuasa :
\Cross {~}Bapa, dan Putera, dan Roh Kudus,}
\BU{Amin.}
\BI{Saudara sekalian, ibadat ekaristi mitoni sudah selesai.
Marilah kita memuji Tuhan.}
\BU{Syukur kepada Allah.}
\BI{Marilah pergi, kita diutus}
\BU{Amin.}

\lagu{Lagu Penutup - nDherek Dewi Mariyah}

%\small
\itshape{
\begin{center}
NDHEREK DEWI MARIYAH\end{center} 
\begin{verse}
Ndherek Dewi Mariyah temtu geng kang manah \\
Mboten yen kuwatosa Ibu njangkung tansah \\
Kanjeng Ratu ing swarga amba sumarah samya \\
Sang Dewi Sang Dewi Mangestonana \\
Sang Dewi Sang Dewi Mangestonana 
\end{verse}
\begin{verse}
Nadyan manah getera dipun godha setan\\ 
Nanging batos engetna wonten pitulungan\\ 
Wit Sang Putri Mariyah mangsa tega anilar\\ 
Sang Dewi Sang Dewi Mangestonana \\
Sang Dewi Sang Dewi Mangestonana 
\end{verse}
}
\normalsize

\newpage
\begin{flushright}
{\Large Ucapan terima kasih}

\noindent Dengan penuh syukur dalam kasih Tuhan, kami mengucapkan banyak
terima kasih kepada:

\textbf{Romo \romo}\\
yang telah berkenan mempimpin perayaan ekaristi pada malam hari ini.

\textbf{Umat lingkungan \lingkungan\\ dan tamu undangan}\\
yang telah mendukung perayaan ekaristi ini.

\textbf{Segenap keluarga dan orang-orang terkasih}\\
yang telah berkenan hadir memberikan cinta dan doa dalam perayaan
ekaristi ini.

Semoga Tuhan memberkati dan memelihara ikatan kasih\\ di antara kita semua.

Amin.

\bigskip 

\small
Bapak dan Ibu \keluarga\\
\calonibu dan\\ \calonayah\\
dan segenap keluarga
\end{flushright}


\end{document}