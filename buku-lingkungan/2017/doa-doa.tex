\chapter{Doa-Doa}

\section{Litani St. Theresia dari Kanak-kanak Yesus}
Tuhan, kasihanilah kami, Tuhan, kasihanilah kami.\\
Kristus kasihanilah kami, Kristus dengarkanlah kami.\\
Kristus, kabulkanlah doa kami.\\
Allah Bapa di Surga, kasihanilah kami.\\
Allah Putera Penebus dunia,\\
Allah Roh Kudus,\\
Allah Tri Tunggal Mahakudus; Allah Yang Maha Esa,\\
Bunda Maria, doakanlah kami.\\
Santa Theresia, Putri kanak-kanak Yesus,\\
Santa Theresia, mempelai Kristus,\\
Santa Theresia, malaikat murni di dunia,\\
Santa Theresia, kegembiraan ayah bundanya,\\
Santa Theresia, teladan anak-anak,\\
Santa Theresia, hiasan para penganjur komuni kudus,\\
Santa Theresia, mempelai Kristus dalam Ekaristi,\\
Santa Theresia, bunga kesempurnaan mistik,\\
Santa Theresia, bunga yang sederhana,\\
Santa Theresia, bunga mawar cinta kasih,\\
Santa Theresia, bunga bakung kemurnian,\\
Santa Theresia, bunga di taman Karmel,\\
Santa Theresia, teladan ketaatan,\\
Santa Theresia, teladan kemurnian,\\
Santa Theresia, keluhuran hidup bertapa,\\
Santa Theresia, yang  mendambakan  mati sebagai martir,\\
Santa Theresia, yang mengandalkan  penyelenggaraan  Tuhan,\\
Santa Theresia, kurban bagi para imam,\\
Santa Theresia, pelindung tanah-tanah misi,\\
Santa Theresia, penghibur orang-orang sakit,\\
Santa Theresia, ketabahan mereka yang berkecil hati,\\
Santa Theresia, tempat pelarian orang yang berkecil hati,\\
Santa Theresia, pembimbing jiwa-jiwa,\\
Santa Theresia, guru cita-cita hidup sebagai anak-anak,\\
Santa Theresia, penolong orang-orang berdosa,\\
Santa Theresia, yang ditakuti setan,\\
Santa Theresia, pemecah aneka kesulitan,\\
Santa Theresia, yang menghapus dosa dunia,\\
Anak Domba Allah yang menghapus dosa dunia,\\
sayangilah kami, ya Tuhan.\\
Anak Domba Allah yang menghapus dosa dunia,\\
kabulkanlah doa kami, ya Tuhan.\\
Anak Domba Allah yang menghapus dosa dunia,\\
kasihanilah kami.\\
Santa Theresia, doakanlah kami.\\
Supaya kami pantas menerima janji Kristus.\\
\\
\textbf{Marilah berdoa}\\
Tuhan Yesus Kristus, Engkau pernah bersabda, barang siapa tidak menyambut kerajaan Allah seperti seorang anak menyambutnya, ia tidk akan measuk kedalamnya?. Kami mohon kepadaMu, semoga dapat mengikuti jejak Santa Theresia, dengan sederhana dan rendah hati sungguh-sungguh, sehingga dapat memperoleh kebahagiaan kekal di surga. Engkau yang hidup dan berkuasa sepanjang masa. 
Amin.

\section{Novena kepada St. Theresia dari Kanak-kanak Yesus}
\normalsize
Santa Theresia, kekasih Kanak-kanak Yesus,\\
aku ikut besyukur kepada Tuhan, \\
karena engkau telah memperoleh rahmat tak terbilang. \\
Aku ikut pula bergembira, \\
karena Engkau telah diperkenankan memiliki 
kemuliaan kekal di surga.\\
Ketika masih hidup di dunia ini, \\
engkau selalu menaati segala kehendak Tuhan Yesus Kristus. \\
Maka kini di surga tentulah Tuhan Yesus Kristus mengabulkan segala permohonanmu. 
Berapa ribu saja orang yang terkabul permohonannya berkat pengantaranmu. \\
Akan sampaikah hatimu tidak mendengarkan permohonanku ini?\\
Santa Theresia, aku mohon bantuanmu. Sudilah membawa permohonanku ini kepada Tuhan Yesus Kristus:\\

(\textit{sebutkan dalam hati} \ldots\ldots\ldots) \\
\\Tunjukkan kini kuasamu.\\ Engkau kini seakan-akan diberi hak menjadi bendahara surga, 
sebab anugerahmu, \\ bagaikan hujan bunga mawar karena banyaknya.\\
Puteri kekasih Kanak-kanak Yesus, aku percaya. \\Mustahil aku tidak engkau bantu. \\
Amin.\\

$ \left.
\begin{array}{c}
\textrm{Bapa kami \ldots} \\
\textrm{Salam Maria \ldots}\\
\textrm{Kemuliaan \ldots} \\
\end{array}
\right\rbrace 3 \times
$
\\

\noindent{\textbf{Marilah berdoa:}}\\
Allah Bapa kami, hambaMu Santa Theresia sudah Engkau muliakan, berkat perantaraannya kamipun dapat melalui jalan yang rendah hati dan iman yang mantap. Demi Kristus, Tuhan dan pengantara kami. \\Amin.

\section{Doa Angelus dan Ratu Surga}
\subsection*{Doa Angelus}
Maria diberi kabar oleh Malaikat TUHAN~ Maka Ia mengandung dari Roh
Kudus~ \\
Salam Maria {\ldots}~  

Aku ini hamba TUHAN~ Terjadilah padaku menurut
perkataanMU.~ \\
Salam Maria {\ldots}~  

Sabda sudah menjadi daging~ Dan tinggal
diantara kita~ \\
Salam Maria~{\ldots}  

Doakanlah kami, ya Santa Bunda
ALLAH~ Supaya kami dapat menikmati janji KRISTUS.~

Marilah berdoa: (\textit{hening sejenak})~ Ya Allah, karena kabar Malaikat kami
mengetahui~bahwa YESUS KRISTUS PutraMU menjadi manusia.~Curahkanlah
rahmatMU ke dalam hati kami,~supaya karena sengsara dan salibNYA,~kami
dibawa kepada kebangkitan yang mulia.~Sebab DIAlah TUHAN dan Pengantara
kami~.  Amin~

\subsection*{Doa Ratu Surga (dalam Masa Paskah)}
Ratu Surga bersukacitalah, alleluya,~ Sebab Ia yang sudi kau kandung,
alleluya,~

 Telah bangkit seperti disabdakan-Nya, alleluya!~ Doakanlah kami pada
Allah, alleluya!~

 Bersukacita dan bergembiralah, Perawan Maria, alleluya,~ sebab Tuhan
sungguh telah bangkit, Alleluya!~

Marilah berdoa (hening sejenak)~ Ya Allah,~ Engkau telah menggembirakan
dunia dengan kebangkitan PutraMu,~Tuhan kami Yesus Kristus.~Kami
mohon,~perkenankanlah kami bersukacita dalam kehidupan kekal bersama
BundaNya, Perawan Maria.~Demi Kristus, pengantara kami.~ Amin.

\subsection*{Sejarah Doa Angelus}
Kita mengenal tradisi doa Angelus yang kita doakan pada jam 6 pagi, jam
12 siang dan jam 6 sore. Doa ini mempunyai 2 rumusan yakni rumusan
untuk dipakai pada masa Paskah dan rumusan untuk masa di luar Paskah.
Di Indonesia doa ini mulanya penggunaannya masih terbatas pada kalangan
kaum religius dan rohaniawan-rohaniwati. Akhir-akhir ini, doa Angelus
sudah semakin sering didoakan oleh umat awam.

\subsubsection[Arti]{Arti}
{\textquotedbl}Angelus{\textquotedbl} berarti
{\textquotedbl}Malaikat{\textquotedbl}.

\subsubsection[Mengapa dinamakan Doa Angelus?]{Mengapa dinamakan Doa
Angelus?}
Dinamakan Angelus karena kata ini merupakan kata pertama dari
{\textquotedbl}Maria diberi kabar oleh Malaikat{\textquotedbl} Yang
dalam bahasa latinnya adalah {\textquotedbl}Angelus domini nuntiavit
Mariae{\textquotedbl}

Doa Angelus sore hari dimulai pada abad ke-13 di Eropa. Oleh karena itu
doa Angelus sore hari ini yang pertama kali digunakan. Selanjutnya pada
pertengahan abad ke-14 barulah doa Angelus pagi hari digunakan di
seluruh Eropa. Doa Angelus pagi dan sore hari didoakan oleh para rahid
sebagai bagian dari doa pagi dan doa malam di biara-biara. Diawali
dengan doa Angelus kemudian dilanjutkan doa-doa harian para rahib
biara. Kemudian pada antara abad 14-15, barulah doa Angelus pada siang
hari muncul dan mulai didoakan.

\subsubsection[Tujuan Doa Angelus]{Tujuan Doa Angelus}
\begin{description} 
\item[Doa Angelus jam 6 pagi: Menghormati kebangkitan Kristus.]

Yesus yang telah bangkit dan bersama Kristus kita memulai dari dengan
semangat kebangkitan.

\item [Doa Angelus jam 12 siang: Menghormati sengsara Kristus.]

Di tengah pekerjaan kita yang berat, kita senantiasa ingat Kristus yang
telah berkorban bagi kita.

\item [Doa Angelus jam 6 sore: Menghormati Inkarnasi Allah menjadi manusia.]

Pada saat kita beranjak untuk beristirahat, ingatlah bahwa Allah selalu
tinggal beserta kita.
\end{description}

\small
\section[Doa masa Advent]{Doa masa Advent}
Ya Allah, Bapa yang Mahakudus kami bersyukur kehadirat-Mu, karena lewat
masa penantian ini Engkau menjanjikan Juruselamat yakni Yesus Kristus
Putra-Mu. Kedatangan-Nya dinubuatkan oleh para nabi dan dinantikan oleh
Perawan Maria dengan cinta mesra. Dialah Adam baru yang memulihkan
persahabatan kami dengan Dikau. Ia penolong yang lemah dan
menyelamatkan yang berdosa.

Ia membawa damai sejati bagi kami dan membuat semakin banyak orang
mengenal Engkau, dan berani melaksanakan kehendak-Mu. Ia datang sebagai
manusia biasa, untuk melaksanakan rencana-Mu dan membukakan jalan
keselamatan bagi kami. Pada akhir zaman ia akan datang lagi dengan
semarak dan mulia untuk menyatakan kebahagiaan yang kami nantikan.

Kami mohon kelimpahan rahmat-Mu, agar selama hidup di dunia ini kami
selalu siap siaga dan penuh harap menantikan kedatangan-Nya yang mulia,
agar pada saat Ia datang nanti, kami Kau perkenankan ikut berbahagia
bersama Dia dan seluruh umat kesayangan-Mu. Sebab Dialah Tuhan,
pengantara kami, kini dan sepanjang masa. (Amin)

\section[Doa masa Natal]{Doa masa Natal}
Allah Bapa disurga, kami memuji Engkau dan bersyukur kepada-Mu karena
sabda-Mu yang menjadi manusia dengan lahir ditengah-tengah kami. Ia
menjadi manusia lemah agar kami yang rapuh dan fana ini diurapi oleh
Daya ilahi yang Abadi.

Dengan kelahiran-Nya di dunia ini, Engkau yang tak dapat dilihat kini
kelihatan sebagai manusia seperti kami, dan cahaya keselamatan-Mu
bersinar ditengah kami, mengusir kegelapan yang menguasai kami.

Curahkanlah rahmat-Mu, agar kami yang kini merayakan misteri inkarnasi
berani menjadi pembawa damai bagi sesama, dan dengan demikian kami pun
menjadi sarana inkarnasi-Mu ditengah-tengah mereka. Dengan pengantaraan
Kristus, Tuhan kami, kini dan sepanjang masa (Amin).

\section[Doa masa PraPaskah]{Doa masa PraPaskah}
Allah Bapa yang maha kuasa, kami bersyukur kepada-Mu atas masa prapaskah
yang Kau anugerahkan kepada kami. Lewat masa prapaskah ini. Engkau
menginginkan kami untuk menyadari segala kebaikan-Mu. Selama masa
prapaskah ini Engkau melimpahkan rahmat untuk menyegarkan iman kami.

Engkau mengajak kami untuk bertobat, menyesali kekurangan dan dosa-dosa
kami. Engkau mendorong kami melepaskan diri dari belenggu nafsu yang
menyesatkan. Engkau mengajar kami untuk hidup sederhana, mensyukuri
segala anugerah-Mu, dan membantu orang-orang yang menderita. Selama
masa prapaskah ini Engkau membimbing para calon baptis yang akan
bersatu dengan kami melalui sakramen baptis. Sambil mendampingi mereka,
kamipun Kau ajak menyegarkan rahmat baptisan yang pernah kami terima
dari-Mu.

Semoga karena rahmat-MU, yang Kau limpahkah selama Masa Prapaskah ini,
kami semakin Suci, semakin bersatu dengan umat kesayangan-MU, dan
berani meneladani Yesus Putra-MU, yang rela menderita sengsara, wafat
dan bangkit untuk menyelamatkan kami. Sebab dialah Tuhan, pengantara
kami, kini dan sepanjang masa (Amin)

\section[Doa Paskah]{Doa Paskah}
Allah Bapa yang mahabaik, kami bersyukur kepada-Mu Karena Yesus Kristus
telah bangkit dari Kubur. Dengan kebangkitan-Nya. kau tumbuhkan
semangat dan harapan baru dalam hati kami; umat baru Kau ciptakan, dan
pintu surga Kaubuka bagi kami. Melalui kebangkitan-Nya kuasa Dosa kau
hancurkan, kami Kau damaikan dengan Dikau dan sesama, dan alam semesta
yang porak poranda Kaupugar kembali.

Dengan kenaikannya Ia merintis jalan kesurga, dan menyediakan tempat
bagi kami. Semoga karena Rahmat kebangkitan-Nya kami menjadi manusia
baru, yang penuh harapan, yang gigih melawan dosa dan kejahatan, yang
setia mengikuti kehendak-MU, dan tak gentar akan derita salib. Demi
Yesus Kristus, pengantara Kami, kini dan sepanjang masa. (Amin)
\normalsize

\section[Doa NOVENA Roh Kudus]{Doa NOVENA Roh Kudus}
Umat Kristen mempunyai kebiasaan mengadakan doa Novena Roh Kudus. Ini
dilaksanakan selama sembilan hari (novena = sembilan), mulai pada hari
sesudah kenaikan Tuhan Yesus ke surga dan berakhir pada hari Sabtu
menjelang Pentekosta. dalam doa ini umat Kristen memuji Tuhan yang
menjanjikan kedatangan Roh Kudus dan memohon rahmat Allah agar siap
menyambut kedatangan Roh Kudus. Doa ini juga bisa dilaksanakan pada
kesempatan lain yang cocok. Yang tersaji disini lebih dimaksudkan untuk
didoakan dalam kelompok; kalau didoakan secara pribadi, dapat
disesuaikan seperlunya.

Kalau Novena ini dipadukan dengan Perayaan Ekaristi, sesudah Mohon Tujuh
Karunia Roh Kudus menyusul Liturgi Ekaristi (persembahan, Doa syukur
Agung, dan seterusnya)

\subsection*{Hari Pertama}
Allah pokok keselamatan kami, karena kebangkitan Kristus kami lahir
kembali dalam pembabtisan dan menjalani hidup baru. Arahkanlah hati
kami kepada Kristus yang kini duduk di sebelah kanan-Mu. Semoga Roh-Mu
menjaga kami sampai Penyelamat kami datang dalam kemuliaan, sebab
Dialah Tuhan, Pengantara kami, kini dan sepanjang masa. Amin

Dilanjutkan dengan Rosario Roh Kudus ...

\subsection*{Hari Kedua}
Allah yang mahabijaksana, Putra-Mu menjanjikan Roh Kudus kepada para
rasul dan memenuhi janji itu sesudah Dia naik ke surga. Semoga kami pun
Kau anugrahi karunia Roh Kudus. Demi Yesus Kristus, Pengantara kami,
kini dan sepanjang masa. Amin

Dilanjutkan dengan Rosario Roh Kudus ...

\subsection*{Hari Ketiga}
Allah, Penyelamat kami, kami percaya bahwa Kristus telah bersatu dengan
Dikau dalam keagungan. Semoga dalam Roh-Nya, Dia selalu menyertai kami
sampai akhir zaman, seperti yang dijanjikan-Nya. Sebab Dialah Tuhan
kami, kini dan sepanjang masa. Amin

Dilanjutkan dengan Rosario Roh Kudus ...

\subsection*{Hari Keempat}
Allah yang mahakudus, semoga kekuatan Roh-Mu turun atas kami, agar kami
mematuhi kehendak-Mu dengan setia dan mengamalkannya dalam cara hidup
kami. Demi Yesus Kristus, Tuhan kami, kini dan sepanjang masa. Amin

Dilanjutkan dengan Rosario Roh Kudus ...

\subsection*{Hari Kelima}
Allah yang mahakuasa dan mahakudus, semoga Roh Kudus turun atas kami dan
berdiam dalam diri kami, sehingga kami menjadi kenisah kemuliaan-Nya.
Demi Yesus Kristus, Tuhan kami, kini dan sepanjang masa. Amin

Dilanjutkan dengan Rosario Roh Kudus ...

\subsection*{Hari Keenam}
Allah yang mahaesa, Engkau telah menghimpun Gereja dalam Roh Kudus.
Semoga kami mengabdi kepada-Mu dengan ikhlas dan bersatu padu dalam
cinta. Demi Yesus Kristus, Tuhan kami, kini dan sepanjang masa. Amin

Dilanjutkan dengan Rosario Roh Kudus ...

\subsection*{Hari Ketujuh}
Allah yang mahakudus, curahkanlah Roh Kudus-Mu ke dalam diri kami,
sehingga kami dapat melaksanakan kehendak-Mu dan layak menjadi
milik-Mu. Demi Yesus Kristus, Tuhan kami, kini dan sepanjang masa. Amin

Dilanjutkan dengan Rosario Roh Kudus ...

\subsection*{Hari Kedelapan}
Allah sumber cahaya kekal, Engkau telah membukakan bagi kami jalan
menuju hidup kekal dengan memuliakan Putra-Mu dan mengutus Roh Kudus.
Semoga cinta bakti dan iman kami selalu bertambah. Demi Yesus Kristus,
Tuhan kami, kini dan sepanjang masa. Amin

Dilanjutkan dengan Rosario Roh Kudus ...

\subsection*{Hari Kesembilan}
Allah yang mahakuasa, kebangkitan Putra-Mu telah menumbuhkan hidup baru
dalam diri kami. Semoga karena bantuan Roh-Mu kami mewujudkan rahmat
kebangkitan dalam hidup kami sehari-hari. Demi Yesus Kristus, Tuhan
kami, kini dan sepanjang masa. Amin

Dilanjutkan dengan Rosario Roh Kudus ...

\section{Rosario Roh Kudus}
Rosario Roh Kudus disusun pada tahun 1892 oleh seorang biarawan
Fransiskan Kapusin di Inggris sebagai sarana bagi umat beriman untuk
menghormati Roh Kudus. Doa ini kemudian memperoleh persetujuan
apostolik dari Paus Leo XIII pada tahun 1902. Rosario ini dimaksudkan
sebagai sarana untuk menghormati Roh Kudus, sama seperti Rosario Bunda
Maria di maksudkan para rahib Dominikan untuk menghormati Bunda Maria.

Rosario ini terdiri atas 5 kelompok manik-manik. Tiap kelompok terdiri
dari 7 manik. Sebelum dan sesudah tiap kelompok terdapat 2 butir manik
besar, sehingga seluruhnya ada 35 butir manik kecil dan 12 butir manik
besar. Sebagai tambahan, terdapat 3 manik kecil pada bagian permulaan.
Pada ketiga manik kecil ini dibuat tanda salib, lalu di daraskan doa
tobat dan himne datanglah Roh Pencipta.

Dalam tiap kelompok manik, diucapkan doa kemuliaan pada ketujuh manik
kecil, dan 1 doa Bapa Kami serta 1 Salam Maria pada kedua manik besar.
Pada 2 manik besar yang tersisa di bagian akhir, diucapkan Syahadat
Para Rasul (Aku percaya .....), doa Bapa Kami dan Salam Maria untuk
mendoakan Bapa Suci.

Pada doa ini terdapat 5 misteri: masing-masing misteri direnungkan pada
setiap kelompok manik-manik. Angka lima merupakan penghormatan atas
lima Luka Suci Yesus yang merupakan sumber rahmat yang dibagikan Roh
Kudus untuk seluruh umat manusia.

Secara berurutan, Rosario Roh Kudus di daraskan sebagai berikut:
\begin{enumerate}

\item Lagu Pembukaan
\item \Cross ~~Tanda salib
\item Doa Tobat

Datanglah Roh Pencipta~ Datanglah hai Roh Pencipta~ Kunjungilah jiwa
kami semua~ Penuhilah dengan rahmat-Mu~hati kami ciptaan-Mu.

Gelar-Mu ialah penghibur~ Rahmat Allah yang mahaluhur~ Sumber Hidup, Api
Kasih~dan Pengurapan Ilahi.

Engkaulah sumber sapta karunia~ Jemari tangan Sang Ilahi.

Engkaulah janji sejati Allah Bapa~yang mempergandakan bahasa.

Terangilah akal budi kami,~ Curahkan cinta di setiap hati.

Segala kelemahan kami~semoga Kau lindungi dan Kau kuatkan.

Jauhkanlah semua musuh segera,~ Anugerahkanlah kedamaian jiwa,~ Dengan
Engkau sebagai penuntun kami~ Kejahatan tak{\textquotesingle}kan
mempengaruhi.

Perkenalkanlah kami kepada Bapa~ Ajarilah agar kami mengakui Putra~serta
Engkau, 

Roh dari Keduanya~yang kami imani dan puji selamanya.

Segala kemuliaan bagi Allah Bapa~dan bagi Sang Putra~yang telah bangkit
dari mati~serta bagi-Mu Roh Kudus pula~sepanjang segala abad.

Amin
\item Misteri-misteri
\begin{enumerate}
\item  Misteri Pertama:{\textquotedbl}Dari Roh Kuduslah Yesus dikandung Perawan
Maria.{\textquotedbl}~ (Renungan Luk 1:35 )

Ujud khusus:~

\textit{Dengan tekun, mintalah bantuan dari Roh Ilahi serta perantaraan Bunda
Maria untuk mengikuti kebajikan-kebajikan Yesus Kristus, contohlah
segala kebajikan-Nya, sehingga kita dapat menjadi serupa dengan citra
Putra Allah.}

Renungan dan doa pribadi ...~ Bapa Kami ...~ Salam Maria ...~ Kemuliaan
... (7x)

\item Misteri Kedua:{\textquotedbl}Roh Allah turun atas Yesus.{\textquotedbl}~
(Renungan Mat3:16 )

Ujud khusus:~

\textit{Peliharalah dengan penuh kesungguhan anugrah yang tak ternilai, rahmat
pengudusan yang dicurahkan dan ditanamkan dalam jiwa kita oleh Roh
Kudus pada saat pembaptisan. Peganglah dengan teguh janji baptis yang
telah kita ucapkan: tingkatkan iman, harapan dan cinta kasih melalui
tindakan nyata, serta hiduplah sebagai anak-anak Allah dan anggota
Gereja Allah yang sejati agar kelak kita dapat memperoleh warisan
surgawi.}

Renungan dan doa pribadi ...~ Bapa Kami ...~ Salam Maria ...~ Kemuliaan
... (7x)

\item Misteri Ketiga:{\textquotedbl}Oleh Roh Kudus, Yesus dibimbing menuju
padang gurun untuk dicobai.{\textquotedbl}~ (Renungan Luk 4:1-2)

Ujud khusus:~

\textit{Bersyukurlah selalu atas ketujuh karunia Roh Kudus yang dicurahkan pada
kita saat menerima Sakramen Penguatan: Roh kebijaksanaan, pengertian,
nasihat, keperkasaan, pengenalan akan Allah, kesalehan, dan rasa takut
akan Allah. Serahkan diri kita dengan setia kepada bimbingan Ilahi-Nya,
sehingga di atas segala godaan dan pencobaan hidup kita berlaku secara
perkasa sebagai seorang Kristen sejati dan prajurit Kristus yang
berani.}

Renungan dan doa pribadi ...~ Bapa Kami ...~ Salam Maria ...~ Kemuliaan
... (7x)

\item Misteri Keempat :{\textquotedbl}Peranan Roh Kudus dalam
Gereja.{\textquotedbl}~ (Renungan Kis2:2 Kis 2:4 Kis 2:11 )

Ujud khusus:~

\textit{Bersyukurlah kepada Tuhan karena Ia menjadikan kita sebagai anggota
Gereja-Nya yang selalu dijiwai dan diarahkan oleh Roh Kudus, Roh yang
diturunkan ke dunia untuk tugas itu pada hari Pentekosta. Dengarlah dan
patuhilah Takhta Suci, wakil Roh Kudus yang tidak dapat salah, serta
Gereja, pilar dan dasar kebenaran. Junjunglah ajaran-ajarannya dan
belalah hak-haknya.}

Renungan dan doa pribadi ...~ Bapa Kami ...~ Salam Maria ...~ Kemuliaan
... (7x)

\item Misteri Kelima:{\textquotedbl}Roh Kudus dalam jiwa-jiwa orang
beriman.{\textquotedbl}~ (Renungan 1 Kor 6:19 1 Tes 5:19 Ef 4:30 )

Ujud khusus:~

\textit{Sadarilah keberadaan Roh Kudus dalam diri kita, peliharalah dengan
seksama kemurnian tubuh dan jiwa, ikutilah dengan setia bimbingan
Ilahi-Nya, sehingga kita dapat menghasilkan buah-buah Roh: kasih,
sukacita, damai sejahtera, kesabaran, kemurahan hati, kebaikan,
kesetiaan, kelemah lembutan, iman, kerendahan hati, penguasaan diri,
dan kemurnian.}

Renungan dan doa pribadi ...~ Bapa Kami ...~ Salam Maria ...~ Kemuliaan
... (7x)~  

Aku Percaya ...~ Bapa Kami ...~ Salam Maria ...
\end{enumerate}
\end{enumerate}

\section{Doa Umat}
Doa umat merupakan bentuk pelaksanaan imamat umum seluruh umat beriman. Doa umat mengakhiri liturgi sabda. Dalam doa umat, jemaat menanggapi sabda Allah yang telah mereka terima dengan penuh iman dan memohon secara resmi untuk keselamatan semua orang dan bukan hanya untuk diri sendiri dan kepentingan kelompok. Dengan demikian, mereka mengamalkan tugas imamat umum yang mereka peroleh dalam pembaptisan. Menurut ketentuan liturgi, doa umat dibawakan dari mimbar atau tempat lain yang sesuai oleh petugas, entah diakon, lektor, atau petugas awam lainnya. 
 
Pada umumnya urutan tradisional doa umat mencakup 4 hal:
\begin{enumerate}
\item  Doa bagi Gereja, khususnya para pemimpin Gereja
\item Doa bagi pemimpin masyarakat dan keselamatan dunia
\item Doa bagi orang-orang yang sedang menderita
\item Doa bagi jemaat setempat (paroki, stasi, wilayah,lingkungan)
\end{enumerate} 
 
Struktur - doa umat memiliki empat unsur:
\begin{enumerate}
\item \textbf{Pembuka}, berupa ajakan pemimpin yg ditujukan kepada jemaat. Pembuka ini bukanlah suatu doa yg dialamatkan kepada Tuhan.
\item \textbf{Usulan ujud dan undangan untuk berdoa}.
Usulan ujud ini disampaikan oleh petugas kepada jemaat, maka selalu diakhiri dengan ajakan "Marilah kita mohon" atau sejenisnya.

Sering terjadi dalam doa umat spontan; meliputi rumusan, alamat, dan isi yang tidak sesuai maksud. Rumusan: usulan ujud diubah menjadi doa. Alamat: kepada jemaat diubah kepada Allah. Isi: Tidak jarang doa umat berubah menjadi doa syukur.
Dalam situasi khusus, kita dapat menekankan ujud ini atau itu. Di samping ujud-ujud yang diucapkan, bisa juga diberikan kesempatan untuk ujud-ujud dalam hati.

\item \textbf{Aklamasi oleh jemaat}.
Inilah bagian yang sungguh berwujud doa. Rumusannya sangat singkat, diserukan jemaat kepada Tuhan: Tuhan, kabulkanlah doa kami; Tuhan, dengarkanlah doa kami; Tuhan, kasihanilah kami, dlsb.
\item \textbf{Penutup}, berbentuk doa singkat sebagai rangkuman atas semua permohonan.
\end{enumerate}

\section{Doa Syukur}
Ada beberapa langkah yang dapat diikuti dalam menyusun sebuah doa yang baik, yang lazim menurut kebiasaan gereja Katolik.

\begin{enumerate}[label=\textbf{\Roman*.}]
\item  \textbf{Sapaan}, mulailah dengan menyapa Allah sambil menyebutkan satu sifatNya (yang sesuai dengan bentuk dan isi doa yang akan dipanjatkan). 

Misalnya, 

\begin{itemize}
\item untuk doa mohon kesembuhan : \textit{Ya Allah, mahakuasa penuh kasih sayang, Engkaulah pemelihara kehidupan kami, jiwa dan raga, Engkaulah yang penuh kuasa dan belas kasihan \ldots }

\item untuk doa syukur ulang tahun : \textit{Ya Allah Bapa maha baik, Engkaulah pemegang tali kehidupan umat manusia, Engkaulah pencipta dan pemelihara kami \ldots}
\end{itemize}

Kita menyapa Allah dengan sifatNya: 
\begin{itemize}
\item \textbf{Esa}: dalam rangka kesatuan hidup, suami-isteri, kerukunan, pertemuan keluarga, pertemuan umat berbeda agama). 
\item \textbf{Maha kuasa}: dalam rangka ulang tahun kehidupan, pengalaman hidup yang khusus, cita-cita/niat/rencana, pelindung perjalanan, mohon keberhasilan suatu usaha baru, sakit). 

\item \textbf{Maha bijaksana}: menghadapi kesulitan, mencari penerangan/ bimbingan/Roh Kudus.  

\end{itemize}

\item \textbf{Isi doa}, sesudah menyapa Allah Bapa, sampaikanlah saat ini apa yang menjadi isi, ujud dari doa itu;  untuk : memuji dan bersyukur kepada Allah, memohon atau meminta sesuatu berkat/kemurahan, mempersembahkan diri/berserah kepada Allah, dll sesuai isi doa.

Contoh memuji/bersyukur : \textit{Kami memuji dan beryukur kepadaMu karena Engkau berkenan mengumpulkan kami bersama keluarga di sini sebagai umatMu. Kami bersyukur pula karena melalui ibadat ini, Kau tunjukkan kepada kami, bagaimana seharusnya kami membangun persaudaraan.}

Untuk keluarga : P\textit{andanglah keluarga yang datang berlindung dan bermohon kepadaMu. Mereka percaya akan diriMu, mereka berharap padaMu saja, mereka memanggil namaMu. Maka, \ldots dst, sesuai isi doa yang dimohonkan keluarga.}

Catatan :
\begin{enumerate}
\item Usahakanlah untuk mengaitkan doa dengan tema ibadat atau dengan pokok pertemuan. Jika pemimpin doa cukup terbiasa membaca Kitab Suci, maka ia dapat juga mengutip ayat-ayat Kitab Suci tertentu dalam doanya; misalnya : \textit{PutraMu Yesus Kristus telah bersabda : di mana dua atau tiga orang berkumpul demi namaKu, Aku ada di tengah-tengah mereka} \ldots atau \ldots \textit{Mintalah maka kamu akan diberi} \ldots. Atau \ldots \textit{RohKu akan Kucurahkan kepadamu} \ldots

\item Sejauh perlu kita dapat menyebutkan situasi alam, situasi khusus, tempat, di mana kita berada . Kita dapat menyebutkan peristiwa yang sementara dialami, dihadapi; dalam rangka apa \ldots Dapat juga menyebutkan siapa saja yang hadir. Hal ini membantu juga untuk menciptakan suatu suasana sehati sejiwa dari orang-orang yang berdoa bersama. Tetaplah berhati-hati untuk setia pada ujud doa. Jangan mencampurkan segala macam doa dalam satu doa. Misalnya : dalam doa makan, kita memfokuskan diri pada doa makan, jadi tidak perlu diselipi doa tobat atau permohonan ampun atas dosa dan salah.
 
\end{enumerate}
\item \textbf{Menutup} dengan rumusan penutup.

Kita memiliki Yesus Kristus sebagai Tuhan dan Pengantara kita, maka semua doa-doa kita diakhiri dengan mempersatukan doa-doa kita dengan Kristus Tuhan sendiri. Biasanya digunakan rumusan : \textit{Demi Kristus Tuhan dan Pengantara kami, yang hidup dan berkuasa kini dan sepanjang masa} \ldots Atau ditutup dengan rumusan penutup Trinitas: 
\textit{Inilah doa yang kami sampaikan kepadaMu dengan perantaraan Kristus PutraMu, pengantara kami, yang bersatu dengan Dikau dan Roh Kudus hidup dan berkuasa, kini dan sepanjang masa. Amin.}
\end{enumerate}
