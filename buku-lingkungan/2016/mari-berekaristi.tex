\section{Mari ber-Ekaristi dengan baik dan benar}
\begin{enumerate}
\item  Masuk ke
Gereja membuat tanda salib. Jangan terburu-buru, tetapi hayatilah dan syukurilah
bahwa karena rahmat Baptis anda bisa bergabung ke dlm persekutuan
Gereja. Jangan membiasakan memberi air suci pada orang lain dgn
mengulurkan jari anda. Ketika anda dibaptis anda dipanggil dgn nama
pribadi anda, berarti sgt personal, maka tanda salib jangan dibuat dgn
asal-asalan
\item Perayaan Ekaristi/ Misa Kudus adalah rangkaian doa.~Maka
tanda salib hanya dilakukan pada AWAL dan AKHIR MISA KUDUS saja yaitu
ketika imam memulai dan mengakhiri misa. Tanda Salib disini menunjuk pada tanda salib biasa dan bukan
penandaan dahi, bibir, dan dada dengan salib yg tetap harus dilakukan
saan bacaan injil. 
\item Ketika doa pembuka, sampaikanlah ujud pribadi
anda dalam hati, singkat saja sambil mengaminkan doa yg dibawakan
imam.~Tuhan sudah tahu masalah anda jadi tidak perlu bertele-tele. Pada
zaman dahulu, kesempatan ini diisi dgn doa spontan oleh umat yg hadir,
yg akhirnya ditutup oleh imam.(Kesempatan lain yg bisa dilakukan untuk
menyampaikan ujud pribadi adalah ketika doa umat, pada waktu yg
disediakan).
\item Tanda salib yg dibuat sebaiknya tanda salib besar,
yaitu dgn menyentuh pusar (sebagai lambang inkarnasi Kristus).~Tidak
membuat tanda salib ketika imam memberi absolusi umum
({\textquotedbl}...semoga Alah mengasihani
kita...dst..{\textquotedbl}), karena yg kita ikuti adalah Misa Kudus
bukan Sakramen Tobat. Tidak salah membuat tanda salib dengan menyentuh
dada ketika berkata {\textquotedbl}Putra{\textquotedbl}.  
\item Berlutut
sebelum duduk, jangan asal-asalan, jangan hanya membungkuk, kecuali
terpaksa.~Yang ada di depan anda adalah Kristus sebenar2nya dalam rupa
Hosti di Tabernakel. Ingatlah sejenak juga akan inkarnasi Kristus.
Hosti dalam Tabernakel, bisa diasosiasikan dgn Kristus dalam rahim
Maria. 

TENTANG PAKAIAN YANG PANTAS untuk menghadap Pencipta anda
sendiri yg ada secara fisik di hadapan anda, anda pasti bisa memilihnya
bukan?~ SEBERAPA SOPAN ANDA BERPAKAIAN MENCERMINKAN SEBERAPA TINGGI
PENGHORMATAN ANDA AKAN KRISTUS DALAM TABERNAKEL~  
\item Nyanyikanlah Tuhan
Kasihanilah kami dan Kemuliaan dengan penuh hormat.~Harap diingat bahwa
Kemuliaan adalah kidung malaikat di padang Efrata ketika kelahiran
Kristus. Jadi, mohon dinyanyikan dengan penuh sukacita dan hormat~  

\item Bacaan kitab suci yg dibacakan dr ambo (mimbar) adalah waktu Allah
berbicara dan kita mendengarkan, yaitu menyimak dengan penuh
perhatian.~Jika paroki anda menyediakan teks misa, anda lebih baik
membaca kutipan bacaan sebelum misa dimulai. TATAP lektor/imamnya
karena Allah sedang berbicara pada anda.Komunikasi yg baik dalam
percakapan adalah SALING MENATAP bukan? PEMBACAAN INJIL -dan bukannya
homili - adalah PUNCAK LITURGI SABDA. Harap diingat, suara yg anda
dengar adalah Suara Kristus sendiri karena imam bertindak IN PERSONA
CHRISTI (mewakili Kristus sepenuh-penuhnya)~  
\item Mohon menyanyikan
KUDUS dengan sepenuh hati, dengan keagungan, jangan asal2an.Dikarenakan
bahwa ketika menyanyikan/mengucapkan KUDUS kita bergabung dengan
seluruh penghuni surga yang memuji Allah tak henti.~  
\item Ketika
konsekrasi (Inilah TubuhKU, Inilah DarahKu atau ketika Hosti diangkat
dan Piala diangkat) anda boleh mengangkat kedua tangan yg terkatup
seperti ritus ibadat di pura Hindu, NAMUN SEBENARNYA berlutut sudah
merupakan ungkapan PENYEMBAHAN.Yang terpenting ketika konsekrasi adalah
anda harus menatapNya. Harap diingat, Suara yg anda dengar (Inilah
TubuhKU, Inilah darahKU, adalah Suara Kristus sendiri. Lagi, hal ini
dikarenakan Imam bertindak IN PERSONA CHRISTI. Jadi? Tataplah Hosti dan
Piala itu dgn penuh hormat, yakinkan pada diri anda kalau itu adalah
Kristus sendiri, bukannya sibuk dengan permohonan dalam hati.~  

\item Ketika imam mengucapkan/menyanyikan : {\textquotedbl}Dengan perantaraan
Kristus, bersama dia, dan dalam Dia...dst...{\textquotedbl} IKUTILAH
DALAM HATI. TATAPLAH HOSTI DAN PIALA YG DIANGKAT.~Ketika
{\textquotedbl}AMIN{\textquotedbl} dinyanyikan (dlm bahasa inggris
disebut THE GREAT AMEN{\textquotedbl}). Mohon dinyanyikan dengan
sepenuh hati, dengan suara terindah yg anda miliki. Dikarenakan bahwa
THE GREAT AMEN ini adalah PUNCAK LITURGI EKARISTI.~  
\item Jangan menadahkan tangan seperti imam, pada waktu berdoa atau menyanyikan Bapa
Kami.~Dikarenakan imam sedang berdoa atas nama Gereja atau IN PERSONA
ECCLESIA. Sikap yg benar adalah mengatupkan tangan, tanda berdoa.
Hayatilah doa Bapa Kami. Sadarilah bahwa
{\textquotedbl}rezeki{\textquotedbl} yg anda minta itu terutama adalah
{\textquotedbl}Roti Hidup{\textquotedbl} dalam Ekaristi. (dlm bahasa
aslinya (Aram), doa Bapa Kami menggunakan kata
{\textquotedbl}roti{\textquotedbl} bukan rezeki. Pun,dalam bahasa latin
digunakan kata {\textquotedbl}PANEM{\textquotedbl} yg berarti roti.)~ 

\item TIDAK MENGUCAPKAN DOA PRESIDENSIAL (yg boleh diucapkan oleh imam
saja) doa: {\textquotedbl}..jangan perhitungkan dosa kami tetapi
perhatikanlah iman GerejaMu{\textquotedbl}~Jika Imam mengucapkan
{\textquotedbl}marilah kita mohon damai Tuhan{\textquotedbl} dsb
sebelum doa ini, bukan berarti kita harus ikut mengucapkan doa ini.
Ucapkan dalam hati saja KEMUDIAN DIAMINKAN DENGAN IMAN.~  
\item Ketika menerima komuni, TATAPLAH terlebih dahulu hosti yg diangkat sebelum
ditaruh di tangan anda. AMIN HARUS DIUCAPKAN DENGAN PENUH IMAN.~  
\item Tidak perlu ikut menghormat ketika imam menghormati Tabernakel dan
altar (juga pada waktu awal misa). Tidak masalah jika anda tetap
melakukannya karena merupakan kebiasaaan yg saleh.~Namun kalau anda
menghadiri misa di luar negeri, jangan kaget kalau di negara tertentu
praktik ini tidak dilakukan.~  
\item Tanda salib pada saat keluar Gereja,
sebenarnya tidak perlu dilakukan. Tanda salib sebelum anda masuk
sebenarnya kurang lebih berfungsi seperti wudhu, yaitu untuk menyucikan
(dan mengingatkan akan Baptis). Ketika anda selesai misa, Kristus yang
Maha Suci sudah masuk dalam tubuh anda, tidak diperlukan lagi sarana
penyucian lain. Namun demikian, tidak ada salahnya kalau dilakukan,
asal jangan karena latah, namun harus disertai kesadaran iman, bahwa anda
kini diutus untuk mewartakan karya salib Kristus lewat perkataan dan
perbuatan.~  
\end{enumerate}


Anda dapat menjadi contoh bagi orang lain. Anda dapat mensosialisasikan hal-hal di atas pada siapa saja yg menghadiri misa bersama Anda.

\textbf{Tambahan} : Info ini BUKAN TPE BARU. TPE yg berlaku
tetap TPE 2005. Info ini hanya merupakan hasil olahan. Coba perhatikan
dengan seksama bahwa sama sekali tidak ada yg berubah. Yang ditulis di
atas lebih ke arah praktikal, terutama bagaimana sebenarnya menghayati
apa yg kita lakukan atau katakan atau nyanyikan setiap kali kita
menghadiri Misa.~  

Sampaikan dengan sopan pada saudara dari persekutuan
gerejawi lain (Protestan) agar mereka tidak ikut mengambil komuni,
namun boleh menerima berkat seperti katekumen yaitu dengan menyilangkan
tangan di depan dada, sehingga yang memberikan komuni tahu bahwa dia
bukanlah seorang katolik. Walaupun mereka tergabung dalam semacam
persekutuan dengan Gereja Katolik berkat Sakramen Baptis, namun komuni
hanya diperuntukkan bagi mereka yg berada dalam persekutuan penuh
dengan Uskup Roma (Paus sebagai penerus Petrus), dengan kata lain
komuni hanya eksklusif untuk umat Katolik.~  

Tambahan bagi perempuan
katolik: Jangan merasa terhalang menerima komuni jika anda sedang
mengalami datang bulan. Tuhan Yesus tidak mempermasalahkan sesuatu yg
manusiawi. Konsep terhalang karena datang bulan tidak ada di dalam Gereja Katolik.
