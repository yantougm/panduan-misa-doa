\section{Aturan dan Kebijakan Lingkungan St. Theresia \tahun} 

\begin{itemize}
\item Iuran lingkungan tetap Rp.6.000,- /KK/bulan yang
terdiri dari iuran wajib Rp.3.000,- dan iuran sosial Rp.3.000,-.
\item Bagi warga St. Theresia yang opname di rumah sakit mendapat sumbangan sebesar
Rp. 100.000,-/org/tahun. Bila umat berkenan menambah sumbangan
lingkungan tersebut dengan melakukan sumbangan serkiler, maka
dipersilahkan untuk mengumpulkan dana pribadi tanpa adanya suatu
pemaksaan. 
\item Jika ada warga dari lingkungan lain yang sakit (opname), kita
bersepakat untuk membezuk tanpa ada dana tunjangan dari Lingkungan St.
Theresia, tetapi tali kasih yang diberikan melalui dana serkiler
(pribadi).
\item Permintaan (ujud) misa dari umat Lingk. St. Theresia. Bila
menghendaki diiringi koor dari Lingkungan, maka biaya konsumsi selama
latihan koor menjadi tanggungan dari umat yg meminta ujud.
\item Jika lingkungan St. Petrus123 mendapat tugas untuk koor di Gereja,
maka selama latihan koor biaya konsumsi ditanggung oleh lingkungan St.
Petrus123 bersama atau bergantian. 
\item Untuk jaringan informasi via sms, tiap warga berkewajiban untuk
menyampaikan informasi tersebut sesuai dengan jalurnya yang sudah
ditentukan.
\end{itemize}

\section{Tata cara persiapan dan pelaksanaan ujud  misa/ibadat pribadi}
\label{sec:tataCara}
\begin{enumerate}
\item Persiapan (oleh umat bersangkutan dan pengurus lingkungan):

\begin{enumerate}
\item penentuan waktu oleh umat
\item menghubungi Romo/petugas
\item persiapan koor (bila ada)
\item persiapan peralatan misa (bila ada misa)
\item pembuatan dan pengedaran undangan
\end{enumerate}
\item Pelaksanaan (oleh umat bersangkutan bersama dengan pengurus
lingkungan):

\begin{enumerate}
\item pengaturan tempat
\item pengaturan Altar
\item penjemputan Romo/petugas (bila perlu)
\item pelaksanaan misa/ibadat
\item penyerahan stipendium atau iura stolae untuk Romo
\item penggantian biaya hosti dan anggur
\end{enumerate}
\end{enumerate}
Catatan:~Segala kegiatan doa/misa pribadi yang dipersiapkan dan
dilaksanakan sendiri (tanpa melibatkan Lingkungan) dengan melibatkan
banyak umat, keluarga bersangkutan wajib memberikan laporan kepada
Ketua Lingkungan untuk diteruskan ke Paroki.
