\section{Doa NOVENA Roh Kudus}
\scriptsize
    Umat Kristen mempunyai kebiasaan mengadakan doa Novena Roh Kudus. Ini dilaksanakan selama sembilan hari (novena = sembilan), mulai pada hari sesudah kenaikan Tuhan yesus ke surga dan berakhir pada hari Sabtu menjelang Pentekosta. dalam doa ini umat Kristen memuji Tuhan yang menjanjikan kedatangan Roh Kudus dan memohon rahmat Allah agar siap menyambut kedatangan Roh Kudus. Doa ini juga bisa dilaksanakan pada kesempatan lain yang cocok. Yang tersaji disini lebih dimaksudkan untuk didoakan dalam kelompok; kalau didoakan secara pribadi, dapat disesuaikan seperlunya.

    Kalau Novena ini dipadukan dengan Perayaan Ekaristi, sesudah Mohon Tujuh Karunia Roh Kudus menyusul Liturgi Ekaristi (persembahan, Doa syukur Agung, dan seterusnya)
 
\normalsize
\subsection*{Hari Pertama}

    Allah pokok keselamatan kami, karena kebangkitan Kristus kami lahir kembali dalam pembabtisan dan menjalani hidup baru. Arahkanlah hati kami kepada Kristus yang kini duduk di sebelah kanan-Mu. Semoga Roh-Mu menjaga kami sampai Penyelamat kami datang dalam kemuliaan, sebab Dialah Tuhan, Pengantara kami, kini dan sepanjang masa. Amin
    
    \emph{Dilanjutkan dengan Rosario Roh Kudus ...}
    
\subsection*{Hari Kedua}

    Allah yang mahabijaksana, Putra-Mu menjanjikan Roh Kudus kepada para rasul dan memenuhi janji itu sesudah Dia naik ke surga. Semoga kami pun Kau anugrahi karunia Roh Kudus. Demi Yesus Kristus, Pengantara kami, kini dan sepanjang masa. Amin
    
    \emph{Dilanjutkan dengan Rosario Roh Kudus ...}
    
\subsection*{Hari Ketiga}

    Allah, Penyelamat kami, kami percaya bahwa Kristus telah bersatu dengan Dikau dalam keagungan. Semoga dalam Roh-Nya, Dia selalu menyertai kami sampai akhir zaman, seperti yang dijanjikan-Nya. Sebab Dialah Tuhan kami, kini dan sepanjang masa. Amin
    
    \emph{Dilanjutkan dengan Rosario Roh Kudus ...}
    
\subsection*{Hari Keempat}

    Allah yang mahakudus, semoga kekuatan Roh-Mu turun atas kami, agar kami mematuhi kehendak-Mu dengan setia dan mengamalkannya dalam cara hidup kami. Demi Yesus Kristus, Tuhan kami, kini dan sepanjang masa. Amin
    
    \emph{Dilanjutkan dengan Rosario Roh Kudus ...}
    
\subsection*{Hari Kelima}

    Allah yang mahakuasa dan mahakudus, semoga Roh Kudus turun atas kami dan berdiam dalam diri kami, sehingga kami menjadi kenisah kemuliaan-Nya. Demi Yesus Kristus, Tuhan kami, kini dan sepanjang masa. Amin
    
    \emph{Dilanjutkan dengan Rosario Roh Kudus ...}
    
\subsection*{Hari Keenam}

    Allah yang mahaesa, Engkau telah menghimpun Gereja dalam Roh Kudus. Semoga kami mengabdi kepada-Mu dengan ikhlas dan bersatu padu dalam cinta. Demi Yesus Kristus, Tuhan kami, kini dan sepanjang masa. Amin
    
    \emph{Dilanjutkan dengan Rosario Roh Kudus ...}

\subsection*{Hari Ketujuh}

    Allah yang mahakudus, curahkanlah Roh Kudus-Mu ke dalam diri kami, sehingga kami dapat melaksanakan kehendak-Mu dan layak menjadi milik-Mu. Demi Yesus Kristus, Tuhan kami, kini dan sepanjang masa. Amin
    
    \emph{Dilanjutkan dengan Rosario Roh Kudus ...}
    
\subsection*{Hari Kedelapan}

    Allah sumber cahaya kekal, Engkau telah membukakan bagi kami jalan menuju hidup kekal dengan memuliakan Putra-Mu dan mengutus Roh Kudus. Semoga cinta bakti dan iman kami selalu bertambah. Demi Yesus Kristus, Tuhan kami, kini dan sepanjang masa. Amin
    
    \emph{Dilanjutkan dengan Rosario Roh Kudus ...}
    
\subsection*{Hari Kesembilan}

    Allah yang mahakuasa, kebangkitan Putra-Mu telah menumbuhkan hidup baru dalam diri kami. Semoga karena bantuan Roh-Mu kami mewujubkan rahmat kebangkitan dalam hidup kami sehari-hari. Demi Yesus Kristus, Tuhan kami, kini dan sepanjang masa. Amin
    
    \emph{Dilanjutkan dengan Rosario Roh Kudus ...}
    
\section{ROSARIO ROH KUDUS}

\scriptsize
Rosario Roh Kudus disusun pada tahun 1892 oleh seorang biarawan Fransiskan Kapusin di Inggris sebagai sarana bagi umat beriman untuk menghormati Roh Kudus. Doa ini kemudian memperoleh persetujuan apostolik dari Paus Leo XIII pada tahun 1902. Rosario ini dimaksudkan sebagai sarana untuk menghormati Roh Kudus, sama seperti Rosario Bunda Maria di maksudkan para rahib Dominikan untuk menghormati Bunda Maria.

Rosario ini terdiri atas 5 kelompok manik-manik. Tiap kelompok terdiri dari 7 manik. Sebelum dan sesudah tiap kelompok terdapat 2 butir manik besar, sehingga seluruhnya ada 35 butir manik kecil dan 12 butir manik besar. Sebagai tambahan, terdapat 3 manik kecil pada bagian permulaan. Pada ketiga manik kecil ini dibuat tanda salib, lalu di daraskan doa tobat dan himne datanglah Roh Pencipta.

Dalam tiap kelompok manik, diucapkan doa kemuliaan pada ketujuh manik kecil, dan 1 doa Bapa Kami serta 1 Salam Maria pada kedua manik besar. Pada 2 manik besar yang tersisa di bagian akhir, diucapkan Sahadat Para Rasul (Aku percaya .....), doa Bapa Kami dan Salam Maria untuk mendoakan Bapa Suci.

Pada doa ini terdapat 5 misteri: masing-masing misteri direnungkan pada setiap kelompok manik-manik. Angka lima merupakan penghormatan atas lima Luka Suci Yesus yang merupakan sumber rahmat yang dibagikan Roh Kudus untuk seluruh umat manusia.

\normalsize
Secara berurutan, Rosario Roh Kudus di daraskan sebagai berikut:

{\bf Tanda salib}

{\bf Doa Tobat}

Datanglah Roh Pencipta\\
Datanglah hai Roh Pencipta\\
kunjungilah jiwa kami semua\\
penuhilah dengan rahmat-Mu\\
hati kami ciptaan-Mu.

Gelar-Mu ialah penghibur\\
rahmat Allah yang mahaluhur\\
Sumber Hidup, Api Kasih\\
dan Pengurapan Ilahi.

Engkaulah sumber sapta karunia\\
jemari tangan Sang Ilahi.

Engkaulah janji sejati Allah Bapa\\
yang mempergandakan bahasa.

Terangilah akal budi,\\
curahkan cinta di setiap hati.

Segala kelemahan kami\\
semoga Kau lindungi dan Kau kuatkan.

Jauhkanlah semua musuh segera,\\
anugrahkanlah kedamaian jiwa,\\
dengan Engkau sebagai penuntun kami\\
kejahatan tak'kan mempengaruhi.

Perkenalkanlah kami kepada Bapa\\
ajarilah agar mengakui Putra\\
serta Engkau, Roh dari Keduanya\\
yang kami imani dan puji selamanya.

Segala kemuliaan bagi Allah Bapa\\
dan bagi Sang Putra\\
yang telah bangkit dari mati\\
serta bagi-Mu Roh Kudus pula\\
sepanjang segala abad.

Amin

Misteri Pertama:"Dari Roh Kuduslah Yesus dikandung Perawan Maria."\\
(Renungan Luk1:35 )

{\bf Ujub khusus:}\\
Dengan tekun, mintalah bantuan dari Roh Ilahi serta perantaraan Bunda maria untuk mengikuti kebajikan-kebajikan Yesus Kristus, contohlah segala kebajikan-Nya, sehingga kita dapat menjadi serupa dengan citra Putra Allah.

{\it Renungan dan doa pribadi ...\\
Bapa Kami ...\\
Salam Maria ...\\
Kemuliaan ... (7x)}

Misteri Kedua:"Roh Allah turun atas Yesus."\\
(Renungan Mat3:16 )

{\bf Ujub khusus:}\\
Peliharalah dengan penuh kesungguhan anugrah yang tak ternilai, rahmat pengudusan yang dicurahkan dan ditanamkan dalam jiwa kita oleh Roh Kudus pada saat pembabtisan. Peganglah dengan teguh janji baptis yang telah kita ucapkan: tingkatkan iman, harapan dan cinta kasih melalui tindakan nyata, serta hiduplah sebagai anak-anak Allah dan anggota Gereja Allah yang sejati agar kelak kita dapat memperoleh warisan surgawi.

{\it Renungan dan doa pribadi ...\\
Bapa Kami ...\\
Salam Maria ...\\
Kemuliaan ... (7x)}

Misteri Ketiga:"Oleh Roh Kudus, Yesus dibimbing menuju padang gurun untuk dicobai."\\
(Renungan Luk4:1-2)

{\bf Ujub khusus:}\\
Bersyukurlah selalu atas ketujuh karunia Roh Kudus yang dicurahkan pada kita saat menerima Sakramen Penguatan: Roh kebijaksanaan, pengertian, nasihat, keperkasaan, pengenalan akan Allah, kesalehan, dan rasa takut akan Allah. Serahkan diri kita dengan setia kepada bimbingan Ilahi-Nya, sehingga di atas segala godaan dan pencobaan hidup kita berlaku secara perkasa sebagai seorang Kristen sejati dan prajurit Kristus yang berani.

{\it Renungan dan doa pribadi ...\\
Bapa Kami ...\\
Salam Maria ...\\
Kemuliaan ... (7x)}

Misteri Keempat:"Peranan Roh Kudus dalam Gereja."\\
(Renungan Kis2:2 Kis2:4 Kis2:11 )

{\bf Ujub khusus:}\\
Bersyukurlah kepada Tuhan karena Ia menjadikan kita sebagai anggota Gereja-Nya yang selalu dijiwai dan diarahkan oleh Roh Kudus, Roh yang diturunkan ke dunia untuk tugas itu pada hari Pentekosta. Dengarlah dan patuhilah Takhta Suci, wakil Roh Kudus yang tidak dapat salah, serta Gereja, pilar dan dasar kebenaran. Junjunglah ajaran-ajarannya dan belalah hak-haknya.

{\it Renungan dan doa pribadi ...\\
Bapa Kami ...\\
Salam Maria ...\\
Kemuliaan ... (7x)}

Misteri Kelima:"Roh Kudus dalam jiwa-jiwa orang beriman."\\
(Renungan 1Kor6:19 1Tes5:19 Ef4:30 )

{\bf Ujub khusus:}\\
Sadarilah keberadaan Roh Kudus dalam diri kita, peliharalah dengan seksama kemurnian tubuh dan jiwa, ikutilah dengan setia bimbingan Ilahi-Nya, sehingga kita dapat menghasilkan buah-buah Roh: kasih, sukacita, damai sejahtera, kesabaran, kemurahan hati, kebaikan, kesetiaan, kelemah lembutan, iman, kerendahan hati, penguasaan diri, dan kemurnian.

{\it Renungan dan doa pribadi ...\\
Bapa Kami ...\\
Salam Maria ...\\
Kemuliaan ... (7x)\\ \\
Aku Percaya ...\\
Bapa Kami ...\\
Salam Maria ...}    