\newpage
\section{Tata Urutan Ibadat Lingkungan}

\begin{description}
\item [Nyanyian Pembukaan]
      {\it untuk membuka ibadat, mempersatukan umat.  Hendaknya dinyanyikan bersama.}
\item [Tanda Salib]
		{\it menyadari Tuhan hadir di antara kita.}
\item [Tema/Pengantar]
		{\it menjelaskan tujuan ibadat.}	
		
\item [Doa Tobat]
		{\it membuka hati bagi Tuhan yang hadir agar Ia mempersatukan kita yang tercerai-berai. Dapat juga diganti dengan doa syukur, misalnya mazmur.}
		
\item [Doa Pembukaan]
		{\it menyapa Allah Bapa secara resmi.}
		
\item [Ibadat Sabda]
		{\it menyadari Tuhan hadir dalam Sabda-Nya.}
		\begin{description}
		\item [Bacaan I (dan Bacaan II, bila perlu)]
				{\it mendengarkan Sabda Allah melalui Perjanjian Lama atau surat rasul.}
		\item [Nyanyian Renungan]
				{\it merenungkan kembali Sabda Allah. Hendaknya sesuai dengan bacaan.}
		\item [Bacaan Injil]
				{\it mendengarkan Sabda Yesus Kristus.
				\begin{itemize}
				\item Tuhan sertamu \dots
				\item Inilah Injil Suci \dots
				\item Demikian Sabda Tuhan \dots
				\end{itemize}
				}
		\item [Homili]
				{\it menyadari Sabda Allah bagi hidup kita. Dapat juga diadakan tukar menukar pengalaman iman, tetapi bukan diskusi.}
		\end{description}				

\item [Kolekte]
		{\it untuk pengumpulan dana bagi kebutuhan umat. Diiringi dengan nyanyian.}
\item [Doa Umat]
		{\it menjawab Sabda Allah dengan mohon agar terlaksana dalam hidup, mendoakan kepentingan kita bersama. Doa umat ditutup dengan:}
\item [Bapa Kami]
		{\it bersatu sebagai anak Allah dalam doa Kristus.}
\item [Penutup dan Doa Penutup]
		{\it menyadari tugas perutusan dalam hidup di dunia. Secara resmi berterima kasih pada Allah dan sanggup untuk melaksanakan kehendakNya.}
\item [Berkat]
		{\it mohon bantuan bagi pelaksanaan tugas kita di dunia.}
\item [Nyanyian penutup]				
		{\it berterima kasih pada Tuhan atas apa yang kita terima dalam ibadat.}
\end{description}