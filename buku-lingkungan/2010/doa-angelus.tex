\section{DOA ANGELUS}

\emph{Maria diberi kabar oleh Malaikat TUHAN} \\
bahwa Ia akan mengandung dari Roh Kudus \\
Salam Maria ...\\ \\
\emph{Aku ini hamba TUHAN}\\
terjadilah padaku menurut perkataanMU.\\
Salam Maria ... \\ \\
\emph{Sabda sudah menjadi daging}\\
dan tinggal diantara kita\\
Salam Maria\\ \\
\emph{Doakanlah kami, ya Santa Bunda ALLAH}\\
supaya kami dapat menikmati janji KRISTUS.\\ \\

\emph{Marilah berdoa (hening sejenak)}\\
Ya ALLAH, karena kabar Malaikat kami mengetahui\\
bahwa YESUS KRISTUS PutraMU menjadi manusia.\\
Curahkanlah rahmatMU ke dalam hati kami,\\
supaya karena sengsara dan salibNYA,\\
kami dibawa kepada kebangkitan yang mulia.\\
Sebab DIAlah TUHAN dan Pengantara kami \\
(Amin)\\

\section{DOA RATU SURGA (dalam Masa Paskah)}

\emph{Ratu Surga bersukacitalah, alleluya,}\\
sebab Ia yang sudi kau kandung, alleluya,\\
\emph{telah bangkit seperti disabdakan-Nya, alleluya!}\\
Doakanlah kami pada Allah, alleluya!\\
\emph{Bersukacita dan bergembiralah, Perawan Maria, alleluya,}\\
sebab Tuhan sungguh telah bangkit, Alleluya!\\

\emph{Marilah berdoa (hening sejenak)}\\
Ya Allah, \\
Engkau telah menggembirakan dunia dengan kebangkitan PutraMu,\\ 
Tuhan kami Yesus Kristus. \\
Kami mohon, \\
perkenankanlah kami bersukacita dalam kehidupan kekal bersama BundaNya, Perawan Maria. \\
Demi Kristus, pengantara kami. \\
Amin.


\subsection*{SEJARAH DOA ANGELUS}
\scriptsize

Kita mengenal tradisi doa Angelus yang kita doakan pada jam 6 pagi, jam 12 siang dan jam 6 sore.
Doa ini mempunyai 2 rumusan yakni rumusan untuk dipakai pada masa Paskah dan rumusan untuk masa di luar Paskah.
Di Indonesia doa ini mulanya penggunaannya masih terbatas pada kalangan kaum religius dan rohaniawan-rohaniwati.
Akhir-akhir ini, doa Angelus sudah semakin sering didoakan oleh umat awam.

\subsubsection*{Arti}
"Angelus" berarti "Malaikat".

\subsubsection*{Mengapa dinamakan Doa Angelus?}
Dinamakan Angelus karena kata ini merupakan kata pertama dari "Maria diberi kabar oleh Malaikat"
Yang dalam bahasa latinnya adalah "Angelus domini nuntiavit Mariae"

{~}\newpage \thispagestyle{empty}{~} \newpage \thispagestyle{empty}{~} \newpage {~}

\subsubsection*{Sejarah Doa Angelus}
Doa Angelus sore hari dimulai pada abad ke-13 di Eropa.
Oleh karena itu doa Angelus sore hari ini yang pertama kali digunakan.
Selanjutnya pada pertengahan abad ke-14 barulah doa Angelus pagi hari digunakan di seluruh Eropa.
Doa Angelus pagi dan sore hari didoakan oleh para rahid sebagai bagian dari doa pagi dan doa malam di biara-biara.
Diawali dengan doa Angelus kemudian dilanjutkan doa-doa harian para rahib biara.
Kemudian pada antara abad 14-15, barulah doa Angelus pada siang hari muncul dan mulai didoakan.

\normalsize
\subsubsection*{Tujuan Doa Angelus}

\begin{description}
\item[Doa Angelus jam 6 pagi]
Menghormati kebangkitan Kristus.

Yesus yang telah bangkit dan bersama Kristus kita memulai dari dengan semangat kebangkitan.

\item[Doa Angelus jam 12 siang]
Menghormati sengsara Kristus.

Di tengah pekerjaan kita yang berat, kita senantiasa ingat Kristus yang telah berkorban bagi kita.

\item[Doa Angelus jam 6 sore]
Menghormati Inkarnasi Allah menjadi manusia.

Pada saat kita beranjak untuk beristirahat, ingatlah bahwa Allah selalu tinggal beserta kita.
\end{description}