\section{Tata Urutan Ibadat Lingkungan} 

\begin{enumerate}
\item {\large \textbf{Pembukaan}}
\begin{itemize}
\item \begin{description}
\item[Nyanyian Pembukaan]
Untuk membuka ibadat, mempersatukan umat. Hendaknya dinyanyikan bersama.
\end{description}

\item \begin{description}
\item[\Cross ~~Tanda Salib]
Menyadari Tuhan hadir di antara kita.
\end{description}

\item \begin{description}
\item[Tema/Pengantar]: Menjelaskan tujuan/tema ibadat secara singkat, dan
mengajak umat untuk mempersiapkan batin memohon pengampunan dari Tuhan
agar layak di hadapanNya.
\end{description}
\item \begin{description}
\item [Doa Tobat]: Memohon ampun dan membuka hati bagi Rahmat Allah. Dapat
juga diganti dengan doa syukur, misalnya mazmur.
\end{description}
\item \begin{description}
\item [Doa Pembuka]: Menyapa Allah Bapa secara resmi.
\end{description}
\end{itemize}

\item {\large \textbf{Ibadat Sabda}}
\begin{itemize}

\item \begin{description}
\item[Bacaan] : Mendengarkan Sabda Allah melalui Perjanjian Lama atau
surat rasul.
\end{description}
\item \begin{description}

\item[Nyanyian Renungan]: Hendaknya sesuai dengan bacaan dan akan lebih baik jika mengacu pada Mazmur yang sesuai. 
\end{description}
\item \begin{description}
\item[Bacaan Injil]
\end{description}

\begin{itemize}
\item Semoga Tuhan beserta kita {\dots}{\dots}{\dots}
\item Inilah Injil Suci tulisan {\dots}{\dots}{\dots}
\item Demikian Injil Tuhan {\dots}{\dots}{\dots}
\end{itemize}
\item \begin{description}
\item[Khotbah/Homili/Sharing]

Menyadari Sabda Allah bagi hidup kita. Dapat juga diadakan tukar-menukar
pengalaman iman, tetapi bukan diskusi.

\end{description}
\end{itemize}
\item \begin{description}
\item[Aku Percaya]
\end{description}
\item \begin{description}
\item[Doa Umat ]
\end{description}
\begin{itemize}
\item Doa pengantar doa Umat 
\item Doa Umat
\item Doa Penutup Doa Umat
\end{itemize}

\item \begin{description}
\item[Bapa Kami]: Bersatu sebagai anak Allah dalam doa yang diajarkan
Kristus sendiri.
\end{description}
\item \begin{description}
\item[Penutup] : Menyadari tugas perutusan dalam hidup di dunia. Secara
resmi berterima kasih pada Allah dan sanggup untuk melaksanakan
kehendakNya.
\end{description}
 \begin{itemize}
\item \textbf{Doa Penutup dan Mohon Berkat}

Mohon bantuan dan berkat Allah Bapa bagi pelaksanaan tugas kita di
dunia.

\item \textbf{Nyanyian penutup} sekaligus mengiringi Kolekte.

\end{itemize}
\end{enumerate}

\section[Tata urutan Doa Rosario]{Tata urutan Doa Rosario}
Tata urutan Doa Rosario
\begin{enumerate}[label=\Roman*.]

\item  \textbf{Pembuka:}oleh Petugas Doa Rosario
\begin{itemize}
\item \textbf{Lagu Pembuka}
\item  \textbf{\Cross ~~Tanda Salib}
\item \textbf{Pengantar}:
\begin{itemize}
\item Menyampaikan Peristiwa yang ingin diambil dalam doa ini.
\item Mengajak umat untuk mempersiapkan bathin.

\end{itemize}	
\item \textbf{Tobat: }(Ungkapan Tobat)
\item \textbf{Doa Pembuka:} (Misalnya: memohon agar Allah berkenan mendengarkan segala doa kita yang akan kita sampaikan/doakan dengan perantaraan/bersama Bunda Maria).

\end{itemize}

\item  \textbf{Bacaan Injil}

Lagu Pengantar masuk ke suasana Rosario (Lagu Maria)

\item  \textbf{Doa Rosario: }oleh Petugas Doa Rosario
\begin{itemize}
\item \textbf{Aku Percaya} \ldots
\item \textbf{Kemuliaan} \ldots
\item \textbf{Bapa Kami }\ldots
\item \textbf{3 Salam} (Puteri Allah Bapa, Bunda Allah Putera, Mempelai Allah RohKudus), Salam Maria \ldots
\item \textbf{Kemuliaan }\ldots
\item \textbf{Terpujilah} \ldots
\item \textbf{Peristiwa Rosario} 

(Sebutkan 'Peristiwanya'. Lihat Tema Peristiwa sesuai dengan tema masing-masing hari)

\begin{enumerate}[label=\alph*.]
\item  \textbf{Tiga Misteri Kudus }
(3x persepuluhan pertama)
\begin{itemize}
\item Sebutkan Misteri Kudus, dan dilanjutkan dengan ujudnya
\item Bapa kami \ldots
\item Salam Maria  \ldots(10x)
\item Kemuliaan  \ldots
\item Terpujilah  \ldots
\item Ya Yesus yang Baik  \ldots

(Setelah doa 'Ya Yesus yang Baik', dilanjutkan dengan Misteri Kudus berikutnya dengan urutan doa yang sama seperti di atas.)

\end{itemize}

\item \textbf{Lagu Selingan} 

\item  \textbf{Dua Misteri Kudus} 

(2x persepuluhan ke dua). Urutan doanya sama seperti di atas.


\end{enumerate}
\end{itemize}


\item  \textbf{Ibadat Penutup}
\begin{itemize}
\item Doa Penutup dan mohon berkat Tuhan
\item Kolekte dan Lagu Penutup. 

Lagu Penutup selain untuk mengakhiri doa Rosario juga mengiringi Kolekte.
\end{itemize}
\end{enumerate}
