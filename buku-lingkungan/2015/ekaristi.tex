\chapter{Ekaristi}
\section[Pendahuluan]{Pendahuluan}
Karena Ekaristi adalah Yesus Kristus sendiri, Ekaristi menjadi
{\textquoteleft}jantung{\textquoteright} dari iman Katolik. Katekismus
Gerja Katolik mengajarkan bahwa Ekaristi adalah
{\textquotedblleft}\textbf{sumber dan puncak seluruh kehidupan
Kristiani}{\textquotedblright} (KGK 1324) dan
{\textquotedblleft}\textbf{hakikat dan rangkuman
iman}~kita{\textquotedblright} (KGK 1327). Tentu idealnya semua orang
Katolik mengetahui hal ini, tetapi sayangnya, kenyataan berbicara lain.
Di Amerika, menurut polling pendapat yang diadakan
oleh~\emph{Gallup~poll}~pada tahun 1992, pengertian ini tidak dimiliki
oleh sebagian besar umat Katolik.[1] Hal yang serupa mungkin pula
terjadi di Indonesia.

Hasil yang diperoleh cukup menggambarkan bahwa banyak orang Katolik yang
tidak tahu dengan persis bahwa Yesus sungguh-sungguh hadir dalam
Ekaristi:

\begin{itemize}
\item 30\% percaya bahwa mereka sungguh-sungguh dan benar-benar menerima
Tubuh, Darah, Jiwa dan ke-Allahan Yesus Kristus dalam rupa roti dan
anggur.
\item 29\% percaya bahwa mereka menerima roti dan anggur
yang\emph{melambangkan}~Tubuh dan Darah Kristus.
\item 10\% percaya mereka menerima~\emph{roti dan anggur}~di mana di
dalamnya Yesus juga hadir.
\item 24\% percaya mereka menerima Tubuh dan Darah Yesus
karena~\emph{iman mereka sendiri}~mengatakan demikian.
\end{itemize}
Orang yang benar-benar mengerti akan pengajaran Gereja Katolik akan
mengetahui bahwa pilihan yang benar itu hanya pilihan pertama,
sedangkan pilihan yang lain itu keliru. Sayangnya,~\textbf{hanya 30\%
umat Katolik yang mengerti akan kebenaran ini}; sedangkan 70\% yang
lain sepertinya {\textquoteleft}bingung{\textquoteright} atau memegang
kepercayaan gereja lain yang bukan Katolik. Mari kita bertanya pada
diri kita sendiri, termasuk golongan mana kita ini?

\section[Apa yang diajarkan oleh Gereja Katolik tentang
Ekaristi?]{Apa yang diajarkan oleh Gereja Katolik tentang Ekaristi?}
\subsection[1.~Kehadiran Yesus Kristus yang real dan substansial di
dalam Ekaristi]{1.~\textbf{Kehadiran Yesus Kristus yang real dan
substansial di dalam Ekaristi}}
Selama kira-kira 2000 tahun, Gereja Katolik selalu mengajarkan bahwa
Yesus Kristus sungguh hadir, real dan substansial, di dalam Ekaristi,
yaitu Tubuh, Darah, Jiwa dan ke-Allahan-Nya di dalam rupa roti dan
anggur (KGK 1374). Pada saat imam selesai mengucapkan doa konsekrasi --
{\textquotedblleft}Inilah Tubuh-Ku{\textquotedblright} dan
{\textquotedblleft}Inilah darah-Ku{\textquotedblright}, Tuhan secara
ajaib mengubah roti dan anggur menjadi Tubuh dan Darah-Nya. Kejadian
ini disebut sebagai
{\textquotedblleft}\textbf{\emph{transubstansiasi}}{\textquotedblleft},
yang mengakibatkan substansi dari roti dan anggur berubah menjadi Tubuh
dan Darah Kristus (lih. KGK 1376). Jadi yang tinggal hanyalah rupa roti
dan anggur, tetapi substansi roti dan anggur sudah lenyap, digantikan
dengan kehadiran Yesus.

\textbf{Yesus hadir seutuhnya}~di dalam roti itu, bahkan sampai di
partikel yang terkecil dan di dalam setiap tetes anggur. Pemecahan roti
bukan berarti pemecahan Kristus, sebab kehadiran Kristus utuh, tak
berubah dan tak berkurang di dalam setiap partikel. Dengan demikian
kita dapat menerima Kristus di dalam rupa roti saja, atau anggur saja,
atau kedua bersama-sama (lih. KGK 1390). Dalam setiap hal ini, kita
menerima Yesus yang utuh di dalam sakramen.

Karena Yesus sungguh-sungguh hadir di dalam Ekaristi, maka kita memberi
hormat di depan tabernakel, kita berlutut dan menundukkan diri sebagai
tanda penyembahan kepada Tuhan. Itulah sebabnya Gereja memperlakukan
Hosti Kudus dengan hormat, dan melakukan prosesi untuk menghormati
Hosti suci yang disebut Sakramen Maha Kudus, dan mengadakan adorasi di
hadapan-Nya dengan meriah (lih. KGK 1378).

Kehadiran Kristus di dalam Ekaristi bermula pada waktu konsekrasi dan
berlangsung selama rupa roti dan anggur masih ada (KGK 1377), maksudnya
pada saat roti dan anggur itu dicerna di dalam tubuh kita dan sudah
tidak lagi berbentuk roti, maka itu sudah bukan Yesus. Jadi kira-kira
Yesus bertahan dalam diri kita [dalam rupa hosti] selama 15 menit.
Sudah selayaknya kita menggunakan waktu itu untuk berdoa menyembah-Nya,
karena untuk sesaat itu kita sungguh-sungguh menjadi tabernakel Allah
yang hidup!

\textbf{Kristus sendiri yang mengundang kita untuk menyambut Dia dalam
Ekaristi}~(KGK 1384), dan karena itu kita harus mempersiapkan diri
untuk saat yang agung dan kudus ini, dengan melakukan pemeriksaan
batin. Karena Ekaristi itu sungguh-sungguh Allah, maka kita tidak boleh
menyambutNya dalam keadaan berdosa berat. Untuk menyambut-Nya dengan
layak kita harus berada dalam keadaan berdamai dengan Allah. Jika kita
sedang dalam keadaan berdosa berat, kita harus menerima pengampunan
melalui Sakramen Tobat sebelum kita dapat menyambut Komuni Kudus (KGK
1385).

\subsection[2.~Keutamaan Ekaristi disebabkan karena di dalamnya
terkandung Kristus sendiri]{2.~\textbf{Keutamaan Ekaristi disebabkan
karena di dalamnya terkandung Kristus sendiri}}
Ekaristi disebut sebagai sumber dan puncak kehidupan Kristiani (LG 11)
karena di dalamnya terkandung seluruh kekayaan rohani Gereja,
yaitu\textbf{Kristus}~sendiri (KGK 1324). Pada perjamuan terakhir, pada
malam sebelum sengsara-Nya, Kristus menetapkan Ekaristi sebagai tanda
kenangan yang dipercayakan oleh Kristus kepada mempelai-Nya yaitu
Gereja (KGK 1324). Kenangan ini berupa kenangan akan wafat dan
kebangkitan Kristus yang disebut sebagai~\textbf{Misteri Paska}, yang
menjadi puncak kasih Allah yang membawa kita kepada keselamatan (KGK
1067). Keutamaan Misteri Paska dalam rencana Keselamatan Allah
mengakibatkan keutamaan Ekaristi, yang menghadirkan Misteri Paska
tersebut, di dalam kehidupan Gereja (KGK 1085).

Gereja Katolik mengajarkan bahwa kurban salib Kristus terjadi hanya
sekali untuk selama-lamanya (Ibr 9:28). Kristus tidak disalibkan
kembali di dalam setiap Misa Kudus, tetapi kurban yang satu dan sama
itu dihadirkan kembali oleh kuasa Roh Kudus (KGK 1366). Hal itu
dimungkinkan karena Yesus yang mengurbankan Diri adalah Tuhan yang
tidak terbatas oleh waktu dan kematian. Kristus telah mengalahkan maut,
karenanya Misteri Paska-Nya tidak hanya terbenam sebagai masa lampau,
tetapi dapat dihadirkan di masa sekarang (KGK 1085). Karena bagi Tuhan,
segala waktu adalah {\textquoteleft}saat ini{\textquoteright}, sehingga
masa lampau maupun yang akan datang terjadi sebagai
{\textquoteleft}saat ini{\textquoteright}. Dan kejadian Misteri Paska
sebagai {\textquoteleft}saat ini{\textquoteright} itulah yang
dihadirkan kembali di dalam Ekaristi, dengan cara yang berbeda, yaitu
secara sakramental. Dengan demikian, Ekaristi menjadi~\textbf{kenangan
hidup}~akan Misteri Paska dan akan segala karya agung yang telah
dilakukan oleh Tuhan kepada umat-Nya, dan sekaligus~\textbf{harapan
nyata}~untuk Perjamuan surgawi di kehidupan kekal (lih. KGK
1362,1364,1340,1402,1405).

\subsection[3.~Beberapa nama Ekaristi dan
artinya]{3.~\textbf{Beberapa nama Ekaristi dan artinya}}
Ekaristi berasal dari kata
{\textquoteleft}\emph{eucharistein}{\textquoteleft} yang
artinya~\textbf{ucapan terima kasih}kepada Allah (KGK 1328). Ekaristi
adalah kurban pujian dan syukur kepada Allah Bapa, di mana Gereja
menyatakan terima kasihnya kepada Allah Bapa untuk segala kebaikan-Nya
di dalam segala sesuatu: untuk penciptaan, penebusan oleh Kristus, dan
pengudusan. Kurban pujian ini dinaikkan oleh Gereja kepada Bapa melalui
Kristus: oleh Kristus, bersama Dia dan untuk diterima di dalam Dia.
(KGK 1359-1361)

Ekaristi adalah~\textbf{Perjamuan Tuhan}, yang memperingati perjamuan
malam yang diadakan oleh Kristus bersama dengan murid-murid-Nya.
Perjamuan ini juga merupakan antisipasi perjamuan pernikahan Anak Domba
di surga (KGK 1329).

Ekaristi adalah~\textbf{kenangan akan}~\textbf{kesengsaraan dan
kebangkitan Tuhan}(KGK 1330). Ekaristi diadakan untuk memenuhi perintah
Yesus untuk merayakan kenangan akan hidup-Nya, kematian-Nya,
kebangkitan-Nya dan akan pembelaan-Nya bagi kita di depan Allah Bapa
(KGK 1341).

Ekaristi adalah~\textbf{Kurban kudus}, karena ia menghadirkan kurban
tunggal Yesus, dan juga kurban penyerahan diri Gereja yang mengambil
bagian dalam kurban Yesus, Kepalanya (KGK 1330, 1368). Sebagai kenangan
Paska Kristus, Ekaristi menghadirkan dan mempersembahkan secara
sakramental kurban Kristus satu-satunya dalam liturgi Gereja (KGK 1362,
1365). Ekaristi menghadirkan kurban salib dan memberikan buah-buahnya
yaitu pengampunan dosa (KGK 1366).

Ekaristi adalah~\textbf{Komuni kudus}, karena di dalam sakramen ini kita
menerima Kristus sendiri (KGK 1382) dan dengan demikian kita menyatukan
diri dengan Kristus, yang mengundang kita mengambil bagian di dalam
Tubuh dan Darah-Nya, supaya kita membentuk satu Tubuh dengan-Nya (KGK
1331).

Ekaristi dikenal juga dengan~\textbf{Misa kudus}, karena perayaan
misteri keselamatan ini berakhir dengan pengutusan umat beriman
(\emph{missio}) supaya mereka melaksanakan kehendak Allah dalam
kehidupan sehari-hari.

\subsection[4.~Buah{}-buah Ekaristi/ Komuni
kudus]{4.~\textbf{Buah-buah Ekaristi/ Komuni kudus}}
\begin{itemize}
\item Komuni~\textbf{memperdalam persatuan}~kita dengan Yesus, hal ini
berdasarkan atas perkataan Yesus, {\textquotedblleft}Barangsiapa makan
daging-Ku dan minum Darah-Ku, ia tinggal dalam Aku dan Aku di dalam
Dia{\textquotedblright} (KGK 1391).
\item Komuni~\textbf{memisahkan kita dari dosa}, karena dengan
mempersatukan kita dengan Kristus kita sekaligus dibersihkan dari dosa
yang telah kita lakukan dan melindungi kita dari dosa-dosa yang baru
(KGK 1393).
\item Ekaristi~\textbf{membangun Gereja di dalam kesatuan}. Oleh
Ekaristi Kristus mempersatukan kita dengan semua umat beriman menjadi
satu Tubuh, yaitu Gereja. Ekaristi memperkuat kesatuan dengan Gereja
yang telah dimulai pada saat pembaptisan (KGK 1396). Kesatuan dengan
Gereja ini mencakup Gereja yang masih berziarah di dunia, Gereja yang
sudah jaya di Surga, dan Gereja yang masih dimurnikan di dalam Api
Penyucia (lih. KGK 954)
\item Ekaristi~\textbf{mewajibkan kita terhadap kaum miskin}, sebab
dengan bersatu dengan Kristus dalam Ekaristi, kita juga mengakui
Kristus yang hadir di dalam orang-orang termiskin yang juga menjadi
saudara-saudara-Nya (KGK 1397), yang di dalam Dia, menjadi
saudara-saudara kita juga.
\item Ekaristi~\textbf{mendorong kita ke}~\textbf{persatuan umat
beriman}, sebab Ekaristi, menurut perkataan Santo Agustinus adalah
{\textquoteleft}sakramen kasih sayang, tanda kesatuan dan ikatan
cinta,{\textquoteright} (KGK 1398) yang seharusnya secara penuh dialami
bersama oleh semua orang yang beriman di dalam Kristus.
\end{itemize}
\section[Dasar pengajaran tentang Ekaristi dari Alkitab]{Dasar
pengajaran tentang Ekaristi dari Alkitab}
\subsubsection[1.~Perjanjian Lama:]{1.~\textbf{Perjanjian Lama}:}
\begin{itemize}
\item \textbf{Imam Agung}~\textbf{Melkisedek}~mempersembahkan roti dan
anggur (Kej 14:18) yang menggambarkan Perjamuan Yesus pada Perjamuan
Terakhir. Yesus sendiri dikatakan sebagai Imam Besar menurut peraturan
Melkisedek (Ibr 6:20).
\item \textbf{Kurban anak domba Paska}~yang menyelamatkan umat Israel
merupakan kurban yang dimakan sebagai makanan untuk menguatkan mereka
menempuh perjalanan ke Tanah Terjanji (Kej 12:1-20). Hal ini
menggambarkan Ekaristi yang merupakan kurban Anak Domba Allah, yaitu
Yesus, yang dimakan sebagai makanan untuk menjadi bekal perjalanan kita
ke Tanah Terjanji, yaitu surga.
\item \textbf{Roti Manna~}yang menjadi simbol Ekaristi pada Perjanjian
Lama. Yesus sendiri mengatakan bahwa Ia adalah Roti manna yang turun
dari surga (lih. Yoh 6:32-51). Seperti halnya bahwa manna menguatkan
bangsa Israel sepanjang perjalanan di gurun dan berhenti dicurahkan
setelah mereka sampai di Tanah Terjanji; Ekaristi juga diberikan untuk
menguatkan kita di perjalanan hidup di dunia, dan berhenti setelah kita
sampai di surga.
\item Pada\textbf{~}\textbf{Tabut Perjanjian Lama~}menggambarkan
tabernakel pada gereja Katolik di manapun, yang merujuk pada
Ekaristi.~\textbf{Dua loh batu}~(Kel 25:16) menggambarkan sabda
kehidupan yang terkandung dalam Ekaristi.~\textbf{Manna}~(Kel 16:34)
menggambarkan Ekaristi sebagai roti hidup yang turun dari surga (Yoh
6:51).~\textbf{Tongkat Harun}~(Bil 17: 5) yang menandai imamatnya,
menggambarkan peran Imamat kudus dalam Kristus, yaitu tubuhNya. Seperti
tongkat Harun yang bertunas, tubuh Yesus yang ditembus oleh tombak
mengeluarkan air dan darah yang melambangkan sakramen Pembaptisan dan
Ekaristi.[2]
\end{itemize}
\subsection[2.~Perjanjian Baru:]{2.~\textbf{Perjanjian Baru}:}
Yesus sungguh-sungguh hadir di dalam Ekaristi, seperti dinyatakan:

\begin{itemize}
\item Pada~\textbf{Perjamuan Terakhir}~Yesus memerintahkan
murid-murid-Nya untuk mengenangkan Dia dengan merayakan perjamuan
tersebut. Yesus berkata, {\textquotedblleft}Inilah Tubuh-Ku{\dots}
(bukan ini~\emph{melambangkan}~Tubuh-Ku){\dots} (lih Mat 26:26-28; Mrk
14:22-24; Luk 22:15-20).{\textquotedblright}
\item Yesus mengatakan sendiri bahwa Ia adalah
{\textquotedblleft}\textbf{Roti hidup}~yang turun dari surga. Jikalau
seorang~\textbf{makan}~dari roti ini, dia akan hidup selama-lamanya;
dan roti yang Ku-berikan itu ialah daging-Ku yang Kuberikan untuk hidup
dunia (Yoh 6:35, 51).
\item Pengajaran ini diberikan setelah Yesus mengadakan mukjizat
pergandaan roti, yaitu mukjizat yang ditulis di dalam ke-empat Injil
(Mat 14:13-21; Mrk 6:32-44; Luk 9:10-17; Yoh 6:1-15). Lima roti yang
sama yang dibagikan oleh para rasul dapat memberi makan 5000 orang,
dengan sisa 12 keranjang. Ini menggambarkan~\textbf{Yesus yang satu dan
sama}~hadir dalam Ekaristi, dapat dibagikan kepada semua orang, tanpa
Dia sendiri menjadi terbagi-bagi atau berkurang/ hilang.
\item Yesus berkata bahwa~\textbf{Ia lebih tinggi nilainya dari pada
manna}~yang diberikan kepada orang Israel di gurun. Padahal mukjizat
manna adalah suatu mukjizat yang besar, setiap harinya berjuta orang
Israel menerima 1 omer (1.1 liter) roti manna per orang, sehingga tiap
harinya ada beberapa ratus ton roti manna tercurah dari langit, selama
40 tahun.[3] Yesus mengatakan bahwa mukjizat-Nya lebih hebat daripada
mukjizat manna ini, sehingga kita dapat menyimpulkan bahwa di dalam
Ekaristi, roti dapat sungguh-sungguh diubah Yesus menjadi diri-Nya
sendiri, seperti yang dikatakan-Nya.
\item Orang-orang yang mendengarkan
pengajaran~\textbf{{\textquoteleft}}\textbf{Roti
Hidup{\textquoteright}}~ini memahami bahwa Yesus mengajarkan sesuatu
yang~\textbf{literal}~(tidak figuratif/ simbolis), sehingga mereka
meninggalkan Yesus sambil berkata, {\textquotedblleft}Bagaimana Ia ini
dapat memberikan daging-Nya untuk dimakan{\textquotedblright} (Yoh
6:52)
\item Yesus menggunakan gaya bahasa yang kuat untuk menjelaskan arti
literal pengajaran ini dengan~\textbf{mengulangi}~pengajaran ini
sampai~\textbf{6 kali di dalam 6 ayat}~(ay. 53-58),{\dots} jikalau kamu
tidak makan daging Anak Manusia dan minum darah-Nya, kamu tidak
mempunyai hidup di dalam dirimu (Yoh 6:53); Sebab daging-Ku adalah
benar-benar makanan dan darah-Ku adalah benar-benar minuman (Yoh 6:55).
Ini adalah gaya bahasa yang bukan kiasan/ simbolis!
\item Banyak murid tidak dapat menerima pengajaran ini, dan meninggalkan
Yesus (ay.66), tetapi~\textbf{Yesus tidak menarik kembali
pengajaran-Nya tentang diri-Nya sebagai {\textquotedblleft}Roti
Hidup{\textquotedblright}}. Dia tidak mengatakan bahwa Dia hanya
berkata secara figuratif/simbolis. Pada beberapa kesempatan, jika Ia
berbicara secara figuratif, Yesus menerangkan kembali maksud
perkataan-Nya pada para murid-Nya yang mengartikannya secara literal.
(Contohnya pada Yoh 4:31-34, Yesus menjelaskan bahwa
{\textquoteleft}makanan-Nya yang tidak mereka kenal{\textquoteright}
adalah melakukan kehendak Bapa yang mengutus-Nya. Atau pada Mat
16:5-12; tentang ragi orang-orang Farisi dan Saduki, maksudnya adalah
bukan ragi secara literal, tetapi pengajaran mereka)[4]
\item Setelah banyak yang meninggalkan Dia karena pengajaran ini, Yesus
bahkan bertanya kepada ke dua-belas rasulNya,
{\textquotedblleft}\textbf{Apakah kamu tidak mau pergi
juga}?{\textquotedblright}(Yoh 6:67). Namun Petrus menjawab,
{\textquotedblleft}Tuhan kepada siapakah kami akan pergi? Perkataan-Mu
adalah perkataan hidup yang kekal (Yoh 6:69). Pertanyaan yang sama
ditujukan pada kita, apakah kita mau percaya akan pengajaran ini
seperti Petrus, ataukah kita seperti murid-murid lain yang meninggalkan
Dia?
\item Rasul Paulus mengingatkan jemaat agar tidak menerima Ekaristi
secara tidak layak, supaya tidak berdosa terhadap Tubuh dan Darah Tuhan
(1 Kor 11:27). Rasul Paulus juga menambahkan, jika seseorang makan dan
minum tanpa mengakui Tubuh Tuhan, ia mendatangkan hukuman atas dirinya
sendiri (1 Kor 11:28-29). Pengajaran ini tidak masuk di akal, jika
kehadiran Yesus dalam Ekaristi hanya simbolis belaka. Kesimpulannya,
St. Paulus jelas mengajarkan bahwa Yesus sungguh-sungguh hadir di dalam
Ekaristi.
\end{itemize}
\subsection[3. Bukti dari para Bapa Gereja di abad awal]{3. Bukti
dari para Bapa Gereja di abad awal}
Tulisan para Bapa Gereja di abad awal merupakan bukti yang sangat
penting tentang {\textquoteleft}keaslian{\textquoteright} pengajaran
tentang Ekaristi. Para Bapa Gereja merupakan saksi yang menjamin
keaslian pengajaran Alkitab, karena mereka sungguh-sungguh menyaksikan
para rasul mengajar dan menuliskan Injil, seperti Rasul Matius, Yohanes
dan St. Paulus menuliskan surat-suratnya. Melalui tulisan-tulisan
mereka, kita mengetahui Tradisi Suci para Rasul, seperti Kehadiran
Yesus dalam Ekaristi, Misa Kudus, kepemimpinan Rasul Petrus, devosi
kepada Maria, Api penyucian, dll. Semua pengajaran ini adalah
pengajaran yang diteruskan oleh Gereja Katolik. Berikut ini adalah para
Bapa Gereja yang mengajarkan tentang kehadiran Yesus di dalam Ekaristi:

\begin{enumerate}
\item \textbf{Ignatius dari Antiokhia}, murid dan pembantu Rasul
Yohanes, uskup ke-3 di Antiokhia. Tahun 110 ia menulis 7 surat kepada
gereja-gereja sebelum kematiannya sebagai martir di Roma. Pada suratnya
ke gereja di Smyrna, St. Ignatius menyebutkan bahwa mereka yang tidak
percaya kepada {\textquoteleft}Kehadiran Yesus di dalam
Ekaristi{\textquoteright} adalah sesat
({\textquoteleft}\emph{heretics}{\textquoteleft}).[5] Kepada gereja di
Roma, St. Ignatius menuliskan imannya tentang Ekaristi yang
sungguh-sungguh adalah Tubuh dan Darah Yesus.[6]
\item \textbf{St. Yustinus Martir,~}pengikut Kristus pada tahun 130,
yang mendapat pengajaran dari Rasul Yohanes,
seorang~\emph{Apologist}~yang terkenal di abad ke-2. Pada tulisannya
kepada Emperor di Roma, yaitu
{\textquotedblleft}\emph{Apology}{\textquotedblright} pada tahun 150,
St. Yustinus juga menjelaskan kebenaran pengajaran tentang kehadiran
Yesus di dalam Ekaristi.[7]
\item \textbf{St. Irenaeus,~}uskup Lyons, hidup tahun 140-202. Ia murid
St. Polycarpus yang adalah murid Rasul Yohanes. Dengan menuliskan
bukunya yang terkenal, {\textquotedblleft}\emph{Against
Heresies}{\textquotedblright} (195), ia menghancurkan pandangan sesat
yang bertentangan dengan kepercayaan Gereja yang dipegang oleh para
rasul.[8]
\item \textbf{St. Cyril dari Yerusalem,~}pada tahun 350 mengajarkan agar
kita sebagai pengikut Kristus percaya sepenuhnya akan kehadiran Yesus
di dalam Ekaristi, sebab Yesus sendiri yang mengatakannya[9]
\item \textbf{St. Hilary,~}uskup Poitiers, Perancis, tahun 315-367.
Dengan karyanya, {\textquotedblleft}\emph{On the
Trinity}{\textquotedblright} (356), St. Hilary mengajarkan kehadiran
Kristus dalam Ekaristi yang kita terima menjadikan kita tinggal di
dalam Kristus dan Kristus di dalam kita.[10]
\end{enumerate}
Para Bapa Gereja ini membuktikan bahwa jemaat Kristen awal percaya akan
Kehadiran Yesus di dalam Ekaristi. Perhatikanlah bahwa St. Ignatius
adalah murid Rasul Yohanes, sedangkan St. Yustinus Martir dan St.
Irenaeus belajar langsung dari murid-murid Rasul Yohanes. Mereka semua
mendapat pengajaran dari Rasul Yohanes yang menulis tentang Yesus
sebagai {\textquotedblleft}Roti Hidup{\textquotedblright} (Yoh 6).
Siapa yang dapat mengatakan bahwa ia lebih memahami pengajaran Yesus
tentang {\textquoteleft}Roti Hidup{\textquoteright} ini dari pada
mereka yang mendengar langsung/ murid dari Rasul Yohanes?

\section[Kesimpulan]{Kesimpulan}
Jika kita dengan hati terbuka mempelajari Alkitab, dan tulisan para Bapa
Gereja, kita akan melihat bahwa kenyataan menunjukkan bukti yang kuat
yang mendasari pengajaran Gereja Katolik tentang~\textbf{Kehadiran
Yesus secara real dan substansial di dalam Ekaristi}. Yesus sendiri
hadir di dalam Ekaristi, di dalam rupa roti dan anggur, dan sudah
menjadi kehendak-Nya agar kita mengenangkan Dia melalui perjamuan ini,
agar kita dapat mengambil bagian di dalam Misteri Paska-Nya yang
mendatangkan keselamatan bagi dunia. Ekaristi adalah cara yang dipilih
Yesus agar kita dapat tinggal di dalam Dia dan Dia di dalam kita.
Percaya penuh akan kehadiran-Nya di dalam Ekaristi dan menerima
Ekaristi dengan sikap yang benar merupakan bentuk perwujudan iman dan
kasih kita kepada Tuhan yang terlebih dahulu mengasihi kita sampai
wafat di salib. Mari kita menerima dengan hati terbuka, cara Yesus
mengasihi kita di dalam Ekaristi. Mari kita berdoa, agar makin hari
kita makin dapat menghayati kasih-Nya yang tak terbatas, yang tercurah
pada kita melalui Sakramen yang Maha Kudus ini{\dots}

