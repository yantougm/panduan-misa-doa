\documentclass[12pt,a4paper, twocolumn]{article}
\usepackage[latin1]{inputenc}
\usepackage{amsmath}
\usepackage{amsfonts}
\usepackage{amssymb}
\usepackage[width=142.50mm, height=210.00mm, left=10.00mm, right=10.00mm, top=10.00mm, bottom=10.00mm]{geometry}
\author{Yohanes Suyanto}
\setlength{\columnsep}{0.9cm}
\begin{document}
\pagestyle{empty}
\newcommand{\doa}{ %
\begin{center}
\textbf{\LARGE Doa kepada \\Santa Theresia \\ Kanak-kanak Yesus}
\end{center}

\scriptsize
\begin{quote}
\emph{Santo Theresia meninggal dunia pada tanggal 30 September 1897  ketika berusia 24 tahun. Sebelum meninggal Santo Theresia mengatakan, "Dari surga aku akan berbuat kebaikan bagi dunia" dan ia menepati janjinya! Semua orang dari seluruh dunia yang memohon bantuan Santo Theresia untuk mendoakan mereka kepada Tuhan telah memperoleh jawaban atas
doa-doa mereka.}
\end{quote}

\small
\begin{center}
\textbf{O Santa Theresia dari Kanak-Kanak Yesus}

\textbf{tolong petikkan bagiku sekuntum mawar}

\textbf{dari taman surgawi dan kirimkan padaku} 

\textbf{dengan suatu amanat cinta.}

\textbf{O Bunga Kecil dari Yesus}

\textbf{mintalah kepada Allah hari ini}

\textbf{untuk menganugerahkan rahmat yang sangat kubutuhkan \ldots \ldots 
(\textit{katakan permohonanmu})}

\textbf{Santa Theresia, bantulah aku untuk senantiasa percaya
kepada belaskasih Allah} 

\textbf{yang sedemikian besar,}

\textbf{sebagaimana telah engkau wujudkan} 

\textbf{di dalam hidupmu,}

\textbf{sehingga aku boleh mengikuti} 

\textbf{'Jalan Kecil'mu setiap hari.}

\textbf{Amin.}
\end{center}
\normalsize
}
\doa
\vspace{1cm}
\doa

\doa
\vspace{1.3cm}
\doa
\end{document}