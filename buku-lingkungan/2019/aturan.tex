\section{Informasi Paroki}
\begin{itemize}
	\item Misa Hari Minggu: dilayani romo paroki Marganingsih Kalasan, gereja stasi 3 kali, gereja wilayah lama (Prambanan, Temanggal, Payak, Manisrenggo) 2 kali, gereja wilayah baru (Karanglo, Jragung) 1 kali.

	\item Misa lingkungan dilayani mulai bulan Mei 2017 oleh Romo pendamping rayon, maksimal 2 kali setahun. Dalam misa lingkungan, ketua lingkungan yang ketempatan bisa mengundang ketua lingkungan sekitar sehingga juga sekaligus berfungsi koordinasi apabila ada informasi yang bersifat Parokial sekaligus tersampaikan.

	Jumawil dirasa belum bisa menghadirkan umat secara signifikan maka kemudian diganti menjad misa lingkungan.

	\item Baptis bayi sangat baik dilakukan dalam misa di gereja paroki/stasi/wilayah, dilayani oleh Romo Paroki.

	\item Baptis dewasa dilayani setahun sekali dalam perayaan Paskah, disiapkan satu tahun.
	
	\item Perkawinan dilayani dengan misa. Jika dilaksanakan pada hari Minggu/hari raya bacaan Injil harus menggunakan bacaan hari yang bersangkutan. 

	\item Minyak suci dilayani kapan saja.
	
	\item Pemberkatan Jenasah dilayani dengan misa dan diharapkan dilaksanakan 1,5 jam sebelum pemakaman. Dalam upacara melepas jenazah banyak terjadi ketika sambutan yang bertele-tele, akan lebih baik jika sambutan singkat dan padat saja.

	\item Misa ujud di rumah keluarga pada hari Sabtu malam Minggu dan hari Minggu tetap dilayani walaupun tidak dianjurkan.
	
	\item Misa ujud di rumah keluarga pada hari yang bertepatan dengan misa Jumat I kalau mungkin dihindari. Jika terpaksa harus hari itu, maka dilaksanakan setelah misa Jumat I. Khusus untuk stasi Macanan karena berlindung pada ‘Tyas Dalem’ maka misa ujud di rumah keluarga pada Jumat I ditiadakan.
\end{itemize}

\section{Aturan dan Kebijakan Lingkungan St. Theresia \tahun} 

\begin{itemize}
\item Iuran lingkungan tetap Rp.8.000,- /KK/bulan yang
terdiri dari iuran wajib Rp.5.000,- dan iuran sosial Rp.3.000,-.
\item Bagi warga St. Theresia yang opname di rumah sakit mendapat sumbangan sebesar
Rp. 200.000,-/org/tahun. Bila umat berkenan menambah sumbangan
lingkungan tersebut dengan melakukan sumbangan serkiler, maka
dipersilahkan untuk mengumpulkan dana pribadi tanpa adanya suatu
pemaksaan. 
\item Jika ada warga dari lingkungan lain yang sakit (opname), kita
bersepakat untuk membezuk tanpa ada dana tunjangan dari Lingkungan St.
Theresia, tetapi tali kasih yang diberikan melalui dana serkiler
(pribadi).
\item Permintaan (ujud) misa dari umat Lingk. St. Theresia. Bila
menghendaki diiringi koor dari Lingkungan, maka biaya konsumsi selama
latihan koor menjadi tanggungan dari umat yg meminta ujud.
\item Berhubung sudah beberapa lama informasi lingkungan via jalur WhatsApp maka informasi via sms, akan dilakukan hanya kepada keluarga yang tidak memasang aplikasi WhatsApp.
\end{itemize}

\section{Tata cara persiapan dan pelaksanaan ujud  misa/ibadat pribadi}
\label{sec:tataCara}
\begin{enumerate}
\item Persiapan (oleh umat bersangkutan dan pengurus lingkungan):

\begin{enumerate}
\item penentuan waktu oleh umat
\item menghubungi Romo Paroki/petugas
\item persiapan koor (bila ada)
\item persiapan peralatan misa (bila ada misa)
\item pembuatan dan pengedaran undangan
\end{enumerate}
\item Pelaksanaan (oleh umat bersangkutan bersama dengan pengurus
lingkungan):

\begin{enumerate}
\item pengaturan tempat
\item pengaturan Altar
\item penjemputan Romo/petugas (bila perlu)
\item pelaksanaan misa/ibadat
\item penyerahan stipendium atau iura stolae untuk Romo
\item penggantian biaya hosti dan anggur
\end{enumerate}
\end{enumerate}
Catatan:~Segala kegiatan doa/misa pribadi yang dipersiapkan dan
dilaksanakan sendiri (tanpa melibatkan Lingkungan) dengan melibatkan
banyak umat, keluarga bersangkutan wajib memberikan laporan kepada
Ketua Lingkungan untuk diteruskan ke Paroki.
