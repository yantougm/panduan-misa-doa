\documentclass[a5paper,titlepage,12pt]{scrbook}
\usepackage[a5paper,backref]{hyperref}
\usepackage[papersize={148.5mm,215mm},twoside,bindingoffset=0.5cm,hmargin={2cm,2cm},
				vmargin={2cm,2cm},footskip=1.1cm,driver=dvipdfm]{geometry}
\usepackage{palatino}
\usepackage[ascii]{inputenc}
\usepackage[T1]{fontenc}
\usepackage[bahasa]{babel}
\usepackage{amsmath}
\usepackage{amssymb,amsfonts,textcomp}
\usepackage{array}
\usepackage{supertabular}
\usepackage{hhline}
\usepackage[pdftex]{graphicx}
\usepackage{enumerate}
\makeatletter
\newcommand\arraybslash{\let\\\@arraycr}
\makeatother
\setlength\tabcolsep{1mm}
\renewcommand\arraystretch{1.3}
\title{}

\makeatletter
\renewcommand{\@makeschapterhead}[1]{%
  {\parindent \z@ \centering \normalfont
    \interlinepenalty\@M \Large \bfseries #1\par\nobreak \vskip 20\p@ }}
\renewcommand{\section}{\@startsection {section}{1}{\z@}%
                                   {-3.5ex \@plus -1ex \@minus -.2ex}%
                                   {2.3ex \@plus.2ex}%
%                                   {\normalfont\normalsize\bfseries\centering}}
                                   {\normalfont\normalsize\bfseries}}
\renewcommand\subsection{\@startsection{subsection}{2}{\z@}%
                                     {-3.25ex\@plus -1ex \@minus -.2ex}%
                                     {1.5ex \@plus .2ex}%
                                     {\normalfont\normalsize\bfseries}}
\renewcommand\subsubsection{\@startsection{subsubsection}{3}{\parindent}%
                                    {3.25ex \@plus1ex \@minus.2ex}%
                                    {-1em}%
                                    {\normalfont\normalsize\bfseries}}
\makeatother

\def\thesection{\arabic{section}.}
\setlength{\parindent}{0pt}

\hyphenation{Ke-ra-ja-an-Nya}
\hyphenation{me-re-ka}
\hyphenation{a-da-lah}
\hyphenation{me-ngan-dung}
\hyphenation{ke-su-kar-an}
\hyphenation{di-lin-dung-i}

\newcommand{\terbaptis}{satu dua tiga }
\begin{document}


\chapter*{DOA ROSARIO HARI KE-21, \\SYUKUR ATAS PEMBAPTISAN, dan DOA PERSIAPAN OPERASI}
\begin{center}
\textit{TEMA UTAMA\\
Merenungkan serta memuji {\textquoteleft}peran{\textquoteright} Bunda Maria dalam Karya Keselamatan PuteraNya, Yesus Kristus, bagi kita.}
\end{center}

\section{Pembuka}
\subsection*{Lagu Pembuka}
\subsection*{Tanda Salib}
Dalam nama Bapa dan Putra dan Roh Kudus Amin

Rahmat Tuhan kita Yesus Kristus, cinta kasih Allah dan persekutuan Roh Kudus beserta kita

Sekarang dan selama-lamanya.

\subsection*{Pengantar dan Tobat} 
Ibu, Bapak, dan Saudara sekalian hari ini kita patur bersyukur kepada Tuhan karena 3 putri \terbaptis Bapak dan Ibu Mateus Daryanto beberapa hari yang lalu telah dibaptis dan resmi menjadi katolik. Dengan demikian putri-putri tersebut telah dipersatukan dalam persekutuan umat Allah. Kita berdoa semoga karena sakramen baptis yang diterimanya,  mereka mampu bertumbuh dakam iman, harapan, dan kasih akan Allah Tritunggal.

Selain itu, karena beberapa hari lagi Bapak Mateus Daryanto akan menjalani operasi mata, kita juga mohon berkat Tuhan agar operasi terlaksana dengan baik dan lancar, sehingga saudara kita akan cepat sembuh dari sakitnya.

Ibu, Bapak, dan Saudara sekalian, sadar bahwa kita sering melanggar janji-janji baptis kita, marilah di awal ibadat syukur ini, kita menyesali dosa-dosa kita dan memohon pengampunan dari Tuhan agar kita layak untuk melaksanakan ibadat syukur ini.
 
Saya mengaku \dots


\subsection*{Doa Pembuka}
Marilah berdoa,

Allah Bapa yang mahakuasa dan kekal pada persitiwa pembaptisan di Sungai Yordan, PuteraMu telah Kauurapi dengan Roh Kudus dan kepadaNya Kaukenakan kekuatan Ilahi. Kami bersyukur atas rahmat pembaptisan bagi \terbaptis. Tolonglah mereka mengalahkan kejahatan dengan kebaikan, ajarilah mereka mengasihi Allah dan sesama. Kami mohon pula berilah Bapak Mateus Daryanto keberanian untuk menghadapi operasi tanpa cemas, takut, dan ragu-ragu.
Demi Yesus Kristus, Tuhan dan Penyelamat kami, yang hidup dan berkuasa bersama Dikau dalam persekutuan Roh Kudus, kini dan sepanjang masa.

Amin. 

\subsection*{Lagu Pengantar Rosario}

\section{Doa Rosario dan Bacaan}
\begin{itemize}
	\item Aku Percaya{\dots}.
	\item Kemuliaan{\dots}.
	\item Bapa Kami{\dots}..
	\item 3 Salam (Puteri Allah Bapa, Bunda Allah Putera, Mempelai Allah Roh Kudus), Salam Maria{\dots}{\dots}
	\item Kemuliaan{\dots}..
	\item Terpujilah{\dots}{\dots}
	\item Peristiwa Rosario, peristiwa gembira 
	
	\begin{enumerate}[a.]
		\item Tiga Misteri Kudus (3x persepuluhan pertama)		
			\begin{itemize}
				\item 
				 	\begin{enumerate}[\bfseries{Peristiwa }I.]
						\item \textbf{Yesus di baptis di sungai Yordan}
Bagi saudara kami \terbaptis yang beberapa hari yang lalu telah dibaptis;

Ya Bapa, lewat sakramen pembaptisan saudara2 kami telah dipersatukan dalam persekutuan umat Allah. Semoga kelak mereka juga memperoleh warisan surgawi bersama para kudus di Surga 

Kami mohon:
						
						\item \textbf{Yesus menyatakan diri-Nya dalam pesta pernikahan di Kana}
						  
Bagi umat lingkungan kami;

Ya Bapa, semoga sebagaimana Yohanes Pembaotis menanti-nantikan penyelamat, selalu melaksanakan kelembutan dan kerendahan hatiNya dalam hidup kami sehari-hari
						
						Kami mohon:

						\item \textbf{Yesus memberitakan Kerajaan Allah dan menyerukan pertobatan}

Ya Bapa, berilah hiburan dan harapan baru bagi mereka yang sedang menderita dengan warta gembira Kristus

Kami mohon:

					\end{enumerate}
				\item Bapa kami{\dots}{\dots}
				\item Salam Maria{\dots}.(10x)
				\item Kemuliaan{\dots}{\dots}
				\item Terpujilah{\dots}{\dots}
				\item Ya Yesus yang baik{\dots}{\dots}.
			\end{itemize}

		\item Bacaan
			\begin{description}
				\item [Bacaan dari Surat 1 Yoh 5:1-9] 
					{~}

Setiap orang yang percaya, bahwa Yesus adalah Kristus, lahir dari Allah; dan setiap orang yang mengasihi Dia yang melahirkan, mengasihi juga Dia yang lahir dari pada-Nya.

Inilah tandanya, bahwa kita mengasihi anak-anak Allah, yaitu apabila kita mengasihi Allah serta melakukan perintah-perintah-Nya.
Sebab inilah kasih kepada Allah, yaitu, bahwa kita menuruti perintah-perintah-Nya. Perintah-perintah-Nya itu tidak berat,
sebab semua yang lahir dari Allah, mengalahkan dunia. Dan inilah kemenangan yang mengalahkan dunia: iman kita.

Siapakah yang mengalahkan dunia, selain dari pada dia yang percaya, bahwa Yesus adalah Anak Allah?

Inilah Dia yang telah datang dengan air dan darah, yaitu Yesus Kristus, bukan saja dengan air, tetapi dengan air dan dengan darah. Dan Rohlah yang memberi kesaksian, karena Roh adalah kebenaran.

Sebab ada tiga yang memberi kesaksian (di dalam sorga: Bapa, Firman dan Roh Kudus; dan ketiganya adalah satu.
Dan ada tiga yang memberi kesaksian di bumi): Roh dan air dan darah dan ketiganya adalah satu.

Kita menerima kesaksian manusia, tetapi kesaksian Allah lebih kuat. Sebab demikianlah kesaksian yang diberikan Allah tentang Anak-Nya.

				\item [Lagu Selingan]

				\item [Bacaan Injil menurut Yohanes (1:19-28)]
				{~}
				
Dan inilah kesaksian Yohanes ketika orang Yahudi dari Yerusalem mengutus beberapa imam dan orang-orang Lewi kepadanya untuk menanyakan dia: "Siapakah engkau?"

Ia mengaku dan tidak berdusta, katanya: "Aku bukan Mesias."

Lalu mereka bertanya kepadanya: "Kalau begitu, siapakah engkau? Elia?" Dan ia menjawab: "Bukan!" "Engkaukah nabi yang akan datang?" Dan ia menjawab: "Bukan!"

Maka kata mereka kepadanya: "Siapakah engkau? Sebab kami harus memberi jawab kepada mereka yang mengutus kami. Apakah katamu tentang dirimu sendiri?"

Jawabnya: "Akulah suara orang yang berseru-seru di padang gurun: Luruskanlah jalan Tuhan! seperti yang telah dikatakan nabi Yesaya."

Dan di antara orang-orang yang diutus itu ada beberapa orang Farisi.
Mereka bertanya kepadanya, katanya: "Mengapakah engkau membaptis, jikalau engkau bukan Mesias, bukan Elia, dan bukan nabi yang akan datang?"

Yohanes menjawab mereka, katanya: "Aku membaptis dengan air; tetapi di tengah-tengah kamu berdiri Dia yang tidak kamu kenal,
yaitu Dia, yang datang kemudian dari padaku. Membuka tali kasut-Nyapun aku tidak layak."

Hal itu terjadi di Betania yang di seberang sungai Yordan, di mana Yohanes membaptis.

 
			\end{description}

		\item Renungan

Poin-poin renungan:
\begin{itemize}
\item Materi sakramen pembaptisan: garam dan air.
\item Simbol: lilin, minyak, dan kain putih
\item ``Kamu adalah garam dan terang dunia''
\item Baptisan Yohanes, bukan soal upacara, alihkan ke Yesus
\item Yesus adalah mata air abadi yang tidak pernah kering
\item Yohanes memugar prinsip-prinsip: kebenaran, nilai, moral, dan hukum.
\end{itemize}

Ibu, Bapak, dan saudara sekalian kita sudah dibaptis dan menjadi orang katolik. Namun kita tidak imun terhadap dosa. Kita sadar bahwa di dunia ini masih ada tempat bagi segala bentuk kelemahan. Kita patut berdoa semoga Tuhan tetap menyertai kita dan dengan bantuan Roh Kudus kita dimampukan untuk hidup sesuai dengan janji-janji baptis dan kredo yang kita ucapkan.

Kemuliaan kepada Allah Bapa, Putra, dan Roh Kudus, Amin.

	\item Lanjutan Peristiwa Rosario, peristiwa sedih 
		\begin{enumerate}[{Peristiwa }I.]
		    \setcounter{enumii}{3}
			\item \textbf{Yesus menampakan kemuliaan-Nya}

Bagi Bapak Mateus Daryanto;

Jagailah dia ya Bapa, selama proses operasi nanti berlangsung. Semoga Dikau sendiri hadir menguatkan hatinya dan meringankan deritanya terutama saat sakit atau nyeri sedang dirasakannya			

Kami mohon:
			
			\item \textbf{Yesus menetapkan Ekaristi}

Kami serahkan Bapak Mateus Daryanto dalam naunganMu. PadaMu kami percaya. Sudilah kiranya tinggal dalam hati saudara kami ini untuk menjadi daya hidup dan penyembuh sejati.
			
Kami mohon:

		\end{enumerate}
	\end{enumerate}
	\item Lagu
\end{itemize}

\section{Ibadat Penutup}
\begin{itemize}
\item Doa Penutup dan mohon berkat Tuhan

	Marilah berdoa,

Allah Bapa yang mahabaik, dalam ibadat syukur ini kami mohon baptislah kami dengan RohMu, agar kami dengan tulus ikhlas bersikap baik terhadap sesama, penuh perhatian terhadap yang lemah, menghibur dan menguatkan mereka. Semoga dengan demikian kerajaanMu benar-benar datang di tengah-tengah kami.

Demi Kristus Tuhan dan pengantara kami. 
Amin.

Saudara-saudari yang terkasih, marilah kita mengakhiri ibadat syukur dan doa rosario kita, dengan mohon berkat Tuhan.

Tuhan beserta kita

sekarang dan selama-lamanya

Semoga kita sekalian yang hadir di sini, dilindungi dan dibimbing oleh berkat Allah yang Mahakuasa dalam nama Bapa, Putera, dan Roh Kudus

Amin.

\item Kolekte dan Lagu Penutup. 
\end{itemize}
\end{document}

Marilah berdoa,

Allah Yang maha kasih
Kami bersyukur telah Kau percaya
untuk mengemban tugas sebagai wali baptis bagi
…..……… NN ……………


Bantulah kami agar dalam perkataan dan perbuatan
senantiasa selaras dengan ajaran PuteraMu
sehingga kami pun bisa menjadi contoh
bagi anak-anak baptis kami.
Berkatilah ………NN………
Agar senantiasa mendapatkan
Suasana yang mendukung
bagi perkembangan imannya.
Semoga iman yang diterimanya dalam pembaptisan
semakin tumbuh subur dan menghasilkan buah.
Anugerahkanlah kepadanya
apa saja yang perlu
bagi keselamatan jiwa raganya
sesuai dengan kehendakMu.
Demi Kristus Tuhan dan Pengantara kami
kini dan sepanjang masa. Amin

Bapa Kami …… Salam Maria …… Kemuliaan ……

Santo/a ……… (Pelindung NN)
Doakanlah …………NN ……..
Amin.
						
