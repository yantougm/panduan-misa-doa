% This file was converted to LaTeX by Writer2LaTeX ver. 1.0.2
% see http://writer2latex.sourceforge.net for more info
\documentclass{scrartcl}
\usepackage[a5paper,vmargin={2cm,2cm},hmargin={2cm,2cm}]{geometry}
\usepackage[ascii]{inputenc}
\usepackage[T1]{fontenc}
\usepackage[bahasa]{babel}
\usepackage{amsmath}
\usepackage{amssymb,amsfonts,textcomp}
\usepackage{microtype}
\usepackage{palatino}
\usepackage{graphicx}
\usepackage{marvosym}
\setlength\tabcolsep{1mm}
\renewcommand\arraystretch{1.3}
\title{DOA NOVENA-ROSARIO\\ ROH KUDUS}
\author{Untuk lingkungan St. Theresia Maguwo}
\date{~}
\newcommand{\roleSay}[2]{\begin{itemize} \item[#1:] #2 \end{itemize}}
\newcommand{\BU}[1]{\roleSay{U}{#1}}
\newcommand{\BI}[1]{\roleSay{I}{#1}}
\newcommand{\BP}[1]{\roleSay{P}{#1}}

\newcounter{urut}
\newcommand{\bait}[1]{%
  \begin{enumerate}
  \slshape
  \setcounter{enumi}{\value{urut}}
  \item #1
  \setcounter{urut}{\value{enumi}}
  \end{enumerate}	
}

\begin{document}
\maketitle
\section{DATANGLAH, YA ROH PENCIPTA\\ (Puji Syukur, No. 565)}

\roleSay{Bersama}{Datanglah, ya Roh Pencipta. Hati kami kunjungilah. Penuhi
dengan RahmatMu. Jiwa kami~ciptaanMu.}

\roleSay{Putri}{Kau digelari Penghibur, karunia Allah yang luhur. Kau hidup,
api, dan kasih, dan pengurapan~ilahi.}

\roleSay{Putra}{Dikau sapta karunia, dan tangan kanan ilahi. Engkau yang
Bapa janjikan Kau pergandakan~bahasa.}

\roleSay{Putri}{Sinari hati umatMu, dan curahkanlah cintaMu. Semoga Dikau
kuatkan yang rapuh dalam~tubuhnya.}

\roleSay{Putra}{Halaulah musuh umatMu. Berilah kami damaiMu, agar dengan
tuntunanMu kami hindarkan~yang jahat.}

\roleSay{Putri}{Buatlah kami mengenal serta mengimani terus Bapa dan Putra yang
tunggal dan Engkau Roh~keduanya.}

\roleSay{Bersama}{Dipujilah Allah Bapa dan Putra yang sudah bangkit, serta Roh
Kudus Penghibur, kini dan sepanjang masa. Amin.}

\section{PEMBUKAAN}

\BP{Demi nama Bapa, dan Putera, dan Roh Kudus.}

\BU{Amin.}

\BP{Semoga Allah Bapa serta Putera-Nya, Tuhan kita Yesus Kristus,
memberikan Roh Kudus kepada kita.}

\BU{Sekarang dan selama-lamanya.}

\BP{Saudara-saudara. Setelah Tuhan Yesus naik ke surga, ibu Maria dan
para rasul berkumpul di ruang makan lantai dua sebuah rumah,
bersama-sama berdoa MOHON TURUNNYA ROH KUDUS(Kisah Para Rasul 1:13).

Marilah kita bersama ibu Maria dan para rasul berdoa denganintensi
(ujub)~agar:
\begin{enumerate}
\item Kita masing-masing beserta
keluarga
\begin{enumerate}
\item diberkati Tuhan dan
dicurahi Roh
Kudus. 
\item Agar kuasa kegelapan diusir dari kita, keluarga dan rumah kita.

\item Agar kita dianugerahi iman-pengharapan-kasih,
kebijaksanaan-kesucian-kesetiaan, kebahagiaan dan keceriaan, baik dalam
untung maupun malang.

\item Kesehatan badan-jiwa, dan dibebaskan dari bahaya
jasmani-rohani.

\item Pekerjaan yang lancar dan rejeki yang cukup.
\end{enumerate} 

\item Agar anak-anak kita menjadi {\textquotedblleft}Pria
Utama{\textquotedblright} dan {\textquotedblleft}Wanita
Utama{\textquotedblright}.

\item Dan {\dots}{\dots}{\dots}.~(Hening sejenak).
\end{enumerate}
}
\section{DOA ROH KUDUS\\(bersama-sama)}
\begin{itemize}
\item Datanglah Roh Kudus, penuhilah hati umat-Mu, dan nyalakanlah di
dalamnya~\textbf{\emph{API CINTAMU}}. Utuslah Roh-Mu, maka semuanya
akan dijadikan lagi.

\item Dan Engkau akan memperbaharui muka bumi.

\item Marilah berdoa.

\item Ya Allah / Engkau yang mengajar hati umat-Mu dengan penerangan Roh
Kudus. / Berilah kami, dengan pengantaraan Roh
Kudus,\textbf{\emph{KEBIJAKSANAAN YANG SEJATI}}, / serta
karunia~\textbf{\emph{SELALU MERASA GEMBIRA}}~karena penghiburan-Nya. /
Demi Kristus, Tuhan dan pengantara kami..~Amin.
\end{itemize}

\section{BACAAN KITAB SUCI}
\textbf{\textit{Bacaan dapat diambil dari daftar pada halaman \pageref{bacaanLain}}}
\section{RENUNGAN}
Mari kita~renungkan~Bacaan Kitab Suci tadi. (minimal 3 menit).

\section{DOA PERMOHONAN}
Mari kita sampaikan kepada Allah
Bapa-Putera-Roh-Kudus permohonan kita tersebut di atas dan permohonan
pribadi kita masing-masing.

\section{DOA-DOA LAIN}.~(Di sini dapat juga ditambahi Doa-doa
lain).

\section{MOHON TUJUH KARUNIA ROH KUDUS\\{\small Yesaya
11:2-3.}}

\roleSay{Bersama}{Datanglah, ROH NASIHAT. Dampingilah kami dalam hidup yang
penuh gejolak ini. Tunjukkanlah kepada kami: yang baik dan benar; Tuhan serta rahmat-Nya dan doronglah kami mencintai-Nya serta
menjauhi dosa.}

\roleSay{Putri}{Datanglah, ROH PENGERTIAN. Terangilah budi kami agar dapat
melihat dan memilah-milah baik-buruk benar-salah. Memahami ajaran
Yesus, serta melihat hasilnya yang baik kalau kami melaksanakannya.}

\roleSay{Putra}{Datanglah, ROH YANG MENGENAL ALLAH. Ajarilah kami agar mampu
{\textquotedblleft}melihat{\textquotedblright} Tuhan dan rahmat-Nya di
sekitar kami. {\textquotedblleft}Melihat{\textquotedblright}
ke-selamatan dalam tugas yang berat.
{\textquotedblleft}Melihat{\textquotedblright} sifat-sementara dari hal
yang duniawi dan {\textquotedblleft}melihat{\textquotedblright} yang
jahat serta malapetaka yang diakibatkannya.}

\roleSay{Putri}{Datanglah, ROH TAKUT KEPADA ALLAH. Ajarlah kami takut kepada
Tuhan. Dan membenci dosa serta takut kepada akibat buruk dari padanya.}

\roleSay{Putra}{Datanglah, ROH HIKMAT KEBIJAKSANAAN. Ajarilah kami menjadi
bijaksana. Mampu menghargai, memilih, dan mencintai yang baik, terutama
cita-cita surga, walaupun berat dan pahit. Agar kami lebih mencintai
Tuhan dan kesucian, daripada diri kami sendiri. Dan agar kami membenci
yang jahat dan dosa, serta Kau bebaskan daripadanya}

\roleSay{Putri}{Datanglah, ROH KESALEHAN. Dampingilah dan bimbinglah kami
untuk~\emph{melaksanakan}~yang baik dan yang berkenan pada-Mu. Menjadi
orang yang~\emph{tahu berterima kasih}~atas segala kebaikan Tuhan. Agar
kami~\emph{taat}~dan~\emph{mengabdi}Tuhan di manapun kami
berada.~\emph{Menunaikan tugas}sebagai anggota keluarga dan masyarakat.
Menjadi teladan-kesalehan bagi lingkungan kami. Serta mampu menggunakan
hal-hal duniawi dengan bijaksana, hanya demi kemuliaan-Mu saja.}

\roleSay{Bersama}{Datanglah, ROH KEPERKASAAN. Kuatkanlah hamba-Mu yang lemah
ini, agar~\emph{setia}~melaksanakan kebaikan dan kebenaran dalam hidup
kami sehari-hari, serta menjauhi yang jahat, betapapun berat dan
pahitnya. Agar kami\emph{tabah}~dalam segala kesulitan dan derita.
Kuatkanlah kami bilamana kami selalu memegang tangan-Mu yang senantiasa
menuntun kami. Amin.}

\section{Rosario Roh Kudus}

\newcommand{\BSK}{\begin{quote}
\textbf{Bapa Kami \dots (1X)\\
Salam Maria \dots (1X)\\
Kemuliaan \dots (7X)}
\end{quote}
}

\subsection*{Misteri Pertama\\
"Dari Roh Kuduslah Yesus dikandung Perawan Maria."}
Renungan:  "\textit{Roh Kudus akan turun atasmu dan kuasa Allah yang Mahatinggi akan menaungi engkau; sebab itu anak yang akan kau lahirkan itu akan disebut kudus, Anak Allah.}"

Dengan tekun, mintalah bantuan dari Roh Ilahi serta perantaraan Bunda Maria untuk mengikuti kebajikan-kebajikan Yesus Kristus, contohlah segala kebajikan-Nya, sehingga kita dapat menjadi serupa dengan citra Putra Allah.

\BSK

\subsection*{Misteri Kedua\\"Roh Allah turun atas Yesus."}
Renungan : \textit{"Sesudah dibaptis, Yesus segera keluar dari air, dan pada waktu itu juga langit terbuka dan Ia melihat Roh Allah seperti burung merpati turun ke atasnya."}

Peliharalah dengan penuh kesungguhan anugrah yang tak ternilai, rahmat pengudusan yang dicurahkan dan ditanamkan dalam jiwa kita oleh Roh Kudus pada saat pembabtisan. Peganglah dengan teguh janji baptis yang telah kita ucapkan: tingkatkan iman, harapan dan cinta kasih melalui tindakan nyata, serta hiduplah sebagai anak-anak Allah dan anggota Gereja Allah yang sejati agar kelak kita dapat memperoleh warisan surgawi.

\BSK 

\subsection*{Misteri Ketiga\\"Oleh Roh Kudus, Yesus dibimbing menuju padang gurun untuk dicobai."}

Renungan : \textit{"Yesus, yang penuh dengan Roh Kudus, kembali dari Sungai Yordan, lalu dibawa oleh Roh Kudus ke padang gurun. Di situ Ia tinggal empat puluh hari lamanya dan dicobai Iblis."}

Bersyukurlah selalu atas ketujuh karunia Roh Kudus yang dicurahkan pada kita saat menerima Sakramen Penguatan: Roh kebijaksanaan, pengertian, nasihat, keperkasaan, pengenalan akan Allah, kesalehan, dan rasa takut akan Allah. Serahkan diri kita dengan setia kepada bimbingan Ilahi-Nya, sehingga di atas segala godaan dan pencobaan hidup kita berlaku secara perkasa sebagai seorang Kristen sejati dan prajurit Kristus yang berani.

\BSK

\subsection*{Misteri Keempat\\"Peranan Roh Kudus dalam Gereja."}

Renungan : \textit{"Tiba-tiba turunlah dari langit suatu bunyi seperti tiupan angin keras yang memenuhi seluruh rumah di mana mereka duduk.... Maka penuhlah mereka dengan Roh Kudus, lalu mereka mulai berkata-kata ... tentang perbuatan-perbuatan besar yang dilakukan Allah."}

Bersyukurlah kepada Tuhan karena Ia menjadikan kita sebagai anggota Gereja-Nya yang selalu dijiwai dan diarahkan oleh Roh Kudus, Roh yang diturunkan ke dunia untuk tugas itu pada hari Pentekosta. Dengarlah dan patuhilah Takhta Suci, wakil Roh Kudus yang tidak dapat salah, serta Gereja, pilar dan dasar kebenaran. Junjunglah ajaran-ajarannya dan belalah hak-haknya.

\BSK

\subsection*{Misteri Kelima\\"Roh Kudus dalam jiwa-jiwa orang beriman."}

Renungan : \textit{"Tidak tahukah kamu, bahwa tubuhmu adalah bait Roh Kudus yang diam di dalam kamu?"; "Janganlah padamkan Roh."; "Dan janganlah kamu mendukakan Roh Kudus Allah, yang telah memateraikan kamu menjelang hari penyelamatan."}

Sadarilah keberadaan Roh Kudus dalam diri kita, peliharalah dengan seksama kemurnian tubuh dan jiwa, ikutilah dengan setia bimbingan Ilahi-Nya, sehingga kita dapat menghasilkan buah-buah Roh: kasih, sukacita, damai sejahtera, kesabaran, kemurahan hati, kebaikan, kesetiaan, kelemah lembutan, iman, kerendahan hati, penguasaan diri, dan kemurnian.

\BSK

\begin{quote}
\textbf{Aku Percaya \ldots\\
Bapa Kami \ldots\\
Salam Maria \ldots}
\end{quote}


\section{PENUTUP}

\BP{Semoga Tuhan beserta kita.}

\BU{Sekarang dan selama-lamanya.}

\BP{Semoga kita semua, beserta keluarga dan karya kita, diberkati oleh
Allah yang mahakuasa : \Cross{} Bapa, dan Putera, dan Roh Kudus.}

\BU{Amin.}

\BP{Sdr-sdr dengan ini Novena Roh Kudus, hari ke-\ldots sudah selesai.}

\BU{Syukur kepada Allah.}

\section*{CURAHKAN RAHMAT DALAM HATIKU}

\roleSay{\textit{Refr.}}{\textit{\textbf{Curahkan rahmat dalam hatiku.\\ Ciptakan hati, dan semangat baru.}}}

\bait{Engkau Kucucikan dan Kubersihkan dari cinta-diri.\par Engkau Kuhidupkan dan Kukobarkan cinta di hati.}

\bait{Hatimu yang kaku, keras, dan beku, Kuambil darimu.\\
Ambillah dari-Ku semangat baru dalam karyamu.} 


\section*{BACAAN-BACAAN NOVENA ROH KUDUS}
\label{bacaanLain}
\newcommand{\hari}[2]{\roleSay{Hari-#1}{#2}}

\hari{1}{Pada awal mula Allah menciptakan langit dan bumi. Bumi belum
berbentuk dan kosong, dan~\textbf{\emph{Roh Allah~}}melayang-layang di
atas permukaan air~(Kejadian 1:1).~Tuhan Allah membentuk manusia dari
debu tanah dan menghembuskan~\textbf{\emph{Nafas Hidup~}}ke dalam
hidungnya; demikianlah manusia itu menjadi mahluk yang hidup (Kejadian
2:7; Yohanes 1:3; Kolose 1:15-16).

\textit{\textbf{Inti bacaan:}}
\begin{enumerate}
\item Yang mencipta bumi dan manusia
adalah Allah Tritunggal Mahakudus (Allah Bapa, Allah Putera, dan Allah
Roh Kudus). 
\item Ingat. Dalam diri kita dan diri pendosa, ada Roh Kudus
yang sedang menggarap kita tiap hari menjadi Manusia seutuhnya, Citra
Allah, Anak Allah.
\end{enumerate}
}

\hari{2}{\textbf{\emph{Roh Tuhan~}}membawa aku ke suatu lembah yang
penuh tulang-tulang~(=umat Allah yang kehilangan martabatnya).~Kemudian
atas perintah Allah, tulang-tulang menyatu, mendapat daging dan kulit,
menjadi manusia-manusia tetapi belum hidup. Lalu~\textbf{\emph{Nafas
Hidup~}}dihembuskan kedalam mereka sehingga mereka
menjadi~\textbf{\emph{manusia yang hidup}}. Mereka menjejakkan kakinya.
Oh, suatu~\textbf{\emph{bala tentara}}~yang sangat besar
jumlahnya~(Yeheski-el 37:1-10). 

\textit{\textbf{Inti bacaan:}}
\begin{enumerate}
\item Dengan Roh Kudus, tiap hari Allah
menggarap kita serta para pendosa yang rusak karena dosanya, kembali
menjadi manusia seutuhnya, Citra Allah, Anak Allah. \item Apa tujuannya?
Menjadi bala tentara Tuhan dan Kalifatullah (Duta Wakil Allah) yang
harus memperindah dan memakmurkan bumi lahir batin (=membentuk
pribadi-pribadi yang indah, hubungan mesra dengan Tuhan, dan
Persaudaraan Sejati Keluarga Allah).
\end{enumerate}
}


\hari{3}{Pada hari ke-3 setelah wafat, Tuhan Yesus bangkit lalu
menampakkan Diri kepada murid-murid-Nya dalam suatu rumah yang terkunci
(karena takut pada orang Yahudi). Kata Tuhan Yesus kepada mereka,
{\textquotedblleft}Damai sejahtera bagimu{\textquotedblright}.
Kemudian~\textbf{\emph{Ia menghembusi mereka~}}dan berkata
{\textquotedblleft}Terimalah\textbf{\emph{Roh
Kudus{\textquotedblright}}}~(Yohanes 20:19-23).~

\textit{\textbf{Inti bacaan:}}
\begin{enumerate}
\item Dengan berdosa,
Adam-Hawa dan juga kita mengusir Roh Kudus dari hati kita.
\item Kedatangan Tuhan Yesus ke dunia adalah untuk mengembalikan Roh Kudus
kepada semua orang~\emph{tanpa kecuali}, agar memperbaiki dan
memulihkan mereka menjadi manusia seutuhnya, Citra Allah, Anak Allah.
\end{enumerate}
}

\hari{4}{Ada seorang lumpuh berbaring di tepi kolam Betesda. Sudah 38
tahun dia mau mengambil kesempatan pertama untuk mencelupkan tangannya
ke kolam ketika air kolam digoyangkan malaikat, supaya disembuhkan.
Tapi tak pernah berhasil, karena lumpuh. Tuhan Yesus datang padanya dan
bertanya {\textquotedblleft}Maukah engkau sembuh?{\textquotedblright}.
Jawabnya~\emph{{\textquotedblleft}Ya Tuhan saya
mau{\textquotedblright}}. Lalu~\textbf{\emph{Tuhan Yesus ber-sabda
kepadanya {\textquotedblleft}Bangunlah dan angkatlah tilammu dan
berjalanlah}}{\textquotedblright} Maka orang itu sembuh~(Yohanes
5:12).~

\textit{\textbf{Inti bacaan:}}
\begin{enumerate}
\item Orang lumpuh tersebut gambaran kita dan umat manusia yang
berdosa (Tak mampu berdoa dan beramal. Doa dan amal pendosa itu palsu).
\item Kedatangan Tuhan Yesus ke dunia adalah untuk memulihkan Roh Kudus
dan kekuatan illahi dalam umat manusia. Menjadi kuat, baik, dan suci.
\end{enumerate}
}


\hari{5}{Maria Magdalena~\textbf{\emph{berjumpa dengan Tuhan Yesus}},
maka dia dibebaskan dari tujuh roh jahat~(Yohanes 8:3; Lukas 8:2;
Matius 26:6-13).Zakeus~\textbf{\emph{berjumpa dengan Tuhan
Yesus,~}}maka dia diubah menjadi sosial, berjanji akan memberikan
separo miliknya kepada orang miskin; dan kalau ada orang yang
diperasnya, dia akan mengembalikan uangnya 4x lipat~(Lukas
19:1-10).~Saulus \textbf{\emph{berjumpa dengan Tuhan Yesus,}} maka dia
diubah menjadi Paulus Rasul Yesus Kristus (Kisah 9). Perbaikan pendosa
bukan pertama-tama dengan doa dan amal, tetapi dengan datangnya kasih
Allah (Roh Kudus) yang berkarya dalam diri kita. Baru~kedua, kita harus
menanggapinya secara positif dengan doa dan amal, agar rahmat menjadi
efektif dalam diri kita. Doa dan amal itu mutlak namun yang kedua. Yang
pertama, karya agung Allah.
}

\hari{6}{Inilah tanda~\textbf{\emph{orang yang percaya (kepada Roh
Kudus):}} Diamenerima kuasa mengusir setan, berbahasa gaib, memegang
ular berbisa, diracun tidak celaka dan menyembuhkan orang sakit (Markus
16:17-18). Tidak usah menunggu romo atau pendeta atau dukun. Orang
beriman bisa berbuat apa-apa saja dalam DIA yang menguatkannya (Filipi
4:13; 1 Yohanes 5:4).}

\hari{7}{\textbf{\emph{Buah roh}}\textbf{\emph{~}}ialah : kasih,
sukacita, damai sejahtera, kesabaran, murah-hati, kebaikan, setia,
lemah-lembut, ugahari dan pengandalian diri.~(Galatia
5:22-23).~\textbf{\emph{Kasih}}~itu sabar, murah hati, tidak cemburu,
tidak sombong, selalu sopan, tidak mencari keuntungan diri, tidak
pemarah, tidak pendendam, tidak senang kalau orang lain diperlakukan
tidak adil, cinta kebenaran, menutupi segala sesuatu, tidak membocorkan
rahasia, punya pengharapan yang besar, kuat menanggung derita~(1
Korintus 13:4-7) 

\textit{\textbf{Inti bacaan:}}

Ugahari = tahu batas.~Pribadi yang berhasil digarap
oleh Roh Kudus menjadi pribadi yang indah seperti tersebut di atas.
Bacalah sekali lagi keindahan pribadi kita itu.
}

\hari{8}{(Kisah Para Rasul 2:41-47; 4:32-37).~Panca tugas umat Allah
(yang penuh Roh Kudus) = Paguyuban, doa, pendalaman Injil, bakti
sosial, kemasyarakatan. Komunitas kita harus berciri 5~ini.

\textit{\textbf{Inti bacaan:}}
Setelah Tuhan Yesus naik ke surga, para rasul menyebar ke
seluruh dunia dan mewartakan Injil. Orang-orang yang~\textbf{\emph{mau
menerima Firman,~}}minta dibaptis dan berkat karya Roh Kudus,
mereka~
\begin{enumerate}

\item ikut aktif dalam paguyuban,\item rajin dalam doa-doa bersama,
ikut Ekaristi,\item tekun dalam pendalaman Injil,\item segala kepunyaan
mereka diserahkan menjadi milik bersama.\item mereka disukai semua
orang.~
\end{enumerate}
}

\hari{9}{Roma
8:1-17, Perbuatan~daging~telah nyata, yaitu : percabulan, kecemaran,
hawa nafsu, menyembah berhala, guna-guna dan sant\`et, perseteruan dan
perselisihan, iri hati, amarah, kepentingan diri, percideraan, roh
pemecah belah, dengki, mabuk-mabukan, pesta pora, dsb. Galatia
5:19-21

Saudara-saudara. Jika kamu menyambut Roh Allah dalam
hatimu, \textbf{\emph{Roh Allah }}memang diam dalam kamu. Kamu tidak
lagi hidup dalam daging, melainkan \textbf{\emph{hidup dalam Roh}}.
Kamu menjadi anak Allah (ay.14) dan akan mewarisi
sorga (ay.17).Selanjutnya kamu hanya memikirkan hal-hal
yang \textbf{\emph{dari}} \textbf{\emph{Roh}}(ay.5). Kamu \textbf{\emph{mematikan
perbuatan-perbuatan daging}} (ay.14).Dan
oleh \textbf{\emph{Roh }}kamu \textbf{\emph{bisa
berdoa }}{\textquotedblleft}Abba ya Bapa{\textquotedblright} .
}

\begin{center}
$\prec\smallsmile\succ$
\end{center}
\end{document}
