\documentclass[a5paper,titlepage,12pt]{scrbook}
\usepackage[a5paper,backref]{hyperref}
\usepackage[papersize={148.5mm,215mm},twoside,bindingoffset=0.5cm,hmargin={2cm,2cm},
				vmargin={2cm,2cm},footskip=1.1cm,driver=dvipdfm]{geometry}
\usepackage{palatino}
\usepackage[ascii]{inputenc}
\usepackage[T1]{fontenc}
\usepackage[bahasa]{babel}
\usepackage{amsmath}
\usepackage{amssymb,amsfonts,textcomp}
\usepackage{array}
\usepackage{supertabular}
\usepackage{hhline}
\usepackage[pdftex]{graphicx}
\usepackage{enumerate}
\makeatletter
\newcommand\arraybslash{\let\\\@arraycr}
\makeatother
\setlength\tabcolsep{1mm}
\renewcommand\arraystretch{1.3}
\title{}

\makeatletter
\renewcommand{\@makeschapterhead}[1]{%
  {\parindent \z@ \centering \normalfont
    \interlinepenalty\@M \Large \bfseries #1\par\nobreak \vskip 20\p@ }}
\renewcommand{\section}{\@startsection {section}{1}{\z@}%
                                   {-3.5ex \@plus -1ex \@minus -.2ex}%
                                   {2.3ex \@plus.2ex}%
%                                   {\normalfont\normalsize\bfseries\centering}}
                                   {\normalfont\normalsize\bfseries}}
\renewcommand\subsection{\@startsection{subsection}{2}{\z@}%
                                     {-3.25ex\@plus -1ex \@minus -.2ex}%
                                     {1.5ex \@plus .2ex}%
                                     {\normalfont\normalsize\bfseries}}
\renewcommand\subsubsection{\@startsection{subsubsection}{3}{\parindent}%
                                    {3.25ex \@plus1ex \@minus.2ex}%
                                    {-1em}%
                                    {\normalfont\normalsize\bfseries}}
\makeatother

\def\thesection{\arabic{section}.}
\setlength{\parindent}{0pt}

\hyphenation{Ke-ra-ja-an-Nya}
\hyphenation{me-re-ka}
\hyphenation{a-da-lah}
\hyphenation{me-ngan-dung}
\hyphenation{ke-su-kar-an}
\hyphenation{di-lin-dung-i}

\begin{document}


\chapter*{DOA ROSARIO HARI KE-12}
\begin{center}
\textit{TEMA UTAMA\\
Merenungkan serta memuji {\textquoteleft}peran{\textquoteright} Bunda Maria dalam Karya Keselamatan PuteraNya, Yesus Kristus, bagi kita.}
\end{center}

\section{Pembukaan}
\subsection*{Lagu Pembukaan}
\subsection*{Tanda Salib}
\subsection*{Pengantar} 
\begin{itemize}
\item Tema 'Iman Maria yang rendah hati'
\item Menyampaikan Peristiwa yang ingin diambil dalam doa ini, yaitu persitiwa mulia.
\item Mengajak umat untuk mempersiapkan batin.
\end{itemize}

\subsection*{Tobat} 

Bapak Ibu dan Saudara terkasih, menyadari bahwa kita adalah manusia yang lemah dan sering terjatuh dalam dosa, maka marilah memohon ampun kepada Allah agar kita pantas dan diperkenankan untuk memuliakan Tuhan dan menyampaikan segala permohonan kita kepada Allah Bapa dengan perantaran PuteraNya, Yesus Kristus. 

Saya mengaku \ldots

Semoga Allah Bapa yang mahakuasa melindungi kita, mengampuni dosa-dosa kita, dan menghantar kita ke hidup yang kekal, Amin.

Tuhan kasihanilah kami 2$\times$

Kristus kasihanilah kami 2$\times$

Tuhan kasihanilah kami 2$\times$


\subsection*{Doa Pembukaan}

Marilah berdoa,

Bapa yang mahabaik kami bersyukur atas segala rahmat dan karuniaMu, sehingga pada saat ini kami umatMu di lingkungan St Petrus dapat berkumpul untuk memuliakan namaMu dengan perantaraan doa rosario. Kami mohon ya Bapa, agar apa yang kami dapatkan dalam pertemuan ini semakin menguatkan iman kami dan semakin memuliakan namaMu. Semua ini kami mohon
dengan pengantaraan Yesus Kristus PuteraMu Tuhan kami yang bersama dengan Dikau dalam persekutuan Roh Kudus hidup dan berkuasa, Allah sepanjang segala masa.

Amin.

\section{Doa Rosario dan Bacaan}
\begin{itemize}
	\item Aku Percaya{\dots}.
	\item Kemuliaan{\dots}.
	\item Bapa Kami{\dots}..
	\item 3 Salam (Puteri Allah Bapa, Bunda Allah Putera, Mempelai Allah Roh Kudus), Salam Maria{\dots}{\dots}
	\item Kemuliaan{\dots}..
	\item Terpujilah{\dots}{\dots}
	\item Peristiwa Rosario, peristiwa mulia 
	
	\begin{enumerate}[a.]
		\item Tiga Misteri Kudus (3x persepuluhan pertama)		
			\begin{itemize}
				\item 
				 	\begin{enumerate}[\bfseries{Peristiwa }I.]
						\item \textbf{Yesus bangkit dari kematian}
						
Ya Yesus kebangkitanMu sebagai salah satu bukti keallahanMu. Kuatkanlah iman kami, agar semakin kokoh dalam menghadapi benturan-benturan dalam kehidupan kami.

Kami mohon:						
						
						\item \textbf{Yesus naik ke surga}
						
Ya Yesus, Engkau naik ke surga guna menyiapkan tempat bagi kami. Ingatkanlah kami bahwa tujuan akhir kami adalah hidup kekal bersamaMu di surga. Semoga kami dapat dengan yakin menolak godaan dunia yang rasanya lebih menyenangkan yang ternyata hanya sesaat.

Kami mohon:

						\item \textbf{Roh Kudus turun atas para rasul}

Ya Bapa, Kau kirim Roh Kudus atas para rasul sehingga mereka lebih kuat dan tangguh dalam menyebarkan kabar baik dariMu. Kirimlah juga Roh Kudus atas kami sehingga kami menjadi lebih kuat dalam iman dan lebih giat dalam bekerja demi berkembangnya kerajaanMu. 

Kami mohon:

					\end{enumerate}
				\item Bapa kami{\dots}{\dots}
				\item Salam Maria{\dots}.(10x)
				\item Kemuliaan{\dots}{\dots}
				\item Terpujilah{\dots}{\dots}
				\item Ya Yesus yang baik{\dots}{\dots}.
			\end{itemize}

		\item Bacaan
			\begin{description}
				\item [Bacaan dari Injil Lukas  1:46-56] 
					{~}
 
Lalu kata Maria: ``Jiwaku memuliakan Tuhan,dan hatiku bergembira karena Allah, Juruselamatku,
sebab Ia telah memperhatikan kerendahan hamba-Nya. Sesungguhnya, mulai dari sekarang segala keturunan akan menyebut aku berbahagia,
karena Yang Mahakuasa telah melakukan perbuatan-perbuatan besar kepadaku dan nama-Nya adalah kudus.

Dan rahmat-Nya turun-temurun atas orang yang takut akan Dia.
Ia memperlihatkan kuasa-Nya dengan perbuatan tangan-Nya dan mencerai-beraikan orang-orang yang congkak hatinya;
Ia menurunkan orang-orang yang berkuasa dari takhtanya dan meninggikan orang-orang yang rendah;
Ia melimpahkan segala yang baik kepada orang yang lapar, dan menyuruh orang yang kaya pergi dengan tangan hampa;
Ia menolong Israel, hamba-Nya, karena Ia mengingat rahmat-Nya,
seperti yang dijanjikan-Nya kepada nenek moyang kita, kepada Abraham dan keturunannya untuk selama-lamanya.''

Dan Maria tinggal kira-kira tiga bulan lamanya bersama dengan Elisabet, lalu pulang kembali ke rumahnya.

			\end{description}
	
	   \item Lagu selingan		

		\item Renungan

Bulan Mei dan Oktober biasanya dikenal sebagai bulan Maria. Pada bulan-bulan itu umat beriman secara khusus menghormati Bunda Gereja, sekaligus juga Bunda kita dengan berdoa rosario. Tentu saja berdevosi kepada Bunda Maria tidak semata-mata pada bulan Mei dan Oktober saja, dan dapat kita laksanakan setiap saat.

Injil Lukas 1:39-56, mengisahkan tentang Bunda Maria mengujungi Elisabeth saudarinya. Ketika Bunda Maria tiba di kediaman Elisabeth, ia sangat bergembira karena ibu Tuhan mau bersedia datang ke rumahnya, dan anak yang dikandung oleh Elisabeth dalam usia tua melonjak kegirangan. Keduanya sangat bahagia karena keduanya dipilih secara khusus oleh Allah.

Bunda Maria adalah teladan iman yang sederhana. Ia tidak banyak berbicara, tidak banyak menuntut, tapi dalam kepasrahannya ia menyerahkan segalanya kepada penyelenggaraan Ilahi. Ia sangat mendengarkan Putera-Nya. Kesetiaan dan kepasrahannya itu terlihat dalam ``Kidung Maria''. Kidung Maria yang terdiri atas 9 bait menunjukkan betapa Tuhan itu sangat berbelas kasih dan kasih setianya turun temurun. ``Jiwaku memuliakan Tuhan, dan hatiku bergembira karena Allah Juruselamatku, sebab Ia telah memperhatikan kerendahan hamba-Nya. Sesungguhnya, mulai sekarang segala keturunan akan menyebut aku bahagia''.

Pujian  manusia paling agung  bagi Tuhan adalah memuliakan Nama-Nya. Pujian itu menyentuh seluruh jiwa karena Tuhan telah menyelamatkan umat-Nya. Kidung Maria yang kita baca pada hari ini merupakan kidung kita sebagai manusia yang menantikan kehadiran Allah dalam hidup manusia.

Hidup kita adalah anugerah Ilahi yang diberikan secara gratis  ketika Tuhan datang menjenguk umat-Nya yang sedang berziarah di tengah ganasnya dunia.  Pilihan Allah menjadi manusia dalam diri Yesus adalah solidaritas tertinggi Allah kepada umat-Nya.  Bunda Maria yang mewakili kita yang bersyukur kepada Allah atas kehadiran-Nya dalam hidup Maria. Sikap Maria terhadap panggilan   Allah itulah yang perlu kita contoh. Maria menjadi model dalam hidup kita yang selalu bersikap rendah hati. Kerendahan hati Bunda Maria sangat mengagumkan kita semua, sehingga kita perlu menyontoh sikap hidupnya.  Sikap rendah hati, mengakui segala kelemahan dan menerima kehendak Tuhan adalah sikap yang paling pas dalam hidup kita sebagai umat beriman. 

Bunda Maria adalah contoh sekaligus teladan kerendahan hati. Banyak orang sulit untuk rendah hati. Dan tidak sedikit pula mereka yang menjadi sombong, seolah hanya dialah yang paling mampu.

Pada hal kerendahan hati adalah pangkal kekudusan hidup. Dengan mencontoh kerendahan hati Bunda Maria, sekaligus kita bisa ikut meneladani hidupnya. Kidung Maria begitu indah. Tak ada salahnya kalau kita bisa merenungkan dan mencontoh hidup Bunda Maria. 



	\item Lanjutan Peristiwa Rosario, peristiwa mulia 
		\begin{enumerate}[{Peristiwa }I.]
		    \setcounter{enumii}{3}
			\item \textbf{Maria diangkat ke surga}

Ya, Bunda, doakan kami agar kami dapat meneladani hidupmu dengan kerendahan hati serta kepasrahan terhadap penyelenggaraan Ilahi. Amin. 
			
Kami mohon:
			
			\item \textbf{Maria dimahkotai di surga}

Ya, Bapa, curahkan Roh-Mu yang kudus agar kami terus memiliki sikap kerendahan hati seperti  Bunda Maria, Amin. 
			
Kami mohon:
		\end{enumerate}
	\end{enumerate}
	\item Lagu
\end{itemize}

\section{Ibadat Penutup}
\begin{itemize}
\item Doa Penutup dan mohon berkat Tuhan

	Marilah berdoa,

Allah Bapa sumber kehidupan, kami telah Engkau segarkan dengan sabdaMu dan pengalaman iman. KehadiranMu memberi keteguhan bagi hidup kami sehari-hari dalam berusaha memenuhi kehendakMu melalui pelayanan dan  pekerjaan kami. Penuhilah kami dengan sikap rendah hati selaras dengan teladan Bunda Maria. Doa ini kami panjatkan dengan perantaraan Kristus Tuhan kami, Amin.

Saudara-saudari yang terkasih, marilah kita mengakhiri doa rosario kita, dengan mohon berkat Tuhan.

Tuhan beserta kita

sekarang dan selama-lamanya

Semoga kita sekalian yang hadir di sini, dilindungi dan dibimbing oleh berkat Allah yang Mahakuasa dalam nama Bapa, Putera, dan Roh Kudus

Amin.

\item Kolekte dan Lagu Penutup. 
\end{itemize}
\end{document}
