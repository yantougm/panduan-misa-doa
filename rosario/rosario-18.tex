\documentclass[a5paper,titlepage,12pt]{scrbook}
\usepackage[a5paper,backref]{hyperref}
\usepackage[papersize={148.5mm,215mm},twoside,bindingoffset=0.5cm,hmargin={2cm,2cm},
				vmargin={2cm,2cm},footskip=1.1cm,driver=dvipdfm]{geometry}
\usepackage{palatino}
\usepackage[ascii]{inputenc}
\usepackage[T1]{fontenc}
\usepackage[bahasa]{babel}
\usepackage{amsmath}
\usepackage{amssymb,amsfonts,textcomp}
\usepackage{array}
\usepackage{supertabular}
\usepackage{hhline}
\usepackage[pdftex]{graphicx}
\usepackage{enumerate}
\makeatletter
\newcommand\arraybslash{\let\\\@arraycr}
\makeatother
\setlength\tabcolsep{1mm}
\renewcommand\arraystretch{1.3}
\title{}

\makeatletter
\renewcommand{\@makeschapterhead}[1]{%
  {\parindent \z@ \centering \normalfont
    \interlinepenalty\@M \Large \bfseries #1\par\nobreak \vskip 20\p@ }}
\renewcommand{\section}{\@startsection {section}{1}{\z@}%
                                   {-3.5ex \@plus -1ex \@minus -.2ex}%
                                   {2.3ex \@plus.2ex}%
%                                   {\normalfont\normalsize\bfseries\centering}}
                                   {\normalfont\normalsize\bfseries}}
\renewcommand\subsection{\@startsection{subsection}{2}{\z@}%
                                     {-3.25ex\@plus -1ex \@minus -.2ex}%
                                     {1.5ex \@plus .2ex}%
                                     {\normalfont\normalsize\bfseries}}
\renewcommand\subsubsection{\@startsection{subsubsection}{3}{\parindent}%
                                    {3.25ex \@plus1ex \@minus.2ex}%
                                    {-1em}%
                                    {\normalfont\normalsize\bfseries}}
\makeatother

\def\thesection{\arabic{section}.}
\setlength{\parindent}{0pt}
\newcommand{\arwah}{Ibu Maria Fransiska Sariyem }

\hyphenation{Ke-ra-ja-an-Nya}
\hyphenation{me-re-ka}
\hyphenation{a-da-lah}
\hyphenation{me-ngan-dung}
\hyphenation{ke-su-kar-an}
\hyphenation{di-lin-dung-i}

\begin{document}


\chapter*{DOA ROSARIO HARI KE-18 dan DOA ARWAH HARI KE-40}
\begin{center}
\textit{TEMA UTAMA\\
Merenungkan serta memuji {\textquoteleft}peran{\textquoteright} Bunda Maria dalam Karya Keselamatan PuteraNya, Yesus Kristus, bagi kita.}
\end{center}

\section{Pembukaan}
\subsection*{Lagu Pembukaan}
\subsection*{Tanda Salib}
\subsection*{Pengantar} 
\begin{itemize}
\item Tema 'Iman yang kukuh dan kebimbangan'
\item Menyampaikan Peristiwa yang ingin diambil dalam doa ini.
\item Mengajak umat untuk mempersiapkan batin.
\end{itemize}

\subsection*{Tobat} 

Marilah kita dengan jujur dan ikhlas kita mengakui segala  dosa dan kelemahan kita dihadapan Allah dan sesama ; marilah  kita mohon ampun atas segala kelemahan dan kedosaan yang  telah kita perbuat. Secara khusus kita mohonkan ampun juga  atas kesalahan dan dosa dari saudara kita \arwah. yang dipanggil  Bapa di surga 40 hari yang lalu. 

Saya mengaku \dots


\subsection*{Doa Pembukaan}

Marilah berdoa,

Allah Bapa di surga, kami hunjukkan syukur kepadaMu karena malam ini kami Kau berkati sehingga dapat berkumpul di tempat ini. Allah mahakasih, Engkaulah pencipta dan penebus kami. Tuhan Yesus telah berjaya dengan mengalahkan maut dan  masuk kedalam kemuliaan abadiMu. Semoga ya Bapa,  hambaMu ini \arwah juga mengalahkan  maut dan masuk dalam kemuliaanMu untuk selama-lamanya. Demi Yesus Kristus Putra-Mu, Tuhan dan  pengantara kini dan sepanjang masa.
 
Amin.

\subsection*{Lagu Pengantar Rosario}

\section{Doa Rosario dan Bacaan}
\begin{itemize}
	\item Aku Percaya{\dots}.
	\item Kemuliaan{\dots}.
	\item Bapa Kami{\dots}..
	\item 3 Salam (Puteri Allah Bapa, Bunda Allah Putera, Mempelai Allah Roh Kudus), Salam Maria{\dots}{\dots}
	\item Kemuliaan{\dots}..
	\item Terpujilah{\dots}{\dots}
	\item Peristiwa Rosario, peristiwa sedih 
	
	\begin{enumerate}[a.]
		\item Tiga Misteri Kudus (3x persepuluhan pertama)		
			\begin{itemize}
				\item 
				 	\begin{enumerate}[\bfseries{Peristiwa }I.]
						\item \textbf{Yesus berdoa kepada Bapa-Nya di surga dalam sakratul maut}
						
Bunda Maria doakanlah arwah \arwah yang telah menghadap
Bapa, semoga ia menempati tempat yang telah disediakan olek Yesus bagi orang-rang yang
mengimaninya. Kami mohon:						
						
						\item \textbf{Yesus didera}
						
						Bunda Maria doakanlah sanak keluarga yang ditinggalkan oleh  \arwah. Semoga Bapa di surga memberikan penghiburan,
ketabahan, serta keteguhan kepada keluarga yang ditinggalkan.  Dan  berkat  pengharapan yang
kuat, mereka yang masih meneruskan peziarahan di dunia ini Kauberi terang dalam meneruskan
tugas dan panggilan mereka. Kami mohon:

						\item \textbf{Yesus dimahkotai duri}

Bagi orang-orang yang telah meninggal dan
yang masih mengharapkan belah kasihan Allah.
Semoga, berkat belas kasihan dan kerahiman
Allah, Ia memberikan pengampunan kepada mereka semua, 
sehingga mereka boleh beristirahat dalam kebahagiaan abadi bersama Bapa
di surga. Kami mohon:

					\end{enumerate}
				\item Bapa kami{\dots}{\dots}
				\item Salam Maria{\dots}.(10x)
				\item Kemuliaan{\dots}{\dots}
				\item Terpujilah{\dots}{\dots}
				\item Ya Yesus yang baik{\dots}{\dots}.
			\end{itemize}

		\item Bacaan
			\begin{description}
				\item [Bacaan dari Kitab  2 Mak 12:42-45] 
					{~}

						Merekapun lalu mohon dan minta, semoga dosa yang telah dilakukan itu dihapus semuanya. Tetapi Yudas yang berbudi luhur memperingatkan khalayak ramai, supaya memelihara diri tanpa dosa, justru oleh karena telah mereka saksikan dengan mata kepala sendiri apa yang sudah terjadi oleh sebab dosa orang-orang yang gugur itu.

Kemudian dikumpulkannya uang di tengah-tengah pasukan. Lebih kurang dua ribu dirham perak dikirimkannya ke Yerusalem untuk mempersembahkan korban penghapus dosa. Ini sungguh suatu perbuatan yang sangat baik dan tepat, oleh karena Yudas memikirkan kebangkitan.

Sebab jika tidak menaruh harapan bahwa orang-orang yang gugur itu akan bangkit, niscaya percuma dan hampalah mendoakan orang-orang mati.

Lagipula Yudas ingat bahwa tersedialah pahala yang amat indah bagi sekalian orang yang meninggal dengan saleh. Ini sungguh suatu pikiran yang mursid dan saleh. Dari sebab itu maka disuruhnyalah mengadakan korban penebus salah untuk semua orang yang sudah mati itu, supaya mereka dilepaskan dari dosa mereka. 

				\item [Lagu Selingan]

				\item [Bacaan Injil menurut Yohanes (5:6-15)]
				{~}
				
				Ketika Yesus melihat orang itu berbaring di situ dan karena Ia tahu, bahwa ia telah lama dalam keadaan itu, berkatalah Ia kepadanya: "Maukah engkau sembuh?"

Jawab orang sakit itu kepada-Nya: "Tuhan, tidak ada orang yang menurunkan aku ke dalam kolam itu apabila airnya mulai goncang, dan sementara aku menuju ke kolam itu, orang lain sudah turun mendahului aku."

Kata Yesus kepadanya: "Bangunlah, angkatlah tilammu dan berjalanlah."

Dan pada saat itu juga sembuhlah orang itu lalu ia mengangkat tilamnya dan berjalan. Tetapi hari itu hari Sabat.
Karena itu orang-orang Yahudi berkata kepada orang yang baru sembuh itu: "Hari ini hari Sabat dan tidak boleh engkau memikul tilammu."

Akan tetapi ia menjawab mereka: "Orang yang telah menyembuhkan aku, dia yang mengatakan kepadaku: Angkatlah tilammu dan berjalanlah."

Mereka bertanya kepadanya: "Siapakah orang itu yang berkata kepadamu: Angkatlah tilammu dan berjalanlah?"

Tetapi orang yang baru sembuh itu tidak tahu siapa orang itu, sebab Yesus telah menghilang ke tengah-tengah orang banyak di tempat itu. Kemudian Yesus bertemu dengan dia dalam Bait Allah lalu berkata kepadanya: "Engkau telah sembuh; jangan berbuat dosa lagi, supaya padamu jangan terjadi yang lebih buruk."

Orang itu keluar, lalu menceriterakan kepada orang-orang Yahudi, bahwa Yesuslah yang telah menyembuhkan dia.

 
			\end{description}

		\item Renungan


Sejak awal mula Gereja meyakini bahwa hidup kita tidak dilenyapkan oleh kematian melainkan diubah. Gereja menemani perjalanan orang beriman yang meninggal dan sedang menuju ke Allah itu melalui segala macam doa. Praktek doa Gereja bagi mereka yang meninggal dunia didasarkan, selain pada iman akan kebangkitan Kristus, juga pada \textit{communio sanctorum}, persekutuan orang-orang kudus dalam Gereja. Persekutuan itu meliputi: mereka yang sudah mulia bersama Allah di surga, mereka yang masih hidup di dunia ini, dan mereka yang sudah meninggal namun belum masuk secara penuh dalam kemuliaan Allah dan masih perlu mengalami penyucian. Untuk kelompok yang terakhir inilah doa peringatan arwah diadakan. Sebagai warga Gereja, mereka semua entah yang masih hidup atau sudah mati tetap saling berhubungan dan saling dukung dalam cinta dan doa.

Bapa melalui Putera-Nya Yesus menghendaki agar semua orang selamat dan akan dibangkitkan pada akhir zaman.
Walau demikian, tidak semua percaya kepada-Nya, bahkan mencaci dan membunuh-Nya utusan Allah atau Allah sendiri yang  hadir di tengah-tengah dunia ini. Semua orang tahu bahwa akan ada penghakiman terakhir, namun banyak yang tidak siap. Banyak yang tidak peduli atau lebih tertarik dengan tawaran dunia ini. Inilah saatnya bagi kita untuk mendoakan saudara-saudara kita yang sudah tiada, biarlah Tuhan melalui Yesus memberikan  pengampunan dan kerahiman sehingga mereka diperbolehkan  menikmati kehidupan bersama para kudus lainnya. 

Bantuan yang kita berikan untuk keselamatan kekal bagi yang telah meninggal adalah dengan berbuat silih. Silih itu bisa dengan mendoakan mereka, bisa pula dengan berbuat amal kasih bagi orang-orang yang membutuhkan amalan dan kasih kita demi mereka yang telah meninggal; agar mereka beristirahat dalam damai. (RIP: Requescat In Pace).

Dari bacaan Injil, kita bisa ambil contoh pola Yesus dalam memberi bantuan:

\textbf{Yang pertama, Yesus melihat sebuah kebutuhan}. Yohanes 5:5 menyatakan disitu ada seseorang yang sudah tiga puluh delapan tahun lamanya sakit. Seperti apakah rasanya sakit selama tiga puluh delapan tahun? Kalau itu saya, saya sudah menyerah dengan keadaan tersebut dan saya mulai melupakan kalau saya bisa sembuh. Bayangkan saja, orang tersebut sudah sakit demikian lama. Dia sudah mencoba apa yang bisa dia coba. Dia sudah ke dokter, kemanapun yang menjanjikan bisa menyembuhkan dia. Bahkan dia sudah mencoba mencari keajaiban dari kolam itu. Hanya saja, dia selalu terlambat untuk masuk ke kolam karena tidak ada yang menolongnya. Sepertinya orang-orang di sekitar dia tidak memperdulikan dia lagi. Tidak ada lagi orang yang siap sedia untuk menolong orang tersebut.
Sakitnya bukan hanya sakit secara fisik, tetapi sakit karena pengabaian yang dilakukan oleh orang di sekitarnya. Mungkin dulu dia memiliki istri, dan sekarang mungkin sudah meninggalkannya. Tidak ada yang tahu, bahkan namanya pun tidak tercatat di dalam alkitab, siapakah dia. Ada kebutuhan fisik dan psikologis. Kebutuhan fisik, karena dia ingin sembuh dari penyakitnya. Dan kebutuhan psikologis, karena dia ingin bersama-sama dengan teman-temannya dan 'dianggap' oleh orang disekitarnya.

\textbf{Yang kedua, Yesus menanyakan keinginan orang tersebut}. Sebenarnya terasa aneh ketika Yesus menanyakan ``Maukah engkau sembuh?'' Sangat jelas jika orang tersebut mau sembuh. Buktinya orang tersebut ada di sana dan menunggu air kolam bergerak dan dia berencana terjun ke dalam kolam tersebut. Tetapi mengapa Yesus menanyakan maukah orang tersebut sembuh?

Hal ini seakan-akan mengajarkan ke saya untuk tidak menebak-nebak kebutuhan seseorang. Yesus menanyakan keinginan ornag tersebut walaupun sebenarnya sudah sangat jelas. Tetapi Dia tidak bersikap sok tahu dan menyimpulkan apa yang menjadi keinginan orang tersebut. Pertanyaan tersebut untuk memastikan apa yang dibutuhkan orang tersebut. selain itu juga untuk menguji seberapa besar orang tersebut merasakan kebutuhan itu. Dengan jawaban dari orang tersebut, memperlihatkan bahwa dia sudah berusaha keras mengupayakan kesembuhan untuk dirinya sendiri.

\textbf{Yang ketiga, Yesus menantang orang tersebut melakukan apa yang tidak bisa dilakukan.} Unik sekali ketika Yesus menyuruh orang tersebut mengangkat tilam. Apakah Yesus itu buta, dah tahu kalau orang tersebut tidak bisa berjalan kok malah disuruh mengangkat tilam? Mengapa Dia tidak angkat orang tersebut? Ketika Dia menyuruh orang tersebut mengangkat tilam dan berjalan, Dia tidak menjanjikan kesembuhan. Dia tidak berkata, ``Aku akan menyembuhkanmu, karena itu angkat tilammu dan berlajanlah.''

Yesus tidak memanjakan. Dia tidak melakukan apa yang bisa dilakukan oleh orang tersebut. Bahkan terkesan, Yesus tidak melakukan apa-apa. Yesus hanya memberikan tantangan untuk melakukan sesuatu yang selama ini tidak pernah dilakukan. Memang, saya percaya, Yesus menggunakan kuasaNya untuk menyembuhkan. Tantangan tersebut memperlihatkan iman orang tersebut. Salah satu yang menentukan kesembuhan adalah iman orang tersebut. Jika orang tersebut tidak mau mengangkat tilam dan berjalan, mungkin orang tersebut tidak akan sembuh. Keinginan yang kuat dalam diri orang tersebutlah yang membuat dirinya melakukan apa yang tidak bisa dia lakukan.

\textbf{Yang keempat, Yesus memperingatkan bisa terjadi hal yang lebih buruk.} Peringatan jangan berbuat dosa lagi, mengingatkan orang tersebut untuk berubah. Tidak melakukan yang selama ini dia lakukan dan berubah kehidupannya. Jika itu tidak dilakukan, hati-hati, bisa terjadi hal yang lebih buruk.
Banyak orang yang sudah tertolong tetapi tetap dalam  keadaan miskin bahkan lebih parah lagi. Saya mendengar orang di daerah tertentu justru memiliki mental pengemis setelah ditolong sekian lama. Karena dia tetap dalam keadaan semula, dia tidak berubah. Pola pikirnya dan mentalnya tidak berubah. Jadi menurutnya, sudah menjadi kewajiban orang disekitarnya untuk menolong dia. Kalau orangt ersebut tidak menolong, maka dirinya marah.

Kita perlu memperingatkan kalau mereka harus berubah. Seperti yang Yesus lakukan, kalau tidak berubah maka keadaan bisa menjadi lebih buruk. Mereka harus mengubah pola pikir mereka. Mereka perlu menyadari bahwa diri merekalah yang paling menentukan dalam proses penyembuhan mereka. Tanpa perubahan dan usaha, mereka tidak akan berubah kehidupannya.

Inilah pola bantuan yang bisa kita pelajari dari Yesus. Bagaimana menolong seseorang tetapi tidak membuat mereka tergantung pada sang penolong. Bagaimana pertolongan tersebut benar-benar membebaskan orang-orang yang kita tolong.

	\item Lanjutan Peristiwa Rosario, peristiwa sedih 
		\begin{enumerate}[{Peristiwa }I.]
		    \setcounter{enumii}{3}
			\item \textbf{Yesus memanggul salibNya}
			
			Yesus Engkau telah memberi teladan dengan cara memanggul salibMu dengan mantap tidak ragu-ragu, tidak mengeluh. Kuatkanlah kami ya Yesus agar kami kuat dan tabah memanggul salib kami masing-masing. Kami mohon:
			
			\item \textbf{Yesus wafat di salib}
			
			Yesus, dengan wafatMu kami telah Kau bebaskan dari mati. Kami mohon ya Tuhan agar dalam menghadapi kematian kami tetap penuh harapan untuk hidup kembali bersama Kristus. Kami mohon:
		\end{enumerate}
	\end{enumerate}
	\item Lagu
\end{itemize}

\section{Ibadat Penutup}
\begin{itemize}
\item Doa Penutup dan mohon berkat Tuhan

	Marilah berdoa,

Allah Bapa sumber kehidupan, kami telah Engkau segarkan dengan sabdaMu dan pengalaman iman. KehadiranMu memberi keteguhan bagi hidup kami sehari-haru dalam berusaha memenuhi kehendakMu melalui pelayanan dan  pekerjaan kami. Allah Bapa kami yang mahabaik, semoga doa kami berguna bagi keselamatan  \arwah yang sudah
dipanggil Bapa 40 hari yang lalu. Ya Bapa, bebaskanlah dan bersihkanlah dia dari segala dosanya, limpahkanlah penebusan-Mu kepadanya. Demi Kristus Tuhan kami. 
Amin.

Saudara-saudari yang terkasih, marilah kita mengakhiri ibadat kita untuk \arwah dan doa rosario, dengan mohon berkat Tuhan.

Tuhan beserta kita

sekarang dan selama-lamanya

Semoga kita sekalian yang hadir di sini, dilindungi dan dibimbing oleh berkat Allah yang Mahakuasa dalam nama Bapa, Putera, dan Roh Kudus

Amin.

\item Kolekte dan Lagu Penutup. 
\end{itemize}
\end{document}
