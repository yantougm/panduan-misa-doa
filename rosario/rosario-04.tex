\documentclass[a5paper,titlepage,12pt]{scrbook}
\usepackage[a5paper,backref]{hyperref}
\usepackage[papersize={148.5mm,215mm},twoside,bindingoffset=0.5cm,hmargin={2cm,2cm},
				vmargin={2cm,2cm},footskip=1.1cm,driver=dvipdfm]{geometry}
\usepackage{palatino}
\usepackage[ascii]{inputenc}
\usepackage[T1]{fontenc}
\usepackage[bahasa]{babel}
\usepackage{amsmath}
\usepackage{amssymb,amsfonts,textcomp}
\usepackage{array}
\usepackage{supertabular}
\usepackage{hhline}
\usepackage[pdftex]{graphicx}
\usepackage{enumerate}
\makeatletter
\newcommand\arraybslash{\let\\\@arraycr}
\makeatother
\setlength\tabcolsep{1mm}
\renewcommand\arraystretch{1.3}
\title{}

\makeatletter
\renewcommand{\@makeschapterhead}[1]{%
  {\parindent \z@ \centering \normalfont
    \interlinepenalty\@M \Large \bfseries #1\par\nobreak \vskip 20\p@ }}
\renewcommand{\section}{\@startsection {section}{1}{\z@}%
                                   {-3.5ex \@plus -1ex \@minus -.2ex}%
                                   {2.3ex \@plus.2ex}%
%                                   {\normalfont\normalsize\bfseries\centering}}
                                   {\normalfont\normalsize\bfseries}}
\renewcommand\subsection{\@startsection{subsection}{2}{\z@}%
                                     {-3.25ex\@plus -1ex \@minus -.2ex}%
                                     {1.5ex \@plus .2ex}%
                                     {\normalfont\normalsize\bfseries}}
\renewcommand\subsubsection{\@startsection{subsubsection}{3}{\parindent}%
                                    {3.25ex \@plus1ex \@minus.2ex}%
                                    {-1em}%
                                    {\normalfont\normalsize\bfseries}}
\makeatother

\def\thesection{\arabic{section}.}
\setlength{\parindent}{0pt}
\newcommand{\arwah}{Vincentius Muharto }

\hyphenation{Ke-ra-ja-an-Nya}
\hyphenation{me-re-ka}
\hyphenation{a-da-lah}
\hyphenation{me-ngan-dung}
\hyphenation{ke-su-kar-an}
\hyphenation{di-lin-dung-i}

\begin{document}


\chapter*{DOA ROSARIO HARI KE-4 dan DOA ARWAH HARI KE-7}
\begin{center}
\textit{TEMA UTAMA\\
Merenungkan serta memuji {\textquoteleft}peran{\textquoteright} Bunda Maria dalam Karya Keselamatan PuteraNya, Yesus Kristus, bagi kita.}
\end{center}

\section{Pembukaan}
\subsection*{Lagu Pembukaan}
\subsection*{Tanda Salib}
\subsection*{Pengantar} 
\begin{itemize}
\item Tema 'Kepercayaan itu sulit'
\item Menyampaikan Peristiwa yang ingin diambil dalam doa ini.
\item Mengajak umat untuk mempersiapkan batin.
\end{itemize}

\subsection*{Tobat} 

Bapak Ibu dan Saudara terkasih, menyadari bahwa kita adalah manusia yang lemah dan sering terjatuh dalam dosa, maka marilah memohon ampun kepada Allah agar kita pantas dan diperkenankan untuk menyampaikan segala permohonan kita kepada Allah Bapa dengan perantaran Bunda PuteraNya, Yesus Kristus. 

Saya mengaku \dots


\subsection*{Doa Pembukaan}

Marilah berdoa,

Allah Bapa di surga, kami hunjukkan syukur kepadaMu karena malam ini kami Kau berkati sehingga dapat berkumpul di tempat ini. Bapa pada kesempatan ini kami memohon karuniaMu bagi saudara kami, \arwah yang telah Engkau panggil 7 hari yang lalu.  Persatukanlah kami dalam pengharapan akan kebahagiaan kekal di surga, kendati masih harus berjuang di dunia ini. Bantulah kami agar memiliki sukacita sejati dan saling menghibur satu sama lain. Bunda Maria dukunglah doa-doa kami malam ini agar berkenan pada Allah Bapa. Dengan pengantaraan Yesus Kristus PuteraMu Tuhan kami yang bersama dengan Dikau dalam persekutuan Roh Kudus hidup dan berkuasa, Allah sepanjang segala masa.

Amin.

\section{Doa Rosario dan Bacaan}
\begin{itemize}
	\item Aku Percaya{\dots}.
	\item Kemuliaan{\dots}.
	\item Bapa Kami{\dots}..
	\item 3 Salam (Puteri Allah Bapa, Bunda Allah Putera, Mempelai Allah Roh Kudus), Salam Maria{\dots}{\dots}
	\item Kemuliaan{\dots}..
	\item Terpujilah{\dots}{\dots}
	\item Peristiwa Rosario, peristiwa sedih 
	
	\begin{enumerate}[a.]
		\item Tiga Misteri Kudus (3x persepuluhan pertama)		
			\begin{itemize}
				\item 
				 	\begin{enumerate}[\bfseries{Peristiwa }I.]
						\item \textbf{Yesus berdoa kepada Bapa-Nya di surga dalam sakratul maut}
						
Bagi arwah saudara kita \arwah yang telah menghadap
Bapa, semoga ia menempati tempat yang telah disediakan olek Yesus bagi orang-rang yang
mengimaninya. Kami mohon:						
						
						\item \textbf{Yesus didera}
						
						Bagi sanak keluarga yang ditinggalkannya. Semoga Bapa di surga memberikan penghiburan,
ketabahan, serta keteguhan kepada keluarga yang ditinggalkan.  Dan  berkat  pengharapan yang
kuat, mereka yang masih meneruskan peziarahan di dunia ini Kauberi terang dalam meneruskan
tugas dan panggilan mereka. Kami mohon:

						\item \textbf{Yesus dimahkotai duri}

Bagi orang-orang yang telah meninggal dan
yang masih mengharapkan belah kasihan Allah.
Semoga, berkat belas kasihan dan kerahiman
Allah, Ia memberikan pengampunan kepada mereka semua, 
sehingga mereka boleh beristirahat dalam kebahagiaan abadi bersama Bapa
di surga. Kami mohon:

					\end{enumerate}
				\item Bapa kami{\dots}{\dots}
				\item Salam Maria{\dots}.(10x)
				\item Kemuliaan{\dots}{\dots}
				\item Terpujilah{\dots}{\dots}
				\item Ya Yesus yang baik{\dots}{\dots}.
			\end{itemize}

		\item Bacaan
			\begin{description}
				\item [Bacaan dari Kitab  Wahyu 21:1-5a] 
					{~}

					Lalu aku melihat langit yang baru dan bumi yang baru, sebab langit yang pertama dan bumi yang pertama telah berlalu, dan lautpun tidak ada lagi.
					
					Dan aku melihat kota yang kudus, Yerusalem yang baru, turun dari sorga, dari Allah, yang berhias bagaikan pengantin perempuan yang berdandan untuk suaminya.
					
					Lalu aku mendengar suara yang nyaring dari takhta itu berkata: ``Lihatlah, kemah Allah ada di tengah-tengah manusia dan Ia akan diam bersama-sama dengan mereka. Mereka akan menjadi umat-Nya dan Ia akan menjadi Allah mereka.
					
					Dan Ia akan menghapus segala air mata dari mata mereka, dan maut tidak akan ada lagi; tidak akan ada lagi perkabungan, atau ratap tangis, atau dukacita, sebab segala sesuatu yang lama itu telah berlalu.''
					
					Ia yang duduk di atas takhta itu berkata: ``Lihatlah, Aku menjadikan segala sesuatu baru!'' 

				\item [Lagu Selingan]

				\item [Bacaan Injil menurut Lukas (1:26-34)]
				{~}
				
				Dalam bulan yang keenam Allah menyuruh malaikat Gabriel pergi ke sebuah kota di Galilea bernama Nazaret,
				kepada seorang perawan yang bertunangan dengan seorang bernama Yusuf dari keluarga Daud; nama perawan itu Maria.
				
				Ketika malaikat itu masuk ke rumah Maria, ia berkata: ``Salam, hai engkau yang dikaruniai, Tuhan menyertai engkau.''
				
				Maria terkejut mendengar perkataan itu, lalu bertanya di dalam hatinya, apakah arti salam itu.
				
				Kata malaikat itu kepadanya: ``Jangan takut, hai Maria, sebab engkau beroleh kasih karunia di hadapan Allah.
				Sesungguhnya engkau akan mengandung dan akan melahirkan seorang anak laki-laki dan hendaklah engkau menamai Dia Yesus. Ia akan menjadi besar dan akan disebut Anak Allah Yang Mahatinggi. Dan Tuhan Allah akan mengaruniakan kepada-Nya takhta Daud, bapa leluhur-Nya, dan Ia akan menjadi raja atas kaum keturunan Yakub sampai selama-lamanya dan KerajaanNya tidak akan berkesudahan.''
				
				Kata Maria kepada malaikat itu: ``Bagaimana hal itu mungkin terjadi, karena aku belum bersuami?''
 
			\end{description}

		\item Renungan

Pertanyaan yang ditanya oleh orang-orang a-dalah, bagaimanakah mungkin seorang perawan dapat hamil?
Janganlah kita salah, sebab tidak ada yang mustahil bagi Allah. Bukan hanya kelahiran
Yesus Kristus saja yang merupakan suatu mujizat, tetapi kehamilan dan kelahiran seorang bayi
biasa pun itu adalah suatu mujizat yang Tuhan lakukan.

Maria harus membayar harga dalam menerima mujizat dari Tuhan. Nama baiknya dapat tercemar 
karena dia hamil saat belum bersuami. Maria tahu persis konsekuensi dari mujizat yang dia 
terima, tetapi dia tetap terima dan mengucap syukur.

Jika kita ingin menerima mujizat, kita juga harus siap untuk membayar harga. 
\textit{There is no crown without the cross}.

Sebagai sosok manusia yang terdekat dengan Tuhan, tentunya Maria mempunyai banyak nilai yang bisa kita teladani. 
Maria adalah seseorang yang Tuhan rencanakan sejak awal untuk kedatanganNya ke dunia ini. Tuhan sangat kuasa, Dia mempersiapkan segala sesuatu di luar pikiran kita, dan Dia pun mempersiapkan kedatanganNya di dunia ini, Dia telah menyiapkan Maria untuk mengandung oleh Roh Kudus. 

Seperti Tuhan telah menyiapkan Yohanes Pembabtis untuk kedatanganNya demikian pula dengan sosok bunda yang menyatu dengan diriNya, tentunya Tuhan telah mempersiapkannya.

Lihatlah apa yang dilakukan Maria dan apa yang Tuhan berikan setelah Maria menjawab panggilan Tuhan. 
Dengan kerendahan hatinya, mengetahui siapa dirinya dan mengetahui kuasa Allah Bapa, Maria mencontohkan kepada kita bagaimana menjalani hidup ini, dia mencontohkan bagaimana dia berhubungan dengan Tuhan, dan apakah dia pernah membanggakan dirinya sebagai bunda yang melahirkan Yesus? 

Pada saat pertama kali Tuhan menyampaikan salam kepada Maria melalui malaikan Gabriel, 'Salam hai engkau yang dikaruniai, Tuhan menyertai engkau' (Lukas 1:28). 'Kata malaikat itu kepadanya: "Jangan takut, hai Maria, sebab engkau beroleh kasih karunia di hadapan Allah. Sesungguhnya engkau akan mengandung dan akan melahirkan seorang anak laki-laki dan hendaklah engkau menamai Dia Yesus (Lukas 1:30-31) 

Dan reaksi Maria adalah sesuatu yang sangat logis sebagai manusia, dia mengatakan 'Bagaimana hal itu mungkin terjadi, karena aku belum bersuami?' (Lukas 1:34) 

Bagi kita, terutama seorang gadis, kabar itu adalah kabar yang sangat mengagetkan bukan, dimana Tuhan menyapanya dan memberitahukan bahwa dia akan mengandung dan dia masih seorang gadis, bagaimana tanggapan lingkungannya , apa pikiran orang lain terhadapnya? apakah ini kabar baik bagi Maria? apakah ini suatu kabar yang menakutkan dia. 

Tapi Allah telah mengatur semua itu dan malaikat itu menyampaikan hal ini kepada Maria : 'Roh Kudus akan turun atasmu dan kuasa Allah Yang Mahatinggi akan menaungi engkau; sebab itu anak yang akan kaulahirkan itu akan disebut kudus, Anak Allah' (Lukas 1:35) 

Dari perkataan itu Maria menanggapi semua itu, Maria pasrahkan semua itu, dia tau siapa dirinya, dan dia tahu kehendap Allah bapa. 

Jika Tuhan memanggil kita, apakah kita siap seperti Maria, apakah kita siap meninggalkan segala sesuatu, menanggung semua ini dan berkata seperti Maria? ataukah kita mengelak, tidak mendengarkan? 

'Bagaimana hal itu mungkin terjadi, karena aku belum bersuami?'

Pertanyaan Maria di sini bukanlah suatu ungkapan ketidakpercayaan atas perkataan malaikat Gabriel. Maria tidak meminta tanda dari Tuhan sebagai bukti ataupun jaminan atas pesan-Nya melalui malaikat Gabriel. Sebaliknya, pertanyaan ini menyatakan:

(1). Iman Maria. Maria percaya bahwa hal itu akan terjadi.

(2). Penerimaan Maria. Maria menerima rencana Tuhan. Dia mengajukan pertanyaan ini karena ingin mengetahui apakah ada sesuatu yang dapat dilakukannya untuk menggenapi rencana Tuhan. Misalnya: Apakah dia harus menikah dengan Yusuf atau seseorang lain untuk memenuhi rencana Tuhan? Dia melihat kondisi ``tidak menikah''-nya sebagai suatu hambatan atau sebagai suatu ketidakmampuannya dalam merealisasikan kehendak Tuhan. Dengan jujur dan rendah hati dia membawa kondisi ketidakmampuannya ini kepada Tuhan. Jadi, ini merupakan suatu jawaban ``ya'' yang aktif dan rendah hati. Maria tidak hanya menerima rencana Tuhan atau panggilan Tuhan secara pasif, tetapi dia mengambil inisiatif untuk bekerja sama dengan membawa seluruh keberadaan dirinya, termasuk kelemahan dan ketidakmampuannya. Dia bertanya kepada Tuhan bagaimana menghadapi hambatan ini. Kita lihat bahwa dia tidak berputus asa, karena dia percaya bahwa pesan itu akan terlaksana. Jadi, iman Maria adalah iman yang aktif, iman yang mencari penjelasan. Dalam hidup rohani kita, kita juga menemui kesukaran-kesukaran dalam melaksanakan kehendak Tuhan. Maria mengajak kita untuk membawa seluruh diri kita, termasuk kelemahan dan ketidakmampuan juga kesukaran-kesukaran kita semua kepada Tuhan dalam iman dan harapan. Dia juga mengajak kita memakai akal budi kita untuk memahami lebih baik apa yang telah kita percayai dalam iman, untuk melaksanakan kehendak-Nya. Memang percaya penuh pada Tuhan itu sulit. Tetapi dengan campur tangan Tuhan, segala kesulitan bisa diatasi.

	\item Lanjutan Peristiwa Rosario, peristiwa sedih 
		\begin{enumerate}[{Peristiwa }I.]
		    \setcounter{enumii}{3}
			\item \textbf{Yesus memanggul salibNya}
			
			Yesus Engkau telah memberi teladan dengan cara memanggul salibMu dengan mantap tidak ragu-ragu, tidak mengeluh. Kuatkanlah kami ya Yesus agar kami kuat dan tabah memanggul salib kami masing-masing. Kami mohon:
			
			\item \textbf{Yesus wafat di salib}
			
			Yesus, dengan wafatMu kami telah Kau bebaskan dari mati. Kami mohon ya Tuhan agar dalam menghadapi kematian kami tetap penuh harapan untuk hidup kembali bersama Kristus. Kami mohon:
		\end{enumerate}
	\end{enumerate}
	\item Lagu
\end{itemize}

\section{Ibadat Penutup}
\begin{itemize}
\item Doa Penutup dan mohon berkat Tuhan

	Marilah berdoa,

Allah Bapa sumber kehidupan, kami telah Engkau segarkan dengan sabdaMu dan pengalaman iman. KehadiranMu memberi keteguhan bagi hidup kami sehari-haru dalam berusaha memenuhi kehendakMu melalui pelayanan dan  pekerjaan kami. Semoga hambaMu \arwah yang telah menghadapMu kini berbahagia bersamaMu dan persatukanlah kami kelak dengannya serta para kudus di surga. Dengan perantaraan Kristus Tuhan kami, Amin.

Saudara-saudari yang terkasih, marilah kita mengakhiri ibadat kita untuk \arwah dan doa rosario, dengan mohon berkat Tuhan.

Tuhan beserta kita

sekarang dan selama-lamanya

Semoga kita sekalian yang hadir di sini, dilindungi dan dibimbing oleh berkat Allah yang Mahakuasa dalam nama Bapa, Putera, dan Roh Kudus

Amin.

\item Kolekte dan Lagu Penutup. 
\end{itemize}
\end{document}
