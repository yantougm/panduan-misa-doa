\documentclass[a5paper,headsepline,titlepage,12pt,nnormalheadings,DIVcalc]{scrbook}
\usepackage[a5paper,backref]{hyperref}
\usepackage[papersize={145.25mm,215mm},twoside,bindingoffset=0.5cm,hmargin={1cm,1cm},
				vmargin={2cm,2cm},footskip=1.1cm,driver=dvipdfm]{geometry}
\usepackage[utf8]{inputenc} 
\usepackage{calc}
\usepackage{setspace} 
\usepackage{fixltx2e} 
\usepackage{graphicx}
\usepackage{multicol} 
\usepackage[normalem]{ulem} 
\usepackage[bahasa]{babel} 
\usepackage{color}
\usepackage{hyperref} 
\usepackage{pstricks}
\usepackage{fancyhdr}
\usepackage{pst-node}

\setlength{\parindent}{0mm}
\makeatletter
\newcommand{\lagu}[1]{%
  {\parindent \z@ 
    \interlinepenalty\@M \slshape \mdseries \large \textit{#1}\par\nobreak \vskip 10\p@ }}
\newcommand{\keterangan}[1]{%
  {\parindent \z@ 
    \interlinepenalty\@M \slshape \mdseries \textit{#1}\par\nobreak \vskip 10\p@ }}
\makeatother

\renewcommand{\footrulewidth}{0.5pt}
\lhead[\fancyplain{}{\thepage}]%
      {\fancyplain{}{~}}
\rhead[\fancyplain{}{~}]%
      {\fancyplain{}{\thepage}}
\pagestyle{fancy}
\lfoot[\emph{Doa Lingkungan}]{}
\rfoot[]{\emph{27 Juli 2012}}
\cfoot{}
\usepackage{palatino}
\usepackage{enumerate}
\hyphenation{sa-u-da-ra-ku}
\hyphenation{ke-ri-ngat}
\hyphenation{je-ri-tan}
\hyphenation{hu-bung-an}
\hyphenation{me-nya-dari}
\hyphenation{Eng-kau}
\hyphenation{ke-sa-lah-an}
\hyphenation{ba-gai-ma-na}
\hyphenation{Tu-han}
\hyphenation{di-per-ca-ya-kan}
\hyphenation{men-ja-uh-kan}
\hyphenation{bu-kan-lah}
\hyphenation{per-sa-tu-kan-lah}
\hyphenation{ma-khluk}
\hyphenation{Sem-buh-kan-lah}
\hyphenation{ja-lan}
\hyphenation{mem-bu-tuh-kan}
\hyphenation{be-ri-kan-lah}
\hyphenation{me-ra-sa-kan}
\hyphenation{te-man-ilah}
\hyphenation{mem-bi-ngung-kan}
\hyphenation{di-ka-gum-i}
\hyphenation{ta-ngis-an-Mu}
\hyphenation{mi-lik-ilah}


\setlength{\parindent}{0mm}
\setlength{\parskip}{2mm}

\newcommand{\BU}[1]{\begin{itemize} \item[U:] #1 \end{itemize}}
\newcommand{\BI}[1]{\begin{itemize} \item[I:] #1 \end{itemize}}
\newcommand{\BIU}[1]{\begin{itemize} \item[I+U:] #1 \end{itemize}}
\newcommand{\BP}[1]{\begin{itemize} \item[P:] #1 \end{itemize}}
\newcommand{\inputlagu}[1]{\begin{textit} \input{#1} \end{textit}}

\begin{document}

\chapter*{Pesta \\St. Yoakim dan St. Anna \\ 27 Juli}

\section*{Pembuka}
\subsection*{Salam}
\subsection*{Tobat}
\subsection*{Doa pembuka}

\BP{Bapa kami yang maha kasih!
Kami sangat bersyukur kepadaMu
karena kami Kau kumpulkan di sini untuk lebih mengenal sabda dan karyaMu.
Karuniakanlah kepada kami Roh Kudus,
supaya kami kuat melawan keinginan - keinginan jahat
yang ada di dalam maupun di luar diri kami,
dengan tunduk dan taat kepadaMu.
Kuatkanlah kami mengalahkan segala kuasa kegelapan
yang dapat menghalangi pertumbuhan kami
sebagai murid-murid Yesus. Doa ini kami sampaikan
dengan pengantaraan Yesus Kristus, Putra-Mu, Tuhan kami, yang bersama dengan Dikau dalam persatuan Roh Kudus, hidup dan berkuasa, Allah, sepanjang segala masa. }
\BU{Amin.}

\section*{Ibadat Sabda}

\subsection*{Bacaan - Sirakh 44:10-15}

Tetapi yang berikut ini adalah orang kesayangan, yang kebajikannya tidak sampai terlupa;semuanya tetap tinggal pada keturunannya sebagai warisan baik yang berasal dari mereka.

Keturunannya tetap setia kepada perjanjian-perjanjian, dan anak-anak merekapun demikian pula keadaannya.
Keturunan mereka akan tetap tinggal untuk selama-lamanya, dan kemuliaannya tidak akan dihapus.
Dengan tenteram jenazah mereka ditanamkan, dan nama mereka hidup terus turun-temurun.
Bangsa-bangsa bercerita tentang kebijaksanaannya, dan pujian mereka diwartakan jemaah.


\subsection*{Injil - Mat 13:16-17}
\textit{"Berbahagialah matamu karena melihat dan telingamu karena mendengar"
}

"Tetapi berbahagialah matamu karena melihat dan telingamu karena mendengar. Sebab Aku berkata kepadamu: Sesungguhnya banyak nabi dan orang benar ingin melihat apa yang kamu lihat, tetapi tidak melihatnya, dan ingin mendengar apa yang kamu dengar, tetapi tidak mendengarnya." 

\subsubsection*{Renungan}

Santa Anna dan Santo Yoakim adalah orangtua kandung Santa Perawan Maria, Bunda Yesus, Putera Allah. Keduanya dikenal sebagai keturunan raja Daud yang setia menjalankan kewajiban agamanya serta dengan ikhlas mengasihi dan mengabdi Allah dan sesamanya. Oleh karena itu keduanya layak dihadapan Allah turut serta dalam karya keselamatan Allah. Santa Anna dihormati sebagai pelindung kaum ibu, khususnya yang sedang hamil dan sibuk mengurus keluarganya. 

Diceritakan bahwa sejak perkawinan antara Anna dan Yoakim, tak henti-hentinya Anna mengharapkan karunia Tuhan berupa seorang anak. Cukup lama ia menantikan tibanya karunia Allah itu. Anna sesekali menganggap keadaan dirinya yang tak dapat menghasilkan keturunan itu sebagai hukuman Allah atas dirinya, sebagaimana anggapan umum masyarakat Yahudi pada waktu itu. Karena itu diceritakan bahwa ia tak henti-hentinya tanpa putus asa berdoa kepada Allah agar kenyataan pahit itu ditarik Allah dari padanya. Setiap tahun, Anna dan Yohakim berziarah ke Bait Allah Yerusalem untuk berdoa. Ia berjanji, kalau Tuhan menganugerahkan anak kepadanya, maka anak itu akan dipersembahkan kembali kepada Tuhan.

Suatu hari malaikat Tuhan mengunjungi Anna yang sudah lanjut usia itu membawa warta gembira : ”Tuhan berkenan mendengarkan doa ibu! Ibu akan melahirkan seorang anak perempuan, yang akan membawa sukacita besar bagi seluruh dunia!.” Dengan kegembiraan dan kebahagiaan yang besar, Anna menceritakan warta malaikat Tuhan itu kepada Yoakim.

Setelah genap waktunya, lahirlah seorang anak wanita yang manis. Bayi ini diberi nama Maria, yang kelak akan mengandung putera Allah, Yesus Kristus, Juru Selamat Dunia. Bagi Anna, Maria lebih merupakan buah rahmat Allah daripada buah kodrat manusia. Kelahiran Maria menyemarakkan dan menyucikan kehidupan keluarganya

Orang-orang Yunani mendirikan sebuah basilik khusus di Konstantinopel pada tahun 550 untuk menghormati Santa Anna. Dikalangan Gereja Barat, Paus Gregorius XIII (1572-1585) menggalakkan penghormatan kepada Santa Anna di seluruh Gereja pada tahun 1584.

Nama Yoakim dan Anna sungguh sesuai dengan maksud pilihan Allah. Yoakim berarti ”Persiapan bagi Tuhan”, sedangkan Anna berarti “Rahmat atau Karunia”.


Berrefleksi atas bacaan-bacaan dalam rangka mengenangkan pesta St.Yoakim dan St.Anna, orangtua SP Maria, hari ini saya sampaikan catatan-catatan sederhana sebagai berikut:

Ada orangtua calon seminaris di Seminari Mertoyudan, ketika berwawancara dengan Tim Finansial untuk membicarakan sumbangan bagi anaknya yang diterima di Seminari Mertoyudan, begitu pelit dan alot untuk memberi sumbangan, dan memang kesediaan untuk memberi sumbangan akhirnya memang tidak sesuai dengan kemampuannya. 

Namun setelah beberapa bulan ketika seminaris memperoleh kesempatan berlibur ke rumah ada suatu perubahan yang mengesan. Orangtua sangat terkesan bahwa anaknya yang baru beberapa bulan di Seminari Mertoyudan telah berubah: rajin, siap sedia membantu orangtua untuk mencuci pakaiannya sendiri, membersihkan rumah dst.. Dan dengan rendah hati akhirnya orangtua tersebut datang ke Seminari Mertoyudan seraya minta maaf dan menyatakan diri akan memberi sumbangan lebih dari apa yang disanggupkan sebelumnya, bahkan secara nominal melebihi rata-rata beaya per seminaris per bulan. 

Benarlah bahwa "melihat dan mendengarkan" sungguh mempengaruhi cara hidup dan cara bertindak seseorang. Hari ini kita kenangkan St.Yoakim dan St Anna, orangtua SP Maria; kiranya sebagai orangtua sungguh bahagia ketika melihat dan mendengar anaknya terpilih untuk menjadi Bunda Penyelamat Dunia, dengan hamil karena Roh Kudus dan akan melahirkan Penyelamat Dunia yang dinantikan kedatanganNya oleh seluruh umat manusia. 

Pada hari pesta St.Yoakim dan St.Anna hari ini kami mengingatkan dan mengajak para orangtua untuk mawas diri perihal sikap terhadap anak-anaknya. Kebahagiaan sejati orangtua terhadap anak-anaknya hemat saya terletak ketika orangtua melihat dan mendengar bahwa anak-anaknya tumbuh berkembang sebagai pribadi yang cerdas beriman, dikasihi oleh Tuhan dan sesamanya. Maka sudah selayaknya para orangtua sungguh mendidik dan membina anak-anaknya sedemikian rupa sehingga tumbuh berkembang menjadi pribadi yang cerdas beriman, dan ketika ada anaknya yang terpanggil khusus untuk menjadi imam, bruder atau suster, hendaknya didukung dan difasilitasi, tidak dipersulit dan dihambat.

\begin{quote}
\textit{"Tetapi yang berikut ini adalah orang kesayangan, yang kebajikannya tidak sampai terlupa; semuanya tetap tinggal pada keturunannya sebagai warisan baik yang berasal dari mereka. Keturunannya tetap setia kepada perjanjian-perjanjian, dan anak-anak merekapun demikian pula keadaannya. Keturunan mereka akan tetap tinggal untuk selama-lamanya, dan kemuliaannya tidak akan dihapus. Dengan tenteram jenazah mereka ditanamkan, dan nama mereka hidup terus turun-temurun}" (Sir 44:10-14). 
\end{quote}

Para orangtua kiranya memiliki dambaan atau harapan bahwa kelak anak-cucunya maupun keturunannya senantiasa mengenangnya, seperti santo-santa di lingkungan Gereja Katolik atau para pahlawan bangsa yang namanya dikenang dengan digunakan sebagai nama jalan, bangunan maupun taman dst.. Jika anda mendambakan atau mengharapkan yang demikian itu kami harapkan anda mempersiapkan diri sebaik mungkin sejak sekarang, sedini mungkin. Salah satu usaha persiapan yang baik adalah orangtua senantiasa mengasihi anak-anaknya, mendidik dan membinanya sesuai dengan kehendak Allah. 

Kehendak Allah bagi umat manusia adalah manusia senantiasa dalam keadaan baik sebagaimana ketika mereka diciptakan, sebagai gambar atau citra Allah, sehingga senantiasa dalam keadaan selamat dan bahagia, terutama jiwa dan hatinya. Kebahagiaan sejati jiwa dan hati manusia kiranya terletak ketika yang bersangkutan hidup sesuai dengan panggilan Allah. Maka kami berharap kepada para orangtua agar mendidik dan membina anak-anaknya dalam dan dengan semangat cintakasih dan kebebasan sejati, sebagaimana anda berdua menjadi suami-isteri juga karena cintakasih dan kebebasan sejati. 

Setiap manusia juga diciptakan dalam dan dengan cintakasih dan kebebasan sejati, maka akan tumbuh berkembang dengan baik jika dididik dan dibina dalam dan dengan cintakasih dan kebebasan sejati. Mau jadi apakah anak nanti setelah dewasa, berilah kebebasan dan cintakasih untuk memilih dan menentukannya sesuai dengan kehendak Allah.

Mata dan telinga, melihat dan mendengarkan merupakan indera dan kegiatan yang sangat berpengaruh dalam pertumbuhan dan perkembangan pribadi seseorang. Apa yang dilihat dan didengarkan akan menentukan sikap mental atau kwalitas pribadi yang bersangkutan. Anak atau bayi yang masih di dalam rahim atau kandungan ibu sudah dapat mendengarkan, dan begitu ia dilahirkan maka indera mata langsung fungsional juga. Maka kami mengajak dan mengingatkan para orangtua atau pendidik/pembina, lebih-lebih bagi anak-anak kecil atau balita, hendaknya kepada mereka diperdengarkan dan diperlihatkan apa-apa yang baik, dan memang pada waktunya baru diperdengarkan dan diperlihatkan yang tidak atau kurang baik juga. Teruitama dan pertama-tama kami mengingatkan mereka yang memiliki anak-anak usia balita, hendak indera mata dan telinga sungguh menjadi perhatian. Aneka macam hiasan dinding atau gambar yang dipasang di dinding hendaknya apa yang bersifat mendidik dan membina anak-anak agar tumbuh berkembang menjadi pribadi yang berbudi pekerti luhur. Hiasan atau gambar seperti tokoh-tokoh agama dan bangsa, flora atau fauna, tanaman atau binatang, pemandangan alam, dst.. kiranya cukup baik untuk diperlihatkan kepada anak-anak alias yang menghiasi dinding-dinding ruangan atau kamar. Dalam hal pendengaran hendaknya aneka omongan atau wacana yang dapat didengarkan oleh anak-anak apa-apa yang baik. “Berbahagialah matamu karena melihat dan telingamu karena mendengar”, demikian sabda Yesus. Sabda ini mengajak dan memanggil kita semua untuk melihat dan mendengarkan karya Allah dalam seluruh ciptaanNya, dalam aneka bentuk pertumbuhan dan perkembangan ciptaanNya, dan tentu saja pertama-tama dan terutama dalam diri manusia, yang diciptakan sebagai gambar atau citra Allah.

· "Sesungguhnya Aku akan datang kepadamu dalam awan yang tebal, dengan maksud supaya dapat didengar oleh bangsa itu apabila Aku berbicara dengan engkau, dan juga supaya mereka senantiasa percaya kepadamu.”(Kel 19:9), demikian sabda Tuhan kepada Musa.. Awan yang tebal menjadi tanda kehadiran Tuhan di tengah umatNya. Kehadiran dan karya Tuhan di dunia ini atau dalam seluruh ciptaanNya juga dapat kita lihat dan dengarkan melalui berbagai tanda atau gejala yang ada dalam ciptaan-ciptaanNya. “Allah tinggal dalam ciptaan-ciptaanNya: dalam unsur-unsur, memberi ‘ada’nya; dalam tumbuh-tumbuhan, memberi daya tumbuh; dalam binatang-binatang, daya rasa; dalam manusia, memberi pikiran” (St,Ignatius Loyola, LR no 235). Mungkin baik pertama-tama dan terutama kita dengarkan buah-buah pikiran dari sesama manusia atau saudara-saudari kita, dengan kata lain marilah kita saling tukar pengalaman apa yang kita pikirkan alias bercurhat. Tentu saja masing-masing dari kita harus dijiwai sikap percaya, sehingga kita saling percaya dan dengan demikian setelah saling tukar buah pikiran kita semua semakin teguh dan handal dalam saling percaya satu sama lain. Saling percaya satu sama lain dan jauh dari aneka macam bentuk kecuirigaan hemat saya menjadi dambaan atau kerinduan kita semua, maka marilah kita bersama-sama mewujudkan, antara lain sering bertemu untuk tukar buah pikiran atau bercurhat, entah dengan pertemuan formal bersama, sedang berrekreasi, sedang makan bersama , dst… Dengan saling melihat dan mendengarkan kebersamaan atau persaudaraan kita semakin baik dan menarik atau memikat bagi siapapun yang melihat dan mendengarnya.

\begin{quote}
\textit{"TUHAN telah menyatakan sumpah setia kepada Daud, Ia tidak akan memungkirinya: "Seorang anak kandungmu akan Kududukkan di atas takhtamu; Sebab TUHAN telah memilih Sion, mengingininya menjadi tempat kedudukan-Nya: "Inilah tempat perhentian-Ku selama-lamanya, di sini Aku hendak diam, sebab Aku mengingininya.}" (Mzm 132:11.13-14)
\end{quote}

\subsection*{Doa Umat}

\BP{Sudilah melimpahkan berkat-Mu atas keluarga-keluarga di lingkungan St. Petrus dalam wujud :kesehatan, kegembiraan, Kerukunan dan cinta kasuh yang tulus ikhlas. Hiasilah rumah tangga ini dengan suasana Penuh kedamaian,seperti keluarga kudus di Nazaret, supaya semua yang tinggal disini Merasa damai dan tenteram .

Kami mohon:}

\BU{Kabulkanlah doa kami,ya Tuhan.}

\BP{Jauhkanlah semua keluarga di lingkungan ini dari salah paham, pertengkaran, perselisihan ataupun iri dan dengki Sesama manusia. Dampingilah keluarga-keluarga ini bila sekali waktu mengalami di fitnah, dan Jauhkanlah segala yang dapat mengguncangkan ketenteraman serta keutuhahan keluarga. 

Kami mohon:}

\BU{Kabulkanlah doa kami,ya Tuhan.}

\BP{Hiasilah keluarga kami dengan sikap lemah lembut serta ramah tamah kepada sesama,dan Hormat bakti kepada Allah, sehingga di dalam rumah kami orang dapat melihat pengalaman Ajaran-Mu yang utama, yakni:cinta kepada Allah dan kasih kepada sesama. 

Kami mohon}

\BU{Kabulkanlah doa kami,ya Tuhan.}

\BP{Anugrahilah kami bagian dari kebikjasanaan-Mu yang suci, agar kami dapat Mendidik anak-anak seturut ajaran-Mu supaya anak-anak pandai berbakti Kepada Allah,penuh hormat kepada orang tua,dan berguna bagi Negara serta sesama. 

Kami mohon:}

\BU{Kabulkanlah doa kami,ya Tuhan.}


\subsection*{Bapa Kami}

\section*{Penutup}
\subsection*{Doa penutup}
\BP{\textit{"Terpujilah Engkau, Tuhan, Allah nenek moyang kami, yang patut dihormati dan ditinggikan selama-lamanya. Terpujilah nama-Mu yang mulia dan kudus, yang patut dihormat dan ditinggikan selama-lamanya. Terpujilah Engkau dalam Bait-Mu yang mulia dan kudus, Engkau patut dinyanyikan dan dimuliakan selama-lamanya. Terpujilah Engkau di atas takhta kerajaan-Mu, Engkau patut dinyanyikan dan ditinggikan selama-lamanya. Terpujilah Engkau yang mendugai samudera raya dan bersemayam di atas kerub-kerub, Engkau patut dihormat dan ditinggikan selama-la-manya. Terpujilah Engkau di bentangan langit, Engkau patut dinyanyikan dan dimuliakan selama-lamanya” (Dan 3:52-56)
}

Ya Allah Bapa, terima kasih Engkau telah memimpin kami selama ibadah ini. 
Selanjutnya kami akan pulang ke rumah kami masing-masing, berkatilah kami di jalan dan sampai di rumah dengan membawa warta sukacita dariMu. Demi Kristus Tuhan dan pengantara kami.}


\BU{Amin}

\subsection*{Berkat}
\BP{Saudara sekalian, dengan ini Ibadat Lingkungan ( pertemuan doa Lingkungan ) sudah selesai.}

\BU{Syukur kepada Allah.}

\BP{Semoga Tuhan melindungi kita terhadap dosa, menghantar kita ke hidup yang kekal, dan memberkati kita dalam nama Bapa dan Putra dan Roh Kudus}
\BU{Amin} 
\end{document}