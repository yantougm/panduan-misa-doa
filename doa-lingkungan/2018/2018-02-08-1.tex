\documentclass[a4paper,12pt]{article}
\usepackage[latin1]{inputenc}
\usepackage[papersize={215mm,297mm},twoside,bindingoffset=0.25cm,hmargin={1.5cm,1.5cm},
vmargin={1.5cm,1.5cm},footskip=0.55cm,driver=dvipdfm]{geometry}
\usepackage[T1]{fontenc}
\usepackage{amsmath}
\usepackage{amssymb,amsfonts,textcomp}
\usepackage{array}
\usepackage{palatino}
\usepackage{microtype}
\usepackage{hhline}
\usepackage{setspace}
\title{Doa lingkungan} 
\date{08-02-2018}
\author{Yohanes S}

\newcommand{\BU}[1]{\begin{itemize} \item[U:] #1 \end{itemize}}
\newcommand{\BI}[1]{\begin{itemize} \item[I:] #1 \end{itemize}}
\newcommand{\BP}[1]{\begin{itemize} \item[P:] #1 \end{itemize}}
\newcommand{\BIP}[1]{\begin{itemize} \item[Bpk:] #1 \end{itemize}}
\newcommand{\BIW}[1]{\begin{itemize} \item[Ibu:] #1 \end{itemize}}
\newcommand{\BPU}[1]{\begin{itemize} \item[P+U:] #1 \end{itemize}}

\begin{document}
\maketitle
\onehalfspacing
\BP{Pembacaan dari 1 Raj 11:4-13
	
	Sebab pada waktu Salomo sudah tua, isteri-isterinya itu mencondongkan hatinya kepada allah-allah lain, sehingga ia tidak dengan sepenuh hati berpaut kepada TUHAN, Allahnya, seperti Daud, ayahnya.
	
	Demikianlah Salomo mengikuti Asytoret, dewi orang Sidon, dan mengikuti Milkom, dewa kejijikan sembahan orang Amon,
	
	dan Salomo melakukan apa yang jahat di mata TUHAN, dan ia tidak dengan sepenuh hati mengikuti TUHAN, seperti Daud, ayahnya.
	
	Pada waktu itu Salomo mendirikan bukit pengorbanan bagi Kamos, dewa kejijikan sembahan orang Moab, di gunung di sebelah timur Yerusalem dan bagi Molokh, dewa kejijikan sembahan bani Amon.
	
	Demikian juga dilakukannya bagi semua isterinya, orang-orang asing itu, yang mempersembahkan korban ukupan dan korban sembelihan kepada allah-allah mereka.
	
	Sebab itu TUHAN menunjukkan murka-Nya kepada Salomo, sebab hatinya telah menyimpang dari pada TUHAN, Allah Israel, yang telah dua kali menampakkan diri kepadanya,
	
	dan yang telah memerintahkan kepadanya dalam hal ini supaya jangan mengikuti allah-allah lain, akan tetapi ia tidak berpegang pada yang diperintahkan TUHAN.
	
	Lalu berfirmanlah TUHAN kepada Salomo: "Oleh karena begitu kelakuanmu, yakni engkau tidak berpegang pada perjanjian dan segala ketetapan-Ku yang telah Kuperintahkan kepadamu, maka sesungguhnya Aku akan mengoyakkan kerajaan itu dari padamu dan akan memberikannya kepada hambamu.
	
	Hanya, pada waktu hidupmu ini Aku belum mau melakukannya oleh karena Daud, ayahmu; dari tangan anakmulah Aku akan mengoyakkannya.
	
	Namun demikian, kerajaan itu tidak seluruhnya akan Kukoyakkan dari padanya, satu suku akan Kuberikan kepada anakmu oleh karena hamba-Ku Daud dan oleh karena Yerusalem yang telah Kupilih."
	
	Demikianlah sabda Tuhan
}

\BP{Semoga Tuhan beserta kita}

\BU{sekarang dan selama-lamanya}

\BP{Inilah Injil Yesus Kristus menurut Markus (Mrk 7:24-30)}

\BU{Dimuliakanlah Tuhan}
	
\BP{Lalu Yesus berangkat dari situ dan pergi ke daerah Tirus. Ia masuk ke sebuah rumah dan tidak mau bahwa ada orang yang mengetahuinya, tetapi kedatangan-Nya tidak dapat dirahasiakan.
	
	Malah seorang ibu, yang anaknya perempuan kerasukan roh jahat, segera mendengar tentang Dia, lalu datang dan tersungkur di depan kaki-Nya.
	
	Perempuan itu seorang Yunani bangsa Siro-Fenisia. Ia memohon kepada Yesus untuk mengusir setan itu dari anaknya.
	
	Lalu Yesus berkata kepadanya: "Biarlah anak-anak kenyang dahulu, sebab tidak patut mengambil roti yang disediakan bagi anak-anak dan melemparkannya kepada anjing."
	
	Tetapi perempuan itu menjawab: "Benar, Tuhan. Tetapi anjing yang di bawah meja juga makan remah-remah yang dijatuhkan anak-anak."
	
	Maka kata Yesus kepada perempuan itu: "Karena kata-katamu itu, pergilah sekarang sebab setan itu sudah keluar dari anakmu."
	
	Perempuan itu pulang ke rumahnya, lalu didapatinya anak itu berbaring di tempat tidur, sedang setan itu sudah keluar.
	

    Demikianlah Injil Tuhan}

\BU{Terpujilah Kristus}

\section{Renungan}

SETELAH berdiskusi dengan kaum Farisi dan ahli taurat, Yesus rupanya ingin mencari tempat nyaman untuk beristirahat. Dari diskusinya itu tampak jelas posisi Yesus berhadapan dengan tradisi dan peraturan agama Yahudi. Tetapi bagaimana sikap Yesus terhadap orang-orang non-Yahudi? Yesus pergi ke Tirus, sebuah daerah pinggiran tempat orang-orang bukan Yahudi. Di sana menjadi jelas pula posisi Yesus dalam berhadapan dengan orang-orang non Yahudi, yang dianggap rendah dan disebut kafir oleh orang-orang Yahudi.

Yesus pergi diam-diam ke daerah Tirus dan masuk di sebuah rumah. Walaupun demikian orang tetap mengetahui kedatanganNya. Rupanya Yesus sudah begitu terkenal sehingga orang-orang di wilayah bukan Yahudi pun mengetahuinya. Seorang wanita Yunani dari bangsa Siro-Fenisia mengetahui kedatangan Yesus, dan datang kepadaNya. Dia tersungkur di depan kaki Yesus dan memohon supaya anak perempuannya disembuhkan dari kerasukan roh jahat.  Tentu wanita ini amat ingin puterinya yang menderita kerasukan roh jahat itu disembuhkan. Dan dia menaruh pengharapan besar pada Yesus yang telah didengarnya banyak menyembuhkan berbagai penyakit, serta tidak pernah menolak permohonan orang. Jadi walaupun wanita ini tergolong kaum kafir menurut pandangan orang Yahudi, tetapi dia memiliki iman dan pengharapan yang besar pada Yesus. Cara dia meminta penyembuhan berupa sembah sujud kepada Dia, Sang penyembuh ilahi.

Jawaban Yesus barangkali mengherankan. Jawaban Yesus yang menolak permintaan wanita itu sepertinya sejalan dengan sikap kaum Yahudi pada umumnya yang menolak kaum kafir. Yesus berkata kepada wanita itu: ?Biarlah anak-anak kenyang dahulu sebab tidak patut mengambil roti yang disediakan bagi anak-anak dan melemparkannya kepada anjing?. Wanita itu tentu mengerti kata-kata Yesus. Dia faham siapa yang disebut anak-anak dan siapa yang disebut anjing. Orang Yahudi sudah biasa menyebut kaum kafir sebagai anjing-anjing liar, yang berbeda dari mereka sebagai kaum terpilih, anak-anak Allah sendiri.

Walaupun demikian wanita itu tidak merasa terhina, melainkan tetap kuat iman dan pengharapannya. Dia yakin Yesus akan mau menolongnya. Dia tidak mundur. Dia seakan-akan tahu isi hati Yesus. Karena itu dia menjawab Yesus ?Benar Tuan, tetapi anjing-anjing yang di bawah meja pun makan dari remah-remah yang dijatuhkan anak-anak.? Wanita itu membenarkan kata-kata Yesus, karena dia pun sudah tahu bahwa orang Yahudi sering menyamakan kaum kafir itu dengan anjing. Tetapi dia tidak merasa dirinya sebagai anjing liar yang jauh melainkan anjing rumah yang masih mendapat kasih sayang dari tuannya, walaupun tidak diperlakukan sama seperti anak-anak.

Saya yakin pada waktu Yesus mengucapkan kata-kata penolakan itu, pasti wajahnya menunjukkan penerimaan. Rupanya Yesus tetap ramah dan mengucapkan kata-kata tersebut sambil tersenyum. Yesus sebenarnya hanya menguji kesungguhan iman wanita itu. Yesus tentu tidak megucapkan kata-kata itu dari hati sebagaimana kaum Yahudi lainnya. Dan wanita tersebut membuktikan kesungguhan imannya dengan tidak mundur melainkan tetap mendesak Yesus supaya menyembuhkan anaknya. Apa lagi argumennya amat meyakinkan sekaligus menunjukkan kerendahan hati serta penuh pengharapan. Dan ketika mendengar jawaban wanita itu Yesus tidak dapat menolak lagi. Dia berkata: ?Karena kata-katamu itu, pergilah sekarang sebab setan itu sudah keluar dari anakmu?.

Walaupun demikian, apa yang dikatakan Yesus kepada wanita itu tetap mengandung kebenaran. Dia datang pertama-tama memang kepada orang Yahudi. Tetapi tidak hanya kepada mereka melainkan juga kepada bangsa-bangsa lain. Yesus datang untuk semua orang, walaupun melalui bangsa Yahudi.

Melalui jawaban wanita itu kita tidak hanya menemukan suatu kerendahan hati dan perjuangan yang tak kenal lelah demi kebaikan anak perempuannya, melainkan sebuah iman yang mengagumkan. Dia percaya bahwa remah-remah yang jatuh dari meja pun sudah cukup untuk menyelamatkan puterinya. Dia juga seperti orang-orang Genesaret yang amat percaya, bahwa hanya dengan menyentuh jubah Yesus mereka menjadi sembuh. Dia tahu isi hati Yesus, yang pasti mau menunjukkan belaskasihnya kepada orang dari bangsa manapun dan karena itu dia akan menyembuhkan puterinya.

Belaskasih Tuhan melampaui segalanya: melampaui ikatan kesukuan. Dia menerima setiap permohonan yang disampaikan dengan segala kerendahan hati serta pengharapan yang teguh. Bahkan belaskasih dan kerahiman Tuhan melampaui dosa dan ketidak-layakan kita.

Barangkali kita pernah merasa tidak dipedulikan oleh Tuhan dalam penderitaan atau kesulitan hidup kita. Ketika kita sungguh membutuhkan Dia, seolah-olah Dia jauh dari jangkauan kita.

Wanita Yunani berkebangsaan Siro-Fenesia memohon kepada Yesus untuk menunjukkan belas kasih kepada anak perempuannya yang kerasukan roh jahat.

Pada pandangan pertama Yesus tampaknya tidak mempedulikan permohonan ibu itu, malahan Dia melontarkan pertanyaan dengan memakai kata   melemparkan roti kepada anjing-anjing  . Ketika itu orang-orang Yahudi suka mengibaratkan orang-orang kafir sebagai   anjing-anjing yang kotor .

Hal tersebut dibuat Yesus bukan untuk meremehkan atau menghina atau mengabaikan sang ibu, yang adalah seorang Yunani dan bukan Yahudi, tetapi untuk menguji imannya.

  Benar, Tuhan! Tetapi anjing di bawah meja pun makan remah-remah yang dijatuhkan anak-anak.   Jawaban wanita ini mengagumkan Yesus dan ia dipuji karena imannya. Ketika pulang ke rumahnya ia mendapati anaknya sembuh, bebas dari kuasa setan. Seperti orang-orang lain pada umumnya, wanita ini pun merindukan keselamatan Allah.

Barang siapa yang dengan sepenuh hati dan iman mencari Yesus tidak akan pernah ditolak. Ketika Yesus menjadi bagian penuh dari kehidupan seseorang, hal ini bagaikan ungkapan Adam:   Inilah dia, tulang dari tulangku dan daging dari dagingku.   Artinya, dia adalah bagian penuh dari hidupku sendiri, dia dan aku tidak dapat dipisahkan.

Demikianlah, orang yang beriman tidak dapat terpisahkan dari Dia yang diimaninya. Dalam kesatuan dengan Yesus dan iman akan Dia, segalanya menjadi mungkin.

Tuhan Yesus Kristus, satukanlah hidupku dengan hidup-Mu, dan tambahkanlah senantiasa imanku akan kuasa-Mu yang menyembuhkan, serta bebaskanlah aku dari segala yang jahat. Amin.
	
\end{document}