% This file was converted to LaTeX by Writer2LaTeX ver. 1.4
% see http://writer2latex.sourceforge.net for more info
\documentclass{article}
\usepackage[ascii]{inputenc}
\usepackage[T1]{fontenc}
\usepackage{amsmath}
\usepackage{amssymb,amsfonts,textcomp}
\usepackage{array}
\usepackage{hhline}
\title{}
\author{}
\date{2018-02-08}
\begin{document}
TATA PERAYAAN IBADAT SABDA

Oleh:

Vinsensius Morita Bayudita Eliananda (Yudith)

1.~~~~~~Ritus pembuka

~ ~ ~Lagu pembuka

(dapat disesuaikan dengan situasi atau bertepatan dengan perayaan-perayaan pada tahun liturgi)

~Tanda Salib

~ ~ ~ ~ ~ ~ ~ ~P : Marilah kita tandai diri kita dengan tanda kemenangan Kristus.~

Dalam Nama Bapa, dan Putera dan Roh Kudus

~ ~ ~ ~ ~ ~ ~ ~U: Amin

~ ~ Salam Pembuka

P : Semoga Kasih karunia, dan damai sejahtera dari Allah Bapa dan dari Tuhan kita Yesus Kristus, selalu beserta kita.

~ ~ ~ ~ ~ ~ U : Sekarang dan selama-lamanya.

~ ~Kata Pengantar

~ ~ ~ ~ ~Dapat di buat secara spontan sesuai dengan tema serta~~dengan bacaan injil atau situasi ~ ~ ~ ~ ~ ~ ~ ~ ~
~setempat pada waktu perayaan sabda di mulai
{\dots}{\dots}{\dots}{\dots}{\dots}{\dots}{\dots}{\dots}{\dots}{\dots}{\dots}{\dots}{\dots}..

Ibu-Bapak serta saudara-saudari yang di kasihi Tuhan, puji dan syukur kita haturkan kepada Allah yang Maha Kasih karena
kasih-Nya telah menyatukan kita di tempat keluarga ini untuk kembali memuji dan memuliakan Dia didalam doa-doa kita.

Dengan kerendahan hati dan kesadaran diri, kita telah berkumpul di sini.

Semoga pikiran kita selalu dibuka oleh Allah untuk mengerti Sabda-Nya~~yang akan kita dengar dan kita renungkan bersama.

Biarkanlah diri kita dibimbing dan dikuasai oleh Roh Kudus sehingga ibadat kita ini berguna bagi kehidupan kita dan iman
kita kepada Tuhan semakin hari semakin kuat kepada-Nya.

Menyadari segala salah dan dosa kita, marilah dengan rendah hati kita memohon ampun kepada Tuhan, agar layak dan pantas
merayakan ibadat sabda ini.

( Marilah kita hening sejenak{\dots}..)

~ ~ ~Pernyataan tobat

P : Ibu-Bapak serta saudara-saudari yang di kasihi Tuhan

Setelah kita menyadari akan kehadiran Tuhan di tengah-tengah kita, marilah kita dari hati yang penuh syukur kepada
Tuhan, kita mengakui segala dosa kita dan memohon ampun kepada Bapa yang Maha Pengampun agar kita tetap menjadi
putera-puteri Bapa yang berkenan di hati-Nya dan layak mengikuti ibadat ini.

(Marilah kita hening sejenak{\dots}{\dots}{\dots}.)

P : Saya mengaku kepada Allah Yang Maha Kuasa{\dots}..

P : Semoga Allah Yang Maha Kuasa mengasihani kita, mengampuni dosa-dosa kita, dan menghantar Kita kehidupan yang kekal.

U : Amin.

~ ~Tuhan Kasihanilah Kami

Cara A:

P : Tuhan kasihanilah kami.

U : Tuhan Kasihanilah kami.

P : Kristus kasihanilah Kami.

U : Kristus Kasihanilah Kami.

P : Tuhan Kasihanilah Kami.

U : Tuhan Kasihanilah Kami.

~Atau Cara B

P : Tuhan Yesus Kristus, kami mudah tersinggung dan sukar mengampuni, kurang sabar dan cepat marah.

Tuhan kasihanilah kami

U :~Tuhan kasihanilah kami

P : Di dalam pergaulan dengan sesama kami sering terkurung dalam kesempitan cinta diri, dan jarang mau mencintai mereka
dengan sungguh.

Kristus, kasihanilah kami

U~ :~Kristus, kasihanilah kami

P : Kami mudah sekali mengadili orang lain dan lekas iri hati kepada sesama

Tuhan, kasihanilah kami

U :~Tuhan, kasihanilah kami

Atau Cara C

P : Tuhan Yesus Kristus, Engkau diutus untuk menyembuhkan orang yang remuk redam hatinya

Tuhan, kasihanilah kami

U~ :~Tuhan, kasihanilah kami

P : Engkau datang memanggil orang yang berdosa

Kristus, kasihanilah kami

U~ :~Kristus, kasihanilah kami

P : Engkau duduk di sisi Bapa sebagai pengantara kami

Tuhan, kasihanilah kami

U :~Tuhan, kasihanilah kami

(Atau dinyanyikan dari Madah Bakti No. 177-187)

~ ~Gloria (Madah Pujian)

P~~~~ : Ibu-Bapak serta saudara-saudari yang dikasihi Tuhan, marilah kita memuji dan memuliakan Tuhan Allah kita dengan
madah Pujian

P-U : Kemuliaan kepada Allah di surga{\dots}{\dots}{\dots}{\dots}{\dots}.

[FC?]${\blacksquare}{\blacksquare}$Kalau diucapkan dari Madah Bakti No. 105

[FC?]${\blacksquare}{\blacksquare}$Kalau dinyanyikan dari Madah Bakti No. 192-202

~ ~ Doa Pembuka

~~~ ~ ~ ~ ~ (di bawakan oleh pemimpin ibadat dan disesuaikan dengan ujud doa)

~ ~ ~ ~ ~ Contoh Doa Cara A:

~ ~ ~ ~ ~ P~~~ : Marilah kita bersatu dalam doa

Allah Bapa yang penuh kasih dan cinta. Pada hari ini, kami berkumpul di hadapan-Mu untuk memuji dan memuliakan Dikau
serta mengenangkan ajaran dan teladan Putera-Mu Yesus Kristus Tuhan kami. Dengan rendah hati, kami mohon curahkanlah
belaskasihan-Mu kepada kami semua supaya kami yang rapuh dan lemah ini, jangan binasa dalam dunia yang fana ini. Demi
Yesus Kristus Putera-Mu, pengantara yang hidup dan berkuasa kini dan sepanjang masa

U~~ : Amin

Contoh Doa Cara B:

P~~~ : Marilah kita berdoa

Ya Bapa yang Maha Baik kami bersyukur kepada-Mu atas segala kebaikan-Mu yang boleh kami terima hingga saat ini. Hadirlah
di tengah-tengah kami dan bukalah pintu hati dan pikiran kami agar kami semua dapa mendengarkan Sabda-Mu dengan baik.
Meresapkannya dalam hati dan mampu melaksanakan perintah-Mu serta bisa membuka diri hingga dapat merasakan kebaikan dan
peran serta-Mu dalam hidup ini. Demi Kristus Tuhan dan pengantara kami

U~~ : Amin

Contoh Doa Cara C:

P~~~ : Marilah kita hening, untuk masuk dalam doa kita

Allah Bapa Yang Pengasih dan Penyayang, Engkau membangkitkan dalam hati kami kerinduan akan kedatangan Putera-Mu.
Berilah kami roh hikmat dan pengertian, roh nasihat dan keperkasaan, roh pengenalan dan takwa kepada-Mu, agar hati kami
sungguh pantas menyongsong Kristus, Raja yang akan dating. Sebab Dialah Tuhan pengantara kami, yang bersama Dikau hidup
dan berkuasa kini dan sepanjang segala masa.

U~~ : Amin

2.~Liturgi Sabda

P~~~ : Marilah kita mendengarkan dalam hati kita bacaa-bacaan suci pada hari ini

[56?]${\blacksquare}{\blacksquare}$Bacaan I

(disesuaikan dengan kalender Liturgi atau tema dalam perayaan Sabda)

[56?]${\blacksquare}{\blacksquare}$Lagu antar bacaan

(diambil dari Madah Bakti No. 208-223)

[56?]${\blacksquare}{\blacksquare}$Aklamasi Injil

P : semoga Tuhan beserta kita

U : sekarang dan selama-lamanya

[56?]${\blacksquare}{\blacksquare}$Bacaan Injil

(disesuaikan dengan kalender Liturgi atau tema dalam perayaan Sabda)

P~ : Inilah Injil Yesus Kristus menurut Matius (contoh)

U~ : Dimuliakanlah Tuhan

(Kemudian Injil dibacakan oleh pemimpin dan disambung dengan aklamasi sesudah Injil)

P~ : Demikianlah Injil Tuhan

U~ : Terpujilah Kristus

[56?]${\blacksquare}{\blacksquare}$Khotbah atau Homili atau Renungan

(Pemimpin atau yang mewakili dalam memberikan renungan dapat disesuaikan dengan tema atau sesuai dengan bacaan Injil
yang dibacakan pada waktu itu)

Jikalau tidak memakai renungan, maka pemimpin cukup mengucapkan:

~P~ :~Ibu-Bapak serta saudara-saudari yang terkasih, setelah kita mendengarkan bacaan Injil pada ~ ~ ~ ~hari ini, maka
marilah bersama-sama meresapkannya di dalam hati kita, agar sabda Tuhan hari ~ ~ ~ ~ini sungguh hidup dalam diri kita

U~ :~Amin

[56?]${\blacksquare}{\blacksquare}$Credo atau Aku percaya atau Syahadat Singkat atau Pengakuan Iman

(jikalau diucapkan dari Madah Bakti No. 114 dan jikalau dinyanyikan dari Madah Bakti No. 224-227)

P~ :~Ibu-Bapak serta saudara-saudari yang terkasih.Marilah kita mengungkapkan iman dan ~ ~ ~ ~ ~ ~ ~ ~ ~ ~kepercayaan
kita kepada Allah Bapa di Surga

P~ : Aku percaya akan Allah,

U~ :~Bapa yang mahakuasa{\dots}{\dots}{\dots}{\dots}{\dots}..

[56?]${\blacksquare}{\blacksquare}$Doa Umat

(Alangkah baiknya jika doa dapat disampaikan secara spontan, agar umat dapat ambil bagian dalam peribadatan)

P~ :~Ibu-Bapak serta saudara-saudari yang dikasihi oleh Tuhan. Allah itu Maha baik dan Maha Pemberi. Ia dengan setia dan
sabar menanti kita untuk dapat berbicara dengan Dia. Datanglah kepada-Nya maka kita akan memperoleh kebahagiaan.

Untuk itu marilah kita sekarang memanjatkan doa-doa permohonan serta ungkapan hati kita di hadapan Tuhan.

1.~~~Bagi Gereja.

Ya Bapa yang pengasih berkatilah Gereja-Mu yang sedang mengembara di dunia ini. Jauhkanlah Gereja-Mu dari perpecahan dan
kehancuran sehingga Gereja-Mu dapat membawa sukacita Kristus kepada semua orang.

Marilah kita mohon{\dots}.

U~~~~~~~~ :~Kabulkanlah doa kami Ya Tuhan.

2. Bagi para Pemimpin Negara kami.

Bapa yang kekal dan kuasa, berkatilah dan lindungilah serta pimpinlah para pemimpin Negara kami, agar mereka dapat
memimpin Negara kami dengan kebijaksanaan dan keadilan-Mu sehingga tercipta kehidupan yang damai tentram serta
kesejahteraan bagi~~seluruh umat manusia.

Marilah kita mohon{\dots}..

U~~~~~~~~ :~Kabulkanlah doa kami Ya Tuhan.

3. Bagi mereka yang sedang sakit, terkena luka batin dan yang sedang menderita.

Bapa yang pengasih dan penyayang, kami mohon kepada-Mu berilah keteguhan dan penghiburan bagi mereka yang sedang sakit,
terkena luka batin dan yang sedang menderita baik jiwa maupun raganya. Semoga mereka dapat memperoleh jamahan tangan
kasih-Mu, sehingga mereka semua dapat sembuh dan kembali untuk dapat memuliakan nama-Mu dalam perjalanan hidup mereka.
Dan juga melalui penghiburan-Mu mereka dapat menghayati penderitaannya sebagai keikutsertaan dalam penderitaan-Mu,
bukan sebagai hukuman.

Marilah kita mohon{\dots}..

U~~~~~~~~ :~Kabulkanlah doa kami Ya Tuhan.

4. Bagi umat manusia yang kini masih belum mengenal Dikau.

Ya Bapa, kami berdoa bagi umat manusia yang sampai pada hari ini, masih belum mengenal akan Dikau. Semoga melalui belas
kasihan-Mu, bukakan mata, hati, dan pikiran mereka untuk dapat melihat bahwa Engkaulah Allah yang hidup, pemilik alam
semesta ini, yang mampu memberikan kehidupan dan kebahagiaan bagi semua orang yang percaya pada-Mu. Sehingga mereka pun
dapat bersama-sama dengan kami untuk memuji dan memuliakan nama-Mu sepanjang kehidupan kami.

Marilah kita mohon{\dots}..

U~~~~~~~~ :~Kabulkanlah doa kami Ya Tuhan.

~~~~~~~~~~~~~~~~~~~~~~~~~~~~~~~~~~~~~~~~~~~~~~~

5. Bagi kami yang berkumpul di tempat ini.

Ya Bapa Yang Maha Baik, berikanlah kami kesehatan jiwa dan raga yang baik agar kami boleh memuji dan memuliakan Dikau di
dalam kegembiraan dan kebahagiaan hati kami hari ini dan selalu. Kuatkanlah iman kami dalam menghadapi situasi
pengalaman pahit dan kenyataan hidup yang menggoncangkan iman kami.

Marilah kia mohon{\dots}.

U~~~~~~~~ :~Kabulkanlah doa kami Ya Tuhan.

P~ :~Ya Bapa Yang Maha Baik, Engkau berkendak agar semua orang menjadi selamat dan tidak ada yang binasa maka kami mohon
kabulkanlah doa umat-Mu ini, moga-moga karena pimpinan-Mu dunia hidup dalam perdamaian dan keadilan serta kiranya
umat-Mu memuliakan Dikau dalam ketenangan dan kebahagiaan. Demi Kristus, Tuhan dan pengantara kami yang hidup dan
berkuasa, kini dan sepanjang segala masa.

U~ : Amin

[56?]${\blacksquare}{\blacksquare}$Persembahan

a.~~~~~~Lagu Persembahan

(diselingi dengan lagu persembahan diambil dari MB No.228-247)

b.~~~~~~Doa Persembahan

P~ :~Ya Bapa yang Maha Baik. Puji dan syukur kami haturkan kepada-Mu untuk semua cinta dan kebaikan yang kami terima
dari pada-Mu. Kini kami persembahkan kepada-Mu seluruh kehidupan kami, seluruh suka dan duka kami. Terimalah
persembahan kami ini dan jadikanlah kami semua yang hadir di sini hanya menurut kehendak-Mu hari ini dan selalu. Demi
Kristus Tuhan dan pengantara kami.

U~~ :~Amin

[56?]${\blacksquare}{\blacksquare}$Bapa Kami

P~~~~~~~~~ :~Ibu-Bapak serta saudara-saudari yang terkasih, kita telah menerima Roh Kudus yang menjadikan kita
Putera-Puteri Allah. Maka dengan kuasa Roh Kudus itu kita berani berdoa seperti yang diajarkan oleh Yesus kepada kita.

P-U~~~~~ :.Bapa kami yang ada di Surga{\dots}{\dots}{\dots}

(dapat dinyanyikan dari Madah Bakti No. 142-144 atau 632)

~~

3.~~~Ritus penutup

[56?]${\blacksquare}{\blacksquare}$Doa penutup

Contoh Doa Cara A:

P~ : Marilah kita akhiri ibadat kita ini dengan berdoa

~ ~~Ya Bapa yang Maha Kuasa, sinarilah hati kami dengan buah-buah Roh Kudus-Mu agar semakin hari ~ ~ ~ ~ ~ ~ ~semakin
dewasa dalam beriman kepada-Mu. Jauhkanlah kami dari semua kesulitan hidup serta ~ ~ ~ ~ ~ ~ ~ ~ ~ ~ ~ ~persatukanlah
kami dengan kurban Yesus Kristus Putera-Mu yang hidup dan berkuasa kini dan sepanjang ~ ~ ~ ~masa.

U~ : Amin

Contoh Doa Cara B:

P~ : Marilah kita tutup perayaan Sabda dengan doa penutup

~~~ ~Allah Bapa yang bertahta dalam Kerajaan Surga, terima kasih Engkau telah menyertai sepanjang ibadat ini. ~ ~ ~
Semoga Sabda-Mu tetap tinggal di antara kami dan sungguh menjadi pedoman hidup kami. Sehingga apa ~ ~ ~ ~ yang kami
lakukan selalu sesuai denan kehendak-Mu. Serta iman kami semakin tumbuh dan berkembang. ~ ~ ~ Demi Kristus yang hidup
bersatu dengan Roh Kudus kini dan sepanjang masa.

U~ : Amin

Contoh Doa Cara C:

P~ : Marilah kita bersatu di dalam doa

~ ~ ~ ~Allah Yang penuh cinta dan kasih, kami bersyukur karena cinta yang Engkau tunjukan kepada kami. ~ ~ ~ ~ ~ ~
Semoga kami hidup takwa sesuai dengan perintah dan larangan-Mu, sambil menyesal atas dosa dan ~ ~ ~ ~ ~ ~ ~ ~ kesalahan
kami. Semoga kami mengalami belas kasih-Mu dan memuji cinta-Mu yang begitu limpah. ~ ~ ~ ~ ~ ~ ~ ~Peliharalah kami,
sehingga kami hidup tanpa noda dosa, sampai kelak tibalah saatnya kami boleh menghadap ~ ~Putera-Mu. Sebab Dialah Tuhan
dan pengantara kami, yang hidup dan berkuasa kini dan sepanjang~~segala ~ ~ ~masa.

U~ : Amin

[56?]${\blacksquare}{\blacksquare}$Pengumuman

[56?]${\blacksquare}{\blacksquare}$Berkat

P :~~Marilah kita mohon berkat dari Tuhan

P :~~Semoga Tuhan beserta kita.

U :~~sekarang dan selama-lamanya

P : Semoga kita semua yang hadir saat ini, orang-orang yang kita doakan,sanak keluarga kita, ~ ~ ~ ~ ~ ~ ~ ~saudara
kita, aktivitas, pekerjaan, pendidikan serta perjalanan hidup~~kita selalu diberkati oleh ~ ~ ~ ~Allah Bapa Yang Maha
Kuasa~

P~ : Dalam Nama Bapa dan Putera dan Roh Kudus

U~ : Amin

[56?]${\blacksquare}{\blacksquare}$Pengutusan

~ ~ ~P :~~Ibu-Bapak serta saudara-saudari yang terkasih Ibadat sabda kita telah selesai.

U :~~Syukur kepada Allah.

P~ : Marilah kita berani mewartakan sabda Tuhan

U~ : Amin

Lagu penutup

(dapat diambil dari Madah Bakti dan disesuaikan dengan tema atau perayaan-perayaan yang bertepatan pada perayaan Sabda
pada hari itu)

\end{document}
