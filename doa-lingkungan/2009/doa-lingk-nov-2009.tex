\documentclass[a5paper,headsepline,titlepage,10pt,nnormalheadings,DIVcalc]{scrbook}
\usepackage[a5paper,backref]{hyperref}
\usepackage{fancyhdr}
\usepackage{graphicx}
\renewcommand{\footrulewidth}{0.5pt}
\lhead[\fancyplain{}{\thepage}]%
      {\fancyplain{}{\rightmark}}
\rhead[\fancyplain{}{\leftmark}]%
      {\fancyplain{}{\thepage}}
\pagestyle{fancy}
\lfoot[\emph{Lingk St Petrus }]{}
\rfoot[]{\emph{Maguwo}}
\cfoot{}
%\sloppy


\makeatletter
\newcommand{\judul}[1]{%
  {\parindent \z@ \centering 
    \interlinepenalty\@M \Large \bfseries #1\par\nobreak \vskip 20\p@ }}
\newcommand{\subjudul}[1]{%
  {\parindent \z@ 
    \interlinepenalty\@M \bfseries #1\par\nobreak \vskip 10\p@ }}
\newcommand{\lagu}[1]{%
  {\parindent \z@ 
    \interlinepenalty\@M \slshape \mdseries \Large \textsl{#1}\par\nobreak \vskip 20\p@ }}
\newcommand{\keterangan}[1]{%
  {\parindent \z@  \slshape \large
    \interlinepenalty\@M \textsl{#1}\par\nobreak  \vskip 5\p@}}

\renewenvironment{description}
               {\list{}{\labelwidth\z@ \itemindent-\leftmargin
                        \let\makelabel\descriptionlabel}}
               {\endlist}
\renewcommand*\descriptionlabel[1]{\hspace\labelsep 
                                \normalfont\bfseries #1 }


\makeatother

\newcommand{\BU}[1]{\begin{itemize} \item[U:] #1 \end{itemize}}
\newcommand{\BP}[1]{\begin{itemize} \item[P:] #1 \end{itemize}}

%lagu-lagu
\newcommand{\lagupembukaan}{~}
\newcommand{\laguantarbacaan}{~}
\newcommand{\lagupersembahan}{~}
\newcommand{\lagupenutup}{~}


\usepackage[bahasa]{babel}
\selectlanguage{bahasa}
\hyphenation{a-kan}
\hyphenation{ba-gi-mu}
\hyphenation{ber-a-da}
\hyphenation{ber-du-a}
\hyphenation{be-ri-kan}
\hyphenation{ber-ka-ta}
\hyphenation{ber-nya-nyi}
\hyphenation{ber-sa-ma}

\hyphenation{dah-syat}
\hyphenation{DA-RAH-KU}
\hyphenation{da-tang}
\hyphenation{di-ka-ta-kan}
\hyphenation{di-pim-pin}
\hyphenation{di-se-rah-kan}
\hyphenation{di-tum-pah-kan}

\hyphenation{Eng-kau}
\hyphenation{ha-dap-an}
\hyphenation{han-tar-kan-lah}
\hyphenation{ha-rap-an}

\hyphenation{ja-lan}
\hyphenation{ja-ngan-lah}

\hyphenation{ka-nak}
\hyphenation{ka-re-na}
\hyphenation{kau-lim-pah-kan}
\hyphenation{Kau-cip-ta-kan}
\hyphenation{ke-bang-kit-an-Nya}
\hyphenation{ke-da-tang-an}
\hyphenation{ke-da-tang-an-Nya}
\hyphenation{ke-dua}
\hyphenation{ke-na-ik-kan-nya}
\hyphenation{ke-pa-daMu}
\hyphenation{ke-ra-him-an}
\hyphenation{ke-se-jah-te-ra-an-mu}
\hyphenation{ko-men-tar}

\hyphenation{la-ma-nya}
\hyphenation{lim-pah-kan}

\hyphenation{ma-nu-sia}
\hyphenation{me-nga-da-kan}
\hyphenation{me-ngan-dung-lah}
\hyphenation{me-ngu-kuh-kan}
\hyphenation{me-la-lui}
\hyphenation{me-lim-pah-kan}
\hyphenation{me-lu-hur-kan}
\hyphenation{me-me-cah-me-cah-kan}
\hyphenation{mem-per-sem-bah-kan}
\hyphenation{me-nan-da-ta-ngan-i}
\hyphenation{men-cin-tai}
\hyphenation{meng-a-lir-kan}
\hyphenation{me-nga-sihi}
\hyphenation{me-nge-lu-ar-kan}
\hyphenation{meng-u-cap-kan}
\hyphenation{meng-ung-kap-kan}
\hyphenation{me-num-buh-kan}
\hyphenation{me-nya-ta-kan}
\hyphenation{me-nye-la-mat-kan}
\hyphenation{me-nye-rah-kan}
\hyphenation{me-nye-rah-kanNya}
\hyphenation{me-ra-ya-kan}

\hyphenation{o-rang}
\hyphenation{o-rang-o-rang}

\hyphenation{pa-sang-kan-lah}
\hyphenation{pa-tut}
\hyphenation{pe-ne-ri-ma-an}
\hyphenation{pe-ngam-pun-an}
\hyphenation{Pe-ngan-ta-ra}
\hyphenation{peng-hi-bur-an}
\hyphenation{per-bu-at-an-nya}
\hyphenation{per-ka-ta-an}
\hyphenation{per-ka-win-an}
\hyphenation{per-ni-kah-an}
\hyphenation{per-se-ku-tu-an}
\hyphenation{per-sem-bah-an}
\hyphenation{rom-bong-an}

\hyphenation{se-la-ma}
\hyphenation{se-ka-li-an}
\hyphenation{se-pan-jang}
\hyphenation{se-ra-ya}
\hyphenation{Su-dar-yan-to}

\hyphenation{te-ta-pi}
\hyphenation{ta-ngan-Mu}
\hyphenation{Tu-han}
\hyphenation{tu-lang}
\hyphenation{tu-lang-tu-lang}

\hyphenation{u-mat-Mu}
\hyphenation{wa-kil}

\hyphenation{ba-gi-mu}
\hyphenation{di-se-rah-kan}
\hyphenation{me-la-lui}
\hyphenation{ka-nak}
\hyphenation{ka-re-na}
\hyphenation{ber-ka-ta}
\hyphenation{te-ta-pi}
\hyphenation{per-ka-win-an}
\hyphenation{pa-tut}
\hyphenation{me-lu-hur-kan}
\hyphenation{ber-nya-nyi}
\hyphenation{di-tum-pah-kan}
\hyphenation{pe-ngam-pun-an}
\hyphenation{ber-a-da}
\hyphenation{kau-lim-pah-kan}
\hyphenation{ke-bang-kit-an-Nya}
\hyphenation{per-ka-ta-an}
\hyphenation{pa-sang-kan-lah}
\hyphenation{DA-RAH-KU}
\hyphenation{ke-na-ik-kan-nya}
\hyphenation{per-sem-bah-an}
\hyphenation{per-se-ku-tu-an}



\title{DOA LINGKUNGAN}
\author{
Santo Petrus Maguwo \\
oleh Y Suyanto} 
\date{5 November 2009}

\begin{document}
\sffamily
\maketitle
\thispagestyle{empty}

\lagu{Lagu pembukaan - \lagupembukaan}

\subjudul{Tanda Salib dan Salam}
\BP{Demi Nama Bapa dan Putera dan Roh Kudus}
\BU{Amin}

\subjudul{Kata Pengantar}
\BP{Bapak, Ibu, dan Saudara-saudara, pada kesempatan ini saya mengajukan tema 'hukum yang paling utama' yang berhubungan dengan masalah kasih, mengasihi Allah, dan mengasihi manusia. Kasih adalah esensi dari perintah Allah. Maka, perintah terbesar sepenuhnya adalah perkara kasih (1 Kor. 13:13), kasih terhadap Allah dan kasih terhadap manusia. Apakah bukti nyata kita mengasihi Allah?}

\subjudul{Pernyataan Tobat}
\BU{Bapak/Ibu/Saudara terkasih, marilah kita menyesali segala dosa-dosa kita dan
mohon ampun atas segala dosa-dosa kita.}
\BP{Saya mengaku \dots}
\BU{Kepada Allah yang Maha Kuasa dan kepada saudara sekalian bahwa saya telah
berdosa dengan pikiran dan perkataan, dengan perbuatan dan kelalaian. Saya
berdosa, saya berdosa, saya sungguh berdosa. Oleh sebab itu saya mohon kepada
Santa Perawan Maria, kepada para malaikat dan orang kudus, dan kepada saudara
sekalian supaya mendoakan saya kepada Allah Tuhan kita.}
\BP{Semoga Allah yang Maha Kuasa mengasihi kita, mengampuni dosa kita, dan
menghantar kita ke hidup yang kekal.}
\BU{Amin}

\lagu{Tuhan Kasihanilah Kami}

\subjudul{Doa Pembukaan}
\BP{Marilah kita berdoa

Allah Bapa yang mahamurah, Engkau berkenan mengampuni orang berdosa yang percaya kepadaMu. Engkau menyediakan api penyucian untuk membersihkan dan menyucikan umatMu yang masih sering jatuh ke dalam dosa saat masih mengembara di dunia. Semoga kami yang masih didunia ini Engkau bukakan hati kami untuk mendoakan mereka yang sudah Engkau panggil, agar penderitaan mereka di api penyucian sedikit berkurang. 
Demi Yesus Kristus, Putra-Mu, Tuhan dan pengantara kami, yang bersatu dengan Dikau dan Roh Kudus hidup dan berkuasa kini dan sepanjang masa.}

\BU{Amin.}

\judul{LITURGI SABDA}

\subjudul{Bacaan Kitab Suci}
\BP{Surat St Paulus kepada Umat di Roma 14:7-12	

Sebab tidak ada seorangpun di antara kita yang hidup untuk dirinya sendiri, dan tidak ada seorangpun yang mati untuk dirinya sendiri. Sebab jika kita hidup, kita hidup untuk Tuhan, dan jika kita mati, kita mati untuk Tuhan. Jadi baik hidup atau mati, kita adalah milik Tuhan.
Sebab untuk itulah Kristus telah mati dan hidup kembali, supaya Ia menjadi Tuhan, baik atas orang-orang mati, maupun atas orang-orang hidup.
Tetapi engkau, mengapakah engkau menghakimi saudaramu? Atau mengapakah engkau menghina saudaramu? Sebab kita semua harus menghadap takhta pengadilan Allah.

Karena ada tertulis: "Demi Aku hidup, demikianlah firman Tuhan, semua orang akan bertekuk lutut di hadapan-Ku dan semua orang akan memuliakan Allah."
Demikianlah setiap orang di antara kita akan memberi pertanggungan jawab tentang dirinya sendiri kepada Allah.}

\subjudul{Bacaan Injil}
\BP{Tuhan sertamu}
\BU{Dan sertamu juga}
\BP{Inilah Injil Suci Yesus Kristus menurut Lukas 15:1-10}
\BU{Terpujilah Kristus}

\BP{Para pemungut cukai dan orang-orang berdosa biasanya datang kepada Yesus untuk mendengarkan Dia.
Maka bersungut-sungutlah orang-orang Farisi dan ahli-ahli Taurat, katanya: "Ia menerima orang-orang berdosa dan makan bersama-sama dengan mereka."

Lalu Ia mengatakan perumpamaan ini kepada mereka:
"Siapakah di antara kamu yang mempunyai seratus ekor domba, dan jikalau ia kehilangan seekor di antaranya, tidak meninggalkan yang sembilan puluh sembilan ekor di padang gurun dan pergi mencari yang sesat itu sampai ia menemukannya?
Dan kalau ia telah menemukannya, ia meletakkannya di atas bahunya dengan gembira,
dan setibanya di rumah ia memanggil sahabat-sahabat dan tetangga-tetanggan serta berkata kepada mereka: Bersukacitalah bersama-sama dengan aku, sebab dombaku yang hilang itu telah kutemukan.

Aku berkata kepadamu: Demikian juga akan ada sukacita di sorga karena satu orang berdosa yang bertobat, lebih dari pada sukacita karena sembilan puluh sembilan orang benar yang tidak memerlukan pertobatan."
"Atau perempuan manakah yang mempunyai sepuluh dirham, dan jika ia kehilangan satu di antaranya, tidak menyalakan pelita dan menyapu rumah serta mencarinya dengan cermat sampai ia menemukannya?
Dan kalau ia telah menemukannya, ia memanggil sahabat-sahabat dan tetangga-tetangganya serta berkata: Bersukacitalah bersama-sama dengan aku, sebab dirhamku yang hilang itu telah kutemukan.

Aku berkata kepadamu: Demikian juga akan ada sukacita pada malaikat-malaikat Allah karena satu orang berdosa yang bertobat."

Demikianlah Injil Tuhan}

\BU{Terpujilah Kristus}

\subjudul{Renungan}
\section*{Tradisi}
Contoh tradisi-tradisi yang berkembang di beberapa tempat sehubungan dengan Peringatan Arwah Semua Orang Beriman. Di Meksiko, sanak-saudara merangkai karangan-karangan bunga dan dedaunan, juga membuat salib-salib dari bunga-bunga segar maupun bunga-bunga kertas beraneka warna guna diletakkan pada makam sanak-saudara yang telah meninggal, di pagi hari Peringatan Arwah Semua Orang Beriman. Keluarga akan menghabiskan sepanjang hari itu di pemakaman. Imam akan mengunjungi makam, menyampaikan khotbah dan mempersembahkan doa-doa bagi mereka yang meninggal, serta memberkati makam-makam satu per satu. Permen “Tengkorak” dibagikan kepada anak-anak.

Praktek serupa didapati pula di Louisiana. Sanak-saudara membersihkan serta melabur batu-batu nisan, mempersiapkan karangan-karangan bunga dan dedaunan, juga salib-salib dari bunga-bunga segar maupun bunga-bunga kertas untuk menghiasi makam. Pada siang hari Peringatan Arwah Semua Orang Beriman, imam berarak sekeliling makam, memberkati makam-makam dan mendaraskan rosario. Lilin-lilin dinyalakan dekat kubur pada senja hari; satu untuk setiap anggota keluarga yang telah meninggal dunia. Pada Peringatan Arwah Semua Orang Beriman, biasanya Misa dirayakan di pemakaman. Dua contoh praktek kebudayaan ini berpusat pada pentingnya mengenangkan mereka yang telah meninggal dunia serta mendoakan jiwa-jiwa mereka.

Tiap tanggal 2 November kita memperingati arwah semua orang beriman.
Bahkan Gereja mengkhususkan 1 bulan tersebut untuk kita berdoa dan berbuat silih untuk jiwa-jiwa yang masih berada di api penyucian.

Mengapa jiwa-jiwa orang yang meninggal itu masih harus 'mampir' di api penyucian?
Bagaimana kita berdoa dan berbuat silih untuk mereka? Kita percaya bahwa jiwa orang yang meninggal segera mendapat tiga alternatif 'tempat' yakni neraka, surga, dan api penyucian. Mereka yang sengaja menolak Allah niscaya segera menuju neraka. Mereka sama sekali tidak menginginkan hidup bahagia bersama Tuhan. Sikap ini tampak dari keseluruhan hidupnya; mereka sengaja menolak Tuhan.


Bagaimana dengan mereka yang segera menuju surga?
Dalam buku 'Rahasia Jiwa-jiwa di Api Penyucian' (Jakarta: Marian Centre Indonesia, 2003), Maria Simma, seorang mistik Austria yang mendapat karunia dikunjungi jiwa-jiwa di api penyucian selama kurun waktu 50 tahun menyebut tiga kunci emas untuk langsung masuk ke surga (hlm 27-28), yakni:

\begin{enumerate}
\item Kasih

Dalam perumpamaan tentang pengadilan terakhir (Mat 25:31-46) Tuhan Yesus tidak menyebut kriteria pengetahuan untuk masuk surga: apa agamamu, siapa nabimu dan sebagainya. Satu-satunya kriteria yang dipakai adalah apa yang kita lakukan terhadap mereka yang hina dan menderita sebab dalam diri mereka Tuhan sendiri hadir (ayat 40). Sementara menurut Paulus , di antara berbagai karunia yang kita terima, karunia kasihlah yang paling besar dan tidak berkesudahan (1 Kor 13:8.13). Tentang ciri-ciri kasih Paulus menyebutnya dalam 1 Kor 13:4-7. Bila kasih yang menjadi tolok ukur masuk surga, niscaya semua orang tanpa pandang agama juga berpeluang untuk masuk surga.

\item Kerendahan Hati

Kerendahan hati bukanlah perkara mudah untuk diupayakan mengingat akar dosa manusia adalah kesombongan. Hawa terpikat oleh bujuk rayu si iblis untuk menyamai Tuhan, berhak menentukan sendiri mana yang baik dan mana yang buruk (bdk. Kej 3:5). Artinya, manusia tidak mau tunduk pada kuasa Allah. Rendah hati berarti mengakui keterbatasan diri dan mau terbuka pada Allah, tidak mengandalkan kekuatan sendiri. Inilah yang juga disebut sebagai sikap miskin di hadirat Allah, yang dijanjikan sebagai empunya Kerajaan Surga (Mat 5:3).

\item Taat pada Kehendak Allah

Hidup seturut kehendak Allah sudah tentu merupakan salah satu syarat masuk surga. Tuhan Yesus sendiri menegaskan, 'Bukan setiap orang yang berseru kepadaKu: Tuhan, Tuhan! Akan masuk ke dalam Kerajaan Surga melainkan dia yang melakukan kehendak BapaKu yang di surga (Mat 7:21).
Mereka yang mendengarkan Sabda Allah dan melaksanakannya akan disebut sebagai saudara-saudari Yesus sendiri (Mat 12:46-50).
\end{enumerate}

Jiwa-jiwa yang berada di api penyucian (dari kata 'suci'; bukan pencucian) tentu tidak menolak Allah. Mereka justru rindu bersatu dengan Tuhan. Namun mereka sendiri menyadari ketidakpantasan mereka karena berbagai dosa. Karena itu di api penyucian mereka mengalami 'pemurnian' agar pantas menghadap hadirat Tuhan. Situasi jauh dari Tuhan inilah yang membuat jiwa-jiwa merasa tersiksa.

Namun mereka tidak bisa berbuat apa-apa. Dosa mereka memang sudah diampuni Tuhan. Tetapi mereka masih harus menanggung siksa dosanya. Doa dan silih kita bagi jiwa-jiwa di api penyucian bisa meringankan siksa dosa mereka.

\subsubsection*{Penderitaan sebagai silih}

Kita juga dapat mempersembahkan penderitaan kita sebagai silih bagi jiwa-jiwa di api penyucian.
Entah itu penderitaan sukarela seperti berpuasa, pengorbanan diri, dan sebagainya ataupun penderitaan yang tidak disengaja seperti penyakit, dukacita, kegagalan dan penolakan.

Semua penderitaan ini bisa kita jadikan silih asalkan kita menjalaninya dengan sabar dan menerimanya dengan rendah hati. Sebaliknya: bila kita terus berkeluh-kesah dan marah, tidak mungkin penderitaan itu kita jadikan silih. Sanggup tidak; kita menerima tiap penderitaan ini dengan lapang dada dan berjiwa besar?

Hal menarik lagi yang disharingkan oleh Maria Simma adalah ternyata jiwa orang yang meninggal tahu siapa saja yang hadir di hari penguburannya. Dia juga tahu apakah kita sungguh-sungguh berdoa baginya atau hadir sekadar 'absen' dan unjuk muka.

Jiwa-jiwa yang datang pada Maria Simma mengeluh bahwa orang-orang yang hadir di pemakaman tidak mengucapkan sepatah doa pun kepada Tuhan tetapi mengeluarkan banyak air mata. Padahal ini tiada bermanfaat (him 29). Mereka lebih membutuhkan doa doa kita. 

Kita di dunia dapat membantu jiwa-jiwa menderita di api penyucian dengan doa, amal, perbuatan-perbuatan baik, dan khususnya dengan Perayaan Misa. Tindakan-tindakan kita itu dapat membantu mengurangi masa tinggal mereka di api penyucian. Jika suatu jiwa telah dibersihkan sepenuhnya, jiwa tersebut akan segera menuju surga untuk menikmati kebahagiaan bersama Yesus, Bunda Maria, semua orang kudus dan para malaikat untuk selama-lamanya! Kita yakin bahwa mereka akan menjadi pendoa bagi kita kepada Tuhan. Kita menolong mereka dan mereka menolong kita. Semuanya ini adalah bagian dari menjadi Keluarga Allah: persekutuan para kudus.

Betapa baiknya Tuhan itu yang menjadikan kita semua bagian dari Keluarga yang Sungguh Luar biasa ini! Mari kita luangkan waktu setiap hari, terutama selama bulan November, untuk berdoa bagi mereka yang telah meninggal dunia. Meraka pantas mendapatkan cinta dan doa kita.

Semoga apa yang saya sampaikan ini dapat direnungkan lebih lanjut. Mungkin itu akan kita perdalam melalui pengalaman bagirasa selanjutnya didalam kesempatan doa arwah tahun depan.

Semoga kita senantiasa diberkati Tuhan. Amin.

\lagu{Kolekte diiringi lagu}

\subjudul{Doa Umat}

\BP{Allah Bapa yang Maha Baik pada peringatan hari arwah sedunia dimana Gereja Katolik mendoakan semua orang yang telah meninggal dunia, perkenankan kami mendoakan para kudus Allah, para martirMu yang jaya, para saksi iman, para pembela iman, anggota keluarga kami yang sudah Kau panggil, 
jiwa-jiwa yang menderita,
jiwa-jiwa korban bencana alam,
jiwa-jiwa korban perang,  
jiwa-jiwa yang terlupakan, 
jiwa-jiwa dibunuh karena ketidakadilan,
Semoga belas kasihMu tercurah dan kerahimanMu mengantar mereka ke tempat yang kekal.
Marilah kita mohon:}

\BU{Kabulkanlah doa kami ya Tuhan}

\BP{Tuhan,
Berilah mereka istirahat kekal dan sinarilah mereka dengan cahaya abadi. Semoga semua orang yang sudah meninggal beristirahat dalam damai.
 Marilah kita mohon:}

\BU{Kabulkanlah doa kami ya Tuhan}


\BP{Bapa,
Selamatkanlah saudara-saudara kami, kaum beriman, dan semua orang lain yang telah meninggal dunia. Berikanlah istirahat kekal kepada mereka dan kepada semua saudara yang meninggal dalam Kristus. Kasihanilah dan sambutlah mereka dalam pangkuan-Mu.
Marilah kita mohon:}

\BU{Kabulkanlah doa kami ya Tuhan}

\BP{Bapa yang Maha Kasih, jamahlah warga Santo Petrus yang saat ini menderita
sakit, kurangilah penderitaannya, sembuhkanlah mereka dari sakitnya, sehingga mereka
dapat aktif mengikuti kegiatan gereja bersama-sama kami. Marilah kita mohon:}

\BU{Kabulkanlah doa kami ya Tuhan}

\subjudul{Bapa Kami}
\BP{Atas petunjuk Penyelamat kita dan menurut ajaran Ilahi, maka beranilah
kita berdoa}

\BU{Bapa Kami yang ada di surga \dots}


\subjudul{Doa Penutup}
\BP{Marilah berdoa

Allah Bapa kami yang mahabaik, kami bersyukur kepadaMu, sebab Engkau sumber kehidupan, Engkaulah yang menganugerahkan keselamatan serta kebangkitan orang mati. Kami mohon semoga arwah-arwah saudara kami Kau bersihkan dari segala dosa dan hukuman sehingga layak masuk dalam KerajaanMu yang abadi, serta boleh bersama semua orang kudus memuliakan Dikau selama-lamanya di surga. 

Demi Kristus, Tuhan dan pengantara kami, yang hidup dan berkuasa kini dan sepanjang masa.}
\BU{Amin.}

\subjudul{Berkat Penutup}
\BP{Semoga kita sekalian diberkati oleh Allah yang mahakuasa, dalam nama Bapa dan
Putera dan Roh Kudus}
\BU{Amin.}

\lagu{Lagu Penutup - \lagupenutup}

\end{document}
