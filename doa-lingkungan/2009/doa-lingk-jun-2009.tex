\documentclass[a4paper,12pt]{scrartcl}
\usepackage{fancyhdr}
\usepackage[a4paper,top=2cm,bottom=2cm]{geometry}
\usepackage{graphicx}
\setlength{\voffset}{0.5in}
\setlength{\oddsidemargin}{28pt}
\setlength{\evensidemargin}{0pt}
\renewcommand{\footrulewidth}{0.5pt}
\lhead[\fancyplain{}{\Large \thepage}]%
      {\fancyplain{}{\rightmark}}
\rhead[\fancyplain{}{\leftmark}]%
      {\fancyplain{}{\Large \thepage}}
\pagestyle{fancy}
\lfoot[\emph{Lingk St Petrus }]{}
\rfoot[]{\emph{Maguwo}}
\cfoot{}
\topmargin=-0.5in
\textheight=9in
%\sloppy


\makeatletter
\newcommand{\judul}[1]{%
  {\parindent \z@ \centering 
    \interlinepenalty\@M \Large \bfseries #1\par\nobreak \vskip 20\p@ }}
\newcommand{\subjudul}[1]{%
  {\parindent \z@ 
    \interlinepenalty\@M \bfseries #1\par\nobreak \vskip 10\p@ }}
\newcommand{\lagu}[1]{%
  {\parindent \z@ 
    \interlinepenalty\@M \slshape \mdseries \Large \textsl{#1}\par\nobreak \vskip 20\p@ }}
\newcommand{\keterangan}[1]{%
  {\parindent \z@  \slshape \large
    \interlinepenalty\@M \textsl{#1}\par\nobreak  \vskip 5\p@}}

\renewenvironment{description}
               {\list{}{\labelwidth\z@ \itemindent-\leftmargin
                        \let\makelabel\descriptionlabel}}
               {\endlist}
\renewcommand*\descriptionlabel[1]{\hspace\labelsep 
                                \normalfont\bfseries #1 }


\makeatother

\newcommand{\BU}[1]{\begin{itemize} \item[U:] #1 \end{itemize}}
\newcommand{\BP}[1]{\begin{itemize} \item[P:] #1 \end{itemize}}

%lagu-lagu
\newcommand{\lagupembukaan}{~}
\newcommand{\laguantarbacaan}{~}
\newcommand{\lagupersembahan}{~}
\newcommand{\lagupenutup}{~}


\usepackage[bahasa]{babel}
\selectlanguage{bahasa}
\hyphenation{a-kan}
\hyphenation{ba-gi-mu}
\hyphenation{ber-a-da}
\hyphenation{ber-du-a}
\hyphenation{be-ri-kan}
\hyphenation{ber-ka-ta}
\hyphenation{ber-nya-nyi}
\hyphenation{ber-sa-ma}

\hyphenation{dah-syat}
\hyphenation{DA-RAH-KU}
\hyphenation{da-tang}
\hyphenation{di-ka-ta-kan}
\hyphenation{di-pim-pin}
\hyphenation{di-se-rah-kan}
\hyphenation{di-tum-pah-kan}

\hyphenation{Eng-kau}
\hyphenation{ha-dap-an}
\hyphenation{han-tar-kan-lah}
\hyphenation{ha-rap-an}

\hyphenation{ja-lan}
\hyphenation{ja-ngan-lah}

\hyphenation{ka-nak}
\hyphenation{ka-re-na}
\hyphenation{kau-lim-pah-kan}
\hyphenation{Kau-cip-ta-kan}
\hyphenation{ke-bang-kit-an-Nya}
\hyphenation{ke-da-tang-an}
\hyphenation{ke-da-tang-an-Nya}
\hyphenation{ke-dua}
\hyphenation{ke-na-ik-kan-nya}
\hyphenation{ke-pa-daMu}
\hyphenation{ke-ra-him-an}
\hyphenation{ke-se-jah-te-ra-an-mu}
\hyphenation{ko-men-tar}

\hyphenation{la-ma-nya}
\hyphenation{lim-pah-kan}

\hyphenation{ma-nu-sia}
\hyphenation{me-nga-da-kan}
\hyphenation{me-ngan-dung-lah}
\hyphenation{me-ngu-kuh-kan}
\hyphenation{me-la-lui}
\hyphenation{me-lim-pah-kan}
\hyphenation{me-lu-hur-kan}
\hyphenation{me-me-cah-me-cah-kan}
\hyphenation{mem-per-sem-bah-kan}
\hyphenation{me-nan-da-ta-ngan-i}
\hyphenation{men-cin-tai}
\hyphenation{meng-a-lir-kan}
\hyphenation{me-nga-sihi}
\hyphenation{me-nge-lu-ar-kan}
\hyphenation{meng-u-cap-kan}
\hyphenation{meng-ung-kap-kan}
\hyphenation{me-num-buh-kan}
\hyphenation{me-nya-ta-kan}
\hyphenation{me-nye-la-mat-kan}
\hyphenation{me-nye-rah-kan}
\hyphenation{me-nye-rah-kanNya}
\hyphenation{me-ra-ya-kan}

\hyphenation{o-rang}
\hyphenation{o-rang-o-rang}

\hyphenation{pa-sang-kan-lah}
\hyphenation{pa-tut}
\hyphenation{pe-ne-ri-ma-an}
\hyphenation{pe-ngam-pun-an}
\hyphenation{Pe-ngan-ta-ra}
\hyphenation{peng-hi-bur-an}
\hyphenation{per-bu-at-an-nya}
\hyphenation{per-ka-ta-an}
\hyphenation{per-ka-win-an}
\hyphenation{per-ni-kah-an}
\hyphenation{per-se-ku-tu-an}
\hyphenation{per-sem-bah-an}
\hyphenation{rom-bong-an}

\hyphenation{se-la-ma}
\hyphenation{se-ka-li-an}
\hyphenation{se-pan-jang}
\hyphenation{se-ra-ya}
\hyphenation{Su-dar-yan-to}

\hyphenation{te-ta-pi}
\hyphenation{ta-ngan-Mu}
\hyphenation{Tu-han}
\hyphenation{tu-lang}
\hyphenation{tu-lang-tu-lang}

\hyphenation{u-mat-Mu}
\hyphenation{wa-kil}

\hyphenation{ba-gi-mu}
\hyphenation{di-se-rah-kan}
\hyphenation{me-la-lui}
\hyphenation{ka-nak}
\hyphenation{ka-re-na}
\hyphenation{ber-ka-ta}
\hyphenation{te-ta-pi}
\hyphenation{per-ka-win-an}
\hyphenation{pa-tut}
\hyphenation{me-lu-hur-kan}
\hyphenation{ber-nya-nyi}
\hyphenation{di-tum-pah-kan}
\hyphenation{pe-ngam-pun-an}
\hyphenation{ber-a-da}
\hyphenation{kau-lim-pah-kan}
\hyphenation{ke-bang-kit-an-Nya}
\hyphenation{per-ka-ta-an}
\hyphenation{pa-sang-kan-lah}
\hyphenation{DA-RAH-KU}
\hyphenation{ke-na-ik-kan-nya}
\hyphenation{per-sem-bah-an}
\hyphenation{per-se-ku-tu-an}



\title{DOA LINGKUNGAN}
\author{
Santo Petrus Maguwo \\
oleh Y Suyanto} 
\date{4 Juni 2009}

\begin{document}
\sffamily
\maketitle
\thispagestyle{empty}

\lagu{Lagu pembukaan - \lagupembukaan}

\subjudul{Tanda Salib dan Salam}
\BP{Demi Nama Bapa dan Putera dan Roh Kudus}
\BU{Amin}

\subjudul{Kata Pengantar}
\BP{Bapak, Ibu dan Saudara-saudara, pada kesempatan ini saya mengajukan tema 'hukum yang paling utama' yang berhubungan dengan masalah kasih, mengasihi Allah, dan mengasihi manusia. Kasih adalah esensi dari perintah Allah. Maka, perintah terbesar sepenuhnya adalah perkara kasih (1 Kor. 13:13), kasih terhadap Allah dan kasih terhadap manusia. Apakah bukti nyata kita mengasihi Allah?}

\subjudul{Pernyataan Tobat}
\BU{Bapak/Ibu/Saudara terkasih, marilah kita menyesali segala dosa-dosa kita dan
mohon ampun atas segala dosa-dosa kita.}
\BP{Saya mengaku \dots}
\BU{Kepada Allah yang Maha Kuasa dan kepada saudara sekalian bahwa saya telah
berdosa dengan pikiran dan perkataan, dengan perbuatan dan kelalaian. Saya
berdosa, saya berdosa, saya sungguh berdosa. Oleh sebab itu saya mohon kepada
Santa Perawan Maria, kepada para malaikat dan orang kudus, dan kepada saudara
sekalian supaya mendoakan saya kepada Allah Tuhan kita.}
\BP{Semoga Allah yang Maha Kuasa mengasihi kita, mengampuni dosa kita, dan
menghantar kita ke hidup yang kekal.}
\BU{Amin}

\lagu{Tuhan Kasihanilah Kami}

\subjudul{Doa Pembukaan}
\BP{Marilah kita berdoa

Allah yang mahakudus, Engkau berkenan menguduskan umat-Mu di semua negara dan bangsa dengan mengutus Roh Kudus. Dialah yang membaharui segala sesuatu, dan menyatupadukan seribu satu bahasa dalam pengakuan iman yang satu dan sama. Tanamkanlah sabda suci-Mu dalam lubuk hati kami, dan teguhkanlah persatuan umat-Mu di lingkungan St Petrus ini, serta baharuilah semangat cinta kasih kami satu sama lain.
Demi Yesus Kristus, Putra-Mu, Tuhan dan pengantara kami, yang berstu dengan Dikau dan Roh Kudus hidup dan berkuasa kini dan sepanjang masa.}

\BU{Amin.}

\judul{LITURGI SABDA}

\subjudul{Bacaan Injil}
\BP{Tuhan sertamu}
\BU{Dan sertamu juga}
\BP{Inilah Injil Suci Yesus Kristus menurut Markus 12:28b-34}
\BU{Terpujilah Kristus}

\BP{
Lalu seorang ahli Taurat, yang mendengar Yesus dan orang-orang Saduki bersoal jawab dan tahu, bahwa Yesus memberi jawab yang tepat kepada orang-orang itu, datang kepada-Nya dan bertanya: "Hukum manakah yang paling utama?"

Jawab Yesus: "Hukum yang terutama ialah: Dengarlah, hai orang Israel, Tuhan Allah kita, Tuhan itu esa.

Kasihilah Tuhan, Allahmu, dengan segenap hatimu dan dengan segenap jiwamu dan dengan segenap akal budimu dan dengan segenap kekuatanmu.

Dan hukum yang kedua ialah: Kasihilah sesamamu manusia seperti dirimu sendiri. Tidak ada hukum lain yang lebih utama dari pada kedua hukum ini."

Lalu kata ahli Taurat itu kepada Yesus: "Tepat sekali, Guru, benar kata-Mu itu, bahwa Dia esa, dan bahwa tidak ada yang lain kecuali Dia.

Memang mengasihi Dia dengan segenap hati dan dengan segenap pengertian dan dengan segenap kekuatan, dan juga mengasihi sesama manusia seperti diri sendiri adalah jauh lebih utama dari pada semua korban bakaran dan korban sembelihan."

Yesus melihat, bagaimana bijaksananya jawab orang itu, dan Ia berkata kepadanya: "Engkau tidak jauh dari Kerajaan Allah!" Dan seorangpun tidak berani lagi menanyakan sesuatu kepada Yesus.
Demikianlah Injil Tuhan}

\BU{Terpujilah Kristus}

\subjudul{Renungan}
\section*{Hukum manakah yang paling utama?}
Ada begitu banyak aturan dan tatanan hidup atau hukum yang diberlakukan dalam hidup bersama: hidup bermasyarakat, berbangsa dan bernegara maupun beragama. Kiranya tidak ada orang yang hafal atas semua tatanan, aturan atau hukum tersebut.  Hemat saya semuanya itu dijiwai oleh hukum yang terutama sebagaimana disabdakan oleh Yesus: “Kasihilah Tuhan, Allahmu, dengan segenap hatimu dan dengan segenap jiwamu dan dengan segenap akal budimu dan dengan segenap kekuatanmu. Dan hukum yang kedua ialah: Kasihilah sesamamu manusia seperti dirimu sendiri”, maka marilah kita sikapi dan laksanakan aneka tatanan, aturan dan hukum yang terkait dengan hidup, panggilan dan tugas pengutusan kita dengan ‘hukum yang terutama’ tersebut. 

Mengasihi dengan ‘segenap hati, segenap jiwa, segenap akal budi dan segenap kekuatan’  kepada Tuhan dan sesama manusia itulah panggilan dan tugas pengutusan kita. “Dengan segenap”  berarti utuh dan tidak kurang sedikitpun; kalau kurang utuh berarti sakit, maka menjadi sakit hati/ pembenci/ pemarah, sakit jiwa/ sinthing/ gila, sakit akal budi/ bodoh dan sakit tubuh. Orang yang sedang menderita sakit jelas mengalami keterbatasan untuk mengasihi. Semua aturan, tatanan atau hukum jika disikapi dan dihayati dalam dan oleh kasih akan enak dan nikmat adanya. “Kasih itu sabar; kasih itu murah hati; ia tidak cemburu. Ia tidak memegahkan diri dan tidak sombong. Ia tidak melakukan yang tidak sopan dan tidak mencari keuntungan diri sendiri. Ia tidak pemarah dan tidak menyimpan kesalahan orang lain. Ia tidak bersukacita karena ketidakadilan, tetapi karena kebenaran. Ia menutupi segala sesuatu, percaya segala sesuatu, mengharapkan segala sesuatu, sabar menanggung segala sesuatu” (1Kor 13:4-7). Jika kita hidup dalam kasih sebagaimana disabdakan oleh Yesus dan diajarkan oleh Paulus kiranya ‘tidak ada seorangpun yang berani menanyakan sesuatu pada kita’, dan mereka akan mengikuti apa yang kita hayati.

(I Yohanes 3:18) Anak-anakku, marilah kita mengasihi bukan dengan perkataan atau dengan lidah,
tetapi dengan perbuatan dan dalam kebenaran.

\lagu{Kolekte diiringi lagu}

\subjudul{Doa Umat}

\BP{Bapa yang mahabaik, semoga para pemimpin Gereja dengan ketulusan hati dan
dipenuhi oleh Roh Kudus-Mu, mampu menjadi sarana pengejawantahan dan tanda
belaskasih-Mu yang merengkuh setiap orang dengan kasih yang sama. 
Marilah kita mohon:}

\BU{Kabulkanlah doa kami ya Tuhan}
\BP{Bapa penuh kasih, semoga para pemimpindan calon pemimpin negara senantiasa
mendengarkan tuntunan-Mu serta mampu menciptakan komunikasi yang hangat dengan warga
masyarakat sebagai bentuk perhatian bagi warganya. Dengan demikian, kebijakan yang
mereka buat, berlandaskan pada nilai-nilai kemanusiaan yang hakiki. 
 Marilah kita mohon:}

\BU{Kabulkanlah doa kami ya Tuhan}


\BP{Bapa yang Pemurah, bantulah kami menjadi pembangun jembatan dan bukannya
tembok pemisah. Lebih dari itu teguhkanlah iman kami sehingga takkan pernah menolak
PutraMu dan siapapun juga. Jadikanlah kami
bertanggungjawab atas setiap berkat, talenta dan kebaikanMu.
Marilah kita mohon:}

\BU{Kabulkanlah doa kami ya Tuhan}

\BP{Bapa yang penuh kemurahan, semoga anak-anak dan remaja di stasi Maguwo
khususnya lingkungan Santo Petrus semakin mampu dan mau terlibat dalam kehidupan
menggereja karena dukungan para orangtua, pengurus Dewan Stasi dan Lingkungan yang
memberi kesempatan luas bagi kiprah anak, remaja dan kaum muda.  Semoga kesemarakan
kegiatan Gereja dihiasi oleh kesadaran kaum muda untuk ambil bagian secara penuh
dalam mewujudkan iman.  
Marilah kita mohon:}

\BU{Kabulkanlah doa kami ya Tuhan}

\BP{Bapa yang Maha Kasih, jamahlah warga Santo Petrus yang saat ini menderita
sakit, kurangilah penderitaannya, sembuhkanlah mereka dari sakitnya, sehingga mereka
dapat aktif mengikuti kegiatan gereja bersama-sama kami. Marilah kita mohon:}

\BU{Kabulkanlah doa kami ya Tuhan}

\BP{Bapa yang Maha Pemurah, semoga kami yang berhimpun di sekitar meja perjamuan
ini, diteguhkan dalam kasih dan persaudaraan. Bantulah kami  membangun persaudaraan
dengan ia yang paling terkucil, terbuang, tersepelekan di rumah, di tempat kerja, di
lingkungan kami.
Marilah kita mohon:}

\BU{Kabulkanlah doa kami ya Tuhan}

\subjudul{Bapa Kami}
\BP{Atas petunjuk Penyelamat kita dan menurut ajaran Ilahi, maka beranilah
kita berdoa}

\BU{Bapa Kami yang ada di surga \dots}


\subjudul{Doa Penutup}
\BP{Marilah berdoa

Allah Bapa kami yang mahabaik, dengan turunnya Roh Kudus Kau tanamkan harapan, kasih, dan semangat baru dalam hati kami. Semoga dengan dipenuhi kasih ini sungguh menjadi pendorong kami semua dalam mengusahakan kebahagiaan bagi kami dan semua orang yang kami jumpai. Satukanlah kami dalam cinta kasih-Mu, dan doronglah kami membangun suatu masyarakat yang guyub rukun, sehati sejiwa, tempat tiada lagi yang merasa diasingkan. Demi Kristus, Tuhan kami, yang hidup dan berkuasa kini dan sepanjang masa.}
\BU{Amin.}

\subjudul{Berkat Penutup}
\BP{Semoga kita sekalian diberkati oleh Allah yang mahakuasa, dalam nama Bapa dan
Putera dan Roh Kudus}
\BU{Amin.}

\lagu{Lagu Penutup - \lagupenutup}

\end{document}
