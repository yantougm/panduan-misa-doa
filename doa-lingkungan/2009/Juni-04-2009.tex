\documentclass[12pt,a4paper,twoside]{article}
\usepackage[utf8]{inputenc}
\usepackage{ucs}
\usepackage{amsmath}
\usepackage{amsfonts}
\usepackage{amssymb}
\usepackage{setspace}
\author{Yohanes Suyanto}
\title{Hukum Manakah yang Paling Utama?}
\begin{document}
\onehalfspacing
\section*{Bacaan Injil: Markus 12:28b-34}
Lalu seorang ahli Taurat, yang mendengar Yesus dan orang-orang Saduki bersoal jawab dan tahu, bahwa Yesus memberi jawab yang tepat kepada orang-orang itu, datang kepada-Nya dan bertanya: "Hukum manakah yang paling utama?"

Jawab Yesus: "Hukum yang terutama ialah: Dengarlah, hai orang Israel, Tuhan Allah kita, Tuhan itu esa.

Kasihilah Tuhan, Allahmu, dengan segenap hatimu dan dengan segenap jiwamu dan dengan segenap akal budimu dan dengan segenap kekuatanmu.

Dan hukum yang kedua ialah: Kasihilah sesamamu manusia seperti dirimu sendiri. Tidak ada hukum lain yang lebih utama dari pada kedua hukum ini."

Lalu kata ahli Taurat itu kepada Yesus: "Tepat sekali, Guru, benar kata-Mu itu, bahwa Dia esa, dan bahwa tidak ada yang lain kecuali Dia.

Memang mengasihi Dia dengan segenap hati dan dengan segenap pengertian dan dengan segenap kekuatan, dan juga mengasihi sesama manusia seperti diri sendiri adalah jauh lebih utama dari pada semua korban bakaran dan korban sembelihan."

Yesus melihat, bagaimana bijaksananya jawab orang itu, dan Ia berkata kepadanya: "Engkau tidak jauh dari Kerajaan Allah!" Dan seorangpun tidak berani lagi menanyakan sesuatu kepada Yesus.

\section*{Hukum manakah yang paling utama?}
Ada begitu banyak aturan dan tatanan hidup atau hukum yang diberlakukan dalam hidup bersama: hidup bermasyarakat, berbangsa dan bernegara maupun beragama. Kiranya tidak ada orang yang hafal atas semua tatanan, aturan atau hukum tersebut.  Hemat saya semuanya itu dijiwai oleh hukum yang terutama sebagaimana disabdakan oleh Yesus: “Kasihilah Tuhan, Allahmu, dengan segenap hatimu dan dengan segenap jiwamu dan dengan segenap akal budimu dan dengan segenap kekuatanmu. Dan hukum yang kedua ialah: Kasihilah sesamamu manusia seperti dirimu sendiri”, maka marilah kita sikapi dan laksanakan aneka tatanan, aturan dan hukum yang terkait dengan hidup, panggilan dan tugas pengutusan kita dengan ‘hukum yang terutama’ tersebut. 

Mengasihi dengan ‘segenap hati, segenap jiwa, segenap akal budi dan segenap kekuatan’  kepada Tuhan dan sesama manusia itulah panggilan dan tugas pengutusan kita. “Dengan segenap”  berarti utuh dan tidak kurang sedikitpun; kalau kurang utuh berarti sakit, maka menjadi sakit hati/pembenci/pemarah, sakit jiwa/sinthing/gila, sakit akal budi/bodoh dan sakit tubuh. Orang yang sedang menderita sakit jelas mengalami keterbatasan untuk mengasihi. Semua aturan, tatanan atau hukum jika disikapi dan dihayati dalam dan oleh kasih akan enak dan nikmat adanya. “Kasih itu sabar; kasih itu murah hati; ia tidak cemburu. Ia tidak memegahkan diri dan tidak sombong. Ia tidak melakukan yang tidak sopan dan tidak mencari keuntungan diri sendiri. Ia tidak pemarah dan tidak menyimpan kesalahan orang lain. Ia tidak bersukacita karena ketidakadilan, tetapi karena kebenaran. Ia menutupi segala sesuatu, percaya segala sesuatu, mengharapkan segala sesuatu, sabar menanggung segala sesuatu” (1Kor 13:4-7). Jika kita hidup dalam kasih sebagaimana disabdakan oleh Yesus dan diajarkan oleh Paulus kiranya ‘tidak ada seorangpun yang berani menanyakan sesuatu pada kita’, dan mereka akan mengikuti apa yang kita hayati.

      Seperti Bapa telah mengasihi Aku, demikianlah juga Aku telah mengasihi
kamu; tinggallah di dalam kasihKu itu... Yesus mengundang kita untuk tinggal
di dalam kasihNya. Inilah undangan yang menyelamatkan kita. Siapakah kita
ini sehingga diundang untuk tinggal di dalam kasihNya? Apakah artinya bahwa
kita tinggal di dalam kasihNya? Tinggal di dalam kasihNya berarti kita
bersatu dengan Yesus mengasihi kita, Yesus yang lebih dahulu mengenal dan
mengasihi kita.

       Percaya atau tidak, bila kita tinggal di dalam kasihNya, hidup kita
akan terjamin oleh kasihNya. Ia tidak pernah menolak kita. Kasih dari
manusia bisa berubah menjadi kebencian, namun kasih Yesus tidak pernah
berubah, bahkan ketika kita tidak setia, Ia tetap setia dan menerima kita
apa adanya. Tinggal di dalam kasihNya, tidak akan pernah membuat kita sakit
hati, atau terluka. Sebab justru Ia sendiri rela terluka saat kita tinggal
di dalam kasihNya, justru karena kelemahan dan dosa kita.

CONTEMPLATIO:
Pandanglah Yesus...resapkanlah dalam hati yang jernih dan hening
kata-kataNya: Tinggallah dalam kasihKu... Betapa indah tinggal di dalam
kasihNya..., hidupmu aman, terjamin! Ia rela menderita demi keselamatanmu.
Bahkan hatiNya dirobek demi cinta untuk mengasihi orang berdosa...termasuk
dirimu. Peganglah hatiNya, rasakanlah kasihNya, hiduplah di dalam Dia.

ORATIO:
Tuhan Yesus Kristus, terima kasih Engkau berkenan mengundang aku untuk
tinggal dalam kasihMu. Amin.

MISSIO:
Hari ini aku akan tinggal dalam kasihNya dengan semakin banyak berdoa dan
merasakan kehadiranNya dalam hidup, pekerjaan dan aktivitasku.

\end{document}