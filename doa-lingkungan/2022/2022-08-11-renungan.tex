\documentclass[11pt,a5paper]{article}
\usepackage[T1]{fontenc}
\usepackage{amsmath}
\usepackage{amsfonts}
\usepackage{amssymb}
\usepackage{makeidx}
\usepackage{graphicx}
\usepackage[left=2.00cm, right=1.00cm, top=2.00cm, bottom=1.00cm]{geometry}
\author{Yohanes Suyanto}
\begin{document}
\section*{Arti Mengampuni 70 Kali 7 Kali yang Diajarkan oleh Tuhan Yesus
}

Apa arti mengampuni? Banyak orang yang seringkali jengkel terhadap orang lain yang berbuat salah. Kita seringkali jengkel jika teman kita berbuat kesalahan yang sama setiap harinya. Namun, pengajaran Tuhan Yesus selalu menitikberatkan pada pengampunan. 

Petrus pernah bertanya kepada Tuhan Yesus, ?Guru, berapa kalikah kami harus mengampuni saudara kami, yang bersalah kepada kami??, pada saat itu mungkin saja Petrus yang tidak pernah bisa menangkap ikan sering diledek atau direndahkan saudaranya. Tuhan Yesus menjawab; ?70 kali 7 kali?. Pernahkah kamu menghitung dengan teliti arti ucapan Yesus? Apa arti mengampuni 70 kali 7 kali? Nah, artikel kali ini akan membahas lebih lanjut mengenai pengertian mengampuni dalam agama Kristen.

Dikutip dari buku Konsep dan Upaya Pemberantasan Tindak Pidana Korupsi di Indonesia yang ditulis oleh F. H. Edy Nugroho (2019: 40), permintaan maaf menurut pandangan agama Kristen Protestan dan Katolik dipandang sebagai suatu cara efektif untuk memperbaiki suatu hubungan yang rusak atau terganggu. Dalam Injil Lukas pada Bab 6 ayat 37 disebutkan, ?Janganlah kamu menghakimi, maka kamu pun tidak akan dihakimi. Dan janganlah kamu menghukum, maka kamu pun tidak akan dihukum; ampunilah dan kamu akan diampuni.?

Injil Matius 18 ayat 21?22 menyebutkan sikap kita terhadap orang yang meminta maaf atas kesalahannya, sebagai berikut:
Kemudian datanglah Petrus dan berkata kepada Yesus: ?Tuhan, sampai berapa kali aku harus mengampuni saudaraku jika ia berbuat dosa terhadap aku? Sampai tujuh kali?? Yesus berkata kepadanya: ?Bukan sampai tujuh kali, melainkan sampai tujuh puluh kali tujuh.?

Berdasarkan Injil Matius tersebut dapat diketahui bahwa Yesus meminta agar dalam memberikan maaf kepada orang lain yang bersalah. Kita diwajibkan untuk memberikan maaf secara tulus, bahkan pemberian maaf tersebut adalah pemberian maaf yang diberikan tidak cukup hanya satu kali saja, tetapi diberikan berkali lipat dan dilakukan dengan sepenuh hati serta tanpa syarat.

Memang bagi kita manusia, ?mengampuni? adalah sesuatu yang tidak mudah, dan karenanya kita perlu memohon kekuatan dari Tuhan. Sebenarnya, Tuhan Yesus tidak mengajarkan bahwa orang yang bersalah kepada kita itu harus minta maaf terlebih dahulu baru kemudian ?layak? kita ampuni. Berikut ini adalah beberapa ayat Alkitab yang menunjukkan bahwa kita harus mengampuni tanpa syarat, seperti yang diajarkan oleh Tuhan:

\begin{enumerate}
	\item Dalam doa Bapa Kami, kita setiap kali berdoa, ?Ampunilah kesalahan kami seperti kamipun mengampuni yang bersalah kepada kami?? (Mat 6: 12). Di sini tidak dikatakan ?asalkan mereka minta maaf kepada kami?. Jadi sesungguhnya apapun yang terjadi, Tuhan menghendaki agar kita mengampuni orang yang bersalah pada kita- tanpa ada syarat apa-apa lagi.
	\item Pada khotbah-Nya di bukit, Yesus berkata, ?Kasihilah musuhmu dan berdoalah bagi mereka yang menganiaya kamu.? (Mt 5:44) Di sini tidak dikatakan apakah musuh itu harus minta maaf atau menyesal dahulu, baru kita ampuni/ kasihi. Makna ?kasihilah? di sini adalah sesuatu yang lebih dalam daripada mengampuni, karena mengampuni saja sudah sulit, apalagi mengasihi dan mendoakan mereka.
	\item Yesus memberikan sendiri contoh yang sempurna terhadap pengajaran-Nya ini dengan menyerahkan Diri-Nya di kayu salib. Pada saat Ia tergantung di salib, ketika tangan-Nya terentang antara langit dan bumi, Ia berkata, ?Ya Bapa, ampunilah mereka, sebab mereka tidak tahu apa yang mereka perbuat ? (Luk 23: 34). Dalam kesatuan-Nya dengan Allah Bapa, Yesus mengampuni mereka yang telah menyalibkan Dia, walaupun pada saat itu mereka tidak bertobat atau minta ampun pada Yesus.
	\item Rasul Paulus mengatakan, ?Akan tetapi Allah menunjukkan kasih-Nya kepada kita, oleh karena Kristus telah mati untuk kita ketika kita masih berdosa.? (Rom 5:8). Jadi Kristus memilih untuk wafat di salib untuk menebus dosa-dosa kita manusia, meskipun pada waktu itu manusia belum bertobat. Dan kasih Allah yang besar inilah yang sesungguhnya malah mengantar kita kepada pertobatan.
	\item Sebenarnya kita mengampuni bukanlah melulu demi orang yang bersalah kepada kita, seolah-olah jika kita mengampuni maka ?dia yang untung dan kita yang rugi?. Sebaliknya, jika kita mengampuni sesungguhnya itu adalah untuk kebaikan kita sendiri, karena dengan kita mengampuni, kita dibenarkan oleh Tuhan karena kita mengikuti teladan-Nya dan kita menjauhkan dari diri kita segala bentuk sakit penyakit badani dan rohani yang berkaitan dengan kekecewaan, kesesakan, kepahitan dan sakit hati yang terpendam. Kitab Mazmur mengatakan, ?Kasihanilah aku ya, Tuhan, sebab aku merasa sesak; karena sakit hati mengidaplah mataku, meranalah jiwa dan tubuhku. Sebab hidupku habis dalam duka dan tahun-tahun umurku dalam keluh kesah; kekuatanku merosot karena sengsaraku, dan tulang-tulangku menjadi lemah?? (Mz 31: 10-11). Tentulah karena Tuhan mengasihi kita, maka Ia ingin agar kita belajar mengampuni, agar kita tidak menyimpan sakit hati yang dapat mendatangkan hal-hal negatif terhadap diri kita sendiri, baik rohani maupun jasmani.
\end{enumerate}

Contoh yang paling indah saya rasa adalah bagaimana Bapa Paus Yohanes Paulus II yang mengampuni Mehmet Ali Agca, yang telah berusaha membunuhnya, dengan menembaknya pada tgl 13 mei 1981. Begitu Bapa Paus sembuh, beliau mengunjungi Ali di penjara, dan menyatakan bahwa beliau mengampuni Ali, walaupun setahu saya, tidak didahului oleh permintaan maaf dari Ali. Entah bagaimana jika kita yang ada di posisi Bapa Paus, sanggupkah kita mengampuni orang yang telah berusaha membunuh kita?

Memang, mengampuni bukan sesuatu yang mudah, namun itu adalah pengajaran Tuhan yang tak bisa ditawar. Maka kita semua memang harus berusaha untuk melakukannya, tentu dengan bantuan rahmat Tuhan. Jika kita dizinkan Tuhan untuk mengalami pengalaman disakiti oleh orang lain, maka kita diberi kesempatan oleh-Nya untuk merasakan sedikit dari penderitaan-Nya di kayu salib. Dan untuk itu, obat yang paling mujarab adalah: kita kembali mempersembahkan rasa sakit hati/ hati yang hancur kita di hadapan Tuhan (lih. Mzm 51:19), dan mempersatukannya dengan korban Yesus dalam Ekaristi Kudus, agar kita memperoleh buah-buahnya, yaitu dosa kita diampuni, sakit hati kita disembuhkan, dan kita diberi kekuatan oleh Tuhan untuk mengampuni, dengan kekuatan yang bukan berasal dari diri kita sendiri, tetapi dari Tuhan. Dengan pertolongan rahmat Tuhan, maka kita akan dapat mengampuni sesama yang bersalah pada kita, walaupun yang bersangkutan tidak minta maaf pada kita. Hal ini dapat terjadi sekaligus, ataupun merupakan perjuangan yang bertahap, namun kita harus terus mengusahakannya, sebab inilah yang dikehendaki oleh Tuhan bagi kita, ?Dengan demikian semua orang akan tahu, bahwa kamu adalah murid-murid-Ku, yaitu jikalau kamu saling mengasihi.? (Yoh 13:35).

\end{document}