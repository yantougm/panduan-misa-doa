Pada saat ini sangat sering terdengar adanya pertentangan antara beberapa orang saat ada ibadat lingkungan terkait bagaimana seharusnya ibadat dilakukan. Sangat disayangkan bahwa ibadat lingkungan yang merupakan sarana untuk sampai pada Tuhan justru jadi tempat untuk tempat berdebat. Melihat hal ini, bulletin membahas tentang Misa dan ibadat lingkungan menurut Actio Pastoralis, intruksi kongregasi Ibadat mengenai Misa untuk kelompok-kelompok Khusus. Tulisan ini dikutip dari tulisan Bosco Da Cunha,O.Carm wakil ketua Komisi Liturgi KWI dalam majalah Liturgi edisi Maret-April 2008.
Dalam Konstitusi Liturgi (KL) no 19, Gereja mengemukakan pemikiran yang merupakan dasar Misa Lingkungan, Misa Kategoial dan Ibadat Lingkungan. Pemikiran Gereja tersebut adalah:
Sasaran dari kegiatan pastoral Gereja adalah untuk meningkatkan keterlibatan para beriman dalam hidup menjemaat
Kelompok-kelompok khusus untuk melayani keperluan-keperluan khusus umat beriman sehingga penghayatan hidup Kristiani dapat meningkat sesuai keperluan dan kemampuan anggota kelompok yang bersangkutan. Hal ini untuk mengembangkan suatu ikatan rohani dan apostolis khusus dan semangat saling mendukung dalam perkembangan rohani.
Apabila diatur dan diarhakan secara bijak, perayaan samasekali tidak menghambat kesatuan di Paroki. Justru menunjang kegiatan missioner, akrena mampu menjangkau seluruh umat secara lebih personal, dan berdampak lebih terhadap pembinaan iman orang lain.

Misa Lingkungan
Terdapat beberapa kutipan peraturan, antara lain:
Perlunya suatu pertimbangan apakah mengadakan misa atau ibadat
Tempat Ekaristi adalah ruang ibadat, ruang keluarga, wisma, dan bukan ruang tidur, tetapi misa jangan dilakukan hanya di rumah beberapa orang saja karena alasan luas atau keindahan tempat
Misa tidak boleh selalu bersifat eksklusive suatu kelompok khusus, karena misa adalah perayaan Gereja, dan sakramen kesatuan.
Agar Ekaristi dapat serasi dengan kondisi umat maka:
1. umat hendaklah diajak untuk berperan serta
2. sesuai dengan kemampuan kelompok yang bersangkutan maka perayaan dapat didahului dengan Bimbingan Rohani atau renungan Kitab Suci
3. boleh dipilih bacaan yang sesuai, apabila situasi kongkrit menyarankan
4. Hendaklah umat aktif dalam doa umat
5. Komuni dalam dua rupa tidak diperkenankan; kecuali viaticum atau komuni bekal suci
6. memakai nyanyian liturgis, bukan nyanyian Rohani atau devosional dalam hal ini
Jangan sampai penyeseuaian-penyesuaian khusus yang diperbolehkan untuk kelompok khusus dimasukkan dalam Gereja Paroki.
Ekaristi di Rumah Warga tidak dapat dilaksanakan pada hari minggu dan pesta wajib, demi memupuk paguyuban umat di paroki, kecuali atas izin Uskup melalui Pastor paroki.
Para Gembala umat hendaknya memperhitungkan daya guna perayaan untuk pembinaan Rohani jemaat. Suatu perayaan akan sungguh bermanfaat apabila:
1. membawa peserta masuk dalam pemahaman misteri Kristiani secara lebih mendalam
2. penghayatan ibadat lebih hangat
3. umat lebih melibatkan diri dalam paguyuban jemaat paroki
4. giat dalam karya kerasulan dan social karitatif di tengah lingkungan dan paroki.

Perkembangan saat ini menunjukkan bahwa beberapa orang ada yang merasa dirinya “maju” karena menampilkan sesuatu yang baru yang justru kadang tidak berkualitas sebab bentuk perayaan liturgy justru dibuat seenaknya. Oleh karena itu perlu ditaati instruksi yang disusun oleh Kongregasi Ibadat atas pengarahan Paus Paulus VI, agar tercipta Liturgy yang mendalam demi perkembangan iman umat

Doa Lingkungan
Kita ketahui bahwa doa lingkungan yang sering diadakan bukanlah Ekaristi, maka susunan atau struktur dalam doa lingkungan lebih terbuka menurut situasi umat, peristiwa dan atau intensi keluarga yang bersangkutan. Dalam hal ini memang dianjurkan agar struktur Doa di lingkungan mirip dengan Liturgi sabda dalam perayaan Ekaristi.
Doa lingkungan merupakan sarana efektif nuk dapat meneladani sikap jemaat perdana yang sering berkumpul dan berdoa mendengarkan Sabda Tuhan, serta memupuk sika solider atas berbagai persoalan hidup serta dapat digunakan agar dapat mewujudnyatakan program yang ada di Paroki.
Dalam doa lingkungan dan juga kegiatan devosional, Sabda menjadi unsur pokok. Dari Sabda tersebut mengalirlah segala doa dan pujian. Doa Lingkungan dipimpin oleh awam baik laki-laki ataupun perempuan, jadi bukan imam namun kalau mereka hadir, maka pembacaan Injil serta renungan atau pengajaran diserahkan kepada mereka.
Rancangan doa lingkungan yang sering dimungkinkan untuk dipakai adalah sebagai berikut:
Sapaan awal: dari ketua lingkungan atau wakil
Lagu pembukaan: disiapkan oleh petugas nyanyian
Tanda Salib
Kata Pembuka : secara singkat disampaikan tentang bacaan-bacaan Kitab Suci yang akan didengarkan
Doa Pembukaan : oleh pemimpin doa
Bacaan : oleh Lektor. (bacaan dapat diambil dari bacaan Misa hari yang bersangkutan)
Lagu antar bacaan
Injil : dapat diambil dari bacaan Misa hari yang bersangkutan
Renungan
Mazmur-mazmur: dapat daimbil dari buku Puji Syukur
Doa permohonan atau doa bersama dari Puji Syukur
Doa Bapa Kami
Doa Penutup
Kata penutup: Saudara sekalian, dengan ini Ibadat Lingkungan (pertemuan doa lingkungan) sudah selesai.
Umat menjawab : Amin
Semoga Tuhan memberkati kita, melindungi kita terhadap dosa dan menghantar kita ke hidup yang kekal
Umat menjawab : Amin (sambil setiap orang membuat tanda salib)
Nyanyian Penutup

Beberapa keterangan praktis dalam doa lingkungan
Dalam usulan rancangan ini, praktis umat membawa Puji Syukur. Hanya pemimpin doa yang membawa buku tambahan untuk Doa pembuka dan Penutup, namun dalam doa lingkungan, spontan lebih baik dengan tujuan agar sesuai dengan situasi dan kepentingan.
Selama masa Advent dan Prapaskah dapat ditambahkan pernyataan tobat setelah kata pembuka dan setelah memohon ampun dan dilanjutkan dengan doa pembuka.
Selama Bulan Mei dan Oktober atau pada doa lingkungan untuk arwah dapat dilanjutkan dengan doa Rosario setelah renungan singkat.
Dalam doa lingkungan tugas-tugas hendaknya dibagi kepada beberapa orang, bahkan kotbah atau renungan dapat juga dalam bentuk sharing yang kemudian dirangkum oleh pemimpin
Pengumuman hendaknya dilaksanakan setelah doa lingkungan selesai, jangan di awal ataupun dipertengahan, sebab akan mengganggu suasana doa.

Franz Daru H