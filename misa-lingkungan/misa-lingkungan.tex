\documentclass[12pt,a4paper]{beamer}
\usepackage[latin1]{inputenc}
\usepackage{amsmath}
\usepackage{amsfonts}
\usepackage{amssymb}
\usepackage{graphicx}
\author{Penutupan Bulan Rosario}
\title{Misa Lingkungan St. Petrus}
\subtitle{Bersama Rm. Ketut, SVD}
\date{27 Oktober 2013 }
\usecolortheme{albatross}
%\usetheme[height=7mm]{Rochester}

\begin{document}
\maketitle
\begin{frame}{Pembuka}

\begin{itemize}
\item Lagu Pembuka: Kelana MB-160
\item Tanda Salib
\item Salam
\end{itemize}
\begin{description}[1cm]
\item [I] 	Semoga kasih karunia, rahmat dan damai sejahtera dari Allah Bapa dan dari Putera-Nya Yesus Kristus besertamu.
\item [U] 	Dan sertamu juga.
\end{description} 

\begin{itemize}
\item Kata Pengantar
\end{itemize}
\end{frame}

\begin{frame}{Pembuka}
\begin{itemize}
\item Tobat
\end{itemize}

\begin{description}[1cm]
\item [I]  Marilah kita membuka hati dan menghayati dengan sungguh kehadiran Tuhan di tempat ini, sekaligus memohon belaskasih pengampunanNya, agar kita pantas merayakan perjamuan syukur ini.
\item [U]	Saya mengaku \ldots
\item [I]	Semoga Allah yang mahakuasa mengampuni dosa kita dan menghantar kita ke hidup yang kekal. 
\item [U] Amin
\end{description}
\begin{itemize}
\item Tuhan kasihanilah MB-189 + MB-206
\end{itemize}
\end{frame}

\begin{frame}{Pembuka}
\begin{itemize}
\item Doa Pembukaan
\end{itemize}
\begin{description}[1cm]
\item [I] Marilah berdoa :\\
	Tuhan Allah Bapa kami, kami bersyukur kepadaMu atas karyaMu yang agung di tengah kami. Engkau telah memberi kesempatan kepada kami keluarga besar lingkungan St Petrus, stasi Maguwo, dalam memekarkan diri menjadi lingkungan-lingkungan yang baru. Engkau pula yang setia membimbing dan menuntun kami sehingga kami boleh bertekun menapaki hari demi hari selama satu bulan untuk berdoa rosario. Begitu banyak rahmat yang telah kami terima dariMu. 
\end{description}
\end{frame}

\begin{frame}{Pembuka}
\begin{description}[1cm]
\item [U] Berkatilah umatMu di lingkungan ini agar dapat meneladan keteguhan St. Petrus, kesetiaan dalam berdoa St. Monika, dan kerendahan hati St Theresia, yang dengan sepenuh iman percaya dan berserah diri pada Allah. Berilah kami semua ketenangan hati, kerukunan, semangat keteguhan, kesetian, kerendahan hati dan semangat persaudaraan agar bisa saling melayani dengan lebih baik bagi perkembangan gerejaMu. Demi Yesus Kristus, Putera-Mu, Tuhan dan Pengantara kami, yang hidup dan berkuasa bersama Dikau dan Roh Kudus, kini dan sepanjang segala masa. Amin.

\end{description}
\end{frame}

\begin{frame}{Liturgi Sabda}
\textit{Bacaan: Rom.8 : 31.b - 39}

\textit{Keyakinan Iman}

\begin{description}[1cm]
\item [P] \textit{Pembacaan dari Surat Rasul Paulus kepada jemaat di Roma }

      Sebab itu apakah yang akan kita katakan tentang semuanya itu? Jika Allah di pihak kita, siapakah yang akan melawan kita? Ia, yang tidak menyayangkan Anak-Nya sendiri, tetapi yang menyerahkan-Nya bagi kitasemua, bagaimanakah mungkin Ia tidak mengaruniakan segala sesuatu kepada kita, bersama-sama dengan Dia? Siapakah yang akan menggugat orang--orang pilihan Allah? Allah, yang membenarkan mereka? Siapakah yang akan menghukum mereka? 
\end{description}      
\end{frame}

\begin{frame}{Liturgi Sabda}
\begin{description}[1cm]

\item Kristus Yesus, yang telah mati? Bahkan lebih lagi: yang telah bangkit, yang juga duduk  di sebelah kanan Allah, yang malah menjadi Pembela bagi kita? Siapakah yang akan memisahkan kita dari kasih Kristus? Penindasan atau kesesakan atau penganiayaan, atau kelaparan atau ketelanjangan, atau bahaya, atau pedang? 
      
      Seperti ada tertulis :
"Oleh karena Engkau kami ada dalam bahaya mau sepanjang hari, kami telah dianggap sebagai domba-domba sembelihan"
\end{description}

\end{frame}

\begin{frame}{Liturgi Sabda}
\begin{description}[1cm]
\item Tetapi dalam semuanya itu kita lebih dari pada orang-orang yang menang, oleh Dia yang telah mengasihi kita. Sebab aku yakin, bahwa baik maut, maupun hidup, baik malaikat-malaikat, maupun pemerintah-pemerintah, baik yang ada sekarang, maupun yang akan datang, atau kuasa-kuasa, baik yang di atas, maupun yang yang di bawah, ataupun sesuatu makhluk lain, tidak akan dapat memisahkan kita dari kasih Allah, yang ada dalam Kristus Yesus, Tuhan kita.

       Demikianlah sabda Tuhan

\item [U] Syukur kepada Allah 
\end{description}
\end{frame}

\begin{frame}{Liturgi Sabda}
\begin{itemize}
\item Mazmur Tanggapan: Keheningan Hati (teks)
\item Bacaan Injil
\end{itemize}

\begin{description}[1cm]
\item [I]  Tuhan beserta kita 
\item [U]   Sekarang dan selama-lamanya
\item [I]    Inilah Injil Tuhan kita Yesus Kristus, menurut Santo  Lukas (13 : 31-35)

	\textit{Yesus harus mati di Yerusalem}
	
            Pada waktu itu datanglah beberapa orang Farisi dan berkata kepada Yesus ; "Pergilah, tinggalkanlah tempat ini, karena Herodes hendak membunuh Engkau". Jawab esus kepada mereka: "Pergilah dan katakanlah kepada  kepada si serigala itu: Aku mengusir setan dan menyembuhkan orang, pada hari ini dan besok, dan pada hari yang ketiga Aku akan selesai. 
\end{description}

\end{frame}            
\begin{frame}{Liturgi Sabda}
\begin{description}[1cm]
\item Tetapi hari ini dan besok dan lusa Aku harus meneruskan perjalanan-Ku, sebab tidaklah semestinya seorang nabi dibunuh kalau tidak di Yerusalem. Yerusalem, Yerusalem, engkau yang membunuh nabi-nabi dan melempari dengan batu orang-orang yang diutus kepadamu! Berkali-kali Aku rindu mengumpulkan anak-anakmu, sama seperti induk ayam mengumpulkan anak-anaknya di bawah sayapnya, tetapi kamu tidak mau. Sesungguhnya rumahmu ini akan ditinggalkan dan menjadi sunyi. Tetapi Aku berkata kepadamu: Kamu tidak akan melihat Aku lagi hingga saat kamu berkata : "Diberkatilah Dia yang datang dalam nama Tuhan!"

Demikianlah Injil Tuhan

\item [U]    Terpujilah Kristus

\end{description}
\end{frame}

\begin{frame}{Liturgi Sabda}

\begin{itemize}
\item Homili
\item Syahadat
\item Doa Umat
\end{itemize}
\begin{description}[1cm]
\item [I] Marilah berdoa kepada Allah Bapa dan Tuhan kita Yesus Kristus, yang telah mencurahkan darah-Nya demi keselamatan semua manusia:
\item [P] Bagi Gereja : Kami bersyukur atas Bapa Paus, para Uskup, Para Imam, Biarawan dan Biarawati yang dengan setia dan penuh iman mewartakan kerajaanMu. Berikanlah senantiasa rahmat-Mu  bagi mereka semua. Kami mohon \ldots
\end{description}
\end{frame}

\begin{frame}{}
\begin{description}[1cm]
\item [P]	Kami bersyukur atas lingkungan kami St. Petrus yang akan mekar bertambah dengan St. Monika, dan St. Theresia, sehingga kami Kau berkati dan dapat memekarkan wilayah dalam suasana yang harmonis. Kami senantiasa memohon limpahan kasih dan berkat-Mu agar lingkungan kami semakin mampu hidup meng-gereja untuk mewartakan Injil-Mu  Kami mohon \ldots

\item [P]	Kami bersyukur ya Tuhan, atas persaudaran dan persahabatan yang boleh terjalin selama  ini. Semoga kami tetap setia bergandeng tangan dan saling merengkuh satu sama lain dalam meneruskan perjuangan hidup kami di masa depan. Kami mohon \dots
\end{description}
\end{frame}

\begin{frame}{}
\begin{description}[1cm]
\item [P]  Kami mohon berkatMu ya Tuhan, bagi saudara-saudara kami yang saat ini sedang menderita sakit dan menghadapi masalah hidup. Semoga mereka senantiasa kuat dalam pencobaan dan selalu percaya akan pertolonganMu. Kami mohon \dots
\item [P]	Dan akhirnya ya Tuhan, kami bersyukur boleh menjadi umat-Mu. Curahkanlah Roh KudusMu agar membimbing dan menguatkan iman kami. Semoga dengan perantaraan Bunda Maria kami semakin pantas meneladan putra-Mu. Kami mohon \ldots
\item [I] 	Ya Tuhan, Engkau selalu dekat dengan mereka yang berharap kepada-Mu. Kepada-Mu kami percayakan seluruh syukur dan pengharapan kami ini. Demi Yesus Kristus Tuhan dan pengantara kami.
\item [U]	Amin.
\end{description}
\end{frame}

\begin{frame}{Liturgi Ekaristi}
\begin{itemize}
\item Persiapan Persembahan
\item Lagu Persembahan: Semuanya kuserahkan Kepada-Mu
\item Doa Persembahan
\end{itemize}

\begin{description}[1cm]
\item [I] Allah, Bapa yang mahakuasa, kami mohon, kuduskanlah pesembahan ini dengan berkat-Mu dan teguhkanlah niat kami untuk mengembangkan diri dan siap melayani Engkau dan sesama. Demi Yesus Kristus, Tuhan dan Pengantara kami. Amin.
\end{description}

\begin{itemize}
\item Prefasi dan Kudus

Lagu kudus: MB-260

\item Doa Syukur Agung
\item Bapa Kami 

	\textit{Putut Pudyantoro}
\end{itemize}
\end{frame}

\begin{frame}{Liturgi Ekaristi}
\begin{itemize}
\item Doa Damai
\end{itemize}
\begin{description}[1cm]
\item [I] 	Saudara-saudara, Tuhan kita Yesus Kristus mengutus Roh Kudus, agar dari segala bangsa dikumpulkan-Nya Gereja yang bersatu dalam cinta kasih. Sebab itu marilah kita mohon:
\item [U] 	Tuhan Yesus Kristus \ldots
\end{description}

\begin{itemize}
\item Anak Domba Allah
\item Komuni

	Lagu: Doaku (teks)
\item Madah Syukur
\item Doa sesudah komuni
\end{itemize}
\end{frame}

\begin{frame}{Penutup}
\begin{itemize}
\item Doa penutup
\end{itemize}
\begin{description}[1cm]
\item [I] 	Marilah berdoa
Ya Allah, kami bersyukur atas sabda dan santapan kudus yang kami terima pada malam ini. Semoga ekaristi ini meneguhkan dan mengobarkan hati  kami supaya lebih beriman dan rendah hati seturut teladan St. Theresia. Demi Yesus Kristus Tuhan dan Pengantara kami. 
\item [U] 	Amin
\end{description}


\begin{itemize}
\item Berkat 
\item Lagu Penutup: Salam Bintang Laut MB-677
\item Ucapan Terimakasih kepada Yth. Rm Ketut, SVD
\end{itemize}
\end{frame}
\end{document}