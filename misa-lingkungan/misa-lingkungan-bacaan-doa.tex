\documentclass[12pt,a4paper]{article}
\usepackage[latin1]{inputenc}
\usepackage{amsmath}
\usepackage{amsfonts}
\usepackage{amssymb}
\usepackage{graphicx}
\author{Penutupan Bulan Rosario}
\title{Misa Lingkungan St. Petrus}
\date{27 Oktober 2013 }

\begin{document}
\maketitle
\section*{Liturgi Sabda}
\textit{Bacaan: Rom.8 : 31.b - 39}

\textit{Keyakinan Iman}

\begin{description}
\item [P] \textit{Pembacaan dari Surat Rasul Paulus kepada jemaat di Roma }

      Sebab itu apakah yang akan kita katakan tentang semuanya itu? Jika Allah di pihak kita, siapakah yang akan melawan kita? Ia, yang tidak menyayangkan Anak-Nya sendiri, tetapi yang menyerahkan-Nya bagi kitasemua, bagaimanakah mungkin Ia tidak mengaruniakan segala sesuatu kepada kita, bersama-sama dengan Dia? Siapakah yang akan menggugat orang--orang pilihan Allah? Allah, yang membenarkan mereka? Siapakah yang akan menghukum mereka? 

 Kristus Yesus, yang telah mati? Bahkan lebih lagi: yang telah bangkit, yang juga duduk  di sebelah kanan Allah, yang malah menjadi Pembela bagi kita? Siapakah yang akan memisahkan kita dari kasih Kristus? Penindasan atau kesesakan atau penganiayaan, atau kelaparan atau ketelanjangan, atau bahaya, atau pedang? 
      
      Seperti ada tertulis :
"Oleh karena Engkau kami ada dalam bahaya mau sepanjang hari, kami telah dianggap sebagai domba-domba sembelihan"
 
  Tetapi dalam semuanya itu kita lebih dari pada orang-orang yang menang, oleh Dia yang telah mengasihi kita. Sebab aku yakin, bahwa baik maut, maupun hidup, baik malaikat-malaikat, maupun pemerintah-pemerintah, baik yang ada sekarang, maupun yang akan datang, atau kuasa-kuasa, baik yang di atas, maupun yang yang di bawah, ataupun sesuatu makhluk lain, tidak akan dapat memisahkan kita dari kasih Allah, yang ada dalam Kristus Yesus, Tuhan kita.

       Demikianlah sabda Tuhan

\item [U] Syukur kepada Allah 
\end{description}

\begin{description}
\item [I]  Tuhan beserta kita 
\item [U]   Sekarang dan selama-lamanya
\item [I]    Inilah Injil Tuhan kita Yesus Kristus, menurut Santo  Lukas (13 : 31-35)

	\textit{Yesus harus mati di Yerusalem}
	
            Pada waktu itu datanglah beberapa orang Farisi dan berkata kepada Yesus ; "Pergilah, tinggalkanlah tempat ini, karena Herodes hendak membunuh Engkau". Jawab esus kepada mereka: "Pergilah dan katakanlah kepada  kepada si serigala itu: Aku mengusir setan dan menyembuhkan orang, pada hari ini dan besok, dan pada hari yang ketiga Aku akan selesai. 

\item Tetapi hari ini dan besok dan lusa Aku harus meneruskan perjalanan-Ku, sebab tidaklah semestinya seorang nabi dibunuh kalau tidak di Yerusalem. Yerusalem, Yerusalem, engkau yang membunuh nabi-nabi dan melempari dengan batu orang-orang yang diutus kepadamu! Berkali-kali Aku rindu mengumpulkan anak-anakmu, sama seperti induk ayam mengumpulkan anak-anaknya di bawah sayapnya, tetapi kamu tidak mau. Sesungguhnya rumahmu ini akan ditinggalkan dan menjadi sunyi. Tetapi Aku berkata kepadamu: Kamu tidak akan melihat Aku lagi hingga saat kamu berkata : "Diberkatilah Dia yang datang dalam nama Tuhan!"

Demikianlah Injil Tuhan

\item [U]    Terpujilah Kristus

\end{description}

\newpage
\section*{Doa Umat}

\begin{description}

\item [I] Marilah berdoa kepada Allah Bapa dan Tuhan kita Yesus Kristus, yang telah mencurahkan darah-Nya demi keselamatan semua manusia:
\item [P] Bagi Gereja : Kami bersyukur atas Bapa Paus, para Uskup, Para Imam, Biarawan dan Biarawati yang dengan setia dan penuh iman mewartakan kerajaanMu. Berikanlah senantiasa rahmat-Mu  bagi mereka semua. Kami mohon \ldots

\item [P]	Kami bersyukur atas lingkungan kami St. Petrus yang akan mekar bertambah dengan St. Monika, dan St. Theresia, sehingga kami Kau berkati dan dapat memekarkan wilayah dalam suasana yang harmonis. Kami senantiasa memohon limpahan kasih dan berkat-Mu agar lingkungan kami semakin mampu hidup meng-gereja untuk mewartakan Injil-Mu  Kami mohon \ldots

\item [P]	Kami bersyukur ya Tuhan, atas persaudaran dan persahabatan yang boleh terjalin selama  ini. Semoga kami tetap setia bergandeng tangan dan saling merengkuh satu sama lain dalam meneruskan perjuangan hidup kami di masa depan. Kami mohon \dots

\item [P]  Kami mohon berkatMu ya Tuhan, bagi saudara-saudara kami yang saat ini sedang menderita sakit dan menghadapi masalah hidup. Semoga mereka senantiasa kuat dalam pencobaan dan selalu percaya akan pertolonganMu. Kami mohon \dots
\item [P]	Dan akhirnya ya Tuhan, kami bersyukur boleh menjadi umat-Mu. Curahkanlah Roh KudusMu agar membimbing dan menguatkan iman kami. Semoga dengan perantaraan Bunda Maria kami semakin pantas meneladan putra-Mu. Kami mohon \ldots
\item [I] 	Ya Tuhan, Engkau selalu dekat dengan mereka yang berharap kepada-Mu. Kepada-Mu kami percayakan seluruh syukur dan pengharapan kami ini. Demi Yesus Kristus Tuhan dan pengantara kami.
\item [U]	Amin.
\end{description}

\end{document}