LAGU PEMBUKAAN 

TANDA SALIB DAN SALAM 

I : Dalam Nama Bapa dan Putera dan Roh Kudus 
U : Amin 
I : Semoga Allah yang telah membangkitkan Yesus Kristus, 
PuteraNya dari alam maut, melimpahkan penghiburan dan kekuatan 
iman kepada kita sekalian. 
U : Sekarang dan selama-lamanya 
KATA PEMBUKAAN 

I : Saudara-saudariku sekalian, keluarga yang berduka yang terkasih 
dalam Tuhan. Sebentar lagi kita akan berpisah secara jasmani dengan 
…………….… ini. Maka sebelum kita berpisah dengan dia, baiklah 
kalau sekali lagi kita mengucapkan selamat jalan kepadanya. Semoga 
doa dan salam yang kita ucapkan pada makam ini dapat 
melambangkan cinta, meringankan duka dan meneguhkan iman kita. 
Sebab kita berharap akan berjumpa lagi dengan ………….… ini 
dalam keluarga abadi, yaitu bila Kristus sendiri datang sebagai 
pemenang atas maut untuk mengumpulkan semua sahabatNya dalam 
kerajaan Bapa. 
I : Marilah berdoa: 
Allah yang Maha Kuasa dan Maha rahim, kehidupan dan kematian 
kami berada di dalam tanganMu. Engkau telah memanggil 
……………… dari kehidupan di dunia ini untuk menghadap 
hadiratMu. Dengan hati sedih kami berdiri di sini untuk 

60 



membaringkan jenazahnya dalam makam, namun dengan penuh 
harapan kami menantikan kebangkitan, sebab Kristus telah bangkit 
sebagai yang pertama dari antara orang-orang mati. Maka, 
kasihanilah dia ya Tuhan, kasihanilah dia dan terimalah dia dalam 
pelukan cintaMu. Demi Kristus, Tuhan dan Pengantara kami. 

U : Amin 
BACAAN INJIL: Dari Yohanes 6:37-40 

I : Tuhan sertamu 
U : Dan sertamu juga 
I : Inilah Injil Suci Yesus Kristus menurut Santo Yohanes 
U : Dimuliakanlah Tuhan 
I : 6:37 Pada waktu itu, Yesus bersabda, “Semua yang diberikan 
Bapa kepada-Ku akan datang KepadaKu, barang siapa datang 
kepada-Ku, ia tidak akan Kubuang. 6:38 Sebab Aku telah turun dari 
sorga bukan untuk melakukan kehendak-Ku, tetapi untuk melakukan 
kehendak Dia yang telah mengutus Aku. 6:39 Dan Inilah kehendak 
Dia yang telah mengutus Aku, yaitu supaya dari semua yang telah 
diberikan-Nya kepada-Ku jangan ada yang hilang, tetapi supaya 
Kubangkitkan pada akhir zaman. 6:40 Sebab inilah kehendak bapa-
Ku, yaitu supaya setiap orang, yang melihat anak dan yang percaya 
kepada-Nya beroleh hidup yang kekal, dan supaya Aku 
membangkitkannya pada akhir zaman.” 
U : Terpujilah Kristus 
LAGU UNTUK MENGIRINGI PEMBERKATAN 

PEMBERKATAN MAKAM 

I : Marilah berdoa: 
Tuhan Yesus Kristus, Engkau sendiri berbaring dalam makam 
selama tiga hari. Kami mohon sucikanlah (+) makam ini, agar 

61 



hambaMu ……...........… yang kami istirahatkan di sini akhirnya 
bangkit bersama Engkau dan hidup mulia sepanjang segala masa. 

U : Amin 
PENGUBURAN 

(makam direciki dengan air suci dan didupai. Kemudian peti jenazah 
diturunkan ke liang lahat. Umat dapat mengiringinya dengan 
nyanyian yang sesuai) 

(setelah peti jenazah diturunkan): 

DIRECIKI AIR SUCI 

I : Ketika dibaptis kita disatukan dengan Kristus dan turut mati 
bersama dengan Dia. ………………… yang kita kasihi ini sekarang 
mati bersama dengan Kristus. Semoga ia hidup pula dalam keadaan 
baru seperti Kristus. 
U : Amin 
DIDUPAI 

I : Semoga doa-doa kita mengiringi ………………..… dalam 
perjalanannya menuju rumah Bapa 
U : Amin 
DITABURI BUNGA 

I : Semoga kuntum hidup ilahi yang telah ditanamkan dalam diri 
………………..… ini, akan mekar bagaikan bunga yang semerbak 
harum mewangi 
U : Amin 
DITABURI TANAH YANG SUDAH DIBERKATI 

62 



I : Manusia diciptakan dari tanah dan ia kembali ke tanah. Semoga 
Kristus mengalahkan kebinasaan maut dan memulihkan 
……………….… ini dalam kebangkitan mulia. 
U : Amin. 
DITANDAI SALIB 

I : Saudara terkasih, masuklah dalam kehidupan abadi dengan 
membawa tanda kemenangan Kristus: Demi nama Bapa dan (+) 
Putera dan Roh Kudus 
U : Amin 
DOA UMAT 

I : Saudara-saudari sekalian yang terkasih, marilah kita berdoa 
kepada Allah, Bapa yang maharahim, bagi ……………….… yang 
kita kasihi ini, yang telah meninggal dalam persatuan dengan Tuhan. 
Semoga dosa-dosanya diampuni. Marilah kita memohon ……… 
U : Kabulkanlah doa kami, ya Tuhan 
I : Semoga amal baktinya di dunia ini diterima dengan baik. Marilah 
kita mohon ……… 
U : Kabulkanlah doa kami, ya Tuhan 
I : Semoga ia menikmati kehidupan kekal dalam kemuliaan Allah 
Bapa. Marilah kita mohon …..… 
U : Kabulkanlah doa kami , ya Tuhan 
I : Marilah kita berdoa pula bagi semua orang yang berkabung atas 
kematian …………... ini. Semoga kesepian mereka dipenuhi dengan 
cinta kasih Allah. Marilah kita mohon …..… 
U : Kabulkanlah doa kami, ya Tuhan 
I : Semoga mereka dihibur dalam kesusahan mereka. Dan semoga 
iman dan harapan mereka diperteguh. Marilah kita mohon …..… 
U : Kabulkanlah doa kami, ya Tuhan 
63 



I : Semoga hati kita tidak tenggelam dalam urusan-urusan duniawi, 
tetapi selalu terbuka bagi segala rencana dan kehendak Tuhan. 
Marilah kita mohon ……… 
U : Kabulkanlah doa kami, ya Tuhan 
I : Allah yang kekal dan kuasa, Engkaulah Tuhan bagi orang hidup 
dan juga Tuhan bagi orang-orang mati. Kami mohon belas 
kasihanMu bagi ……………… yang sudah mendahului kami dalam 
imannya. Ampunilah segala dosanya, agar ia bergembira atas diri-Mu 
dan tak henti-hentinya memuji dan memuliakan Engkau. Demi 
Kristus, Tuhan dan Pengantara kami 
U : Amin 
BAPA KAMI 

I : Marilah kita satukan semua doa permohonan dan kerinduan hati 
kita, dalam doa yang diajarkan Kristus sendiri: 
U : Bapa kami yang ada di surga, dimuliakanlah namaMu, datanglah 
kerajaanMu, jadilah kehendakMu, di atas bumi seperti di dalam 
surga. Berilah kami rezeki pada hari ini dan ampunilah kesalahan 
kami, seperti kami pun mengampuni yang bersalah kepada kami, dan 
janganlah masukkan kami ke dalam percobaan, tetapi bebaskanlah 
kami dari yang jahat. 
I : Ya Bapa, bebaskanlah kami dari segala yang jahat dan berilah 
kami damaiMu. Kasihanilah dan bantulah kami, supaya selalu bersih 
dari noda dosa dan terhindar dari segala gangguan, sehingga kami 
dapat hidup dengan tenteram, sambil mengharapkan kedatangan 
Penyelamat kami, Yesus Kristus. 
U : Sebab Engkaulah Raja yang mulia dan berkuasa untuk selamalamanya 
I : Damai Tuhan kita Yesus Kristus, besertamu 
64 



U : Dan sertamu juga 
BERKAT PENUTUP 

I : Tuhan, berilah dia istirahat kekal 
U : Dan sinarilah dia dengan cahaya abadi 
I : Semoga ia beristirahat dalam damai 
U : Amin 
I : Saudara-saudariku sekalian, upacara pelepasan dan pemakaman 
……………… yang kita kasihi ini, telah selesai. 
U : Syukur kepada Allah 
I : Pulanglah dalam damai Tuhan 
U : Amin 
LAGU PENUTUP 

(sementara menyanyikan lagu penutup, keluarga dan semua yang 
hadir diundang untuk menaburkan bunga dan/atau tanah pada liang 
lahat. Setelah itu, jenazah ditimbuni dengan tanah dan umat yang 
hadir dapat mengiringinya dengan doa rosario dan lagu-lagu yang 
sesuai). 

Requiescat In Pace 

65 



