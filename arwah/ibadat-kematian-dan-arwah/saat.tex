
\section*{PEMBUKAAN} 

\BI{ Dalam Nama Bapa dan Putera dan Roh Kudus }
\BU{Amin }
\BP{ Tuhan yang memberi, Tuhan yang mengambil, terpujilah 
nama Tuhan }
\BU{Sekarang dan selama-lamanya }
\BI{ Saudara/i yang terkasih, Allah Sang Sumber dan Tujuan Hidup, 
telah memberikan kehidupan kepada saudara/i kita \nama, 
dan kini ia telah dipanggil untuk kembali kepangkuan-Nya. Kita 
percaya bahwa seluruh hidup manusia ada ditangan Allah. Kita hanya 
bisa bersembah sujud kepada-Nya dan percaya penuh kepada 
penyelenggaran dan kehendak-Nya. Sekarang ini \nama  
telah sampai pada akhir perjalanan hidupnya. Ia telah sampai kepada 
Sang Pencipta dan Penyelamat. Maka marilah kita mempersiapkan 
\nama. dengan ibadat perawatan jenazah, supaya dia 
menghadap Bapa di surga dengan pantas. }
\BI{ Marilah berdoa: (hening sejenak): Allah Bapa yang Maha baik, 
Engkau menciptakan kami menurut citra-Mu dan mengangkat kami 
menjadi anak-anak-Mu berkat Putera-Mu Yesus Kristus. Baru saja 
Engkau memanggil \nama untuk berpulang kepada-Mu di 
surga. Kini jenazah akan kami siapkan supaya ia pergi menghadap-
Mu secara pantas. Kami percaya bahwa Engkau berkenan 
memandang dan menerimanya dengan penuh kasih. Kami mohon, 
bersihkanlah \nama dari segala dosa dan kekurangannya. 
Tuhan, berilah ia pakaian pesta dan terimalah ia dalam keluarga-Mu 
yang bahagia di surga. }
\BU{Amin. }

\section*[BACAAN]{BACAAN\\Pembacaan dari surat Santo Paulus kepada umat di Roma (Rm.6:8-9)}
\BP{Jadi jika kita telah mati dengan Kristus, kita percaya bahwa kita akan hidup juga dengan Dia. Karena kita tahu bahwa Kristus sesudah bangkit dari antara orang mati, tidak mati lagi, maut tidak berkuasa lagi atas Dia. Demikianlah Sabda Tuhan. }

\BU{Syukur kepada Allah. }
\BP{Saudara/i marilah kita mengiringi kepergian \nama  
dengan doa dan penyerahan yang ikhlas kepada Allah. Marilah kita 
memanjatkan permohonan-permohonan dengan menjawab seruan-seruan 
berikut dengan "\textbf{hantarkanlah dia kehadapan Allah}" (hening 
sejenak).}
\BP{Tuhan Yesus Kristus, Engkau telah memanggil \nama sambutlah dan terimalah kedatangannya ya Yesus.}
\BU{Hantarkanlah dia kehadapan Allah.} 
\BP{ Para Kudus Allah datanglah menolong, para malaikat Allah 
datanglah menyongsong.} 
\BU{Hantarkanlah dia kehadapan Allah.}
\BP{ Para malaikat Allah bawalah dia ke pangkuan Abraham.}
\BU{Hantarkanlah dia kehadapan Allah.}
\BP{ Bunda Maria ulurkanlah tanganmu, ajaklah dia menuju 
surga abadi.}
\BU{Hantarkanlah dia kehadapan Allah.}
\BP{ Tuhan, berilah dia istirahat kekal dan semoga cahaya 
yang kekal menerangi perjalanan dia.}
\BU{Hantarkanlah dia kehadapan Allah.}

\textit{PEMIMPIN IBADAT MENGULURKAN TANGAN KEATAS 
JENAZAH }

\BP{ Allah sang pencipta segala sesuatu, janganlah lupa akan ciptaan-
Mu yang pernah Kau anugerahi hidup. Engkau sendiri yang 
memanggil \nama untuk meninggalkan dunia ini. 
Terimalah dia di tempat kediaman-Mu, sebab dialah putera-Mu, 
ciptaan-Mu, dan milik-Mu. Ya Bapa, ia belum dapat menyelesaikan 
semua tugas selama hidup di dunia ini, tetapi Engkau sudah 
memanggilnya. Kami percaya Engkau berkenan menyempurnakan 
apa yang belum dapat diselesaikannya dalam batas waktu yang 
Engkau berikan. Kini jantungnya tidak berdenyut lagi, dan kelopak 
matanya sudah Engkau katupkan. Hanya satu yang masih 
dirindukannya, yakni Engkau, Allahnya dan Allah kami. Di dalam 
Engkaulah ya Bapa, ia akan berbahagia selama-lamanya.}
\BU{Tidak seorangpun hidup bagi dirinya sendiri, tidak seorangpun 
mati bagi dirinya sendiri, kita hidup dan mati bagi Allah, sebab kita ini milik Allah.}
\BP{ Kami mohon pula, ya Bapa, bagi kaum kerabat dan handai taulan 
yang ditinggalkannya. Sudilah Engkau mendampingi dan 
menguatkan hati mereka sehingga dapat menerima kenyataan ini 
dengan hati yang tabah dan berserah sepenuhnya kepada-Mu. 
Semoga kami makin menyadari bahwa hidup dan mati sepenuhnya 
ada di tangan-Mu. Teguhkanlah iman kami agar senantiasa percaya 
akan kebijaksanaan-Mu, karena Kristus Tuhan, pengantara kami yang 
hidup dan berkuasa, kini dan sepanjang masa.}
\BU{Amin.}

\textit{PEMIMPIN IBADAT MEMERCIKI JENAZAH DENGAN AIR 
SUCI }

\BP{ Semoga Tuhan membersihkan hatimu dan mengubah tubuhmu 
menjadi serupa dengan tubuh-Nya yang mulia.}
\BU{Amin.}
\BP{ Tuhan, berikanlah saudara kami ini istirahat kekal.}
\BU{Amin.}
\BP{ Semoga cahaya yang kekal menerangi perjalanan dia.}
\BU{Amin.}
\BP{ Semoga ia beristirahat dalam ketentraman abadi.}
\BU{Amin.}

\textit{Pada kematian orang dewasa: }

\BP{ Marilah kita berdoa: (hening sejenak) Allah yang maharahim, 
kami percayakan saudara \nama ini kepada-Mu. Ampunilah 
dengan murah hati segala dosa yang telah ia lakukan karena 
kerapuhan manusiawi. Bapa, selagi masih hidup ia menikmati apa 
yang diharapkan pada-Mu, yakni hidup bahagia bersama Engkau. 
Demi Kristus, Tuhan kami.}
\BU{Amin.}

\textit{Pemimpin ibadat membuat tanda salib pada dahi jenazah, kemudian melanjutkan: }

\BP{ Kami mohon pula, ya Bapa, bagi kami dan handai taulan yang 
ditinggalkannya. Sudilah Engkau mendampingi dan meneguhkan 
mereka sehingga dapat menerima kenyataan ini dengan hati yang 
tabah, karena mereka percaya akan kebijaksanaan-Mu yang tidak 
kami selami. Semoga perhatian dan pertolongan dari saudara-saudari 
yang berbelasungkawa menghibur dan menguatkan hati mereka. 
Demi Kristus Tuhan kami, yang hidup dan berkuasa, kini dan 
sepanjang masa.}

\BU{Amin.}

\textit{Pada kematian seorang anak: }

\BP{ Saudara-saudari yang terkasih, Tuhan Yesus pernah bersabda, 
"Biarlah anak-anak datang kepada-Ku, sebab orang-orang seperti 
merekalah yang menjadi warga kerajaan surga." Percaya akan sabda 
ini, marilah kita berdoa. }

\textit{Pemimpin ibadat mengulurkan tangan ke atas jenazah: }

\BP{ Allah yang mahapengasih, terdorong oleh cinta-Mu, Engkau 
menciptakan anak ini dan menyerahklan dia kepada orang tuanya 
supaya dijaga dan dipelihara seturut kehendak-Mu. Belum puas 
mereka mengasuh dan membesarkan dia, namun kehendak-Mu yang 
bijaksana memutuskan bahwa anak ini harus pulang ke rumah-Mu. 
Maka kami menyerahkan dia kepada-Mu dengan ikhlas. Sudilah 
menerima anak ini dalam kedamaian rumah-Mu yang abadi. 
Perkenankanlah ia bersuka cita dan bergembira ria bersama para 
malaikat dan orang kudus-Mu. Demi Kristus Tuhan kami.}
\BU{Amin. }

\textit{Pemimpin ibadat membuat tanda salib pada dahi jenazah, kemudian melanjutkan}

\BP{ Allah Bapa di surga, kami berdoa juga bagi semua yang 
ditinggalkan anak ini. Teguhkanlah kepercayaan mereka, dan sudilah 
menghibur ayah-ibu serta seluruh keluarga. Kuatkanlah mereka 
dengan teladan Bunda Maria yang telah mengikhlaskan puteranya 
memenuhi kehendak-Mu. Ya Bapa tambahlah kepercayaan kami, dan 
teguhkanlah harapan kami. Demi Kristus Tuhan kami, yang hidup 
dan berkuasa kini dan sepanjang masa.}

\BU{Amin.}

\section*{DOA PENUTUP}

\BP{ Semoga Tuhan membuka pintu surga bagi putera-Nya yang pulang 
ke rumah Bapa di Kerajaan Allah. Disana tiada duka cita, tetapi 
hanya ada kebahagiaan untuk selama-lamanya. Bapa, berilah dia 
istirahat yang kekal dalam ketenteraman abadi bersama Engkau di 
surga.}
\BU{Amin.}

\subsection*{BERKAT PENGUTUSAN }

\BP{ Saudara-saudari terkasih dalam Kristus, sampai disinilah ibadat 
kita guna mendampingi saudara \nama pada saat mulia, 
peralihannya ke rumah Bapa. Marilah kita membantu keluarga yang 
mengalami kesusahan ini dengan menghibur mereka, dengan 
mengurus jenazah, dan mengatur pemakamannya. Semoga dalam 
seluruh pelaksanaan ini nanti kita diliputi berkat Allah yang 
mahakuasa, Bapa dan Putera dan Roh Kudus.}

\BU{Amin.}

\textit{Note: Sambil menunggu jenazah untuk dimandikan, dapat didaraskan 
Mazmur atau doa rosario. }

