\documentclass[10pt,a5paper,openany,fancyhdr]{memoir}
\usepackage[left=2cm,right=2cm,top=2cm,bottom=2cm]{geometry}
\usepackage[utf8x]{inputenc}
\usepackage[bahasa]{babel}
\usepackage{amsmath}
\usepackage{amsfonts}
\usepackage{amssymb}
\usepackage{palatino}
\usepackage{xspace}
\usepackage{verse}

%\author{Yohanes Suyanto}
\author{Gereja St Anna -- Duren Sawit \\  
Jakarta Timur}
\title{Ibadat Kematian\\ 
dan \\
Peringatan Arwah \\
{~}\\
\vspace{2cm}
~\\
(Edisi Percobaan) \\
~\\
sekretariat\_stanna@yahoo.com \\
~\\
~\\
}
\date{2009}
\setlength{\parindent}{0mm}
\makeatletter
\newcommand{\lagu}[1]{%
  {\parindent \z@ 
    \interlinepenalty\@M \slshape \mdseries \large \textit{#1}\par\nobreak \vskip 10\p@ }}
\newcommand{\keterangan}[1]{%
  {\parindent \z@ 
    \interlinepenalty\@M \slshape \mdseries \textit{#1}\par\nobreak \vskip 10\p@ }}
\makeatother

\newcommand{\BU}[1]{\begin{itemize}\itemsep0pt \item[U:] #1 \end{itemize}}
\newcommand{\BI}[1]{\begin{itemize}\itemsep0pt \item[I:] #1 \end{itemize}}
\newcommand{\BIU}[1]{\begin{itemize}\itemsep0pt \item[I+U:] #1 \end{itemize}}
\newcommand{\BPU}[1]{\begin{itemize}\itemsep0pt \item[P+U:] #1 \end{itemize}}
\newcommand{\BP}[1]{\begin{itemize}\itemsep0pt \item[P:] #1 \end{itemize}}

\newcommand{\inputlagu}[1]{\begin{textit} \input{#1} \end{textit}}

\newcommand{\nama}{Bapak VF Parlan\xspace}

\newlength\chaptitlelength
\newlength\chaptitlerlength

\makeatletter 
\newcommand\thickhrulefill[1]{%
  \leavevmode \leaders \hrule height #1 \hfill \kern \z@} 
\setlength\midchapskip{10pt} 
\makechapterstyle{mystyle}{
  \setlength\beforechapskip{0pt}
  \renewcommand\chapternamenum{} 
  \renewcommand\printchaptername{}
  \renewcommand\printchapternum{} 
  \renewcommand\chaptitlefont{\small\scshape\centering}
  \renewcommand*{\printchaptertitle}[1]{%
    \settowidth\chaptitlelength{\hspace*{1em}\chaptitlefont##1\hspace*{1em}}%
    \ifnum\chaptitlelength>\dimexpr0.7\textwidth\relax%
      \setlength\chaptitlelength{0.7\textwidth}%
    \fi%
    \setlength\chaptitlerlength{\textwidth}%
    \addtolength\chaptitlerlength{-\chaptitlelength}%
    \addtolength\chaptitlerlength{-2em}%
    \noindent\parbox[c]{.5\chaptitlerlength}{\normalsize\thickhrulefill{0.3ex}\par\vskip-1.5ex\thickhrulefill{0.2ex}}\hspace*{1em}%
    \parbox[c]{\chaptitlelength}{\chaptitlefont##1}\hspace*{1em}%
    \parbox[c]{.5\chaptitlerlength}{\normalsize\thickhrulefill{0.3ex}\par\vskip-1.5ex\thickhrulefill{0.2ex}}%
  }%
}
\makeatother
\chapterstyle{mystyle}
\chapterstyle{wilsondob}
\setcounter{secnumdepth}{2}

\begin{document}
\pagestyle{Ruled}
\begin{titlingpage}
\maketitle
\end{titlingpage}

Kematian merupakan peristiwa iman. Pada saat kematian, kita mengambil 
bagian dalam misteri Paskah Kristus. Bersama Yesus Kristus kita beralih 
dari dunia fana ke dalam kehidupan kekal. Kematian adalah pintu masuk ke 
dalam pemurnian diri manusia menuju pada keabadian. Kematian juga 
menghantar kita pada kepenuhan hidup di dalam dan bersama Kristus 
Tuhan kita. 

 



%Daftar Isi 
\renewcommand{\cftchapterfont}{\normalfont\rmfamily}   
\renewcommand{\cftsectionfont}{\normalfont\rmfamily}   
\newpage
\tableofcontents



\chapter{IBADAT PERINGATAN ARWAH HARI KE-40} 
\section*{Pembuka} 

\BP{Saudara-saudari terkasih, 
Pada 40 hari yang lalu, saudara kita 
tercinta \nama telah berpulang ke rumah Bapa dan 
meninggalkan kita semua. Kita pada hari ini berkumpul 
bersama-sama untuk mendoakan arwah saudara 
kita \nama. 
Kita berdoa agar Allah yang Maharahim 
senantiasa mengampuni dosanya dan memberikan ganjaran 
berkat amal baiknya sewaktu hidup bersama kita. Semoga 
Allah Bapa senantiasa mengampuni dosa dan 
kelemahannya sehingga almarhum senantiasa menerima 
kasih Tuhan untuk selamanya. 
Saudara-saudari terkasih marilah kita awali doa kita dengan 
menyanyikan lagu pembukaan.}

\subsection*{Lagu Pembuka}

\subsection*{Tanda Salib} 

\BP{Dalam nama Bapa dan Putera dan Roh Kudus.} 
\BU{Amin} 
\BP{Semoga Allah Bapa mengasihi kita dengan belas 
kasihNya. Semoga Allah Bapa senantiasa menerangi kita 
dengan sabdaNya dan semoga Allah Roh Kudus 
mempersatukan kita semua.}
\BU{Sekarang dan selama-lamanya.}

\subsection*{Tobat} 

\BP{Saudara-saudari sekalian menyadari kita manusia berdosa 
dan dosa serta kesalahan saudara kita tercinta yang 
meninggal 40 hari yang lalu \nama maka marilah kita 
dengan rendah hati bersujud dihadapanNya untuk 
memohon belas kasihanNya dann pengampunanNya.}
\BPU{Saya mengaku \ldots\ldots\ldots }

\BP{Semoga Allah yang mahakuasa mengasihani kita, 
mengampuni dosa kita dan mengantar kita ke dalam hidup 
yang kekal.}
\BU{Amin}

\subsection*{Doa Pembuka} 

\BP{Marilah Berdoa 

Allah Bapa yang mahamurah, Engkau telah menyerahkan 
Yesus, Putra-Mu kepada kematian, semua ini harus terjadi 
untuk melepaskan kami dari segala kuasa kegelapan dan 
dosa. Ya Bapa, anugerahkanlah hidup kekal kepada 
 \nama yang telah menghadap 
kehadiratMu 40 hari yang lalu. Ya Bapa, ampunilah 
segala dosa dan kesalahannya dan bukalah pintu 
kehidupan kekal baginya. Terimalah saudara kami 
tercinta ini kedalam keluarga kudusMu di tahta surgawi.} 
\BU{Amin}

\section*{BACAAN KITAB SUCI} 

\BP{Saudara-saudari terkasih marilah kita mempersiapkan hati 
dan budi untuk mendengarkan sabda Tuhan} 

\subsection*{Pembacaan dari Kitab Nabi Yesaya (25: 7-9)}

\BP{Dan diatas gunung ini Tuhan akan mengoyakkan kain 
perkabungan yang diselubungkan kepada segala suku bangsa 
dan tudung yang ditudungkan kepada segala bangsa-bangsa. Ia 
akan meniadakan maut untuk seterusnya ; dan Tuhan Allah 
akan menghapuskan air mata dari pada segala muka ; dan aib 
umatNya akan dijauhkanNya dan seluruh bumi, sebab Tuhan 
telah mengatakannya. Pada waktu itu orang akan berkata : 
“Sesungguhnya, inilah Allah kita, yang kita nanti-nantikan, 
supaya kita diselamatkan. Inilah Tuhan yang kita nantinantikan; marilah kita bersorak-sorak dan bersukacita oleh 
karena keselamatan yang diadakan-Nya !”


Demikianlah sabda Tuhan}
\BU{Syukur kepada Allah} 

\subsection*{Mazmur tanggapan atau Lagu} 

\subsection*{Bacaan Injil} 

\BP{Tuhan sertamu}
\BU{Dan sertamu juga}
\BP{Inilah Injil Yesus Kristus menurut Lukas (7:11-17)}
\BU{Dimuliakanlah Tuhan}

\BP{Yesus membangkitkan anak muda di Nain 

Kemudian Yesus pergi ke suatu kota yang bernama Nain. 
Murid-muridNya pergi bersama-sama dengan Dia, dan juga 
orang banyak menyertaiNya, berbondong-bondong. Setelah Ia 
dekat pintu gerbang kota, ada orang mati diusung ke luar, anak 
laki-laki, anak tunggal ibunya yang sudah janda, dan banyak 
orang dari kota itu menyertai janda itu. Dan ketika Tuhan 
melihat janda itu, tergeraklah hatiNya oleh belas kasihan, lalu 
Ia berkata kepadanya :”Jangan menangis!” Sambil 
menghampiri usungan itu Ia menyentuhnya, dan sedang para 
pengusung berhenti, Ia berkata : “Hai anak muda, Aku berkata 
kepadamu, bangkitlah !” Maka bangunlah orang itu dan duduk 
dan mulai berkata-kata, dan Yesus menyerahkannya kepada 
ibunya. Semua orang itu ketakukan dan mereka memuliakan 
Allah, sambil berkata :”Seorang nabi besar telah muncul di 
tengah-tengah kita,” dan “Allah telah melawat umatNya.” 
Maka tersiarlah kabar tentang Yesus di seluruh Yudea dan di 
seluruh daerah sekitarnya. 

Demikianlah Injil Tuhan}
\BU{Terpujilah Kristus}

\subsection*{Renungan Singkat} 

\subsection*{Doa Umat}

\BP{Saudara-saudari, 

Hati Yesus tergerak oleh belas kasihan karena melihat 
seorang ibu yang sedang mengalami duka yang 
mendalam, maka marilah kita bersama-sama berdoa, 
semoga Tuhan Yesus tergerak pula hatiNya untuk 
memperhatikan dan mengabulkan permohonan-permohonan 
kita bersama.}

\BP{Bagi \nama
 
Semoga kemurahatian dan belas kasih Kristus yang telah 
membangkitkan anak muda dari Naim juga 
membangkitkan \nama dari kematiannya dan 
menerima karunia hidup kekal. 

\textit{Hening sejenak}

marilah kita mohon} 
\BU{Kabulkanlah doa kami} 
\BP{Bagi semua orang yang sedang berduka karena kematian 
sanak saudara mereka semoga Tuhan Mahakasih 
membukakan selubung duka hati mereka dan memberikan 
semangat penuh harapan bahwa hidup ini bukan menuju 
kepada kematian akan tetapi menuju kepada kepenuhan 
hidup manusia dalam kehendak Bapa sendiri. 
\textit{Hening sejenak}

marilah kita mohon} 
\BU{Kabulkanlah doa kami} 
\BP{Marilah kita juga berdoa bagi mereka yang sedang 
mengalami sakratul maut, semoga mereka akhirnya dapat 
meninggal dengan damai dan tenang. Semoga dengan 
belaskasihan Kristus mereka dapat masuk dalam kesatuan 
dengan para kudus di surga abadi. 
\textit{Hening sejenak}

marilah kita mohon} 
\BU{Kabulkanlah doa kami} 
\BP{Saudara-saudari sekalian marilah kita persatukan semua 
doa permohonan dan harapan kita dengan doa yang 
diajarkan Yesus sendiri.}

\section*{BAPA KAMI} 

\BP{Ya Allah, Bapa kami,
 
Semoga Engkau berkenan mengabulkan doa-doa yang 
kami panjatkan ke hadiratMu. Selamatkanlah saudara 
kami yang tercinta \nama serta semua orang yang sudah 
meninggal. Anugerahkanlah istirahat dan damai abadi 
bagi mereka semua, sebab Engkaulah Tuhan kami, 
sepanjang segala masa.}
\BU{Amin}

\section*{PENUTUP} 

\subsection*{Doa Penutup} 

\BP{Marilah berdoa,
 
Allah Bapa kami yang mahapengasih dan penyanyang, 
semoga kebangkitan putraMu juga menjadi kebangkitan 
saudara kami \nama, Bapa semoga Engkau senantiasa 
membangkitkan semangat kami untuk terus menerus hidup 
seturut nasihat InjilMu. Bapa, semoga doa-doa yang kami 
panjatkan kehadiratMu mampu mengantar saudara-saudari 
kami yang sudah meninggal untuk memasuki kerajaanMu 
yang abadi di surga.} 
\BU{Amin}

\subsection*{Berkat Pengutusan} 

\BP{Saudara-saudari sekalian, 
Dengan ini upacara doa kita sudah selesai, semoga kita 
selalu diberkati oleh Allah yang mahakuasa: Bapa dan 
Putera dan Roh Kudus.}
\BU{Amin}

\subsection*{Lagu Penutup} 




\chapter*{LAGU-LAGU PUJIAN}
\addcontentsline{toc}{chapter}{Lagu-lagu Pujian} 

\renewcommand{\thesection}{\arabic{section}.}
\renewcommand{\thesubsection}{\arabic{subsection}.}
\newlength{\saveleftmargini} % define a temp variable for the original margin
\setlength{\saveleftmargini}{\leftmargini} % write the original margin in this variable
\setsecnumdepth{subsection}

\subsection{TETAP CINTA YESUS} 
\begin{verse}
Kumau cinta Yesus selamanya \\
Kumau cinta Yesus selamanya \\
Meskipun badai silih berganti\\ 
Dalam hidupku\\ 
Kutetap cinta Yesus selamanya\\
\end{verse}
\begin{altverse}
Ya Bapa, Bapa ini aku anakMu \\
Layakkanlah seluruh hidupku \\
Ya Bapa, Bapa ini aku anakMu \\
Pakailah sesuai dengan rencanaMu
\end{altverse}

\subsection{KASIH DARI SURGA}
\begin{altverse}
Kasih dari surga memenuhi tempat ini \\
Kasih dari bapa surgawi \\
Kasih dari Yesus mengalir di hatiku \\
Membuat damai di hidupku 
\end{altverse}
\begin{verse}
Mengalir kasih dari tempat tinggi \\
Mengalir kasih dari tahta Allah Bapa \\
Mengalir, mengalir, mengalir dan mengalir \\
Mengalir memenuhi hidupku \\
\end{verse}

\subsection{KINI SAATNYA} 
\begin{altverse}
Kini saatnya berdiri di altarNya\\ 
S’bab Allah Maha Kudus hadir di sini\\ 
Marilah memuji angkat tangan menyembah\\ 
S’bab Allah Maha Kudus hadir di sini
\end{altverse}
\begin{verse}
Kita masuk tahta suciNya \\
Bersama para malaikat menyembah\\ 
Mari puji Yesusku \\
Kita masuk hadiratNya Maha Kudus 
\end{verse}

\subsection{SEJAUH TIMUR DARI BARAT} 
\begin{altverse}
Sejauh timur dari barat Engkau membuang dosaku\\ 
Tiada kau ingat lagi pelanggaranku\\ 
Jauh ke dalam tubir laut, Kau melemparkan dosaku\\ 
Tiada Kau perhitungkan kesalahanku 
\end{altverse}
\begin{verse}
Betapa besar kasih pengampunanMu, Tuhan\\ 
Tak Kau pandang hina hati yang hancur\\ 
Kuberterima kasih kepadaMu ya Tuhan \\
Pengampunan yang Kau bri pulihkanku 
\end{verse}

\subsection{BERI PENGAMPUNAN} 
\begin{altverse}
Dengan rendah hati aku mengaku \\
Atas dosaku s’lama ini \\
Ya Tuhan, Maha Pengasih \\
B’ri pengampunan untukku 
\end{altverse}
\begin{verse}
Kristus Putra bapa, Krsitus Juru S’lamat \\
Sungguh kusesali dosaku \\
Aku bersalah padaMu \\
B’ri pengampunan untukku 
\end{verse}

\subsection{KUSIAPKAN HATIKU TUHAN} 
\begin{altverse}
Kusiapkan hatiku tuhan ‘tuk dengar FirmanMu saat ini \\
Kusujud menyembahMu Tuhan Dalam hadiratMu saat ini 
\end{altverse}
\begin{verse}
Curahkan urapanMu Tuhan Bagi jemaatMu saat ini \\
Kusiapkan hatiku tuhan’tuk dengar FirmanMu 
\end{verse}
\begin{altverse}
\textit{Reff:} \\
FirmanMu Tuhan tiada berubah \\
Dahulu sekarang selama-lamanya, tiada berubah \\
FirmanMu Tuhan Penolong hidupku \\
Kusiapkan hatiku Tuhan ‘tuk dengar FirmanMu 
\end{altverse}

\subsection{FIRMANMU PELITA BAGI KAKIKU} 
\begin{altverse}
Firmanmu p’lita bagi kakiku terang bagi jalanku\\ 
FirmanMu p’lita bagi kakiku terang bagi jalanku 
\end{altverse}

\begin{verse}
Waktu kubimbang dan hilang jalanku \\
Tetaplah Kau di sisiku \\
Dan tak’kan ku takut sal Kau di dekatku\\ 
Besertaku selamanya 
\end{verse}


\subsection{BETAPA HATIKU} 
\begin{altverse}
Betapa hatiku berterima kasih Yesus,\\ 
Kau mengasihiku, Kau memilikiku
\end{altverse} 

\begin{verse}
Hanya ini Tuhan persembahanku, \\
Segenap hidupku jiwa dan ragaku \\
Sbab tak kumiliki harta kekayaan, \\
Yang cukup berarti ‘tuk kupersembahkan
\end{verse} 

\begin{altverse}
Hanya ini Tuhan permohonanku,\\ 
Terimalah Tuhan persembahanku \\
Pakailah hidupku sebagai alatMu,\\ 
Seumur hidupku
\end{altverse} 

\subsection{BAPA SURGAWI} 
\begin{altverse}
Bapa surgawi, ajarku mengenal \\
Betapa dalamnya kasihMu \\
Bapa surgawi buatku mengerti \\
Betapa kasihMu padaku
\end{altverse} 

\begin{verse}
Semua yang terjadi di dalam hidupku \\
Ajarku menyadari Kau selalu sertaku \\
Bri hatiku selalu, bersyukur padaMu \\
Karna rencanaMu indah bagiku
\end{verse} 

\subsection{BAYU SENJA} 
\begin{altverse}
Hidup bagai biduk di laut lepas \\
Aku pelaut tunggal siap melaju
\end{altverse}
 
\textit{Reff:} 
\begin{verse}
O bayu senja, hembusan sang Ilahi \\
Bawa bidukku ke tepian cerah \\
Pantai umat tebusan
\end{verse} 

\begin{altverse}
Arus gelombang tantang biduk tak daya \\
Hati merayu Tuhan nada nan cemas 
\end{altverse}

\textit{Reff}
 
\begin{verse}
Malam kabut nan pekat bintang pun lenyap\\ 
Setia kunanti Dikau wujud tak tampak \\
\textit{Reff}
\end{verse}
 

\subsection{ AVE MARIA}
\begin{altverse}
Engkau yang dipilih Allah Bapa di surga\\
Untuk melahirkan PutraNya yang kudus\\
Engkaulah Bunda Kristus\\
Bunda sang Penebus s’gala dosa manusia
\end{altverse}

\begin{verse}
Bunda Maria p’rawan yang tiada bernoda\\
Hatimu bersinar putih tiada bercela\\
Engkau Bunda Almasih yang diangkat\\
Ke surga penuh kemuliaan
\end{verse}

\begin{altverse}
Ave maria, ave Maria\\
Terpujilah Bunda, terpuji namaMu\\
S’panjang s’gala masa\\
Ave maria, ave Maria Syukur kepadaNya\\
Tuhan yang Pengasih S’lama-lamanya
\end{altverse}

\subsection{ TUHAN ADALAH GEMBALAKU}
\begin{altverse}
Tuhan adalah gembalaku, Tak’kan kekurangan aku\\
Ia membaringkan aku, Di padang yang berumput hijau
\end{altverse}

\begin{verse}
\textit{Reff:}\\
Ia membimbingku ke air yang tenang\\
Ia menyegarkan jiwaku\\
Ia menuntunku ke jalan yang benar\\
Oleh karna namaNya\\
Sekalipun aku berjalan, dalam lembah kekelaman
\end{verse}

\begin{altverse}
Aku tidak takut bahaya, Sebab Engkau besertaku\\
GadaMu dan tongkatMu, Itulah yang menghibur aku
\end{altverse}

\subsection{ KAU YANG TERINDAH}
\begin{altverse}
Kau yang terindah di dalam hidup ini\\
Tiada Allah Tuhan yang seperti Engkau\\
Besar perkasa penuh kemuliaan
\end{altverse}

\begin{verse}
Kau yang termanis di dalam hidup ini\\
Kucinta Kau lebih dari segalanya\\
Besar kasih setiaMu kepadaku
\end{verse}

\begin{altverse}
\textit{Reff:}\\
Kusembah Kau ya Allahku, Kutinggikan namaMu selalu\\
Tiada lutut tak bertelut, Menyembah Yesus Tuhan Rajaku
\end{altverse}

\begin{verse}
Kusembah Kau ya Allahku, Kutinggikan namaMu selalu\\
Selalu lidah kan mengaku, Engkaulah Yesus Tuhan Rajaku
\end{verse}

\subsection{ BAPA SURGAWI}
\begin{altverse}
Bapa surgawi ajarku mengenal\\
Betapa dalamnya kasihMu\\
Bapa surgawi buatku mengerti
\end{altverse}

\begin{verse}
Betapa kasihMu padaku\\
Semua yang terjadi, di dalam hidupku\\
Ajarku menyadari Kau selalu sertaku\\
Bri hatiku selalu, bersyukur padaMu\\
Karna rencanaMu indah bagiku
\end{verse}

\subsection{ ALLAH PEDULI}
\begin{altverse}
Banyak perkara yang tak dapat kumengerti\\
Mengapakah harus terjadi\\
Didalam kehidupan ini
\end{altverse}

\begin{verse}
Satu perkara yang kusimpan dalam hati\\
Tiada sesuatu kan terjadi tanpa Allah peduli
\end{verse}

\begin{altverse}
Allah mengerti, Allah peduli\\
Segala persoalan yang kita hadapi\\
Tak akan pernah dibiarkannya\\
Kubergumul sendiri\\
S’bab Allah peduli
\end{altverse}

\subsection{ ONE DAY AT THE TIME}
\begin{altverse}
I’m only human, I’m just a man\\
Help me believe in what I could be\\
And all that I am\\
Show me the stairway I have to climb\\
Lord for my sake; Teach me to take\\
One day at a time
\end{altverse}

\begin{verse}
\textit{Chorus:}\\
One day at a time sweet Jesus\\
That’s all I’m asking from you\\
Just give me a strength to do everything\\
What I have to do\\
Yesterday’s gone sweet Jesus\\
Tomorrow may never be mine
\end{verse}

\begin{verse}
Help me today show me the way\\
One daya at a time
\end{verse}

\begin{altverse}
Do you remember\\
When You walk among men\\
Well Jesus know\\
If you are looking below\\
It’s worst now and then\\
Cheating and stealing\\
Violence and crimed\\
So for may sake Lord teach me to take\\
One day at a time\\
\textit{back to chorus}
\end{altverse}

\subsection{ IN MOMENT LIKE THIS}
\begin{altverse}
In moment like this, I sing out a song\\
I sing out a love song to Jesus\\
In moment like this, I lift up my hands\\
I lift up my hands to the Lord
\end{altverse}

\subsection{ GIVE THANKS}
\begin{altverse}
Give thanks with a greatful heart\\
Give thanks to the Holy One\\
Give thanks because He’s given\\
Jesus Christ, His son:
\end{altverse}

\begin{verse}
And now let the weak say I’m strong\\
Let the poor say I’m rich\\
Because of what the Lord has done for us:\\
Give thanks

\end{verse}

\subsection{ TUBUHKU YESUS}
\begin{altverse}
TubuhMu Yesus sucikan daku\\
TubuhMu Yesus bebaskanku\\
TubuhMu Yesus ubahkan daku\\
Ku dijadikan baru
\end{altverse}

\subsection{ DARAHMU YESUS}
\begin{altverse}
DarahMu Yesus sucikan daku\\
DarahMu Yesus bebaskanku\\
DarahMu Yesus ubahkan adaku\\
Ku dijadikan baru
\end{altverse}

\subsection{ EL SHADAI}
\begin{altverse}
Tak usah ku takut, Allah menjagaku\\
Tak usah ku bimbang, Yesus p’liharaku\\
Tak usah ku susah, Roh Kudus hiburku\\
Tak usah ku cemas, Dia memberkatiku
\end{altverse}

\begin{verse}
El Shadai, El Shadai, Allah Maha kuasa\\
Dia besar, Dia besar, el Shadai mulia\\
El Shadai, El Shadai, Allah Maha kuasa\\
BerkatNya melimpah, El Shadai
\end{verse}

\subsection{ TIAP LANGKAHKU}
\begin{altverse}
Tiap langkahku diatur oleh Tuhan\\
Dan tangan kasihNya memimpinku\\
Di tengah badai dunia menakutkan\\
Hatiku tetap tenang teduh
\end{altverse}

\begin{verse}
\textit{Reff:}\\
Tiap langkahku ku tahu Tuhan yang pimpin\\
Ke tempat tinggi ku dihantarNya\\
Hingga sekali nanti aku tiba
\end{verse}

\begin{altverse}
Di rumah bapa, surga yang baka\\
Di waktu imanku mulai goyah\\
Dan bila jalanku hampir sesat
\end{altverse}

\begin{verse}
Kupandang Tuhanku Penebus dosa\\
Ku teguh sebab Dia dekat\\
\textit{Reff}
\end{verse}


\begin{altverse}
Di dalam tuhan saja harapanku\\
Sebab di tanganNya sejahtera\\
DibukaNya Yerusalem yang baru\\
Kota Allah yang suci mulia\\
\textit{Reff}
\end{altverse}



\subsection{ DI DOA IBUKU}
\begin{altverse}
Di waktuku masih kecil gembira dan senang\\
Tiada duka kukenal, tak kunjung mengerang\\
Di sore hari nan sepi ibuku bertelut\\
Sujud berdoa kudengar, namaku disebut
\end{altverse}

\begin{verse}
Seringlah ini kukenang di masa yan benar\\
Di kala hidup mendesak dan nyaris kusesat\\
Melintas gambar ibuku sewaktu bertelut\\
Kembali sayup kudengar namaku disebut
\end{verse}

\begin{altverse}
\textit{Reff:}\\
Di doa ibuku namaku disebut\\
Di doa ibu kudengar ada namaku disebut
\end{altverse}

\begin{verse}
Sekarang dia telah pergi ke rumah yang senang\\
Namun kasihnya padaku selalu kukenang\\
Kelak di sana kami pun bersama bertelut\\
Memuji Tuhan yang dengar namaku disebut\\
\textit{Reff}
\end{verse}

\subsection{ TUHAN BERIKANLAH}
\begin{altverse}
Tuhan berikanlah istirahat\\
Abadi dan tenang bagi yang wafat\\
Beri pengampunan segala dosanya\\
Karna Maha murah hatiMu Allah
\end{altverse}

\begin{verse}
Kami berimankan sabda Putra\\
Aku kebangkitan dan kehidupan\\
Barang siapalah percaya ‘kan daku\\
Ia akan hidup untuk selamanya
\end{verse}

\begin{altverse}
Kami menantikan saat itu\\
Maut akan lenyap diganti hidup\\
Smoga kami kelak memandang wajahMu\\
Di sinari terang dalam rumahMu
\end{altverse}

\subsection{ INDAH RENCANAMU TUHAN}
\begin{altverse}
Indah rencanaMU Tuhan di dalam hidupku\\
Walau ku tak tahu dan ku tak mengerti semua jalanMu\\
Dulu ku tak tahu Tuhan berat kurasakan\\
Hati menderita namun tak kuasa menghadapi semua
\end{altverse}

\begin{verse}
\textit{Reff:}\\
Tapi kumengerti s’karang Kau tolong padaku
\end{verse}

\subsection{ BAPA SUNGGUH BAIK}
\begin{altverse}
Bapa, Engkau sungguh baik\\
KasihMu melimpah di hidupku\\
Bapa, ku bert’rima kasih\\
BerkatMu hari ini, yang Kau sediakan bagiku
\end{altverse}

\begin{verse}
Kunaikkan syukurku buat hari yang Kau b’ri\\
Tak habis-habisnya kasih dan rahmatMu\\
Slalu baru dan tak pernah terlambat pertolonganMu\\
Besar setiaMu di s’panjang hidupku
\end{verse}

\subsection{ JALAN TUHAN}
\begin{altverse}
Ada waktu di hidupku\\
Pencobaan berat menekan\\
Aku berseru mengapa ya Tuhan\\
Nyatakan kehendakMu
\end{altverse}

\begin{verse}
Jalan Tuhan bukan jalanku\\
Jangan bimbang ataupun ragu\\
Nantikan tuhan jadikan semua\\
Indah pada waktunya
\end{verse}

\begin{altverse}
\textit{Reff:}\\
Pada Tuhan masa depanku\\
Pada Tuhan kus’rahkan hidupku\\
Nantikan Tuhan berkarya\\
Indah pada waktunya
\end{altverse}

\begin{verse}
Hari esok tiada kutahu\\
Namun tetap langkahku maju\\
Ku yakin Tuhan jadikan semua\\
Indah pada waktunya
\end{verse}

\subsection{ TANGAN TUHAN}
\begin{altverse}
Apa yang kau alami kini\\
Mungkin tak dapat engkau mengerti\\
Satu hal tanamkan di hati\\
Indah semua yang Tuhan beri
\end{altverse}

\begin{verse}
Tuhanku tak akan memberi\\
Ular beracun pada yang minta roti\\
Cobaan yang engkau alami\\
Tak melebihi kekuatanmu\\
\textit{Reff:}\\
Tangan Tuhan sedang merenda\\
Suatu karya yang agung mulia\\
Saatnya ‘kan tiba nanti\\
Kau lihat pelangi kasihNya
\end{verse}

\subsection{ HANYA KEPADAMU}
\begin{altverse}
Hanya kepadaMu kami dapat berpasrah\\
Tuhan Yesus Kristus\\
Di dalam tanganMu hidup dan mati kami\\
Engkaulah Penebus
\end{altverse}

\begin{verse}
CintaMu ya tuhan mengalahkan maut\\
Hanya Dikau pangkal kehidupan\\
Yang selalu kami dambakan
\end{verse}

\begin{altverse}
Ya Yesus sambutlah saudara kami ini\\
Dalam rumah Bapa\\
Karna Engkau wafat supaya kami hidup\\
Untuk selamanya
\end{altverse}

\begin{verse}
Hidup atau mati kami milik Tuhan\\
Maka Tuhan bimbinglah umatMu,\\ 
Di jalan menuju padaMu
\end{verse}

\subsection{ DOAKAN KAMI BUNDA}
\begin{altverse}
Cantik hatimu tiada bernoda\\
Bunda penolong umat manusia\\
Trimalah nyanyian puji bagimu\\
Kar’na kekagumanku pada iman dalam bening hatimu
\end{altverse}

\begin{verse}
Kau perawan suci nan lembut hati\\
Terberkati dalam rahmat Ilahi\\
Dengarkan selalu doaku O Bunda yang setia\\
Kaulah perantara doa kami pada Yesus PutraMu
\end{verse}

\begin{altverse}
\textit{Reff:}\\
Maria, Maria terpujilah engkau untuk selamanya\\
Doakan kami Bunda saat ini dan saat ajal nanti
\end{altverse}

\subsection{ BUNDA PEMBANTU ABADI}
\begin{altverse}
Pada wajahmu yang suci\\
Matamu nampak bening sejuk lembut\\
Kau pandang para abdimu berdoa\\
Oh Bunda pembantu abadi\\
Engkau pagku anakmu Yesus Putra Allah\\
Sumber suka dan duka hatimu\\
Hanya engkau sendirilah yang tahu\\
Pahit dan manisnya hidupku
\end{altverse}

\begin{verse}
Bukanlah kepadamu oh bunda\\
Pandangan penuh cintaNya tertuju\\
Salib dan tombak bengis dilihatNya\\
Oh Bunda pembantu abadi\\
Tangan Bunda dipegang didekapNya erat\\
Gambaran gelisah manusia\\
Bagaikan terbayang sengsara maut\\
Siksaan salah manusia
\end{verse}

\begin{altverse}
Matamu ya Bunda suci\\
Memberitakan pesanNyaterindah\\
Wahau kamu orang-orang berdosa\\
Lihatlah juru selamatmu\\
Terdengar pesan indah namun kami lemah\\
Terbawa gelombang masa kini\\
Kami pinta doamu pada Bapa\\
Oh Bunda Pembantu abadi
\end{altverse}

\begin{verse}
Pandanglah dunia ini oh bunda\\
Dunia yang penuh dengan kebencian\\
Doakan perdamaian yang sejati\\
Sadarkan hati manusia\\
Arahkan pikian tingkah laku kami\\
Biarkan tampak cinta sesame\\
Bantulah di saat ajalku tiba\\
Oh Bunda Pembantu abadi


\end{verse}


\end{document}

