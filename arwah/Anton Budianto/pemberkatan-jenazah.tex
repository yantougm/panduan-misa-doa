\documentclass[a5paper,headsepline,titlepage,11pt,nnormalheadings,DIVcalc]{scrbook}
\usepackage[a5paper,backref]{hyperref}
\usepackage[papersize={165mm,215mm},twoside,bindingoffset=0.5cm,hmargin={2cm,2cm},
				vmargin={2cm,2cm},footskip=1.1cm,driver=dvipdfm]{geometry}
\usepackage[latin1]{inputenc} 
\usepackage{calc}
\usepackage{setspace} 
\usepackage{fixltx2e} 
\usepackage{graphicx}
\usepackage{multicol} 
\usepackage[normalem]{ulem} 
\usepackage[bahasa]{babel} 
\usepackage{color}
\usepackage{hyperref} 
\usepackage{pstricks}
\usepackage{fancyhdr}
\usepackage{pst-node}

\setlength{\parindent}{0mm}
\makeatletter
\newcommand{\lagu}[1]{%
  {\parindent \z@ 
    \interlinepenalty\@M \slshape \mdseries \large \textit{#1}\par\nobreak \vskip 10\p@ }}
\newcommand{\keterangan}[1]{%
  {\parindent \z@ 
    \interlinepenalty\@M \slshape \mdseries \textit{#1}\par\nobreak \vskip 10\p@ }}
\makeatother

\renewcommand{\footrulewidth}{0.5pt}
\lhead[\fancyplain{}{\thepage}]%
      {\fancyplain{}{~}}
\rhead[\fancyplain{}{~}]%
      {\fancyplain{}{\thepage}}
\pagestyle{fancy}
\lfoot[\emph{Misa Pemberkatan}]{}
\rfoot[]{\emph{Antonius Budianto}}
\cfoot{}

\newcommand{\BU}[1]{\begin{itemize} \item[U:] #1 \end{itemize}}
\newcommand{\BI}[1]{\begin{itemize} \item[I:] #1 \end{itemize}}
\newcommand{\BIU}[1]{\begin{itemize} \item[I+U:] #1 \end{itemize}}
\newcommand{\BP}[1]{\begin{itemize} \item[P:] #1 \end{itemize}}
\newcommand{\inputlagu}[1]{\begin{textit} \input{#1} \end{textit}}
\newcommand{\namaalm}{Bapak Antonius Budianto~}
\newcommand{\namaromo}{Setiawan SJ}
\newcommand{\tanggal}{29 Mei 2010}

\hyphenation{sa-u-da-ra-ku}
\hyphenation{ke-ri-ngat}
\hyphenation{je-ri-tan}
\hyphenation{hu-bung-an}
\hyphenation{me-nya-dari}
\hyphenation{Eng-kau}
\hyphenation{ke-sa-lah-an}
\hyphenation{ba-gai-ma-na}
\hyphenation{Tu-han}
\hyphenation{di-per-ca-ya-kan}
\hyphenation{men-ja-uh-kan}
\hyphenation{bu-kan-lah}
\hyphenation{per-sa-tu-kan-lah}
\hyphenation{ma-khluk}
\hyphenation{Sem-buh-kan-lah}
\hyphenation{ja-lan}
\hyphenation{mem-bu-tuh-kan}
\hyphenation{be-ri-kan-lah}
\hyphenation{me-ra-sa-kan}
\hyphenation{te-man-ilah}
\hyphenation{mem-bi-ngung-kan}
\hyphenation{di-ka-gum-i}
\hyphenation{ta-ngis-an-Mu}
\hyphenation{mi-lik-ilah}

\begin{document}
\thispagestyle{empty}
\psset{unit=1in}
\begin{pspicture}(4in,6.0in)
% set up the fonts we use
\DeclareFixedFont{\PT}{T1}{ppl}{b}{it}{0.4in}
\DeclareFixedFont{\PTsmall}{T1}{ppl}{b}{it}{0.3in}
\DeclareFixedFont{\PTsmallest}{T1}{ppl}{b}{it}{0.2in}
\DeclareFixedFont{\PTtext}{T1}{ppl}{b}{it}{11pt}
\DeclareFixedFont{\Logo}{T1}{pbk}{m}{n}{0.2in}
% place the front cover picture
%\rput[cb](2.3,2.5){\usebox\IBox}
% put the text on the front cover
\rput[cb](2.5,5.3){\PTsmall {Misa Pemberkatan Jenazah}}
\rput[cb](2.5,4.8){\PTsmall {\namaalm}}
\rput[cb](2.5,1.1){\PTsmall {29 Mei 2010}}
\rput[cb](2.5,0.6){\PTsmallest {St Michael Panti Rapih}}
\rput[cb](2.5,0.3){\PTsmallest {Yogyakarta}}

%\rput[cb](3,-1){\PTsmallest {\namagereja}} 

\end{pspicture}
%\tableofcontents 
\newpage
%% \begin{center} 
%% MISA KUDUS PEMBERKATAN JENAZAH\\
%% \namaalm \\
%% \tanggal\\
%% \end{center} 

\section*{RITUS PEMBUKA}

\lagu{Lagu pembukaan}

\subsection*{Tanda salib dan salam} 

\BI{Dalam Nama Bapa dan Putera dan Roh Kudus}
\BU{Amin} 
\BI{Terpujilah Allah, Bapa Tuhan kita Yesus Kristus, Bapa yang 
penuh belaskasihan dan Allah sumber segala penghiburan, yang 
menghibur kita dalam segala penderitaan.}
\BU{Sekarang dan selama-lamanya}

\subsection*{Pengantar} 

\BI{Ibu-Bapak, saudara-saudariku sekalian, keluarga yang berduka 
yang terkasih dalam Kristus. Dengan rasa sedih hati kita mengadakan 
upacara perpisahan dengan \namaalm yang kita kasihi ini. Sebentar lagi 
kita akan menghantar dia ke tempat peristirahatannya yang terakhir. 
Namun kita tidak boleh putus asa seperti orang yang tidak 
mempunyai harapan. Sebab kita menaruh harapan kepada Kristus 
yang telah menghancurkan kekuasaan maut dengan kebangkitan-
Nya yang mulia. Maka marilah kita mendoakan keselamatan kekal 
baginya dan peneguhan iman bagi kita semua, terutama bagi keluarga 
yang ditinggalkan.} 

\subsection*{Pernyataan Tobat} 

\BI{Agar doa-doa yang kita panjatkan ke hadirat Allah yang 
Maharahim diterima, marilah kita terlebih dahulu menyesali dan 
memohon ampun dariNya atas segala dosa dan kesalahan yang telah 
kita lakukan.}

\BI{Ya Tuhan Yesus Kristus, Engkau mengalami kematian sebagai 
manusia, tetapi dibangkitkan oleh kekuasaan Bapa dalam Roh Kudus. 
Tuhan kasihanilah kami.} 
\BU{Tuhan, kasihanilah kami.} 
\BI{Engkaulah kebangkitan dan kehidupan; barang siapa percaya 
kepada-Mu akan memperoleh kehidupan yang kekal. Kristus, 
kasihanilah kami.} 
\BU{Kristus, kasihanilah kami} 
\BI{Engkau akan datang dengan mulia untuk mempersatukan kami 
semua dalam kerajaan Bapa. Tuhan kasihanilah kami.} 
\BU{Tuhan, kasihanilah kami.} 
\BI{Semoga Allah yang Mahakuasa mengasihani kita, mengampuni 
dosa-dosa kita dan menghantar kita ke hidup yang kekal.}
\BU{Amin}

\BI{Marilah Berdoa: 

Allah, Bapa kami yang Maharahim dengan penuh harapan kami 
menyerahkan hambaMu ini \namaalm ke tanganMu. 
Berikanlah dia istirahat penuh terang dan damai dalam kerajaanMu. 
Ampunilah ya Bapa, segala dosa dan kesalahannya. Semoga ia 
bersatu dengan para hambaMu untuk menikmati cahaya kebahagiaan 
kekal dan memuji kebaikanMu. 
Kami juga berdoa bagi kami semua yang masih berziarah di 
bumi ini, terutama bagi keluarga yang sedang berkabung. Engkaulah, 
ya Bapa sumber belaskasihan dan penghiburan, Engkau mencintai 
kami dengan kasih abadi; Engkau mengubah maut yang gelap gulita 
menjadi fajar yang gilang-gemilang berkat kebangkitan mulia 
PuteraMu. Maka kami mohon, semoga keluarga yang sedang 
berkabung ini dapat menanggung duka-citanya dengan tabah dan 
menaruh kepercayaan serta harapannya kepadaMu. 

Demi Yesus Kristus, Puteramu, Tuhan dan pengantara kami, 
yang hidup dan berkuasa bersama dengan Dikau dalam persekutuan 
dengan Roh Kudus, kini dan sepanjang segala masa. }

\BU{Amin}

\section*{LITURGI SABDA}

\subsection*{Bacaan I}

\BU{\emph{Apakah yang dapat memisahkan kita dari cinta kasih Kristus}

Pembacaan dari surat Rasul Paulus kepada umat di Roma 
(8:31b-35,37-39). 

Jika Allah di pihak kita, siapakah yang akan melawan kita? 
Ia, yang tidak menyayangkan Anak-Nya sendiri, tetapi yang 
menyerahkanNya bagi kita semua, bagaimanakah mungkin Ia tidak 
mengaruniakan segala sesuatu kepada kita bersama-sama dengan 
Dia? Siapakah yang akan menggugat orang-orang pilihan Allah? 
Allah, yang membenarkan mereka? Siapakah yang akan menghukum 
mereka? 

Kristus Yesus, yang telah mati? Bahkan lebih lagi: 
yang telah bangkit, yang juga duduk di sebelah kanan Allah, yang 
malah menjadi Pembela bagi kita?
Siapakah yang akan 
memisahkan kita dari kasih Kristus? Penindasan atau kesesakan atau 
penganiayaan, atau kelaparan atau ketelanjangan , atau bahaya, atau 
pedang? 

Tetapi dalam semuanya itu kita lebih daripada orang-
orang yang menang baik malaikat-malaikat, maupun pemerintah-pemerintah, 
baik yang ada sekarang, maupun yang akan datang, atau kuasa-kuasa, baik yang di atas, maupun yang di bawah, ataupun 
sesuatu makhluk lain, tidak akan dapat memisahkan kita dari kasih 
Allah, yang ada dalam Kristus Yesus, Tuhan kita. 

Demikian Sabda Tuhan}

\BU{Amin}

\lagu{Lagu pengantar Injil} 

\subsection*{Bacaan Injil: (Lukas 23-33,39-43)} 

\BI{Tuhan sertamu} 
\BU{Dan sertamu juga} 
\BI{Inilah Injil Suci Yesus Kristus menurut Santo Lukas} 
\BU{Dimuliakanlah Tuhan} 
\BI{\emph{Hari ini juga engkau akan ada bersama-sama dengan aku di dalam firdaus}

Ketika mereka sampai di tempat yang bernama Tengkorak, 
mereka menyalibkan Yesus di situ dan juga kedua orang penjahat itu, 
yang seorang di sebelah kanan-Nya dan yang lain di sebelah kiri-
Nya.

Seorang dari penjahat yang di gantung itu menghujat 
Yesus, katanya: "Bukankah Engkau adalah Kristus? Selamatkanlah 
diri-Mu dan kami! Tetapi yang seorang menegor dia, katanya: 
"Tidakkah engkau takut, juga tidak kepada Allah, sedang engkau 
menerima hukuman yang sama? Kita memang selayaknya 
dihukum, sebab kita menerima balasan yang setimpal dengan 
perbuatan kita, tetapi orang ini tidak berbuat sesuatu yang salah." 

Lalu ia berkata kepada Yesus: "Yesus, ingatlah akan aku, 
apabila Engkau datang sebagai Raja." Kata Yesus kepadanya: 
"Aku berkata kepadamu, sesungguhnya hari ini juga engkau akan ada 
bersama-sama dengan Aku di dalam Firdaus"

Demikianlah Injil Tuhan}

\BU{Terpujilah Kristus}

\subsection*{Homili}
 

\subsection*{Doa Umat}

\BI{Ya Allah yang Maha murah, Engkau adalah sumber kehidupan 
kami yang selalu setia menuntun segala langkah kehidupan kami di 
kala suka maupun duka. Kami percaya, bersama dengan Dikau kami 
mampu untuk membuat hidup ini menjadi makin bermakna. Karena 
itu ya Tuhan, sudilah Engkau mendengarkan doa-doa umatMu yang 
berhimpun di sini.}

\BP{Allah, Bapa kami di dalam surga, kepadaMu kami menyerahkan 
\namaalm yang kami kasihi ini. Engkau telah menciptakan dan 
menempatkan dia di dunia ini untuk mengabdi kepadaMu dan 
sesamanya. Engkaulah yang memanggil dia. Semoga ia pantas 
bertemu dengan Dikau dan berbahagia bersama Engkau di surga. 

Marilah kita memohon}

\BU{Kabulkanlah doa kami, ya Tuhan}
 
\BP{Semoga Engkau dengan murah hati mengasihani dia yang telah 
kembali kepadamu. 

Marilah kita memohon}

\BU{Kabulkanlah doa kami, ya Tuhan.}
 
\BP{Semoga ia yang telah Kau murnikan dengan air permandian, 
tetap Kau sucikan pula dengan kemurahan belaskasihanMu. 

Marilah kita memohon}

\BU{Kabulkanlah doa kami, ya Tuhan.}

\BP{Semoga ia yang selama hidupnya memperjuangkan kebenaran, 
keadilan, cinta kasih dan damai, Kau beri pahala di surga. 

Marilah kita memohon}

\BU{Kabulkanlah doa kami, ya Tuhan.}
 
\BP{Semoga di akhirat, ia Kau beri tempat yang tenteram dan terang 
bersama dengan orang-orang kudusMu. 

Marilah kita memohon} 

\BU{Kabulkanlah doa kami, ya Tuhan.}

\BP{Semoga kami yang bersedih hati atas kematiannya, terutama 
keluarga yang berduka, Kau hibur dengan harapan akan persatuan 
kelak di surga. 

Marilah kita memohon}

\BU{Kabulkanlah doa kami, ya Tuhan.}
 
\BI{Allah yang kekal dan kuasa, Engkau senantiasa menyayangi 
umatMu. Engkaulah kebahagiaan bagi semua orang yang meninggal 
dalam Dikau. Semoga dalam kerahimanMu beristirahatlah 
\namaalm yang kami kasihi. Kabulkanlah dengan rela doadoa 
kami ini. Demi Kristus, Tuhan dan Pengantara kami.}

\BU{Amin} 

\section*{LITURGI EKARISTI}

\lagu{Lagu Persembahan}

\subsection*{Persiapan Persembahan}

\BI{Terpujilah Engkau ya Tuhan, Allah semesta alam, sebab dari kemurahanMu, kami menerima roti yang kami siapkan ini. Inilah hasil dari bumi dan dari usaha manusia yang bagi kami akan menjadi roti kehidupan.}

\BU{Terpujilah Allah selama-lamanya}

\BI{Terpujilah Engkau ya Tuhan, Allah semesta alam, sebab dari kemurahanMu, kami menerima anggur yang kami siapkan ini. Inilah hasil dari pohon anggur dan dari usaha manusia yang bagi kami akan menjadi minuman rohani.}

\BU{Terpujilah Allah selama-lamanya}

\BI{Berdoalah saudara-saudari, supaya persembahanku dan persembahanmu berkenan pada Allah Bapa yang Mahakuasa.}

\BU{Semoga persembahan ini diterima demi kemuliaan Tuhan dan keselamatan kita serta seluruh umat Allah yang kudus.}

\subsection*{Doa Persiapan Persembahan}

\BI{Allah Bapa sumber kekudusan, terimalah bahan persembahan roti dan anggur yang kami hunjukkan untuk keselamatan hambaMu \namaalm. Perkenankanlah dia memasuki rumahMu di surga, dan anugerahkan kami semua ketekunan dalam pengharapan. Demi Kristus Tuhan dan pengantara kami.}

\BU{Amin.}


\subsection*{Doa Syukur Agung}

\lagu{Kudus}

\lagu{Anak Domba Allah}

\subsection*{Ajakan menyambut komuni}

\BI{Saudara-saudari terkasih, Tuhan Yesus bersabda; "Datanglah kepadaKu, kalian yang telah memikul beban berat, maka Aku akan memberikan rasa lega kepadamu." Beban dosa kitapun akan dihapus. Maka berbahagialah kita yang diundang ke perjamuan Tuhan.}

\BU{Ya Tuhan, saya tidak pantas Tuhan datang pada saya, tetapi bersabdalah saja, maka saya akan sembuh.}

\lagu{Lagu Komuni}

\subsection*{Doa sesudah komuni}

\BI{Marilah berdoa,

Allah Bapa sumber kehidupan, kami telah Engkau segarkan dengan santapan kehidupan, Tubuh dan Darah Yesus PutraMu. KehadiranNya memberi keteguhan bagi hidup kami sehari-hari dalam berusaha memenuhi kehendakMu melalui pelayanan da  pekerjaan kami. Semoga hambaMu \namaalm yang telah satu tahun menghadapMu kini berbahagia bersamaMu dan persatukanlah kami kelak dengannya serta para kudus di surga. Dengan perantaraan Kristus Tuhan kami,}

\BU{Amin.}


\section*{Pemberkatan Jenazah} 

\BI{Ibu-Bapak, saudara-saudariku sekalian yang terkasih dalam 
Kristus Tuhan. Sebentar lagi kita akan berpisah secara jasmani 
dengan \namaalm yang kita kasihi ini. Maka dengan hati yang 
tabah, kita memberikan penghormatan yang terakhir kepadanya, 
karena kita berharap bahwa ia akan bangkit bersama Kristus yang 
telah diimaninya untuk kehidupan yang kekal. Maka air suci akan 
direciki di atas dia sebagai lambang pembaptisannya dan jenazahnya 
akan didupai, supaya keharuman amal baktinya di dunia ini berkenan 
kepada Tuhan. }
\BI{Ya Tuhan, siramilah hambaMu ini \namaalm yang masuk ke 
alam kekal, dengan air kehidupan.} 
\BU{Terimalah dia, ya Tuhan.} 
\BI{Supaya ia hidup bahagia selamanya bersama para kudusMu dalam 
kerajaan surga.} 
\BU{Terimalah dia, ya Tuhan.} 
\BI{Dari bumi aku berseru kepadaMu, ya Tuhan. Sudilah Engkau 
mendengarkan seruanku.} 
\BU{Terimalah dia, ya Tuhan.} 
\BI{PadaMu, ya Tuhan, ada pengampunan dosa agar semua orang 
mengabdi kepadaMu dengan hati yang gembira.} 
\BU{Terimalah dia, ya Tuhan.} 
\BI{Aku percaya kepadaMu, ya Tuhan; jiwaku percaya akan 
SabdaMu.}
\BU{Terimalah dia, ya Tuhan.}
\BI{Pada-Mu, ya Tuhan, ada belaskasihan serta penebusan berlimpah 
ada pada-Mu}
\BU{Terimalah dia, ya Tuhan} 
\BI{Hai para kudus dan para malaikat Allah, datanglah menyongsong 
\namaalm ini dan hantarkanlah dia kepada Kristus.} 
\BU{Di hadapan Allah yang Mahatinggi.}
\BI{Semoga Kristus menyambutmu, sebab Dia-lah yang telah 
memanggil engkau. Semoga para malaikat mengiringi dan 
menjemput engkau ke pangkuan Abraham.}
\BU{Di hadapan Allah yang Mahatinggi}
\BI{Tuhan, berilah dia istirahat kekal dan sinarilah dia dengan cahaya 
abadi}
\BU{Di hadapan Allah yang Mahatinggi}
\BI{Marilah berdoa: 

Ya Tuhan, kehidupan dan kematian kami berada di dalam tanganMu 
sendiri. Engkau telah menciptakan manusia karena kasih dan 
cintaMu. Ya Tuhan, lihatlah kami putera-puteri-Mu yang kini 
berhimpun di sekitar peti jenazah dari \namaalm yang kami 
kasihi ini. Kami semua berduka cita atas kematiannya. Maafkan kami 
ya Tuhan, jika kami belum sempat mengucapkan terima kasih kami 
kepadanya atas segala kebaikan yang telah dilakukannya kepada 
kami. Kami juga menyesal jika kami belum sempat meminta maaf 
atas segala dosa dan kesalahan yang telah kami lakukan terhadapnya. 
Akan tetapi kami percaya bahwa kasihMu jauh lebih kuat daripada 
keinginan manusiawi kami. Kami mohon berkat belaskasihan-Mu 
kepada \namaalm , janganlah Engkau serahkan dia 
kepada kekuasaan maut, tetapi bebaskanlah dia demi jasa Kristus, 
Putera-Mu. Biarkanlah darah dan air yang tercurah dari lambung 
Kristus, PuteraMu itu membersihkan dia dari segala dosa dan 
kesalahannya sehingga ia, dengan jiwa yang bersih dapat menghadap 
Engkau, Penciptanya, Bapanya dan juga Bapa kami bersama. Tuhan, 
dalam kehidupan di dunia ini, ia telah dikuatkan dan disegarkan 
dengan santapan Tubuh dan Darah Kristus Putera-Mu. Maka kami 
mohon, perkenankanlah ia kini mengambil bagian dalam perjamuan 
surgawi-Mu. Disanalah kami semua akan dipertemukan kembali 
untuk memuji dan memuliakan Dikau dalam keabadian. Demi 
Kristus Tuhan dan Pengantara kami. }

\BU{Amin.}
\BI{\namaalm , terima kasih atas segala kebaikan, jasa dan 
pengorbanan hidupmu yang telah engkau perbuat selagi masih hidup 
bersama dengan kami. Atas segala tanda kasihmu itu, kami hanya 
sanggup mengucapkan terima kasih dan selamat jalan, bawalah selalu 
tanda kemenangan Kristus: Bapa dan (+) Putera Roh 
Kudus.}
\BU{Amin.} 


\keterangan{(Lalu jenazah direciki dengan air suci dan didupai. Keluarga 
diperkenankan untuk menyirami jenazah dengan minyak wangi yang 
sudah diberkati. Sebaiknya diiringi dengan lagu yang sesuai)}


\section*{RITUS PENUTUP} 

\BI{Marilah berdoa: 

Allah dan Bapa kami yang Maha baik, kami mempercayakan 
\namaalm yang kami kasihi ini kepada kerahimanMu. Kami 
percaya, bahwa semua orang yang meninggal dalam Kristus akan 
hidup bersama Kristus. Kepercayaan ini memberi kami harapan dan 
menabahkan hati kami dalam kesusahan. Dengarkanlah doa umatMu 
ini dan bukalah pintu surga bagi dia yang sudah Engkau panggil. 
Semoga kami yang masih berziarah di bumi ini, terutama semua 
anggota keluarga yang ditinggalkannya selalu saling membantu 
dalam segala tantangan hidup ini; dan semoga kami tetap menaruh 
kepercayaan yang teguh akan sabda-Mu, bahwa kami juga akan 
menyongsong Kristus untuk bersatu dengan Dia selama-lamanya. 
Sebab Dialah PuteraMu, Tuhan dan Pengantara kami kini dan 
sepanjang segala masa.}

\BU{Amin}

\subsection*{Berkat} 

\BI{Tuhan sertamu}
\BU{Dan sertamu juga} 
\BI{Semoga kita sekalian, semua yang kita doakan dan perjalanan kita 
menghantar \namaalm ke tempat peristirahatannya yang 
terakhir, senantiasa dilindungi, dibimbing dan diberkati oleh Allah 
yang Mahakuasa: Bapa dan (+) Putera dan Roh Kudus} 
\BU{Amin.}
\BI{Saudara-saudariku sekalian, marilah kita berangkat ke pekuburan 
untuk menghantar \namaalm yang kita kasihi ini ke tempat 
istirahatnya yang terakhir. Semoga damai Tuhan menyertai kita.} 
\BU{Sekarang dan selama-lamanya}

\subsection*{Lagu Penutup} 


\begin{center}
TERIMA KASIH ATAS DUKUNGAN DOA DARI\\
BAPAK, IBU, DAN SAUDARA\\
SEMOGA KEBAIKAN BAPAK, IBU, DAN SAUDARA\\
TUHAN BERKENAN MEMBERKATI
\end{center}
\end{document}
