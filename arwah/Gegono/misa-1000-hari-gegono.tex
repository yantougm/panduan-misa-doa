\documentclass[a5paper,titlepage,10pt,openany]{scrbook}
\usepackage[a5paper,backref]{hyperref}
\usepackage[papersize={148.5mm,215mm},twoside,bindingoffset=0.5cm,hmargin={2cm,2cm},
				vmargin={2cm,2cm},footskip=1.1cm,driver=dvipdfm]{geometry}
%\usepackage{palatino}
\usepackage{graphicx}
\usepackage{wrapfig}
\usepackage[bahasa]{babel}
\usepackage{fancyhdr}
\usepackage{pst-text}

\renewcommand{\footrulewidth}{0.5pt}
\lhead[\fancyplain{}{\thepage}]%
      {\fancyplain{}{\rightmark}}
\rhead[\fancyplain{}{\leftmark}]%
      {\fancyplain{}{\thepage}}
\pagestyle{fancy}
\lfoot[\emph{Misa peringatan 1000 hari \namaalm}]{}
\rfoot[]{\emph{Misa peringatan 1000 hari \namaalm}}
\cfoot{}

\makeatletter
\newcommand{\judul}[1]{%
  {\parindent \z@ \centering 
    \interlinepenalty\@M \Large \bfseries #1\par\nobreak \vskip 20\p@ }}
\newcommand{\subjudul}[1]{%
  {\parindent \z@ 
    \interlinepenalty\@M \bfseries #1\par\nobreak \vskip 10\p@ }}
\newcommand{\lagu}[1]{%
  {\parindent \z@ 
    \interlinepenalty\@M \slshape \bfseries \normalsize \textit{#1}\par\nobreak \vskip 10\p@ }}
\newcommand{\keterangan}[1]{%
  {\parindent \z@  \slshape 
    \interlinepenalty\@M \textsl{#1}\par\nobreak  \vskip 5\p@}}

\renewenvironment{description}
               {\list{}{\labelwidth\z@ \itemindent-\leftmargin
                        \let\makelabel\descriptionlabel}}
               {\endlist}
\renewcommand*\descriptionlabel[1]{\hspace\labelsep 
                                \normalfont\bfseries #1 }


\makeatother

\newcommand{\BU}[1]{\begin{itemize} \item[U:] #1 \end{itemize}}
\newcommand{\BI}[1]{\begin{itemize} \item[I:] #1 \end{itemize}}
\newcommand{\BIU}[1]{\begin{itemize} \item[I+U:] #1 \end{itemize}}
\newcommand{\BP}[1]{\begin{itemize} \item[P:] #1 \end{itemize}}
\newcommand{\inputlagu}[1]{\itshape{\input{#1}}}
\newcommand{\namaalm}{Bpk Yohanes Markus Gegono Purwadi }
\newcommand{\namaromo}{H Asodo OMI}

\title{Misa Peringatan 1000 Hari}
\author{\namaalm}
\date{oleh \\ Rm \namaromo\\19 Desember 2011}
\hyphenation{sa-u-da-ra-ku}
\hyphenation{ke-ri-ngat}
\hyphenation{je-ri-tan}
\hyphenation{hu-bung-an}
\hyphenation{me-nya-dari}
\hyphenation{Eng-kau}
\hyphenation{ke-sa-lah-an}
\hyphenation{ba-gai-ma-na}
\hyphenation{Tu-han}
\hyphenation{di-per-ca-ya-kan}
\hyphenation{men-ja-uh-kan}
\hyphenation{bu-kan-lah}
\hyphenation{per-sa-tu-kan-lah}
\hyphenation{ma-khluk}
\hyphenation{Sem-buh-kan-lah}
\hyphenation{ja-lan}
\hyphenation{mem-bu-tuh-kan}
\hyphenation{be-ri-kan-lah}
\hyphenation{me-ra-sa-kan}
\hyphenation{te-man-ilah}
\hyphenation{mem-bi-ngung-kan}
\hyphenation{di-ka-gum-i}
\hyphenation{ta-ngis-an-Mu}
\hyphenation{mi-lik-ilah}

\DeclareFixedFont{\PT}{T1}{ppl}{b}{it}{0.5in}
\DeclareFixedFont{\PTsmall}{T1}{ppl}{b}{it}{0.3in}
\DeclareFixedFont{\PTsmallest}{T1}{ppl}{b}{it}{0.2in}
\DeclareFixedFont{\PTtext}{T1}{ppl}{b}{it}{11pt}
\DeclareFixedFont{\Logo}{T1}{pbk}{m}{n}{0.3in}

\begin{document}
%\maketitle
\thispagestyle{empty}
\begin{pspicture}(14cm,16cm)
\psset{unit=1cm}
\rput[cb](5,14){\PTsmall {EKARISTI}}
\rput[cb](5,13){\PTsmall {PERINGATAN 1000 HARI}}
\rput[cb](5,10){\PTsmall {\namaalm}}
\rput[cb](5,3){\PTsmallest {oleh}} 
\rput[cb](5,2){\PTsmallest {Rm \namaromo}}
\rput[cb](5,1){\PTsmallest {19 Desember 2011}}
\end{pspicture}

\newpage
\thispagestyle{empty}
{~}
\newpage

\section*{RITUS PEMBUKA} 

 

\lagu{Lagu Pembukaan}  
\begin{center}
\itshape{Tuhan Beserta Kita}
\end{center}

\small
\begin{verse}
\itshape{
Pabila hatimu dalam duka karna saudara berpulang\\
Ingatlah Tuhanmu yang setia Dia selalu beserta kita\\
Allah pengasih sumber harapan tempat curahan suka duka\\
Allah perkasa jadi lindungan dalam Dia amanlah kita.\\
{~}\\
Di atas dunia tiada daya satu pun takkan kuasa\\
memisahkan Tuhan dari kita karna kasihNya yg menyatukan\\
Jikalau Tuhan bersama kita duka cita berganti tawa\\
Cinta kasihNya tetap bernyala kuasa gelap hilang dan musna.\\
}
\end{verse}
\normalsize 

\subjudul{Tanda Salib} 

\BI{Demi nama  Bapa dan Putera dan Roh Kudus}

\BU{Amin}

 

\subjudul{Salam}

\BI{Semoga kasih karunia, rahmat dan damai sejahtera dari 
Allah Bapa dan dari PuteraNya Yesus Kristus beserta 
saudara.} 

\BU{Sekarang dan selama-lamanya.}

 

\subjudul{Pengantar}

\BI{Terpujilah Allah Bapa di surga: Ia yang memiliki, Ia yang 
memberi dan memelihara, Ia pula yang mengambilnya 
kembali. Terpujilah Allah Bapa di surga.

\namaalm adalah milik Bapa di surga. Karena kasihNya 
kepada kita semua, kita telah menikmati kehadirannya.
KepadaNya pulalah dia telah 
kembali.

Kini kita bersama-
sama berdoa menghadap Allah Bapa di surga untuk 
bersyukur atas kehadirannya, atas teladan kehidupannya 
dan memohon berkat Allah untuk arwahnya 
agar supaya Allah Bapa berkenan mengampuni dosa-dosanya 
dan menerimanya dalam rumah abadi dalam 
damai dan kemuliaan Allah Bapa di surga. 

Kita juga memohon kepada Allah Bapa untuk berkatNya 
agar kita dapat meneruskan kebiasaan baik dari \namaalm , 
terutama dalam kehidupan spiritualitas dan 
sosialitas kita.}

 

\subjudul{Tobat}

\BI{Saudara-saudara, menyadari bahwa kita hanyalah serupa 
debu bernoda di depan alas kaki Allah Bapa, marilah kita 
bersyukur bahwa kita masih diperkenankan berdoa dan 
bermohon kepada Allah Bapa. }

 

\lagu{Tuhan Kasihanilah kami - MB 184 Misa Dolo-dolo}


\BI{Semoga Allah Bapa yang Maha Kuasa, mengasihani kita, 
mengampuni dosa kita dan mengantar kita ke dalam 
kehidupan kekal.}

\BU{Amin}

 

\subjudul{Doa Pembuka}

\BI{Marilah kita berdoa 

Allah Bapa di surga, segala kelemahan dan dosa kami 
terbentang di hadapan Engkau. Karena dosa-dosa itu 
pula, Engkau telah mengutus PuteraMu sendiri Tuhan 
kami Yesus Kristus, untuk datang dan menyelamatkan 
kami dan menyiapkan tempat bagi kami dalam rumah 
kekalMu. Ialah jalan dan kehidupan kami. 

Kami memohon ampunanMu untuk semua dosa-dosa kami 
dan terlebih untuk dosa-dosa \namaalm dan juga dosa-dosa para leluhur kami. 

Semua ini kami mohon demi Yesus Kristus PuteraMu dan 
pengantara kami yang bersatu dengan Dikau dan Roh 
Kudus, hidup dan berkuasa kini dan sepanjang masa.}

\BU{Amin}

 

\section*{LITURGI SABDA} 

\keterangan{Bacaan dari Dan 6 : 12 - 28}

\BP{Kemudian mereka menghadap raja dan menanyakan kepadanya tentang larangan raja: "Bukankah tuanku mengeluarkan suatu larangan, supaya setiap orang yang dalam tiga puluh hari menyampaikan permohonan kepada salah satu dewa atau manusia kecuali kepada tuanku, ya raja, akan dilemparkan ke dalam gua singa?" Jawab raja: "Perkara ini telah pasti menurut undang-undang orang Media dan Persia, yang tidak dapat dicabut kembali."

Lalu kata mereka kepada raja: "Daniel, salah seorang buangan dari Yehuda, tidak mengindahkan tuanku, ya raja, dan tidak mengindahkan larangan yang tuanku keluarkan, tetapi tiga kali sehari ia mengucapkan doanya."

Setelah raja mendengar hal itu, maka sangat sedihlah ia, dan ia mencari jalan untuk melepaskan Daniel, bahkan sampai matahari masuk, ia masih berusaha untuk menolongnya.

Lalu bergegas-gegaslah orang-orang itu menghadap raja serta berkata kepadanya: "Ketahuilah, ya raja, bahwa menurut undang-undang orang Media dan Persia tidak ada larangan atau penetapan yang dikeluarkan raja yang dapat diubah!"

Sesudah itu raja memberi perintah, lalu diambillah Daniel dan dilemparkan ke dalam gua singa. Berbicaralah raja kepada Daniel: "Allahmu yang kausembah dengan tekun, Dialah kiranya yang melepaskan engkau!"

Maka dibawalah sebuah batu dan diletakkan pada mulut gua itu, lalu raja mencap itu dengan cincin meterainya dan dengan cincin meterai para pembesarnya, supaya dalam hal Daniel tidak dibuat perubahan apa-apa.

Lalu pergilah raja ke istananya dan berpuasalah ia semalam-malaman itu; ia tidak menyuruh datang penghibur-penghibur, dan ia tidak dapat tidur.

Pagi-pagi sekali ketika fajar menyingsing, bangunlah raja dan pergi dengan buru-buru ke gua singa;
dan ketika ia sampai dekat gua itu, berserulah ia kepada Daniel dengan suara yang sayu. Berkatalah ia kepada Daniel: "Daniel, hamba Allah yang hidup, Allahmu yang kausembah dengan tekun, telah sanggupkah Ia melepaskan engkau dari singa-singa itu?"

Lalu kata Daniel kepada raja: "Ya raja, kekallah hidupmu!
Allahku telah mengutus malaikat-Nya untuk mengatupkan mulut singa-singa itu, sehingga mereka tidak mengapa-apakan aku, karena ternyata aku tak bersalah di hadapan-Nya; tetapi juga terhadap tuanku, ya raja, aku tidak melakukan kejahatan."

Lalu sangat sukacitalah raja dan ia memberi perintah, supaya Daniel ditarik dari dalam gua itu. Maka ditariklah Daniel dari dalam gua itu, dan tidak terdapat luka apa-apa padanya, karena ia percaya kepada Allahnya.

Raja memberi perintah, lalu diambillah orang-orang yang telah menuduh Daniel dan mereka dilemparkan ke dalam gua singa, baik mereka maupun anak-anak dan isteri-isteri mereka. Belum lagi mereka sampai ke dasar gua itu, singa-singa itu telah menerkam mereka, bahkan meremukkan tulang-tulang mereka.

Kemudian raja Darius mengirim surat kepada orang-orang dari segala bangsa, suku bangsa dan bahasa, yang mendiami seluruh bumi, bunyinya: "Bertambah-tambahlah kiranya kesejahteraanmu!
Bersama ini kuberikan perintah, bahwa di seluruh kerajaan yang kukuasai orang harus takut dan gentar kepada Allahnya Daniel, sebab Dialah Allah yang hidup, yang kekal untuk selama-lamanya; pemerintahan-Nya tidak akan binasa dan kekuasaan-Nya tidak akan berakhir.
Dia melepaskan dan menolong, dan mengadakan tanda dan mujizat di langit dan di bumi, Dia yang telah melepaskan Daniel dari cengkaman singa-singa."

Dan Daniel ini mempunyai kedudukan tinggi pada zaman pemerintahan Darius dan pada zaman pemerintahan Koresh, orang Persia itu.

Demikianlah Sabda Tuhan.}

\BU{Syukur kepada Allah}

 

\lagu{Lagu}
\begin{center}
\itshape{Kurenungkan SabdaMu}
\end{center}

\small
\begin{verse}
\itshape{
Ref:\\
Kurenungkan kuresapkan sabdaMu ya Tuhan\\
Dan pimpinlah kami Tuhan hidup seturut sabdaMu\\
{~}\\
Tuhan cairkanlah hatiku yang beku dan kaku
}
\end{verse}
\normalsize 

\subjudul{Injil Luk 21:20-28}

\BI{Tuhan sertamu}

\BU{dan sertamu juga} 

\BI{Inilah Injil Yesus Kristus menurut Lukas}

\BU{Dimuliakanlah Tuhan}

\BI{"Apabila kamu melihat Yerusalem dikepung oleh tentara-tentara, ketahuilah, bahwa keruntuhannya sudah dekat.

Pada waktu itu orang-orang yang berada di Yudea harus melarikan diri ke pegunungan, dan orang-orang yang berada di dalam kota harus mengungsi, dan orang-orang yang berada di pedusunan jangan masuk lagi ke dalam kota, sebab itulah masa pembalasan di mana akan genap semua yang ada tertulis.

Celakalah ibu-ibu yang sedang hamil atau yang menyusukan bayi pada masa itu! Sebab akan datang kesesakan yang dahsyat atas seluruh negeri dan murka atas bangsa ini,
dan mereka akan tewas oleh mata pedang dan dibawa sebagai tawanan ke segala bangsa, dan Yerusalem akan diinjak-injak oleh bangsa-bangsa yang tidak mengenal Allah, sampai genaplah zaman bangsa-bangsa itu."

"Dan akan ada tanda-tanda pada matahari dan bulan dan bintang-bintang, dan di bumi bangsa-bangsa akan takut dan bingung menghadapi deru dan gelora laut.
Orang akan mati ketakutan karena kecemasan berhubung dengan segala apa yang menimpa bumi ini, sebab kuasa-kuasa langit akan goncang.

Pada waktu itu orang akan melihat Anak Manusia datang dalam awan dengan segala kekuasaan dan kemuliaan-Nya.
Apabila semuanya itu mulai terjadi, bangkitlah dan angkatlah mukamu, sebab penyelamatanmu sudah dekat."}

\subjudul{Aklamasi}

\BI{Demikianlah Injil Tuhan}

\BU{Terpujilah Kristus}

 

\subjudul{Homili}

\subjudul{Syahadat} 

\subjudul{Doa Umat}

\BI{Terpujilah Engkau ya Allah Bapa di surga karena besarlah kuasa 
dan kasihMu. Kami menghaturkan puji dan sembah atas segala 
kurnia yang telah Engkau limpahkan kepada kami: atas keluarga 
kami; atas rumah tempat tinggal kami; atas segala sesuatu yang 
telah kami terima dan nikmati mulai dari doa, teladan, seluruh 
kebutuhan hidup dan pendidikan yang Engkau berikan melalui 
orang-orang yang mengasihi kami.}

\BP{Untuk berkatMu bagi pemurnian diri kami sekeluarga dan 
bagi semua yang menyatu dengan kami sekeluarga; serta 
bagi keselamatan dan kelancaran usaha kami dalam mencari 
nafkah secara jujur; 

Marilah kita mohon,}

\BU{kabulkanlah doa kami ya Tuhan.} 

\BP{Untuk berkatMu bagi arwah \namaalm; bagi arwah saudara 
dan handai taulan; serta bagi arwah para leluhur. Semoga 
Engkau berkenan mengampuni dosa-dosa mereka dan 
memberi mereka karunia kebahagiaan abadi dalam rumahMu 
yang kudus. Marilah kita mohon,}

\BU{kabulkanlah doa kami ya Tuhan.} 

\BP{Untuk kehidupan kami; Allah Bapa di surga, Engkaulah 
sumber hidup, tuntunan dan keselamatan. Setiap orang 
mungkin bisa memperdaya dan meninggalkan kami, namun 
hanya Engkau sajalah yang tidak akan pernah memperdaya 
kami. Engkau menunjukkan kepada kami betapa kasihMu itu 
suci dan sejati; Engkau membimbing kami menuju kebebasan 
sejati; Engkau membimbing jalan kami. KepadaMu kami 
mohon agar kami Kauperkenankan kembali kepadaMu: 
harapan dan kebebasan jiwa kami, kebenaran dan 
kegembiraan batin kami. Kami mohon janganlah biarkan 
kami jauh dari padaMu ya Allah. Marilah kita mohon,}

\BU{kabulkanlah doa kami ya Tuhan.} 

\BP{Untuk keluarga kami. Allah Bapa di surga, keluarga adalah 
kurniaMu yang Kaupercayakan kepada manusia; keluarga 
adalah percikan dari surga yang dibagikan kepada semua 
manusia: keluarga adalah buaian di mana kami dilahirkan dan 
yang kami terus-menerus dilahirkan kembali dalam cinta. 
Allah Bapa di surga, kami mohon masuklah ke dalam rumah-
rumah kami dan pimpinlah kami dalam nyanyian kehidupan. 
Kami mohon perbaharuilah cahaya cinta dan buatlah kami 
merasakan keindahan menjadi terikat satu dengan yang 
lainnya dalam sebuah rangkulan kehidupan: sebuah 
kehidupan yang dihangatkan oleh nafas Allah sendiri, nafas 
dari Allah yang adalah Cinta. Ya Allah Bapa di surga, mohon 
selamatkanlah keluarga kami; lindungilah keluarga kami dari 
fitnah dan mara bahaya dan selamatkanlah hidup itu sendiri. 
Marilah kita mohon,}

\BU{kabulkanlah doa kami ya Tuhan.} 

\BP{Untuk semua orang. Ya Allah Bapa di surga, kami mohonkan 
pula berkatMu bagi semua orang yang memerlukan dan 
merindukan berkatMu terutama bagi mereka yang miskin, 
sakit dan lapar dan bagi mereka yang sedang berada dalam 
kesulitan dan beban berat; Marilah kita mohon,}

\BU{kabulkanlah doa kami ya Tuhan.} 

\section*{LITURGI EKARISTI}

\lagu{Lagu Persembahan}
\begin{center}
\itshape{Ku Persembahkan}
\end{center}

\small
\begin{verse}
\itshape{
Ku persembahkan bagi-Mu Yesus\\
yang terbaik dan terindah.\\
Segala apa yang kumiliki\\
itu semua anugrah-Mu.\\
{~}\\
Ref:\\
Terimalah Tuhan persembahan ini \\
dengan tulus hati dan dengan hati bersyukur\\ 
terimalah Tuhan persembahan ini\\ 
segala puji hormat bagimu.\\
}
\end{verse}
\normalsize


\BI{Kami memuji Engkau ya Bapa, Allah semesta alam, sebab 
dari kemurahanMu kami menerima roti dan anggur yang 
kami persembahkan ini. Inilah hasil dari bumi dan usaha 
manusia yang bagi kami akan menjadi santapan rohani.}

\BU{Terpujilah Allah selama-lamanya}

\BI{Berdoalah saudara-saudara supaya persembahan kita ini 
diterima oleh Allah Bapa yang mahakuasa.}

\BU{Semoga persembahan ini diterima demi kemuliaan Tuhan 
dan keselamatan kita serta seluruh umat Allah yang kudus.}

\BI{Ya Allah Bapa di surga, pengampunanMu menjadi sumber 
kedamaian dan kekuatan baru di dalam hati kami untuk 
mengikuti PuteraMu dengan setia. Maka kami mohon 
pandanglah dengan rela persembahan di atas altar ini dan 
teguhkanlah hati kami berkat korban Yesus Kristus, Tuhan 
dan Pengantara kami kini dan sepanjang masa.}

\BU{Amin} 

\subjudul{DOA SYUKUR AGUNG}


\subjudul{Kudus -- MB 255 - Misa Dolo-dolo}

\subjudul{Bapa Kami}

\subjudul{Anak Domba Allah}

\subjudul{Komuni}

\subjudul{Lagu Komuni}
\small
\begin{center}
\itshape{Tuhanku Gembala}
\end{center}

\begin{verse}
\itshape{
Tuhanku gembalaku, aku diantar olehNya\\
di padang rumput menghijau.\\
{~}\\
Ref:\\
Tuhanku gembalaku, tiada aku kekurangan, \\
tiada aku kekurangan.\\
Tuhanku gembalaku, tiada aku kekurangan, \\
Tuhanku gembalaku, tiada aku kekurangan.\\
{~}\\
Aku dihantar olehNya ke sumber-sumber istirahat,\\
disegarkannya jiwaku.\\
Ref:\\
{~}\\
Biarku lewat jurang kelam tiada ku takut bencana,\\
Engkau menyertai daku.\\
Ref:\\
{~}\\
}
\end{verse}
\normalsize 

\subjudul{Doa sesudah komuni}

\BI{Marilah berdoa: Terima kasih ya Allah, atas anugerah 
terbesar yaitu kami telah diberi rahmat untuk mengenal 
Yesus Kristus. Dalam Dia manusia yang berdosa mendapat 
pengampunan dan perlindungan. Semoga kami berani 
melepaskan semuanya dan menjadi serupa dengan Dia 
dalam kematianNya. Sebab Dialah Tuhan dan pengantara 
kami kini dan sepanjang masa.}

\BU{Amin.}

 

\section*{RITUS PENUTUP}

\BI{Tuhan beserta kita}

\BU{Sekarang dan selama-lamanya}

\BI{Semoga saudara sekalian diberkati oleh Allah Yang 
Mahakuasa † Bapa dan Putera dan Roh Kudus.}

\BU{Amin}

 

\subjudul{Pengutusan}

\BI{Saudara sekalian, Perayaan Ekaristi untuk memohon 
berkat Allah Bapa bagi arwah \namaalm telah selesai.}

\BU{Syukur kepada Allah}

\BI{Kita diutus untuk mewartakan bahwa Tuhan Yesus adalah 
jalan, kebangkitan dan hidup.}

\BU{Amin.}

 

\lagu{Lagu Penutup}
\scriptsize
\begin{center}
\itshape{Ave Maria}
\end{center}

\begin{verse}
\itshape{
Salam Maria penuh rahmat, Tuhan sertamu selalu\\
Terpuji di antara wanita dan terpuji buah tubuhmu.\\
{~}\\
Salam Maria penuh rahmat, Ratu surga dan dunia\\
Terpilih di antara wanita, menjadi Bunda Yesus Tuhan.
{~}\\
Ref:\\
Ave Maria 3$\times$, Bunda pengantara rahmat\\
Ave Maria 3$\times$, Bunda penuh cinta.\\
}
\end{verse}
\normalsize 

\newpage
\begin{flushright}
{\Large Ucapan terima kasih}
\noindent Dengan penuh syukur dalam kasih Tuhan, kami mengucapkan banyak
terima kasih kepada:
\large

\textbf{Romo \namaromo}\\
yang telah berkenan mempimpin perayaan ekaristi peringatan 1000 hari meninggalnya\\ \namaalm
ini.

\textbf{Koor lingkungan Santo Petrus Maguwo}\\
yang telah menyemarakkan perayaan ekaristi ini.

\textbf{Umat lingkungan St. Petrus, para undangan, segenap keluarga, dan orang-orang terkasih}\\
yang telah berkenan hadir memberikan cinta dan doa dalam perayaan
ekaristi ini.

Semoga Tuhan memberkati dan memelihara ikatan kasih\\ di antara kita semua.

Amin.

\bigskip 

\textit{Ibu Catharina Setya Prihatiningtyas\\
dan segenap keluarga}
\end{flushright}

\end{document} 
