\documentclass[titlepage,10pt,openany]{scrbook}
\usepackage[papersize={107.5mm,148.5mm},twoside,bindingoffset=0.5cm,hmargin={1cm,1cm},
				vmargin={1.5cm,1.5cm},footskip=0.5cm,driver=dvipdfm]{geometry}
\usepackage[utf8]{inputenc}
\usepackage{graphicx}
\usepackage{wrapfig}
\usepackage[bahasa]{babel}
\usepackage{fancyhdr}
\usepackage{pst-text}
\usepackage{palatino}
\usepackage{marvosym}
\usepackage{pdfpages}
\usepackage{hyphenat}

\renewcommand{\footrulewidth}{0.5pt}
\lhead[\fancyplain{}{\thepage}]%
      {\fancyplain{}{\rightmark}}
\rhead[\fancyplain{}{\leftmark}]%
      {\fancyplain{}{\thepage}}
\pagestyle{fancy}
\lfoot[\emph{\footnotesize Misa Peringatan 100 hari \namaalm}]{}
\rfoot[]{\emph{\footnotesize Misa Peringatan 100 hari \namaalm}}
\cfoot{}

\makeatletter
\newcommand{\judul}[1]{%
  {\parindent \z@ \centering 
    \interlinepenalty\@M \Large \bfseries #1\par\nobreak \vskip 20\p@ }}
\newcommand{\subjudul}[1]{%
  {\parindent \z@ 
    \interlinepenalty\@M \bfseries #1\par\nobreak \vskip 10\p@ }}
\newcommand{\lagu}[1]{%
  {\parindent \z@ 
    \interlinepenalty\@M \slshape \bfseries \normalsize \textit{#1}\par\nobreak \vskip 10\p@ }}
\newcommand{\keterangan}[1]{%
  {\parindent \z@  \slshape 
    \interlinepenalty\@M \textsl{#1}\par\nobreak  \vskip 5\p@}}

\renewenvironment{description}
               {\list{}{\labelwidth\z@ \itemindent-\leftmargin
                        \let\makelabel\descriptionlabel}}
               {\endlist}
\renewcommand*\descriptionlabel[1]{\hspace\labelsep 
                                \normalfont\bfseries #1 }


\makeatother

\newcommand{\BU}[1]{\begin{itemize} \item[U:] #1 \end{itemize}}
\newcommand{\BI}[1]{\begin{itemize} \item[I:] #1 \end{itemize}}
\newcommand{\BIU}[1]{\begin{itemize} \item[I+U:] #1 \end{itemize}}
\newcommand{\BPU}[1]{\begin{itemize} \item[P+U:] #1 \end{itemize}}
\newcommand{\BP}[1]{\begin{itemize} \item[P:] #1 \end{itemize}}
\newcommand{\inputlagu}[1]{\itshape{\input{#1}}}
\newcommand{\namaalm}{Bapak Vincentius Fererius Parlan }
\newcommand{\namaromo}{Wagiman Wigyo Sumantara Pr.}

\title{Misa Peringatan 100 Hari}
\author{\namaalm}
\date{oleh \\ Rm \namaromo\\27 September 2015}
\hyphenation{a-kan}
\hyphenation{ba-gi-mu}
\hyphenation{ber-a-da}
\hyphenation{ber-du-a}
\hyphenation{be-ri-kan}
\hyphenation{ber-ka-ta}
\hyphenation{ber-nya-nyi}
\hyphenation{ber-sa-ma}

\hyphenation{dah-syat}
\hyphenation{DA-RAH-KU}
\hyphenation{da-tang}
\hyphenation{di-ka-ta-kan}
\hyphenation{di-pim-pin}
\hyphenation{di-se-rah-kan}
\hyphenation{di-tum-pah-kan}

\hyphenation{Eng-kau}
\hyphenation{ha-dap-an}
\hyphenation{han-tar-kan-lah}
\hyphenation{ha-rap-an}

\hyphenation{ja-lan}
\hyphenation{ja-ngan-lah}

\hyphenation{ka-nak}
\hyphenation{ka-re-na}
\hyphenation{kau-lim-pah-kan}
\hyphenation{Kau-cip-ta-kan}
\hyphenation{ke-bang-kit-an-Nya}
\hyphenation{ke-da-tang-an}
\hyphenation{ke-da-tang-an-Nya}
\hyphenation{ke-dua}
\hyphenation{ke-na-ik-kan-nya}
\hyphenation{ke-pa-daMu}
\hyphenation{ke-ra-him-an}
\hyphenation{ke-se-jah-te-ra-an-mu}
\hyphenation{ko-men-tar}

\hyphenation{la-ma-nya}
\hyphenation{lim-pah-kan}

\hyphenation{ma-nu-sia}
\hyphenation{me-nga-da-kan}
\hyphenation{me-ngan-dung-lah}
\hyphenation{me-ngu-kuh-kan}
\hyphenation{me-la-lui}
\hyphenation{me-lim-pah-kan}
\hyphenation{me-lu-hur-kan}
\hyphenation{me-me-cah-me-cah-kan}
\hyphenation{mem-per-sem-bah-kan}
\hyphenation{me-nan-da-ta-ngan-i}
\hyphenation{men-cin-tai}
\hyphenation{meng-a-lir-kan}
\hyphenation{me-nga-sihi}
\hyphenation{me-nge-lu-ar-kan}
\hyphenation{meng-u-cap-kan}
\hyphenation{meng-ung-kap-kan}
\hyphenation{me-num-buh-kan}
\hyphenation{me-nya-ta-kan}
\hyphenation{me-nye-la-mat-kan}
\hyphenation{me-nye-rah-kan}
\hyphenation{me-nye-rah-kanNya}
\hyphenation{me-ra-ya-kan}

\hyphenation{o-rang}
\hyphenation{o-rang-o-rang}

\hyphenation{pa-sang-kan-lah}
\hyphenation{pa-tut}
\hyphenation{pe-ne-ri-ma-an}
\hyphenation{pe-ngam-pun-an}
\hyphenation{Pe-ngan-ta-ra}
\hyphenation{peng-hi-bur-an}
\hyphenation{per-bu-at-an-nya}
\hyphenation{per-ka-ta-an}
\hyphenation{per-ka-win-an}
\hyphenation{per-ni-kah-an}
\hyphenation{per-se-ku-tu-an}
\hyphenation{per-sem-bah-an}
\hyphenation{rom-bong-an}

\hyphenation{se-la-ma}
\hyphenation{se-ka-li-an}
\hyphenation{se-pan-jang}
\hyphenation{se-ra-ya}
\hyphenation{Su-dar-yan-to}

\hyphenation{te-ta-pi}
\hyphenation{ta-ngan-Mu}
\hyphenation{Tu-han}
\hyphenation{tu-lang}
\hyphenation{tu-lang-tu-lang}

\hyphenation{u-mat-Mu}
\hyphenation{wa-kil}

\hyphenation{ba-gi-mu}
\hyphenation{di-se-rah-kan}
\hyphenation{me-la-lui}
\hyphenation{ka-nak}
\hyphenation{ka-re-na}
\hyphenation{ber-ka-ta}
\hyphenation{te-ta-pi}
\hyphenation{per-ka-win-an}
\hyphenation{pa-tut}
\hyphenation{me-lu-hur-kan}
\hyphenation{ber-nya-nyi}
\hyphenation{di-tum-pah-kan}
\hyphenation{pe-ngam-pun-an}
\hyphenation{ber-a-da}
\hyphenation{kau-lim-pah-kan}
\hyphenation{ke-bang-kit-an-Nya}
\hyphenation{per-ka-ta-an}
\hyphenation{pa-sang-kan-lah}
\hyphenation{DA-RAH-KU}
\hyphenation{ke-na-ik-kan-nya}
\hyphenation{per-sem-bah-an}
\hyphenation{per-se-ku-tu-an}


\DeclareFixedFont{\PT}{T1}{ppl}{b}{it}{0.5in}
\DeclareFixedFont{\PTsmall}{T1}{ppl}{b}{it}{0.2in}
\DeclareFixedFont{\PTsmallest}{T1}{ppl}{b}{it}{0.15in}
\DeclareFixedFont{\PTtext}{T1}{ppl}{b}{it}{11pt}
\DeclareFixedFont{\Logo}{T1}{pbk}{m}{n}{0.3in}

\hyphenation{me-nyi-ap-kan pan-jat-kan se-jah-te-ra Par-lan o-rang bang-kit}
\begin{document}
%\maketitle
\thispagestyle{empty}
%\begin{pspicture}(8cm,10cm)
%\psset{unit=1cm}
%\rput[cb](4,10){\PTsmall {EKARISTI}}
%\rput[cb](4,9){\PTsmall {PERINGATAN 100 HARI}}
%\rput[cb](4,8){\PTsmall {\namaalm}}
%\rput[cb](4,3){\PTsmallest {oleh}} 
%\rput[cb](4,2.5){\PTsmallest {Rm \namaromo}}
%\rput[cb](4,1){\PTsmallest {27 September 2015}}
%\end{pspicture}

\includepdf{misa-100-hari-vfp-cover}

\section*{RITUS PEMBUKA} 

 

\lagu{Lagu Pembuka}  
\begin{center}
\itshape{Semua Baik}
\end{center}


\small
\begin{verse}
\itshape{
Dari semula\\
tlah Kau tetapkan,\\
hidupku dalam tangan-Mu\\
dalam rencanaMu Tuhan.\\
{~}\\
Rencana indah\\
tlah kau siapkan,\\
bagi masa depanku\\
yang penuh harapan.\\
{~}\\
(*)\\
Semua baik, semua baik\\
apa yg tlah Kau perbuat\\
di dalam hidupku\

Semua baik\\
sungguh teramat baik\\
Kau jadikan hidupku berarti
}
\end{verse}
\normalsize 

\subjudul{Tanda Salib} 

\BI{Demi nama  Bapa dan Putera dan Roh Kudus}

\BU{Amin}

 

\subjudul{Salam}

\BI{Semoga kasih karunia, rahmat dan damai sejahtera dari 
Allah Bapa dan dari PuteraNya Yesus Kristus beserta 
saudara.} 

\BU{Sekarang dan selama-lamanya.}

 

\subjudul{Pengantar}

\BI{Saudara-saudari terkasih,
Hari ini kita bersama-sama berdoa bersama untuk
mendoakan arwah dari saudara kita \textbf{\namaalm} yang pada 100
hari yang lalu di panggil Bapa. Kita percaya bahwa segala
doa yang kita panjatkan untuk mereka yang sudah
meninggal sangat bermanfaat demi terwujudnya harapan
iman mereka untuk berdiam di rumah Tuhan selama-lamanya.}


\subjudul{Tobat}

\BI{Saudara-saudara, menyadari bahwa kita hanyalah serupa 
debu bernoda di depan alas kaki Allah Bapa, marilah kita 
bersyukur bahwa kita masih diperkenankan berdoa dan 
bermohon kepada Allah Bapa. 

Saya mengaku, \ldots \ldots
} 



\BI{Semoga Allah Bapa yang Maha Kuasa, mengasihani kita, 
mengampuni dosa kita dan mengantar kita ke dalam 
kehidupan kekal.}

\BU{Amin}

\lagu{Kyrie} 
\begin{center}
\itshape{Kyrie}
\end{center}

\small
\begin{verse}
\itshape{
Kyrie eleison, 2$\times$\\
Christe eleison, 2$\times$\\
Kyrie eleison, 2$\times$
}
\end{verse}
\normalsize 

\subjudul{Doa Pembuka}

\BI{Marilah kita berdoa 

Allah Bapa di surga, segala kelemahan dan dosa kami 
terbentang di hadapan Engkau. Karena dosa-dosa itu 
pula, Engkau telah mengutus PuteraMu sendiri Tuhan 
kami Yesus Kristus, untuk datang dan menyelamatkan 
kami dan menyiapkan tempat bagi kami dalam rumah 
kekalMu. Ialah jalan dan kehidupan kami. 

Kami memohon ampunan\hyp Mu untuk semua dosa\hyp dosa kami 
dan terlebih untuk dosa\hyp dosa \namaalm dan juga dosa\hyp dosa para leluhur kami. 

Semua ini kami mohon demi Yesus Kristus Putera\hyp Mu dan pengantara kami yang bersatu dengan Dikau dan Roh Kudus, hidup dan berkuasa kini dan sepanjang masa.}

\BU{Amin}

 

\section*{LITURGI SABDA} 

\keterangan{Pembacaan dari Surat Pertama Rasul Paulus kepada umat di
Korintus (1Kor 15:12-23)}

\BP{Jadi, bilamana kami beritakan, bahwa Kristus dibangkitkan
dari antara orang mati, bagaimana mungkin ada di antara kamu
yang mengatakan, bahwa tidak ada kebangkitan orang mati?

Kalau tidak ada kebangkitan orang mati, maka Kristus juga
tidak dibangkitkan. Tetapi andaikata Kristus tidak dibangkitkan
maka sia-sialah pemberitaan kami dan sia-sialah juga
kepercayaan kamu. Lebih dari pada itu kami ternyata berdusta
terhadap Allah, karena tentang Dia kami katakan, bahwa Ia telah
membangkitkan Kristus -- padahal Ia tidak membangkitkan\hyp Nya,
kalau andaikata benar, bahwa orang mati tidak dibangkitkan.

Sebab jika benar orang mati tidak dibangkitkan, maka Kristus
juga tidak dibangkitkan. Dan jika Kristus tidak dibangkitkan,
maka sia-sialah kepercayaan kamu dan kamu masih hidup
dalam dosamu. 
Demikianlah binasa juga orang-orang yang mati
dalam Kritus. 

Jikalau kita hanya dalam hidup ini saja
menaruh pengharapan pada Kristus, maka kita adalah orang-\-orang yang paling malang dari segala manusia. 

Tetapi yang
benar ialah, bahwa Kristus telah dibangkitkan dari antara orang
mati, sebagai yang sulung dari orang-orang yang telah
meninggal. Sebab sama seperti maut datang karena satu ~o- rang
manusia, demikian juga kebangkitan orang mati datang karena
satu orang manusia. Karena sama seperti semua orang mati
dalam persekutuan dengan Adam, demikian pula semua orang
akan dihidupkan kembali dalam persekutuan dengan Kristus.

Tetapi tiap-tiap orang menurut urutannya : Kristus sebagai buah
sulung ; sesudah itu mereka yang menjadi milik\hyp Nya pada waktu
kedatangan\hyp Nya.

Demikianlah Sabda Tuhan.}

\BU{Syukur kepada Allah}

 

\lagu{Lagu tanggapan sabda}
\begin{center}
\itshape{Kurenungkan SabdaMu, Tuhan}
\end{center}

\small
\begin{verse}
\itshape{
\textbf{Ref:}\\
Kurenungkan sabdaMu Tuhan,\\ 
sabda penuh kebenaran,\\
Kuresapkan di dalam kalbu\\ 
agar selalu jadi milikku\\
{~}\\
Sabda o, sabda-Mu\\ 
memecah hatiku yang membatu,\\
mencairkan budi yang membeku\\
dan membuka cakrawala baru.\\
{~}\\
Buka pintu hati\\ 
luaskan arah pandanganku,\\
dan hilangkan ketegaranku, \\
aku sedia menjawab sabda-Mu.
}
\end{verse}
\normalsize 

\subjudul{Injil Yoh 6:37-40}

\BI{Tuhan sertamu}

\BU{dan sertamu juga} 

\BI{Inilah Injil Yesus Kristus menurut Yohanes }

\BU{Dimuliakanlah Tuhan}

\BI{Semua yang diberikan Bapa kepada\hyp Ku akan datang
kepada\hyp Ku, dan barangsiapa datang kepada\hyp Ku, ia tidak akan
Kubuang. Sebab Aku telah turun dari sorga bukan untuk
melakukan kehendak\hyp Ku, tetapi untuk melakukan kehendak Dia
yang telah mengutus Aku. Dan inilah kehendak Dia yang telah
mengutus Aku, yaitu supaya dari semua yang telah
diberikanNya kepada-Ku jangan ada yang hilang, tetapi supaya
Kubangkitkan pada akhir zaman. Sebab inilah kehendak
BapaKu, yaitu supaya setiap orang, yang melihat Anak dan
yang percaya kepada\hyp Nya beroleh hidup yang kekal, dan
supaya Aku membangkitkannya pada akhir zaman.}


\BI{Demikianlah Injil Tuhan}

\BU{Terpujilah Kristus}

 

\subjudul{Homili}

\subjudul{Syahadat} 

\subjudul{Doa Umat}

\BI{Saudara-saudari terkasih,
Marilah kita panjatkan doa-doa bagi arwah saudara kita
yang sudah meninggal juga bagi kita semua dan Gereja
yang terus menerus.}

\BP{Bagi saudara kita \textbf{\namaalm} yang sudah dipanggil Bapa 100
hari yang lalu. Semoga melalui pembaptisan yang telah
diterima saudara kita \textbf{\namaalm} dan berkat iman akan Yesus
sepanjang hidupnya, ia dianugerahi hidup kekal yang
telah dijanjikan Allah sendiri kepadanya.

\textit{Hening sejenak \ldots\ldots} 

Marilah kita mohon :}

\BU{Kabulkanlah doa kami ya Tuhan}

\BP{Bagi para uskup dan para imam kita yang sudah
meninggal.
Semoga para pemimpin Gereja kita yang sudah dipanggil
menghadap Bapa, diikutsertakan dalam perayaan surgawi.

\textit{Hening sejenak \ldots\ldots} 

Marilah kita mohon :}

\BU{Kabulkanlah doa kami, ya Tuhan}

\BP{Bagi siapa saja yang sudah meninggal dengan harapan akan bangkit kembali. Semoga semua orang yang meninggal, dengan harapan akan bangkit lagi, diterima dalam pangkuan surgawi abadi.

\textit{Hening sejenak \ldots\ldots} 

Marilah kita mohon :}

\BU{Kabulkanlah doa kami, ya Tuhan}

\BP{Bagi kita semua yang ada di tempat ini.
Semoga iman pengharapan kita akan Yesus Kristus
semakin dikuatkan dalam perjuangan hidup keseharian
kita.

\textit{Hening sejenak \ldots\ldots} 

Marilah kita mohon :}

\BU{Kabulkanlah doa kami, ya Tuhan}

\BI{Bapa kasihMu tiada batas, kesabaranMu begitu besar. Semoga dalam pengharapan iman yang benar, kami senantiasa dalam limpahan rahmat-Mu, dengan pengantaraan Kristus Tuhan kami.}

\BU{Amin}

\section*{LITURGI EKARISTI}

\lagu{Lagu pengantar persembahan}
\begin{center}
\itshape{Serahkanlah semua}
\end{center}

\small
\begin{verse}
\itshape{
Ya Bapa ajar kami percaya kepada-Mu\\
Bagaikan burung di langit biru\\ 
tak menabur tak menuai\\ 
namun hidup penuh damai\\
{~}\\
Ya Bapa ajar kami berserah kepada-Mu\\
Turuti segala kehendakMu\\
percaya pada kasih dan cinta-Mu\\
{~}\\
Serahkanlah semua kekuatiran hidupmu\\ 
biarkanlah Dia bekerja bagimu.\\
Percayalah Tuhan menjagamu\\
Dia akan setia menyertaimu.
{~}\\
Serahkanlah pada-Nya
}
\end{verse}
\normalsize

\BI{Kami memuji Engkau ya Bapa, Allah semesta alam, sebab 
dari kemurahanMu kami menerima roti dan anggur yang 
kami persembahkan ini. Inilah hasil dari bumi dan usaha 
manusia yang bagi kami akan menjadi santapan rohani.}

\BU{Terpujilah Allah selama-lamanya}

\BI{Berdoalah saudara-saudara supaya persembahan kita ini 
diterima oleh Allah Bapa yang mahakuasa.}

\BU{Semoga persembahan ini diterima demi kemuliaan Tuhan 
dan keselamatan kita serta seluruh umat Allah yang kudus.}

\BI{Ya Allah Bapa di surga, pengampunanMu menjadi sumber 
kedamaian dan kekuatan baru di dalam hati kami untuk 
mengikuti PuteraMu dengan setia. Maka kami mohon 
pandanglah dengan rela persembahan di atas altar ini dan 
teguhkanlah hati kami berkat korban Yesus Kristus, Tuhan 
dan Pengantara kami kini dan sepanjang masa.}

\BU{Amin} 

\subjudul{DOA SYUKUR AGUNG}


\subjudul{Sanctus}
\begin{center}
\itshape{Sanctus}
\end{center}

\small
\begin{verse}
\itshape{
Sanctus, Sanctus, Sanctus,\\
Dominus Deus Sabbaoth;\\
Pleni sunt caeli et terra gloria Tua.\\
Hosanna in excelsis.\\
Benedictus qui venit in nomine Domini.\\
Hosanna in excelsis.\\
}
\end{verse}
\normalsize

\subjudul{Bapa Kami}

\subjudul{Agnus Dei}
\begin{center}
\itshape{Agnus Dei}
\end{center}

\small
\begin{verse}
\itshape{
Agnus Dei, qui tollis peccata mundi, dona eis requiem. 2$\times$\\
Agnus Dei, qui tollis peccata mundi, dona eis requiem sempiternam.}
\end{verse}
\normalsize 
\subjudul{Komuni}

\subjudul{Lagu Komuni}
\begin{center}
\itshape{Tinggallah di hatiku}
\end{center}

\small
\begin{verse}
\itshape{
Oh Tuhan, kami bersyukur kepadaMu\\
Karena boleh menyambut Tubuh Kristus\\
yang  menyatukan kami anak anakMu\\
Satu sebagai jemaatMu\\
{~}\\
\textbf{Ref:}\\
Tuhan, Kau satukan hidupku\\ 
dalam Sakramen Mahakudus.\\
Tinggallah di hatiku slalu,\\
kini dan sampai selamanya\\
{~}\\
Kami percaya santapan jiwa ini\\ 
adalah Tubuh Kristus.\\
Kau berikan Putra-Mu kepada kami,\\
maka terimalah hidupku.\\
{~}\\
Oh Tuhan, kuatkanlah kami selalu,\\
supaya tak jatuh dalam dosa.\\
Ingatkan kami untuk slalu mengabdi\\
demi kemuliaan nama-Mu
}
\end{verse}
\normalsize 

\subjudul{Doa sesudah komuni}

\BI{Marilah berdoa: Terima kasih ya Allah, atas anugerah 
terbesar yaitu kami telah diberi rahmat untuk mengenal 
Yesus Kristus. Dalam Dia manusia yang berdosa mendapat 
pengampunan dan perlindungan. Semoga kami berani 
melepaskan semuanya dan menjadi serupa dengan Dia 
dalam kematianNya. Sebab Dialah Tuhan dan pengantara 
kami kini dan sepanjang masa.}

\BU{Amin.}

 

\section*{RITUS PENUTUP}

\BI{Tuhan beserta kita}

\BU{Sekarang dan selama-lamanya}

\BI{Semoga saudara sekalian diberkati oleh Allah Yang 
Mahakuasa \Cross ~Bapa dan Putera dan Roh Kudus.}

\BU{Amin}

 

\subjudul{Pengutusan}

\BI{Saudara sekalian, Perayaan Ekaristi untuk memohon 
berkat Allah Bapa bagi arwah \namaalm telah selesai.}

\BU{Syukur kepada Allah}

\BI{Kita diutus untuk mewartakan bahwa Tuhan Yesus adalah 
jalan, kebangkitan dan hidup.}

\BU{Amin.}

 

\lagu{Lagu Penutup}

\vspace{1cm}

\begin{center}
\itshape{Sakjêgé aku ndhèrèk Gusti}
\end{center}


\small
\begin{verse}
\itshape{
Sakjêgé aku ndhèrèk Gusti\\ 
uripku tansah dibêrkahi\\
atiku ayêm têntrêm\\ 
atiku ayêm têntrêm\\ 
kabèh iku Gusti Yésus kang maringi\\
{~}\\
Sakjêgé aku ndhèrèk Gusti\\ 
uripku tansah dibêrkahi\\
atiku ayêm têntrêm\\ 
atiku ayêm têntrêm\\ 
kabèh iku Gusti Yésus kang maringi\\
{~}\\
Matur nuwun matur nuwun matur nuwun \\
Gusti Yésus kula matur nuwun\\ 
Matur nuwun matur nuwun matur nuwun \\
Gusti Yésus kula matur nuwun\\ 
}
\end{verse}
\normalsize 

\newpage
\begin{flushright}
{\Large Ucapan terima kasih}\\
\noindent Dengan penuh syukur dalam kasih Tuhan, kami mengucapkan banyak
terima kasih kepada:
\large

\textbf{Romo \namaromo}\\
yang telah berkenan memimpin perayaan ekaristi peringatan 100 hari meninggalnya\\ \namaalm
ini.

\textbf{Koor Calista}\\
yang telah menyemarakkan perayaan ekaristi ini.

\textbf{Umat lingkungan St. Yustinus, para undangan, segenap keluarga, dan orang-orang terkasih}\\
yang telah berkenan hadir memberikan cinta dan doa dalam perayaan
ekaristi ini.

Semoga Tuhan memberkati dan memelihara ikatan kasih\\ di antara kita semua.

Amin.

\bigskip 

\textit{Ibu MG Waldjijah\\
dan segenap keluarga}
\end{flushright}

\end{document} 
