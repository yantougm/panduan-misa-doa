\documentclass[12pt, twoside, a4paper,startany]{scrbook}
\usepackage{ifpdf}
\usepackage[colorlinks,bookmarksopen]{hyperref}
\begin{document}

\chapter*{Bacaan Kitab Suci}

\section*{Bacaan : 2 Kor 5:1-10}
\section*{Injil : Mat 26: 26-29}

Dalam misa arwah 7 hari, Pastor Anton meminta beberapa dari kita untuk sharing pengalaman soal kesan yang kita peroleh selagi hidup bersama pak Paul. Pada kesempatan ini, saya tidak akan minta untuk melakukan hal yang sama, tapi saya mengajak kita untuk melihat bahwa ketika kita mengengankan saudara saudari kita yang telah meninggal itu berarti kita menghubungkan kembali diri kita dengan dirinya. Amatlah sangat penting bahwa dalam hidup ini kita mengenangkan kembali orang-orang yang dulu pernah hidup bersama dengan kita. Kita menghubungkan hidup kita dengan hidup mereka. Seiring dengan itu, kesedihan kita pun semakin berkurang. Dan di sinilah kemampuan daya ingat memainkan peranan yang besar. Setiap kali kita mengenang kembali kehidupan orang yang telah meninggal, kita pasti akan merasa sedih. 

Sedih atau bahkan sampai menangis bukanlah sesuatu yang buruk. Bahkan dikatakan, itu baik dan bahkan perlu. Kalau kita berusaha untuk menekan perasaan sedih itu maka kita pasti akan merasa sakit. 

Kita semua tentu ingin dikenang. Jesus sendiri pun ingin dikenang. Dia meninggalkan bagi kita suatu cara untuk mengenang Dia yaitu melalui Ekaristi. Pada perjamuan malam terakhir Dia bahkan mengatakan : ‘Lakukan ini sebagai peringatan akan Daku’ 

Dan hal yang menakjubkan dalam hal ini adalah bahwa ketika kita mengenang Yesus dengan cara ini, Dia akan hadir bersama kita. Bukan suatu kehadiran fisik tetapi secara rohani: Suatu kehadiran nyata yang mengatasi ruang dan waktu. Dengan demikian kita bisa masuk dalam suatu relasi dengan dia lebih dalam daripada relasi melalui kehadiran fisik. Kita tidak hanya berkomunikasi dengan Dia tetapi kita bersatu dengan Dia. 

Orang-orang yang kita cintai yang telah meninggal dunia, tidak pernah hilang, tidak pernah dipisahkan dari kita. Jika kita mengenang mereka, maka mereka pun hadir bersama dengan kita. Bukan hanya dalam kenangan tetapi sungguh hadir…kita tidak bisa melihat, tetapi kita bisa merasakan.

Pada kesempatan ini, pertama-tama kita diajak untuk bersyukur kepada Allah atas anugerah hidup yang telah Ia berikan kepada almarhum…juga atas anugerah hidup yang telah kita terima melalui dia. 

Dari sharing-sharing yang dulu, saya mendapat kesan bahwa kita semua berbangga bahwa dalam hidup ini kita mengenal dan bahkan hidup bersama Pak Paul Saya yakin keluarga pasti sangat berbangga memiliki seorang ayah seperti ini. Namun satu hal yang ingin saya katakan bahwa betapa pun baiknya seorang ayah, cintanya yang pernah kita terima masih merupakan cinta seorang manusia yang terbatas dan tidak sempurna. Kita merindukan sebuah cinta yang sungguh dapat dipercayai, sebuah cinta yang sungguh sempurna, dimana hanya Allah sajalah yang mampu memberikannya. Hanya Allah dapat memberikan apa yang kita rindukan.

Cinta seorang ayah bagi anak-anaknya mengingatkan kita akan cinta Allah. Cinta seorang ayah merupakan refleksi atas cinta Allah. Berulang kali di dalam Injil, Yesus berbicara tentang Allah dengan membandingkan cinta seorang ayah kepada anak-anaknya. Dan ketika Yesus mati, dia mempercayakan roh-Nya kepada Bapa-Nya seperti seorang anak yang menjatuhkan dirinya ke dalam rangkulan sang ayah. 

Kita adalah anak-anak Allah. Ketika Allah menciptakan kita, Ia menciptakan menurut gambar dan rupa-Nya sendiri. Ketika Allah memandang kita, Allah melihat gambaran diri-Nya di dalam kita yang membuat-Nya selalu mencintai kita. 

Seringkali kita diingatkan akan cinta Allah pada peristiwa-peristiwa kematian. Ketika sanak keluarga meninggal, kita merasa tak berdaya, bahwa semuanya itu berada di luar kontrol diri kita. Kita seakan berjalan sendirian. Tetapi...Allah selalu ada dalam situasi apapun. Allah tidak meninggalkan kita dalam situasi seperti ini. 

Cinta Allah itu memampukan kita untuk meninggalkan dunia ini menuju suatu dunia yang baru dengan penuh harapan. 

Dalam bacaan pertama tadi, rasul Paulus menegaskan kepada umat di Korintus bahwa kita hidup dalam dua dunia yakni dunia yang sekarang dan dunia yang akan datang. Dunia yang sekarang ini bersifat sementara, dunia yang penuh dengan tekanan sosial, politik, keamanan dan karena tidak kekal, hasil buatan manusia maka dunia sekarang ini bisa dibongkar. 

Sebaliknya dunia yang akan datang sifatnya kekal, dunia yang penuh dengan kedamaian. Pembangunnya ialah Allah sendiri. 

Allah telah memberi kita kunci untuk bisa masuk ke dalam dunia yang akan datang. Namun satu hal yang dituntut dari setiap kita yang mau masuk ke dalamnya adalah Iman, harap dan kasih. Semoga. 

Pastor Tonny Blikon, SS.CC

\end{document}
