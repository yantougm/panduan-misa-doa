\documentclass[a5paper,headsepline,titlepage,11pt,nnormalheadings,DIVcalc]{scrbook}
\usepackage[a5paper,backref]{hyperref}
\usepackage[papersize={148.5mm,215mm},twoside,bindingoffset=0.5cm,hmargin={2cm,2cm},
				vmargin={2cm,2cm},footskip=1.1cm,driver=dvipdfm]{geometry}
%\usepackage{palatino}
\usepackage{graphicx}
\usepackage{wrapfig}
\usepackage[bahasa]{babel}
\usepackage{fancyhdr}
\usepackage{pst-text}
\usepackage{pst-grad}
\usepackage{marvosym}

%\setlength{\voffset}{0.5in}
%\setlength{\oddsidemargin}{28pt}
%\setlength{\evensidemargin}{0pt}
\renewcommand{\footrulewidth}{0.5pt}
\lhead[\fancyplain{}{\thepage}]%
      {\fancyplain{}{\rightmark}}
\rhead[\fancyplain{}{\leftmark}]%
      {\fancyplain{}{\thepage}}
\pagestyle{fancy}
\lfoot[\emph{Misa peringatan 2 tahun \namaalm}]{}
\rfoot[]{\emph{Misa peringatan 2 tahun \namaalm}}
\cfoot{}

\makeatletter
\newcommand{\judul}[1]{%
  {\parindent \z@ \centering 
    \interlinepenalty\@M \Large \bfseries #1\par\nobreak \vskip 20\p@ }}
\newcommand{\subjudul}[1]{%
  {\parindent \z@ 
    \interlinepenalty\@M \bfseries #1\par\nobreak \vskip 10\p@ }}
\newcommand{\lagu}[1]{%
  {\parindent \z@ 
    \interlinepenalty\@M \slshape \mdseries \Large \textit{#1}\par\nobreak \vskip 10\p@ }}
\newcommand{\keterangan}[1]{%
  {\parindent \z@  \slshape 
    \interlinepenalty\@M \textsl{#1}\par\nobreak  \vskip 5\p@}}

\renewenvironment{description}
               {\list{}{\labelwidth\z@ \itemindent-\leftmargin
                        \let\makelabel\descriptionlabel}}
               {\endlist}
\renewcommand*\descriptionlabel[1]{\hspace\labelsep 
                                \normalfont\bfseries #1 }


\makeatother

\newcommand{\BU}[1]{\begin{itemize} \item[U:] #1 \end{itemize}}
\newcommand{\BI}[1]{\begin{itemize} \item[I:] #1 \end{itemize}}
\newcommand{\BIU}[1]{\begin{itemize} \item[I+U:] #1 \end{itemize}}
\newcommand{\BP}[1]{\begin{itemize} \item[P:] #1 \end{itemize}}
\newcommand{\inputlagu}[1]{\begin{textit} \input{#1} \end{textit}}
\newcommand{\namaalm}{Ibu Monika Suarti~}
\newcommand{\namaromo}{Kalictus SVD~}

\title{Misa Peringatan 2 tahun}
\author{\namaalm\\
(13 Februari 2010)}
\date{oleh \\ Rm \namaromo\\16 Februari 2012}
\hyphenation{a-kan}
\hyphenation{ba-gi-mu}
\hyphenation{ber-a-da}
\hyphenation{ber-du-a}
\hyphenation{be-ri-kan}
\hyphenation{ber-ka-ta}
\hyphenation{ber-nya-nyi}
\hyphenation{ber-sa-ma}

\hyphenation{dah-syat}
\hyphenation{DA-RAH-KU}
\hyphenation{da-tang}
\hyphenation{di-ka-ta-kan}
\hyphenation{di-pim-pin}
\hyphenation{di-se-rah-kan}
\hyphenation{di-tum-pah-kan}

\hyphenation{Eng-kau}
\hyphenation{ha-dap-an}
\hyphenation{han-tar-kan-lah}
\hyphenation{ha-rap-an}

\hyphenation{ja-lan}
\hyphenation{ja-ngan-lah}

\hyphenation{ka-nak}
\hyphenation{ka-re-na}
\hyphenation{kau-lim-pah-kan}
\hyphenation{Kau-cip-ta-kan}
\hyphenation{ke-bang-kit-an-Nya}
\hyphenation{ke-da-tang-an}
\hyphenation{ke-da-tang-an-Nya}
\hyphenation{ke-dua}
\hyphenation{ke-na-ik-kan-nya}
\hyphenation{ke-pa-daMu}
\hyphenation{ke-ra-him-an}
\hyphenation{ke-se-jah-te-ra-an-mu}
\hyphenation{ko-men-tar}

\hyphenation{la-ma-nya}
\hyphenation{lim-pah-kan}

\hyphenation{ma-nu-sia}
\hyphenation{me-nga-da-kan}
\hyphenation{me-ngan-dung-lah}
\hyphenation{me-ngu-kuh-kan}
\hyphenation{me-la-lui}
\hyphenation{me-lim-pah-kan}
\hyphenation{me-lu-hur-kan}
\hyphenation{me-me-cah-me-cah-kan}
\hyphenation{mem-per-sem-bah-kan}
\hyphenation{me-nan-da-ta-ngan-i}
\hyphenation{men-cin-tai}
\hyphenation{meng-a-lir-kan}
\hyphenation{me-nga-sihi}
\hyphenation{me-nge-lu-ar-kan}
\hyphenation{meng-u-cap-kan}
\hyphenation{meng-ung-kap-kan}
\hyphenation{me-num-buh-kan}
\hyphenation{me-nya-ta-kan}
\hyphenation{me-nye-la-mat-kan}
\hyphenation{me-nye-rah-kan}
\hyphenation{me-nye-rah-kanNya}
\hyphenation{me-ra-ya-kan}

\hyphenation{o-rang}
\hyphenation{o-rang-o-rang}

\hyphenation{pa-sang-kan-lah}
\hyphenation{pa-tut}
\hyphenation{pe-ne-ri-ma-an}
\hyphenation{pe-ngam-pun-an}
\hyphenation{Pe-ngan-ta-ra}
\hyphenation{peng-hi-bur-an}
\hyphenation{per-bu-at-an-nya}
\hyphenation{per-ka-ta-an}
\hyphenation{per-ka-win-an}
\hyphenation{per-ni-kah-an}
\hyphenation{per-se-ku-tu-an}
\hyphenation{per-sem-bah-an}
\hyphenation{rom-bong-an}

\hyphenation{se-la-ma}
\hyphenation{se-ka-li-an}
\hyphenation{se-pan-jang}
\hyphenation{se-ra-ya}
\hyphenation{Su-dar-yan-to}

\hyphenation{te-ta-pi}
\hyphenation{ta-ngan-Mu}
\hyphenation{Tu-han}
\hyphenation{tu-lang}
\hyphenation{tu-lang-tu-lang}

\hyphenation{u-mat-Mu}
\hyphenation{wa-kil}

\hyphenation{ba-gi-mu}
\hyphenation{di-se-rah-kan}
\hyphenation{me-la-lui}
\hyphenation{ka-nak}
\hyphenation{ka-re-na}
\hyphenation{ber-ka-ta}
\hyphenation{te-ta-pi}
\hyphenation{per-ka-win-an}
\hyphenation{pa-tut}
\hyphenation{me-lu-hur-kan}
\hyphenation{ber-nya-nyi}
\hyphenation{di-tum-pah-kan}
\hyphenation{pe-ngam-pun-an}
\hyphenation{ber-a-da}
\hyphenation{kau-lim-pah-kan}
\hyphenation{ke-bang-kit-an-Nya}
\hyphenation{per-ka-ta-an}
\hyphenation{pa-sang-kan-lah}
\hyphenation{DA-RAH-KU}
\hyphenation{ke-na-ik-kan-nya}
\hyphenation{per-sem-bah-an}
\hyphenation{per-se-ku-tu-an}



\begin{document}
%\maketitle
\thispagestyle{empty}
\newsavebox\IBox
\sbox\IBox{\includegraphics[scale=0.25]{../gambar/Kerawang8.png}}
% set up the picture environment
\psset{unit=1in}
\begin{pspicture}(4in,5in)
% set up the fonts we use
\DeclareFixedFont{\PT}{T1}{ppl}{b}{it}{0.5in}
\DeclareFixedFont{\PTsmall}{T1}{ppl}{b}{it}{0.4in}
\DeclareFixedFont{\PTsmallest}{T1}{ppl}{b}{it}{0.2in}
\DeclareFixedFont{\PTtext}{T1}{ppl}{b}{it}{11pt}
\DeclareFixedFont{\Logo}{T1}{pbk}{m}{n}{0.3in}
% place the front cover picture
\rput[cb](2,1.5){\usebox\IBox}
% put the text on the front cover
\rput[cb](2,4.5){\PTsmall {EKARISTI}}
\rput[cb](2,4.1){\PTsmall {PERINGATAN 2 TAHUN}}
\rput[cb](2,3.0){\PTsmall {\namaalm}}
\rput[cb](2,2.6){\PTsmallest(13 Februari 2010)}
\rput[cb](2,-0.4){\PTsmallest {oleh}} 
\rput[cb](2,-0.8){\PTsmallest {Rm \namaromo}}
\rput[cb](2,-1.2){\PTsmallest {16 Februari 2012}}

%\rput[cb](3,-1){\PTsmallest {\namagereja}} 

\end{pspicture}

\newpage
\thispagestyle{empty}
{~}
\newpage

\section*{RITUS PEMBUKA} 

 

\lagu{Lagu Pembuka}  
%\begin{center}
Ya Tuhan Pandang HambaMu
\end{center}

\begin{verse}
Ya Tuhan pandang hambaMu \\
yang sujud menyembah.\\
Penuh syukur kepadaMu \\
dan hati berserah.\\
\end{verse}

\begin{verse}
Sembah dan bakti umatMu\\ 
pujian kemuliaanMu\\
seutuhnya terimalah \\
dan ampunMu limpahkanlah\\
Berpalinglah kepada hambaMu
\end{verse}

 

\subjudul{Tanda Salib} 

\BI{Demi nama  Bapa dan Putera dan Roh Kudus}

\BU{Amin}

 

\subjudul{Salam}

\BI{Semoga kasih karunia, rahmat dan damai sejahtera dari 
Allah Bapa dan dari PuteraNya Yesus Kristus beserta 
saudara.} 

\BU{Sekarang dan selama-lamanya.}

 

\subjudul{Pengantar}

\BI{Terpujilah Allah Bapa di surga: Ia yang memiliki, Ia yang 
memberi dan memelihara, Ia pula yang mengambilnya 
kembali. Terpujilah Allah Bapa di surga.

\namaalm adalah milik Bapa di surga. Karena kasihNya 
kepada kita semua, kita telah menikmati kehadirannya.
KepadaNya pulalah dia telah 
kembali.

Kini kita bersama-
sama berdoa menghadap Allah Bapa di surga untuk 
bersyukur atas kehadirannya, atas teladan kehidupannya 
dan memohon berkat Allah untuk arwahnya 
agar supaya Allah Bapa berkenan mengampuni dosa-dosanya 
dan menerimanya dalam rumah abadi dalam 
damai dan kemuliaan Allah Bapa di surga. 

Kita juga memohon kepada Allah Bapa untuk berkatNya 
agar kita dapat meneruskan kebiasaan baik dari \namaalm , 
terutama dalam kehidupan spiritualitas dan 
sosialitas kita.}

 

\subjudul{Tobat}

\BI{Tuhan Yesus, Engkau mengalami kematian sebagai manusia tetapi dibangkitkan oleh kekuasaan Bapa dalam Roh Kudus.

Tuhan kasihanilah kami}

\BU{Tuhan kasihanilah kami}

\BI{Engkaulah kebangkitan dan kehidupan, barang siapa percaya kepadaMu akan memperoleh kehidupan yang kekal.

Kristus kasihanilah kami}

\BU{Kristus kasihanilah kami}

\BI{Engkau akan datang kembali dengan mulia untuk mempersatukan kami semua dalam kerajaan surga.

Tuhan kasihanilah kami}

\BU{Tuhan kasihanilah kami}

\BI{Semoga Allah Bapa yang Maha Kuasa, mengasihani kita, 
mengampuni dosa kita dan mengantar kita ke dalam 
kehidupan kekal.}

\BU{Amin}

 

\subjudul{Doa Pembuka}

\BI{Marilah kita berdoa 

Allah Bapa di surga, segala kelemahan dan dosa kami 
terbentang di hadapan Engkau. Karena dosa-dosa itu 
pula, Engkau telah mengutus PuteraMu sendiri Tuhan 
kami Yesus Kristus, untuk datang dan menyelamatkan 
kami dan menyiapkan tempat bagi kami dalam rumah 
kekalMu. Ialah jalan dan kehidupan kami. 

Kami memohon ampunanMu untuk semua dosa-dosa kami 
dan terlebih untuk dosa-dosa \namaalm dan juga dosa-dosa para leluhur kami. 

Semua ini kami mohon demi Yesus Kristus PuteraMu dan 
pengantara kami yang bersatu dengan Dikau dan Roh 
Kudus, hidup dan berkuasa kini dan sepanjang masa.}

\BU{Amin}

 

\section*{LITURGI SABDA} 

\keterangan{Bacaan dari Yehezkiel 37 : 1 - 6}

\BP{Lalu kekuasaan TUHAN meliputi aku dan Ia membawa aku ke luar dengan perantaraan Roh-Nya dan menempatkan aku di tengah-tengah lembah, dan lembah ini penuh dengan tulang-tulang.
Ia membawa aku melihat tulang-tulang itu berkeliling-keliling dan sungguh, amat banyak bertaburan di lembah itu; lihat, tulang-tulang itu amat kering.

Lalu Ia berfirman kepadaku: "Hai anak manusia, dapatkah tulang-tulang ini dihidupkan kembali?" Aku menjawab: "Ya Tuhan ALLAH, Engkaulah yang mengetahui!"

Lalu firman-Nya kepadaku: "Bernubuatlah mengenai tu-lang-tulang ini dan katakanlah kepadanya: Hai tulang-tulang yang kering, dengarlah firman TUHAN!
Beginilah firman Tuhan ALLAH kepada tulang-tulang ini: Aku memberi nafas hidup di dalammu, supaya kamu hidup kembali.

Aku akan memberi urat-urat padamu dan menumbuhkan daging padamu, Aku akan menutupi kamu dengan kulit dan memberikan kamu nafas hidup, supaya kamu hidup kembali. Dan kamu akan mengetahui bahwa Akulah TUHAN."

Demikianlah Sabda Tuhan.}

\BU{Syukur kepada Allah}

 

\lagu{Lagu Antar Bacaan}


\subjudul{Injil Markus 4:30-34}

\BI{Tuhan beserta kita}

\BU{Sekarang dan selama-lamanya} 

\BI{Inilah Injil Yesus Kristus menurut Markus}

\BU{Dimuliakanlah Tuhan}

\BI{Dalam suatu kesempatan Yesus mengatakan: "Dengan apa hendak kita membandingkan Kerajaan Allah itu, atau dengan perumpamaan manakah hendaknya kita menggambarkannya?

Hal Kerajaan itu seumpama biji sesawi yang ditaburkan di tanah. Memang biji itu yang paling kecil dari pada segala jenis benih yang ada di bumi.

Tetapi apabila ia ditaburkan, ia tumbuh dan menjadi lebih besar dari pada segala sayuran yang lain dan mengeluarkan cabang-cabang yang besar, sehingga burung-burung di udara dapat bersarang dalam naungannya."

Dalam banyak perumpamaan yang semacam itu Ia memberitakan firman kepada mereka sesuai dengan pengertian mereka,
dan tanpa perumpamaan Ia tidak berkata-kata kepada mereka, tetapi kepada murid-murid-Nya Ia menguraikan segala sesuatu secara tersendiri.}

\subjudul{Aklamasi}

\BI{Berbahagialah orang yang mendengar Sabda Tuhan dan 
tekun melaksanakannya.}

\BU{Tanamkanlah SabdaMu ya Tuhan dalam hati kami.}

 

\subjudul{Homili}

\subjudul{Syahadat} 

\subjudul{Doa Umat}

\BI{Terpujilah Engkau ya Allah Bapa di surga karena besarlah kuasa 
dan kasihMu. Kami menghaturkan puji dan sembah atas segala 
kurnia yang telah Engkau limpahkan kepada kami: atas keluarga 
kami; atas rumah tempat tinggal kami; atas segala sesuatu yang 
telah kami terima dan nikmati mulai dari doa, teladan, seluruh 
kebutuhan hidup dan pendidikan yang Engkau berikan melalui 
orang-orang yang mengasihi kami.}

\BP{Untuk berkatMu bagi arwah \namaalm; bagi arwah saudara 
dan handai taulan; serta bagi arwah para leluhur. Semoga 
Engkau berkenan mengampuni dosa-dosa mereka dan 
memberi mereka karunia kebahagiaan abadi dalam rumahMu 
yang kudus. Marilah kita mohon,}

\BU{kabulkanlah doa kami ya Tuhan.} 

\BP{Untuk kehidupan kami; Allah Bapa di surga, Engkaulah 
sumber hidup, tuntunan dan keselamatan. Setiap orang 
mungkin bisa memperdaya dan meninggalkan kami, namun 
hanya Engkau sajalah yang tidak akan pernah memperdaya 
kami. Engkau menunjukkan kepada kami betapa kasihMu itu 
suci dan sejati; Engkau membimbing kami menuju kebebasan 
sejati; Engkau membimbing jalan kami. KepadaMu kami 
mohon agar kami Kauperkenankan kembali kepadaMu: 
harapan dan kebebasan jiwa kami, kebenaran dan 
kegembiraan batin kami. Kami mohon janganlah biarkan 
kami jauh dari padaMu ya Allah. Marilah kita mohon,}

\BU{kabulkanlah doa kami ya Tuhan.} 

\BP{Untuk keluarga kami. Allah Bapa di surga, keluarga adalah 
kurniaMu yang Kaupercayakan kepada manusia; keluarga 
adalah percikan dari surga yang dibagikan kepada semua 
manusia: keluarga adalah buaian di mana kami dilahirkan dan 
yang kami terus-menerus dilahirkan kembali dalam cinta. 
Allah Bapa di surga, kami mohon masuklah ke dalam rumah-
rumah kami dan pimpinlah kami dalam nyanyian kehidupan. 
Kami mohon perbaharuilah cahaya cinta dan buatlah kami 
merasakan keindahan menjadi terikat satu dengan yang 
lainnya dalam sebuah rangkulan kehidupan: sebuah 
kehidupan yang dihangatkan oleh nafas Allah sendiri, nafas 
dari Allah yang adalah Cinta. Ya Allah Bapa di surga, mohon 
selamatkanlah keluarga kami; lindungilah keluarga kami dari 
fitnah dan mara bahaya dan selamatkanlah hidup itu sendiri. 
Marilah kita mohon,}

\BU{kabulkanlah doa kami ya Tuhan.} 

\BP{Untuk semua orang. Ya Allah Bapa di surga, kami mohonkan 
pula berkatMu bagi semua orang yang memerlukan dan 
merindukan berkatMu terutama bagi mereka yang miskin, 
sakit dan lapar dan bagi mereka yang sedang berada dalam 
kesulitan dan beban berat; Marilah kita mohon,}

\BU{kabulkanlah doa kami ya Tuhan.} 

\section*{LITURGI EKARISTI}

\lagu{Lagu Persembahan}

\BI{Kami memuji Engkau ya Bapa, Allah semesta alam, sebab 
dari kemurahanMu kami menerima roti dan anggur yang 
kami persembahkan ini. Inilah hasil dari bumi dan usaha 
manusia yang bagi kami akan menjadi santapan rohani.}

\BU{Terpujilah Allah selama-lamanya}

\BI{Berdoalah saudara-saudara supaya persembahan kita ini 
diterima oleh Allah Bapa yang mahakuasa.}

\BU{Semoga persembahan ini diterima demi kemuliaan Tuhan 
dan keselamatan kita serta seluruh umat Allah yang kudus.}

\BI{Ya Allah Bapa di surga, pengampunanMu menjadi sumber 
kedamaian dan kekuatan baru di dalam hati kami untuk 
mengikuti PuteraMu dengan setia. Maka kami mohon 
pandanglah dengan rela persembahan di atas altar ini dan 
teguhkanlah hati kami berkat korban Yesus Kristus, Tuhan 
dan Pengantara kami kini dan sepanjang masa.}

\BU{Amin} 

\subjudul{DOA SYUKUR AGUNG}

\subjudul{KOMUNI}

\subjudul{Bapa Kami}

\subjudul{Komuni}

\subjudul{Lagu Komuni}
 
\subjudul{Doa sesudah komuni}

\BI{Marilah berdoa: Terima kasih ya Allah, atas anugerah 
terbesar yaitu kami telah diberi rahmat untuk mengenal 
Yesus Kristus. Dalam Dia manusia yang berdosa mendapat 
pengampunan dan perlindungan. Semoga kami berani 
melepaskan semuanya dan menjadi serupa dengan Dia 
dalam kematianNya. Sebab Dialah Tuhan dan pengantara 
kami kini dan sepanjang masa.}

\BU{Amin.}

\section*{RITUS PENUTUP}

\BI{Tuhan beserta kita}

\BU{Sekarang dan selama-lamanya}

\BI{Semoga saudara sekalian diberkati oleh Allah Bapa yang 
mahakuasa \Cross ~Bapa dan Putera dan Roh Kudus.}

\BU{Amin}

 

\subjudul{Pengutusan}

\BI{Saudara sekalian, Perayaan Ekaristi untuk memohon 
berkat Allah Bapa bagi arwah \namaalm serta seluruh leluhur dan anggota keluarga yang 
telah meninggal dunia telah selesai.}

\BU{Syukur kepada Allah}

\BI{Kita diutus untuk mewartakan bahwa Tuhan Yesus adalah 
jalan, kebangkitan dan hidup.}

\BU{Amin.}

 

\lagu{Lagu Penutup}


\newpage
\begin{flushright}
{\Large Ucapan terima kasih}
\noindent Dengan penuh syukur dalam kasih Tuhan, kami mengucapkan banyak
terima kasih kepada:
\large

\textbf{Romo \namaromo}\\
yang telah berkenan mempimpin perayaan ekaristi peringatan 2 tahun meninggalnya \namaalm
ini.

\textbf{Umat lingkungan Santo Petrus Maguwo dan tamu undangan}\\
yang telah mendukung perayaan ekaristi ini.

\textbf{Segenap keluarga dan orang-orang terkasih}\\
yang telah berkenan hadir memberikan cinta dan doa dalam perayaan
ekaristi ini.

Semoga Tuhan memberkati dan memelihara ikatan kasih\\ di antara kita semua.

Amin.

\bigskip 

Matius Daryanto\\
Theresia Adriani\\
dan segenap keluarga
\end{flushright}

\end{document} 
