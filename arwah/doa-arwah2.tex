\documentclass[a4paper, 12pt]{article}
\usepackage[left=2.00cm, right=2.00cm, top=2.00cm, bottom=2.00cm]{geometry}
\usepackage{palatino}
\usepackage{microtype}

\newcommand{\BU}[1]{\begin{itemize} \item[U:] #1 \end{itemize}}
\newcommand{\BI}[1]{\begin{itemize} \item[I:] #1 \end{itemize}}
\newcommand{\BP}[1]{\begin{itemize} \item[P:] #1 \end{itemize}}
%\newcommand{\arwah}{Bapak Yohanes Markus Gegono Purwadi }
\newcommand{\arwah}{Ibu Theresia Suci Wahyuningsih}



\begin{document}
\setlength{\parindent}{0mm}

\section*{Doa pembuka}	
\BP{Marilah kita berdoa,
Allah Yang Kekal, Tuhan Yang Kudus,
Bapa dan Pelindung segala ciptaan,
kami mengucap syukur kepada-Mu
karena Engkau telah mengumpulkan kami di sini.
Kami mengarahkan hati kepada-Mu  dan memanjatkan doa
bagi mereka yang telah meninggal dunia,
khususnya Ibu Suci Wahyuningsih.
Mereka semua adalah putra-putri-Mu yang kami kasihi,
para orang tua, kerabat, teman, saudara kami,
mereka yang telah berjasa kepada kami selama hidupnya,
dan tak lupa juga bagi sesama yang telah menyakiti kami,
yang sangat membutuhkan pengampunan kami.
Ya Allah, kami persembahkan doa Rosario hari ke-10 ini, untuk menebus segala dosa yang telah mereka perbuat
dan mohon terimalah mereka dalam persekutuan surgawi.
Demi Yesus Kristus Putera-Mu, Tuhan dan pengantara kami, kini dan selama-lamanya,}

\BU{Amin}

\section*{Bacaan Injil}

\BP{(Luk 10:38-42)

\textit{Marta menerima Yesus di rumahnya.  
Maria telah memilih bagian yang paling baik.
}

Inilah Injil Yesus Kristus menurut Lukas:

Dalam perjalanan ke Yerusalem 
Yesus dan murid-murid-Nya tiba di sebuah kampung. 
Seorang wanita bernama Marta menerima Dia di rumahnya.
Wanita itu mempunyai seorang saudara yang bernama Maria. 
Maria itu duduk dekat kaki Tuhan 
dan terus mendengarkan sabda-Nya.

Tetapi Marta sangat sibuk melayani. 
Ia mendekati Yesus dan berkata, 
"Tuhan, tidakkah Tuhan peduli, 
bahwa saudariku membiarkan daku melayani seorang diri? 
Suruhlah dia membantu aku."
Tetapi Tuhan menjawabnya, 
"Marta, Marta, 
engkau kuatir dan menyusahkan diri dengan banyak perkara,
padahal hanya satu saja yang perlu. 
Maria telah memilih bagian yang terbaik, 
yang tidak akan diambil dari padanya."

Demikianlah Injil Tuhan.}

\BU{Terpujilah Kristus}

\section*{Doa Umat}

\BP{Ya Bapa, kedatangan Yesus Kristus di tengah-tengah kami menghadirkan tahun rahmat Tuhan kepada umat manusia. Semoga, setelah 1 tahun meninggalnya saudara kami \arwah ini pun menghadirkan rahmat bagi saudara-saudari yang ditinggalkan dan kami semua yang hadir di sini.. Kami mohon:}

\BU{Kabulkanlah doa kami ya Tuhan.}

\BP{Ajarkanlah kepada kami iman, harapan, dan kasih yang sejati, agar kami mampu mengalami sukacita sejati baik di dunia ini maupun dalam persekutuan para kudus sebagaimana telah dialami oleh saudari kami \arwah ini. . Kami mohon:}

\BU{Kabulkanlah doa kami ya Tuhan.}

\BP{Bagi orang-orang yang telah meninggal dan
yang masih mengharapkan belah kasihan Allah.
Semoga, berkat belas kasihan dan kerahiman
Allah, Ia memberikan pengampunan kepada
mereka semua, sehingga mereka boleh beristirahat dalam kebahagiaan abadi bersama Bapa
di surga. Kami mohon:}

\BU{Kabulkanlah doa kami ya Tuhan.}

\BP{Bagi kami yang hadir dalam peringatan ini.
Semoga kami yang telah mendengarkan dan merenungkan sabda Tuhan, mampu mengimani
janji Yesus Kristus, yang terwujud dalam perjuangan hidup dan karya kami sehari-hari.
Kami mohon:}

\BU{Kabulkanlah doa kami ya Tuhan.}


\section*{Doa penutup}
\BP{Marilah kita berdoa,
	Allah Bapa Yang Mahakuasa dan kekal,
	kami menyatakan kepercayaan kami kepada-Mu,
	bahwa mereka yang kami doakan
	dalam doa novena arwah hari ini,
	kini berada dalam perdamaian-Mu.
	Kami percaya, bahwa kami bersama mereka
	yang telah Kau panggil, merupakan satu keluarga,
	satu tubuh dalam Kristus.
	Maka berilah kami berkatMu, agar kami senantiasa ingat akan persatuan yang membahagiakan itu.
	Demi Yesus Kristus Putera-Mu, Tuhan dan pengantara kami, kini dan selama-lamanya,}
\BU{Amin.}
	
\BP{Berkat penutup}
\BP{Semoga Tuhan beserta kita}
\BU{Sekarang dan selama-lamanya}
\BP{Semoga kita selalu dibimbing dan diberkati oleh Allah yang mahakuasa, Dalam nama Bapa dan Putera dan Roh Kudus.}
\BU{Amin.}
\end{document}

