\documentclass[10pt,a5paper,fancyhdr]{memoir}
\usepackage[left=2cm,right=2cm,top=2cm,bottom=2cm]{geometry}
\usepackage[utf8x]{inputenc}
\usepackage[bahasa]{babel}
\usepackage{amsmath}
\usepackage{amsfonts}
\usepackage{amssymb}
\usepackage{palatino}

%\author{Yohanes Suyanto}
\author{Gereja St Anna -- Duren Sawit \\  
Jakarta Timur}
\title{Ibadat Kematian\\ 
dan \\
Peringatan Arwah \\
{~}\\
\vspace{2cm}
~\\
(Edisi Percobaan) \\
~\\
sekretariat\_stanna@yahoo.com \\
~\\
~\\
}
\date{2009}
\setlength{\parindent}{0mm}
\makeatletter
\newcommand{\lagu}[1]{%
  {\parindent \z@ 
    \interlinepenalty\@M \slshape \mdseries \large \textit{#1}\par\nobreak \vskip 10\p@ }}
\newcommand{\keterangan}[1]{%
  {\parindent \z@ 
    \interlinepenalty\@M \slshape \mdseries \textit{#1}\par\nobreak \vskip 10\p@ }}
\makeatother

\newcommand{\BU}[1]{\begin{itemize}\itemsep0pt \item[U:] #1 \end{itemize}}
\newcommand{\BI}[1]{\begin{itemize}\itemsep0pt \item[I:] #1 \end{itemize}}
\newcommand{\BIU}[1]{\begin{itemize}\itemsep0pt \item[I+U:] #1 \end{itemize}}
\newcommand{\BPU}[1]{\begin{itemize}\itemsep0pt \item[P+U:] #1 \end{itemize}}
\newcommand{\BP}[1]{\begin{itemize}\itemsep0pt \item[P:] #1 \end{itemize}}
\newcommand{\inputlagu}[1]{\begin{textit} \input{#1} \end{textit}}
\newcommand{\nama}{\ldots\ldots\ldots}

\newlength\chaptitlelength
\newlength\chaptitlerlength

\makeatletter 
\newcommand\thickhrulefill[1]{%
  \leavevmode \leaders \hrule height #1 \hfill \kern \z@} 
\setlength\midchapskip{10pt} 
\makechapterstyle{mystyle}{
  \setlength\beforechapskip{0pt}
  \renewcommand\chapternamenum{} 
  \renewcommand\printchaptername{}
  \renewcommand\printchapternum{} 
  \renewcommand\chaptitlefont{\small\scshape\centering}
  \renewcommand*{\printchaptertitle}[1]{%
    \settowidth\chaptitlelength{\hspace*{1em}\chaptitlefont##1\hspace*{1em}}%
    \ifnum\chaptitlelength>\dimexpr0.7\textwidth\relax%
      \setlength\chaptitlelength{0.7\textwidth}%
    \fi%
    \setlength\chaptitlerlength{\textwidth}%
    \addtolength\chaptitlerlength{-\chaptitlelength}%
    \addtolength\chaptitlerlength{-2em}%
    \noindent\parbox[c]{.5\chaptitlerlength}{\normalsize\thickhrulefill{0.3ex}\par\vskip-1.5ex\thickhrulefill{0.2ex}}\hspace*{1em}%
    \parbox[c]{\chaptitlelength}{\chaptitlefont##1}\hspace*{1em}%
    \parbox[c]{.5\chaptitlerlength}{\normalsize\thickhrulefill{0.3ex}\par\vskip-1.5ex\thickhrulefill{0.2ex}}%
  }%
}
\makeatother
\chapterstyle{mystyle}
\chapterstyle{wilsondob}

\begin{document}
\pagestyle{Ruled}
\begin{titlingpage}
\maketitle
\end{titlingpage}

Kematian merupakan peristiwa iman. Pada saat kematian, kita mengambil 
bagian dalam misteri Paskah Kristus. Bersama Yesus Kristus kita beralih 
dari dunia fana ke dalam kehidupan kekal. Kematian adalah pintu masuk ke 
dalam pemurnian diri manusia menuju pada keabadian. Kematian juga 
menghantar kita pada kepenuhan hidup di dalam dan bersama Kristus 
Tuhan kita. 

 



%Daftar Isi 
\renewcommand{\cftchapterfont}{\normalfont\rmfamily}   
\renewcommand{\cftsectionfont}{\normalfont\rmfamily}   
\newpage
\tableofcontents



\chapter{IBADAT MENJELANG SAAT KEMATIAN}

Doa untuk mendampingi orang yang menghadapi ajal, supaya tetap 
tenang dan tabah, serta menerima kematian. 

\itshape
Catatan: 
\begin{itemize} \itemsep0pt
\item Ibadat ini hendaknya diadakan pada waktu si sakit mulai 
menghadapi sakratul maut. 
\item Mengingat keadaan si sakit, ibadat ini hendaknya dilaksanakan 
dengan khidmat dan tenang. 
\item Bila tiba-tiba si sakit dalam keadaan sakratul maut, langsung 
didoakan (doa 1 atau 2) 
\item Perlengkapan yang diperlukan: lilin, salib, dan air suci.
\end{itemize}

\normalfont
 
\section*{Tanda Salib dan Salam} 

\BP{Dalam nama Bapa dan Putera dan Roh Kudus}
\BU{Amin.} 
\BP{Semoga Tuhan memberkati saudara dan semua yang hadir di sini.} 
\BU{Sekarang dan selama-lamanya.}
 
\section*{Pemercikan dengan air suci} 

\textit{
Pemimpin ibadat memerciki si sakit, tempat tidurnya, dan semua 
yang hadir sambil berkata:} 

\BP{Ya, Tuhan, bersihkanlah saudara/i kami \nama ini dan juga kami 
semua yang hadir disekitarnya agar menjadi murni, basuhlah kami 
agar menjadi putih melebihi salju.}

\section*{Doa Pembuka}

\BP{Marilah kita berdoa: (\textit{hening sejenak})

Ya, Allah, pencipta alam 
semesta, Engkau berkuasa atas hidup dan mati. Kami menyerahkan 
diri sepenuhnya kepada kehendak-Mu yang kudus dan 
kebijaksanaan-Mu yang tak terselami. Hidup kami ini sungguh 
seperti suatu perjalanan, tujuannya adalah rumah Bapa sendiri. Lama 
perjalanan hidup kami masing-masing tidaklah sama. Maka kami 
mohon, semoga kami menerima dengan ikhlas kebijaksanaan-Mu. 
Dan bila Engkau memanggil saudara kami \nama, 
berkenanlah menganugerahi dia hidup abadi. Semoga ia Engkau 
dapati dalam keadaan siap sewaktu-waktu Engkau memanggilnya. 
Demi Yesus Kristus Putera-Mu, Tuhan dan pengantara kami, yang 
bersatu dengan Dikau dan Roh Kudus, hidup dan berkuasa kini dan 
sepanjang masa.} 

\BU{Amin.}
 
\section*{Bacaan singkat} 


\textit{Pemimpin ibadat menyerahkan salib kepada si sakit supaya 
dicium/dipegang, membesarkan hati si sakit dengan mengucapkan 
salah satu kutipan singkat ini: 
}
\begin{itemize}
\item Saudaraku tercinta, entah hidup, entah mati, kita ini milik Tuhan 
(Rom.14:8) 
\item Saudaraku tercinta, rumah kita yang abadi dan sejati ada di surga. 
(2Kor.5:1) 
\item Saudaraku tercinta, Tuhan Yesus bersabda, ”Aku menghendaki 
agar semua orang diserahkan Bapa kepada-Ku, tinggal bersama-Ku 
di tempat Aku berada”. (Yoh.17:24) 
\item Saudaraku tercinta, Tuhan Yesus bersabda, “Percayalah, hari ini 
engkau akan bersama Aku di Firdaus”. (Luk.23:43) 
\item Saudaraku tercinta, Tuhan Yesus bersabda, “Setiap orang yang 
percaya kepada Putera Manusia, akan hidup selama-lamanya”. 
(Yoh.6:40)
\end{itemize}
 
\textit{Kemudian salib diambil dan dikembalikan ketempat semula.} 

\subsection*{Mazmur: (fakultatif)} 


\BP{Saudara sekalian, marilah kita saling membesarkan hati dengan mendoakan Mazmur 90. Setiap ayat hendaklah ditanggapi dengan seruan: “Ya Tuhan, pada-Mulah aku percaya”.} 
\BP{Hendaklah orang yang berlindung pada Allah yang mahatinggi, menikmati malam yang aman dalam
naungan Tuhan.} 
\BU{Ya Tuhan, pada-Mulah aku percaya. }
\BP{Hanya Tuhanlah yang akan melepaskan dikau dari perangkap, dan melindungi engkau terhadap wabah
yang berkecamuk.} 
\BU{Ya Tuhan, pada-Mulah aku percaya. }
\BP{Ia akan menudungi engkau dangan kepaknya. Dibawah sayap-Nya engkau akan berlindung, dan
lengan-Nya akan menjadi perisai serta jabang bagi-mu. }
\BU{Ya Tuhan, pada-Mulah aku percaya. }
\BP{Engkau tak usah takut akan bahaya di waktu malam, akan panah yang mengancam di waktu siang, akan
wabah yang menular dalam kegelapan, atau akan bencana yang mengamuk di siang hari. }
\BU{Ya Tuhan, pada-Mulah aku percaya. }
\BP{Maka engkau takkan ditimpa malapetaka, dan kemahmu takkan diserang wabah, sebab Allah akan
mengutus malai-kat-Nya untuk menjaga engkau ke manapun engkau pergi.} 
\BU{Ya Tuhan, pada-Mulah aku percaya. }
\BP{Mereka akan menatang engkau dengan tangan mereka, jangan sampai kakimu tersandung pada batu.
Singa dan harimau akan kaulangkahi, ular dan naga akan kauinjak-injak.} 
\BU{Ya Tuhan, pada-Mulah aku percaya. }
\BP{Kemuliaan kepada Bapa dan Putera dan Roh Kudus. }
\BU{Seperti pada permulaan, sekarang, selalu dan sepanjang segala abad. }
\BPU{Amin.} 

\section*{Syahadat}

\BP{Saudara-saudari terkasih, bersama dan atas nama saudara 
\nama ini, marilah kita mengulangi janji baptisnya dengan 
mendoakan ringkasan iman kita }
\BU{Aku percaya akan Allah \ldots\ldots\ldots} 

\textit{Pemimpin ibadat menyalakan lilin khusus untuk dipegang oleh 
kerabat si sakit, dan berkata:} 

\BP{Tuhan, kasihanilah kami,} \BU{Tuhan, kasihanilah kami.} 
\BP{Kristus, kasihanilah kami,} \BU{Kristus, kasihanilah kami.} 
\BP{Tuhan, kasihanilah kami,} \BU{Tuhan, kasihanilah kami.} 
\BP{Marilah berdoa: (hening sejenak):
 
Ya Tuhan, terimalah hamba-Mu, saudara \nama ini, dalam 
kebahagiaan yang ia harapkan karena belas kasih-Mu.} 
\BU{Amin.} 
\BP{Ya Tuhan, luputkanlah hamba-Mu ini dari segala kema-langan.} 
\BU{Amin.} 
\BP{Ya Tuhan, selamatkanlah hamba-Mu ini, seperti Engkau 
menyelamatkan Nuh dari air bah. }
\BU{Amin. }
\BP{Ya Tuhan, bebaskanlah hamba-Mu ini dari penderitaan, seperti 
Engkau membebaskan Ayub.} 
\BU{Amin. }
\BP{Ya Tuhan, luputkanlah hamba-Mu ini dari kejahatan, seperti 
Engkau membebaskan Musa dari tangan Firaun. }
\BU{Amin.} 
\BP{Ya Tuhan, luputkanlah hamba-Mu ini dari bahaya, seperti Engkau 
membebaskan Daniel dari moncong singa.} 
\BU{Amin.} 
\BP{Ya Tuhan, luputkanlah hamba-Mu ini dari tuduhan palsu, seperti 
Engkau membebaskan Susana dari fitnahan.} 
\BU{Amin.} 
\BP{Ya Tuhan, luputkanlah hamba-Mu ini dari malapetaka, seperti 
Engkau membebaskan ketiga pemuda itu dari tanur api yang 
berkobar-kobar, dan dari tangan raja yang lalim.} 
\BU{Amin.} 
\BP{Ya Tuhan, luputkanlah hamba-Mu ini dari ancaman musuh, seperti 
Engkau membebaskan Petrus dan Paulus dari penjara.} 
\BU{Amin.} 
\BP{Allah yang maharahim, Yesus Kristus menanggung kematian bagi 
kami, dan menganugerahkan hidup abadi. Maka demi Yesus Kristus, 
luputkanlah hamba-Mu ini dari segala kemalangan. Sebab ia menaruh 
harapannya kepada Yesus, Tuhan kami, yang hidup dan berkuasa, 
kini dan sepanjang masa.} 
\BU{Amin.} 
\BP{Ya Santa Maria, Bunda Allah yang sangat murah hati, Engkaulah 
penghibur yang penuh kasih bagi orang yang berduka cita. Sudilah 
Engkau mempercayakan saudara \nama, ini kepada Yesus 
Putera-Mu. Semoga karena pertolongan-MU ia tidak takut 
menghadapi maut, sebaliknya bersama Engkau ia tabah hati 
berangkat menuju kediaman abadi. Demi Kristus, Tuhan kami, yang 
hidup dan berkuasa, kini dan sepanjang masa.} 
\BU{Amin.} 

\textit{Catatan : \\
Pada saat si sakit dalam keadaan sakratul maut, para hadirin 
berlutut di sekeliling pembaringan. Pemimpin ibadat membisikan 
“Yesus, Yesus, Yesus” pada telinga si sakit. 
Kemudian dengan suara lembut pemimpin ibadat mengucapkan 
beberapa atau semua seruan pendek-pendek di bawah ini, kalau 
mau, setiap seruan bisa diulangi oleh hadirin:} 

\begin{verse}
Tuhan, Allahku, kepada-Mu kuarahkan hatiku.\\ 
Tuhan, siapakah dapat memisahkan aku dari cinta-Mu?\\ 
Tuhan, benteng hidupku, siapakah akan kugentari? \\
Tuhan ampunilah kesalahan kami seperti kami mengampuni yang 
bersalah kepada kami.\\ 
Tuhan, ke dalam tangan-Mu kuserahkan hidupku. \\
Sekarang ya Tuhan, perkenankanlah hamba-Mu berpulang dalam 
damai. \\
Tuhan, Yesus Kristus, terimalah aku.\\ 
Tuhan, Yesus Kristus, mari datanglah. \\
Maria, Bunda rahmat, lindungilah aku disaat ajal.\\ 
Santo Yusuf, doakanlah aku. \\
Santa Maria dan Santo Yusuf, bukakan aku pangkuan kerahiman 
Tuhan. \\
Yesus, Maria dan Yusuf, tabahkan hatiku menghadapi ajal ini. \\
Yesus, Maria dan Yusuf, temanilah aku dalam sakratul maut ini. \\
Yesus, Maria dan Yusuf, biarlah aku tidur dan istirahat dalam 
ketentraman. \\
\end{verse}

\BU{Tak seorangpun hidup bagi diri sendiri, tak seorangpun mati bagi 
diri sendiri. Kita hidup dan mati bagi Allah, sebab kita milik Allah. }
\BP{Saudara-saudari, kita adalah putera-puteri Bapa di surga. Marilah 
menghayati kesatuan kita sebagai saudara dengan mengucapkan doa 
yang diajarkan Yesus sendiri.} 
\BPU{Bapa kami yang ada di surga \ldots\ldots\ldots}

\BP{Kemuliaan kepada Allah Bapa, dan Putera, dan Roh Kudus} 
\BU{Seperti pada permulaan sekarang selalu dan sepanjang segala abad} 
\BP{Terpujilah nama Yesus, Bunda Maria, dan Santo Yosef} 
\BU{Untuk selama-lamanya.} 
\BP{Dalam nama Bapa dan Putera dan Roh Kudus.} 
\BU{Amin.} 

\chapter{DOA PENYERAHAN PADA SAAT ORANG BARU MENINGGAL}
P: Pertolongan kita atas nama Tuhan. 
U: Yang menjadikan langit dan bumi. 
P: Datanglah bergegas, hai para orang kudus Allah, jemputlah hai 
para malaikat Tuhan, terimalah dia. 
U: Perkenankanlah dia datang ke hadirat Yang Mahatinggi. 
P: Semoga engkau diterima Kristus, yang telah memanggil engkau 
kembali kepangkuan-Nya. 
U: Perkenankanlah dia datang ke hadirat Yang Mahatinggi. 
P: Tuhan, berilah ia istirahat yang kekal dan semoga cahaya yang 
kekal menerangi dia. 
U: Perkenankanlah dia datang ke hadirat Yang Mahatinggi. 
P: Tuhan, kasihanilah kami. 
U: Kristus, kasihanilah kami. 
P+U: Bapa kami yang ada di surga ………………………………… 

P: Tuhan kabulkanlah doa kami. 
U: Dan seruan kami sampai ke hadirat-Mu. 

Doa: 

Bila yang meninggal orang dewasa: 

P: Marilah berdoa: (hening sejenak): Allah yang mahakuasa dan 
kekal, kami mohon kepada-Mu, terimalah hamba-Mu 
………….……, ini ke dalam kemuliaan-Mu. Ampunilah segala dosa 
dan kesalahan yang telah dibuatnya selama hidup di dunia ini. 
Biarlah dia berpulang dan beristirahat dalam damai bersama para 
kudusMu dan malaikat di surga, kini dan sepanjang masa. 
U: Amin. 
P: ………………….., di dalam iman kami kepada Yesus Kristus, 
kami melepaskan engkau pergi. Semoga kita dapat bertemu kembali 
di dalam surga abadi demi nama Bapa yang telah menciptakan 
engkau, demi nama Putera yang telah menyelamatkan engkau, demi 
nama Roh Kudus yang telah menghidupi engkau. Bunda Maria, 
jemputlah dia dan hantarlah dia kepada Yesus Putera-Mu penyelamat 
kami, yang hidup dan bertahta bersama Bapa dan Roh Kudus kini 
dan sepanjang masa. 
U: Amin. 
Bila yang meninggal orang yang masih muda: 

P: Marilah kita berdoa: (hening sejenak): Allah yang maharahim, 
Engkau telah memanggil hamba-Mu …………………., ini ke dalam 
haribaan-Mu. Terjadilah semuanya menurut kehendak-Mu. 
Perkenankanlah dia masuk dalam kemuliaan abadi. Rahasia-Mu tak 
dapat diselami, kasih-Mu yang kekal abadi kami imani demi Kristus, 
Tuhan dan pengantara kami. 
U: Amin. 
P: …………………, kami melepaskan engkau pergi menghadap 
Bapa. Karena Tuhan begitu mencintaimu, Tuhan menghendaki 
engkau kembali kepadaNya dalam usiamu yang masih muda ini. 
Terpujilah Bapa yang menciptakan engkau, Terpujilah Putera yang 
menyelamatkan engkau, Terpujilah Roh Kudus yang menghidupi 
12 



engkau, Bunda Maria, jemputlah anak-Mu dan hantarlah dia kepada 
Yesus Putera-Mu, yang hidup dan bertahta bersama Bapa dan Roh 
Kudus kini dan sepanjang masa. 

U: Amin. 
Bila yang meninggal seorang anak kecil: 

P: Marilah berdoa: (hening sejenak): Ya Allah yang mahakuasa dan 
kekal. Rahasia-Mu tak dapat kami selami, Engkau telah menghendaki 
anak kami kembali kepada-Mu dalam usia yang masih sangat muda. 
Kami serahkan saudara/i…….ini ke dalam rumahMu yang kekal 
abadi di surga. Bapa, dalam kerahiman dan kasihMu terimalah dia 
kembali ke pangkuan-Mu. Demi Kristus, Tuhan dan penyelamat 
kami. 
U: Amin. 
P: …………………, semoga engkau berpulang dengan tenang dan 
damai. Engkau tahu bahwa kami selama ini begitu mencintai dan 
mengasihi engkau. Tetapi Tuhan menghendaki engkau kembali 
kepada-Nya. Dalam iman kami percaya Tuhan mempunyai rencana-
Nya sendiri terhadap engkau. Semoga Allah Bapa menerimamu 
kembali kepangkuan-Nya, semoga Allah Putera menyelamatkanmu, 
semoga Allah Roh Kudus menerangi jalanmu kembali ke surga 
abadi. 
U: Amin. 
13 



\chapter{IBADAT SESUDAH SAAT KEMATIAN} 

PEMBUKAAN: 

I: Dalam Nama Bapa dan Putera dan Roh Kudus 
U: Amin 
P: Tuhan yang memberi, Tuhan yang mengambil, terpujilah 
nama Tuhan 
U: Sekarang dan selama-lamanya 
I: Saudara/i yang terkasih, Allah Sang Sumber dan Tujuan Hidup, 
telah memberikan kehidupan kepada saudara/i kita ……………….., 
dan kini ia telah dipanggil untuk kembali kepangkuan-Nya. Kita 
percaya bahwa seluruh hidup manusia ada ditangan Allah. Kita hanya 
bisa bersembah sujud kepada-Nya dan percaya penuh kepada 
penyelenggaran dan kehendak-Nya. Sekarang ini …………………… 
telah sampai pada akhir perjalanan hidupnya. Ia telah sampai kepada 
Sang Pencipta dan Penyelamat. Maka marilah kita mempersiapkan 
…………….. dengan ibadat perawatan jenazah, supaya dia 
menghadap Bapa di surga dengan pantas. 
I: Marilah berdoa: (hening sejenak): Allah Bapa yang Maha baik, 
Engkau menciptakan kami menurut citra-Mu dan mengangkat kami 
menjadi anak-anak-Mu berkat Putera-Mu Yesus Kristus. Baru saja 
Engkau memanggil ……………… untuk berpulang kepada-Mu di 
surga. Kini jenazah akan kami siapkan supaya ia pergi menghadap-
Mu secara pantas. Kami percaya bahwa Engkau berkenan 
memandang dan menerimanya dengan penuh kasih. Kami mohon, 
bersihkanlah ……………… dari segala dosa dan kekurangannya. 
Tuhan, berilah ia pakaian pesta dan terimalah ia dalam keluarga-Mu 
yang bahagia di surga. 
U: Amin. 
14 



BACAAN: Pembacaan dari surat Santo Paulus kepada umat di Roma 
(Rm.6:8-9) 

Jadi jika kita telah mati dengan Kristus, kita percaya bahwa kita akan 
hidup juga dengan Dia. Karena kita tahu bahwa Kristus sesudah 
bangkit dari antara orang mati, tidak mati lagi, maut tidak berkuasa 
lagi atas Dia. Demikianlah Sabda Tuhan. 

U: Syukur kepada Allah. 
P : Saudara/i marilah kita mengiringi kepergian ……..……………. 
dengan doa dan penyerahan yang ikhlas kepada Allah. Marilah kita 
memanjatkan permohonan-permohonan dengan menjawab seruanseruan 
berikut dengan “hantarkanlah dia kehadapan Allah” (hening 
sejenak). 
P : Tuhan Yesus Kristus, Engkau telah memanggil ……………….. 
sambutlah dan terimalah kedatangannya ya Yesus. 
U: Hantarkanlah dia kehadapan Allah. 
P: Para Kudus Allah datanglah menolong, para malaikat Allah 
datanglah menyongsong. 
U: Hantarkanlah dia kehadapan Allah. 
P: Para malaikat Allah bawalah dia ke pangkuan Abraham. 
U: Hantarkanlah dia kehadapan Allah. 
P: Bunda Maria ulurkanlah tanganmu, ajaklah dia menuju 
surga abadi. 
U: Hantarkanlah dia kehadapan Allah. 
P: Tuhan, berilah dia istirahat kekal dan semoga cahaya 
yang kekal menerangi perjalanan dia. 
U: Hantarkanlah dia kehadapan Allah. 
PEMIMPIN IBADAT MENGULURKAN TANGAN KEATAS 
JENAZAH 

15 



P: Allah sang pencipta segala sesuatu, janganlah lupa akan ciptaan-
Mu yang pernah Kau anugerahi hidup. Engkau sendiri yang 
memanggil ………………… untuk meninggalkan dunia ini. 
Terimalah dia di tempat kediaman-Mu, sebab dialah putera-Mu, 
ciptaan-Mu, dan milik-Mu. Ya Bapa, ia belum dapat menyelesaikan 
semua tugas selama hidup di dunia ini, tetapi Engkau sudah 
memanggilnya. Kami percaya Engkau berkenan menyempurnakan 
apa yang belum dapat diselesaikannya dalam batas waktu yang 
Engkau berikan. Kini jantungnya tidak berdenyut lagi, dan kelopak 
matanya sudah Engkau katupkan. Hanya satu yang masih 
dirindukannya, yakni Engkau, Allahnya dan Allah kami. Di dalam 
Engkaulah ya Bapa, ia akan berbahagia selama-lamanya. 
U: Tidak seorangpun hidup bagi dirinya sendiri, tidak seorangpun 
mati bagi dirinya sendiri, kita hidup dan mati bagi Allah, sebab kita 
ini milik Allah. 
P: Kami mohon pula, ya Bapa, bagi kaum kerabat dan handai taulan 
yang ditinggalkannya. Sudilah Engkau mendampingi dan 
menguatkan hati mereka sehingga dapat menerima kenyataan ini 
dengan hati yang tabah dan berserah sepenuhnya kepada-Mu. 
Semoga kami makin menyadari bahwa hidup dan mati sepenuhnya 
ada di tangan-Mu. Teguhkanlah iman kami agar senantiasa percaya 
akan kebijaksanaan-Mu, karena Kristus Tuhan, pengantara kami yang 
hidup dan berkuasa, kini dan sepanjang masa. 
U: Amin. 
PEMIMPIN IBADAT MEMERCIKI JENAZAH DENGAN AIR 
SUCI 

P: Semoga Tuhan membersihkan hatimu dan mengubah tubuhmu 
menjadi serupa dengan tubuh-Nya yang mulia. 
U: Amin. 
P: Tuhan, berikanlah saudara kami ini istirahat kekal. 
U: Amin. 
P: Semoga cahaya yang kekal menerangi perjalanan dia. 
U: Amin. 
16 



P: Semoga ia beristirahat dalam ketentraman abadi. 
U: Amin. 
Pada kematian orang dewasa: 

P: Marilah kita berdoa: (hening sejenak) Allah yang maharahim, 
kami percayakan saudara ……………. ini kepada-Mu. Ampunilah 
dengan murah hati segala dosa yang telah ia lakukan karena 
kerapuhan manusiawi. Bapa, selagi masih hidup ia menikmati apa 
yang diharapkan pada-Mu, yakni hidup bahagia bersama Engkau. 
Demi Kristus, Tuhan kami. 
U: Amin. 
Pemimpin ibadat membuat tanda salib pada dahi jenazah, kemudian 
melanjutkan: 

P: Kami mohon pula, ya Bapa, bagi kami dan handai taulan yang 
ditinggalkannya. Sudilah Engkau mendampingi dan meneguhkan 
mereka sehingga dapat menerima kenyataan ini dengan hati yang 
tabah, karena mereka percaya akan kebijaksanaan-Mu yang tidak 
kami selami. Semoga perhatian dan pertolongan dari saudara-saudari 
yang berbelasungkawa menghibur dan menguatkan hati mereka. 
Demi Kristus Tuhan kami, yang hidup dan berkuasa, kini dan 
sepanjang masa. 
U: Amin. 
Pada kematian seorang anak: 

P: Saudara-saudari yang terkasih, Tuhan Yesus pernah bersabda, 
“Biarlah anak-anak datang kepada-Ku, sebab orang-orang seperti 
merekalah yang menjadi warga kerajaan surga.” Percaya akan sabda 
ini, marilah kita berdoa. 
Pemimpin ibadat mengulurkan tangan ke atas jenazah: 

17 



P: Allah yang mahapengasih, terdorong oleh cinta-Mu, Engkau 
menciptakan anak ini dan menyerahklan dia kepada orang tuanya 
supaya dijaga dan dipelihara seturut kehendak-Mu. Belum puas 
mereka mengasuh dan membesarkan dia, namun kehendak-Mu yang 
bijaksana memutuskan bahwa anak ini harus pulang ke rumah-Mu. 
Maka kami menyerahkan dia kepada-Mu dengan ikhlas. Sudilah 
menerima anak ini dalam kedamaian rumah-Mu yang abadi. 
Perkenankanlah ia bersuka cita dan bergembira ria bersama para 
malaikat dan orang kudus-Mu. Demi Kristus Tuhan kami. 
U: Amin. 
Pemimpin ibadat membuat tanda salib pada dahi jenazah, kemudian 
melanjutkan: 

P: Allah Bapa di surga, kami berdoa juga bagi semua yang 
ditinggalkan anak ini. Teguhkanlah kepercayaan mereka, dan sudilah 
menghibur ayah-ibu serta seluruh keluarga. Kuatkanlah mereka 
dengan teladan Bunda Maria yang telah mengikhlaskan puteranya 
memenuhi kehendak-Mu. Ya Bapa tambahlah kepercayaan kami, dan 
teguhkanlah harapan kami. Demi Kristus Tuhan kami, yang hidup 
dan berkuasa kini dan sepanjang masa. 
U: Amin. 
DOA PENUTUP: 
P: Semoga Tuhan membuka pintu surga bagi putera-Nya yang pulang 
ke rumah Bapa di Kerajaan Allah. Disana tiada duka cita, tetapi 
hanya ada kebahagiaan untuk selama-lamanya. Bapa, berilah dia 
istirahat yang kekal dalam ketenteraman abadi bersama Engkau di 
surga. 
U: Amin. 
BERKAT PENGUTUSAN 

P: Saudara-saudari terkasih dalam Kristus, sampai disinilah ibadat 
kita guna mendampingi saudara ……………… pada saat mulia, 
peralihannya ke rumah Bapa. Marilah kita membantu keluarga yang 
18 



mengalami kesusahan ini dengan menghibur mereka, dengan 
mengurus jenazah, dan mengatur pemakamannya. Semoga dalam 
seluruh pelaksanaan ini nanti kita diliputi berkat Allah yang 
mahakuasa, Bapa dan Putera dan Roh Kudus. 

U: Amin. 
Note: Sambil menunggu jenazah untuk dimandikan, dapat didaraskan 
Mazmur atau doa rosario. 

19 



\chapter{IBADAT MERAWAT / MEMANDIKAN JENAZAH} 

PEMBUKAAN: 

P: Dalam nama Bapa dan Putera dan Roh Kudus. 
U: Amin 
P: Tuhan yang memberi, Tuhan juga yang akan mengambil, 
terpujilah nama Tuhan. 
U: Sekarang dan selama-lamanya. 
P: Saudara/i yang terkasih, Allah Sang Sumber dan Tujuan Hidup, 
yang telah memberikan kehidupan kepada saudara/i kita 
……………….., dan kini dia telah dipanggil untuk kembali 
kepangkuan-Nya. Kita percaya bahwa seluruh hidup manusia ada 
ditangan Allah. Kita hanya bisa bersembah sujud dan percaya kepada 
Penyelenggaraan serta kehendak-Nya. Sekarang ini.........telah sampai 
pada akhir perjalanan hidupnya. Ia telah sampai kepada Sang 
Pencipta dan Penyelamat. Maka marilah kita mempersiapkan 
…………….. dengan ibadat pemandian jenazah, agar dia pantas 
untuk menghadap Bapa di surga. 
P: Bersihkanlah hamba-Mu ini ya Tuhan dari segala dosanya. 
U: Dan basuhlah dia agar pantas menghadap-Mu. 
Catatan : pihak keluarga, sahabat, handai taulan dapat mencurahkan 
air ke jenazah mulai dari kepala ke arah kaki, sebelum petugas 
melaksanakan pemandian jenazah dengan hormat dan khidmat. 
Hendaknya petugas menyapa dan mohon permisi kepada orang yang 
meninggal agar diizinkan untuk memandikannya. 
Selama menunggu jenazah dimandikan dapat didaraskan Mazmur 
atau doa rosario. 

20 



\chapter{IBADAT MENGENAKAN PAKAIAN} 
P: Dalam nama Bapa dan Putera dan Roh Kudus 
U: Amin 
P: Saudara/i yang terkasih, Allah Sang Sumber dan Tujuan Hidup, 
yang telah memberikan kehidupan kepada saudara/i kita 
……………….., dan kini telah dipanggil untuk kembali kepangkuan-
Nya. Kita percaya bahwa seluruh hidup manusia ada ditangan Allah. 
Kita hanya bisa bersembah sujud kepada-Nya dan percaya kepada 
penyelenggaraan serta kehendak-Nya. Sekarang ini 
…………………… telah sampai pada akhir perjalanan hidupnya. Ia 
telah sampai kepada Allah Bapa Sang Pencipta dan Penyelamat kita. 
Maka marilah kita mempersiapkan …………….. dengan 
mengenakan pakaian pesta, agar dia menghadap Bapa di surga 
dengan pantas. 
P: Saudara/i ……………….. kenakanlah pakaian pesta ini untuk 
menghadap Tuhan dengan pantas. 
U: Amin. 

\chapter{IBADAT MEMASUKKAN JENAZAH KE PETI} 
LAGU PEMBUKAAN: (PS. 712, atau yang lain) 

PEMBERKATAN PETI: 

P: Dalam nama Bapa dan Putera dan Roh Kudus 
U: Amin 
P: Ya Allah, berkatilah peti ini yang akan menjadi tempat 
pembaringan …………………. Semoga peti pembaringan ini juga 
menghadirkan tempat kediaman-Mu sendiri, sehingga …………….. 
yang telah Engkau panggil dapat beristirahat dalam damai-Mu. 
U: Amin. 
21 



* Peti diperciki dengan air suci. 
P: Dalam nama Bapa dan Putera dan Roh Kudus. 
U: Amin. 
P: …………. Pulanglah dan beristirahatlah dalam damai Tuhan. 
U: Amin. 
* Jenazah dimasukkan kedalam peti, dapat diiringi dengan nyanyian 
atau mendaraskan Mazmur atau doa rosario. 
DOA PENUTUP: 

P : Saudara/i yang terkasih, marilah kita menutup seluruh rangkaian 
upacara ini dengan doa yang diajarkan oleh Tuhan kita. Bapa kami 
yang ada disurga …………………………
. 
P : Kita serahkan …………………..……………. Kepada Bunda 
Maria. Salam Maria penuh rahmat.....
. 
Kemuliaan kepada Bapa dan Putera dan Roh Kudus. 
U: Seperti pada permulaan, sekarang dan sepanjang segala abad. 
Amin. 
P: Ya Tuhan, terimalah hamba-Mu ini kedalam kerajaan-Mu. 
U: Dan perkenankanlah dia beristirahat dalam damai-Mu. 
P: Semoga semua orang yang sudah meninggal beristirahat dan 
mendapat ketentraman karena kerahiman Tuhan. 
U: Amin. 
LAGU PENUTUP: (PS. 714, atau yang lain). 

22 



\chapter{IBADAT KEMATIAN I}


PEMBUKAAN 
LAGU PEMBUKAAN: 
TANDA SALIB DAN SALAM: 

I Dalam Nama Bapa dan Putera dan Roh Kudus. 
U Amin 
I Semoga Allah yang telah membangkitkan Yesus Kristus, 

PuteraNya dari alam maut, melimpahkan penghiburan dan kekuatan 
iman kepada kita sekalian. 

U Sekarang dan selama-lamanya 

KATA PEMBUKAAN: 

PERNYATAAN TOBAT 

Bapak/ibu, saudara/i terkasih dalam Tuhan, agar Perayaan 
Ekaristi yang kita rayakan ini sungguh mendatangkan rahmat dan 
berkat bagi kita semua, terutama bagi …………. yang sudah 
mendahului kita dalam imannya, kekuatan dan penghiburan iman 
bagi semua anggota keluarga yang ditinggalkan dan bagi kita semua 
yang ambil bagian dalam Perayaaan ini, maka marilah kita dengan 
rendah hati memohon ampun dari Tuhan untuk semua dosa dan 
kesalahan yang sudah kita lakukan. Atas nama ……., marilah kita 
juga memohon ampun dari Tuhan atas segala dosa dan kesalahan 
yang telah ia lakukan selagi ia masih hidup bersama dengan kita. 
(hening sejenak). 

I+U Saya mengaku kepada Allah yang Maha Kuasa dan kepada 
saudara sekalian, bahwa saya telah berdosa dengan pikiran dan 
perkataan, dengan perbuatan dan kelalaian. Saya berdosa, saya 

23 



berdosa, saya sungguh berdosa. Oleh sebab itu saya mohon kepada 
Santa Perawan Maria, kepada para malaikat dan orang kudus dan 
kepada saudara sekalian, supaya mendoakan saya pada Allah, Tuhan 
kita. 

I Semoga Allah yang Maha kuasa mengasihani kita, mengampuni 
dosa kita dan mengantar kita ke hidup yang kekal. 
U Amin 

LAGU TUHAN KASIHANILAH KAMI: 

(Kalau didaraskan/didoakan): 

I Tuhan, kasihanilah kami. I Kyrie eleison. 
U Tuhan, kasihanilah kami. U Kyrie eleison. 

I Kristus, kasihanilah kami. I Christe eleison 
U Kristus, kasihanilah kami. U Christe eleison. 

I Tuhan, kasihanilah kami. I Kyrie eleison. 
U Tuhan, kasihanilah kami. U Kyrie eleison. 

DOA PEMBUKA: 

I Marilah berdoa: 
Allah dan Bapa kami yang Maha Kuasa dan kekal, sungguh kuatlah 
keadilanMu dan besarlah kerahimanMu. Pandanglah kami semua 
yang hadir di sini yang dengan penuh syukur mau menghormati 
kehidupan dan kematian hambaMu: ……….…….. Kami bersyukur 
kepadaMu karena kami sebagai anak-anak, cucu-cucu dan 
keluarganya, boleh mengalami cinta dan kasih setiaMu melalui 
persaudaraan, persahabatan dan kerukunan serta perdamaian yang 
telah ditunjukkan oleh orang tua kami ini. Melalui penderitaan yang 
dia alami, ia telah mengajarkan kami ketaatan kepada kehendakMu 
dan senantiasa bersiap-siap untuk menanti kedatanganMu. Kami 
mohon ya Tuhan, agar tiada sesuatu pun dari karya amal baktinya di 

24 



dunia ini, akan hilang begitu saja. Tetapi semoga segala amal 
perbuatannya mendatangkan kebaikan bagi kami, dan apa saja yang 
dianggap suci dan kudus dan baik olehnya, dihormati dan diindahkan 
oleh orang-orang yang ditinggalkan: teristimewa oleh anak-anak dan 
cucu-cucunya serta keluarga dan sahabat kenalan. Ya Tuhan, semoga 
dalam segala hal yang membuat pribadinya besar, tetap tinggal dan 
berbicara kepada kami, lebih-lebih setelah ia meninggalkan kami. 
Bantulah kami agar semakin menyadari rencana kasih sayangMu 
yang ajaib, sehingga kami semakin menyerahkan hidup serta 
kemampuan kami untuk mengabdi dikau dan sesama kami menurut 
teladan hidup……….yang kami doakan saat ini. Dengan 
pengantaraan Yesus Kristus, PuteraMu, Tuhan kami, yang bersama 
dengan Dikau dalam persatuan Roh Kudus, hidup dan berkuasa, 
Allah, sepanjang segala masa. 
U Amin 

LITURGI SABDA 

BACAAN I: Pembacaan dari Wahyu 21: 1-5A, 6B-7 

LIHATLAH, AKU MENJADIKAN SEGALA SESUATU BARU 

L 21:1 Lalu aku melihat langit yang baru dan bumi yang baru, 
sebab langit yang pertama dan bumi yang pertama telah berlalu, dan 
laut pun tidak ada lagi. 21:2 Dan aku melihat kota yang kudus, 
Yerusalem yang baru, turun dari sorga, dari Allah, yang berhias 
bagaikan pengantin perempuan yang berdandan untuk suaminya. 

21:3 Lalu aku mendengar suara yang nyaring dari takhta itu berkata: 
“Lihatlah, kemah Allah ada di tengah-tengah manusia dan Ia akan 
diam bersama-sama dengan mereka. Mereka akan menjadi umat-Nya 
dan Ia akan menjadi Allah mereka. 21:4 Dan Ia akan menghapus 
segala air mata dari mata mereka, dan maut tidak akan ada lagi; tidak 
akan ada lagi perkabungan, atau ratap tangis, atau dukacita, sebab 
segala sesuatu yang lama itu telah berlalu”. 21:5a Ia yang duduk di 
atas takhta itu berkata: “Lihatlah, Aku menjadikan segala sesuatu 
baru!.” 21:6b Aku adalah Alfa dan Omega, yang Awal dan yang 
25 



Akhir. Orang yang haus akan Kuberi minum dengan Cuma-Cuma 
dari mata air kehidupan. 21:7 Barangsiapa menang, ia akan 
memperoleh semuanya ini, dan Aku akan menjadi Allahnya dan ia 
akan menjadi anak-Ku”. (hening sejenak) 
Demikianlah Sabda Tuhan 

U: Syukur kepada Allah 
LAGU ANTAR BACAAN: 

BACAAN INJIL: YOHANES 14: 1-7 

I Tuhan sertamu 
U Dan sertamu juga 

I Inilah Injil Yesus Kristus menurut Santo Yohanes 
U Dimuliakanlah Tuhan 

I AKULAH JALAN DAN KEBENARAN DAN HIDUP 

14:1 Pada waktu itu, Yesus bersabda, “Janganlah gelisah hatimu; 
percayalah kepada Allah, percayalah juga kepada-Ku. 14-2 Di rumah 
Bapa-Ku ada banyak tempat tinggal. Jika tidak demikian, tentu Aku 
mengatakannya kepadamu. Sebab Aku pergi ke situ untuk 
menyediakan tempat bagimu. 14:3 Dan apabila Aku telah pergi ke 
situ dan telah menyediakan tempat bagimu, Aku akan datang kembali 
dan membawa kamu ke tempat-Ku, supaya di tempat di mana Aku 
berada, kamu pun berada. 14:4 Dan ke mana Aku pergi, kamu tahu 
jalan ke situ”. 14:5 Kata Tomas kepada-Nya: ”Tuhan, kami tidak 
tahu ke mana Engkau pergi; jadi bagaimana kami tahu jalan ke situ?” 
14:6 Kata Yesus kepadanya: “Akulah jalan dan kebenaran dan hidup. 
Tidak ada seorang pun yang datang kepada Bapa, kalau tidak melalui 
Aku. 14:7 Sekiranya kamu mengenal Aku, pasti kamu juga mengenal 
Bapa-Ku. Sekarang ini kamu mengenal Dia dan kamu telah melihat 
Dia.” 
26 



I Berbahagialah orang yang mendengarkan Sabda Tuhan dan 
tekun melaksanakannya. 
U SabdaMu adalah jalan, kebenaran dan hidup kami. 


HOMILI: 


DOA UMAT: 


I Ibu-Bapak, saudara-saudariku sekalian yang terkasih. Marilah 
kita berdoa kepada Allah, Bapa yang Mahakuasa, yang 
membangkitkan PuteraNya dari alam maut sebagai jaminan hidup 
abadi bagi semua orang beriman. Kristus sendiri telah menjadi jalan, 
kebenaran dan kehidupan; dan melalui Dia kita dapat sampai kepada 
Bapa. Oleh karena itu, marilah kita menyampaikan doa-doa 
permohonan kita: 


L1 (Dibawakan oleh salah seorang anak dari almarhumah): Allah 
Bapa yang maha rahim, Engkau yang memberi, Engkau pula yang 
mengambil. Kami mohon ke hadiratMu, sudilah kiranya Engkau 
menerima arwah saudar/i kami terkasih......................
. 
Anugerahkanlah rahmat berlimpah kepadanya, terimalah dia dan 
satukan dia dalam persekutuan para kudusMu di surga. Semoga 
Engkau berkenan agar ia menjadi perantara doa-doa kami anakanaknya 
ke hadiratMu. Marilah kita memohon ………..
. 
U Kabulkanlah doa kami ya Tuhan 


L2 (Dibawakan oleh salah seorang mantu almarhumah): Allah 
sumber suka cita kami, Engkau telah menerima hambaMu 
………………. sebagai pengikut puteraMu yang setia sepanjang 
hidupnya. Ia hidup dalam kesederhanaan, penuh iman dan cinta 
dalam pengabdian kepada sesama. Terima kasih Bapa, ia telah 
menjadi pemersatu, contoh dan teladan bagi kami semua 
keluarganya. Semoga berkat anugerahMu kami dikuatkan untuk 
melanjutkan hidup kami dalam pengabdian bagi sesama dan keluarga 
mengikuti jejak langkah almarhumah, sesuai dengan kehendakMu 
sendiri. Marilah kita memohon ……...
. 


27 



U Kabulkanlah doa kami ya Tuhan 


L3 (Dibawakan oleh salah seorang cucu almarhumah – jika 
almarhumah mempunyai cucu): Tuhan Yesus Kristus, Engkau pernah 
bersabda kepada penjahat yang disalibkan di sampingMu. “Hari ini 
juga Engkau akan berada bersama dengan Aku dalam Firdaus.” 
Bersabdalah juga demikian kepada…….kami ini. Ampunilah dosa 
dan kesalahan yang pernah diperbuatnya. Sambutlah tangannya, 
hantarkanlah dia masuk ke dalam kerajaan abadi, yang diliputi 
suasana damai dan penuh suka cita. Semoga kami cucu-cucunya 
hidup rukun selalu, kasih mengasihi, tolong menolong, sesuai dengan 
cita-cita dan harapan almarhumah. Bapa surgawi, terimalah arwah 
……………… .kami ini ke dalam kerajaanMu. Marilah kita 
memohon……… 
U Kabulkanlah doa kami ya Tuhan 

L4 (Dibawakan oleh salah seorang undangan): Bagi keluarga yang 
ditinggalkan. Semoga peristiwa duka ini tidak menggoyahkan iman 
mereka. Semoga mereka dan kami semua yang hadir di sini 
menyadari bahwa hidup dan kematian kami sepenuhnya berada 
dalam tanganMu. Ya Bapa, semoga mereka semakin percaya bahwa 
Engkaulah Pemberi hidup dan Engkau pulalah yang berhak 
mengambil hidup kami. Kuatkanlah iman mereka dan kami semua. 
Semoga semangat persatuan, keakraban dan persaudaraannya tetap 
kokoh terpelihara di antara kami. Marilah kita memohon ……..… 
U Kabulkanlah doa kami ya Tuhan 

U5 (Dibawakan oleh ketua lingkungan atau yang mewakili): Marilah 
kita berdoa juga untuk semua orang yang telah berbuat baik kepada 
almarhumah dan keluarganya. Mereka telah menyatakan kasih dan 
perhatiannya dengan cara mereka masing-masing kepada 
almarhumah………….. Ya Bapa, semoga Engkau juga memberikan 
ganjaran berlimpah kepada mereka. Berkatilah mereka semua, 
keluarga dan segala karya amal bakti hidup mereka sehingga mereka 
selalu berada dalam perlindunganMu. Marilah kita memohon 
………….. 

28 



U Kabulkanlah doa kami ya Tuhan 


I Allah yang Maha rahim, melalui Yesus Kristus PuteraMu, 
Engkau menganugerahkan kami terang yang abadi. Buatlah hati kami 
semakin terbuka terhadap terang yang menjadi pedoman hidup kami. 
Sudilah memberikan kami keberanian serta kekuatan untuk terus 
berjalan menuju cahaya kemuliaanMu yang abadi. Demi Kristus, 
Tuhan dan Pengantara kami. 
U Amin 


BAPA KAMI: 


I Atas petunjuk Penyelamat kita dan menurut ajaran ilahi, maka 
beranilah kita berdoa 


U Bapa kami yang ada di surga, dimuliakanlah namaMu, datanglah 
kerajaanMu, jadilah kehendakMu, di atas bumi seperti di dalam 
surga. Berilah kami rezeki pada hari ini dan ampunilah kesalahan 
kami, seperti kami pun mengampuni yang bersalah kepada kami, dan 
janganlah masukkan kami ke dalam percobaan, tetapi bebaskanlah 
kami dari yang jahat 


I Ya Bapa, bebaskanlah kami dari segala yang jahat dan berilah 
kami damaiMu. Kasihanilah dan bantulah kami, supaya selalu bersih 
dari noda dosa dan terhindar dari segala gangguan, sehingga kami 
dapat hidup dengan tenteram, sambil mengharapkan kedatangan 
Penyelamat kami, Yesus Kristus. 


U Sebab Engkaulah Raja yang mulia dan berkuasa untuk selamalamanya. 


PEMBERKATAN JENAZAH: 


29 



I: Ibu-Bapak, saudara-saudariku yang terkasih dalam Kristus, 
Tuhan, sebentar lagi kita akan berpisah secara jasmani dengan 
…………..… yang kita kasihi ini. Maka dengan hati yang tabah, kita 
memberikan penghormatan yang terakhir kepadanya, karena kita 
berharap bahwa ia akan bangkit bersama Kristus yang telah 
diimaninya untuk kehidupan yang kekal. Maka air suci akan direciki 
di atas dia sebagai lambang pembaptisannya dan jenazahnya akan 
didupai, supaya keharuman amal baktinya di dunia ini berkenan 
kepada Tuhan. 
I: Ya Tuhan, siramilah hambaMu ini: …………………., yang 
masuk ke alam kekal, dengan air kehidupan 
U: Terimalah dia, ya Tuhan 
I: Supaya ia hidup bahagia selamanya bersama para kudusMu 
dalam kerajaan surga 
U: Terimalah dia, ya Tuhan 
I: Dari bumi aku berseru kepadaMu, ya Tuhan. Sudilah Engkau 
mendengarkan seruanku 
U: Terimalah dia, ya Tuhan 
I: PadaMu, ya Tuhan, ada pengampunan dosa agar semua orang 
mengabdi kepadaMu dengan hati yang gembira 
U: Terimalah dia, ya Tuhan 
I: Aku percaya kepadaMu, ya Tuhan; jiwaku percaya akan 
SabdaMu 
U: Terimalah dia, ya Tuhan 
I: Pada-Mu, ya Tuhan, ada belaskasihan serta penebusan 
berlimpah ada pada-Mu 
U: Terimalah dia, ya Tuhan 
I: Hai para kudus dan para malaikat Allah, datanglah menyongsong 
……………... ini dan hantarkanlah dia kepada Kristus. 
30 



U: Di hadapan Allah yang Mahatinggi. 
I: Semoga Kristus menyambutmu, sebab Dia-lah yang telah 
memanggil engkau. Semoga para malaikat mengiringi dan 
menjemput engkau ke pangkuan Abraham 
U: Di hadapan Allah yang Mahatinggi 
I: Tuhan, berilah dia istirahat kekal dan sinarilah dia dengan cahaya 
abadi 
U: Di hadapan Allah yang Mahatinggi 
I: Marilah berdoa: 
Ya Tuhan, kehidupan dan kematian kami berada di dalam 
tanganMu sendiri. Engkau telah menciptakan manusia karena kasih 
dan cintaMu. Ya Tuhan, lihatlah kami putera-puteri-Mu yang kini 
berhimpun di sekitar peti jenazah dari ……………... yang kami 
kasihi ini. Kami semua berduka cita atas kematiannya. Maafkan kami 
ya Tuhan, jika kami belum sempat mengucapkan terima kasih kami 
kepadanya atas segala kebaikan yang telah dilakukannya kepada 
kami. Kami juga menyesal jika kami belum sempat meminta maaf 
atas segala dosa dan kesalahan yang telah kami lakukan terhadapnya. 
Akan tetapi kami percaya bahwa kasihMu jauh lebih kuat daripada 
keinginan manusiawi kami. Kami mohon berkat belaskasihan-Mu 
kepada …………………………, janganlah Engkau serahkan dia 
kepada kekuasaan maut, tetapi bebaskanlah dia demi jasa Kristus, 
Putera-Mu. Biarkanlah darah dan air yang tercurah dari lambung 
Kristus, PuteraMu itu membersihkan dia dari segala dosa dan 
kesalahannya sehingga ia, dengan jiwa yang bersih dapat menghadap 
Engkau, Penciptanya, Bapanya dan juga Bapa kami bersama. Tuhan, 
dalam kehidupan di dunia ini, ia telah dikuatkan dan disegarkan 
dengan santapan Tubuh dan Darah Kristus Putera-Mu. Maka kami 
mohon, perkenankanlah ia kini mengambil bagian dalam perjamuan 
surgawi-Mu. Disanalah kami semua akan dipertemukan kembali 
untuk memuji dan memuliakan dikau dalam keabadian. Demi Kristus 
Tuhan dan Pengantara kami. 

U: Amin. 
31 



I: Saudara/i …………..…….. terima kasih atas segala 
kebaikan, jasa dan pengorbanan hidupmu yang telah engkau lakukan 
selagi masih hidup bersama dengan kami. Atas segala tanda kasihmu 
itu, kami hanya sanggup mengucapkan terima kasih dan selamat 
jalan, bawalah selalu tanda kemenangan Kristus: Dalam nama Bapa 
dan (+) Putera, dan Roh Kudus. 
U: Amin. 
(Lalu jenazah direciki dengan air suci dan didupai. Keluarga 
diperkenankan untuk menyirami jenazah dengan minyak wangi yang 
sudah diberkati) 

DOA PENUTUP: 

I : Marilah berdoa: 
Allah Tuhan kami, Engkaulah sumber keselamatan kekal dan cahaya 
abadi. Perayaan Ekaristi sebagai peringatan wafat dan kebangkitan 
Yesus Kristus PuteraMu telah kami rayakan pada saat ini untuk 
memohon keselamatan kekal dan bahagia akhirat 
bagi………………….yang telah Engkau panggil menghadap 
hadiratMu. Kami mohon, perkenankanlah kiranya ia masuk 
rumahMu, tempat kami boleh menikmati kebahagiaan kekal bersama 
dengan Engkau sendiri. Dan kuatkanlah iman serta harapan kami 
akan kehadiran-Mu yang senantiasa menghantar kami kepada 
keselamatan kekal. Demi Kristus Tuhan kami. 
U : Amin. 
BERKAT PENUTUP 

I : Saudara/i terkasih, marilah kita akhiri perayaan ekaristi kita ini 
dengan memohon berkat Tuhan. 
I : Tuhan sertamu 
U : Dan sertamu juga 
32 



I : Semoga Allah Bapa yang Maha kuasa dan kekal senantiasa 
memberikan rahmat dan anugerah kehidupan bagi kita semua agar 
kita selalu memuji dan memuliakan nama-Nya yang Kudus. 
U : Amin 
I : Semoga karena kebangkitan Kristus dari alam maut, Allah 
berkenan menguatkan hati kita agar dapat menolak segala dosa dan 
senantiasa berjuang untuk hidup baik. 
U : Amin 
I : Semoga karena kemurahan dan belaskasihanNya, Allah berkenan 
menganugerahkan kita kebahagiaan kekal di surga. 
U : Amin 
I : Dan semoga kita sekalian, semua yang kita doakan serta seluruh 
karya amal bakti hidup kita senantiasa dilindungi, dibimbing dan 
diberkati oleh Allah Yang Maha Kuasa: Bapa, dan (+) Putera, dan 
Roh Kudus. 
U : Amin 
I : Saudara/i terkasih, perayaan Ekaristi dalam rangka mendoakan 
keselamatan kekal …………..., telah selesai 
U : Syukur Kepada Allah 
I : Hiduplah dalam kasih Tuhan 
U : Amin 
LAGU PENUTUP : 

Requiescat In pace 

33 



\chapter{IBADAT KEMATIAN 2} 

PEMBUKAAN 

LAGU PEMBUKAAN: 

TANDA SALIB DAN SALAM: 

I : Dalam Nama Bapa dan Putera dan Roh Kudus 
U : Amin 
I : Semoga Allah yang telah membangkitkan Yesus Kristus, 
PuteraNya dari alam maut, melimpahkan penghiburan dan kekuatan 
iman kepada kita sekalian. 
U : Sekarang dan selama-lamanya. 
KATA PEMBUKAAN: 

PERNYATAAN TOBAT: 

I : Bapak/ibu, saudara/i terkasih dalam Tuhan, agar Perayaan Ekaristi 
yang kita rayakan ini sungguh-sungguh mendatangkan rahmat dan 
berkat bagi kita semua, terutama bagi ………….… yang sudah 
mendahului kita dalam imannya, maka marilah kita dengan rendah 
hati memohon ampun kepada Tuhan untuk semua dosa dan kesalahan 
yang sudah kita lakukan. Atas nama ……………., marilah kita juga 
memohon ampun dari Tuhan atas segala dosa dan kesalahan yang 
telah ia lakukan selagi ia masih hidup bersama dengan kita. 
(hening sejenak) 
I : Ya Tuhan Yesus Kristus, Engkau mengalami kematian sebagai 
manusia, tetapi dibangkitkan oleh kekuasaan Bapa dalam Roh Kudus. 
Tuhan, kasihanilah kami. 
U : Tuhan, kasihanilah kami 
34 



I : Engkaulah kebangkitan dan kehidupan; barangsiapa percaya 
kepadaMu akan memperoleh kehidupan yang kekal. Kristus, 
kasihanilah kami. 
U : Kristus, kasihanilah kami 
I : Engkau akan datang dengan mulia untuk mem-persatukan kami 
semua dalam kerajaan Bapa. 
Tuhan, kasihanilah kami 

U : Tuhan, kasihanilah kami 
I : Semoga Allah yang Mahakuasa mengasihani kita, mengampuni 
dosa-dosa kita dan menghantar kita ke hidup yang kekal. 
U : Amin 
LAGU TUHAN KASIHANILAH KAMI 

DOA PEMBUKA: 

I : Marilah Berdoa: 
Allah yang kekal dan kuasa, barangsiapa berdoa kepadaMu dengan 
hati yang ikhlas, tidak pernah engkau kecewakan. Pada hari ini kami 
berkumpul untuk mendoakan hambaMu …………………. yang telah 
Engkau panggil menghadap hadiratMu. Terima kasih ya Bapa, atas 
rahmat dan cinta yang kami rasakan selama kami hidup bersama 
dengan dia. Dengarkanlah doa kami dan curahkanlah belaskasihMu 
kepadanya. Ia telah meninggal sebagai pengikut Kristus, PuteraMu, 
semoga ia tetap bersatu dengan Kristus dalam kebahagiaan kekal 
bersama para kudus di surga. 
Kami berdoa pula bagi keluarga, kaum kerabat dan handai taulan 
yang ditinggalkannya. Sudilah Engkau mendampingi dan 
meneguhkan mereka sehingga dapat menerima kenyataan ini dengan 
tabah hati, karena mereka percaya akan kebijaksanaanMu yang tidak 
terselami. Semoga perhatian dan pertolongan dari saudara-saudari 
yang berbelasungkawa menghibur dan menguatkan hati mereka. Doa 
ini kami panjatkan kepadaMu dengan pengantaraan Kristus 
35 



PuteraMu, Tuhan dan Pengantara kami yang hidup dan berkuasa, 
bersama dengan Dikau dalam persekutuan dengan Roh Kudus, Allah 
kini dan sepanjang segala masa. 

U:Amin 

LITURGI SABDA: 

BACAAN I: Pembacaan dari I TESALONIKA, 4:13-18 

JIKALAU KITA PERCAYA, BAHWA YESUS TELAH MATI 
DAN TELAH BANGKIT, MAKA KITA PERCAYA JUGA 
BAHWA MEREKA YANG TELAH MENINGGAL DALAM 
YESUS AKAN DIKUMPULKAN ALLAH BERSAMA-SAMA 
DENGAN DIA 

I : 4:13 Selanjutnya kami tidak mau, saudara-saudara, bahwa kamu 
tidak mengetahui tentang mereka yang meninggal, supaya kamu 
jangan berdukacita seperti orang-orang lain yang tidak mempunyai 
pengharapan, 4:14 Karena jikalau kita percaya, bahwa Yesus telah 
mati dan telah bangkit, maka kita percaya juga bahwa mereka yang 
telah meninggal dalam Yesus akan dikumpulkan Allah bersama-sama 
dengan Dia. 4:15 Ini kami katakan kepadamu dengan firman Tuhan: 
kita yang hidup, yang masih tinggal sampai kedatangan Tuhan, 
sekali-kali tidak akan mendahului mereka yang telah meninggal. 4:16 
Sebab pada waktu tanda diberi, yaitu pada waktu penghulu malaikat 
berseru dan sangkakala Allah berbunyi, maka Tuhan sendiri akan 
turun dari sorga dan mereka yang mati dalam Kristus akan lebih 
dahulu bangkit: 4:17 sesudah itu, kita yang hidup, yang masih 
tinggal, akan diangkat bersama-sama dengan mereka dalam awan 
menyongsong Tuhan di angkasa. Demikianlah kita akan selamalamanya 
bersama-sama dengan Tuhan. 4:18 Karena itu hiburkanlah 
seorang akan yang lain dengan perkataan-perkataan ini. Demikian 
Sabda Tuhan 
U : Syukur kepada Allah 
36 



LAGU ANTAR BACAAN 

BACAAN INJIL: Yohanes 11:17-27 

I : Tuhan sertamu 
U : Dan sertamu juga 
I : Inilah Injil Suci Yesus Kristus menurut Santo Yohanes 
U : Dimuliakanlah Tuhan 
AKULAH KEBANGKITAN DAN HIDUP BARANGSIAPA 
PERCAYA KEPADA-KU IA AKAN HIDUP WALAUPUN IA 
SUDAH MATI 

I : 11:17 Maka ketika Yesus tiba di Betania, didapati-Nya Lazarus 
telah empat hari berbaring di dalam kubur. 11:18 Betania terletak 
dekat Yerusalem, kira-kira dua mil jauhnya. 11:19 Di situ banyak 
orang Yahudi telah datang kepada Marta dan Maria untuk menghibur 
mereka berhubung dengan kematian saudara mereka: Lazarus. 11:20 
Ketika Marta mendengar, bahwa Yesus datang, ia pergi 
mendapatkan-Nya. Tetapi Maria tinggal di rumah. 11:21 Maka kata 
Marta kepada Yesus: ”Tuhan, sekiranya Engkau ada di sini, 
saudaraku pasti tidak mati. 11:22 Tetapi sekarang pun aku tahu, 
bahwa Allah akan memberikan kepada-Mu segala sesuatu yang 
Engkau minta kepada-Nya.” 11:23 Kata Yesus kepada Marta: 
”Saudaramu akan bangkit.” 11:24 Kata Marta kepada-Nya: ”Aku 
tahu bahwa ia akan bangkit pada waktu orang-orang bangkit pada 
akhir zaman.” 11:25 Jawab Yesus: ”Akulah kebangkitan dan hidup; 
barangsiapa percaya kepada-Ku, tidak akan mati selama-lamanya. 
Percayakah engkau akan hal ini?” 11:27 Jawab Marta: ”Ya Tuhan, 
aku percaya, bahwa Engkaulah Mesias, Anak Allah, Dia yang akan 
datang ke dalam dunia.” 
I : Berbahagialah orang yang mendengarkan Sabda Tuhan dan tekun 
menghayatinya 
U : SabdaMu adalah jalan, kebenaran dan hidup kami. 
37 



HOMILI 

DOA UMAT: 

I : Marilah kita mendoakan ……yang kita kasihi ini kepada Tuhan 
kita Yesus Kristus yang telah menjadi jalan, kebenaran dan 
kehidupan yang menghantar setiap manusia memasuki kehidupan 
yang kekal. 
P1 : Tuhan Yesus Kristus, Engkau telah menangisi kematian Lazarus. 
Semoga Engkau mengusap air mata duka kami. Marilah kita 
memohon …………. 

U : Kabulkanlah doa kami, ya Tuhan 
P2 : Engkau telah menghidupkan kembali pemuda di Nain. Maka 
anugerahkanlah hidup kekal kepada …………….. ini. Marilah kita 
memohon ……..… 

U : Kabulkanlah doa kami, ya Tuhan 
P3 : Engkau menjanjikan firdaus kepada penyamun yang bertobat. 
Terimalah…dalam kebahagiaan kekal di surga. Marilah kita 
memohon ………… 

U : Kabulkanlah doa kami, ya Tuhan 
P4 : ……………….. telah menerima sakramen pembaptisan dan 
sakramen pengurapan orang sakit. Semoga ia Engkau gabungkan 
dalam persatuan para kudusMu. Marilah kita memohon …….… 

U : Kabulkanlah doa kami, ya Tuhan 
P5 : ……………… ini telah menikmati perjamuan Tubuh dan 
DarahMu. Perkenankanlah kini ia boleh mengambil bagian dalam 
perjamuan surgawi-Mu. Marilah kita memohon ……..… 

U : Kabulkanlah doa kami, ya Tuhan 
38 



P6 : Kami berduka cita atas kematian……ini. Maka hiburlah kami 
dengan harapan akan hidup kekal. Marilah kita memohon …….… 

U : Kabulkanlah doa kami, ya tuhan 
I : Bapa Maha Kuasa; Engkaulah Tuhan dan Allah kami. Kehidupan 
dan kematian kami berada dalam tanganMu. Engkau telah 
menciptakan dan memanggil…………kepada kehidupan. Engkau 
telah menebus dia demi darah Kristus, PuteraMu dan menerima dia 
dalam persekutuan kekal. Maka kami mohon kabulkanlah doa-doa 
permohonan dan segala kerinduan hati kami ini, demi Kristus, Tuhan 
dan Pengantara kami. 
U : Amin 
BAPA KAMI 

I : Atas petunjuk Penyelamat kita dan menurut ajaran ilahi, maka 
beranilah kita berdoa. 
U : Bapa kami yang ada di surga, dimuliakanlah namaMu, datanglah 
kerajaanMu, jadilah kehendakMu, di atas bumi seperti di dalam 
surga. Berilah kami rezeki pada hari ini dan ampunilah kesalahan 
kami, seperti kami pun mengampuni yang bersalah kepada kami, dan 
janganlah masukkan kami ke dalam percobaan, tetapi bebaskanlah 
kami dari yang jahat. 
I : Ya Bapa, bebaskanlah kami dari segala yang jahat dan berilah 
kami damaiMu. Kasihanilah dan bantulah kami, supaya selalu bersih 
dari noda dosa dan terhindar dari segala gangguan, sehingga kami 
dapat hidup dengan tenteram, sambil mengharapkan kedatangan 
Penyelamat kami, Yesus Kristus. 
U : Sebab Engkaulah Raja yang mulia dan berkuasa untuk selamalamanya. 
PEMBERKATAN JENAZAH: 

39 



I : Ibu-Bapak, saudara-saudariku sekalian yang terkasih dalam 
Kristus, Tuhan. Sebentar lagi kita akan berpisah secara jasmani 
dengan ………………. yang kita kasihi ini. Maka dengan hati yang 
tabah, kita memberikan penghormatan yang terakhir kepadanya, 
karena kita berharap bahwa ia akan bangkit bersama Kristus yang 
telah diimaninya untuk kehidupan yang kekal. Maka air suci akan 
direciki di atas dia sebagai lambang pembaptisannya dan jenazahnya 
akan didupai, supaya keharuman amal baktinya di dunia ini berkenan 
kepada Tuhan. 
I : Ya Tuhan, siramilah hambaMu ini: …………….…, yang masuk 
kealam kekal dengan air kehidupan 
U : Terimalah dia, ya Tuhan 
I : Supaya ia hidup bahagia selamanya bersama para kudus-
Mu dalam kerajaan surga 
U : Terimalah dia , ya Tuhan 
I : Dari bumi aku berseru kepadaMu, ya Tuhan. Sudilah Engkau 
mendengarkan seruanku 
U : Terimalah dia, ya Tuhan 
I : Pada-Mu, ya Tuhan, ada pengampunan dosa agar semua orang 
mengabdi kepada-Mu dengan hati yang gembira 
U : Terimalah dia, ya Tuhan 
I : Aku percaya kepada-Mu, ya Tuhan, jiwaku percaya akan Sabda-
Mu 
U :Terimalah dia, ya Tuhan. 
I : Pada-Mu, ya Tuhan, ada belaskasihan serta penebusan berlimpah 
ada pada-Mu 
U : Terimalah dia, ya Tuhan 
I :Hai para kudus dan para malaikat Allah, datanglah menyongsong 
…………..… ini dan hantarkanlah dia kepada Kristus 
40 



U : Di hadapan Allah yang Mahatinggi 
I : Semoga Kristus menyambutmu, sebab Dia-lah yang telah 
memanggil engkau. Semoga para malaikat mengiringi dan 
menjemput engkau ke pangkuan Abraham 
U : Di hadapan Allah yang Mahatinggi 
I : Tuhan, berilah dia istirahat kekal dan sinarilah dia dengan cahaya 
abadi 
U : Di hadapan Allah yang Mahatinggi 
I : Marilah Berdoa: 
Allah yang Maharahim, kami mempercayakan …………..… ini 
kepada-Mu. Ampunilah dengan murah hati dosa apapun yang telah ia 
lakukan karena kerapuhan manusiawi-nya. Selagi masih hidup, ia 
percaya dan berharap kepadaMu; maka perkenankanlah kini ia 
menikmati apa yang diharapkannya dari pada-Mu, yakni hidup 
bahagia bersama Engkau. Demi Kristus, Tuhan dan pengantara kami. 
U : Amin 
I : ……………., terima kasih atas segala kebaikan, jasa dan 
pengorbanan hidupmu yang telah engkau perbuat selagi masih hidup 
bersama dengan kami. Atas segala kebaikanmu itu, kami hanya 
sanggup mengucapkan terima kasih dan selamat jalan, bawalah selalu 
tanda kemenangan Kristus: Dalam nama Bapa dan (+) Putera dan 
Roh Kudus. 
U: Amin 
(Lalu jenazah direciki dengan air suci dan di dupai. Keluarga 
diperkenankan untuk menyirami jenazah dengan minyak wangi yang 
sudah diberkati. Sebaiknya diiringi dengan lagu yang sesuai.) 

DOA PENUTUP: 

I : Marilah berdoa: 
41 



Ya Allah sumber kehidupan kami, engkau menghendaki agar kami 
hidup terus sesudah kematian, sebagai manusia yang baru dan utuh. 
Sudilah menganugerahkan kami berkat Perjamuan Ekaristi ini, iman 
serta keyakinan yang teguh bahwa hidup ini pantas dan bermakna, 
dan bahwa kematian bukanlah titik akhir melainkan awal kehidupan 
yang sesungguhnya bersama Engkau. Semoga hamba-Mu yang kami 
doakan pada kesempatan ini ……………..., engkau anugerahi 
kehidupan kekal bersama para kudus dalam Kerajaan surga-Mu. 
Perkenankanlah kini ia boleh mengambil bagian dalam perjamuan 
surgawi-Mu. Semoga kami pun, yang berziarah di bumi ini, selalu 
mempersiapkan diri dalam menyambut hari suka cita-Mu. 
Kabulkanlah seluruh niat dan keinginan baik kami, demi jasa 
puteraMu Yesus Kristus, yang bersama Dikau dan Roh Kudus, hidup 
dan berdaulat kini dan sepanjang masa. 

U : Amin 
BERKAT DAN PENGUTUSAN 

I : Ibu-Bapak, saudara-saudariku sekalian, Tuhan Yesus Kristus telah 
bangkit dengan jaya. Karena kebangkitan-Nya itu, kita dibebaskan 
dari dosa, supaya dapat mengusahakan hidup yang baru. Untuk itu 
marilah kita memohon berkatNya. 
I : Tuhan sertamu 
U : Dan sertamu juga 
I : Semoga karena kebangkitanNya, Tuhan Yesus meng-anugerahkan 
ketabahan sejati dalam perjuangan hidup kita. 
U: Amin 
I : Semoga Yesus sang Penebus memperkenankan kita ikut 
menikmati kebahagiaan abadi di surga. 
U : Amin. 
I : Perayaan Ekaristi kita, sudah selesai 
42 



U : Syukur kepada Allah 
I : Hiduplah dalam kasih Tuhan 
U : Amin 
LAGU PENUTUP 
Requiescat In pace 

43 




\chapter{IBADAT KEMATIAN 3} 

PEMBUKAAN 

LAGU PEMBUKAAN: 

TANDA SALIB DAN SALAM: 

I : Dalam Nama Bapa dan Putera dan Roh Kudus 
U : Amin 
I : Semoga Allah yang telah membangkitkan Yesus Kristus, 
PuteraNya dari alam maut, melimpahkan penghiburan dan kekuatan 
iman kepada kita sekalian 
U : Sekarang dan selama-lamanya 
KATA PEMBUKAAN: 

PERNYATAAN TOBAT: 

I : Bapak/ibu, saudara/i terkasih dalam Tuhan, agar Perayaan Ekaristi 
yang kita rayakan ini sungguh-sungguh mendatangkan rahmat dan 
berkat bagi kita semua, terutama bagi ……………… yang sudah 
mendahului kita dalam imannya, serta kekuatan dan penghiburan 
iman bagi semua anggota keluarga yang ditinggal-kannya bagi kita 
semua yang ambil bagian dalam Perayaan ini, maka marilah kita 
dengan rendah hati memohon ampun kepada Tuhan untuk semua 
dosa dan kesalahan yang sudah kita lakukan. Atas nama 
……………., marilah kita juga memohon ampun dari Tuhan atas 
segala dosa dan kesalahan yang telah ia lakukan selagi ia masih hidup 
bersama dengan kita. 
(hening sejenak). 
I+U : Saya mengaku kepada Allah yang Maha Kuasa dan kepada 
saudara sekalian, bahwa saya telah berdosa dengan pikiran dan 
perkataan, dengan perbuatan dan kelalaian. Saya berdosa, saya 
44 



berdosa, saya sungguh berdosa. Oleh sebab itu saya mohon kepada 
Santa Perawan Maria, kepada para malaikat dan segala orang Kudus 
dan kepada saudara sekalian, supaya mendoakan saya kepada Allah 
Tuhan kita. 

I : Semoga Allah, Bapa yang Maha kuasa dan Maha rahim, 
mengasihani kita, mengampuni dosa-dosa kita dan menghantar kita 
kepada kehidupan yang kekal. 
U : Amin 
LAGU TUHAN KASIHANILAH KAMI: 

DOA PEMBUKA: 

I : Marilah berdoa: 
Allah yang kekal dan kuasa, barang siapa berdoa kepada-Mu 
dengan hati yang ikhlas, tidak pernah Engkau kecewakan. Pada 
malam hari ini kami berkumpul dalam namaMu untuk mendoakan 
keselamatan kekal bagi ……………….. yang sudah Engkau panggil 
menghadap hadiratMu. Dengarkanlah doa kami dan curahkanlah 
belaskasih-Mu kepadanya. Ia telah meninggal sebagai pengikut 
Kristus, maka semoga ia tetap bersatu dengan Kristus dalam 
persatuan para kudus di surga. Sebab Kristus itulah PuteraMu Tuhan 
dan Pengantara kami, yang hidup dan bertahta bersama dengan Dikau 
dalam persekutuan dengan Roh Kudus, kini dan sepanjang masa. 

U : Amin 
LITURGI SABDA 

BACAAN I: Pembacaan dari YESAYA 25:6a, 7-9 

L : TUHAN AKAN MEMBINASAKAN MAUT UNTUK 
SELAMA-LAMANYA 
25:6a TUHAN semesta alam akan menyediakan di gunung Sion 
suatu jamuan yang mewah bagi segala bangsa. 25:7 Dan di atas 

45 



gunung ini TUHAN akan mengoyakkan kain perkabungan yang 
diselubungkan kepada segala suku bangsa dan tudung yang 
ditudungkan kepada segala bangsa-bangsa. 25:8 Ia akan meniadakan 
maut untuk seterusnya; dan Tuhan ALLAH akan menghapuskan air 
mata dari pada segala muka; dan aib umat-Nya akan dijauhkan-Nya 
dari seluruh bumi, sebab TUHAN telah mengatakannya. 25:29 Pada 
waktu itu orang akan berkata: “Sesungguhnya, inilah Allah kita, yang 
kita nanti-nantikan, supaya kita diselamatkan. Inilah TUHAN yang 
kita nanti-nantikan; marilah kita bersorak-sorak dan bersukacita oleh 
karena keselamatan yang diadakan-Nya! . Demikianlah sabda Tuhan. 

U : Syukur kepada Allah 
LAGU ANTAR BACAAN: 

BACAAN INJIL: Dari Mateus 11:25-30 

I : Tuhan sertamu 
U : Dan sertamu juga 
I : Inilah Injil Suci Yesus Kristus menurut Santo Mateus 
U : Dimuliakanlah Tuhan 
I : MARILAH KEPADA-KU, SEMUA YANG LETIH LESU 

DAN BERBEBAN BERAT, AKU AKAN MEMBERI KELEGAAN 
KEPADAMU. 

11:25 Pada waktu itu berkatalah Yesus:”Aku bersyukur kepada-Mu, 
Bapa, Tuhan langit dan bumi, karena semuanya itu Engkau 
sembunyikan bagi orang bijak dan orang pandai, tetapi Engkau 
nyatakan kepada orang kecil. 11:26 Ya Bapa, itulah yang berkenan 
kepada-Mu. 11:27 Semua telah diserahkan kepada-Ku oleh Bapa-Ku 
dan tidak seorang pun mengenal Anak selain Bapa, dan tidak seorang 
pun mengenal Bapa selain Anak dan orang yang kepadanya Anak itu 
berkenan menyatakannya. 11:28 Marilah kepada-Ku, semua yang 
letih lesu dan berbeban berat, Aku akan memberi kelegaan 
kepadamu. 11:29 Pikullah kuk yang Kupasang dan belajarlah pada46 



Ku, karena aku lemah lembut dan rendah hati dan jiwamu akan 
mendapat ketenangan. 11:30 Sebab kuk yang Kupasang itu enak dan 
beban-Ku pun ringan. 

I : Demikianlah Injil Tuhan 
U : Terpujilah Kristus 
HOMILI 

DOA UMAT: 

I : Saudara-saudariku sekalian, Kristus telah bersabda,” Marilah 
kepadaKu kalian semua yang letih lesu dan berbeban berat, Aku akan 
memberikan kelegaan kepadamu”. Maka dalam iman akan Sabda 
Kristus ini, marilah kita menyerahkan ………………. yang kita 
kasihi ini kedalam penyelenggaraan dan kemurahan Allah sendiri. 
P1 : Tuhan Yesus Kristus, Engkau telah menangisi kematian 
Lazarus. Semoga Engkau mengusap air mata duka kami. Marilah kita 
memohon ………. 

U : Kabulkanlah doa kami, ya Tuhan 
P2 : Engkau telah menghidupkan kembali pemuda di Nain. Maka 
anugerahkanlah hidup kekal kepada …………….… Marilah kita 
memohon …….… 

U : Kabulkanlah doa kami, ya Tuhan 
P3 : Engkau menjanjikan firdaus kepada penyamun yang bertobat. 
Terimalah………………kami ini dalam kebahagiaan kekal di surga. 
Marilah kita memohon ……… 

U : Kabulkankah doa kami, ya Tuhan 
P4 : ………………….. telah menerima sakramen pembaptisan dan 
sakramen pengurapan orang sakit. Semoga ia Engkau gabungkan 
dalam persatuan para kudusMu. Marilah kita memohon ………. 

U : Kabulkanlah doa kami, yaTuhan 
47 



P5 : ………………… ini telah menikmati perjamuan Tubuh dan 
DarahMu. Perkenankanlah kini ia boleh mengambil bagian dalam 
perjamuan surgawi-Mu. Marilah kita memohon ……… 

U : Kabulkanlah doa kami, ya Tuhan 
P6 : Kami berduka cita atas kematian ………………. ini. Maka 
hiburlah kami dengan harapan akan hidup kekal. Marilah kita 
memohon ……… 

U : Kabulkanlah doa kami, ya Tuhan 
I : Bapa Maha Kuasa; Engkaulah Tuhan dan Allah kami. Kehidupan 
dan kematian kami berada dalam tanganMu. Engkau telah 
menciptakan dan memanggil ……………..… kepada kehidupan. 
Engkau telah menebus dia demi darah Kristus, PuteraMu dan 
menerima dia dalam persekutuan kekal. Maka kami mohon 
kabulkanlah doa-doa permohonan dan segala kerinduan hati kami, 
demi Kristus, Tuhan dan Pengantara kami. 
U : Amin 
BAPA KAMI: 

I : Atas petunjuk Penyelamat kita dan menurut ajaran ilahi, maka 
beranilah kita berdoa. 
U : Bapa kami yang ada di surga, dimuliakanlah namaMu, datanglah 
kerajaanMu, jadilah kehendakMu, di atas bumi seperti di dalam 
surga. Berilah kami rezeki pada hari ini dan ampunilah kesalahan 
kami, seperti kami pun mengampuni yang bersalah kepada kami, dan 
janganlah masukkan kami ke dalam percobaan, tetapi bebaskanlah 
kami dari yang jahat. 
I : Ya Bapa, bebaskanlah kami dari segala yang jahat dan berilah 
kami damaiMu. Kasihanilah dan bantulah kami, supaya selalu bersih 
dari noda dosa dan terhindar dari segala gangguan, sehingga kami 
dapat hidup dengan tenteram, sambil mengharapkan kedatangan 
Penyelamat kami, Yesus Kristus. 
48 



U : Sebab Engkaulah Raja yang mulia dan berkuasa untuk selamalamanya. 
DOA PENUTUP: 

I : Marilah berdoa: 
Allah Bapa, kehidupan dan kebangkitan kami, kami telah menyambut 
anugerah PuteraMu, yang mengor-bankan diri bagi kami dan bangkit 
dengan mulia jaya. Kami berdoa untuk arwah hambaMu 
……………..… yang sudah dibersihkan berkat wafat dan 
kebangkitan Kristus. Semoga dia Engkau muliakan dengan anugerah 
kebangkitan . Demi Kristus, Tuhan dan Pengantara kami. 
U : Amin 
BERKAT PENUTUP: 

I : Tuhan sertamu 
U : Dan sertamu juga 
I : Semoga kita sekalian, semua yang kita doakan dan yang kita 
kasihi, terutama ……………….… yang kita doakan secara khusus 
pada kesempatan ini, serta seluruh karya amal bakti hidup kita 
senantiasa dilindungi, dibimbing dan diberkati oleh Allah Yang 
Maha Kuasa: Bapa, dan (+) Putera, dan Roh Kudus. 
U : Amin 
I : Saudara/i terkasih, perayaan Ekaristi dalam rangka mendoakan 
keselamatan kekal ……………… telah selesai. 
U : Syukur Kepada Allah. 
I : Hiduplah dalam kasih Tuhan 
U : Amin 
LAGU PENUTUP 

49 



X. UPACARA PELEPASAN JENAZAH 
PELEPASAN JENAZAH 

LAGU PEMBUKAAN: 

TANDA SALIB DAN SALAM: 

I : Dalam Nama Bapa dan Putera dan Roh Kudus 
U : Amin 
I : Terpujilah Allah, Bapa Tuhan kita Yesus Kristus, Bapa yang 
penuh belaskasihan dan Allah sumber segala penghiburan, yang 
menghibur kita dalam segala penderitaan. 
U : Sekarang dan selama-lamanya 
KATA PEMBUKA: 

I : Ibu-Bapak, saudara-saudariku sekalian, keluarga yang berduka 
yang terkasih dalam Kristus. Dengan rasa sedih hati kita mengadakan 
upacara perpisahan dengan………..yang kita kasihi ini. Sebentar lagi 
kita akan menghantar dia ke tempat peristirahatannya yang terakhir. 
Namun kita tidak boleh putus asa seperti orang yang tidak 
mempunyai harapan. Sebab kita menaruh harapan kepada Kristus 
yang telah meng-hancurkan kekuasaan maut dengan kebangkitan-
Nya yang mulia. Maka marilah kita mendoakan keselamatan kekal 
baginya dan peneguhan iman bagi kita semua, terutama bagi keluarga 
yang ditinggalkan. 
PERNYATAAN TOBAT: 

I : Agar doa-doa yang kita panjatkan ke hadirat Allah yang 
Maharahim diterima, marilah kita terlebih dahulu menyesali dan 
50 



memohon ampun dariNya atas segala dosa dan kesalahan yang telah 
kita lakukan. 

I : Ya Tuhan Yesus Kristus, Engkau mengalami kematian sebagai 
manusia, tetapi dibangkitkan oleh kekuasaan Bapa dalam Roh Kudus. 
Tuhan kasihanilah kami. 
U : Tuhan, kasihanilah kami. 
I : Engkaulah kebangkitan dan kehidupan; barang siapa percaya 
kepada-Mu akan memperoleh kehidupan yang kekal. Kristus, 
kasihanilah kami. 
U : Kristus, kasihanilah kami 
I : Engkau akan datang dengan mulia untuk mempersatukan kami 
semua dalam kerajaan Bapa. Tuhan kasihanilah kami. 
U : Tuhan, kasihanilah kami. 
I : Semoga Allah yang Mahakuasa mengasihani kita, me-ngampuni 
dosa-dosa kita dan menghantar kita ke hidup yang kekal. 
U : Amin 
Marilah Berdoa: 

I : Allah, Bapa kami yang Maharahim dengan penuh harapan kami 
menyerahkan hambaMu ini ……………..…, ke tanganMu. 
Berikanlah dia istirahat penuh terang dan damai dalam kerajaanMu. 
Ampunilah ya Bapa, segala dosa dan kesalahannya. Semoga ia 
bersatu dengan para hambaMu untuk menikmati cahaya kebahagiaan 
kekal dan memuji kebaikanMu. 
Kami juga berdoa bagi kami semua yang masih berziarah di 
bumi ini, terutama bagi keluarga yang sedang berkabung. Engkaulah, 
ya Bapa sumber belaskasihan dan penghiburan, Engkau mencintai 
kami dengan kasih abadi; Engkau mengubah maut yang gelap gulita 
menjadi fajar yang gilang-gemilang berkat kebangkitan mulia 
PuteraMu. Maka kami mohon, semoga keluarga yang sedang 

51 



berkabung ini dapat menanggung duka-citanya dengan tabah dan 
menaruh kepercayaan serta harapannya kepadaMu. 

Demi Yesus Kristus, Puteramu, Tuhan dan pengantara kami, 
yang hidup dan berkuasa bersama dengan Dikau dalam persekutuan 
dengan Roh Kudus, kini dan sepanjang segala masa. 

U : Amin 
LITURGI SABDA: 

BACAAN I: 

APAKAH YANG DAPAT MEMISAHKAN KITA DARI CINTA 
KASIH KRISTUS. 

Pembacaan dari surat Rasul Paulus kepada umat di Roma 
(8:31b-35,37-39). 

L : 8:31b Jika Allah di pihak kita, siapakah yang akan melawan kita? 
8:32 Ia, yang tidak menyayangkan Anak-Nya sendiri, tetapi yang 
menyerahkan-Nya bagi kita semua, bagaimanakah mungkin Ia tidak 
mengaruniakan segala sesuatu kepada kita bersama-sama dengan 
Dia? 8:33 Siapakah yang akan menggugat orang-orang pilihan Allah? 
Allah, yang membenarkan mereka? Siapakah yang akan menghukum 
mereka? 8:34 Kristus Yesus, yang telah mati? Bahkan lebih lagi: 
yang telah bangkit, yang juga duduk di sebelah kanan Allah, yang 
malah menjadi Pembela bagi kita? 8:35 Siapakah yang akan 
memisahkan kita dari kasih Kristus? Penindasan atau kesesakan atau 
peng-aniayaan, atau kelaparan atau ketelanjangan , atau bahaya, atau 
pedang? 8:37 Tetapi dalam semuanya itu kita lebih daripada orang-
orang yang menang baik malaikat-malaikat, maupun pemerintahpemerintah, 
baik yang ada sekarang, maupun yang akan datang, 8:39 
atau kuasa-kuasa, baik yang di atas, maupun yang di bawah, ataupun 
sesuatu makhluk lain, tidak akan dapat memisahkan kita dari kasih 
Allah, yang ada dalam Kristus Yesus, Tuhan kita. 
L : Demikian Sabda Tuhan 
U : Amin 
52 



LAGU ANTAR BACAAN: 

BACAAN INJIL (LUKAS 23-33,39-43) 

I : Tuhan sertamu 
U : Dan sertamu juga 
I : Inilah Injil Suci Yesus Kristus menurut Santo Lukas 
U : Dimuliakanlah Tuhan 
I : HARI INI JUGA ENGKAU AKAN ADA BERSAMA-SAMA 
DENGAN AKU DI DALAM FIRDAUS 
23:33 Ketika mereka sampai di tempat yang bernama Tengkorak, 
mereka menyalibkan Yesus di situ dan juga kedua orang penjahat itu, 
yang seorang di sebelah kanan-Nya dan yang lain di sebelah kiri-
Nya, 23:39 Seorang dari penjahat yang di gantung itu menghujat 
Yesus, katanya: ”Bukankah Engkau adalah Kristus? Selamatkanlah 
diri-Mu dan kami! 23:40 Tetapi yang seorang menegor dia, katanya: 
“Tidakkah engkau takut, juga tidak kepada Allah, sedang engkau 
menerima hukuman yang sama? 23:41 Kita memang selayaknya 
dihukum, sebab kita menerima balasan yang setimpal dengan 
perbuatan kita, tetapi orang ini tidak berbuat sesuatu yang salah.“ 
23:42 Lalu ia berkata kepada Yesus: “Yesus, ingatlah akan aku, 
apabila Engkau datang sebagai Raja.“ 23:43 Kata Yesus kepadanya: 
“Aku berkata kepadamu, sesungguhnya hari ini juga engkau akan ada 
bersama-sama dengan Aku di dalam Firdaus 
I : Demikianlah Injil Tuhan 
U : Terpujilah Kristus 
HOMILI 

DOA UMAT 

53 



I : Ya Allah yang Maha murah, Engkau adalah sumber kehidupan 
kami yang selalu setia menuntun segala langkah kehidupan kami di 
kala suka maupun duka. Kami percaya, bersama dengan Dikau kami 
mampu untuk membuat hidup ini menjadi makin bermakna. Karena 
itu ya Tuhan, sudilah Engkau mendengarkan doa-doa umatMu yang 
berhimpun di sini. 
P1 : Allah, Bapa kami di dalam surga, kepadaMu kami menyerahkan 
…………..… yang kami kasihi ini. Engkau telah menciptakan dan 
menempatkan dia di dunia ini untuk mengabdi kepadaMu dan 
sesamanya. Engkaulah yang memanggil dia. Semoga ia pantas 
bertemu dengan Dikau dan berbahagia bersama Engkau di surga. 
Marilah kita memohon …..…. 

U : Kabulkanlah doa kami, ya Tuhan 
P2 : Semoga Engkau dengan murah hati mengasihani dia yang telah 
kembali kepadamu. Marilah kita memohon ……..… 

U : Kabulkanlah doa kami, ya Tuhan. 
P3 : Semoga ia yang telah Kau murnikan dengan air permandian, 
tetap Kau sucikan pula dengan kemurahan belaskasihanMu. Marilah 
kita memohon ……… 

U : Kabulkanlah doa kami, ya Tuhan. 
P4 : Semoga ia yang selama hidupnya memperjuangkan kebenaran, 
keadilan, cinta kasih dan damai, Kau beri pahala di surga. Marilah 
kita memohon …..…. 

U : Kabulkanlah doa kami, ya Tuhan. 
P5 : Semoga di akhirat, ia Kau beri tempat yang tenteram dan terang 
bersama dengan orang-orang kudusMu. Marilah kita memohon 
……… 

U : Kabulkanlah doa kami, ya Tuhan. 
54 



P6 : Semoga kami yang bersedih hati atas kematiannya, terutama 
keluarga yang berduka, Kau hibur dengan harapan akan persatuan 
kelak di surga. Marilah kita memohon …….… 

U : Kabulkanlah doa kami, ya Tuhan. 
I : Allah yang kekal dan kuasa, Engkau senantiasa menyayangi 
umatMu. Engkaulah kebahagiaan bagi semua orang yang meninggal 
dalam Dikau. Semoga dalam kerahimanMu beristirahatlah 
………………..… yang kami kasihi. Kabulkanlah dengan rela doadoa 
kami ini. Demi Kristus, Tuhan dan Pengantara kami. 
U : Amin 
BAPA KAMI 

I : Marilah kita satukan semua doa permohonan dan kerinduan hati 
kita, dalam doa yang diajarkan Kristus sendiri: 
U : Bapa kami yang ada di surga, dimuliakanlah namaMu, datanglah 
kerajaanMu, jadilah kehendakMu, di atas bumi seperti di dalam 
surga. Berilah kami rezeki pada hari ini dan ampunilah kesalahan 
kami, seperti kami pun mengampuni yang bersalah kepada kami, dan 
janganlah masukkan kami ke dalam percobaan, tetapi bebaskanlah 
kami dari yang jahat. 
I : Ya Bapa, bebaskanlah kami dari segala yang jahat dan berilah 
kami damaiMu. Kasihanilah dan bantulah kami, supaya selalu bersih 
dari noda dosa dan terhindar dari segala gangguan, sehingga kami 
dapat hidup dengan tenteram, sambil mengharapkan kedatangan 
Penyelamat kami, Yesus Kristus. 
U : Sebab Engkaulah Raja yang mulia dan berkuasa untuk selamalamanya 
I : Damai Tuhan kita Yesus Kristus, besertamu 
U : Dan sertamu juga 
PEMBERKATAN JENAZAH 

55 



I : Ibu-Bapak, saudara-saudariku sekalian yang terkasih dalam 
Kristus Tuhan. Sebentar lagi kita akan berpisah secara jasmani 
dengan ……………..… yang kita kasihi ini. Maka dengan hati yang 
tabah, kita memberikan penghormatan yang terakhir kepadanya, 
karena kita berharap bahwa ia akan bangkit bersama Kristus yang 
telah diimaninya untuk kehidupan yang kekal. Maka air suci akan 
direciki di atas dia sebagai lambang pembaptisannya dan jenazahnya 
akan didupai, supaya keharuman amal baktinya di dunia ini berkenan 
kepada Tuhan. 
I : Ya Tuhan, siramilah hambaMu ini ……………. yang masuk ke 
alam kekal, dengan air kehidupan. 
U : Terimalah dia, ya Tuhan. 
I : Supaya ia hidup bahagia selamanya bersama para kudusMu dalam 
kerajaan surga. 
U : Terimalah dia, ya Tuhan. 
I : Dari bumi aku berseru kepadaMu, ya Tuhan. Sudilah Engkau 
mendengarkan seruanku. 
U : Terimalah dia, ya Tuhan. 
I : PadaMu, ya Tuhan, ada pengampunan dosa agar semua orang 
mengabdi kepadaMu dengan hati yang gembira. 
U : Terimalah dia, ya Tuhan. 
I : Aku percaya kepadaMu, ya Tuhan; jiwaku percaya akan 
SabdaMu. 
U : Terimalah dia, ya Tuhan. 
I : Pada-Mu, ya Tuhan, ada belaskasihan serta penebusan berlimpah 
ada pada-Mu 
U : Terimalah dia, ya Tuhan 
56 



I : Hai para kudus dan para malaikat Allah, datanglah menyongsong 
……………. ini dan hantarkanlah dia kepada Kristus. 
U : Di hadapan Allah yang Mahatinggi. 
I : Semoga Kristus menyambutmu, sebab Dia-lah yang telah 
memanggil engkau. Semoga para malaikat mengiringi dan 
menjemput engkau ke pangkuan Abraham. 
U : Di hadapan Allah yang Mahatinggi 
I : Tuhan, berilah dia istirahat kekal dan sinarilah dia dengan cahaya 
abadi 
U : Di hadapan Allah yang Mahatinggi 
I : Marilah berdoa: 
Ya Tuhan, kehidupan dan kematian kami berada di dalam tanganMu 
sendiri. Engkau telah menciptakan manusia karena kasih dan 
cintaMu. Ya Tuhan, lihatlah kami putera-puteri-Mu yang kini 
berhimpun di sekitar peti jenazah dari ……………... yang kami 
kasihi ini. Kami semua berduka cita atas kematiannya. Maafkan kami 
ya Tuhan, jika kami belum sempat mengucapkan terima kasih kami 
kepadanya atas segala kebaikan yang telah dilakukannya kepada 
kami. Kami juga menyesal jika kami belum sempat meminta maaf 
atas segala dosa dan kesalahan yang telah kami lakukan terhadapnya. 
Akan tetapi kami percaya bahwa kasihMu jauh lebih kuat daripada 
keinginan manusiawi kami. Kami mohon berkat belaskasihan-Mu 
kepada …………………………, janganlah Engkau serahkan dia 
kepada kekuasaan maut, tetapi bebaskanlah dia demi jasa Kristus, 
Putera-Mu. Biarkanlah darah dan air yang tercurah dari lambung 
Kristus, PuteraMu itu membersihkan dia dari segala dosa dan 
kesalahannya sehingga ia, dengan jiwa yang bersih dapat menghadap 
Engkau, Penciptanya, Bapanya dan juga Bapa kami bersama. Tuhan, 
dalam kehidupan di dunia ini, ia telah dikuatkan dan disegarkan 
dengan santapan Tubuh dan Darah Kristus Putera-Mu. Maka kami 
mohon, perkenankanlah ia kini mengambil bagian dalam perjamuan 
surgawi-Mu. Disanalah kami semua akan dipertemukan kembali 

57 



untuk memuji dan memuliakan Dikau dalam keabadian. Demi 
Kristus Tuhan dan Pengantara kami. 

U : Amin. 
I : …………….,terima kasih atas segala kebaikan, jasa dan 
pengorbanan hidupmu yang telah engkau perbuat selagi masih hidup 
bersama dengan kami. Atas segala tanda kasihmu itu, kami hanya 
sanggup mengucapkan terima kasih dan selamat jalan, bawalah selalu 
tanda kemenangan Kristus: Dalam nama Bapa dan (+) Putera Roh 
Kudus. 
U : Amin. 
(Lalu jenazah direciki dengan air suci dan didupai. Keluarga 
diperkenankan untuk menyirami jenazah dengan minyak wangi yang 
sudah diberkati. Sebaiknya diiringi dengan lagu yang sesuai) 

DOA PENUTUP 

I : Marilah berdoa: 
Allah dan Bapa kami yang Maha baik, kami mempercayakan 
…………….… yang kami kasihi ini kepada kerahimanMu. Kami 
percaya, bahwa semua orang yang meninggal dalam Kristus akan 
hidup bersama Kristus. Kepercayaan ini memberi kami harapan dan 
menabahkan hati kami dalam kesusahan. Dengarkanlah doa umatMu 
ini dan bukalah pintu surga bagi dia yang sudah Engkau panggil. 
Semoga kami yang masih berziarah di bumi ini, terutama semua 
anggota keluarga yang ditinggalkannya selalu saling membantu 
dalam segala tantangan hidup ini; dan semoga kami tetap menaruh 
kepercayaan yang teguh akan sabda-Mu, bahwa kami juga akan 
menyongsong Kristus untuk bersatu dengan Dia selama-lamanya. 
Sebab Dialah PuteraMu, Tuhan dan Pengantara kami kini dan 
sepanjang segala masa. 

U : Amin 
58 



BERKAT 

I : Tuhan sertamu 
U : Dan sertamu juga 
I : Semoga kita sekalian, semua yang kita doakan dan perjalanan kita 
menghantar …………………..… ke tempat peristirahatannya yang 
terakhir, senantiasa dilindungi, dibimbing dan diberkati oleh Allah 
yang Mahakuasa: Bapa dan (+) Putera dan Roh Kudus 
U : Amin. 
I : Saudara-saudariku sekalian, marilah kita berangkat ke pekuburan 
untuk menghantar ……………… yang kita kasihi ini ke tempat 
istirahatnya yang terakhir. Semoga damai Tuhan menyertai kita. 
U : Sekarang dan selama-lamanya 
LAGU PENUTUP 

59 



\chapter{UPACARA DI PEMAKAMAN} 

LAGU PEMBUKAAN 

TANDA SALIB DAN SALAM 

I : Dalam Nama Bapa dan Putera dan Roh Kudus 
U : Amin 
I : Semoga Allah yang telah membangkitkan Yesus Kristus, 
PuteraNya dari alam maut, melimpahkan penghiburan dan kekuatan 
iman kepada kita sekalian. 
U : Sekarang dan selama-lamanya 
KATA PEMBUKAAN 

I : Saudara-saudariku sekalian, keluarga yang berduka yang terkasih 
dalam Tuhan. Sebentar lagi kita akan berpisah secara jasmani dengan 
…………….… ini. Maka sebelum kita berpisah dengan dia, baiklah 
kalau sekali lagi kita mengucapkan selamat jalan kepadanya. Semoga 
doa dan salam yang kita ucapkan pada makam ini dapat 
melambangkan cinta, meringankan duka dan meneguhkan iman kita. 
Sebab kita berharap akan berjumpa lagi dengan ………….… ini 
dalam keluarga abadi, yaitu bila Kristus sendiri datang sebagai 
pemenang atas maut untuk mengumpulkan semua sahabatNya dalam 
kerajaan Bapa. 
I : Marilah berdoa: 
Allah yang Maha Kuasa dan Maha rahim, kehidupan dan kematian 
kami berada di dalam tanganMu. Engkau telah memanggil 
……………… dari kehidupan di dunia ini untuk menghadap 
hadiratMu. Dengan hati sedih kami berdiri di sini untuk 

60 



membaringkan jenazahnya dalam makam, namun dengan penuh 
harapan kami menantikan kebangkitan, sebab Kristus telah bangkit 
sebagai yang pertama dari antara orang-orang mati. Maka, 
kasihanilah dia ya Tuhan, kasihanilah dia dan terimalah dia dalam 
pelukan cintaMu. Demi Kristus, Tuhan dan Pengantara kami. 

U : Amin 
BACAAN INJIL: Dari Yohanes 6:37-40 

I : Tuhan sertamu 
U : Dan sertamu juga 
I : Inilah Injil Suci Yesus Kristus menurut Santo Yohanes 
U : Dimuliakanlah Tuhan 
I : 6:37 Pada waktu itu, Yesus bersabda, “Semua yang diberikan 
Bapa kepada-Ku akan datang KepadaKu, barang siapa datang 
kepada-Ku, ia tidak akan Kubuang. 6:38 Sebab Aku telah turun dari 
sorga bukan untuk melakukan kehendak-Ku, tetapi untuk melakukan 
kehendak Dia yang telah mengutus Aku. 6:39 Dan Inilah kehendak 
Dia yang telah mengutus Aku, yaitu supaya dari semua yang telah 
diberikan-Nya kepada-Ku jangan ada yang hilang, tetapi supaya 
Kubangkitkan pada akhir zaman. 6:40 Sebab inilah kehendak bapa-
Ku, yaitu supaya setiap orang, yang melihat anak dan yang percaya 
kepada-Nya beroleh hidup yang kekal, dan supaya Aku 
membangkitkannya pada akhir zaman.” 
U : Terpujilah Kristus 
LAGU UNTUK MENGIRINGI PEMBERKATAN 

PEMBERKATAN MAKAM 

I : Marilah berdoa: 
Tuhan Yesus Kristus, Engkau sendiri berbaring dalam makam 
selama tiga hari. Kami mohon sucikanlah (+) makam ini, agar 

61 



hambaMu ……...........… yang kami istirahatkan di sini akhirnya 
bangkit bersama Engkau dan hidup mulia sepanjang segala masa. 

U : Amin 
PENGUBURAN 

(makam direciki dengan air suci dan didupai. Kemudian peti jenazah 
diturunkan ke liang lahat. Umat dapat mengiringinya dengan 
nyanyian yang sesuai) 

(setelah peti jenazah diturunkan): 

DIRECIKI AIR SUCI 

I : Ketika dibaptis kita disatukan dengan Kristus dan turut mati 
bersama dengan Dia. ………………… yang kita kasihi ini sekarang 
mati bersama dengan Kristus. Semoga ia hidup pula dalam keadaan 
baru seperti Kristus. 
U : Amin 
DIDUPAI 

I : Semoga doa-doa kita mengiringi ………………..… dalam 
perjalanannya menuju rumah Bapa 
U : Amin 
DITABURI BUNGA 

I : Semoga kuntum hidup ilahi yang telah ditanamkan dalam diri 
………………..… ini, akan mekar bagaikan bunga yang semerbak 
harum mewangi 
U : Amin 
DITABURI TANAH YANG SUDAH DIBERKATI 

62 



I : Manusia diciptakan dari tanah dan ia kembali ke tanah. Semoga 
Kristus mengalahkan kebinasaan maut dan memulihkan 
……………….… ini dalam kebangkitan mulia. 
U : Amin. 
DITANDAI SALIB 

I : Saudara terkasih, masuklah dalam kehidupan abadi dengan 
membawa tanda kemenangan Kristus: Demi nama Bapa dan (+) 
Putera dan Roh Kudus 
U : Amin 
DOA UMAT 

I : Saudara-saudari sekalian yang terkasih, marilah kita berdoa 
kepada Allah, Bapa yang maharahim, bagi ……………….… yang 
kita kasihi ini, yang telah meninggal dalam persatuan dengan Tuhan. 
Semoga dosa-dosanya diampuni. Marilah kita memohon ……… 
U : Kabulkanlah doa kami, ya Tuhan 
I : Semoga amal baktinya di dunia ini diterima dengan baik. Marilah 
kita mohon ……… 
U : Kabulkanlah doa kami, ya Tuhan 
I : Semoga ia menikmati kehidupan kekal dalam kemuliaan Allah 
Bapa. Marilah kita mohon …..… 
U : Kabulkanlah doa kami , ya Tuhan 
I : Marilah kita berdoa pula bagi semua orang yang berkabung atas 
kematian …………... ini. Semoga kesepian mereka dipenuhi dengan 
cinta kasih Allah. Marilah kita mohon …..… 
U : Kabulkanlah doa kami, ya Tuhan 
I : Semoga mereka dihibur dalam kesusahan mereka. Dan semoga 
iman dan harapan mereka diperteguh. Marilah kita mohon …..… 
U : Kabulkanlah doa kami, ya Tuhan 
63 



I : Semoga hati kita tidak tenggelam dalam urusan-urusan duniawi, 
tetapi selalu terbuka bagi segala rencana dan kehendak Tuhan. 
Marilah kita mohon ……… 
U : Kabulkanlah doa kami, ya Tuhan 
I : Allah yang kekal dan kuasa, Engkaulah Tuhan bagi orang hidup 
dan juga Tuhan bagi orang-orang mati. Kami mohon belas 
kasihanMu bagi ……………… yang sudah mendahului kami dalam 
imannya. Ampunilah segala dosanya, agar ia bergembira atas diri-Mu 
dan tak henti-hentinya memuji dan memuliakan Engkau. Demi 
Kristus, Tuhan dan Pengantara kami 
U : Amin 
BAPA KAMI 

I : Marilah kita satukan semua doa permohonan dan kerinduan hati 
kita, dalam doa yang diajarkan Kristus sendiri: 
U : Bapa kami yang ada di surga, dimuliakanlah namaMu, datanglah 
kerajaanMu, jadilah kehendakMu, di atas bumi seperti di dalam 
surga. Berilah kami rezeki pada hari ini dan ampunilah kesalahan 
kami, seperti kami pun mengampuni yang bersalah kepada kami, dan 
janganlah masukkan kami ke dalam percobaan, tetapi bebaskanlah 
kami dari yang jahat. 
I : Ya Bapa, bebaskanlah kami dari segala yang jahat dan berilah 
kami damaiMu. Kasihanilah dan bantulah kami, supaya selalu bersih 
dari noda dosa dan terhindar dari segala gangguan, sehingga kami 
dapat hidup dengan tenteram, sambil mengharapkan kedatangan 
Penyelamat kami, Yesus Kristus. 
U : Sebab Engkaulah Raja yang mulia dan berkuasa untuk selamalamanya 
I : Damai Tuhan kita Yesus Kristus, besertamu 
64 



U : Dan sertamu juga 
BERKAT PENUTUP 

I : Tuhan, berilah dia istirahat kekal 
U : Dan sinarilah dia dengan cahaya abadi 
I : Semoga ia beristirahat dalam damai 
U : Amin 
I : Saudara-saudariku sekalian, upacara pelepasan dan pemakaman 
……………… yang kita kasihi ini, telah selesai. 
U : Syukur kepada Allah 
I : Pulanglah dalam damai Tuhan 
U : Amin 
LAGU PENUTUP 

(sementara menyanyikan lagu penutup, keluarga dan semua yang 
hadir diundang untuk menaburkan bunga dan/atau tanah pada liang 
lahat. Setelah itu, jenazah ditimbuni dengan tanah dan umat yang 
hadir dapat mengiringinya dengan doa rosario dan lagu-lagu yang 
sesuai). 

Requiescat In Pace 

65 



\chapter{IBADAT PERINGATAN ARWAH HARI KE-7} 

PEMBUKA 

P 
: Saudara-saudari yang terkasih, dengan seluruh hati dan 
budi marilah kita pada peringatan 7 hari almarhum di 
panggil Bapa di surga, kita bersama-sama berdoa memohon 
keselamatan bagi arwah saudara kita……. 
Marilah kita awali doa kita dengan menyanyikan lagu 
pembuka…. 

Lagu Pembuka : 

Tanda Salib 

P 
: Demi nama Bapa dan Putra dan Roh Kudus 
U 
: Amin 
P 
: Semoga Allah yang telah membangkitkan Kristus dari 
alam maut, melimpahkan penghiburan iman kepada kita. 
U 
: Sekarang dan selama-lamanya. 
Tobat 

P 
: Saudara-saudari terkasih, 
Bpk/Ibu/Sdr/Sdri terkasih karena kita menyadari 
kedosaan-kedosaan yang ada dalam diri kita maka 
marilah kita memohon pengampunan kepada Bapa yang 
Maharahim sehingga dengan belas kasihanNya kita 
dibebaskan dari kedosaan-kedosaan dan kesalahankesalahan 
kita. Marilah kita mohon belaskasihanNya dan 
bersama-sama mengakukan kesalahan dan dosa kita 
kepadaNya dan juga kepada sesama kita…… 

66 



P 
: Allah yang maharahim…… 
U 
: Aku menyesal atas dosa-dosaku, sebab patut aku Engkau 
hukum, terutama sebab aku telah menghina Engkau, yang 
mahamurah dan mahabaik bagiku. Aku benci akan segala 
dosaku dan berjanji dengan pertolongan rahmat-Mu 
hendak memperbaiki hidupku dan tidak akan berbuat dosa 
lagi. Allah, ampunilah aku orang berdosa. 
P : Semoga Allah yang mahapengasih dan penyayang, 
mengasihani kita dan menghancurkan sumber dosa di dalam 
hati kita, agar kita layak menerima hidup yang kekal. 
U : Amin 

(Atau memilih bentuk doa tobat yang lain) 

Doa Pembuka 

P 
: Marilah berdoa 
Allah yang mahakuasa dan kekal, kami percaya bahwa 
Saudara/i (anak) kami…….yang telah Engkau panggil pada 
7 hari yang lalu kini telah berdiam dalam rumahMu. Kami 
percaya kesempurnaannya kini telah terwujud dengan 
dipersatukannya saudara/saudari kami……….di dalam 
kemuliaan surgawiMu. Kami mohon semoga karena 
persatuan kami dengan Kristus dalam GerejaNya, kelak 
berlanjut dalam persatuan dengan semua saudara seiman di 
surga. 
Demi Yesus Kristus PutraMu, Tuhan dan pengantara kami, 
yang bersatu dengan Dikau dan Roh Kudus, hidup dan 
berkuasa kini dan sepanjang masa. 
U 
: Amin 
BACAAN KITAB SUCI 

P 
: Marilah kita mendengarkan Sabda Tuhan 
67 



Bacaan Pertama 

Pembacaan dari Kitab Nabi Yehezkiel 37:12-14 

“Demikianlah Tuhan Allah bersabda : Sesungguhnya kubur 
kamu akan Kubuka dan Aku akan menaikkan kamu dari dalam 
kubur, hai umat-Ku, serta mendatangkan kamu ke tanah Israel. 
Maka ketahuilah bahwa Aku Allah, yakni apabila Aku 
membuka kuburmu serta menaikkan kamu dari dalam 
kuburmu, hai umat-Ku ; Aku akan menaruh Roh-Ku di dalam 
kamu, lalu kamu akan hidup dan aku menempatkan kamu di 
tanahmu. 
Maka ketahuilah kamu, bahwa Aku Allah, sudah bersabda dan 
berbuat juga. Itulah firman Tuhan Allah.” 

Bacaan alternatif lain bisa diambil dari Surat Kedua Rasul 
Paulus kepada jemaat di Korintus (5:1,6-10) 

L : Demikianlah Sabda Tuhan 
U : Syukur kepada Allah 
Antar Bacaan 

Untuk antar bacaan bisa dinyanyikan sebuah lagu atau 
mendaraskan mazmur 
Lagu : 
Mazmur : 129 (130) : 1-2, 3-4a.4b-5.6-7.8-9 

Ulangan : Dari jurang yang dalam aku berseru kepada-Mu, ya 
Tuhan. 

68 



P : Dari jurang yang dalam aku berseru kepada-Mu, ya 
Tuhan. Tuhanku, dengarkanlah seruanku. Hendaklah 
telinga-Mu menaruh perhatian kepada jeritan doaku. 
U : Dari jurang…… 
P : Jika Engkau menghitung-hitung kesalahan, ya Tuhan, 
siapakah dapat bertahan ? Tetapi syukurlah Engkau suka 
mengampuni, sehingga orang mengabdi Engkau dengan 
takwa. 
U : Dari jurang…… 
P : Aku berharap akan Tuhan, hatiku mengharapkan sabda-
Nya. Hatiku menantikan Tuhan, lebih dari penjaga 
menantikan fajar. Lebih dari penjaga menantikan fajar, 
Israel menantikan Tuhan. 
U : Dari jurang….. 
P : Sebab pada Tuhanlah kasih setia, dan penebusan yang 
berlimpah-limpah. Tuhanlah yang akan membebaskan 
Israel dari segala kesalahannya. 
U : Dari jurang….. 

Bacaan Injil 

P : Tuhan besertamu 
U : Dan serta Rohmu 
P : Inilah Injil Yesus Kristus menurut Yohanes (14:1-6) 
U 
: Dimuliakanlah Tuhan 
“ Janganlah gelisah hatimu ; percayalah kepada Allah, 
percayalah juga kepada-Ku. Di rumah Bapa-Ku banyak 
tempat tinggal. Jika tidak demikian, tentu Aku 
mengatakannya kepadamu. Sebab Aku pergi ke situ untuk 
menyediakan tempat bagimu. Dan apabila Aku telah pergi 
ke situ dan telah menyediakan tempat bagimu, Aku akan 
datang kembali dan membawa kamu ke tampat-Ku, 
supaya di tempat di mana Aku berada, kamu pun berada. 
69 



Dan ke mana Aku pergi, kamu tahu jalan ke situ.” Kata 
Tomas kepadaNya : “Tuhan, kami tidak tahu ke mana 
Engkau pergi ; jadi bagaimana kami tahu jalan ke situ ?” 
Kata Yesus kepadanya : “Akulah jalan dan kebenaran dan 
hidup. Tidak ada seorang pun yang datang kepada Bapa, 
kalau tidak melalui Aku. 

P 
: Demikianlah Injil Tuhan 
U 
: Terpujilah Kristus 
Renungan Singkat 

Doa Umat 

P 
: Saudara-saudari terkasih, 
Allah yang Mahakuasa berkuasa untuk membangkitkan 
orang dari kematian dan mengangkatnya menuju kemuliaan 
abadi di surga. Maka marilah kita menyampaikan doa-doa 
kita kepada-Nya. 

P 
: Bagi saudara kita……yang pada 7 hari yang lalu dipanggil 
Bapa, ya Bapa yang mahabaik berikanlah kepada saudara 
kami ini hidup yang kekal bersama PutraMu terkasih Tuhan 
kami Yesus Kristus. 
Hening sejenak……marilah kita mohon… 
U 
: Kabulkanlah doa kamu ya Tuhan 
P 
: Allah Bapa Maharahim, satukanlah saudara kami tercinta 
ini bersama para kudusMu di surga, ampunilah segala dosadosanya 
ya Bapa, semoga dengan rahmat cinta kasihMu 
70 



saudara kami ini kini berbahagia dalam kesatuan dengan 
para kudusMu di surga. 
Hening sejenak……marilah kita mohon….
. 


U 
: Kabulkanlah doa kami ya Tuhan 
P : Bagi keluarga dan kerabat yang ditinggalkan, semoga 
kekuatanMu senantiasa menguatkan mereka untuk 
senantiasa berserah dan percaya sepenuhnya kepada 
kehendakMu ya Bapa. 
Hening sejenak……kami mohon….. 
U : Kabulkanlah doa kami ya Tuhan 

P 
: Bagi Gereja : Semoga Gereja yang dipercayakan untuk 
memberikan kesaksian akan hidup sesudah kematian, tetap 
setia melaksanakan misinya, sekalipun menghadapi banyak 
tantangan. 
Hening sejenak…..kami mohon…… 
U 
: Kabulkanlah doa kami ya Tuhan 
P 
: Bagi semua orang yang belum mengenal harapan akan 
Allah. Semoga Allah yang mahabaik melimpahkan belas 
kasihNya bagi saudara-saudari kita yang sudah meninggal 
tetapi belum mengenal Allah. Semoga mereka semua juga 
diperbolehkan untuk memandang wajah Allah yang penuh 
kasih kepada siapa saja. 
Hening sejenak……kami mohon……….. 
U 
: Kabulkanlah doa kami ya Tuhan 
BAPA KAMI 

P : Saudara-saudara terkasih, 
71 



Marilah kita menggabungkan seluruh doa-doa permohonan 
kita tadi dengan bersama-sama memuji dan memohon Allah 
dengan Doa Bapa Kami… 

PENUTUP 

Doa Penutup 

P : Marilah kita berdoa (bersama) 
Ya Bapa yang Mahabaik, Maha Pengampun, kesabaran dan 
kuasaMu telah menyelamatkan kami semua, terlebih bagi 
saudara kami……Bapa betapa kami sering jatuh dalam dosa 
dan menyangkalMu tetapi Engkau tetap setia menanti kami 
untuk bertobat. Kami percaya Bapa, Engkau senantiasa 
menghendaki kami bahagia oleh karena itu kami percaya 
bahwa saudara kami……..kini bahagia bersama Engkau 
sendiri. Bagi kami yang masih terus berjuang di dunia ini 
semoga ya Bapa, semoga karena kekuatan dan rahmatMu 
kami senantiasa hidup seturut jalanMu supaya pada akhirnya 
nanti kami dapat berbahagia bersama para kudusMu di surga. 
Semua ini kami serahkan dengan perantaraan Kristus Tuhan 
kami. 

U : Amin 
Berkat Pengutusan 

P : Semoga Tuhan beserta kita 
U : Sekarang dan selama-lamanya. 
P : Allah selalu menopang, menyelamatkan dan menempatkan 
kita dirumah tinggal yang baru yakni Firdaus yang abadi. 
Marilah kita memohon berkat kepada-Nya 
…………………Hening Sejenak…………………………. 

72 



P 
: Semoga berkat Allah Yang Mahakuasa senantiasa 
memberi ketabahan dan keteguhan iman kepada kita yang 
berkumpul disini. Saudara sekalian ibadat 7 hari saudara 
kita………. sudah selesai. Semoga rahmat Tuhan Yesus 
Kristus senantiasa menyertai kita sekalian. Dalam nama 
Bapa dan Putera dan Roh Kudus. 
Lagu Penutup 

73 



\chapter{IBADAT PERINGATAN ARWAH HARI KE-40} 

Pembuka 

P 
: Saudara-saudari terkasih, 
Pada 40 hari yang lalu, saudara kita 
tercinta……………telah berpulang ke rumah Bapa dan 
meninggalkan kita semua. Kita pada hari ini berkumpul 
bersama-sama untuk mendoakan arwah saudara 
kita……………Kita berdoa agar Allah yang Maharahim 
senantiasa mengampuni dosanya dan memberikan ganjaran 
berkat amal baiknya sewaktu hidup bersama kita. Semoga 
Allah Bapa senantiasa mengampuni dosa dan 
kelemahannya sehingga almarhum senantiasa menerima 
kasih Tuhan untuk selamanya. 
Saudara-saudari terkasih marilah kita awali doa kita dengan 
menyanyikan lagu pembukaan. 

Lagu Pembuka : 

Tanda Salib 

P 
: Dalam nama Bapa dan Putera dan Roh Kudus. 
U 
: Amin 
P 
: Semoga Allah Bapa mengasihi kita dengan belas 
kasihNya. Semoga Allah Bapa senantiasa menerangi kita 
dengan sabdaNya dan semoga Allah Roh Kudus 
mempersatukan kita semua. 
U 
: Sekarang dan selama-lamanya. 
74 



Tobat 

P 
: Saudara-saudari sekalian menyadari kita manusia berdosa 
dan dosa serta kesalahan saudara kita tercinta yang 
meninggal 40 hari yang lalu........maka marilah kita 
dengan rendah hati bersujud dihadapanNya untuk 
memohon belas kasihanNya dann pengampunanNya. 
P+U : Saya mengaku…….. 

P 
: Semoga Allah yang mahakuasa mengasihani kita, 
mengampuni dosa kita dan mengantar kita ke dalam hidup 
yang kekal. 
U : Amin 
Doa Pembuka 

P 
: Marilah Berdoa 
Allah Bapa yang mahamurah, Engkau telah menyerahkan 
Yesus, Putra-Mu kepada kematian, semua ini harus terjadi 
untuk melepaskan kami dari segala kuasa kegelapan dan 
dosa. Ya Bapa, anugerahkanlah hidup kekal kepada 
saudara-saudari ……….yang telah menghadap 
kehadiratMu 40 hari yang lalu. Ya Bapa, ampunilah 
segala dosa dan kesalahannya dan bukalah pintu 
kehidupan kekal baginya. Terimalah saudara kami 
tercinta ini kedalam keluarga kudusMu di tahta surgawi. 
U : Amin 
BACAAN KITAB SUCI 

P 
: Saudara-saudari terkasih marilah kita mempersiapkan hati 
dan budi untuk mendengarkan sabda Tuhan. 
Bacaan Pertama 
Pembacaan dari Kitab Nabi Yesaya (25: 7-9) 


75 



Dan diatas gunung ini Tuhan akan mengoyakkan kain 
perkabungan yang diselubungkan kepada segala suku bangsa 
dan tudung yang ditudungkan kepada segala bangsa-bangsa. Ia 
akan meniadakan maut untuk seterusnya ; dan Tuhan Allah 
akan menghapuskan air mata dari pada segala muka ; dan aib 
umatNya akan dijauhkanNya dan seluruh bumi, sebab Tuhan 
telah mengatakannya. Pada waktu itu orang akan berkata : 
“Sesungguhnya, inilah Allah kita, yang kita nanti-nantikan, 
supaya kita diselamatkan. Inilah Tuhan yang kita nantinantikan 
; marilah kita bersorak-sorak dan bersukacita oleh 
karena keselamatan yang diadakan-Nya !” 

P : Demikianlah sabda Tuhan 
U : Syukur kepada Allah 
Antar Bacaan 

Bacaan Injil 

P : Tuhan sertamu 
U : Dan sertamu juga 
P : Inilah Injil Yesus Kristus menurut Lukas (7:11-17) 
U : Dimuliakanlah Tuhan 
Yesus membangkitkan anak muda di Nain 

Kemudian Yesus pergi ke suatu kota yang bernama Nain. 
Murid-muridNya pergi bersama-sama dengan Dia, dan juga 
orang banyak menyertaiNya, berbondong-bondong. Setelah Ia 
dekat pintu gerbang kota, ada orang mati diusung ke luar, anak 
laki-laki, anak tunggal ibunya yang sudah janda, dan banyak 
orang dari kota itu menyertai janda itu. Dan ketika Tuhan 
melihat janda itu, tergeraklah hatiNya oleh belas kasihan, lalu 

76 



Ia berkata kepadanya :”Jangan menangis!” Sambil 
menghampiri usungan itu Ia menyentuhnya, dan sedang para 
pengusung berhenti, Ia berkata : “Hai anak muda, Aku berkata 
kepadamu, bangkitlah !” Maka bangunlah orang itu dan duduk 
dan mulai berkata-kata, dan Yesus menyerahkannya kepada 
ibunya. Semua orang itu ketakukan dan mereka memuliakan 
Allah, sambil berkata :”Seorang nabi besar telah muncul di 
tengah-tengah kita,” dan “Allah telah melawat umatNya.” 
Maka tersiarlah kabar tentang Yesus di seluruh Yudea dan di 
seluruh daerah sekitarnya. 

P : Demikianlah Injil Tuhan 
U : Terpujilah Kristus 
Renungan Singkat 

Doa Umat 

P 
: Saudara-saudari, 
Hati Yesus tergerak oleh belas kasihan karena melihat 
seorang ibu yang sedang mengalami duka yang 
mendalam, maka marilah kita bersama-sama berdoa, 
semoga Tuhan Yesus tergerak pula hatiNya untuk 
memperhatikan dan mengabulkan permohonanpermohonan 
kita bersama. 
P 
: Bagi saudara-saudari ….. 
Semoga kemurahatian dan belas kasih Kristus yang telah 
membangkitkan anak muda dari Naim juga 
membangkitkan saudara…………dari kematiannya dan 
menerima karunia hidup kekal. 
Hening sejenak……………marilah kita mohon : 
U : Kabulkanlah doa kami 
77 



P 
: Bagi semua orang yang sedang berduka karena kematian 
sanak saudara mereka semoga Tuhan Mahakasih 
membukakan selubung duka hati mereka dan memberikan 
semangat penuh harapan bahwa hidup ini bukan menuju 
kepada kematian akan tetapi menuju kepada kepenuhan 
hidup manusia dalam kehendak Bapa sendiri. 
Hening sejenak………..marilah kita mohon : 
U : Kabulkanlah doa kami 
P 
: Marilah kita juga berdoa bagi mereka yang sedang 
mengalami sakratul maut, semoga mereka akhirnya dapat 
meninggal dengan damai dan tenang. Semoga dengan 
belaskasihan Kristus mereka dapat masuk dalam kesatuan 
dengan para kudus di surga abadi. 
Hening sejenak…………marilah kita mohon : 
U : Kabulkanlah doa kami 
P 
: Saudara-saudari sekalian marilah kita persatukan semua 
doa permohonan dan harapan kita dengan doa yang 
diajarkan Yesus sendiri. 
BAPA KAMI 

P : Ya Allah, Bapa kami, 
Semoga Engkau berkenan mengabulkan doa-doa yang 
kami panjatkan ke hadiratMu. Selamatkanlah saudara 
kami yang tercinta……serta semua orang yang sudah 
meninggal. Anugerahkanlah istirahat dan damai abadi 
bagi mereka semua, sebab Engkaulah Tuhan kami, 
sepanjang segala masa. 
U : Amin 

78 



PENUTUP 
Doa Penutup 

P : Marilah berdoa : 
Allah Bapa kami yang mahapengasih dan penyanyang, 
semoga kebangkitan putraMu juga menjadi kebangkitan 
saudara kami……, Bapa semoga Engkau senantiasa 
membangkitkan semangat kami untuk terus menerus hidup 
seturut nasihat InjilMu. Bapa, semoga doa-doa yang kami 
panjatkan kehadiratMu mampu mengantar saudara-saudari 
kami yang sudah meninggal untuk memasuki kerajaanMu 
yang abadi di surga. 
U : Amin 

Berkat Pengutusan 

P 
: Saudara-saudari sekalian, 
Dengan ini upacara doa kita sudah selesai, semoga kita 
selalu diberkati oleh Allah yang mahakuasa : Bapa dan 
Putera dan Roh Kudus. 
U : Amin 
Lagu Penutup 

79 



\chapter{IBADAT PERINGATAN ARWAH HARI KE-100}
 
Pembuka 

P 
: Saudara-saudari terkasih, 
Hari ini kita bersama-sama berdoa bersama untuk 
mendoakan arwah dari saudara kita………yang pada 100 
hari yang lalu di panggil Bapa. Kita percaya bahwa segala 
doa yang kita panjatkan untuk mereka yang sudah 
meninggal sangat bermanfaat demi terwujudnya harapan 
iman mereka untuk berdiam di rumah Tuhan selamalamanya. 
Lagu Pembukaan 

Tanda Salib 

P 
: Dalam nama Bapa dan Putra dan Roh Kudus 
U 
: Amin 
P 
: Semoga Allah memberikan rahmat dan sejahtera yang 
berlimpah kepada kita semuanya. 
U 
: Sekarang dna selama-lamanya 
Tobat 

P 
: Saudara-saudari, 
Marilah kita dengan jujur dan ikhlas kita mengakui segala 
dosa dan kelemahan kita dihadapan Allah dan sesama ; marilah 
kita mohon ampun atas segala kelemahan dan kedosaan yang 
telah kita perbuat. Secara khusus kita mohonkan ampun juga 
atas kesalahan dan dosa dari saudara kita…….yang dipanggil 
Bapa di surga 100 hari yang lalu. 

80 



Ulangan : Sebab kami orang berdosa 


P : Kasihanilah kami ya Bapa 
U : Sebab kami orang berdosa 
P : Tunjukkanlah belas kasihan-Mu kepada kami 
U : Sebab kami orang berdosa 
P : Semoga Allah yang mahakuasa mengasihani kita, 
mengampuni dosa kita, dan mengantar kita ke hidup yang 
kekal. 
U : Amin 

Doa Pembuka 

P : Marilah berdoa : 
Allah 
mahakasih, Engkaulah pencipta dan penebus kami. 
Tuhan Yesus telah berjaya dengan mengalahkan maut dan 
masuk kedalam kemuliaan abadiMu. Semoga ya Bapa, 
hambaMu ini saudara……………..juga mengalahkan 
maut dan masuk dalam kemuliaanMu untuk selamalamanya. 
Demi Yesus Kristus Putra-Mu, Tuhan dan 
pengantara kini dan sepanjang masa. 

U : Amin 
BACAAN KITAB SUCI 

Bacaan Pertama 
Pembacaan dari Surat Pertama Rasul Paulus kepada umat di 
Korintus (1Kor 15:12-23) 

Kebangkitan Kita 

Jadi, bilamana kami beritakan, bahwa Kristus dibangkitkan 
dari antara orang mati, bagaimana mungkin ada di antara kamu 
yang mengatakan, bahwa tidak ada kebangkitan orang mati ? 

81 



Kalau tidak ada kebangkitan orang mati, maka Kristus juga 
tidak dibangkitkan. Tetapi andaikata Kristus tidak dibangkitkan 
maka sia-sialah pemberitaan kami dan sia-sialah juga 
kepercayaan kamu. Lebih dari pada itu kami ternyata berdusta 
terhadap, karena tentang Dia kami katakan, bahwa Ia telah 
membangkitkan Kristus-padahal Ia tidak membangkitkanNya, 
kalau andaikata benar, bahwa orang mati tidak dibangkitkan. 
Sebab jika benar orang mati tidak dibangkitkan, maka Kristus 
juga tidak dibangkitkan. Dan jika Kristus tidak dibangkitkann, 
maka sia-sialah kepercayaan kamu dan kamu masih hidup 
dalam dosamu. Demikianlah binasa juga orang-orang yang mati 
kedalam Kritus. Jikalau kita hanya dalam hidup ini saja 
menaruh pengharapan pada Kristus, maka kita adalah orang-
orang yang paling malang dari segala manusia. Tetapi yang 
benar ialah, bahwa Kristus telah dibangkitkan dari antara orang 
mati, sebagai yang sulung dari orang-orang yang telah 
meninggal. Sebab sama seperti maut datang karena satu orang 
manusia, demikian juga kebangkitan orang mati datang karena 
satu orang manusia. Karena sama seperti semua orang mati 
dalam persekutuan dengan Adam, demikian pula semua orang 
akan dihidupkan kembali dalam persekutuan dengan Kristus. 
Tetapi tiap-tiap orang menurut urutannya : Kristus sebagai buah 
sulung ; sesudah itu mereka yang menjadi milikNya pada waktu 
kedatanganNya. 

L : Demikianlah Sabda Tuhan 
U : Syukur kepada Allah 
Antar Bacaan 
Bisa dengan lagu atau mazmur 
Dalam pendarasan mazmur : 
Ulangan : Tuhanlah gembalaku aku takkan berkekurangan 

82 



P 
: Tuhanlah gembalaku, aku takkan berkekurangan. Ia 
membaringkan daku di padang rumput yang hijau. Ia 
membimbing aku ke air yang tenang dan menyegarkan 
daku. Ia menuntun aku di jalan yang lurus, demi nama-
Nya yang kudus. 
U : Tuhanlah…. 
P 
: Sekalipun berjalan dalam lembah yang kelam, aku tiada 
takut bahaya, sebab Engkau besertaku. Tongkat 
gembalaan-Mu itulah yang menghibur aku. 
U : Tuhanlah….. 
P 
: Engkau menyediakan hidangan bagiku di hadapan segala 
lawanku. Engkau mengurapi kepalaku dengan minyak, 
pialaku penuh berlimpah. 
U : Tuhanlah….. 
P : Kerelaan dan kemurahan-Mu mengiringi aku seumur 
hidupku. Aku akan diam di dalam rumah Tuhan 
sepanjang masa. 
U : Tuhanlah….. 

Bacaan Injil 

P : Tuhan sertamu 
U : Dan sertamu juga 
P : Inilah Injil Yesus Kristus menurut Yohanes (6:37-40) 
U : Dimulikanlah Tuhan 
Semua yang diberikan Bapa kepada-Ku akan datang 
kepada-Ku, dan barangsiapa datang kepada-Ku, ia tidak akan 
Kubuang. Sebab Aku telah turun dari sorga bukan untuk 
melakukan kehedak-Ku, tetapi untuk melakukan kehendak Dia 
yang telah mengutus Aku. Dan inilah kehendak Dia yang telah 

83 



mengutus Aku, yaitu supaya dari semua yang telah 
diberikanNya kepada-Ku jangan ada yang hilang, tetapi supaya 
Kubangkitkan pada akhir zaman. Sebab inilah kehendak 
BapaKu, yaitu supaya setiap orang, yang melihat Anak dan 
yang percaya kepada-Nya beroleh hidup yang kekal, dan 
supaya Aku membangkitkannya pada akhir zaman. 

P : Demikianlah Sabda Tuhan 
U : Terpujilah Kristus 
Renungan Singkat 

Doa Umat 

P 
: Saudara-saudari terkasih, 
Marilah kita panjatkan doa-doa bagi arwah saudara kita 
yang sudah meninggal juga bagi kita semua dan Gereja 
yang terus menerus. 
P 
: Bagi saudara/i kita ………yang sudah dipanggil Bapa 100 
hari yang lalu. Semoga melalui pembaptisan yang telah 
diterima saudara kita…..dan berkat iman akan Yesus 
sepanjang hidupnya, ia dianugerahi hidup kekal yang 
telah dijanjikan Allah sendiri kepadanya. 
Hening sejenak……Marilah kita mohon : 
U : Kabulkanlah doa kami ya Tuhan 
P 
: 
Bagi para uskup dan para imam kita yang sudah 
meninggal. 
Semoga para pemimpin Gereja kita yang sudah dipanggil 
menghadap Bapa, diikutsertakan dalam perayaan surgawi. 
Hening sejenak…….Marilah kita mohon : 

U : Kabulkanlah doa kami, ya Tuhan 
84 



P : Bagi siapa saja yang sudah meninggal dengan harapan 
akan bangkit kembali. 
Semoga semua orang yang meninggal dengan harapan 
akan bangkit lagi, diterima dalam pangkuan surgawi 
abadi. 
Hening sejenak……Marilah kita mohon : 
U : Kabulkanlah doa kami, ya Tuhan 

P 
: Bagi kita semua yang ada di tempat ini. 
Semoga iman pengharapan kita akan Yesus Kristus 
semakin dikuatkan dalam perjuangan hidup keseharian 
kita. 
Hening sejenak…….Marilah kita mohon : 
U 
: Kabulkanlah doa kami, ya Tuhan 
P : Saudara-saudara terkasih, 
Marilah kita satukan semua doa kita dengan doa yang diajarkan 
Kristus sendiri. 
BAPA KAMI 
P + U : Bapa Kami……
. 


PENUTUP 
Doa Penutup 


P 
: Allah Bapa kami yang mahabaik, semoga doa kami 
berguna bagi keselamatan saudara……yang sudah 
dipanggil Bapa 100 hari yang lalu. Ya Bapa, bebaskanlah 
dan bersihkanlah dia dari segala dosanya, limpahkanlah 
penebusan-Mu kepadanya. Demi Kristus Tuhan kami. 
U 
: Amin 
Berkat Pengutusan 

P 
: Saudara-saudari sekalian 
85 



Dengan ini upacara untuk mendoakan keselamatan arwah 
saudara kita………sudah selesai. Semoga kita sekalian 
senantiasa diberkati oleh Allah, Bapa yang mahakuasa. 
Dalam nama Bapa dan Putra dan Roh Kudus 

U : Amin 
Lagu Penutup 

86 



\chapter{IBADAT PERINGATAN ARWAH HARI KE – 1000} 

Pembuka 

P 
: Saudara-saudari terkasih, 
Kematian tidak memutuskan hubungan kita dengan 
orang-orang yang sudah meninggal. Tuhan Allah tetap 
mempersatukan kita dengan saudara kita…….yang 
dipanggil Bapa 1000 hari yang lalu. Karena kesatuan hati 
itulah kita pada hari ini berkumpul bersama untuk 
mendoakan arwah saudara kita. Hal ini kita lakukan 
sebagai bentuk kasih dan kesatuan hati kita dengan 
almarhum. Semoga doa kita semua akan mendatangkan 
keselamatan kekal bagi saudara kita…….. Maka marilah 
kita awali kesatuan doa kita dengan menyanyikan lagu 
pembuka sebagai lagu pujian kita pada Allah. 
Lagu Pembuka 
Tanda Salib 

P : Dalam nama Bapa dan Putra dan Roh Kudus 
U : Amin 
P : Semoga Allah Bapa serta Tuhan kita Yesus Kristus 
memberikan karunia dan kesejahteraan kepada kita 
sekalian. 
U : Sekarang dan selama-lamanya 

Tobat 

P 
: Saudara-saudari sekalian, 
Marilah kita mempersiapkan diri, dengan memeriksa 
batin dan hidup harian kita masing-masing. Mengingat 
kita orang yang berdosa maka marilah kita memohon 
87 



ampun dan belas kasihan dari Allah Bapa yang 
Maharahim. Secara khusus kita juga memohonkan ampun 
bagi dosa-dosa saudara kita……..yang dipanggil Allah 
1000 hari yang lalu. 
Hening sejenak…… 

P : Tuhan Yesus Kristus, Engkau diutus menyembuhkan 
orang yang remuk redam hatinya. 
Tuhan kasihanilah kami 
U : Tuhan kasihanilah kami 

P 
: Engkau datang memanggil orang yang berdosa 
Kristus kasihanilah kami 
U : Kristus kasihanilah kami 
P 
: Engkau datang memanggil orang yang berdosa 
Kristus kasihanilah kami 
U : Kristus kasihanilah kami 
P 
: Engkau duduk di sisi Bapa sebagai pengantara kami. 
Tuhan kasihanilah kami 
U : Tuhan kasihanilah kami 
P 
: Semoga Allah yang mahakuasa mengasihani kita, 
mengampuni dosa kita, dan mengantar kita ke hidup yang 
kekal. 
U : Amin 
Doa Pembuka 

P : Marilah berdoa : 
Allah Bapa yang mahamurah, Engkau telah menebus dosa 
kami dengan wafat dan kebangkitan Putra-Mu terkasih. 
Kasihanilah kiranya hambaMu, saudara kami…….yang sudah 

88 



Engkau panggil 1000 hari yang lalu. Saudara kami ……percaya 
bahwa setelah kematian ada kebangkitan dan kehidupan abadi 
di surga. Semoga saudara kami ini juga Engkau perkenankan 
untuk menikmati kebahagiaan kekal abadi di surga. Demi 
Yesus Kristus, Putra-Mu, Tuhan dan pengantara kami, yang 
bersatu dengan Dikau dan Roh Kudus hidup dan berkuasa kini 
dan sepanjang masa. 

U : Amin 
BACAAN KITAB SUCI 
Bacaan Pertama 
Pembacaan dari Surat Rasul Paulus kepada umat di Kolese 
(1:12-20) 

Saudara-saudari 

Marilah kita mengucap syukur dengan suka cita kepada 
Bapa, yang melayakkan kamu untuk mendapat bagian dalam 
apa yang ditentukan untuk orang orang kudus di dalam kerajaan 
terang. Ia telah melepaskan kita dari kuasa kegelapan dan 
memindahkan kita ke dalam Kerajaan Anak-Nya yang kekasih; 
didalam Dia kita memiliki penebusan kita, yaitu pengampunan 
dosa. Ia adalah gambar Allah yang tidak kelihatan, yang 
sulung, lebih utama dari segala yang diciptakan, karena di 
dalam Dialah telah diciptakan segala sesuatu, yang ada di sorga 
dan yang ada di bumi, yang kelihatan dan yang tidak kelihatan, 
baik singgasana, maupun kerajaan, baik pemerintah, maupun 
penguasa; segala sesuatu diciptakan oleh Dia dan untuk Dia. Ia 
ada terlebih dahulu dari segala sesuatu dan segala sesuatu ada 
di dalam Dia. Ialah kepada tubuh, yaitu jemaat. Ialah kepala 
tubuh, yaitu jemaat. Ialah yang sulung, yang pertama bangkit 
dari antara orang mati, sehingga Ia yang lebih utama dalam 
segala sesuatu. Karena seluruh kepenuhan Allah berkenan diam 
di dalam Dia, dan oleh Dialah Ia memperdamaikan segala 

89 



sesuatu dengan diri-Nya, baik yang ada di bumi, maupun yang 
ada di sorga, sesudah Ia mengadakan pendamaian oleh darah 
salib Kristus. 

L : Demikianlah Sabda Tuhan 
U : Syukur kepada Allah 
Antar Bacaan 
Antar Bacaan bisa menyanyikan sebuah lagu atau mendaraskan 
mazmur. 

Mazmur 25:1-5 
Ulangan : Bimbinglah kami menurut Sabda-Mu, ya Tuhan 

P : Kepada-Mu kuarahkan hatiku, ya Tuhan Allahku 
U : Bimbinglah…. 
P 
: Kepada-Mu aku percaya, janganlah mengecewakan daku, 
janganlah musuh bersukacita atas kemalanganku. 
U : Bimbinglah…. 
P : Perkenalkanlah jalan-Mu kepadaku, ya Tuhan, 
tunjukkanlah lorong-Mu kepadaku. 
P : Bimbinglah aku menurut Sabda-Mu yang benar dan 
ajarilah aku, karena Engkaulah Allah penyelamatku, 
kepada-Mu aku selalu berharap. 
U : Bimbinglah…. 

Bacaan Injil 

P : Tuhan sertamu 
U : Dan sertamu juga 
P : Inilah Injil Yesus Kristus menurut Yohanes (5:24-29) 
U : Dimuliakanlah Tuhan 
90 



Aku berkata kepadamu : Sesungguhnya barangsiapa 
mendengar perkataan-Ku dan percaya kepada Dia yang 
mengutus Aku, ia mempunyai hidup yang kekal dan tidak 
turut dihukum, sebab ia sudah pindah dari dalam maut ke 
dalam hidup. Aku berkata berkata kepadamu : 
Sesungguhnya saatnya akan tiba dan sudah tiba, bahwa 
orang-orang mati akan mendengar suara Anak Allah, dan 
mereka yang mendengarnya, akan hidup. Sebab sama 
seperti Bapa mempunyai hidup dalam diri-Nya sendiri, 
demikian juga diberikan-Nya Anak mempunyai hidup 
dalam diri-Nya sendiri. Dan Ia telah memberikan kuasa 
kepada-Nya untuk menghakimi, karena Ia adalah Anak 
Manusia. Jangalah kamu heran akan hal itu, sebab saatnya 
akan tiba, bahwa semua orang yang di dalam kuburan akan 
mendengar suara-Nya, dan mereka yang telah berbuat baik 
akan keluar dan bangkit untuk hidup yang kekal, tetapi 
mereka yang telah berbuat jahat akan bangkit untuk 
dihukum. 

P : Demikianlah Sabda Tuhan 
U : Terpujilah Kristus 
Renungan Singkat 

Doa Umat 

P 
: Saudara-saudari terkasih, 
Marilah kita menyampaikan dao-doa permohonan kita : 
P 
: Bagi keselamatan saudara kita…… 
Semoga berkat kesetiaannya kepada Yesus sang Penebus 
dan Penyelamat kita, saudara kita……dihantar dan 
dianugerahi masuk ke dalam kerajaan-Nya untuk 
menikmati kedamaian abadi. 
Hening sejenak……Marilah kita mohon : 
91 



U : Kabulkanlah doa kami, ya Tuhan 
P : Bagi kita yang hadir di sini : 
Semoga Kristus Penyelamat kita menghancurkan 

kekuasaan dosa supaya kita juga layak menerima hidup 
abadi dalam Dia. 
Hening sejenak…….Marilah kita mohon 
: 


U : Kabulkanlah doa kami, ya Tuhan 
P 
: Bagi semua yang sedang berdukacita : 
Semoga Kristus penghibur orang yang berdukacita 
menghapuskan air mata serta dukacita dari hati orang-
orang yang sedang berkabung karena kehilangan sanak 
saudara mereka. 
Hening sejenak……..Marilah kita berdoa : 
U : Kabulkalah doa kami, ya Tuhan 
P 
: Bagi semua orang yang percaya kepada Kristus : 
Semoga semua orang yang percaya kepada Kristus kelak 
pada akhir zaman berbahagia dan bersatu dalam surga 
abadi dengan semua orang yang telah meninggal dalam 
iman kepada-Nya. 
Hening sejenak…..Marilah kita mohon : 
U : Kabulkanlah doa kami, ya Tuhan 
P 
: Marilah kita satukan semua doa-doa permohonan kita 
dengan doa yang diajarkan Kristus sendiri. 
Bapa Kami 

P+U : Bapa Kami…… 
Ya Bapa, bebaskanlah kami dari segala kemalangan dan 
berilah kami damai-Mu. Bantulah kami supaya bersih dari 
noda dosa sambil mengharapkan kedatangan Kristus 
Tuhan kami. 

P+U : Amin 

Penutup 

92 



Doa Penutup 

P 
: Marilah berdoa : 
Tuhan Yesus Kristus, bantulah kami dan bimbinglah kami 
dengan Roh Kudus-Mu, supaya kami selalu mendengar, 
percaya dan menghidupi sabdaMu. Hantrkanlah kami 
semua dalam ziarah hidup ini agar sampai ke rumah-Mu 
dan kelak bersatu kembali dengan saudara-saudarai kami 
yang sudah mendahului kami. Sebab Engkaulah Tuhan 
dan Perantara kami. 
P+U : Amin 

Berkat Pengutusan 

P 
: Saudara-saudari sekalian, 
Dengan ini doa kita sudah selesai. Semoga Tuhan selalu 
beserta kita sekalian. Demi nama Bapa dan Putra dan Roh 
Kudus. 
U : Amin 
Lagu Penutup 

93 



\chapter*{LAGU-LAGU PUJIAN}
\addcontentsline{toc}{chapter}{Lagu-lagu Pujian} 

\renewcommand{\thesection}{\arabic{section}.}

\section{TETAP CINTA YESUS} 
Kumau cinta Yesus selamanya 
Kumau cinta Yesus selamanya 
Meskipun badai silih berganti Dalam hidupku 
Kutetap cinta Yesus selamanya 

Ya Bapa, Bapa ini aku anakMu 
Layakkanlah seluruh hidupku 
Ya Bapa, Bapa ini aku anakMu 
Pakailah sesuai dengan rencanaMu 

\section{KASIH DARI SURGA} 
Kasih dari surga memenuhi tempat ini 
Kasih dari bapa surgawi 
Kasih dari Yesus mengalir di hatiku 
Membuat damai di hidupku 

Mengalir kasih dari tempat tinggi 
Mengalir kasih dari tahta Allah bapa 
Mengalir, mengalir, mengalir dan mengalir 
Mengalir memenuhi hidupku 

(03) KINI SAATNYA 
Kini saatnya berdiri di altarNya 
S’bab Allah Maha Kudus hadir di sini 
Marilah memuji angkat tangan menyembah 
S’bab Allah Maha Kudus hadir di sini 

94 



Kita masuk tahta suciNya 
Bersama para malaikat menyembah 
Mari puji Yesusku 
Kita masuk hadiratNya Maha Kudus 

(04) SEJAUH TIMUR DARI BARAT 
Sejauh timur dari barat Engkau membuang dosaku 
Tiada kau ingat lagi pelanggaranku 
Jauh ke dalam tubir laut, Kau melemparkan dosaku 
Tiada Kau perhitungkan kesalahanku 

Betapa besar kasih pengampunanMu, Tuhan 
Tak Kau pandang hina hati yang hancur 
Kuberterima kasih kepadaMu ya Tuhan 
Pengampunan yang Kau bri pulihkanku 

(05) BERI PENGAMPUNAN 
Dengan rendah hati aku mengaku 
Atas dosaku s’lama ini 
Ya Tuhan, Maha Pengasih 
B’ri pengampunan untukku 

Kristus Putra bapa, Krsitus Juru S’lamat 
Sungguh kusesali dosaku 
Aku bersalah padaMu 
B’ri pengampunan untukku 

(06) KUSIAPKAN HATIKU TUHAN 
Kusiapkan hatiku tuhan ‘tuk dengar FirmanMu saat ini 
Kusujud menyembahMu Tuhan Dalam hadiratMu saat ini 

Curahkan urapanMu Tuhan Bagi jemaatMu saat ini 
Kusiapkan hatiku tuhan’tuk dengar FirmanMu 

95 



Reff: 
FirmanMu Tuhan tiada berubah 
Dahulu sekarang selama-lamanya, tiada berubah 
FirmanMu Tuhan Penolong hidupku 
Kusiapkan hatiku Tuhan ‘tuk dengar FirmanMu 

(07) FIRMANMU PELITA BAGI KAKIKU 
Firmanmu p’lita bagi kakiku terang bagi jalanku 
FirmanMu p’lita bagi kakiku terang bagi jalanku 

Waktu kubimbang dan hilang jalanku 
Tetaplah Kau di sisiku 
Dan tak’kan ku takut sal Kau di dekatku 
Besertaku selamanya 

(08) BETAPA HATIKU 
Betapa hatiku berterima kasih Yesus, Kau mengasihiku, Kau 
memilikiku 

Hanya ini Tuhan persembahanku, Segenap hidupku jiwa dan ragaku 
Sbab tak kumiliki harta kekayaan, Yang cukup berarti ‘tuk 
kupersembahkan 

Hanya ini Tuhan permohonanku, Terimalah Tuhan persembahanku 
Pakailah hidupku sebagai alatMu, Seumur hidupku 

(09) BAPA SURGAWI 
Bapa surgawi, ajarku mengenal 
Betapa dalamnya kasihMu 
Bapa surgawi buatku mengerti 
Betapa kasihMu padaku 

Semua yang terjadi di dalam hidupku 
Ajarku menyadari Kau selalu sertaku 

96 



Bri hatiku selalu, bersyukur padaMu 
Karna rencanaMu indah bagiku 

(10) BAYU SENJA 
Hidup bagai biduk di laut lepas 
Aku pelaut tunggal siap melaju 
Reff; 
O bayu senja, hembusan sang Ilahi 
Bawa bidukku ke tepian cerah 
Pantai umat tebusan 

Arus gelombang tantang biduk tak daya 
Hati merayu Tuhan nada nan cemas 
>>reff 
Malam kabut nan pekat bintang pun lenyap 
Setia kunanti Dikau wujud tak tampak >>reff 

(11) AVE MARIA 
Engkau yang dipilih Allah Bapa di surga 
Untuk melahirkan PutraNya yang kudus 
Engkaulah Bunda Kristus 
Bunda sang Penebus s’gala dosa manusia 

Bunda Maria p’rawan yang tiada bernoda 
Hatimu bersinar putih tiada bercela 
Engkau Bunda Almasih yang diangkat 
Ke surga penuh kemuliaan 

Ave maria, ave Maria 
Terpujilah Bunda, terpuji namaMu 
S’panjang s’gala masa 
Ave maria, ave Maria Syukur kepadaNya 
Tuhan yang Pengasih S’lama-lamanya 

97 



(12) TUHAN ADALAH GEMBALAKU 
Tuhan adalah gembalaku, Tak’kan kekurangan aku 
Ia membaringkan aku, Di padang yang berumput hijau 

Reff 
Ia membimbingku ke air yang tenang 
Ia menyegarkan jiwaku 
Ia menuntunku ke jalan yang benar 
Oleh karna namaNya 
Sekalipun aku berjalan, dalam lembah kekelaman 

Aku tidak takut bahaya, Sebab Engkau besertaku 
GadaMu dan tongkatMu, Itulah yang menghibur aku 

(13) KAU YANG TERINDAH 
Kau yang terindah di dalam hidup ini 
Tiada Allah Tuhan yang seperti Engkau 
Besar perkasa penuh kemuliaan 

Kau yang termanis di dalam hidup ini 
Kucinta Kau lebih dari segalanya 
Besar kasih setiaMu kepadaku 

Reff: 
Kusembah Kau ya Allahku, Kutinggikan namaMu selalu 
Tiada lutut tak bertelut, Menyembah Yesus Tuhan Rajaku 

Kusembah Kau ya Allahku, Kutinggikan namaMu selalu 
Selalu lidah kan mengaku, Engkaulah Yesus Tuhan Rajaku 

(14) BAPA SURGAWI 
Bapa surgawi ajarku mengenal 
Betapa dalamnya kasihMu 
Bapa surgawi buatku mengerti 

98 



Betapa kasihMu padaku 
Semua yang terjadi, di dalam hidupku 
Ajarku menyadari Kau selalu sertaku 
Bri hatiku selalu, bersyukur padaMu 
Karna rencanaMu indah bagiku 

(15) ALLAH PEDULI 
Banyak perkara yang tak dapat kumengerti 
Mengapakah harus terjadi 
Didalam kehidupan ini 

Satu perkara yang kusimpan dalam hati 
Tiada sesuatu kan terjadi tanpa Allah peduli 

Allah mengerti, Allah peduli 
Segala persoalan yang kita hadapi 
Tak akan pernah dibiarkannya 
Kubergumul sendiri 
S’bab Allah peduli 

(16) ONE DAY AT THE TIME 
I’m only human, I’m just a man 
Help me believe in what I could be 
And all that I am 
Show me the stairway I have to climb 
Lord for my sake; Teach me to take 
One day at a time 

Chorus: 
One day at a time sweet Jesus 
That’s all I’m asking from you 
Just give me a strength to do everything 
What I have to do 
Yesterday’s gone sweet Jesus 
Tomorrow may never be mine 

99 



Help me today show me the way 
One daya at a time 

Do you remember 
When You walk among men 
Well Jesus know 
If you are looking below 
It’s worst now and then 
Cheating and stealing 
Violence and crimed 
So for may sake Lord teach me to take 
One day at a time>>>back to chorus 

(17) IN MOMENT LIKE THIS 
In moment like this, I sing out a song 
I sing out a love song to Jesus 
In moment like this, I lift up my hands 
I lift up my hands to the Lord 
Singing I love you, Lord (3x) I love you 

(18) GIVE THANKS 
II: Give thanks with a greatful heart 
Give thanks to the Holy One 
Give thanks because He’s given 
Jesus Christ, His son: II 
II: And now let the weak say I’m strong 
Let the poor say I’m rich 
Because of what the Lord has done for us: II 
Give thanks 
(19) TUBUHKU YESUS 
TubuhMu Yesus sucikan daku 
TubuhMu Yesus bebaskanku 

100 



TubuhMu Yesus ubahkan daku 
Ku dijadikan baru 

(20) DARAHMU YESUS 
DarahMu Yesus sucikan daku 
DarahMu Yesus bebaskanku 
DarahMu Yesus ubahkan adaku 
Ku dijadikan baru 

(21) EL SHADAI 
Tak usah ku takut, Allah menjagaku 
Tak usah ku bimbang, Yesus p’liharaku 
Tak usah ku susah, Roh Kudus hiburku 
Tak usah ku cemas, Dia memberkatiku 

El Shadai, El Shadai, Allah Maha kuasa 
Dia besar, Dia besar, el Shadai mulia 
El Shadai, El Shadai, Allah Maha kuasa 
BerkatNya melimpah, El Shadai 

(22) TIAP LANGKAHKU 
Tiap langkahku diatur oleh Tuhan 
Dan tangan kasihNya memimpinku 
Di tengah badai dunia menakutkan 
Hatiku tetap tenang teduh 

Reff: 
Tiap langkahku ku tahu Tuhan yang pimpin 
Ke tempat tinggi ku dihantarNya 
Hingga sekali nanti aku tiba 

Di rumah bapa, surga yang baka 
Di waktu imanku mulai goyah 
Dan bila jalanku hampir sesat 

101 



Kupandang Tuhanku Penebus dosa 
Ku teguh sebab Dia dekat>>Reff 

Di dalam tuhan saja harapanku 
Sebab di tanganNya sejahtera 
DibukaNya Yerusalem yang baru 
Kota Allah yang suci mulia>>Reff 

(23) DI DOA IBUKU 
Di waktuku masih kecil gembira dan senang 
Tiada duka kukenal, tak kunjung mengerang 
Di sore hari nan sepi ibuku bertelut 
Sujud berdoa kudengar, namaku disebut 

Seringlah ini kukenang di masa yan benar 
Di kala hidup mendesak dan nyaris kusesat 
Melintas gambar ibuku sewaktu bertelut 
Kembali sayup kudengar namaku disebut 

Reff…. 
Sekarang dia telah pergi ke rumah yang senang 
Namun kasihnya padaku selalu kukenang 
Kelak di sana kami pun bersama bertelut 
Memuji Tuhan yang dengar namaku disebut 
Reff……. 

(24) TUHAN BERIKANLAH 
Tuhan berikanlah istirahat 
Abadi dan tenang bagi yang wafat 
Beri pengampunan segala dosanya 
Karna Maha murah hatiMu Allah 

Kami berimankan sabda Putra 
Aku kebangkitan dan kehidupan 
Barang siapalah percaya ‘kan daku 

102 



Ia akan hidup untuk selamanya 

Kami menantikan saat itu 
Maut akan lenyap diganti hidup 
Smoga kami kelak memandang wajahMu 
Di sinari terang dalam rumahMu 

(25) INDAH RENCANAMU TUHAN 
Indah rencanaMU Tuhan di dalam hidupku 
Walau ku tak tahu dan ku tak mengerti semua jalanMu 
Dulu ku tak tahu Tuhan berat kurasakan 
Hati menderita namun tak kuasa menghadapi semua 

Reff: 
Tapi kumengerti s’karang Kau tolong padaku 
Kini kumelihat dan kemerasakan indah rencanaMu>>2x 

(26) BAPA SUNGGUH BAIK 
Bapa, Engkau sungguh baik 
KasihMu melimpah di hidupku 
Bapa, ku bert’rima kasih 
BerkatMu hari ini, yang Kau sediakan bagiku 

Kunaikkan syukurku buat hari yang Kau b’ri 
Tak habis-habisnya kasih dan rahmatMu 
Slalu baru dan tak pernah terlambat pertolonganMu 
Besar setiaMu di s’panjang hidupku 

(27) JALAN TUHAN 
Ada waktu di hidupku 
Pencobaan berat menekan 
Aku berseru mengapa ya Tuhan 
Nyatakan kehendakMu 

103 



Jalan Tuhan bukan jalanku 
Jangan bimbang ataupun ragu 
Nantikan tuhan jadikan semua 
Indah pada waktunya 

Reff: 
Pada Tuhan masa depanku 
Pada Tuhan kus’rahkan hidupku 
Nantikan Tuhan berkarya 
Indah pada waktunya 

Hari esok tiada kutahu 
Namun tetap langkahku maju 
Ku yakin Tuhan jadikan semua 
Indah pada waktunya 

(28) TANGAN TUHAN 
Apa yang kau alami kini 
Mungkin tak dapat engkau mengerti 
Satu hal tanamkan di hati 
Indah semua yang Tuhan beri 

Tuhanku tak akan memberi 
Ular beracun pada yang minta roti 
Cobaan yang engkau alami 
Tak melebihi kekuatanmu 
Reff: 
Tangan Tuhan sedang merenda 
Suatu karya yang agung mulia 
Saatnya ‘kan tiba nanti 
Kau lihat pelangi kasihNya 

104 



(29) HANYA KEPADAMU 
Hanya kepadaMu kami dapat berpasrah 
Tuhan Yesus Kristus 
Di dalam tanganMu hidup dan mati kami 
Engkaulah Penebus 

CintaMu ya tuhan mengalahkan maut 
Hanya Dikau pangkal kehidupan 
Yang selalu kami dambakan 

Ya Yesus sambutlah saudara kami ini 
Dalam rumah Bapa 
Karna Engkau wafat supaya kami hidup 
Untuk selamanya 

Hidup atau mati kami milik Tuhan 
Maka Tuhan bimbinglah umatMu, Di jalan menuju padaMu 

(30) DOAKAN KAMI BUNDA 
Cantik hatimu tiada bernoda 
Bunda penolong umat manusia 
Trimalah nyanyian puji bagimu 
Kar’na kekagumanku pada iman dalam bening hatimu 

Kau perawan suci nan lembut hati 
Terberkati dalam rahmat Ilahi 
Dengarkan selalu doaku O Bunda yang setia 
Kaulah perantara doa kami pada Yesus PutraMu 

Reff: 
Maria, Maria terpujilah engkau untuk selamanya 
Doakan kami Bunda saat ini dan saat ajal nanti 

105 



(31) BUNDA PEMBANTU ABADI 
Pada wajahmu yang suci 
Matamu nampak bening sejuk lembut 
Kau pandang para abdimu berdoa 
Oh Bunda pembantu abadi 
Engkau pagku anakmu Yesus Putra Allah 
Sumber suka dan duka hatimu 
Hanya engkau sendirilah yang tahu 
Pahit dan manisnya hidupku 

Bukanlah kepadamu oh bunda 
Pandangan penuh cintaNya tertuju 
Salib dan tombak bengis dilihatNya 
Oh Bunda pembantu abadi 
Tangan Bunda dipegang didekapNya erat 
Gambaran gelisah manusia 
Bagaikan terbayang sengsara maut 
Siksaan salah manusia 

Matamu ya Bunda suci 
Memberitakan pesanNyaterindah 
Wahau kamu orang-orang berdosa 
Lihatlah juru selamatmu 
Terdengar pesan indah namun kami lemah 
Terbawa gelombang masa kini 
Kami pinta doamu pada Bapa 
Oh Bunda Pembantu abadi 

Pandanglah dunia ini oh bunda 
Dunia yang penuh dengan kebencian 
Doakan perdamaian yang sejati 
Sadarkan hati manusia 
Arahkan pikian tingkah laku kami 
Biarkan tampak cinta sesame 
Bantulah di saat ajalku tiba 
Oh Bunda Pembantu abadi 

106 



-oOo


”Hari ini Juga Engkau Akan Ada Bersama Dengan Aku Di Dalam 


Firdaus.” (Luk 23:43) 


107 


\end{document}

