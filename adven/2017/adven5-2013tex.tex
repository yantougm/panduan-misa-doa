\documentclass[12pt,a4paper]{article}
\usepackage[left=3.00cm, right=2.00cm, top=3.00cm, bottom=3.00cm]{geometry}
\usepackage[latin1]{inputenc}
\usepackage{amsmath}
\usepackage{amsfonts}
\usepackage{amssymb}
\usepackage{palatino}
\begin{document}
\section*{Ibadat Adven V 19 Des 2013}

\subsection*{Nasehat Paulus}
Bacaan-bacaan hari ini sungguh berbicara tentang hidup keluarga kristiani. Paulus memberi nasihat kepada umat di Kolose, bahwa kehidupan kristiani sejati harus dihayati di dalam keluarga kristiani sejati pula! Dan ciri-ciri khas keluarga kristiani sejati ialah: selalu saling mengampuni, kasih, damai dan rasa terima kasih menurut teladan Kristus, yang rela memberikan diri-Nya seutuhnya . Di dalam bacaan Injil hari ini, keluarga kudus dari Nasaret hendaknya menjadi pola dan idola setiap keluarga kristiani, terutama Yusuf hendaknya menjadi teladan pada kaum suami atau para bapak.

Keluarga Yusuf, Maria dan anaknya Yesus bukanlah keluarga kaya, tetapi jelas satu keluarga yang sungguh beriman. Iman hidup keluarga ini sudah tampak dalam diri Yusuf dan Maria sebelum mereka berkeluarga dan keimanan hidup keluarga ini semakin disempurnakan dengan kehadiran Yesus dalam keluarga mereka. Mereka adalah keluarga yang saleh, beriman dan kudus, namun mereka tidak lepas dari penderitaan dan persoalan hidup. Ketika Maria hendak melahirkan, mereka ditolak dan tidak mendapat tempat di pengingapan. Sesudah Yesus lahir, mereka harus mengungsi, melarikan diri karena Herodes si gila hormat, sigila kuasa hendak membunuh Yesus. Walau demikian, keluarga ini dapat menghadapi semuanya dan terlepas dari persoalan itu karena iman dan ketaatan mereka melaksanakan perintah Tuhan dan Tuhan sendirilah yang membebaskan mereka.

\subsection*{Pesta Keluarga Kudus}
Bacaan hari ini diambil dari bacaan Pesta Keluarga Kudus.
Pesta Keluarga Kudus dirayakan setiap tanggal 30 Desember, yang merupakan perayaan liturgi Gereja Katolik untuk menghormati Yesus dari Nazaret, ibunya, Perawan Maria dan bapaknya, Santo Yusuf, sebagai bagian dari kesatuan keluarga.

Secara liturgis, sehabis merayakan Hari Raya Natal umat Katolik merayakan "Pesta Keluarga Kudus, Jesus, Maria dan Yusuf" (lazim disebut Keluarga Kudus dari Nazaret). Implementasinya, merayakan Natal tidak hanya berhenti hanya sampai mengadakan ritual di gereja, melakukan silaturahmi dengan para sahabat, dan pesta-pesta saja; Pesan Natal harus dilaksanakan dalam masyarakat dari hari ke hari bersama anggota masyarakat luas.

Hari Raya Keluarga Kudus diresmikan oleh Paus Leo XIII pada tahun 1893. Hingga Januari 1969 dan dirayakan pada hari Minggu dalam Oktaf dari Epifani. Artinya, dirayakan pada hari Minggu manapun yang jatuh antara tanggal 7 Januari sampai dengan 13 Januari, (lihat Kalender Umum Gereja Katolik Roma 1962). Kini hari raya ini dirayakan pada hari Minggu antara Natal dan Tahun Baru, dalam Oktaf Natal. 
Karena tahun ini Natal jatuh pada hari Rabu maka mestinya untuk tahun ini jatuh pada tgl 29 Des 2013 nanti. 
Apabila pada tahun tersebut tidak ada hari Minggu, karena tanggal 25 Desember dan 1 Januarinya jatuh pada hari Minggu, peringatan ini diselenggarakan pada hari Jumat sebelum tanggal 30 Desember pada tahun tersebut.

Perayaan ini tidak termasuk hari suci yang wajib dirayakan, namun kehadiran mengikuti Misa yang dirayakan pada hari Minggunya ketika hari itu diperingati, wajib dilakukan, seperti halnya pula dengan perayaan-perayaan hari suci lainnya yang diperingati pada suatu hari Minggu. Ketika Perayaan Keluarga Kudus dipindahkan ke tanggalnya yang sekarang, tempatnya di dalam kalender diambil oleh Perayaan Baptisan Tuhan.

Di kalangan Katolik ada kebiasan untuk menuliskan "JMJ" di bagian atas surat-surat dan dan catatan pribadi sebagai referensi untuk Yesus, Maria, dan Yusuf, para anggota dari Keluarga Kudus.

Kenapa keluarga Yusuf dan Maria disebut keluarga kudus? Apakah karena mereka tidak pernah cekcok? Apakah karena mereka selalu rajin pergi ke sinagoga atau apakah mereka tidak pernah melakukan kesalahan dalam hidup berkeluarga? Saya yakin bukan! Mereka disebut Keluarga Kudus karena hadir di sana, Yesus Kristus Putra Allah. Kehadiran Yesus menguduskan hidup keluarga itu secara lahir maupun batin. Suasana hidup keluarga dipengaruhi oleh kasih dan damai yang dibawa Yesus.

Hendaknya keluarga-keluarga kristiani menjadi keluarga yang saleh seperti keluarga kudus dari Nazareth. Keluarga yang merasa bahwa doa adalah sebuah kebutuhan. Maria dan Yusuf adalah orang Yahudi sejati dan mereka juga mendidik Yesus untuk bertumbuh dengan ukuran kesalehan sebagai manusia. Dia sungguh-sungguh manusia. Tetapi Yesus juga sungguh-sungguh Allah maka persatuan dengan Bapa di surga memiliki daya kasih yang luar biasa yang juga mempersatukan setiap pribadi. 

Setiap keluarga kristiani yang senantiasa hidup dalam imannya pasti akan selalu diberkati, dilindungi Tuhan dan Tuhan akan menyelamatkan mereka. Dengan demikian, jaminan keutuhan dan kebahagiaan keluarga kristiani itu, bukanlah harta, jabatan, pangkat tetapi iman yang hidup dan dirayakan dalam keluarga. Keluarga kristiani mestinya tidak menganggap bahwa uang atau hartalah yang utama dalam keluarga, sehingga melupakan hidup iman mereka. 


\subsection*{Kejujuran dan Ketulusan}
Dalam memperingati Keluarga Kudus, tokoh Yesus, Maria dan Yusuf menjadi teladan bagi kita semua. Kita tampilkan lebih dulu teladan Yusuf. Segenap sikap dasar hidupnya berlandasan pada kehendak dan sabda Allah. Kepentingan Allah selalu dilihat dan dilaksanakan secara mutlak. Di samping itu, kepentingan sesama manusia pun, justru sebagai konsekuensinya, harus dihayati apabila kita sungguh ingin menjadi orang beriman kristiani sejati. Kejujuran dan ketulusan hati sungguh mutlak sebagai syarat keselamatan. Yusuf adalah teladan orang yang tulus! Meskipun hanya tukang kayu, ia telah berperan sebagai bapak yang menyertai kehidupan Yesus. Memang, Yusuf tidak banyak ditulis dalam Kitab Suci. Dalam kehidupan Gereja pun keteladanan Yusuf tidak banyak mendapat perhatian umat. Padahal, justru dalam kesederhanannya dan dalam kurangnya dikenal serta kurangnya mendapat perhatian itulah letak kebesaran Yusuf. 

Kaum bapak melihat dan meneladan Yusuf sebagai kepala keluarga dalam keluarga kristiani. Kita ketahui bahwa Yusus sebagai kepala keluarga dalam keluarga Nasaret tahun bahwa Yesus anak Maria bukan dari dirinya, namun walaupun demikian iman, ketulusan hatinya tetap membuat dia memelihara dan berusaha menyelamatkan keluarganya khususnya Maria dan Yesus. Secara khusus sebenarnya Yusuf menyelamatkan Yesus, karena Herodes hanya mau membunuh Yesus. Kita semua tahu bahwa Yesus adalah kehadiran Allah dalam hidup manusia, kehadiran Kerajaan Allah yang hendak menyelamatkan manusia.

Peran suami dalam keluarga memang sebagai kepala keluarga, sebagai pemimpin dan bahkan sebagai penyelamat hidup keluarga itu. Namun kenyataannya kadangkala terjadi, para suami menganggap bahwa sebagai kepala keluarga bisa berbuat sesuka hati bagi isteri dan anak-anaknya, menganggap bahwa tugasnya cukup hanya bekerja, mencari nafkah sedangkan tugas untuk mendidik dan memelihara iman anak hanyalah tugas isteri. Walaupun saat ini sudah banyak isteri yang bekerja di luar atau mencari tambahan penghasilan bagi keluarga, demi membantu suami dalam mempertahankan hidup keluarga. Ini terjadi karena merasa sebagai kepala, bos dan tuan dalam keluarga itu, kerjanya hanya mengatur, memerintahkan dan menekan. Pendangan seperti ini sangatlah keliru dan bukan gambaran suami yang kristiani. Suami menjadi penanggungjawab utama dalam iman dan kelangsungan hidup keluarga

Oleh karena itu, dalam usaha kita meneladan Keluarga Kudus, sungguh bergunalah bagi kita untuk menyadari bahwa kehidupan kita sebagai orang beriman katolik sejati harus juga disertai dengan kejujuran dan ketulusan hati seperti Yusuf.

Sudah siapkah kita berperilaku dengan tulus hati dan tidak mau mencemarkan nama baik orang lain di muka umum? Memiliki ketulusan hati berarti suci, bersih dan tiada dosa sedikitpun. Nampaknya hal itu sungguh berat bagi kita semua. Namun demikian, marilah kita saling membantu dan mengingatkan untuk hidup dan bertindak dengan tulus hati di mana pun dan kapan pun. Tulus hati juga berarti jujur, yaitu sikap dan perilaku yang berani mengakui kesalahan, serta rela berkorban untuk kebenaran, dan. tidak suka berbohong dan tidak suka berbuat curang.

Salah satu bentuk konkret ketulusan hati adalah tidak mau mencemarkan nama baik orang lain di muka umum kapan saja dan di mana saja. Banyak di antara kita suka ngrasani atau ngrumpi yang pada umumnya berisi menjelek-jelekan orang lain atau membicarakan kekurangan dan kelemahan orang lain. Dengan demikian, kekurangan atau kelemahan orang yang bersangkutan dieber-eber dan dibesar-besarkan. Hal yang dermikian itu termasuk melanggar hak asasi manusia dan melanggar cintakasih: demikian juga membicarakan kekurangan atau kelemahan orang lain untuk bercanda.


\subsection*{Kasih dan Pasrah}
Kita dapat meneladan Maria dari kasihnya, ketaatannya dan kepasrahannya kepada Tuhan, yang terungkap dalam jawabannya, "Aku ini hamba Tuhan, terjadilah padaku menurut kehendak-Mu".

\subsection*{Rendah Hati}
Kita dapat meneladan Yesus dari segi kerendahan hati-Nya. Walaupun dalam rupa Allah, Yesus telah mengosongkan diri-Nya sendiri, dan mengambil rupa seorang hamba, dan menjadi sama dengan manusia. Dan dalam keadaan sebagai manusia, Ia telah merendahkan diri-Nya dan taat sampai mati, bahkan sampai mati di kayu salib.

\subsection*{Kesimpulan}
Oleh karena itu, kalau kita orang-orang percaya ini mau menjadikan keluarga kita sebagai keluarga kudus, pertama-tama bukan berarti keluarga kita harus bebas dari salah, bukan berarti tiap-tiap anggota tidak melakukan kekeliruan, melainkan keluarga kita mau menerima serta membiarkan diri dipengaruhi oleh Yesus, yang kita imani.
Semoga keluarga dan komunitas kita menjadi kudus dan suci. Semoga kita tetap merindukan kehadiran Yesus di tengah-tengah keluarga kita. Semoga kita pun siap mewartakan kehadiranMu, bukan dengan kata-kata, melainkan dengan perbuatan baik kita sehari-hari. Amin.
\end{document}