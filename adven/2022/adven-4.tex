\documentclass[11pt,a5paper]{article}
\usepackage[T1]{fontenc}
\usepackage{amsmath}
\usepackage{amssymb}
\usepackage{makeidx}
\usepackage{graphicx}
\usepackage{microtype}
\usepackage{palatino}
\usepackage[left=2.00cm, right=1.00cm, top=2.00cm, bottom=1.00cm]{geometry}
\author{Yohanes Suyanto}
\date{22 Desember 2022}
\title{Doa Adven 2022 (keempat)}
\begin{document}
\maketitle
"Catholic Theology in the 21st Century" adalah sebuah buku yang ditulis oleh Thomas P. Rausch yang berisi tentang teologi Katolik di abad ke-21. Buku ini mencoba untuk memberikan gambaran tentang bagaimana teologi Katolik telah berkembang di abad terakhir ini, serta mengeksplorasi beberapa isu teologis yang penting bagi Gereja Katolik saat ini.

Buku ini terbagi menjadi beberapa bab, yang setiap bab membahas topik teologis yang berbeda. Bab-bab tersebut antara lain membahas tentang agama dan masyarakat, teologi kristen dan pluralisme, teologi kristen dan keberagaman, teologi kristen dan masalah-masalah sosial, serta teologi kristen dan masa depan. Buku ini merupakan sumber informasi yang bermanfaat bagi siapa saja yang ingin memahami lebih lanjut tentang teologi Katolik di abad ke-21.


\section*{Doa pembuka}
Ya Allah, kami bersyukur atas kesempatan ini untuk bertemu dalam sarasehan Adven keempat ini. Kami bersyukur atas kehadiran-Mu yang selalu dekat dengan kami dan memberikan kekuatan kepada kami untuk terus belajar dan tumbuh dalam ajaran-Mu. Kami berdoa agar dapat semakin dekat dengan-Mu dan terus menjalani kehidupan ini sesuai dengan ajaran-Mu yang suci.

Ya Allah, kami memohon terang Roh Kudus agar sarasehan Adven keempat ini dapat memberikan kesadaran semangat "berjalan bersama" dan mengembangkan paguyuban di lingkungan maupun paroki sesuai dengan cita-cita ARDAS DAN RIKAS KAS. Kami berdoa agar dapat terus tumbuh dalam kekristenan dan menjadi bagian dari gereja-Mu yang terus berkembang. Ya Allah, kami memohon penyertaan-Mu agar dapat memaknai Adven sebagai masa mempersiapkan kedatangan Kristus sang Juru Selamat Dunia. Kami berdoa agar dapat menjalani kehidupan ini dengan menghayati dan memahami ajaran-Mu yang suci, serta terus tumbuh dalam keimanan dan ketaqwaan kepada-Mu. Kami juga memohon agar dapat selalu dekat dengan-Mu dan terus mengikuti jejak-Mu yang suci. Demi Kristus Tuhan dan pengantara kami. Amin
 

\section*{Doa umat}
\begin{enumerate}
	\item Tuhan, kami memuji Engkau karena Engkau telah mengirimkan Anak-Mu, Yesus, yang telah dijanjikan oleh para nabi. Kami bersyukur atas kehadiran-Nya di dunia ini, yang telah memulihkan hubungan kami dengan Engkau. Kami bersyukur atas Perawan Maria yang dengan cinta mesra menyambut Kedatangan-Nya. Kami berdoa agar kami dapat selalu mengingat kebesaran-Mu dan menjadi saksi tentang kehadiran-Mu di dunia ini. Kami berdoa agar kami dapat mencintai dan mengikuti Jesus sebagai Adam baru yang telah membuka jalan bagi kami untuk kembali kepada Engkau. . 

\item Tuhan, kami bersyukur atas kehadiran Yesus di dunia ini. Ia membawa damai sejati bagi kami dan membantu lebih banyak orang mengenal Engkau dan berani melaksanakan kehendak-Mu. Meskipun ia datang ke dunia ini sebagai manusia biasa, ia memiliki tujuan yang besar: untuk melaksanakan rencana-Mu dan membukakan jalan keselamatan bagi kami. Kami bersyukur atas kemuliaan-Mu yang akan ditunjukkan pada akhir zaman, saat Yesus datang kembali dengan semarak dan mulia untuk menyatakan kebahagiaan yang kami nantikan. Kami berdoa agar kami dapat selalu mengingat bahwa Engkau adalah Tuhan yang agung dan merindukan kehadiran-Mu. .

\item Tuhan, kami mohon kepada-Mu kelimpahan rahmat-Mu. Kami berdoa agar selama hidup di dunia ini, kami selalu siap siaga dan penuh harap menantikan kedatangan-Nya yang mulia. Kami berdoa agar pada saat Jesus datang nanti, Engkau mengizinkan kami ikut berbahagia bersama-Nya dan seluruh umat kesayangan-Mu. Kami percaya bahwa dengan rahmat-Mu, kami dapat menjadi pribadi yang siap menerima kehadiran-Nya. 

\end{enumerate}

\section*{Doa penutup}
Ya Allah, kami bersyukur atas pertemuan yang berjalan lancar ini. Terima kasih atas penyertaan-Mu yang membuat kami dapat mengalami pemahaman yang lebih dalam tentang ajaran-Mu. Kami berdoa agar kami mampu dengan sukacita mengobarkan dan menggelorakan iman kami, serta terus tumbuh dalam kebaikan dan kebenaran.
Ya Allah, kami memohon penyertaan-Mu agar dapat semakin mengobarkan dan menggelorakan iman kami. Kami berdoa agar dapat terus tumbuh dalam keimanan dan ketakwaan kepada-Mu. Kami juga memohon agar dapat dengan sukacita menjalani hidup ini sesuai dengan ajaran-Mu yang mulia.  Demi Kristus Tuhan dan pengantara kami. Amin
\end{document}
