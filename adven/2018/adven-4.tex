\documentclass[11pt]{beamer}
\usepackage{polyglossia}
\usepackage[utf8]{inputenc}
\usepackage[T1]{fontenc}
\usetheme{JuanLesPins}
\usecolortheme{seahorse}

\setdefaultlanguage{bahasai}

\setbeamercolor{block title}{bg=red!37!blue, fg=white}

\begin{document}
	\author{Lingkungan St. Theresia}
	\subtitle{MEMAJUKAN TATA HIDUP BERSAMA}
	\title{Adven ke-4 2018}
	%\logo{}
	%\institute{}
	%\date{}
	%\subject{}
	%\setbeamercovered{transparent}
	%\setbeamertemplate{navigation symbols}{}
	\frame[plain]{\maketitle}
	
	\begin{frame}
		\transboxin
		\frametitle{Pembuka}
	 	\begin{block}{Nyanyian pembuka}
	 	\end{block}	
 	\pause
	 	\begin{block}{Doa pembuka}
	 	
	 	
	 	Allah Bapa yang Maha Kuasa, kami bersyukur boleh hidup di negara dan bangsa Indonesia
	 	yang berdasarkan Pancasila. Semoga nilai-nilai luhur Pancasila mendasari seluruh nafas
	 	kehidupan warga negara kami, menjadi dasar dalam tata hidup bersama kami. Semoga
	 	dalam pertemuan ini, kami semakin Engkau sadarkan untuk semakin menghadirkan kasih
	 	Kristus dala tata kehidupan bersama kami. Demi Kristus, Tuhan dan pengantara kami.
	 	
	 	 Amin.
	 	\end{block}	
	 	
	\end{frame}

	\begin{frame}
		\frametitle{Pembuka}
	\begin{block}{Penyalaan Lilin Korona}
		
		\small Allah Bapa yang Maha Kasih, kami telah memasuki masa Adven, masa dimana kami
		menantikan akan kedatangan Putera-Mu terkasih. Kami mohon, semoga Jilin adven ini
		menerangi hati dan menuntun kami untuk menghadirkan Peradaban Kasih bagi sesama,
		lingkungan dan bangsa kami ini. Semoga dengan bimbingan sabda-Mu kami dapat
		menggiatkan lingkungan sebagai pusat hidup beriman yang semakin terbuka, mampu
		berdialog dan membawa perubahan baru dalam masyarakat. Semoga kami dapat
		menjadikan semua orang untuk semakin sejahtera, bermartabat dan beriman sesuai dengan
		nilai Pancasila. Akhirnya, kami semakin pantas untuk menyambut Putera-Mu yang lahir ditengah-tengah kami. Permohonan ini kami sampaikan kepada-Mu dengan pengantaraan
		Kristus, Tuhan kami yang hidup dan berkuasa bersama Engkau dan Roh Kudus, sepanjang
		segala masa.
		
		Amin
	\end{block}
\end{frame}

\begin{frame}
	\frametitle{Inspirasi}
	\begin{block}{Bacaan Inspiratif}
		"Siapa yang bilang politik itu jahat atau tidak bagus"
	\end{block}
 	\pause
	\begin{block}{Pendalaman Bersama}
		\begin{itemize}
			\item Bagaimana tanggapan Anda dengan kritikan Romo Benny "Dewan paroki datar-
			datar saja. Kalau urusan pembangunan gereja umat antusias, kalau urusan kaderisasi
			umat tidur karena merasa tidak penting." Berikanlah tanggapan Anda
			\item Menurut Anda, apa yang dapat kita lakukan untuk ikut serta terlibat dalam
			memajukan tata hidup bersama? Berikanlah tanggapan Anda?
		\end{itemize}
	\end{block}
\end{frame}

\begin{frame}
	\frametitle{Refleksi Kateketis dan Simpul Pertemuan}
	\begin{block}{Kutipan Kitab Suci Mat 21:12-17}
	\small Lalu Yesus masuk ke Bait Allah clan mengusir semua orang yang berjual beli di halaman Bait
	Allah. Ia membalikkan meja-meja penukar uang dan bangku-bangku pedagang merpati dan
	berkata kepada mereka: "Ada tertulis: Rumah-Ku akan disebut rumah doa. Tetapi kamu
	menjadikannya sarang penyamun." Maka datanglah orang-orang buta dan orang-orang
	timpang kepada-Nya dalam Bait Allah itu dan mereka disembuhkan-Nya. Tetapi ketika imam-
	imam kepala dan ahli-ahli Taurat melihat mujizat-mujiza t yang dibuat-Nya itu dan anak-anak
	yang berseru dalam Bait Allah: "Hosana bagi Anak Daud! " hati mereka sangat jengkel, lalu
	mereka berkata kepada-Nya: "Engkau dengar apa yang dikatakan anak-anak ini?" KataYesus kepada mereka: "Aku dengar; belum pernahkah kamu baca: Dari mulut bayi-bayi dan
	anak-anak yang menyusu Engkau telah menyediakan puji-pujian? "lu Ia meninggalkan
	mereka dan pergi ke luar kota ke Betania dan bermalam di situ.	
	\end{block}
\end{frame}

\begin{frame}
	\frametitle{Refleksi Kateketis dan Simpul Pertemuan}
	\begin{block}{Renungan dan Simpul}
	\begin{itemize}
		\item<1-> Yesus marah karena bait Allah yang menjadi rumah
		doa dijadikan tempat berdagang
		\item<2-> Yesus siap Dirinya menjadi yang terdepan untuk
		menyucikan kembali bait Allah, sekalipun sikapnya itu mendatangkan resiko bagi
		kehidupan-Nya.
		\item<3-> Negara dan pemerintahan adalah rumah bersama seluruh warga. Namun ada pejabat yang telah mencermarkan kesucian pemerintahan.
		\item<4-> Menjadi orang baik saja tidak cukup, kita
		panggil juga untuk memperbaiki tata hidup bersama kita, terutama dari kejahatan
		struktural.
	\end{itemize}
	\end{block}
\end{frame}

\begin{frame}
	\frametitle{Penegasan Bersama dan Penutup}
	\begin{block}{Pengendapan}
		Ada sepenggal kutipan dari Surat Yakobus 1:23-24, yang menyatakan "Sebab jika
		seorang hanya mendengar firman saja dan tidak melakukannya, ia adalah
		seumpama seorang yang sedang mengamat-amati mukanya yang sebenarnya di
		depan cermin. Baru saja ia memandang dirinya, ia sudah pergi atau ia segera lupa
		bagaimana rupanya."?. Bagaimanakah diri Anda, apakah hanya sekedar marah,
		mengkritik dan berdiam diri. Yesus menghendaki agar Anda berbuat sesuatu bagi
		tertatanya kehidupan bersama yang lebih adil, sejahtera dan demokratis.
	\end{block}
	\begin{block}<2->{Berdoa} \small
		Berdoalah agar Roh Kudus membantu kita untuk hidup sesuai panggilan iman kita,
		yaitu terlibat dalam memajukan tata hidup bersama. Kita satukan dalam doa yang
		diajarkan Kristus sendiri yakni doa "Bapa Kami".
	\end{block}
\end{frame}

\begin{frame}
	\frametitle{Penegasan Bersama dan Penutup}
	\begin{block}{Doa penutup}
		Bapa yang Maha Rahim dan Bijaksana, kami bersyukur karena semakin Engkau sadarkan,
		akan tugas panggilan iman kami untuk semakin menghadirkan Kristus dalam keterlibatan
		bermasyarakat dengan memajukan tata hidup bersama. Semoga kami dapat menjadi orang
		Katolik yang mampu melaksanakan secara bebas dan bertanggungjawab mewujudkan
		kesejahteraan bersama dengan lebih bermartabat dan dengan semangat iman akan Kristus.
		Akhirnya, kami semakin Engkau pantaskan menyambut kelahiran Yesus Kristus dalam
		perayaan Natal nanti. Demi Kristus Tuhan kami. Amin.
	\end{block}
	\begin{block}<2->{Nyanyian Penutup}
		MB:
	\end{block}
\end{frame}


\end{document}