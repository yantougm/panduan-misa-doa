\documentclass[a5paper,headsepline,titlepage,11pt,nnormalheadings,DIVcalc]{scrbook}
\usepackage[a5paper,backref]{hyperref}
\usepackage[papersize={165mm,215mm},twoside,bindingoffset=0.5cm,hmargin={2cm,2cm},
				vmargin={2cm,2cm},footskip=1.1cm,driver=dvipdfm]{geometry}
%\usepackage[papersize={148mm,215mm},twoside,bindingoffset=0.5cm,hmargin={2cm,2cm},
%				vmargin={2cm,2cm},driver=dvipdfm]{geometry}
%\usepackage{palatino}
\usepackage{graphicx}
\usepackage{wrapfig}
\usepackage[bahasa]{babel}
\usepackage{fancyhdr}
\usepackage{longtable}
\usepackage{hhline,multirow}
\usepackage{pst-node}

%\setlength{\voffset}{0.5in}
%\setlength{\oddsidemargin}{28pt}
%\setlength{\evensidemargin}{0pt}
\renewcommand{\footrulewidth}{0.5pt}
\lhead[\fancyplain{}{\thepage}]%
      {\fancyplain{}{~}}
\rhead[\fancyplain{}{~}]%
      {\fancyplain{}{\thepage}}
\pagestyle{plain}
\lfoot[\emph{Informasi 2010}]{}
\rfoot[]{\emph{Lingkungan St Petrus Maguwo}}
\cfoot{}

\newcommand{\BU}[1]{\begin{itemize} \item[U:] #1 \end{itemize}}
\newcommand{\BI}[1]{\begin{itemize} \item[I:] #1 \end{itemize}}
\newcommand{\BP}[1]{\begin{itemize} \item[P:] #1 \end{itemize}}
\title{Informasi Lingkungan St. Petrus}
\author{Wilayah Yohanes De Britto \\Stasi Maguwo \\Paroki Kalasan}
\date{2010}
\hyphenation{sa-u-da-ra-ku}
\hyphenation{ke-ri-ngat}
\hyphenation{je-ri-tan}
\hyphenation{hu-bung-an}
\hyphenation{me-nya-dari}
\hyphenation{Eng-kau}
\hyphenation{ke-sa-lah-an}
\hyphenation{ba-gai-ma-na}
\hyphenation{Tu-han}
\hyphenation{di-per-ca-ya-kan}
\hyphenation{men-ja-uh-kan}
\hyphenation{bu-kan-lah}
\hyphenation{per-sa-tu-kan-lah}
\hyphenation{ma-khluk}
\hyphenation{Sem-buh-kan-lah}
\hyphenation{ja-lan}
\hyphenation{mem-bu-tuh-kan}
\hyphenation{be-ri-kan-lah}
\hyphenation{me-ra-sa-kan}
\hyphenation{te-man-ilah}
\hyphenation{mem-bi-ngung-kan}
\hyphenation{di-ka-gum-i}
\hyphenation{ta-ngis-an-Mu}
\hyphenation{mi-lik-ilah}

\renewcommand*\thesection{\arabic{section}.}
\setlength{\parindent}{0mm} 

\begin{document}
\section*{hasjashjaj}
\newpage
{\Large \center DOA SYUKUR 2010\\UMAT ALLAH KEUSKUPAN AGUNG SEMARANG}

Allah Bapa Maha Pemurah,\\
kami bersyukur kepada-Mu\\
sebab Engkau senatiasa membimbing kami,\\
seluruh umat-Mu di Keuskupan Agung Semarang,\\
untuk bersahabat dengan-Mu,\\
mengangkat martabat pribadi manusia,\\
dan melestarikan keutuhan ciptaan.\\

Terlebih kami bersyukur kepada-Mu\\
atas habitus baru dalam paguyuban-paguyuban\\
di tengah umat-Mu,\\
yang menumbuhkembangkan semangat berbagi.\\
Kami juga bersyukur atas keluarga-keluarga\\
yang menjadi basis hidup beriman,\\
atas anak-anak, remaja, dan kaum muda\\
yang semakin terlibat dalam pengembangan umat,\\
dan segala upaya pemberdayaan saudara-saudari kami\\
yang kecil, lemah, miskin, tersingkir, dan difabel.\\

Bersama Bunda Maria,\\
hamba-Mu dan bunda kami,\\
kami mohon,\\
utuslah Roh Kudus-Mu\\
untuk melanjutkan pekerjaan baik\\
yang telah Engkau mulai di tengah kami\\
agar kami dapat menjadi saksi budaya kasih dan kebenaran-Mu\\
bagi masyarakat dan lingkungan hidup kami.\\

Doa syukur dan permohonan ini\\
kami hunjukkan kepada-Mu\\
dengan pengantraan Kristus, Tuhan kami. Amin.

\newpage

\begin{center}
Puasa Katolik
\end{center}

1=C 4/4 \hspace{3cm} D. Bambang Sutrisno, Pr.


\begin{tabular}{l|l|l}
$\overline{5~~~5}~$&$~3~~~\overline{.~~~5}~~~\overline{6~~~5}~~~\overline{6~~~5}~~$&$~~3~~~.~~~.$\\
\multicolumn{3}{l}{Me-nu-ju ~~~~~~pe-ra-ya-~an ~Pas~-~ka}\\
\multicolumn{3}{l}{Per-ta-ma ~~~~~memperbanyak do~-~a}\\
\\
$\overline{5~~~5}~$&$~3~~~\overline{.~~~5}~~~\overline{6~~~5}~~~\overline{4~~~3}~~$&$~~2~~~.~~~0$\\
\multicolumn{3}{l}{Pu-a~-~~sa ~~~~empat pu-luh ha~~-~~ri}\\
\multicolumn{3}{l}{Ke-du~-~a ~~~~~memperdalam i~~-~~man}\\
\\
$\overline{2~~~3}~$&$~4~~~\overline{.~~~4}~~~\overline{6~~~6}~~~\overline{5~~~4}~~$&$~~3~~~.~~~.$\\
\multicolumn{3}{l}{Ka-re~-~na ~~~~li-bur ~~ha-ri Ming-~gu}\\
\multicolumn{3}{l}{Ke-ti~-~ga ~~~~kurang makan mi~-~num}\\
\\
$\overline{\dot{1}~~~7}~$&$~6~~~\overline{.~~~6}~~~\overline{7~~~7}~~~\overline{6~~~7}~~$&$~~\dot{1}~~~.~~~0~~~~~$\\
\multicolumn{3}{l}{Pu-a~-~sa ~~~~~mu~~lai ha-ri ~Ra~-~bu.}\\
\multicolumn{3}{l}{Keem~pat menyumbang yang pa-pa.}\\

\end{tabular}

\subsection*{Lembar isian/Evaluasi Pertemuan I APP KAS 2010}
\begin{enumerate}
\item Apa yang paling mengesan/disukai/disyukuri/yang sudah baik dari kegiatan Aksi Puasa selama ini (pribadi, keluarga, lingkungan, paroki, dst)?

\item Hal-hal pa yang belum dimengerti atau baru pertama kali didengar tentang Aksi Puasa Pembangunan?

\item Apakah ada usul-usul untuk meningkatkan keterlibatan umat dalam kegiatan APP (pendalaman tema APP, Aksi nyata, dan pengumpulan derma APP)?

\item Apakah sudah ada panitia APP di tingkat PAroki? Sejauh mana diketahui tentang peran panitia APP Paroki?

\end{enumerate}
\end{document} 
