\documentclass[12pt,a4paper]{article}
\usepackage[latin1]{inputenc}
\usepackage{microtype}
\usepackage{palatino}
\usepackage[width=142.50mm, height=210.00mm, left=10.00mm, right=10.00mm, top=10.00mm, bottom=10.00mm]{geometry}
\author{Yohanes Suyanto}
\newcommand{\BU}[1]{\begin{itemize} \item[U:] #1 \end{itemize}}
\newcommand{\BP}[1]{\begin{itemize} \item[P:] #1 \end{itemize}}

\begin{document}
\large
\section*{Bacaan dan Renungan 30-10-2015}
\subsection*{Bacaan}
\BP{Semoga Tuhan beserta kita}
\BU{Sekarang dan selama-lamanya}
\BP{Inilah Injil Yesus Kristus menurut Santo Lukas}
\BU{Terpujilah Kristus}
\BP{Pada suatu hari Sabat Yesus datang ke rumah salah seorang pemimpin dari orang-orang Farisi untuk makan di situ. Semua yang hadir mengamat-amati Dia dengan saksama.

Tiba-tiba datanglah seorang yang sakit busung air berdiri di hadapan-Nya.
Lalu Yesus berkata kepada ahli-ahli Taurat dan orang-orang Farisi itu, kata-Nya: "Diperbolehkankah menyembuhkan orang pada hari Sabat atau tidak?"

Mereka itu diam semuanya. Lalu Ia memegang tangan orang sakit itu dan menyembuhkannya dan menyuruhnya pergi.
Kemudian Ia berkata kepada mereka: "Siapakah di antara kamu yang tidak segera menarik ke luar anaknya atau lembunya kalau terperosok ke dalam sebuah sumur, meskipun pada hari Sabat?"
Mereka tidak sanggup membantah-Nya.
}

\subsection*{Renungan}
\BP{
"Waktu hari Minggu, aku lihat tukang bubur yang biasa lewat di depan rumahku, itu tuh Pak Jali, langganan kita, gerobaknya kesenggol bis kota. Sampai tumpah semua buburnya. Kasihan deh," cerita Lili pada Kiki siang itu. "Wah, Pak Jali ya! Terus kamu tolong dia?" tanya Kiki ingin tahu. "Nggak, habis kata Firman Tuhan kan kalo hari Sabat kita nggak boleh bekerja, tapi Pak Jali malah jualan bubur. Jadi aku pikir yah Tuhan lagi menghukum dia," ujar Lili cepat. "Ya ampun Li! Itu kamu dengar dari mana? Firman Tuhan nggak bilang begitu! Kalo Pak Jali nggak jualan, gimana dia bisa kasih makan anak-anaknya? Nanti kita ke rumahnya deh! Aku mau kasih uang jajanku hari ini, biar nggak seberapa. Kasihan Pak Jali, jual bubur kan untungnya nggak seberapa, ini malah buntung," ujar Kiki.

Wah benar, ternyata Tuhan Yesus ingin kita berbuat baik pada hari Sabat! Kenapa ya? Hari Sabat memang pada masa itu disebut sebagai hari perhentian, atau hari istirahat. Biasanya manusia akan datang menyembah Tuhan pada hari Sabat tersebut. Namun bagi Yesus menyembah dan berbakti kepada Tuhan bukan hanya lewat persembahan atau pujian, namun terlebih lagi haruslah lewat perbuatan baik bagi semua orang. Bukankah Tuhan tetap menolong kita ketika kita meminta pertolongan-Nya pada hari Sabat sekalipun?

Bapa di Surga, ajar aku menguduskan hari Sabat dengan melakukan hal-hal baik kepada sesama. Amin.
}

\end{document}