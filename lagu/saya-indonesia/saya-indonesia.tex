\documentclass{article}
\usepackage[paperwidth=15.00cm, paperheight=11.00cm, margin=0.5in]{geometry}
\usepackage{amsmath}
\usepackage{numnot}
\usepackage{palatino}
\begin{document}
\thispagestyle{empty}
\begin{center}
{\huge Saya Indonesia, Saya Pancasila}


{\normalsize Do=F, 4/4 Moderato bersemangat}\\
Syair: Joko Widodo, 2017 \hfill Lagu: Theo Sunu Widodo, 2017
\end{center}
\begin{numnot}[1]{satu}
5, | 1 (. 2) 3 1 | 2 . . 1 | 7, (. 1) 2 7, | 1 . .
L: Sa-ya In-do-ne-sia, sa-ya Pan-ca-si-la.

3 | 4 (. 3) 2 {(1 2)} | 3 . . 3 | 4/ (. 4/) 2 {(3 4/)} | 5 . .
L: Sa-ya In-do-ne-sia, sa-ya Pan-ca-si-la.

5 | 5 (. 4) 3 2 | {1 2 3} 1 | 3 (. 3) 4 5 | 6 . .
L: Sa-ya In-do-ne-sia, sa-ya Pan-ca-si-la.

6 | 6 . . (5 4) | 3 5 . 3 | 4 . (2 1) (. 7,) | 1 . . ||
L: Sa-ya In-do-ne-sia, sa-ya Pan-ca-si-la.
\end{numnot}
\flushright{\textit{pada peringatan}}
\flushright{\textit{Hari Lahir Pancasila, 1 Juni 2017}}
\end{document}

