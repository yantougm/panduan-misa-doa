\documentclass[a5paper,headsepline,titlepage,11pt,nnormalheadings,DIVcalc]{scrbook}
\usepackage[a5paper,backref]{hyperref}
\usepackage[papersize={148.5mm,215mm},twoside,bindingoffset=0.5cm,hmargin={2cm,2cm},
				vmargin={2cm,2cm},footskip=1.1cm,driver=dvipdfm]{geometry}
%\usepackage{palatino}
\usepackage{graphicx}
\usepackage{wrapfig}
\usepackage[bahasa]{babel}
\usepackage{fancyhdr}
\usepackage{marvosym}

\renewcommand{\footrulewidth}{0.5pt}
\lhead[\fancyplain{}{\thepage}]%
      {\fancyplain{}{\rightmark}}
\rhead[\fancyplain{}{\leftmark}]%
      {\fancyplain{}{\thepage}}
\pagestyle{fancy}
\lfoot[\emph{Doa keluarga}]{}
\rfoot[]{\emph{Doa keluarga}}
\cfoot{}

\makeatletter
\newcommand{\judul}[1]{%
  {\parindent \z@ \centering 
    \interlinepenalty\@M \Large \bfseries #1\par\nobreak \vskip 20\p@ }}
\newcommand{\subjudul}[1]{%
  {\parindent \z@ 
    \interlinepenalty\@M \bfseries #1\par\nobreak \vskip 10\p@ }}
\newcommand{\lagu}[1]{%
  {\parindent \z@ 
    \interlinepenalty\@M \slshape \mdseries \Large \textit{#1}\par\nobreak \vskip 10\p@ }}
\newcommand{\keterangan}[1]{%
  {\parindent \z@  \slshape 
    \interlinepenalty\@M \textsl{#1}\par\nobreak  \vskip 5\p@}}

\renewenvironment{description}
               {\list{}{\labelwidth\z@ \itemindent-\leftmargin
                        \let\makelabel\descriptionlabel}}
               {\endlist}
\renewcommand*\descriptionlabel[1]{\hspace\labelsep 
                                \normalfont\bfseries #1 }


\makeatother

\newcommand{\BU}[1]{\begin{itemize} \item[U:] #1 \end{itemize}}
\newcommand{\BI}[1]{\begin{itemize} \item[I:] #1 \end{itemize}}
\newcommand{\BIU}[1]{\begin{itemize} \item[I+U:] #1 \end{itemize}}
\newcommand{\BP}[1]{\begin{itemize} \item[P:] #1 \end{itemize}}
\newcommand{\inputlagu}[1]{\begin{textit} \input{#1} \end{textit}}

\hyphenation{a-kan}
\hyphenation{ba-gi-mu}
\hyphenation{ber-a-da}
\hyphenation{ber-du-a}
\hyphenation{be-ri-kan}
\hyphenation{ber-ka-ta}
\hyphenation{ber-nya-nyi}
\hyphenation{ber-sa-ma}

\hyphenation{dah-syat}
\hyphenation{DA-RAH-KU}
\hyphenation{da-tang}
\hyphenation{di-ka-ta-kan}
\hyphenation{di-pim-pin}
\hyphenation{di-se-rah-kan}
\hyphenation{di-tum-pah-kan}

\hyphenation{Eng-kau}
\hyphenation{ha-dap-an}
\hyphenation{han-tar-kan-lah}
\hyphenation{ha-rap-an}

\hyphenation{ja-lan}
\hyphenation{ja-ngan-lah}

\hyphenation{ka-nak}
\hyphenation{ka-re-na}
\hyphenation{kau-lim-pah-kan}
\hyphenation{Kau-cip-ta-kan}
\hyphenation{ke-bang-kit-an-Nya}
\hyphenation{ke-da-tang-an}
\hyphenation{ke-da-tang-an-Nya}
\hyphenation{ke-dua}
\hyphenation{ke-na-ik-kan-nya}
\hyphenation{ke-pa-daMu}
\hyphenation{ke-ra-him-an}
\hyphenation{ke-se-jah-te-ra-an-mu}
\hyphenation{ko-men-tar}

\hyphenation{la-ma-nya}
\hyphenation{lim-pah-kan}

\hyphenation{ma-nu-sia}
\hyphenation{me-nga-da-kan}
\hyphenation{me-ngan-dung-lah}
\hyphenation{me-ngu-kuh-kan}
\hyphenation{me-la-lui}
\hyphenation{me-lim-pah-kan}
\hyphenation{me-lu-hur-kan}
\hyphenation{me-me-cah-me-cah-kan}
\hyphenation{mem-per-sem-bah-kan}
\hyphenation{me-nan-da-ta-ngan-i}
\hyphenation{men-cin-tai}
\hyphenation{meng-a-lir-kan}
\hyphenation{me-nga-sihi}
\hyphenation{me-nge-lu-ar-kan}
\hyphenation{meng-u-cap-kan}
\hyphenation{meng-ung-kap-kan}
\hyphenation{me-num-buh-kan}
\hyphenation{me-nya-ta-kan}
\hyphenation{me-nye-la-mat-kan}
\hyphenation{me-nye-rah-kan}
\hyphenation{me-nye-rah-kanNya}
\hyphenation{me-ra-ya-kan}

\hyphenation{o-rang}
\hyphenation{o-rang-o-rang}

\hyphenation{pa-sang-kan-lah}
\hyphenation{pa-tut}
\hyphenation{pe-ne-ri-ma-an}
\hyphenation{pe-ngam-pun-an}
\hyphenation{Pe-ngan-ta-ra}
\hyphenation{peng-hi-bur-an}
\hyphenation{per-bu-at-an-nya}
\hyphenation{per-ka-ta-an}
\hyphenation{per-ka-win-an}
\hyphenation{per-ni-kah-an}
\hyphenation{per-se-ku-tu-an}
\hyphenation{per-sem-bah-an}
\hyphenation{rom-bong-an}

\hyphenation{se-la-ma}
\hyphenation{se-ka-li-an}
\hyphenation{se-pan-jang}
\hyphenation{se-ra-ya}
\hyphenation{Su-dar-yan-to}

\hyphenation{te-ta-pi}
\hyphenation{ta-ngan-Mu}
\hyphenation{Tu-han}
\hyphenation{tu-lang}
\hyphenation{tu-lang-tu-lang}

\hyphenation{u-mat-Mu}
\hyphenation{wa-kil}

\hyphenation{ba-gi-mu}
\hyphenation{di-se-rah-kan}
\hyphenation{me-la-lui}
\hyphenation{ka-nak}
\hyphenation{ka-re-na}
\hyphenation{ber-ka-ta}
\hyphenation{te-ta-pi}
\hyphenation{per-ka-win-an}
\hyphenation{pa-tut}
\hyphenation{me-lu-hur-kan}
\hyphenation{ber-nya-nyi}
\hyphenation{di-tum-pah-kan}
\hyphenation{pe-ngam-pun-an}
\hyphenation{ber-a-da}
\hyphenation{kau-lim-pah-kan}
\hyphenation{ke-bang-kit-an-Nya}
\hyphenation{per-ka-ta-an}
\hyphenation{pa-sang-kan-lah}
\hyphenation{DA-RAH-KU}
\hyphenation{ke-na-ik-kan-nya}
\hyphenation{per-sem-bah-an}
\hyphenation{per-se-ku-tu-an}


\setlength{\parindent}{0pt}
\begin{document}

\chapter*{Doa keluarga} 

\section*{Diucapkan bersama- sama: bapak, ibu dan anak- anak}

Allah Bapa di surga, puji syukur kami panjatkan ke hadirat-Mu, atas kesempatan kami bersama- sama menaikkan doa sebagai satu keluarga. Kami percaya bahwa Tuhan Yesus hadir di tengah kami saat ini, sesuai dengan sabda-Nya, ``Sebab di mana dua atau tiga orang berkumpul dalam nama-Ku, di situ Aku ada di tengah- tengah mereka''. Kami percaya ya Yesus, Engkau mendengarkan doa- doa yang kami panjatkan kepada-Mu.

Tuhan Yesus, Engkau adalah pusat Keluarga Kudus, bukan hanya karena Engkau adalah Tuhan dan manusia, tetapi karena sampai akhir hidup-Mu Engkau adalah anak Maria dan Yusuf. Bantulah kami untuk memahami bahwa di dalam keluarga Kristiani, kami membutuhkan kasih-Mu. Berilah kami rahmat kasih yang sejati di dalam keluarga kami sehingga kami rela berkorban satu sama lain, sebab Engkau mengajarkan kasih yang sedemikian kepada kami.

\section*{Doa Bapak (oleh bapak/ papa)}
Tuhan Yesus, aku bersyukur atas keluarga yang Engkau berikan kepadaku. Aku mohon agar aku dapat mencintai keluargaku dengan sepenuh hati. Dalam kesibukanku sehari-hari, bantulah aku agar dapat memberikan waktuku kepada mereka. Berikanlah kepadaku kasih yang berasal daripada-Mu, yaitu kasih sebagaimana Engkau mencintai Gereja-Mu.

Kami mohon, kabulkanlah doa kami, ya Tuhan

\section*{Doa Ibu (oleh ibu/ mama)}
Tuhan Yesus, aku bersyukur atas keluarga yang Engkau berikan kepadaku. Aku mohon agar dapat melayani suami dan anak- anakku dengan kasih yang berasal dari-Mu. Berikanlah aku kerendahan hati dan bantulah aku untuk menjadi pembawa damai bagi keluarga kami, sehingga selalu ada suka cita dan damai sejahtera di dalamnya.

Kami mohon, kabulkanlah doa kami, ya Tuhan

\section*{Doa Bapak dan Ibu}
Ya Tuhan, kami mohon agar Engkau membimbing kami dalam mendidik dan membesarkan anak- anak kami, agar kami dapat mengarahkan mereka kepada-Mu. Berkatilah usaha mereka dalam menuntut ilmu, dalam pergaulan yang baik dan kuatkanlah mereka dalam menghadapi godaan di sekeliling mereka.

Kami mohon, kabulkanlah doa kami, ya Tuhan

\section*{Doa Anak- anak}
O Tuhan, kami bersyukur atas orang tua yang Engkau berikan kepada kami. Kami mohon agar Engkau membimbing mereka dalam menuntun kami menuju hari depan kami. Lindungilah mereka dari bahaya dan segala yang jahat, serta berikanlah kepada mereka kesehatan dan rejeki yang cukup, serta kebijaksanaan sesuai dengan kehendak-Mu. Bantulah kami untuk hidup sesuai dengan perintah- perintah-Mu, taat kepada orang tua dan mengasihi mereka dan saudara- saudari kami.

Kami mohon, kabulkanlah doa kami, ya Tuhan.

\section*{Diucapkan bersama- sama: bapak, ibu dan anak- anak}
Bunda Maria yang terkasih, Bunda Kristus dan Bunda kami, tolonglah kami untuk belajar menimba kekuatan dari devosi kepada Keluarga Kudus yang telah diberikan Allah sebagai teladan bagi kami. Baik di Betlehem, di pengungsian ke Mesir, di Nazaret maupun di Yerusalem, Keluarga Kudusmu selalu merupakan tempat berlindung yang penuh kedamaian. Doakanlah kami, O Bunda, agar damai sejahtera yang datang dari ketaatan untuk melaksanakan perintah- perintah Tuhan, selalu ada di dalam keluarga kami. Doakanlah kami agar kami dapat dengan siap sedia selalu taat kepada kehendak Allah sehingga di dalam hidup kami bersama, kami tidak mementingkan diri sendiri, tetapi menjaga dan memelihara sebagai kepunyaan-Mu segala harta milik yang Tuhan percayakan kepada kami, sebab suatu hari nanti kami akan ditanyai oleh-Nya tentang pertanggungan jawab kami mempergunakan semua itu.

Tuhan segala rahmat dan kebijaksanaan, berikanlah kepada kami rahmat untuk hidup bersama di dalam damai yang kudus dan kebahagiaan. Tunjukkanlah kepada kami bagaimana untuk hidup sabar dan baik, tak mudah berkata- kata kasar dan cepat untuk memaafkan satu sama lain. Semoga kami menjadi seperti Keluarga Kudus di Nazaret- sederhana dan cinta damai, senantiasa rajin membantu sesama kami, dengan suka cita mengemban tugas di dalam Gereja maupun di lingkungan masyarakat kami.
Doa syukur dan permohonan ini kami panjatkan demi Kristus Tuhan kami. Amin.

Bapa Kami \ldots 

Salam Maria \ldots 

Kemuliaan \ldots

Litani Kerendahan Hati
Kardinal Merry dei Vai

Ya Tuhan, ubahlah hatiku, Engkau yang telah menerima penghinaan karena cinta-Mu padaku. Nyatakanlah bagiku kerendahan hati-Mu. Terangilah aku dengan cahaya-Mu, kiranya aku mulai saat ini menghancurkan kesombongan yang ada pada diriku! Ini sumber kemalanganku, rintangan yang membuat aku melawan cinta-Mu! Tuhan, aku telah menjadi musuh bagi diriku sendiri ketika aku mencoba mencari kedamaian dalam diriku, dan bukan dalam diri-Mu. Ya Yesus, yang lembut dan rendah hati, dengarkanlah aku.

Dari keinginan untuk dihargai,                                  bebaskanlah aku.
Dari keinginan untuk dicintai,
Dari keinginan untuk dianggap sebagai orang penting,
Dari keinginan untuk dihormati,
Dari keinginan untuk dipuji,
Dari keinginan untuk lebih disukai daripada menyukai,
Dari keinginan untuk dimintai nasihat,

Dari ketakutan untuk dihina,                                     bebaskanlah aku.                
Dari ketakutan untuk direndahkan,
Dari ketakutan untuk diabaikan,
Dari ketakutan untuk difitnah,
Dari ketakutan untuk dilupakan,
Dari ketakutan untuk diejek,
Dari ketakutan untuk dinodai,
Dari ketakutan untuk dicurigai,

Yesus, berilah aku rahmat untuk berharap;
Supaya orang lain lebih dicintai daripada aku,
Supaya mereka bertambah besar di mata dunia dan aku bertambah kecil,
Supaya mereka dipakai dan aku dikesampingkan,
Supaya mereka mendapat pujian dan aku diabaikan,
Supaya mereka mendapat jalan yang lancar dan aku tersisihkan,
Supaya mereka melebihi aku dalam segala hal,
Supaya mereka lebih suci daripada aku, asalkan aku menjadi suci sesuai kemampuanku.

 
\ps{Doa Kerendahan Hati}
Puji Syukur 1992, No. 141

(Kata “kami /aku” bisa diganti menjadi saya / nama orang … sesuai keperluan)

Allah yang Mahatinggi, Putra-Mu Yesus telah memberikan teladan kerendahan hati yang tiada tara. Walaupun Allah, Ia telah menghampakan diri-Nya, mengambil rupa seorang hamba, dan menjadi sama dengan manusia. Dan dalam keadaan-Nya sebagai manusia, Ia telah merendahkan diri-Nya dengan taat sampai mati, bahkan sampai mati di kayu salib.

Terima kasih, ya Bapa, atas teladan Yesus ini. Berilah kami semangat Yesus sendiri, agar dengan rendah hati kami menganggap orang lain lebih utama daripada kami sendiri.

Bebaskanlah kami dari kesombongan, dan berilah kami ketabahan kalau karena nama-Mu kami direndahkan. Semoga kami tidak sakit hati kalau kami kurang di hargai atau kurang dihormati, kalau kami diabaikan atau dilupakan. Sebaliknya, semoga kami ikut bahagia kalau orang lain berhasil dan mendapat pujian serta penghargaan.

Ya Bapa, jadikanlah hati kami seperti hati Yesus yang lembut dan rendah hati. Sebab Dialah Tuhan, pengantara kami. Amin

Litani Kerendahan Hati
Kardinal Merry dei Vai

Ya Tuhan, ubahlah hatiku, Engkau yang telah menerima penghinaan karena cinta-Mu padaku. Nyatakanlah bagiku kerendahan hati-Mu. Terangilah aku dengan cahaya-Mu, kiranya aku mulai saat ini menghancurkan kesombongan yang ada pada diriku! Ini sumber kemalanganku, rintangan yang membuat aku melawan cinta-Mu! Tuhan, aku telah menjadi musuh bagi diriku sendiri ketika aku mencoba mencari kedamaian dalam diriku, dan bukan dalam diri-Mu. Ya Yesus, yang lembut dan rendah hati, dengarkanlah aku.

Dari keinginan untuk dihargai, bebaskanlah aku.
Dari keinginan untuk dicintai,
Dari keinginan untuk dianggap sebagai orang penting,
Dari keinginan untuk dihormati,
Dari keinginan untuk dipuji,
Dari keinginan untuk lebih disukai daripada menyukai,
Dari keinginan untuk dimintai nasihat,

Dari ketakutan untuk dihina, bebaskanlah aku.
Dari ketakutan untuk direndahkan,
Dari ketakutan untuk diabaikan,
Dari ketakutan untuk difitnah,
Dari ketakutan untuk dilupakan,
Dari ketakutan untuk diejek,
Dari ketakutan untuk dinodai,
Dari ketakutan untuk dicurigai,

Supaya orang lain lebih dicintai daripada aku, Yesus, berilah aku rahmat untuk berharap;
Supaya mereka bertambah besar di mata dunia dan aku bertambah kecil, Yesus, berilah aku rahmat untuk berharap;
Supaya mereka dipakai dan aku dikesampingkan, Yesus, berilah aku rahmat untuk berharap;
Supaya mereka mendapat pujian dan aku diabaikan, Yesus, berilah aku rahmat untuk berharap;
Supaya mereka mendapat jalan yang lancar dan aku tersisihkan, Yesus, berilah aku rahmat untuk berharap;
Supaya mereka melebihi aku dalam segala hal, Yesus, berilah aku rahmat untuk berharap;
Supaya mereka lebih suci daripada aku, asalkan aku menjadi suci sesuai kemampuanku, Yesus, berilah aku rahmat untuk berharap.
Ditulis dalam Doa dan Devosi. Tinggalkan sebuah Komentar »
November 23, 2007 — klinikrohani

Doa Kerendahan Hati
Puji Syukur 1992, No. 141

(Kata “kami /aku” bisa diganti menjadi saya / nama orang … sesuai keperluan)

Allah yang Mahatinggi, Putra-Mu Yesus telah memberikan teladan kerendahan hati yang tiada tara. Walaupun Allah, Ia telah menghampakan diri-Nya, mengambil rupa seorang hamba, dan menjadi sama dengan manusia. Dan dalam keadaan-Nya sebagai manusia, Ia telah merendahkan diri-Nya dengan taat sampai mati, bahkan sampai mati di kayu salib.

Terima kasih, ya Bapa, atas teladan Yesus ini. Berilah kami semangat Yesus sendiri, agar dengan rendah hati kami menganggap orang lain lebih utama daripada kami sendiri.

Bebaskanlah kami dari kesombongan, dan berilah kami ketabahan kalau karena nama-Mu kami direndahkan. Semoga kami tidak sakit hati kalau kami kurang di hargai atau kurang dihormati, kalau kami diabaikan atau dilupakan. Sebaliknya, semoga kami ikut bahagia kalau orang lain berhasil dan mendapat pujian serta penghargaan.

Ya Bapa, jadikanlah hati kami seperti hati Yesus yang lembut dan rendah hati. Sebab Dialah Tuhan, pengantara kami. Amin

Litani Kerendahan Hati
Kardinal Merry dei Vai

Ya Tuhan, ubahlah hatiku, Engkau yang telah menerima penghinaan karena cinta-Mu padaku. Nyatakanlah bagiku kerendahan hati-Mu. Terangilah aku dengan cahaya-Mu, kiranya aku mulai saat ini menghancurkan kesombongan yang ada pada diriku! Ini sumber kemalanganku, rintangan yang membuat aku melawan cinta-Mu! Tuhan, aku telah menjadi musuh bagi diriku sendiri ketika aku mencoba mencari kedamaian dalam diriku, dan bukan dalam diri-Mu. Ya Yesus, yang lembut dan rendah hati, dengarkanlah aku.

Dari keinginan untuk dihargai, bebaskanlah aku.
Dari keinginan untuk dicintai,
Dari keinginan untuk dianggap sebagai orang penting,
Dari keinginan untuk dihormati,
Dari keinginan untuk dipuji,
Dari keinginan untuk lebih disukai daripada menyukai,
Dari keinginan untuk dimintai nasihat,

Dari ketakutan untuk dihina, bebaskanlah aku.
Dari ketakutan untuk direndahkan,
Dari ketakutan untuk diabaikan,
Dari ketakutan untuk difitnah,
Dari ketakutan untuk dilupakan,
Dari ketakutan untuk diejek,
Dari ketakutan untuk dinodai,
Dari ketakutan untuk dicurigai,

Supaya orang lain lebih dicintai daripada aku,Yesus, berilah aku rahmat untuk berharap;
Supaya mereka bertambah besar di mata dunia dan aku bertambah kecil,
Supaya mereka dipakai dan aku dikesampingkan,
Supaya mereka mendapat pujian dan aku diabaikan,
Supaya mereka mendapat jalan yang lancar dan aku tersisihkan,
Supaya mereka melebihi aku dalam segala hal,
Supaya mereka lebih suci daripada aku, asalkan aku menjadi suci sesuai kemampuanku.
Ditulis dalam Doa dan Devosi. Tinggalkan sebuah Komentar »
Devosi Kepada Hati Kudus Yesus
November 23, 2007 — klinikrohani

Novena Kepada Hati Kudus Yesus

(Novena ini dilakukan setiap hari 9X,
berturut-turut pada jam yang sama)

Ya Yesus, Engkau berkata:
“Mintalah maka akan diberkan kepadamu,
carilah maka kamu akan mendapat;
ketuklah maka pintu akan dibukakan bagimu.”
Dengan perantaraan Maria Bunda-Mu
tersuci aku memanggil Engkau,
aku mencari dan memohon kepada-Mu
untuk mendengarkan permohonanku ini.
(Sebutkan karunia yang anda minta)

Ya Yesus, Engkau berkata:
“Apa saya yang kau minta kepada Bapa-Ku dengan
nama-Ku. Dia akan memberikannya kepadamu.”
Aku memohon dengan rendah hati dan penuh
kepercayaan dari Bapa Surgawi dalam nama-Mu,
dengan perantaraan Maria Bunda-Mu tersuci,
untuk mengabulkan permohonanku ini.
(Sebutkan permohonan anda)

Ya Yesus, Engkau berkata:
“Langit dan bumi akan musnah,
tetapi Sabda-Ku tidak akan musnah.”
Dengan perantaraan Maria Bunda-Mu tersuci,
aku percaya bahwa permohonanku akan dikabulkan
(Sebutkan permohonan anda)

Yesusku, Tuhan jiwaku, Engkau berjanji bahwa
Hati Kudus-Mu akan menjadi laut kerahiman
bagi orang-orang yang berharap pada-Mu,
aku sungguh percaya bahwa Engkau
akan mengabulkan apa yang aku minta,
walaupun itu memerlukan mukjizat.
Pada siapa aku akan mengetuk
kalau bukan pada hati-Mu.
Terberkatilah mereka yang berharap pada-Mu.
Ya Yesus, aku mempersembahkan kepada Hati-Mu
(penyakit ini, jiwa ini, permohonan ini).
Pandanglah dan buatlah apa yang hati-Mu kehendaki.

Ya Yesus, aku berharap pada-Mu dan percaya,
kepada-Mu aku mempersembahkan diriku,
di dalam Engkau aku merasa aman.
(1x Bapa Kami … Salam Maria … Kemuliaan …)

Hati Kudus Yesus, aku berharap pada-Mu
(Ulangi 10x dengan penuh semangat)

Ya Yesus yang baik, Engkau berkata:
“Jika engkau hendak menyenangkan Daku,
percayalah kepada-Ku.
Jika engkau hendak lebih menyenangkan Daku,
berharaplah pada-Ku selalu.”
Padamu Tuhan, aku berharap,
agar aku tidak binasa selamanya.
Amin.

Doa Kepada Hati Kudus Yesus

Ya Tuhan, aku berdoa, agar di rumahku ada damai, ketenangan dan kesejahteraan di dalam naungan-Mu. Berkatilah dan lindungilah usahaku, pekerjaanku, segala keingiananku dan semua yang Kauserahkan kepadaku. Usirlah nafsu dari dalam hatiku, rencana palsu dan pikiran jahat. Tuangkanlah di dalam hatiku, cinta kepada sesama dan anugerahkanlah kepadaku semangat penyerahan yang teguh, teristimewa pada saat kemalangan, agar supaya aku bangun dari kebimbangan.

Ya Tuhan, bimbinglah dan lindungilah hidupku dari bahaya-bahaya dan ketidaktentuan dunia. Jangan lupa, ya Yesusku, orang-orang yang kukasihi, baik yang masih hidiup maupun yang sudah meninggal, yang menyebabkan kesedihan kami. Tetapi kami dihibur oleh ketaatan mereka waktu mereka masih hidup, sehingga Engkau tidak menyerahkan mereka kepada maut. Kasihanilah mereka Tuhan, dan bawalah mereka kepada kemuliaan surgawi. Amin.

Devosi Kepada Hati Kudus Yesus

Hati Kudus Yesus, Hati yang penuh Cintakasih, setiap anak yang datang mengaku dengan sungguh karena penuh dosa dan kelemahan, maka Engkau akan tergerak dengan penuh belas kasihan. Ampunilah kami yang senantiasa melukai Hati KudusMu. Aku dengan semua kelemahan diriku menyerahkan hatiku seutuhnya kepadaMu. Ya Yesus jadilah Juruselamat dan Raja pribadiku seutuhnya. Aku membuka hatiku lebar lebar dengan penuh kerinduan padaMu. Masukilah hatiku ini, walaupun nista isinya dan kotor pelatarannya, bahkan hatiku yang rusak dan penuh luka ini. Aku percaya akan penyelenggaraanMu yang senantiasa mengasihi aku terlebih dahulu. Karena Engkau Tuhan, sesungguhnya Engkau memang tidak membutuhkan kasihku, tetapi apa yang selama ini ku rindukan adalah: aku mengasihiMu Allahku, dan diriku kupersembahkan seutuhnya kepadaMu. Inilah permohonanku ………. Maka ampunilah kami dan seluruh dunia ini yang senantiasa melukai Hati KudusMu, ya Tuhan.

Kemuliaan …
Bapa Kami …
Salam Maria …
Kemuliaan …

Litani Hati Kudus Yesus

Tuhan kasihanilah kami
Tuhan kasihanilah kami

Kristus kasihanilah kami
Kristus kasihanilah kami

Tuhan kasihanilah kami; Kristus dengarkanlah kami
Kristus kabulkanlah doa kami

Allah Bapa di surga, kasihanilah kami.
Allah Putra, Penebus dunia,
Allah Roh Kudus,
Allah Tritunggal Mahakudus, Tuhan Yang Maha Esa,
Hati Yesus yang Mahakudus,
Hati Yesus Putra Bapa kekal,
Hati Yesus yang di wujudkan oleh Roh Kudus dalam ribaan Bunda Perawan,
Hati Yesus yang dipersatukan dengan Sabda Allah dalam satu wujud,
Hati Yesus yang mulia tak terbatas,
Hati Yesus Bait Kudus Allah,
Hati Yesus Kemah Allah dan Pintu Surga,
Hati Yesus Perapian Cinta Kasih yang bernyala-nyala,
Hati Yesus Perbendaharaan Keadilan dan Cinta Kasih,
Hati Yesus Lubuk penuh keutamaan,
Hati Yesus amat patut dipuji-puji,
Hati Yesus Raja dan pusat segala hati,
Hati Yesus tempat semua harta kebijaksanaan dan pengetahuan,
Hati Yesus tempat tinggal keallahan seluruhnya,
Hati Yesus yang berkenan kepada Bapa,
Hati Yesus yang kaya raya dan murah hati kepada kami,
Hati Yesus kerinduan bukit-bukit yang kekal,
Hati Yesus yang murah hati bagi semua orang yang berseru kepada-Mu,
Hati Yesus sumber kehidupan dan kesucian,
Hati Yesus kurban pelunas dosa kami,
Hati Yesus yang ditimpa penghinaan,
Hati Yesus yang taat sampai mati,
Hati Yesus yang tertusuk dengan tombak,
Hati Yesus sumber segala penghiburan,
Hati Yesus kehidupan dan kebangkitan kami,
Hati Yesus pokok damai dan pemulihan kami,
Hati Yesus kurban untuk orang berdosa,
Hati Yesus keselamatan bagi orang yang berharap kepada-Mu,
Hati Yesus pengharapan orang yang meninggal dalam Engkau,
Hati Yesus kesukaan orang kudus,

Anak Domba Allah, yang menghapus dosa dunia,
sayangilah kami, ya Tuhan.

Anak Domba Allah, yang menghapus dosa dunia,
kabulkanlah doa kami, ya Tuhan.

Anak Domba Allah, yang menghapus dosa dunia,
kasihanilah kami.

Yesus yang lembut dan murah hati,
jadikanlah hati kami seperti hati-Mu.

Marilah berdoa
Allah yang Mahakuasa dan kekal, terimalah segala pujian dan penghapusan dosa yang dipersembahkan Hati Yesus kepada-Mu atas nama semua orang berdosa. Sudilah Engkau mengampuni dosa-dosa umat-Mu, yang memohon belas kasih-Mu dengan perantaraan Yesus Kristus, Tuhan kami, yang bersatu dengan Dikau dan Roh Kudus, hidup dan berkuasa, kini dan sepanjang masa. Amin

Doa Penyerahan Pribadi Kepada Hati Kudus Yesus
St. Maria Margaretha Alacoque

Aku ……… menyerahkan dan mempersembahkan diriku, hidup, karya, usaha serta penderitaanku kepada Hati Kudus Yesus. Sejak saat ini, dengan segala kekuatanku aku akan berusaha menghormati, memuji dan mencintai Hati Kudus Yesus. Dengan seluruh tenagaku aku akan berusaha menjadi milik-Nya. Aku menolak segala perbuatan yang tidak berkenan di hati-Nya. Aku memilih Hati Kudus Yesus sebagai devosi utama penghormatanku, sebagai pelindung hidup dan jaminan keselamatanku, sebagai obat untuk menyembuhkan kekurangan serta kegoyahan sikapku, untuk menyilih dosa-dosa dari seluruh hidupku dan untuk memperoleh bantuan pada saat ajalku.

Hati Kudus Yesus yang penuh kebaikan, jadilah penyilih dosa-dosaku serta perisai terhadap murka Allah Bapa atas diriku.

Hati Kudus Yesus yang penuh cinta, seluruh harapanku kupasrahkan pada-Mu, lindungilah aku terhadap si jahat, kuatkanlah kehendakku.

Hancurkanlah di dalam diriku segala sesuatu yang tidak berkenan di hati-Mu dan apa saja yang melawan Dikau. Semoga cinta ilahi-Mu meresap sedalam-dalamnya di dalam hati sanubariku agar aku tak pernah melupakan Dikau dan berpisah dari pada-Mu. Karena cinta-Mu yang tak terbatas, aku mohon dengan sangat, goreslah namaku di dalam Hati-Mu, Engkau satu-satunya kerinduan, kebahagiaan serta kebanggaanku. Aku mau hidup dan mati sebagai anak-Mu. Amin
Ditulis dalam Doa dan Devosi. Tinggalkan sebuah Komentar »
Doa Koronka
November 12, 2007 — klinikrohani

Gambar Yesus Maharahim dengan tulisan: JEZU, UFAM TOBIE (YESUS, ENGKAU ANDALANKU) diperlihatkan kepada Sr. Faustina oleh Yesus sendiri pada penampakan-Nya tanggal 22 Februari 1931. Dua sinar pada gambar tersebut melambangkan darah dan air. Sinar yang merah melambangkan darah yang memberi hidup bagi jiwa-jiwa dan sinar yang pucat melambangkan air yang menguduskan jiwa-jiwa. Dua sinar itu keluar dari kerahiman Yesus ketika hati-Nya ditusuk dengan tombak saat disalib.

Doa Koronka

Doa Koronka / Kerahiman diajarkan Tuhan Yesus sendiri pada penampakan tahun 1935 kepada Suster Faustina.
Doa ini didaraskan dengan menggunakan Rosario biasa, dan hendaknya di lakukan pada jam tiga sore.

Dalam Nama Bapa …
Bapa Kami …
Salam Maria …
Aku Percaya …

Kemudian pada manik besar (Bapa Kami) didoakan:
Bapa yang kekal kupersembahkan kepada-Mu, Tubuh dan Darah, Jiwa dan Ke-Illahi-an Putera-Mu terkasih Tuhan kami Yesus Kristus, sebagai pemulihan dosa-dosa kami dan dosa seluruh dunia.

Pada setiap sepuluhan manik kecil (Salam Maria) didoakan:
Demi sengsara Yesus yang pedih, tunjukkanlah belaskasih-Mu kepada kami dan seluruh dunia.

Diakhiri dengan tiga kali mengucapkan
Allah yang kudus, kudus dan berkuasa, kudus dan kekal, kasihanilah kami dan seluruh dunia. Amin


Novena Kerahiman Ilahi

Pada tahun 2002 Paus Yohanes Paulus II mendedikasikan Hari Minggu Paskah II sebagai Pesta Kerahiman Ilahi. Dan tiga tahun kemudian pada tanggal 2 April 2005 , Allah berkenan memberi beliau kehormatan besar untuk meninggal dunia tepat pada hari Pesta Kerahiman Illahi. Kini ia berbahagia di Surga, merayakan Paskah Abadi dan memuji Kerahiman Illahi selamanya. Doakanlah kami o Bapa Suci Yohanes Paulus II, agar kami layak menerima janji dan kerahiman Kristus Tuhan kita.

Novena ini bisa diadakan kapan saja, tetapi teristimewa adalah pada Jumat Agung sampai Minggu Paskah ke II
Bacalah bagian Kitab Suci yang ditentukan pada hari itu, dan renungkanlah … setelah itu doakanlah teks Novena dan disusul dengan Doa Koronka.

Hari Pertama

Meditasi Kitab Suci : Luk15:1-10

Marilah kita berdoa memohon Kerahiman Illahi untuk seluruh umat manusia.

Yesus yang Maharahim, Engkau mengasihi kami dan mengampuni kami. Janganlah Engkau memperhitungkan dosa kami, tetapi perhatikanlah harapan kami kepada kebaikan-Mu yang tak terbatas. Terimalah kami semua ke dalam hati-Mu yang Maharahim dan jangan singkirkan kami dari sana untuk selama-lamanya. Semuanya ini kami mohon demi cinta-Mu yang mempersatukan Dikau dengan Bapa dan Roh Kudus.

Bapa yang kekal, pandanglah dengan mata-Mu yang penuh Kerahiman pada seluruh umat manusia, khususnya pada para pendosa yang merana. Mereka memanjatkan harapan tunggal pada hati Maharahim Putera-Mu dan Tuhan kami Yesus Kristus. Demi kesengsaraan-Nya yang dahsyat, limpahkanlah kerahiman-Mu kepada kami, supaya kami semua dapat memuliakan kekuasaan-Mu untuk selama-lamanya. Amin
dilanjutkan dengan doa Koronka.

Hari Kedua

Meditasi Kitab Suci : Mat9:35-38

Marilah kita berdoa bagi para Imam yang menjadi perantara Kerahiman Allah bagi umat manusia.

Yesus yang Maharahim, sumber segala kebaikan. Lipat gandakanlah rahmat-Mu bagi para Imam dan Religius supaya dengan pantas dan dengan hasil yang baik, mereka melaksanakan tugasnya di kebun anggur-Mu. Supaya melalui sabda dan teladan yang baik, mereka dapat menarik semua orang kepada devosi yang pantas kepada Kerahiman Illahi selama-lamanya.

Bapa yang kekal, pandanglah dengan mata-Mu yang penuh kerahiman pada para pekerja dalam kebun anggur-Mu, pada jiwa para Imam dan Religius yang menjadi pelaksana cinta kasih khusus dari Putera-Mu Tuhan kami Yesus Kristus. Berilah pada mereka kekuatan berkat-Mu dan terang khusus, supaya mereka dapat memimpin banyak orang ke jalan keselamatan dan supaya mereka dapat menurunkan kerahiman-Mu. Amin
dilanjutkan dengan doa Koronka.


Hari Ketiga

Meditasi Kitab Suci : 1Ptr2:1-10

Marilah kita berdoa bagi seluruh umat Allah.

Yesus yang Maharahim, yang memberi rahmat berlimpah dari perbendaharaan Kerahiman Illahi, terimalah seluruh anggota Gereja-Mu yang dengan setia berlindung dalam kemah hati-Mu yang Maharahim. Janganlah menyingkirkan kami dari sana untuk selama-lamanya. Semuanya ini kami mohon demi cinta kasih yang menyatukan Dikau dengan Bapa dan Roh Kudus.

Bapa yang kekal, pandanglah dengan mata penuh kerahiman-Mu pada jiwa-jiwa yang setia, yang merupakan harta Putera-Mu berkat sengsara-Nya yang dahsyat itu. Berikanlah mereka berkat-Mu dan jagalah mereka selalu, supaya mereka tidak pernah mengalami kehilangan cinta dan mutiara iman yang suci. Supaya bersama semua malaikat dan para kudus, mereka dapat memuji kerahiman-Mu yang tak terbatas selama-lamanya. Amin
dilanjutkan dengan doa Koronka.


Hari Keempat

Meditasi Kitab Suci : Why21:5-8

Marilah kita berdoa bagi orang-orang yang belum mengenal Kerahiman Illahi.

Yesus yang Maharahim, sumber cahaya bagi seluruh dunia, terimalah ke dalam hati-Mu yang berbelas kasih, jiwa-jiwa orang kafir yang tidak mau percaya dan yang belum mengenal Dikau. Semoga sinar cahaya rahmat-Mu menerangi mereka, supaya mereka bersama seluruh umat-Mu dapat memuji kerahiman-Mu untuk selama-lamanya.

Bapa yang kekal, pandanglah dengan mata penuh kerahiman pada jiwa-jiwa orang kafir dan yang mereka yang tidak percaya akan Engkau satu-satunya Allah yang benar, yang belum mengenal hati berbelaskasih Putera-Mu dan Tuhan kami Yesus Kristus. Semoga mereka akan tertarik kepada cahaya terang Injil, sehingga dapat mengerti betapa bahagia mencintai Engkau dan memuji kerahiman-Mu untuk selama-lamanya. Amin
dilanjutkan dengan doa Koronka.


Hari Kelima

Meditasi Kitab Suci : Ef4:2-7 ; Ef4:11-16

Marilah kita berdoa bagi orang-orang yang mengaku dirinya Kristen namun terpisah dari Gereja yang Satu, Kudus, Katolik dan Apostolik.

Yesus yang Maharahim, Engkaulah sumber segala kebaikan dan tidak akan menolak permintaan orang yang dengan rendah hati memohon terang-Mu. Terimalah ke dalam kemah hati-Mu yang maharahim jiwa-jiwa orang yang tersesat dan iman yang sejati dan benar yang memisahkan diri dari Gereja Katolik. Rangkulah mereka dengan terang-Mu ke dalam persatuan dengan Gereja Katolik, supaya mereka bersama seluruh umat-Mu memuji kelimpahan kerahiman-Mu selama-lamanya.

Bapa yang kekal, pandanglah dengan mata penuh kerahiman keapda jiwa-jiwa orang yang tersesat dari iman dan yang menghindar dari Gereja Katolik, karena salah memanfaatkan rahmat-Mu. Mereka dengan keras kepala mempertahankan pendapat mereka yang salah. Janganlah memperhitungkan kejahatan mereka demi cinta dan penderitaan Putera-Mu yang pahit itu. Sebelum menelan sengsara, dengan sangat Yesus berdoa, “Supaya mereka semua menjadi satu.” Tolonglah supaya secepat mungkin mereka kembali ke pangkuan Gereja-Mu dan bersama seluruh umat-Mu memuliakan kerahiman-Mu untuk selama-lamanya. Amin
dilanjutkan dengan doa Koronka.


Hari Keenam

Meditasi Kitab Suci : Luk18:15-17

Marilah kita berdoa untuk anak-anak.

Yesus yang Maharahim, Engkau pernah mengatakan, “ Belajarlah dari pada-Ku, karena Aku lemah lembut dan rendah hati.” (Mat11:29). Terimalah ke dalam kemah hati-Mu yang Maharahim, jiwa anak-anak kecil dan semua mereka yang menjadi seperti anak kecil dalam hal kelemahlembutan dan kerendahan hati. Mereka membuat surga mengagumkan dan mereka menjadi seperti bunga harum di hadapan takhta Bapa di surga. Buatlah supaya mereka selalu hadir dalam hati-Mu dan selalu memuji Kerahiman Illahi-Mu.

Bapa yang kekal, pandanglah dengan mata penuh Kerahiman pada jiwa anak-anak kecil dan jiwa-jiwa orang-orang yang lemah lembut dan rendah hati karena mereka menjadi seperti Putera-Mu, yang dengan keharuman-Nya yang utama, mereka mengitari takhta-Mu. Bapa yang Maharahim, kami mohon dengan penuh harapan, supaya melalui cinta dan kegembiraan, Engkau bertakhta dalam hati mereka. Berkatilah seluruh dunia, supaya semua orang memuji kerahiman-Mu untuk selama-lamanya. Amin
dilanjutkan dengan doa Koronka.


Hari Ketujuh

Meditasi Kitab Suci : Luk7:36-50

Marilah kita berdoa bagi orang-orang yang memiliki devosi kepada Kerahiman Illahi.

Yesus yang Maharahim, hati-Mu adalah inti cinta kasih. Terimalah ke dalam kemah hati-Mu yang Maharahim jiwa-jiwa yang secara khusus menghormati dan memuji kebesaran kerahiman Illahi-Mu. Mereka membutuhkan kekuatan Allah, karena mereka berada dalam penderitaan dan kesulitan besar bersama Kristus, memikul seluruh bangsa manusia. Rangkulah mereka dengan kerahiman-Mu yang besar dan tolonglah mereka dengan rahmat kesetiaan, keberanian dan kesabaran.

Bapa yang kekal, pandanglah dengan mata penuh kerahiman pada jiwa orang-orang yang secara khusus memuji dan menghormati sifat-Mu yang paling inti, yaitu kerahiman-Mu yang tak terduga. Mulut mereka penuh dengan nyanyian puji-pujian akan kemuliaan-Mu. Tangan mereka penuh dengan perbuatan belas kasih terhadap sesama. Kami mohon supaya Engkau menunjukkan kepada mereka kerahiman-Mu sesuai harapan mereka pada-Mu dan sesuai dengan janji-Mu, bahwa Engkau akan selalu membela mereka dalam kemuliaan-Mu terutama pada jam kematian mereka. Amin
dilanjutkan dengan doa Koronka.

Hari Kedelapan

Meditasi Kitab Suci : 2Ptr3:9-14

Marilah berdoa bagi jiwa-jiwa di Api Penyucian.

Yesus yang Maharahim, Engkau bersabda, “Hendaklah kamu murah hati, sama seperti Bapa-Mu adalah murah hati.” (Luk6:36). Terimalah ke dalam kemah hati-Mu yang maharahim jiwa-jiwa di Api Penyucian. Mereka sedang melunasi hutang terhadapa keadilan Illahi-Mu (Mat5:26). Semoga aliran air dan darah yang keluar dari hati-Mu memadamkan nyala Api Penyucian, supaya disitu juga terjadi puji-pujian akan kekuatan kerahiman-Mu.

Bapa yang kekal, pandanglah dengan mata penuh kerahiman pada jiwa-jiwa di Api Penyucian. Demi sengsara Yesus Kristus yang dahsyat itu dan karena kepahitan yang memenuhi hati-Mu yang amat kudus, tunjukkanlah belas kasihan kepada mereka yang sekarang berada di bawah pandangan-Mu yang adil. Kami mohon, semoga Engkau melihat mereka hanya melalui luka-luka Putera-Mu terkasih Tuhan kami Yesus Kristus, yang kerahiman-Nya melebihi keadilan. Amin
dilanjutkan dengan doa Koronka.

Hari Kesembilan

Meditasi Kitab Suci : Why3:15-19

Marilah berdoa bagi jiwa orang-orang yang bersikap acuh tak acuh.

Yesus yang Maharahim, Engkaulah kebaikan. Antarlah ke dalam kemah hati-Mu yang Maharahim semua orang yang bersikap acuh tak acuh. Mereka bagaikan mayat yang busuk, yang memenuhi hati-Mu dengan kejijikannya. Benamkanlah mereka ke dalam api cinta kasih-Mu yang bersih dan hangatkanlah mereka dari kedinginan, supaya semangat baru mulai menyala dalam hati mereka, sehingga mereka selalu memuji kerahiman-Mu yang tak terbatas itu.

Bapa yang kekal, pandanglah dengan mata penuh kerahiman pada jiwa-jiwa orang yang acuh tak acuh. Dari hati-Nya yang maharahim, Putera-Mu pernah mengeluarkan keluhan di Bukit Zaitun, “Biarlah cawan ini berlalu daripada-Ku.” (Mat26:39). Kami mohon demi sengsara yang dahsyat dari Putera-Mu terkasih Tuhan kami Yesus Kristus dan demi sakratul maut-Nya selama 3 jam di salib, nyalakanlah semangat baru dalam hati mereka, untuk menjunjung kehormatan-Mu. Tuangkanlah ke dalam hati mereka cinta kasih yang benar, supaya mereka hidup kembali dan mampu melaksanakan perbuatan belas kasih di dunia ini dan akhirnya memuji kerahiman-Mu di surga untuk selama-lamanya. Amin
dilanjutkan dengan doa Koronka.

Litani Kerahiman Ilahi

Tuhan kasihanilah kami
Tuhan kasihanilah kami

Kristus kasihanilah kami
Kristus kasihanilah kami

Tuhan kasihanilah kami; Kristus dengarkanlah kami
Kristus kabulkanlah doa kami

Allah Bapa di surga, kasihanilah kami.
Allah Putra Penebus dunia,
Allah Roh Kudus,
Allah Tritunggal Mahakudus, Tuhan Yang Maha Esa,

Kerahiman Ilahi, sifat pencipta yang paling nyata, Engkaulah andalanku
Kerahiman Ilahi, kesempurnaan penyelamat yang tertinggi,
Kerahiman Ilahi, pengudus sumber cinta yang tak dapat di pahami,
Kerahiman Ilahi, Tritunggal Mahakudus yang tidak dapat dimengerti,
Kerahiman Ilahi, bukti kekuasaan Allah yang tertinggi,
Kerahiman Ilahi, terwujud dalam penciptaan para malaikat,
Kerahiman Ilahi, yang menciptakan kami dari yang tiada,
Kerahiman Ilahi, yang merangkul seluruh dunia,
Kerahiman Ilahi, yang memberikan kami hidup abadi,
Kerahiman Ilahi, yang menjaga kami terhadap siksa yang patut kami pikul,
Kerahiman Ilahi, yang mengangkat kami dari kebusukan dosa,
Kerahiman Ilahi, yang membela kami dengan sabda yang menjelma,
Kerahiman Ilahi, yang terpancar dari luka-luka Yesus,
Kerahiman Ilahi, yang mengalir dari hati Yesus yang Mahakudus,
Kerahiman Ilahi, yang memberikan kami Bunda Maria sebagai Bunda berbelas kasih,
Kerahiman Ilahi, yang terwujud dalam pengadaan Gereja Katolik,
Kerahiman Ilahi, yang terungkap dalam pengadaan Sakramen-sakramen Suci,
Kerahiman Ilahi, yang tampak secara khusus dalam Sakramen Permandian dan Tobat,
Kerahiman Ilahi, yang terwujud dalam Sakramen Ekaristi dan Imamat,
Kerahiman Ilahi, yang terwujud dalam panggilan kita kepada iman yang benar,
Kerahiman Ilahi, yang terwujud dalam pertobatan para pendosa,
Kerahiamn Ilahi, yang terwujud dalam kekudusan para orang jujur,
Kerahiman Ilahi, yang terwujud dalam kesucian para orang saleh,
Kerahiman Ilahi, yang membawa kesembuhan bagi orang sakit dan menderita,
Kerahiman Ilahi, yang menurunkan penghiburan bagi orang yang berada dalm kesulitan,
Kerahiman Ilahi, harapan bagi orang yang berputus-asa,
Kerahiman Ilahi, yang mendampingi kami selalu dan di mana-mana,
Kerahiman Ilahi, yang mendahului kami dengan rahmat,
Kerahiman Ilahi, ketenangan bagi orang yang menghadapi ajal,
Kerahiman Ilahi, kegembiraan surgawi bagi orang yang diselamatkan,
Kerahiman Ilahi, kesejukan dan keringanan bagi jiwa-jiwa di api penyucian,
Kerahiman Ilahi, mahkota semua orang kudus,
Kerahiman Ilahi, sumber mukjizat yang tak terbatas,

Anak domba Allah, yang membuktikan Kerahiman paling tinggi bagi keselamatan dunia dengan salib-Mu,
sayangilah kami, ya Tuhan

Anak domba allah, yang dengan penuh Kerahiman mempersembahkan diri untuk kami dalam setiap Kurban Misa,
kabulkanlah doa kami, ya Tuhan

Anak domba Allah, yang oleh Kerahiman yang tak terbatas menghapus dosa-dosa kami,
kasihanilah kami, ya Tuhan

Kerahiman Ilahi yang melampaui segala perbuatan-Nya,
sebab itu aku akan memuji Kerahiman Ilahi untuk selama-lamanya

Marilah berdoa
Allah, yang kerahiman-Mu tak dapat dipahami dan yang belas kash-Mu tak terbatas, pandanglah kami dengan mata belas kasih-Mu dan tambahkanlah kerahiman-Mu dalam kesulitan sebesar apa pun. Semoga kami selalu berharap pada kehendak-Mu yang selalu hadir dengan kerahiman-Mu. semuanya ini kami mohon dengan perantaraan Yesus Kristus, Raja Kerahiman, yang bersama Engkau dan Roh Kudus menurunkan Kerahiman kepada kami untuk selama-lamanya. Amin

Berikut adalah kumpulan doa-doa Suster Faustina, yang intinya memuliakan, memuji dan merenungkan kerahiman Tuhan.

Doa Utama Kepada Kerahiman Allah

Kata Yesus kepada Suster Faustina:
Serukanlah kerahiman-Ku bagi para pendosa. Aku merindukan keselamatan mereka. Bila dengan hati yang diresapi sikap kerendahan hati dan iman engkau ucapkan doa ini untuk seorang pendosa, akan Kuberikan kepadanya karunia pertobatan. Inilah doa itu:

Darah dan Air yang telah memancar dari Hati Yesus sebagai sumber kerahiman bagi kami, Engkaulah andalanku!

Doa Pujian Kepada Kerahiman Allah

Ya Allah, yang karena kerahiman-Mu telah berkenan memanggil segenap umat manusia dari ketiadaan kepada keberadaan.
Engkau menganugrahinya dengan kodrat dan rahmat. Namun seakan-akan kebaikan-Mu itu belum mencukupi. Karena kerahiman-Mu, Engkau mengizinkan kami masuk ke dalam kebahagiaan kekal serta mengikutsertakan kami dalam kehidupan rohani-Mu sendiri. Semuanya itu Engkau lakukan karena kerahiman-Mu semata-mata. Rahmat-Mu Engkau berikan kepada kami karena Engkau baik dan penuh kasih.
Kami sesunguhnya tidak Kau butuhkan agar diri-Mu bahagia, namun Engkau ingin berbagi kebahagiaan-Mu dengan kami. Terpujilah kerahiman-Mu, ya Tuhan. Aku akan memuliakannya sepanjang masa.

Ya Allah, Engkau tidak membinasakan manusia sesudah jatuh ke dalam dosa. Dalam kerahiman-Mu telah Engkau memaafkan kami secara ilahi, sebab bukan hanya mengampuni kesalahan kami, melainkan malah melimpahi kami dengan rahmat. Terdorong kerahiman-Mu, Engkau turun kepada kami. Maka terjadilah mukjizat kerahiman-Mu, ya Tuhan: Sabda telah menjadi Daging, Allah mulai berdiam di antara kami, Sabda Allah, kerahiman telah menjelma.
Dengan merendahkan diri telah Kau angkat kami kepada ke-Allahan-Mu. Sungguh, kerahiman-Mu berlimpah-limpah. Inilah samudra kerahiman-Mu yang tak terselami.

Langit terkagum-kagum menyaksikan kelimpahan kasih-Mu.
Sekarang tak ada seorang pun yang takut mendekati Engkau. Engkau Allah Yang Maharahim, Engkau berbelas kasih terhadap yang hina. Engkau Allah kami, kami ini umat-Mu. Engkau Bapa kami, kami ini anak-anak-Mu karena kasih-karunia.Terpujilah kerahiman-Mu, Engkau telah sudi turun kepada kami.

Dengan sepatah kata saja, ya Allah, dapat Engkau selamatkan ribuan manusia di dunia. Satu tarikan nafas Yesus saja akan memuaskan keadilan-Mu. Namun Engkau, ya Yesus, sendiri menanggung sengsara begitu mengerikan, karena kasih-Mu semata-mata. Tepat pada saat Engkau menghembuskan napas-Mu di salib, hidup kekal Engkau karuniakan kepada kami. Dengan membiarkan lambung tersuci-Mu dibuka, Engkau telah membuka sumber kerahiman-Mu yang tak pernah dapat ditimba habis. Apa yang paling berharga pada-Mu telah Engkau berikan kepada kami, yaitu darah dan air dari Hati-Mu. Inilah kemahakuasaan kerahiman-Mu. Dari pada-Mu mengalir segala rahmat. Amin

Doa Mohon Kerahiman Allah

Ya Allah, sumber kerahiman yang besar, kebaikan yang tak terhingga. Lihatlah segenap umat manusia yang hina ini berseru kepada kerahiman-Mu, kepada belaskasihan-Mu.

Allah yang berbelas kasih, jangan Engkau menolak doa-doa kami ini.

Ya Tuhan, kebaikan yang tak terpahami, yang sangat mengenal kehinaan kami, dan mengetahui bahwa kami tak mampu mengangkat diri kami kepada-Mu, curahkanlah kami dengan rahmat-Mu dalam diri kami, supaya kami bisa melaksanakan kehendak-Mu dengan setia, sepanjang hidup kami, juga pada saat kematian kami.

Semoga kuasa kerahiman-Mu melindungi kami terhadap serangan-serangan para musuh kepada keselamatan kami, supaya dengan penuh pengharapan kami menantikan kedatangan-Mu yang terakhir, yang saatnya hanya Engkau yang mengetahui.

Kami berharap, akan menerima semua yang Engkau telah janjikan kepada kami melalui Yesus, walaupun kami ini tidak layak di hadapan-Mu, sebab Yesuslah andalan kami. Semoga melalui Hati-Nya yang Maharahim kami akan bersatu dengan-Nya di surga. Amin

Doa Mohon Rahmat untuk Berbelas Kasih Terhadap Sesama

Ya Tuhan, setiap kali aku bernafas, setiap kali jantungku berdetak, setiap kali darahku mengalir dalam tubuhku, selalu aku ingin memuliakan kerahiman-Mu, Tritunggal Yang Mahakudus.

Ya Tuhan, aku ingin seluruh diriku berubah menjadi kerahiman hati-Mu, serta cermin diri-Mu sendiri. Semoga Kerahiman-Mu menenbus hati dan jiwaku dan juga sesamaku.

Ya Tuhan, bantulah aku, agar mataku selalu terang dalam berbelas kasihan, agar aku tidak pernah curiga dan tidak menilai seorang pun secara lahiriahnya, tetapi memandang jiwa sesamaku dengan indah dan selalu siap menolong mereka.

Ya Tuhan, bantulah aku agar telingaku selalu peka dalam berbelas kasihan, agar aku selalu siap melayani kebutuhan sesamaku yang sedang dalam kesulitan.

Ya Tuhan, bantulah aku agar lidahku selalu lembut dalam berbelas kasihan, agar aku tidak berbicara buruk mengenai sesamaku, tetapi selalu mengucapkan kata-kata yang menghibur dan membangkitakn kepada sesamaku.

Ya Tuhan, bantulah aku agar tanganku selalu ringan dalam berbelas kasihan dan selalu melakukan perbuatan yang baik kepada sesamaku walaupun pekerjaan itu memberatkan dan meletihkan.

Ya Tuhan, bantulah aku agar kakiku selalu cepat dalam berbelas kasihan, agar aku selalu bergerak cepat dalam memberikan pertolongan pada sesamaku, semoga aku dapat menguasai keletihan dan kelelahanku dalam melayani sesamaku.

Ya Tuhan, bantulah aku agar hatiku selalu tulus dalam berbelas kasihan, agar aku selalu peka akan tangisan sesamaku dan tidak mengeraskan hatiku dalam berbagi kasih dengan siapa pun juga. Semoga hatiku selalu tulus dalam membantu mereka. Ijinkanlah aku untuk tinggal di dalam Hati Yesus yang Maharahim dan kuatkanlah hatiku saat dalam penderitaanku sendiri. Biarlah kerahiman-Mu tinggal di dalam diriku selamanya.

Engkau sendiri yang menginginkan aku melakukan tiga macam belas kasihan, yaitu: segala macam perbuatan kasih, dan bila aku tidak dapat berbuat maka aku akan berkata-kata dengan kasih dan bila aku tidak dapat berbuat dan berkata-kata dengan kasih maka aku akan berdoa, terutama ke tempat atau orang-orang yang memang tidak dapat aku jangkau.

Ya Tuhan, ubahlah diriku menjadi seperti diri-Mu, sebab hanya Engkaulah yang dapat berbuat segala-galanya. Amin

Doa Agar Mampu Melaksanakan Kehendak Allah

Ya Yesus, yang telah terpaku di kayu salib, aku memohon, berilah aku rahmat supaya selalu setia dalam melaksanakan kehendak Bapa-Mu. Dan bila kehendak Bapa tampak terlalu berat dan sulit untuk aku laksanakan, maka aku memohon pada-Mu, ya Tuhan: semoga dari luka-luka-Mu, Engkau curahkan kekuatan dan ketegaran, dan semoga aku berseru berkata: Terjadilah menurut kehendak-Mu!

Ya Yesus, Juru Selamat dan Pengasih keselamatan manusia, dalam sengsara yang begitu dahsyat Engkau telah melupakan diri-Mu sendiri dan memikirkan keselamatan jiwa-jiwa manusia. O Yesus yang penuh belas kasih, berilah aku rahmat agar mampu melupakan diriku sendiri demi orang-orang lain. semoga aku dapat membantu Engkau dalam misi penyelamatan-Mu, sesuai dengan kehendak Bapa-Mu di surga. Amin

Doa Mohon Kebijaksanaan Ilahi

Ya Yesus, berikanlah aku akal budi yang baik supaya aku dapat lebih mengenal Engkau. Sebab semakin aku mengenal Engkau, semakin aku akan mencintai-Mu.

Ya Yesus, berilah aku akal budi yang mendalam agar aku dapat memahami perkara-perkara Ilahi, agar aku bisa memahami kodrat Ilahi-Mu, dan agar aku dapat menjiwai Sang Tritunggal.

Ya Yesus, aku mengetahui bahwa Engkau memiliki rahmat kebijaksanaan yang luar biasa, karena itu aku memohon agar Engkau memberikan rahmat-Mu itu kepada aku. Semoga Engkau berkenan dengan doa ku ini ya Tuhan. Amin
Ditulis dalam Doa dan Devosi. Tinggalkan sebuah Komentar »
Doa Rosario
November 12, 2007 — klinikrohani


Bunda Maria Ratu Rosari

DOA-DOA POKOK DI DALAM DOA ROSARIO

I. TANDA SALIB
Demi nama Bapa, Putera, dan Roh Kudus. Amin

II. AKU PERCAYA
Aku percaya akan Allah, Bapa yang Maha Kuasa, pencipta langit dan bumi. Dan akan Yesus Kristus, PuteraNya yang tunggal, Tuhan kita. Yang dikandung dari Roh Kudus, dilahirkan oleh perawan Maria. Yang menderita sengsara, dalam pemerintahan Pontius Pilatus, disalibkan, wafat dan dimakamkan. Yang turun ketempat penantian, pada hari ketiga bangkit dari antara orang mati. Yang naik ke surga, duduk disebelah kanan Allah Bapa yang Maha Kuasa. Dari situ Ia akan datang mengadili orang hidup dan mati. Aku percaya akan Roh Kudus, Gereja Katholik yang Kudus, Persekutuan para Kudus, pengampunan dosa, kebangkitan badan, kehidupan kekal. Amin

III. Kemuliaan
Kemuliaan kepada Bapa, Putera, dan Roh Kudus. Seperti pada permulaan sekarang selalu dan sepanjang segala abad. Amin
Terpujilah nama Yesus, Maria dan Santo Yosef. Untuk selama-lamanya

IV. BAPA KAMI
Bapa Kami yang ada di surga, dimuliakanlah namaMu. Datanglah KerajaanMu. Jadilah kehendakMu di atas bumi seperti di dalam surga. Berilah kami rejeki pada hari ini, dan ampunilah kesalahan kami, seperti kami pun mengampuni yang bersalah kepada kami. Dan janganlah masukkan kami ke dalam percobaan. Tetapi bebaskanlah kami dari yang jahat. Amin.

SALAM MARIA
Salam Maria penuh rahmat, Tuhan sertamu. Terpujilah engkau diantara wanita dan terpujialah buah tubuhmu Yesus. Santa Maria, Bunda Allah, doakanlah kami yang berdosa ini, sekarang dan waktu kami. Amin.


ALUR DOA ROSARIO

Tata Cara Doa Rosario :

01. Tanda Salib
02. Aku Percaya
03. Kemuliaan
04. Bapa Kami
05. Salam Putri Allah Bapa, Salam Maria
06. Salam, Bunda Allah Putra, Salam Maria
07. Salam, Mempelai Allah Roh Kudus, Salam Maria
08. Kemulian
09. Pilih salah satu rangkaian peristiwa misteri rosario, disesuaikan dengan hari

    Peristiwa Gembira : Dipakai pada hari Senin dan Kamis; pada masa Adven dan Natal.
    Peristiwa Sedih : Dipakai pada hari Selasa dan Jumat; pada masa Puasa.
    Persitiwa Mulia : Dipakai pada hari Rabu, Sabtu dan Minggu; pada masa Paska.

PERISTIWA-PERISTIWA MISTERI

I. Peristiwa Gembira:
	Peristiwa Gembira 1: Maria Menerima Kabar Gembira dari Malaikat

“Jangan takut, hai Maria, sebab engkau beroleh kasih karunia di hadapan Allah. Sesungguhnya engkau akan mengandung dan melahirkan seorang anak laki-laki dan hendaklah engkau menamai Dia Yesus.
(Lukas 1 : 30-31)

Bapa Kami yang ada di surga…..
Salam Maria, penuh rahmat,…… (10 X)
Kemuliaan Kepada Bapa, Putera, dan Roh Kudus…
Terpujilah Nama Yesus, Maria, dan Yosef…..

	Peristiwa Gembira 2: Maria Mengunjungi Elisabeth Saudaranya

“Diberkatilah Engakau diantara semua perempuan, dan diberkatilah buah rahimmu.
“Siapakah aku ini sampai ibu Tuhankudatang mengunjungi aku?”
(Lukas 1 : 42-43)

Bapa Kami yang ada di surga…..
Salam Maria, penuh rahmat,…… (10 X)
Kemuliaan Kepada Bapa, Putera, dan Roh Kudus…
Terpujilah Nama Yesus, Maria, dan Yosef…..

	Peristiwa Gembira 3: Kelahiran Kristus

Maria melahirakan seorang anak laki-laki, anaknya yang sulung, lalu dibungkusnya dengan lampin dan dibaringkanya dalam palungan, karena tidak ada tempat bagi mereka di rumah penginapan.
(Lukas 2 : 7)

Bapa Kami yang ada di surga…..
Salam Maria, penuh rahmat,…… (10 X)
Kemuliaan Kepada Bapa, Putera, dan Roh Kudus…
Terpujilah Nama Yesus, Maria, dan Yosef…..
	Peristiwa Gembira 4: Yesus dipersembahkan di Bait Allah

“Sesungguhnya. Anak ini di tentukan untuk menjatuhkanatau membangkitkan banyak orang di Israel dan untuk menjadi suatu tanda yang menimbulkan perbantahan (dan suatu pedang akan menembus jiwamu sendiri) supaya menjadi nyata pikiran hati banyak orang.”
(Lukas 2 : 34-35)

Bapa Kami yang ada di surga…..
Salam Maria, penuh rahmat,…… (10 X)
Kemuliaan Kepada Bapa, Putera, dan Roh Kudus…
Terpujilah Nama Yesus, Maria, dan Yosef…..
	Peristiwa Gembira 5: Yesus Diketemukan Kembali di Bait Allah

Yesus pulang bersama-sama mereka ke Nazaret; dan Ia tetap hidup dalam asuhan mereka. dan ibuNya menyimpan semua perkara itu di dalam hatinya. Dan Yesus makin bertambah besar dan bertambah hikmatNya dan besarNya. Dan makin dikasihi oleh Allah dan manusia.
(Lukas 2 : 52-52)

Bapa Kami yang ada di surga…..
Salam Maria, penuh rahmat,…… (10 X)
Kemuliaan Kepada Bapa, Putera, dan Roh Kudus…
Terpujilah Nama Yesus, Maria, dan Yosef…..


II. Peristiwa Sedih:
	Peristiwa Sedih 1: Yesus Berdoa Kepada BapaNya di Surga Dalam Sakrat Maut

Yesus berdoa: “Ya Abba, ya Bapa, tidak ada yang mustahil bagiMu, ambillah cawan ini dari padaKu, tetapi janganlah apa yang Aku kehendaki, melainkan apa yang Engkau kehendaki.”
(Markus 14 : 36)

Bapa Kami yang ada di surga…..
Salam Maria, penuh rahmat,…… (10 X)
Kemuliaan Kepada Bapa, Putera, dan Roh Kudus…
Terpujilah Nama Yesus, Maria, dan Yosef…..

	Peristiwa Sedih 2: Yesus Didera

Pilatus ingin memuaskan hati orang banyak itu, ia membebaskan Barabas bagi mereka. tetapi Yesus diserahkan lalu diserahkannya untuk disalibkan
(Markus 15 : 15)

Bapa Kami yang ada di surga…..
Salam Maria, penuh rahmat,…… (10 X)
Kemuliaan Kepada Bapa, Putera, dan Roh Kudus…
Terpujilah Nama Yesus, Maria, dan Yosef…..

	Peristiwa Sedih 3: Yesus Dimahkotai Duri

Mereka mengenakan jubah ungu kepadaNya, menganyam sebuah mahkota duri dan menaruhnya diatas kepalaNya. Kemudian mereka mulai memberi hormat kepadaNya, katanya:”Salam, hai raja orang Yahudi!”
(Markus 15 : 17-18)

Bapa Kami yang ada di surga…..
Salam Maria, penuh rahmat,…… (10 X)
Kemuliaan Kepada Bapa, Putera, dan Roh Kudus…
Terpujilah Nama Yesus, Maria, dan Yosef…..

	Peristiwa Sedih 4: Yesus Memanggul salibNya ke Gunung Kalvari

Sambil memikul salibNya Ia pergi keluar ketempat yang bernama tempat Tengkorak, dalam bahasa Ibrani: Golgota.
(Yohanes 19 : 17)

Bapa Kami yang ada di surga…..
Salam Maria, penuh rahmat,…… (10 X)
Kemuliaan Kepada Bapa, Putera, dan Roh Kudus…
Terpujilah Nama Yesus, Maria, dan Yosef…..

	Peristiwa Sedih 5: Yesus Wafat di Salib

Yesus berseru dengan suara nyaring: “Ya Bapa, kedalam tanganMu KuserahkannyawaKu.” Dan sesudah berkata demikian Ia menyerahkan nyawaNya.
(Lukas 23 : 46)

Bapa Kami yang ada di surga…..
Salam Maria, penuh rahmat,…… (10 X)
Kemuliaan Kepada Bapa, Putera, dan Roh Kudus…
Terpujilah Nama Yesus, Maria, dan Yosef…..

III. Peristiwa Mulia:
	Peristiwa Mulia 1: Yesus Bankit Dengan Mulia

“Jangan kamu takut; sebab Aku tahu kamu mencari Yesus yang disalibkan itu. Ia tidak ada di sini, sebab Ia telah bangkit, sama seperti telah dikatakanNya.”
(Matius 28 : 5-6)

Bapa Kami yang ada di surga…..
Salam Maria, penuh rahmat,…… (10 X)
Kemuliaan Kepada Bapa, Putera, dan Roh Kudus…
Terpujilah Nama Yesus, Maria, dan Yosef…..

	Peristiwa Mulia 2: Yesus Naik ke Surga

Sesudah Ia mengatakn demikian, terangkatlah Ia disaksikan oleh mereka, dan awan menutupNya dari pandangan mereka, dan berkata kepada mereka: ” Hai orang-orang Gelilea, mengapakah kamu berdiri melihat ke langit? Yesus ini yang terangkat ke Surga meninggalkan kamu, akan datang kembali dengan cara yang sama seperti kamu melihat Dia naik ke Surga.
(Kisah Para Raul 1 : 9,11)

Bapa Kami yang ada di surga…..
Salam Maria, penuh rahmat,…… (10 X)
Kemuliaan Kepada Bapa, Putera, dan Roh Kudus…
Terpujilah Nama Yesus, Maria, dan Yosef…..

	Peristiwa Mulia 3: Roh Kudus Turun Atas Para Rasul

Tiba-tiba turunlah dari langit suatu bunyi seperti tiupan angin keras yang memenuhi seluruh rumah di mana mereka duduk. Maka penuhlah mereka dengan Roh Kudus, lalu mereka mulai berkata-kata dalam bahasa-bahasa lain, seperti yang diberikan oleh Roh itu kepada mereka untuk mengatakannya.
((Kisah Para Raul 2 : 2,4)

Bapa Kami yang ada di surga…..
Salam Maria, penuh rahmat,…… (10 X)
Kemuliaan Kepada Bapa, Putera, dan Roh Kudus…
Terpujilah Nama Yesus, Maria, dan Yosef…..

	Peristiwa Mulia 4: Maria Diangkat ke Surga

Karena jikalau kita percaya, bahwa Yesustelah mati dan telah bangkit, maka kita percaya juga bahwa mereka yang telah meninggal dalam Yesus akan dikumpulkan Allah bersama-sama dengan Dia, sesudah itu, kita yang hidup, yang masih tinggal akan diangkat bersama-sama mereka dalam awan menyongsong Tuhan di angkasa. Demikianlah kita akan selama-lamanya bersama dengan Tuhan.
(I Tesalonika 4 : 14,17)

Bapa Kami yang ada di surga…..
Salam Maria, penuh rahmat,…… (10 X)
Kemuliaan Kepada Bapa, Putera, dan Roh Kudus…
Terpujilah Nama Yesus, Maria, dan Yosef…..

	Peristiwa Mulia 5: Maria Dimahkotai di Surga

Maka tampaklah suatu tanda besar di langit, seorang perempuan berselubungkan matahari, dengan bulan di bawah kakinya dan sebuah mahkota dari dua belas bintang di atas kepalanya.
(Wahyu 12 : 7)

Bapa Kami yang ada di surga…..
Salam Maria, penuh rahmat,…… (10 X)
Kemuliaan Kepada Bapa, Putera, dan Roh Kudus…
Terpujilah Nama Yesus, Maria, dan Yosef…..

IV. Peristiwa Cahaya:
	Peristiwa Cahaya 1: Yesus Dibaptis di Sungai Yordan

“Pada waktu itu datanglah Yesus dari Nazaret di tanah Galilea, dan Ia dibaptis di sungai Yordan oleh Yohanes. Pada saat Ia keluar dari air, Ia melihat langit terkoyak, dan Roh seperti burung merpati turun ke atas-Nya. Lalu terdengarlah suara dari sorga: “Engkaulah Anak-Ku yang Kukasihi, kepada-Mu-lah Aku berkenan.” (Markus 1 : 9-11)

Bapa Kami yang ada di surga…..
Salam Maria, penuh rahmat,…… (10 X)
Kemuliaan Kepada Bapa, Putera, dan Roh Kudus…
Terpujilah Nama Yesus, Maria, dan Yosef…..

	Peristiwa Cahaya 2: Yesus Menyatakan Diri-Nya Dalam Perjamuan Nikah di Kana

“Yesus berkata kepada pelayan-pelayan itu: “Isilah tempayan-tempayan itu penuh dengan air.” Dan merekapun mengisinya sampai penuh. Lalu kata Yesus kepada mereka: “Sekarang cedoklah dan bawalah kepada pemimpin pesta.” Lalu mereka pun membawanya. Pemimpin pesta itu mengecap air, yang telah menjadi anggur itu – dan ia tidak tahu dari mana datangnya..” (Yohanes 2 : 7-9)

Bapa Kami yang ada di surga…..
Salam Maria, penuh rahmat,…… (10 X)
Kemuliaan Kepada Bapa, Putera, dan Roh Kudus…
Terpujilah Nama Yesus, Maria, dan Yosef…..

	Peristiwa Cahaya 3: Yesus Mewartakan Kerajaan Allah Serta Menyerukan Pertobatan

“Yesus pun berkeliling di seluruh Galilea; Ia mengajar dalam rumah-rumah ibadat dan memberitakan Injil Kerajaan Allah serta melenyapkan segala penyakit dan kelemahan di antara bangsa itu.” (Mat 4:23)

Bapa Kami yang ada di surga…..
Salam Maria, penuh rahmat,…… (10 X)
Kemuliaan Kepada Bapa, Putera, dan Roh Kudus…
Terpujilah Nama Yesus, Maria, dan Yosef…..

	Peristiwa Cahaya 4: Yesus Dipermuliakan

“Yesus membawa Petrus, Yohanes dan Yakobus, lalu naik ke atas gunung untuk berdoa. Ketika Ia sedang berdoa, rupa wajah-Nya berubah dan pakaian-Nya menjadi putih berkilau-kilauan. Dan tampaklah dua orang berbicara dengan Dia, yaitu Musa dan Elia. Keduanya menampakkan diri dalam kemuliaan dan berbicara tentang tujuan kepergian-Nya yang akan digenapi-Nya di Yerusalem.” (Luk 9:28-31)

Bapa Kami yang ada di surga…..
Salam Maria, penuh rahmat,…… (10 X)
Kemuliaan Kepada Bapa, Putera, dan Roh Kudus…
Terpujilah Nama Yesus, Maria, dan Yosef…..

	Peristiwa Cahaya 5: Yesus Menetapkan Ekaristi

“Dan ketika mereka sedang makan, Yesus mengambil roti, mengucap berkat, memecah-mecahkannya lalu memberikannya kepada murid-murid-Nya dan berkata: “Ambillah, makanlah, inilah tubuh-Ku. Sesudah itu Ia mengambil cawan, mengucap syukur lalu memberikannya kepada mereka dan berkata: “Minumlah, kamu semua, dari cawan ini. Sebab inilah darah-Ku, darah perjanjian, yang ditumpahkan bagi banyak orang untuk pengampunan dosa” (Mat 26:26-28)

Bapa Kami yang ada di surga…..
Salam Maria, penuh rahmat,…… (10 X)
Kemuliaan Kepada Bapa, Putera, dan Roh Kudus…
Terpujilah Nama Yesus, Maria, dan Yosef…..

\end{document}
