\documentclass[titlepage,11pt,openany]{scrbook}
\usepackage[backref]{hyperref}
\usepackage[papersize={107.5mm,148.5mm},twoside,bindingoffset=0.5cm,hmargin={1cm,1cm},
				vmargin={2cm,2cm},footskip=1.1cm,driver=dvipdfm]{geometry}
\usepackage{palatino}
\usepackage[utf8]{inputenc}

\usepackage{pstricks}
\usepackage{graphicx}
\usepackage[bahasa]{babel} 
\usepackage{lettrine}
\usepackage{pifont}
\usepackage{enumitem}
\usepackage{wrapfig}
\usepackage{indentfirst}
\usepackage{parcolumns}
\usepackage[titles]{tocloft}
\usepackage{longtable}
\usepackage{microtype}
\usepackage{hyphenat}
%\usepackage[raggedright]{titlesec}
%\usepackage{titletoc}


\renewcommand{\cftchapfont}{%
  \fontsize{9}{8}\selectfont
}

\makeatletter
\renewcommand{\@pnumwidth}{1em} 
\renewcommand{\@tocrmarg}{1em}
\makeatother

\author{Lingkungan St. Petrus Maguwo}
\title{Doa dari Puji Syukur}
\setlength{\parindent}{1cm}
\psset{unit=1mm}

\renewcommand*{\thesubsection}{\arabic{subsection}.}

\newcommand{\ps}[2][\numexpr\value{subsection}+1\relax]{%
\setcounter{subsection}{\numexpr#1-1\relax}
\subsection{#2}
}

\begin{document}
\thispagestyle{empty}
\tableofcontents
\ps[141]{Doa Kerendahan Hati}
Puji Syukur 1992, No. 141

(Kata "kami /aku" bisa diganti menjadi saya / nama orang ... sesuai keperluan) 

Allah yang Mahatinggi, Putra-Mu Yesus telah memberikan teladan kerendahan hati yang tiada tara. Walaupun Allah, Ia telah menghampakan diri-Nya, mengambil rupa seorang hamba, dan menjadi sama dengan manusia. Dan dalam keadaan-Nya sebagai manusia, Ia telah merendahkan diri-Nya dengan taat sampai mati, bahkan sampai mati di kayu salib.

Terima kasih, ya Bapa, atas teladan Yesus ini. Berilah kami semangat Yesus sendiri, agar dengan rendah hati kami menganggap orang lain lebih utama daripada kami sendiri.

Bebaskanlah kami dari kesombongan, dan berilah kami ketabahan kalau karena nama-Mu kami direndahkan. Semoga kami tidak sakit hati kalau kami kurang di hargai atau kurang dihormati, kalau kami diabaikan atau dilupakan. Sebaliknya, semoga kami ikut bahagia kalau orang lain berhasil dan mendapat pujian serta penghargaan.

Ya Bapa, jadikanlah hati kami seperti hati Yesus yang lembut dan rendah hati. Sebab Dialah Tuhan, pengantara kami. Amin

 


\ps{Doa Kebijaksanaan}
Puji Syukur 1992, No. 142

Allah yang Mahabijaksana, Engkau telah menciptakan dan menata alam ini dengan kebijaksanaan yang tak terhingga. Engkau pun telah mengajarkan kebijaksanaan sejati kepada kami, yang seringkali tidak kami pahami, karena jalan-Mu jauh berbeda dengan jalan kami, dan pikiran kami jauh berbeda dari pikiran-Mu. Berilah kami bagian dari kebijaksanaa-Mu, supaya seperti Salomo, kami lebih mencintai kebijaksanaan daripada harta dan kuasa yang akan binasa.

Terangilah hati kami dengan Roh Kebijaksanaan-Mu, supaya kami berpengamatan tajam dan luas. Jauhkanlah kami dari segala ketakutan dan kecemasan yang tak berfaedah, dan janganlah membiarkan kami menyeleweng karena pelbagai keinginan yang tidak teratur. Semoga kami selalu waspada terhadap bujuk rayu dan godaan yang menyesatkan.

Ya Allah, anugerahkanlah kepada kami kebijaksanaan yang sejati, supaya kami belajar mencari Engkau di dalam segala sesuatu, dan memahami peristiwa-peristiwa hidup ini sesuai dengan tata kebijaksanaan-Mu. Berilah kami kebijaksanaan sejati, agar dengan pikiran yang jernih kami dapat memilih yang terbaik, dan melangkah di jalan yang lurus, mengikuti jejak Yesus, guru kebijaksanaan sejati. Dialah Tuhan, pengantara kami. Amin

 

\ps{Doa Kesabaran}
Puji Syukur 1992, No. 143

Allah yang Mahamurah, Engkau tetap sabar ketika umat-Mu Israel tidak setia. Dengan penuh kesabaran pula Engkau menuntun orang berdosa untuk bertobat, sebab Engkau tidak menginginkan pendosa menderita atau menjadi binasa. Dengan sabar dan penuh kasih Engkau mengulurkan tangan-Mu dan menunjukan jalan tobat; yang bertobat Engkau ampuni dan Kau rangkul dengan mesra.

Ya Bapa, berilah kami hati yang lapang, agar kami dapat menerima orang lain seperti apa adanya, dan dapat memahami kekurangannya, karena kami pun sering salah dan khilaf. Semoga kami tidak mudah mencela dan berprasangka, tidak pula terlalu cepat mengumpat dan mencerca, atau mengadili dan menghukum sesama. Semoga kami dapat menerima saudara yang bersalah dengan penuh cinta, mangampuni dan memaafkan kesalahannya. Semoga api kasih selalu mengarahkan sikap kami. Sebab kasih itu sabar, kasih itu murah hati. Kasih itu menutupi segala sesuatu, percaya segala sesuatu, mengharapkan segala sesuatu, dan sabar menanggung segala sesuatu.

Ya Bapa, berikanlah kami kesabaran, agar tidak mudah putus asa menghadapi kesulitan dan tantangan; jauhkanlah pula kami dari sikap gegabah dan suka mengambil jalan pintas. semoga dengan penuh kesabaran kami menantikan kerahiman-Mu. Demi Kristus, Tuhan kami. Amin

 

\ps{Doa Kehendak Yang Kuat}
Puji Syukur 1992, No. 144

Ya Allah, Engkau telah memberikan kehendak yang kuat pada Yesus, Tuhan kami. Tanpa rasa takut atau goyah Ia berpegang pada kehendak-Mu meski harus menanggung pengurbanan yang berat. Takala digoda iblis, Ia tidak goyah. Demikian pula ketika harus menderita sengsara sampai mati. Bunda Maria pun Kauberikan kepada kami sebagai panutan yang berkehendak kuat. Berilah kami kehendak yang kuat, agar pada saat goyah kami tidak berbelok arah dan menyeleweng. semoga kami tidak kecil hati menghadapi aneka kesulitan dan tantangan.

Allah, gunung batu kami, berilah kami kehendak yang kuat laksana batu karang yang tetap tegar meski tak henti diterpa gelombang. Semoga kami tetap teguh bila kami digoda untuk menyeleweng, bila kami dibujuk untuk menipu dan berlaku tidak jujur, bila kami digoda berlaku munafik, bila kami digoda untuk berbuat dosa, mencuri, berkhianat; terlebih bila kami dibujuk untuk menghianati Kasih-Mu.

Ya Allah, kekuatan kami, buatlah kami kuat seperti Yesus yang lebih suka mati dari pada menyimpang dari kehendak-Mu. Dialah Tuhan, pengantara kami, kini dan sepanjan masa. Amin

 

\ps{Doa Tanggung Jawab}
Puji Syukur 1992, No. 145

Allah sumber segala sesuatu, Engkau memberikan talenta untuk kami kembangkan. Engkau memuji para hamba yang baik dan setia, yang dengan penuh tanggung jawab memperkembangkan talenta yang mereka terima.

Buatlah kami bersikap tanggung jawab terhadap Yesus, supaya kami senantiasa ingat bahwa Ia begitu mangasihi kami, dan telah mempertaruhkan nyawa-Nya demi kami. semoga kami selalu penuh tanggung jawab terhadap panggilan kami sebagai orang beriman. Bantulah kami terus berusaha menjadi orang beriman yang dewasa dan sungguh terlibat dalam persekutuan jemaat, pewartaan, ibadat dan kesaksian serta pelayanan kepada masyarakat.

Buatlah kami bersikap tanggung jawab terhadap diri kami sendiri, supaya kami tidak menyia-nyiakan karunia yang Kau berikan kepada kami.

Buatlah kami bersikap tanggung jawab terhadap orang tua, supaya kami selalu berusaha berbuat yang terbaik guna membalas kasih sayang dan pemeliharaan yang mereka lakukan terhadap kami.

Semoga kami bersikap tanggung jawab terhadap semua orang yang mendidik kami, supaya semua pelajaran hidup yang mereka berikan dengan penuh kesabaran tidak kami sia-siakan.

Buatlah kami bersikap tanggung jawab terhadap teman-teman kami, supaya kami tidak menghianati sikap persahabatan mereka.

Buatlah kami bersikap tanggung jawab terhadap masyarakat, supaya kami selalu berusaha menyumbang lebih banyak dari pada apa yang kami terima.

Ya Bapa, bantulah kami, supaya selalu mensyukuri apa yang sudah kami terima, dan mempergunakan dengan sebaik-baiknya apa saja yang ada pada kami demi Yesus, Tuhan kami. Amin

 

\ps{Doa Ketabahan}
Puji Syukur 1992, No. 146

Allah, tumpuan hidup kami, Yesus telah Kau tampilkan sebagai teladan ketabahan di dalam pencobaan dan penderitaan. Berpola pada-Nya, Gereja pun tabah takala mengalami penganiyaan.

Pandanglah kami yang sedang dalam kesulitan dan kecemasan, yang dirundung susah dan merasa tertekan. Berilah kami ketabahan, agar kami dapat menghadapi semuanya ini dengan tabah hati, tidak mudah mengeluh, apalagi putus asa. Bukalah hati kami agar dapat melihat tuntutan dan kebijaksanaan-Mu sendiri di balik semua penderitaan ini.

Demikianlah juga kami mohon berkat-Mu bagi teman-teman yang sedang patah semangat karena beban hidup yang amat berat. Jadilah Engkau penopang, supaya mereka pun tetap tabah, seperti Yesus teladan ketabahan kami. Dialah Tuhan, pengantara kami. Amin

 

\ps[152]{Doa Ketaatan}
Puji Syukur 1992, No. 152

Allah yang Mahakuasa, Engkau telah memberi kami teladan ketaantan yang kokoh dalam diri Yesus yang telah taat pada-Mu sampai mati, bahkan sampai mati di salib; demikian juga Engkau memberi kami seorang ibu, Maria, yang mentaati panggilan-Mu dengan menjawab,"Aku ini hamba Tuhan, terjadilah padaku menurut perkataan-Mu."

Tanamkanlah semangat ketaatan Yesus dan Maria dalam hati kami, supaya kami pun taat kepada kehendak-Mu, yang Kaunyatakan lewat para pemimpin jemaat dan pemimpin masyarakat; juga lewat panggilan-Mu, dan terlebih lewat suara hati yang adalah bisikan Roh-Mu sendiri. Semoga kami selalu taat mengikuti bimbingan Roh-Mu, agar kami jangan jatuh ke dalam dosa, tetapi selamat sampai kepada-Mu meniti jalan hidup yang penuh tantangan dan cobaan. Ya bapa, berilah kami semangat ketaatan sejati. Amin




\ps[155]{Doa Orang yang Bertunangan}
%Puji Syukur 1992, No. 155

Ya Bapa di surga, kami bersyukur atas kasih yang telah bersemi dalam hati kami. Dengan bimbingan-Mu kami ingin menguji dan mematangkan hubungan cinta kami dengan lebih sungguh-sungguh. Kami menyadari kelemahan manusiawi kami. Maka kami menyerahkan masa pertunangan ini ke dalam tangan-Mu. Kami ingin mempergunakannya selaras dengan kehendak-Mu.

Ya Bapa, semoga lewat masa pertunangan ini kami semakin mengenal. Bantulah kami agar dapat saling bersikap terbuka, dan jauhkanlah kami dari sikap sengaja menyembunyikan kelemahan, agar keluarga yang akan kami bangun sungguh kokoh.

Bimbinglah kami supaya setia kepada-Mu. Tegurlah kami bila suatu saat melangkah terlalu jauh atau melanggar kehendak-Mu. Jauhkanlah dari kami setiap bahaya dan godaan. Kami berharap bahwa masa pertunangan ini dapat kami lalui dengan selamat, dan mambawa manfaat besar bagi keluarga yang akan kami bangun. Tetapi, kalau Bapa sendiri mempunyai rencana yang lain untuk kami masing-masing, semoga kamipun percaya akan kebijaksanaan dan kehendak-Mu. Demi Yesus Kristus Putra-Mu, Tuhan, pengantara kami. Amin

\ps[160]{Doa Untuk Anak}
%Puji Syukur 1992, No. 160

Kalau anak hanya satu, kata “mereka” diganti “dia”

Ya Allah yang mahakuasa, Engkau telah menciptakan anak kami menurut gambar dan citra-Mu sendiri. Terima kasih atas martabat luhur yang Kau berikan kepada mereka, dan terima kasih bahwa kami boleh menjadi alat-Mu untuk mengasuh mereka kepada kebijaksanaan-Mu. Jagalah mereka agar semakin menyerupai Yesus, yang semakin besar semakin bertambah pula hikmat-Nya, semakin berkenan pada-Mu dan pada sesama. Tuntunlah mereka agar tetap setia pada panggilannya selaku orang Kisten; bantulah mereka menekuni tugas mereka dengan penuh semangat dan tanggung jawab; lindungilah mereka dari segala marabahaya. Terangilah mereka dalam memilih jalan hidup yang selaras dengan kehendak-Mu. Semoga mereka setia kepada jalan hidup yang telah mereka pilih, dan dapat menjadikan panggilannya sebagai sarana pengabdian kepada masyarakat, kepada jemaat, dan kepada-Mu sendiri. Bila mereka mengalami kesulitan, sudilah Engkau selalu mandampingi, jangan sampai mereka lemah semangat apa lagi putus asa.

Kami mohon berkat-Mu bagi anak-anak yang terpaksa berpisah dari orangtua, lalu mengikuti orangtua asuh; semoga dalam keluarga baru ini pun mereka mendapatkan kasih yang mereka perlukan. Kami berdoa pula bagi anak-anak yang karena berbagai sebab tidak memperoleh bimbingan selayaknya. Peliharalah mereka, dan bantulah kami agar dapat turut serta mendampingi mereka menyiapkan masa depan.

Terlebih lagi kami berdoa bagi anak-anak yang terlantar dan gagal. Sudilah Engkau membangkitkan kasih dalam setiap orang untuk membantu mereka membina masa depan yang penuh harapan.

Permohonan ini kami serahkan kepada kebijaksanaan-Mu, Bapa, sebab Engkaulah Bapa sekalian anak, demi Kristus, Tuhan kami. Amin
Ditulis dalam Doa dan Devosi. 

\ps{Doa Untuk Orang Tua}
Puji syukur 1992, No. 161

Ya Allah, Bapa yang penuh kasih sayang, kami bersyukur kepada-Mu atas orangtua kami. Lewat mereka Engkau telah menciptakan kami. Melalui kasih sayang mereka, Engkau menyayangi kami. Mereka mendidik, mendampingi, dan menuntun kami. Mereka membesarkan kami dan menjadi sahabat kami.

Berkatilah mereka senantiasa. Berilah mereka kesabaran. Terangilah akal budi mereka supaya mereka selalu bertindak bijaksana. Berilah mereka kesehatan agar tetap mampu menjalankan tugas mereka sebagai pembina keluarga. Berilah rezeki secukupnya untuk kami semua; dan hindarkanlah orangtua kami dari marabahaya. Sempurnakanlah kasih mereka satu sama lain, sehingga mereka dapat menjaga kelestarian perkawinan, dan tetap setia pada janji perkawinan mereka.

Semoga mereka dapat menjalankan tugas dengan baik bagi gereja, masyarakat, dan keluarga. Buatlah keluarga kami menjadi Gereja kecil yang selalu mengasihi-Mu dan mengasihi Yesus, Putra-Mu.

Kami mohon pula berkat-Mu untuk semua orangtua, yang dengan rela dan penuh tanggung jawab telah menjalankan tugas selaku orangtua atas anak-anak mereka. Semoga pengorabnan mereka tidak sia-sia. Bila mereka menghadapi kesulitan dan tantangan, sudilah Engkau menunjukan jalan keluar yang diperlukan. Jangan biarkan mereka merana karena kegetiran hidup.

Kami berdoa pula bagi para orangtua yang sering dilupakan oleh anak-anak mereka. Sudilah Engkau menghibur dan menguatkan hati mereka. Teristimewa kami berdoa bagi para orangtua yang merasa gagal dalam membangun keluarga dan mendidik anak-anak. Semoga kepedihan ini tidak membuat mereka putus asa, tetapi semakin menyadarkan mereka untuk senantiasa bersandar pada-Mu.

Bapa, semua permohonan ini kami unjukan kepada-Mu demi Yesus Kristus Putra-Mu, yang menjadi teladan kami dalam menghormati dan mengasihi orangtua. Dialah pengantara kami untuk selama-lamanya. Amin
Ditulis dalam Doa dan Devosi. 

\ps{Doa Untuk Keluarga}
Puji Syukur 1992, No. 162

Ya Allh, Bapa sekalian insan, Engkau menciptakan manusia dan menghimpun mereka menjadi satu keluarga, yakni keluarga-Mu sendiri. Engkau pun telah memberi kami keluarga teladan, yakni keluarga kudus Nazaret, yang anggota-anggotanya sangat takwa kepada-Mu dan penuh kasih satu sama lain. Terima kasih, Bapa, atas teladan yang indah ini.

Semoga keluarga kami selalu Kau dorong untuk meneladan keluarga kudus Nazaret. Semoga keluarga kami tumbuh menjadi keluarga Kristen yang sejati yang dibangun atas dasar iman dan kasih: kasih akan Dikau dan kasih antar semua anggota keluarga.Ajarlah kami hidup menurut Injil, yaitu rukun, ramah, bijaksana, sederhana, saling menyayangi, saling menghormati, dan saling membantu dengan ikhlas hati. Hindarkanlah keluarga kami dari marabahaya dan malapetaka; sertailah kami dalam suka dan duka; tabahkanlah kami bila kami sekeluarga menghadapi masalah-masalah. Bantulah kami agar tetap bersatu padu dan sehati sejiwa; hindarkan kami dari perpecahan dan percekcokan.

Jadikanlah keluarga kami ibarat batu yang hidup untuk membangun jemaat-Mu menjadi Tubuh Kristus yang rukun dan bersatu padu.Berilah kepada keluarga kami rezeki yang cukup. Semoga kami sekeluarga selalu berusaha hidup lebih baik di tengah-tengah jemaat dan masyarakat.

Jadikanlah keluarga kami garam dan terang dalam masyarakat. Semoga keluarga kami selalu setia mengamalkan peran ini kendati harus menghadapi aneka tantangan.

Ya Bapa, kami berdoa pula untuk keluarga yang sedang dilanda kesulitan. Dampingilah mereka agar jangan patah semangat. Terlebih kami sangat perihatin untuk keluarga-keluarga yang berantakan. Jangan biarkan mereka ini hancur. Sebaliknya berilah kekuatan kepada para anggotanya untuk membangun kembali keutuhan keluarga.

Semua ini kami mohon kepada-Mu, Bapa keluarga umat manusia, dengan pengataraan Yesus Kristus, Tuhan kami.
Ditulis dalam Doa dan Devosi. 

\ps[168]{Doa Untuk Anggota Keluarga yang Sakit}
Puji Syukur 1992, No. 168

Bapa yang maha pengasih, kami sekeluarga sangat perihatin, karena anggota keluarga kami …….. sedang sakit.

Dalam keperihatinan ini kami ingat akan Yesus Kristus, yang Kauberi kuasa menyembuhkan orang-orang sakit. Percaya akan kuasa-Mu, kami serahkan saudara kami yang sakit ini kepada kebijaksanaan-Mu. Dengan penuh iman dan harapan kami mohon: Kuatkanlah dia dalam deritanya, dampingilah dan hiburlah dia dalam kesunyian dan kesepiannya, dan teguhkanlah dia dalam iman dan harapan. Sudilah Engkau menyembuhkan dia dari penyakit yang dideritanya.

Semoga dalam menanggung sakit ini ia ingat akan Yesus yang menderita sangat hebat demi keselamatan semua orang.Bantulah ia menyatukan sakitnya dengan penderitaan Yesus sendiri, supaya akhirnya ia pun boleh bersatu dengan Yesus yang bangkit dan mulia. Terangilah dia agar mampu memetik hikmat dari pengalaman sakitnya ini. Semoga ia semakin memahami makna kehidupan, bahkan dapat melihat sakitnya sebagai karunia yang mendatangkan aneka karunia.

Kami berdoa juga bagi mereka yang sakitnya tak tersembuhkan. Semoga dengan hati terbuka mereka menerima kebijaksanaa-Mu.

Bagi kami sendiri, semoga peristiwa ini semakin menyadarkan kami akan tanggung jawab kami terhadap mereka yang sakit. Semoga karena berkat-Mu kami selalu berusaha melayani mereka dengan senang hati. Sebab kami sadar bahwa apa pun yang kami perbuat bagi mereka, itu kami perbuat bagi Yesus Kristus sendiri, Tuhan kami, yang hidup dan berkuasa, kini dan sepanjang masa. Amin


\end{document}