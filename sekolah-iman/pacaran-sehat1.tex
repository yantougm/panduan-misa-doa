\documentclass[11pt]{scrartcl}
\usepackage{natbib}
\title{Gaya Pacaran Sehat ditinjau dari Sisi Iman dan Moral}
\author{Kelompok 5 Sekolah Iman}
\date{~}
\begin{document}
\maketitle
\vspace{-2cm}
\section*{Latar belakang}
Hasil penelitian dari BPPM DIY tentang kesehatan reproduksi tahun 2011 mengungkap fakta bahwa 62,7\% remaja SMP dan SMA tidak perawan/perjaka. Data lain menyatakan bahwa remaja yang melakukan aborsi sebanyak 32\% \cite{bppmdiy2011}.   

%\section*{Pacaran}
\subsection*{Pengertian pacaran}
Siapa di antara temen-temen yang udah punya pacar? Atau lagi nyari pacar? Atau malah belum pengen punya pacar? Waktu masih anak-anak, kalau ada orang yang kita suka dengan polosnya kita bilang kalau dia itu pacar saya. Waktu sudah lebih gede dikit, kita marah/malu kalau dibilang pacarnya si A ato si B. Masuk ke awal masa remaja, biasanya di sini kita mulai bener-bener merasa tertarik sama lawan jenis. Istilahnya kita merasakan apa yang disebut “cinta monyet”, di mana kita suka karena kita suka aja. Entah karena dia ranking 1, ato jagoan basket sekolah, ato bahkan karena dia orang pertama yang “nembak” kita. Seiring bertambahnya usia, perasaan cemburu ataupun posesif terhadap pacar mulai kita kenal, tapi itu memang merupakan proses pembelajaran buat kita. Waktu SMA, punya pacar malah bisa buat menaikkan status di mata teman-teman kita. Jujur saja, pasti merupakan suatu kebanggaan tersendiri kalau ada yang bisa dibanggakan dari pacar kita. Tapi sebenarnya tujuan pacaran itu apa sih? Kayak barang yang bisa dipamerkan ke teman-teman? Atau (terutama buat cowok-cowok nih) sekedar memenuhi kebutuhan fisik kita?

Menurut Gereja sendiri, pacaran itu merupakan proses penting buat kita yang ingin menerima sakramen perkawinan. Kenapa pacaran dianggap penting? Karena pacaran merupakan proses untuk saling mengenal dan memberikan gambaran sejelas mungkin keadaan pasangan kita sekaligus memastikan kecocokan kita berdua. Makanya, pacaran yang baik itu sebaiknya diisi dengan upaya-upaya untuk saling menyesuaikan diri terutama yang bersangkutan dengan nilai hidup, pola berpikir, ataupun gaya hidup kita dan pasangan kita. Bisa enggak kita menerima pasangan kita apa adanya?\cite{ryant2008}

Sebagai orang katolik, apa aja sih yang perlu diperhatikan dalam mencari pacar? Kayaknya point terpenting udah jelas deh : satu iman. Lho, memangnya kenapa kalau beda iman? Bukannya Gereja membolehkan kita menikah dengan orang yang beda agama? Karena hal ini menyangkut sesuatu yang paling mendasar, yaitu dasar dan pandangan hidup kita. Kalau dari situ saja beda nantinya dalam proses komunikasi dan penerimaan satu sama lain akan mengalami kesulitan. Karena itu, tempat paling gampang buat nyari pacar ya di gereja. Kan asik tuh bisa berangkat ke gereja dan misa bareng sama pasangan kita, atau merayakan Natal – Paskah dan hari raya lainnya berdua. Nah, berangkat dari situ kita bisa mempertimbangkan kesesuaian antara kita dengan pasangan kita. Bukannya harus sama persis segala sesuatunya lho, tapi apakah kita dan pasangan kita bisa saling melengkapi? Semakin banyak kesesuaian yang ada, semakin mantap hubungan yang kita bangun. Ingat, buat kita sebagai orang katholik, pernikahan hanya sekali (kecuali dipisahkan oleh maut).

\subsubsection*{Kata Alkitab mengenai pacaran}
Meskipun kata “pacaran” tidak ditemukan dalam Alkitab, kita diberikan beberapa prinsip yang kita perlu taati sebelum memasuki jenjang pernikahan. Hal pertama yang harus disadari adalah bahwa kita perlu memisahkan diri dari pandangan dunia mengenai pacaran karena cara Tuhan berbeda dengan cara dunia (2 Petrus 2:20). (Oleh dunia) kita disuruh untuk berpacaran sebanyak yang kita inginkan, dengan sebanyak-banyaknya pacar yang kita bisa dapat. Sebaliknya kita perlu menemukan orang seperti apa yang kita inginkan sebelum betul-betul masuk dalam hubungan pacaran. Kita perlu mengetahui apakah orang yang bersangkutan sudah lahir baru kembali (Yohanes 3:3-8), dan apakah mereka memiliki keinginan yang sama untuk menjadi serupa dengan Kristus (Filipi 2:5). Mengapa hal ini penting dalam mencari pasangan hidup? Orang Kristen perlu berhati-hati jangan sampai menikahi orang yang belum percaya (2 Korintus 6:14-15) karena hal ini dapat melemahkan hubungan Anda dengan Kristus dan menurunkan standar dan moral Anda.

Ketika seseorang masuk dalam hubungan yang serius dengan orang lain, penting untuk mengingat untuk mengasihi Tuhan lebih dari segalanya (Matius 10:37). Mengatakan atau menganggap bahwa orang lain itu adalah “segalanya” bagi Anda atau yang paling penting dalam hidup Anda adalah penyembahan berhala, dan merupakan dosa (Galatia 5:20; Kolose 3:5). Juga jangan menajiskan tubuh Anda dengan melakukan hubungan seks pra-nikah (1 Korintus 6:9, 13, 2 Timotius 2:22). Percabulan bukan saja dosa melawan Tuhan namun juga terhadap diri sendiri (1 Korintus 6:18). Adalah penting untuk mengasihi dan menghormati orang lain sebagaimana Anda mengasihi diri sendiri (Roma 12:9-10) dan hal ini berlaku pula untuk pacaran dan pernikahan. Mengikuti prinsip-prinsip ini adalah cara terbaik untuk memiliki dasar yang kokoh dalam pernikahan. Ini adalah salah satu keputusan yang paling penting yang Anda ambil karena ketika dua orang menikah, mereka menjadi satu daging dan tidak dapat dipisahkan lagi (Kejadian 2:24; Matius 19:5)

\subsection*{Pacaran yang sehat secara iman dan moral}
Masa pacaran adalah masa pengenalan kepribadian secara utuh dari pasangannya. Masa pacaran bukan saat untuk mencoba-coba hingga berhubungan seksual dengan pasangannya sebagaimana suami-istri \cite{kriswanta2009}.

Pacaran ingin berbagi kasih menerima satu sama lain. Dengan tujuan untuk membina rumah tangga yang bahagia. 

Nah, kalau kita mau pacaran, pacaran kita juga harus sehat. Pacaran yang sehat itu gimana? Yang pertama mesti sehat secara \textbf{psikologis}. Pacaran itu merupakan salah satu perwujudan dari mengasihi sesama. Dalam 1Korintus 13:4-6, ada definisi tentang kasih. Dari definisi di sana bisa dikatakan bahwa pacaran menjadi tidak sehat kalau mulai main paksa-paksaan, cemburu berlebihan, terlalu posesif, berantem terus, pokoknya malah bikin stress, ketakutan, tertekan, selalu terpaksa, dll. 
Bahkan kalau sampai ke jenjang pernikahan, kalau dilangsungkan secara terpaksa maupun karena takut, baik dari salah satu pihak maupun keduanya menyebabkan perkawinan itu tidak sah. Hal ini sesuai dengan Kanon 1103 dari KHK \cite{khk2011}.

Pacaran juga harus sehat secara \textbf{fisik}. Hak dan kewajiban orang pacaran itu beda dengan yang sudah menikah. Kita mesti bisa jaga diri untuk tidak melangkah terlalu jauh karena terpengaruh emosi sesaat. 

Dengan orang tua memberikan lampu hijau buat pacaran, berarti mereka percaya sama kita. Coba deh kita hargai kepercayaan mereka dengan tidak menyalahgunakannya.

Terakhir, pacaran juga harus sehat secara \textbf{sosial}. Kita hidup di masyarakat yang memiliki norma dan adat istiadat. Pulang ngapel larut malem melebihi jam malam di kawasan kita, juga suka bikin sebel masyarakat, biasanya ortu bakal keberatan dan jangan-jangan gara-gara dianggap nggak sopan kita malah dilarang pacaran sama mereka. Kalau gaya pacaran kita udah bikin masalah di lingkungan, berarti pacaran kita udah nggak sehat.

Kalau ternyata antara kita dan pasangan kita tidak menemui kecocokan dan tidak bisa dicocokkan, artinya mungkin kita perlu mencari pasangan lain. Lho koq segampang itu memutuskan cari pasangan lain? Pacaran yang ideal yaitu apabila kita bisa menyesuaikan diri dan saling menerima perbedaan kita. Tapi seperti yang temen-temen ketahui dari praktikum Fisika 1, kondisi ideal itu enggak selalu tercapai. Seandainya pun ternyata kita enggak menemui kecocokan dan hubungan gak bisa dilanjutin, percayalah selalu bahwa Tuhan selalu ngasih yang terbaik buat kita. Mungkin aja pasangan kita sekarang bukan orang yang cocok buat kita. Ingatlah bahwa rencana Tuhan itu indah pada saatnya. Patah hati itu kejadian yang umum dan sekali lagi itu juga merupakan proses pembelajaran buat kita. Jadi ga perlu takut pacaran karena takut patah hati.

Proses pacaran sendiri bisa berbeda-beda tiap orang. Ada yang sekali pacaran langsung menemui kecocokan dan menikah, tapi ada juga yang perlu mencoba beberapa kali baru menemukan pasangan hidupnya. Ibaratnya kalau kita dapat nilai E untuk mata kuliah tertentu (mudah-mudahan jangan sampai kejadian beneran deh), apa itu bikin kita langsung memutuskan berhenti kuliah? Hidup itu emang perjuangan, tapi kita beruntung karena Tuhan selalu menyertai kita.


\subsection*{Dampak pacaran tidak sehat}

Masa pacaran seharusnya untuk saling mengenal pribadi masing-masing. Akibat tidak dikenalnya pasangan secara baik bisa muncul:
tekanan batin, kecewa, putus/batal.

Saat pacaran perlu dibincang tentang cita-cita hidup bersama. Kalau tidak maka dapat muncul kesalahpahaman, ketidakjujuran. Masalah perbedaan agama juga dapat menimbulkan kecurigaan. Konflik juga bisa muncul karena perbedaan ekonomi. Suami dan istri mempunyai keingingan sendiri-sendiri yang bisa muncul tidak hanya dari pribadi tetapi dari lingkungan sekitar dan tempat kerja. Jika tidak dikomunikasikan dengan baik dapat menimbulkan konflik. \cite{iman1996}


Perasaan bebas karena belum terikat, tidak ada tekanan, tidak ada kekangan akan membuat pribadi berkembang dengan baik. Kalau perasaan ini
tidak terpelihara dengan baik maka
pribadi bisa terhambat perkembangannya, 
kurang pergaulan, kepercayaan diri juga terhambat. 
Bisa juga hal ini menjadi biang ketidakpercayaan satu sama lain.

Tidak boleh melakukan hubungan seks. Sesuai dengan sepuluh perintah Allah. Jika pasangan yang sedang pacaran sampai melakukan hubungan seksual akibatnya bisa
%\cite{sepuluhperintahAllah}
merasa bersalah terhadap keluarga dan Tuhan,
merasa tidak berharga,
merasa berdosa,
pemurung, pemarah, mudah tersinggung,
kawatir dari sisi perempuan,
hamil,
penyakit menular seksual, abortus,
\textbf{terpaksa} menikah.

\subsection*{Peran pemerintah dan orangtua}
Kalau mengcau ke negara-negara maju, kita bisa melihat bagaimana pemerintah di sana serius menangani masalah remaja khususnya dalam pendidikan tentang seks. Kampanye besar-besaran yang digelar setiap tahun, dikemas dengan pendekatan dunia remaja.
%\cite{Jejak-jejakkebijaksanaan}

Jangan pernah salah memilih pasangan dan jang napernah menunda menghentikan segala pola yang menuju kepada pacaran yang tidak sehat, mungkin mengarah kepada kekerasan saat pacaran. Keluarga adalah pondasi untuk memperkuat posisi kita dalam menghadapi gaya pacaran tidak sehat. Teman-teman terdekat yang mendukung pacaran sehat adalah tempat kita meminta tolong. Jangan percaya terhadap cinta yang menuntut pengorbanan dan digantikan dengan kekerasan. Rasa saling menghormati dan menyayangi adalah landasan utama. Lalu? semoga kita akan hidup lebih baik ari hari ini. Semoga kita bisa menemukan rasa cinta sejati \cite{sony2009}.


\subsection*{Tips Pacaran Bagi Orang Katolik}
\begin{enumerate}
\item Belajarlah untuk mengutamakan Tuhan dalam hidup Anda! Persiapkan diri Anda untuk sebuah pernikahan Katolik! Bacalah Alkitab Anda, berjemaatlah di gereja dimana Anda bertumbuh. Pelajarilah hikmat Tuhan untuk pernikahan, suami-suami dan istri-istri. Alkitab telah memberikan kita satu perintah yang sangat penting untuk bidang ini, yaitu “menjadi pasangan yang seimbang (2 Korintus 6 :14). Pelajarilah ayat ini dan cobalah untuk dapat mengerti arti sebenarnya!

\item  Kenali diri Anda! Ambillah waktu untuk membuat perubahan apapun yang Anda butuhkan untuk dapat menjadi pasangan yang baik bagi seseorang. Anda TIDAK dapat menjadi bahagia dalam pernikahan MANAPUN tanpa bahagia terlebih dahulu dengan diri Anda sendiri!

\item  Mengetahui apa yang Anda butuhkan! Anda harus mengetahui kebutuhan-kebutuhan Anda, dengan demikian Anda dapat mengkomunikasikannya dengan pasangan Anda di masa depan. Ini adalah hal
yang tidak dapat Anda kompromikan! Tanyakan juga kepada pasangan Anda apa yang dia butuhkan. Kemudian carilah tahu apakah Anda berdua dapat saling memenuhi kebutuhan satu sama lain. Hal ini sangatlah penting!

\item  Belajarlah untuk peka terhadap tanda-tanda peringatan yang Anda rasakan ketika Anda sedang pacaran dengan seseorang! Menyadari bahwa seseorang yang sedang menjalin hubungan dengan Anda bukanlah “seseorang yang special” adalah separuh dari perjuangan Anda. Anda bisa saja berusaha agar hubungan itu dapat berjalan dengan baik seumur hidup Anda, yang kemudian pada akhirnya, tidak akan pernah berhasil! Semua orang mempunyai kualitas yang baik dan buruk. Hanya karena Anda tidak cocok
dengan seseorang, bukan berarti bahwa orang itu tidak akan menjadi pasangan yang baik bagi orang lain! Apabila memang tidak “cocok”, hormati diri Anda dan pasangan Anda dengan mengakhiri hubungan Anda. Anda berdua layak untuk memiliki hidup yang berbahagia.

\item  Jangan hidup dalam ketakutan dengan kemungkinan bahwa Anda akan tetap sendiri seumur hidup Anda. Ketakutan akan menumbuhkan kegilaan ketika Anda sedang menjalin hubungan! Kebutuhan Anda menjadi tidak berarti sama sekali bagi diri Anda! Anda bahkan dapat membuat keputusan-keputusan bodoh ketika ketakutan ini mengambil alih diri Anda. Isilah kehidupan Anda dengan hal-hal yang dapat membuat Anda merasa gembira. Serahkan semuanya kepada Tuhan dan TINGGALKAN itu di sana!

\item Minum-minum yang berlebihan (alkoholik, pesta minuman keras di akhir pekan, dll), orang yang bertindak dengan kekerasan dan sejenisnya, adalah orang-orang yang “TIDAK MAMPU” untuk sebuah hubungan dengan komitmen. Orang-orang ini membutuhkan pertolongan dan “penyakit-penyakit” mereka membuat mereka untuk saat itu, tidak mampu membangun suatu hubungan yang
sehat. 
\item Carilah seorang konselor Kristen dengan reputasi yang baik, bila memungkinkan, untuk membantu Anda dalam membuat keputusan yang benar. Pernikahan adalah KOMITMEN untuk SEUMUR HIDUP. Anda bertanggung jawab terhadap diri Anda sendiri untuk membuat keputusan terbaik yang bisa
Anda buat. Menemukan pasangan yang tepat dan membuat komitmen untuk seumur hidup dengan orang tersebut adalah sebuah anugerah yang luar biasa dari Tuhan
\end{enumerate}

\section*{Penutup}
Setiap persoalan mesti adala cara untuk menyelesaikannya. Begitu pula dengan masalah pacaran ini. Kami mengajak siapa saja untuk bersama-sama mendukung pacaran yang sehat. Negara-negara besar seperti AS dan Uni Eropa telah melakukan perhatian yang terintegrasi bagi perkembangan masa depan anak mudanya. Mereka berupaya menghentikan akibat negatif dari pacaran tidak sehat. 

Mari kita bersama melakukan hal yang serupa di Indonesia, khususnya di kalangan mudika. Mri kita satukan tekad demi masa depan kita, gereja, dan Indonesia yang lebih baik.

\renewcommand{\bibname}{Referensi}
\renewcommand{\refname}{Referensi}
\bibliographystyle{plain} 
\bibliography{pacaran-sehat}




\end{document}