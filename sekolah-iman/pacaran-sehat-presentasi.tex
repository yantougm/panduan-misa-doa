\documentclass[12pt,a4paper]{beamer}
\usepackage[latin1]{inputenc}
\usepackage[bahasa]{babel}
\usepackage{amsmath}
\usepackage{amsfonts}
\usepackage{amssymb}
\title{Gaya Pacaran Sehat ditinjau dari Sisi Iman dan Moral}
\author{Kelompok 5 Sekolah Iman}
\date{29 Januari 2013}

\usetheme{Singapore}
\usecolortheme{beaver}
\usefonttheme{professionalfonts}

\begin{document}
\maketitle
\begin{frame}
\frametitle{Latar belakang}
\begin{itemize}[<+->]
\item 62,7\% remaja SMP dan SMA tidak perawan/perjaka
\item 32\% remaja yang melakukan aborsi
\item pernikahan dini
\begin{itemize}[<+->]
\item 2010 : 362 pemohon
\item 2011 : 522 pemohon
\item usia 11 -- 14
\item pendidikan SD -- SMP
\end{itemize}

\end{itemize}
\end{frame}

\begin{frame}
\frametitle{Latar belakang (lanjutan)}
\begin{itemize}[<+->]
\item Data sesungguhnya bisa lebih besar
\item Seks bebas, karena kurang memahami dan mengamalkan agama
\item Peran pendidikan dan agama 
\end{itemize}
\end{frame}


\begin{frame}
\frametitle{Pacaran}
\begin{itemize}[<+->]
\item Cinta monyet?
\item Definisi pacaran
\begin{itemize}[<+->]
\item saling mengenal
\item gambaran tentang pasangan
\item memastikan kecocokan
\item persiapan menuju perkawinan
\end{itemize}
\item Pacaran yang baik saling menyesuaikan diri
\begin{itemize}[<+->]
\item nilai hidup
\item pola berpikir
\item gaya hidup
\end{itemize}
\end{itemize}
\end{frame}

\begin{frame}
\frametitle{Mencari pacar}
\begin{itemize}[<+->]
\item satu iman
\item kesesuaian
\item saling melengkapi
\end{itemize}
\end{frame}

\begin{frame}
\frametitle{Kata Alkitab tentang pacaran}
\begin{itemize}[<+->]
\item kata ``pacaran'' tidak ada
\item sudah lahir kembali
\item ingin serupa dengan Kristus
\item jangan menikahi orang yang belum percaya
\end{itemize}
\end{frame}

\begin{frame}
\frametitle{Pacaran sehat secara iman dan moral}
``Masa pacaran adalah masa pengenalan kepribadian secara utuh dari pasangannya. Masa pacaran bukan saat untuk mencoba-coba hingga berhubungan seksual dengan pasangannya sebagaimana suami-istri''

``Pacaran ingin berbagi kasih menerima satu sama lain. Dengan tujuan untuk membina rumah tangga yang bahagia.''

\end{frame}

\begin{frame}
\frametitle{Pacaran yang sehat secara}
\begin{itemize}[<+->]
\item fisik
\item seksualitas
\item sosial
\end{itemize}
\end{frame}

\begin{frame}
\frametitle{Sehat secara fisik}
\begin{itemize}[<+->]
\item pacaran beda dengan nikah
\item jaga diri
\item hargai kepercayaan orang tua
\end{itemize}
\end{frame}

\begin{frame}
\frametitle{Sehat secara seksualitas}
\begin{itemize}[<+->]
\item pasangan bukan segalanya
\item jangan berbuat cabul
\item hubungan seks diluar nikah $\rightarrow$ najis
\item cabul dosa terhadap Tuhan dan diri sendiri
\item penting mengasihi dan menghormati orang lain
\end{itemize}
\end{frame}

\begin{frame}
\frametitle{Sehat secara sosial}
\begin{itemize}[<+->]
\item lihat waktu
\item tidak cocok $\rightarrow$ siap putus
\end{itemize}
\end{frame}

\begin{frame}
\frametitle{Dampak pacaran tidak sehat}
\begin{itemize}[<+->]
\item Akibat  cita-cita hidup bersama tidak dibicarakan
\begin{itemize}[<+->]
\item kesalahpahaman
\item ketidakjujuran
\item kecurigaan
\item konflik
\end{itemize}
\item Akibat tidak bebas, ada tekanan
\begin{itemize}[<+->]
\item perkembangan pribadi terhambat
\item kurang pergaulan
\item kurang PD
\item ketidakpercayaan
\end{itemize}
\item Akibat hubungan seks di luar nikah
\begin{itemize}[<+->]
\item merasa bersalah terhadap keluarga dan Tuhan 
\item merasa tidak berharga
\item berdosa, pemarah, pemurung, mudah tersinggung
\item hamil, abortus
\item PMS
\end{itemize}

\end{itemize}
\end{frame}

\begin{frame}
\frametitle{Peran pemerintah}
\begin{itemize}[<+->]
\item negara maju serius tentang pendidikan seks
\item kampanye besar-besaran dengan pendekatan remaja
\end{itemize}

\end{frame}

\begin{frame}
\frametitle{Peran orangtua}
\begin{itemize}[<+->]
\item jangan salah pilih
\item jangan tunda untuk stop bila muncul indikasi kekerasan
\item keluarga adalah pondasi
\item jangan percaya begitu saja, cinta menuntut pengorbanan
\item saling menghormati dan menyayangi
\item cinta sejati, dukungan keluarga sehat
\end{itemize}
\end{frame}

\begin{frame}
\frametitle{Tips pacaran orang Katolik}
\begin{itemize}[<+->]
\item \textbf{Utamakan Tuhan dalam hidup Anda!}
\item  \textbf{Kenali diri Anda!} 
\item  \textbf{Mengetahui apa yang Anda butuhkan!} 
\item  \textbf{Siapkan hati untuk berpisah!}
\item  \textbf{Jangan takut tidak laku!}
\item \textbf{Jangan mabuk!}
\item \textbf{Jangan ragu hubungi konselor!}
\end{itemize}

\end{frame}

\begin{frame}
\frametitle{Penutup}
\begin{itemize}[<+->]
\item Berpacaran $\rightarrow$ berkeluarga bahagia sejahtera
\item Penjajagan pribadi, menerima +/-  
\item Pacaran sehat: saling pengertian, keterbukaan, komunikasi
\item Orangtua memberi bimbingan dan doa restu
\item Program ke depan, bagaimana hidup bersama akan dijalani
\item {\Large Terima Kasih}
\end{itemize}
\end{frame}



\end{document}