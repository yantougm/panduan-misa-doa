\newpage
\small
\section*{\center Keranjang Arang dan Alkitab}
Seorang Kakek hidup di suatu perkebunan di suatu pegunungan sebelah timur Negara bagian Kentucky (Amerika) dengan cucu lelakinya yg masih muda. Setiap pagi Kakek bangun lebih awal dan membaca Alkitab di meja makan. Cucu lelaki nya ingin sekali menjadi seperti kakeknya.	

Suatu hari sang cucu nya bertanya,"Kakek! Aku mencoba untuk membaca Alkitab seperti yang kakek lakukan tetapi aku tidak memahaminya, dan apa yang aku pahami aku lupakan secepat aku menutup buku. Apa sih kebaikan dari membaca Alkitab?"Dengan tenang sang Kakek dengan mengambil keranjang tempat arang, memutar sambil melobangi keranjang nya ia menjawab,"Bawa keranjang ini ke sungai dan bawa kemari lagi penuhi dengan air."

Maka sang cucu melakukan seperti yang diperintahkan kakek, tetapi semua air habis menetes sebelum tiba di depan rumahnya. Kakek tertawa dan berkata,"Lain kali kamu harus melakukukannya lebih cepat lagi, ayo coba lagi!"

Sang cucu berlari lebih cepat, tetapi tetap, lagi2 keranjangnya kosong sebelum ia tiba di depan rumah. Dengan terengah-engah, ia berkata kepada kakek nya bahwa mustahil membawa air dari sungai dengan keranjang yang sudah dibolongi, maka sang cucu mengambil ember sebagai gantinya.

Sang kakek berkata,"Aku tidak mau ember itu; aku hanya mau keranjang arang itu. Ayolah, usaha kamu kurang cukup,"maka sang kakek pergi ke luar pintu untuk mengamati usaha cucu laki-lakinya itu.

Cucu nya yakin sekali bahwa hal itu mustahil, tetapi ia tetap ingin menunjukkan kepada kakeknya, biar sekalipun ia berlari secepat-cepatnya, air tetap akan bocor keluar sebelum ia sampai ke rumah. Sekali lagi sang cucu mengambil air ke dalam sungai dan berlari sekuat tenaga menghampiri kakek, tetapi ketika ia sampai didepan kakek keranjang sudah kosong lagi.

Sambil terengah-engah ia berkata,"Lihat Kek, percuma!"

"Jadi kamu pikir percuma?"Jawab kakek. Kakek berkata,"Lihatlah keranjangnya."

Sang cucu menurut, melihat ke dalam keranjangnya dan untuk pertama kalinya menyadari bahwa keranjang itu sekarang berbeda. Keranjang itu TELAH BERUBAH dari keranjang arang yang tua kotor dan kini BERSIH LUAR DAN DALAM.

"Cucuku, hal itulah yang terjadi ketika kamu MEMBACA ALKITAB. Kamu TIDAK BISA MEMAHAMI atau INGAT segalanya, tetapi KETIKA kamu MEMBACANYA LAGI, kamu AKAN BERUBAH, luar dalam. Itu adalah KARUNIA dari ALLAH di dalam hidup kita."

{\noindent \emph{Sumber: salib.net}}
\normalsize