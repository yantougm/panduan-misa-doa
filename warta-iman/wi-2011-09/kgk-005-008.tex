\newpage
\section*{Kompendium Katekese Gereja Katolik}
\setcounter{kgkcounter}{4}
{\small
\kgk{Bagaimana kita dapat bicara tentang Allah?}
Sebagai titik tolak, kita berbicara tentang kesempurnaan manusia dan ciptaan
lainnya, yang -- meskipun terbatas -- merupakan cerminan kesempurnaan Allah
yang tak berkesudahan. Namun, kita perlu terus-menerus memurnikan bahasa kita
sejauh itu mungkin walaupun harus kita sadari bahwa kita tidak akan pernah dapat
mengungkapkan misteri Allah yang tak terbatas.

\kgk{Apa yang diwahyukan Allah kepada manusia?}
Dalam kebaikan dan kebijaksanaan-Nya, Allah mewahyukan Diri. Melalui
sabda dan karya, Allah mewahyukan Diri dan rencana-Nya yang berasal dari cinta kasih yang dalam Kristus telah dinyatakan sejak kekal. Menurut rencana ini,
semua umat manusia, melalui rahmat Roh Kudus, mengambil bagian dalam 
kehidupan ilahi sebagai "anak-anak angkat" dalam Putra Tunggal Allah.

\kgk{Apa saja tahap-tahap awal pewahyuan Allah?}
Sejak awal mula, Allah mengungkapkan Diri-Nya kepada leluhur kita yang
pertama, Adam dan Hawa, dan mengundang mereka untuk masuk ke dalam
persatuan yang intim dengan-Nya. Sesudah kejatuhan mereka ke dalam dosa, Allah
tidak menghentikan pewahyuan-Nya kepada mereka, tetapi menjanjikan penebusan
bagi semua keturunan mereka. Sesudah bencana air bah, Allah membuat perjanjian
dengan Nabi Nuh, perjanjian antara Allah sendiri dengan semua makhluk hidup.

\kgk{Apa saja tahap-tahap selanjutnya wahyu Allah?}
Allah memilih Abram, memanggilnya keluar dari tanah airnya, menjadikannya
"bapa banyak bangsa" (Kej 17:5), dan berjanji melalui dia "semua bangsa di muka
bumi akan mendapat berkat" (Kej 12:3). Bangsa keturunan Abraham akan menjadi
orang-orang kepercayaan dari janji ilahi yang sudah diberikan kepada para bapa
bangsa. Allah membentuk Israel sebagai bangsa terpilih, membebaskan mereka
dari perbudakan di Mesir, menetapkan perjanjian di Gunung Sinai, dan melalui
Nabi Musa, memberikan hukum-Nya kepada mereka. Para nabi memaklumkan
penebusan bagi seluruh umat dan penyelamatan bagi segala bangsa dalam sebuah
perjanjian yang baru dan kekal. Dari bangsa Israel, dari keturunan Raja Daud akan
lahir sang Mesias, yaitu Yesus.


\flushright{(\dots \emph{bersambung} \dots)}
}