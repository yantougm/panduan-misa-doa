\small
\newpage
\section*{\center MEMULIAKAN TUHAN DENGAN BAKAT}

	Luciano Pavarotti adalah seorang lelaki Italia yang dikehendaki ayahnya menjadi professor di Universitas. Luciano tidak mau. Semalam suntuk ia berdebat dengan ayahnya gara-gara itu. Berhari-hari hubungan dengan ayahnya menjadi tegang. Sebab, Luciano sendiri bertekad menjadi penyanyi. Ayahnya berkata, "hanya satu dari seribu orang yang bias hidup dari menyanyi. "Tetapi, Luciano tidak menghiraukan kata ayahnya itu. Dia nekad. Blia anda ingin tahu, siapakah penyanyi opera terbaik abad 20? Luciano Pavarotti lah orangnya.

	Luciano, tanpa diduga oleh siapapun, ternyata melejit menjadi penyanyi opera yang tenar di taraf international. Suara tenor sang super star itu dikagumi banyak orang. Mengenai kehidupannya, ia berkata, "saya ingin mengisi hidup. Mengisinya dengan yang terbaik. Saya sampaikan pandangan saya lewat lagu-lagu saya. Lagu-lagu itulah jeritan hati nurani saya. Dengan itu saya memiliki sesuatu yang membuat orang lain bahagia. Mereka, para penonton itulah yang menjadi perhatian saya yang utama. Lain daripada itu, dengan lagu, saya memuliakan Tuhan. Karena daripada-Nya lah saya memperoleh semuanya ini. Kini saya bahagia dengan seorang istri dan tiga orang anak yang bak-baik." Katanya dengan penuh gairah. 

	Niat yang kuat, disertai dengan tekad yang bulat, siapak yang dapat membendung? Apalagi semua itu sudah didasari dengan pengenalan diri secara benar. Itulah yang dilakukan oleh Luciano Pavarotti. Sebelum melangkah menjadi penyanyi Luciano sudah mengenali betapa Tuhan menganugerahinya kekuatan super untuk suara tenornya. Banyak teman-temannya memuji kekuatan suara tenornya yang sangat kuat. Dengan dasar itulah Luciano bertekad menjadi seorang penyanyi. Meskipun mendapat perlawanan berat dari ayahnya yang ingin menyetir hidupnya, Luciano tetap teguh dengan pilihannya.

	Orangtua selalu mau mengatur segala arah dan keinginan hidup anak-anaknya. Padahal, masing-masing anak sudah diberi bakat oleh Tuhan yang sering lain dengan keinginan orangtua. Dengan memaksakan kehendaknya, sebenarnya orangtua tidak membantu anak, melainkan malah menjerumuskan. Oleh karena itu, orangtua harus membantu anak secara benar agar anak dapat menemukan bakat yang masih terpendam, namun sudah ditanam oleh Tuhan dalam diri sang anak. Peranan orangtua adalah menggali segala potensi dalam diri  anak. Bila anak sungguh-sungguh dapat mengembangkan bakat demi kebahagiaan sesamanya dan demi kemuliaan Tuhan. Ia akan menjadi orang yang tahu bersyukur kepada Allah dalam arti yang sesungguhnya. Orang yang demikian bukan hanya menjadi orang terkenal, melainkan juga mulia, harum namanya.
\begin{center} ***\end{center}
\normalsize
