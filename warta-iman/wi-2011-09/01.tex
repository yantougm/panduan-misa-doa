\section*{Kitab Suci dan Bulan Kitab Suci}

\subsection*{Sejarah Kitab Suci}

Kitab  Suci  disebut juga Alkitab dari bahasa Arab. Kata "Al"  maksudnya  "sang". Jadi  Alkitab adalah buku   yang    paling  luhur  dan paling unggul yakni "buku  suci"  atau  "Kitab  Suci".  Yang  dimaksudkan  ialah seluruh buku  iman kristiani, baik yang  disebut  Perjanjian Lama maupun Perjanjian Baru. 

Kitab Suci yang kita miliki sekarang merupakan salinan- salinan yang ditulis oleh baik para nabi ataupun sekretaris/wakilnya yang kita sebut Perjanjian Lama dan juga oleh para rasul yang hidup bersama Yesus Kristus merefleksikan imannya yang ditulis dengan bahasa, latar belakang budaya dan tahun yang berbeda-beda. Kitab Suci yang kita miliki sekarang memiliki dua bagian yaitu Kitab Suci Perjanjian Lama dan Perjanjian Baru. Kedua bagian itu diterima sebagai kitab suci kita setelah melalui proses kanonisasi yaitu proses di mana dilihat sejauh mana kitab-kitab yang ditemukan itu layak/pantas dinormakan atau dikanonisasikan untuk menjadi patokan iman dan juga moral dalam kehidupan kita sehari-hari.  

\subsubsection*{Isi Kitab Suci Katolik}
Kitab Suci berisi Sabda Allah itu sendiri. Sabda Allah pada jaman Israel disebut Kitab Perjanjian Lama dan pada Perjanjian Baru pun diterima sebagai sabda Allah sendiri yang diwahyukan melalui Yesus Kristus dan juga akhirnya direfleksikan oleh para rasul yang pada saat itu hidup bersama dengan Yesus Kristus. Kitab Suci Perjanjian Lama terdiri dari 46 Kitab. Dari ke-46 kitab masih terbagi menjadi 4 bagian:
   
\begin{tabular}{rp{8cm}}
5&Kitab Pentateukh \\
16&Kitab Sejarah (Dimana 7 Kitab termasuk dalam Deuterokanonika)\\
7&Kitab Puitis dan Hikmat\\
18&Kitab Para Nabi
\end{tabular}

Untuk ke-27 Kitab Perjanjian Baru terdiri dari 4 Injil (Yohanes, Markus, Lukas, Matius), Kisah Para Rasul, Epistola atau surat-surat dan Kitab Wahyu dengan tahun penulisan yang berbeda-beda.

\subsubsection*{Bulan September sebagai Bulan Kitab Suci pada Bulan September}
Latar belakangnya adalah hembusan angin segar Konsili Vatikan II (aggiornamento/pembaharuan) yang mengajak seluruh umat beriman Katolik untuk membaca, merenungkan, mendalami dan menghayati Kitab suci dalam kehidupan sehari-hari. Dari hembusan angin segar itu, gereja dalam hal ini Paus dalam kesatuannya dengan para Uskup mulai menentukan ujub-ujub khusus dan ujub-ujub umum sesuai dengan kebutuhan Gereja di mana dia berada. Dari ujub-ujub itu maka Tahta Suci dalam hal ini Paus bersama para Uskup menetapkan bulan September sebagai Bulan Kitab Suci yang ditetapkan oleh Konferensi Wali Gereja Indonesia.


\subsubsection*{Umat Katolik pernah dilarang membawa Kitab Suci ke Gereja}
Karena adanya pandangan klerikal yuridiksi yang menyatakan bahwa Kitab Suci hanya bisa dipegang, dibaca, disampaikan, diwartakan hanya oleh para klerus dan itu berlangsung sampai dengan Konsili Vatikan II tahun 1962. Sebelum itu juga karena ada pandangan bahwa gereja dan dunia di mana gereja harus hadir di dalam dunia pada waktu itu umat cukup identik dengan suatu keadaan yang penuh dengan kedosaan, setelah ada aggionarmento timbul kesempatan bahwa umat sekarang boleh memegang, boleh membaca sebagai lektor. Di sini pandangan Gereja adalah umat Allah dan umat sekarang mempunyai hak dan wewenang yang sama tapi tidak terlalu sama dengan kaum tertahbis tetap ada batasannya. Umat boleh berkumpul, bersatu, membuka kitab suci, mengadakan pendalaman, meresapi, bertanya dengan klerus.

Kaum tertahbis, yaitu Paus sebagai Pemimpin Hirarki, Uskup dan Para Imam, serta tahbisan sebelum Imam adalah Diakon yang mempunyai kewajiban untuk membacakan kitab suci pada waktu sebelum Konsili Vatikan II.

Kitab Suci berisi sabda Tuhan dan juga merupakan refleksi iman Gereja Perdana dan bangsa Israel. Singkatnya peranan kitab suci dalam kehidupan sehari-hari adalah sebagai patokan/norma iman dan juga norma moral yang konkrit. Banyak norma moral yang terkandung dalam Kitab Suci.

Sebenarnya gereja tidak melarang umat Katolik untuk menafsirkan Kitab Suci. Namun kita perlu membedakan kata menafsirkan, merefleksikan dan menerjemahkan. Menafsirkan berarti memberikan penjelasan /penerangan lebih lanjut kepada umat. Oleh karena itu dalam hal menafsirkan Gereja menentukan orang-orang yang memang mempunyai keahlian dalam bidang Kitab Suci, yang mengetahui seluk beluk, latar belakang budaya bahasa kitab suci apalagi kitab suci kita bisa dalam berbagai bahasa yang sulit dipahami. Dalam hal ini gereja meyakini bahwa magisterium atau kuasa mengajar gereja yang tidak dapat sesat yang dapat menafsirkan kitab suci. Dan juga para ahli Kitab Suci yang mempunyai keahlian dan mendapat surat keputusan khusus dari Tahta Suci untuk bisa menafsirkan Kitab Suci.

Kita bisa membagikan /mensharingkan kepada orang lain asalkan tidak jauh berbeda dan tidak sesat atau tidak berlawanan dengan ajaran-ajaran gereja kita.

\subsubsection*{Para kudus mendalami Kitab Suci}
St. Agustinus menghayati dan mendalami kitab suci sebagai sabda Allah sendiri sehingga dibutuhkan suatu konsentrasi penuh penghayatan untuk mendengarkan sabda Allah. Dia mengatakan bahwa jika ketika kita mendengarakan sabda Allah tidak dengan penuh konsentrasi sama buruknya seperti mengambil piala sesudah Konsekrasi lalu menginjak-injaknya dengan kaki. Ketika kita tidak mendengarkan sabda Allah dengan serius berarti kita tidak menghargai kitab suci yang menjadi sumber sabda Allah.

Ibu Teresa dari Calcuta walaupun sebagai seorang suster berpakaian seperti orang kebanyakan dan melakukan banyak hal dengan tulus hati. Dia mengatakan bahwa Allah itu artinya memberi. Dia berusaha mencontoh semangat hidup Allah dan memberikan dirinya untuk orang lain dengan melayani penderita lepra yang dalam kitab suci merupakan orang yang berdosa.

\subsubsection*{Kitab Deuterokanonika}
Kitab-kitab yang terdapat dalam Kanon disebut kitab-kitab kanonik. Orang Yahudi hanya menerima kitab suci yang ditulis dalam bahasa Ibrani sedangkan yang ditulis dalam bahasa Yunani tidak diterima. Jumlah kitab suci yang diterima sebanyak 39 kitab yang kemudian diterjemahkan dalam bahasa Yunani. Terjemahan itu diberi nama Septuaginta yang biasanya ditulis dengan simbol LXX. Dalam Septuaginta terdapat kitab yang sudah diterjemahkan ditambah dengan beberapa tulisan asli Yunani yang tadi tidak diterima dan kemudian diterima oleh gereja Katolik. Isi dari Kitab Deuterokanonika ada 7 yaitu Makabe, Sirakh, Kebijaksanaan, Yudit, Tobit, Tambahan Daniel, Tambahan Kitab Ester dan Surat Yeremia. Kita menerima karena di dalam kitab-kitab tersebut terdapat norma iman yang bisa dijadikan patokan.   

\subsubsection*{Cara membaca Kitab Suci dengan baik dan benar}
Kita perlu membaca Kitab Suci karena Kitab Suci merupakan pedoman hidup kita baik iman maupun moral maka kita perlu membaca sebagai penuntun hidup kita sehari-hari. Kita bisa merefleksikan makna kitab suci itu sendiri misalnya makna kuk yang merupakan alat untuk memikul barang.  

Membaca Kitab Suci dengan benar yaitu membaca dan merenungkan sehingga kita dapat menangkap makna di dalamnya. Membaca sesuai dengan kalender liturgi dari hari ke hari karena kalender liturgi mengikuti pola/peristiwa hidup Yesus dan Para Kudus juga merupakan cara yang tepat.

Waktu yang baik atau tepat untuk membaca Kitab Suci adalah
setiap hari sebelum melakukan aktivitas harian. Menyediakan waktu dengan sengaja untuk Tuhan karena yang terpenting bukan kuantitas tetapi kualitas pertemuan kita dengan Tuhan sendiri melalui firmanNya.

\subsubsection*{Meningkatkan minat anak untuk membaca Kitab Suci sejak dini}
Orang tua harus memberikan teladan pada anak-anaknya untuk membaca Kitab Suci sejak dini atau sejak masih kecil. Bisa dengan membacakan kisah-kisah menarik yang ada di dalam Kitab Suci, bisa dengan memberikan Kitab Suci dalam bahasa sehari-hari yang dilengkapi dengan gambar-gambar, bisa juga dengan nyanyian dan lain-lain.

Untuk membaca kitab suci tidak diperlukan karunia khusus. Kita membaca tidak hanya membaca tapi membaca perlahan-lahan, menangkap isinya, merenungkan salah satu perikop yang sesuai dengan pengalaman hidup sehari-hari. Kalau mau dilatih bisa melalui meditasi: membaca, merenungkan dan mengkonkritkan dalam tindakan.(\emph{yes})

\begin{center} ***\end{center}
