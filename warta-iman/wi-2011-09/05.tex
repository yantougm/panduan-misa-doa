\newpage
\section*{\center\LARGE{Empat Prinsip untuk Menginterpretasikan Alkitab}}
%\small
Alkitab merupakan Sabda Allah yang disampaikan melalui tulisan penulis kitab yang ditunjuk oleh Allah untuk menuliskan hanya yang diinginkan oleh Tuhan. Maka jika kita ingin memahami Alkitab, kita perlu mengetahui makna yang disampaikan oleh para pengarang kitab dan apakah yang ingin disampaikan oleh Allah melalui tulisannya. Dan karena Alkitab bersumber pada Allah yang satu, maka kita harus melihat keseluruhan Alkitab sebagai satu kesatuan yang saling melengkapi. Inilah yang menjadi dasar bagaimana kita memperoleh pengertian yang mendalam tentang Kitab Suci, dan dengan cara demikianlah jemaat awal mengartikan Kitab Suci.

\section*{Ke-4 Prinsip Mengartikan Alkitab}
Secara umum, Alkitab mempunyai dua macam arti. Yang pertama disebut 'literal/harafiah' sedangkan yang kedua disebut sebagai 'spiritual/rohaniah'. Kemudian arti rohaniah ini terbagi menjadi 3 macam, yaitu: alegoris, moral dan anagogis. Ke-empat macam arti ini secara jelas menghubungkan Perjanjian Lama dan Perjanjian Baru.
\begin{enumerate}
\item \textbf{Arti literal/ harafiah.}

Arti harafiah adalah arti yang berdasarkan atas penuturan teks yang ada secara tepat. Mengikuti ajaran St. Thomas Aquinas, kita harus berpegang bahwa, "Tiap arti [Kitab Suci] berakar di dalam arti harafiah".[4] Jadi dalam membaca Kitab suci, kita harus mengerti akan arti kata-kata yang dimaksud secara harafiah yang ingin disampaikan oleh pengarangnya, baru kemudian kita melihat apakah ada maksud rohani yang lain. Arti rohani ini timbul berdasarkan arti harafiah.

\item \textbf{Arti alegoris}
Arti alegoris adalah arti yang lebih mendalam yang diperoleh dari suatu kejadian, jika kita menghubungkan peristiwa tersebut dengan Kristus. Contohnya:
\begin{enumerate}
\item Penyeberangan bangsa Israel melintasi Laut Merah adalah tanda kemenangan yang diperoleh umat beriman melalui Pembaptisan (lih.Kel14:13-31; 1Kor 10:2).
\item Kurban anak domba Paska di Perjanjian Lama merupakan tanda kurban Yesus Sang Anak Domba Allah pada Perjanjian Baru (Kel 12: 21-28; 1 Kor 5:7)).
\item Abraham yang rela mengurbankan anaknya Ishak adalah gambaran dari Allah Bapa yang rela mengurbankan Yesus Kristus Putera-Nya (Kej 22: 16; Rom 8:32).
\item Tabut Perjanjian Lama adalah gambaran dari Bunda Maria, Sang Tabut Perjanjian Baru. Karena pada tabut Perjanjian Lama tersimpan dua loh batu kesepuluh perintah Allah (Kel 25:16) dan roti manna (Kel 25:30); sedangkan pada rahim Maria Sang Tabut Perjanjian Baru tersimpan Sang Sabda yang menjadi manusia (Yoh 1:14), Sang Roti Hidup (Yoh 6:35).
\end{enumerate}

\item \textbf{Arti moral}
Arti moral adalah arti yang mengacu kepada hal-hal yang baik yang ingin disampaikan melalui kejadian- kejadian di dalam Alkitab. Hal-hal itu ditulis sebagai "contoh bagi kita \dots sebagai peringatan" (1 Kor 10:11).
\begin{enumerate}
\item Ajaran Yesus agar kita duduk di tempat yang paling rendah jika diundang ke pesta (Luk 14:10), maksudnya adalah agar kita berusaha menjadi rendah hati.
\item Peringatan Yesus yang mengatakan bahwa ukuran yang kita pakai akan diukurkan kepada kita (Mrk 4: 24) maksudnya agar kita tidak lekas menghakimi orang lain.
\item Melalui mukjizat Yesus menyembuhkan dua orang buta, yang berteriak-teriak, "Yesus, Anak Daud, kasihanilah kami!" (Mat 20: 29-34) Yesus mengajarkan agar kita tidak lekas menyerah dalam doa permohonan kita.
\end{enumerate}

\item  \textbf{Arti anagogis}
Arti anagogis adalah arti yang menunjuk kepada surga sebagai ‘tanah air abadi’. Contohnya adalah:
\begin{enumerate}
\item Gereja di dunia ini melambangkan Yerusalem surgawi (lih. Why 21:1-22:5).
\item Surga adalah tempat di mana Allah akan menghapuskan setiap titik air mata (Why 7:17).
\end{enumerate}
\end{enumerate}

\section*{Contoh interpretasi Alkitab menggunakan ke-4 prinsip}
Maka semua kejadian di dalam Alkitab memiliki makna harafiah, walaupun dapat mengandung arti rohaniah juga. Contohnya adalah kisah Allah menurunkan roti manna di padang gurun (Kel 16).
\begin{itemize}
\item \textbf{Secara harafiah}, memang Allah memberi makan bangsa Israel dengan manna yang turun dari langit selama 40 tahun saat mereka mengembara di padang gurun.
\item \textbf{Secara alegoris}, roti manna menjadi gambaran Ekaristi, di mana Yesus sebagai Roti Hidup adalah Roti yang turun dari surga (Yoh 6:51), menjadi santapan rohani kita umat beriman yang masih berziarah di dunia ini.
\item \textbf{Secara moral}, kisah ini mengajarkan kita untuk tidak cepat mengeluh dan bersungut-sungut (Kel 16:2-3) kepada Allah. Umat Israel yang bersungut-sungut akhirnya dihukum Allah sehingga tak ada dari generasi mereka yang dapat masuk ke tanah terjanji (selain Yoshua dan Kaleb).
\item \textbf{Secara anagogis}, kita diingatkan bahwa seperti roti manna yang berhenti diturunkan setelah bangsa Israel masuk ke Tanah Kanaan, maka Ekaristi juga akan berakhir pada saat kita masuk ke Surga, yaitu saat kita melihat Tuhan muka dengan muka.
\end{itemize}

\section*{Peran Gaya Bahasa dalam Alkitab}
Seperti halnya pada sebuah karya tulis pada umumnya, peran gaya bahasa adalah sangat penting. Demikian juga pada Alkitab, sebab Allah berbicara pada kita dengan menggunakan bahasa manusia. Maka kita perlu memahami gaya bahasa yang digunakan, agar dapat lebih memahami isinya. Secara umum, gaya bahasa yang digunakan dalam Alkitab sebenarnya tidaklah rumit, sehingga orang kebanyakan dapat menangkap maksudnya. Dalam hampir semua perikop Alkitab, sebenarnya cukup jelas, apakah pengarang Injil sedang membicarakan hal yang harafiah atau yang rohaniah. Memang ada kekecualian pada perikop-perikop tertentu, sehingga kita perlu mengetahui beberapa prinsipnya:
\begin{enumerate}
\item \textbf{Simili}: adalah perbandingan langsung antara kedua hal yang tidak serupa. Misalnya, pada kitab Dan 2:40, digambarkan kerajaan yang ke-empat ‘yang keras seperti besi’, maksudnya adalah kekuatan kerajaan tersebut, yang dapat menghancurkan kerajaan lainnya.
\item \textbf{Metafor}: adalah perbandingan tidak langsung dengan mengambil sumber sifat-sifat yang satu dan menerapkannya pada yang lain. Contohnya, "Jiwaku haus kepada Allah Yang hidup" (Mzm 42:3). Sesungguhnya, jiwa yang adalah rohani tidak mungkin bisa haus, seperti tubuh haus ingin minum. Jadi ungkapan ini merupakan metafor untuk menjelaskan kerinduan jiwa kepada Allah.
\item \textbf{Bahasa perkiraan}: adalah penggambaran perkiraan, seperti jika dikatakan pembulatan angka-angka perkiraan. Misalnya,"Yesus memberi makan kepada lima ribu orang laki-laki" (Mat 14: 21; Mrk 6:44; Luk 9:14; Yoh 6:10) dapat berarti kurang lebih 5000 orang, dapat kurang atau lebih beberapa puluh.
\item \textbf{Bahasa fenomenologi}: adalah penggambaran sesuatu seperti yang nampak, dan bukannya seperti mereka adanya. Kita mengatakan ‘matahari terbit’ dan ‘matahari terbenam’, meskipun kita mengetahui bahwa kedua hal tersebut merupakan akibat dari perputaran bumi. Demikian juga dengan ucapan bahwa ‘matahari tidak bergerak’ (Yos 10: 13-14).
\item \textbf{Personifikasi/antropomorfis}: adalah pemberian sifat-sifat manusia kepada sesuatu yang bukan manusia. Contohnya adalah ungkapan ‘wajah Tuhan’ atau ‘tangan Tuhan’ (Kel 33: 20-23), meskipun kita mengetahui bahwa Tuhan adalah Allah adalah Roh (Yoh 4:24) sehingga tidak terdiri dari bagian-bagian tertentu.
\item \textbf{Hyperbolisme}: adalah pernyataan dengan penekanan efek yang besar, sehingga kekecualian tidak terucapkan. Contohnya adalah ucapan rasul Paulus, "Semua orang telah berbuat dosa dan telah kehilangan kemuliaan Allah" (Rom 3:23); di sini tidak termasuk Yesus, yang walaupun Tuhan juga sungguh-sungguh manusia dan juga tidak termasuk Bunda Maria yang walaupun manusia tetapi sudah dikuduskan Allah sejak dalam kandungan (tanpa dosa asal).
\end{enumerate}

Selanjutnya, ada juga kekecualian juga terjadi pada kondisi berikut:
\begin{enumerate}
\item Jika Alkitab jelas mengatakan bahwa yang disampaikan adalah perumpamaan. Contoh Yoh 10:6 "Itulah yang dikatakan Yesus dalam perumpamaan kepada mereka…" yang kemudian dilanjutkan oleh Yesus, yang mengumpamakan Ia sebagai ‘pintu’ (Yoh 10:7). Demikian juga dengan Mat 13:33 yang mengatakan bahwa Yesus mengajar dengan perumpamaan. Di sini perumpamaan belum tentu terjadi secara nyata.
\item Interpretasi harafiah dilakukan sejalan dengan akal sehat, namun jika tidak masuk akal, maka tidak mungkin dimaksudkan secara harafiah. Jadi misalnya, pada saat Yesus mengatakan bahwa raja Herodes adalah ‘serigala’ (Luk 13:32), maka kita tidak akan mengartikan bahwa pada waktu itu pemerintah di jaman Yesus dikepalai oleh mahluk mamalia, berambut, berekor, berkuping lancip yang bernama Herodes.
\item Jika pengartian secara harafiah malah menujukkan kontradiksi pada Allah, maka gaya bahasa yang diucapkan tidak dimaksudkan untuk diartikan secara harafiah. Dalam hal ini penting sekali kita melihat ayat-ayat lain untuk melihat gambaran yang lebih jelas akan makna ayat tersebut. Contoh: Dalam Mat 23:9, Yesus berkata "Jangan memanggil seorangpun sebagai bapa di bumi ini", padahal baru sesaat sebelumnya Yesus mengulangi perintah ke-4 dari kesepuluh perintah Allah, "Hormatilah ibu bapa-mu" (Mat 19:19) dan Ia juga menyebut Abraham sebagai "bapa" (Mat 3:9). Selanjutnya kita melihat bagaimana Rasul Paulus kemudian menyebut dirinya sendiri sebagai "bapa" bagi umat di Korintus (1 Kor 4:15) dan kepada Onesimus (Flm 10). Maka ayat Mat 23:9 tidak mungkin diartikan secara harafiah. Dalam hal ini, Yesus menggunakan gaya bahasa hyperbolisme untuk menyatakan otoritas ilahi yang mengatasi otoritas duniawi.
\end{enumerate}

\section*{Tips utama dan contohnya}
Jadi di sini kita perlu mengingat bahwa jika bahasa yang dipakai tidak menunjuk kepada arti figuratif, dan jika tidak ada kondisi kekecualian seperti yang disebutkan di atas, maka kita harus menginterpretasikan perikop secara harafiah, kecuali adanya argumentasi yang sangat meyakinkan untuk mengartikan sebaliknya. Kita tidak boleh memilih-milih ayat mana yang kelihatannya baik dan mudah untuk dicerna, dan mana yang tidak, untuk menentukan apakah dapat diartikan secara harafiah atau tidak. Misalnya, ada banyak orang tidak menyukai adanya neraka, maka mereka menganggap perkataan Yesus tentang neraka hanya sebagai ucapan simbolis. Ini tentu saja keliru! Atau misalnya, banyak orang salah mengartikan perikop tentang Roti Hidup pada Injil Yohanes 6. Mereka tidak dapat menerima ucapan Yesus secara harafiah,"Jikalau kamu tidak makan daging-Ku dan minum darah-Ku, kamu tidak mempunyai hidup di dalam dirimu; dan barangsiapa makan daging-Ku dan minum darah-Ku, ia mempunyai hidup yang kekal …" (Yoh 6:53-54). Mereka mengartikannya bahwa Yesus hanya berbicara secara simbolik saja. Hal ini tentu adalah \textbf{sikap yang keliru}, yaitu \textbf{mengartikan suatu perikop secara harafiah atau simbolik hanya berdasarkan ‘selera’ saja atau terbatas pada pemikiran yang sempit}.

Jika seseorang menganggap perikop Roti Hidup sebagai ‘ayat yang sulit sehingga lebih baik tidak diartikan secara literal tetapi figuratif saja’, maka orang itu memasukkan dirinya dalam golongan orang-orang yang pada jaman Yesus juga menganggap ayat itu terlalu sulit, dan memilih untuk meninggalkan Yesus. "Perkataan ini keras, siapakah yang sanggup mendengarkannya?" (Yoh 6:60). Dan sungguh banyak murid-murid-Nya yang pergi mengundurkan diri dan tidak lagi mengikuti Dia, setelah Yesus mengajarkan demikian. (Yoh 6:66). Jika pengajaran ini hanya bermaksud simbolis, tentu Yesus akan mencegah mereka pergi. Namun Alkitab mengatakan yang sebaliknya. Menanggapi hal ini, Yesus malah bertanya kepada para rasulnya, apakah mereka mau pergi juga. Dan Petrus, mewakili para rasul menjawabNya, "Tuhan, kepada siapakah kami akan pergi? PerkataanMu adalah perkataan hidup yang kekal" (Yoh 6: 68). Maka kita ketahui bahwa hanya para Rasul dan mereka yang setia memegang ajaran ini, adalah mereka yang kepadanya Yesus telah berjanji, "Barangsiapa yang memakan Aku, akan hidup oleh Aku… ia akan hidup selama-lamanya." (Yoh 6: 57-58). Sekarang memang kita perlu menilik ke dalam diri kita, termasuk golongan manakah kita ini: yang menerima ayat tersebut secara harafiah ataukah yang figuratif? Jika kita menerima ayat itu secara harafiah sesuai kehendak Yesus, dan kita sudah percaya kepada kehadiran Yesus yang nyata dalam Ekaristi, selanjutnya, apakah sikap kita dalam menyambut Ekaristi sudah mencerminkan iman kita itu?

Contoh yang lain adalah cerita Nabi Yunus yang ditelan oleh ikan besar selama 3 hari (Yun 1:17), sebelum dimuntahkan ke laut. Banyak orang menganggap kisah ini tidak masuk akal, sehingga lebih baik dianggap figuratif saja. Namun bagi kita yang percaya pada Sabda Allah, maka sesungguhnya tidaklah sulit bagi kita untuk percaya bahwa hal ini harafiah terjadi, apalagi kisah inilah yang dipakai oleh Yesus untuk menggambarkan kematian-Nya sebelum Ia bangkit pada hari ketiga (Mat 12:39-41; Luk 11:29-32). Melihat pentingnya misteri wafat dan kebangkitan-Nya, tentulah Yesus tidak sekedar hanya mengambil kisah simbolis, namun kisah yang sungguh terjadi.

Di sini kita melihat, jika kita mulai mempertanyakan terus dan hanya mau menerima apa yang dapat dibuktikan dengan akal, maka kita dapat terjebak pada memilih-milih ayat sesuai dengan keinginan kita, dan akhirnya dapat mempertanyakan segala mukjizat yang ada dalam Kitab Suci. Hal inilah yang dimiliki oleh banyak ahli Kitab suci jaman modern, yang berusaha merasionalisasikan Alkitab, dan sedapat mungkin mencoret unsur mukjizat dan intervensi ilahi. Sikap yang demikian bukanlah sikap yang rendah hati yang disyaratkan untuk membaca Sabda Tuhan, dan kita sungguh perlu berdoa agar kita tidak mempunyai sikap yang demikian.

\section*{Kesimpulan}
\begin{footnotesize}
Keempat prinsip untuk menginterpretasikan Alkitab adalah pedoman bagi kita untuk mendapatkan pengertian yang lebih mendalam akan ayat-ayat Kitab Suci. Prinsip-prinsip tersebut membantu kita untuk dapat "membaca dan menginterpretasikan Kitab Suci dengan semangat roh yang sama dengan bagaimana kitab tersebut dituliskan",[8] dan dengan demikian kita dapat mendapatkan gambaran yang lebih menyeluruh tentang makna ayat-ayat dalam Kitab suci, karena kita melihat juga kaitan satu ayat dengan ayat-ayat yang lain. Sudah menjadi Tradisi Gereja bahwa ayat-ayat Alkitab tidak untuk dipertentangkan satu dengan yang lain, tetapi selalu dilihat dalam satu kesatuan yang utuh dan saling melengkapi. Mari kita belajar dari teladan kebaikan Tuhan, yang walaupun tetap mempertahankan kebenaran dan kekudusan-Nya, telah sedemikian menyesuaikan Diri-Nya untuk menjangkau kita semua dengan menggunakan bahasa manusia. Mari kita melakukan bagian kita, dengan berusaha untuk memahami apa yang hendak disampaikan-Nya kepada kita.
\end{footnotesize}
\begin{center} ***\end{center}
\normalsize