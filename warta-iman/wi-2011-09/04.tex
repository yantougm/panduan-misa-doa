\newpage
\section*{\center KANONISASI}
 
Kanon sebetulnya berarti tongkat.
Tetapi kemudian tongkat  itu  juga  dipakai  sebagai  ukuran
(serupa  dengan  tongkat  yang dipakai untuk mengukur kain).
Dari situ kata kanon mendapat arti ukuran atau patokan.  Dan
berhubungan  dengan  Kitab  Suci, kanon berarti ukuran untuk
tulisan-tulisan yang sungguh-sungguh  termasuk  Kitab  Suci.
Dalam  praktek,  kanon  berarti daftar buku-buku yang diakui
sebagai bagian dari Kitab Suci.  Sebab  di  samping  tulisan
Kitab  Suci, adajuga tulisan-tulisan lain yang serupa, namun
yang tidak sungguh-sungguh merupakan tanggapan  iman  Gereja
atas  Sabda Allah. Tulisan-tulisan seperti itu tidak diakui,
sehingga tidak termasuk kanon.

Proses kanonisasi  sudah  terjadi
sejak  semula.  Dalam penggunaan tulisan-tulisan Kitab Suci,
umat sendiri, baik umat PL maupun umat  PB  membuat  seleksi
antara tulisan-tulisan yang ada. Baru kemudian, mulai dengan
abad kedua Masehi, dibuat daftar-daftar  yang  kurang  lebih
resmi.  Dan daftar yang sungguh resmi sebetulnya baru dibuat
oleh Konsili Trente pada abad ke-16.

Proses   pembakuan/kanonisasi   terjadi   dan
dilaksanakan di kalangan jemaat. Mereka membedakan buku-buku
yang betul-betul mereka akui sebagai buku yang mengungkapkan
iman  Gereja,  dan  buku-buku  yang  hanya merupakan tulisan
seseorang. Bisa jadi bahwa  tulisan-tulisan  perorangan  itu
baik-baik, tetapi tidak merupakan ungkapan iman Gereja. Maka
oleh umat, di bawah bimbingan pimpinan Gereja,  tulisan  itu
tidak diakui sebagai Kitab Suci


Kriteria/syarat yang dipakai untuk pembakuan itu ada tiga.

\begin{enumerate} 
\item Isi

Oleh umat dilihat apakah isinya benar-benar
mengungkapkan iman Gereja. Bukan hanya  perasaan  atau  iman
seseorang, tetapi betul-betul iman seluruh Gereja.
 
\item Universal

Secara universal diterima sebagai Kitab Suci, buku-buku yang oleh seluruh  Gereja  dan
dimana-mana diakui sebagai Kitab Suci.

\item Umur pemakaian
Hanya  diakui  sebagai   Kitab   Suci jika
buku-buku itu dari awal diterima oleh Gereja dimana-mana.
\end{enumerate}
 
Tentu  saja dalam penerapan dan penggunaan kriteria-kriteria
itu orang bekerja sesuai dengan  kemampuan  dan  kemungkinan
situasinya. Kriteria-kriteria itu tidak bersifat ilmiah
dan juga cara menyelidikinya  tidak  terlalu  ilmiah.  Suatu
buku diterima atau ditolak, menurut keyakinan umat.


Kanonisasi dibuat olehumat,  seluruh  Gereja.  Memang  dibawah   bimbingan
hierarki,  namun  tidak  oleh  hierarki saja. Malahan justru
penggunaan liturgis dalam  jemaat  adalah  cara  yang  utama
dalam menentukan kanon itu.

Proses kanonisasi sungguh
ditentukan oleh umat. Umat perdana. Setelah ditetapkan - dan
itu praktis terjadi dalam abad kedua atau ketiga - tidak ada
keragu-raguan lagi. Juga tidak ada diskusi  lagi  dan  tidak
ada   perubahan   lagi.   Maka  kalau  dikatakan  kanonisasi
dilakukan  oleh  umat,  itu   tidak   berarti   bahwa   umat
sewaktu-waktu  dapat  meninjau  kembali  kanon  Kitab  Suci.
Sekarang ini kanon Kitab Suci,  baik  PL  maupun  PB,  telah
ditetapkan dan tidak bisa diubah lagi.


Ditetapkannya kanon Kitab Suci,
supaya orang tahu, mana buku-buku yang  oleh  Gereja  diakui
sebagai  Kitab  Suci.  Supaya  tahu  dimana ada sumber iman.
Supaya tahu mana pedoman pokok untuk iman orang kristiani.
 
{\noindent \emph{Sumber: Permasalahan Sekitar Kitab Suci oleh Dr. Tom Jacobs, SJ, Kanisius, 1996}}

\begin{center} ***\end{center}