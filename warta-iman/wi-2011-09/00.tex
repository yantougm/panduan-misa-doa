\newpage

\section*{\center Dari Redaksi}

\indent{Berkah Dalem,}

Sekarang kita telah memasuki bulan September, bulan yang oleh gereja digunakan sebagai Bulan Kitab Suci Nasional. Bulan Kitab Suci Nasional merupakan bulan yang secara khusus digunakan untuk membahas atau mengenal lebih jauh kitab suci, melalui sarasehan-sarasehan ataupun dalam bentuk lain. Dalam bulan ini baik di gereja maupun di lingkungan pasti senantiasa diperdengarkan hal-hal yang berkaitan dengan kitab suci. Ada baiknya momen bulan kitab suci ini menjadikan saat bagi kita untuk kembali mendalami kitab suci atau yang paling sederhana kita membaca, entah satu atau dua ayat. 

Gereja telah menerbitkan kalender liturgi setiap tahunnya yang memuat bacaan-bacaan kitab suci harian maupun mingguan yang dapat kita gunakan sebagai pedoman bagi kita untuk membaca kitab suci. Saat ini pula tepat bagi kita (tidak mengesampingkan saat-saat yang lain) untuk menjadikan kitab suci menjadi bacaan yang inspiratif bagi kehidupan kita walaupun kadangkala hal itu menjadi sesuatu yang kurang menarik. Atau paling sangat sederhana kita tengok apakah kita masih mempunyai kitab suci atau tidak dan kondisinya bagaimana (Jangan-jangan karena tidak pernah disentuh, kitab suci kita hilang entah kemana).

Edisi ini juga memuat cuplikan tentang isi Kompendium Katekese Gereja Katolik, yang merupakan kelanjutan dari cuplikan di edisi sebelumnya.

\begin{center}***\end{center} 

\vspace*{1.3cm}

\noindent{\framebox{\parbox{10cm}{\scriptsize
Warta Iman\\
Media komunikasi dan informasi umat lingkungan St. Petrus\\
Alamat Redaksi: Lingkungan St. Petrus Maguwo\\
E-mail: stpetrusmgw@gmail.com
}}}
