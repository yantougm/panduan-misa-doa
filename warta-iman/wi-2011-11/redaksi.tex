\newpage

\chap{Dari Redaksi}

\indent{Berkah Dalem,}

Kematian adalah sesuatu hal yang sudah pasti namun sekaligus timbul ketidakpastian. Kita semua yang hidup di dunia ini suatu saat akan mati. Kapan itu terjadi? Tidak ada yang tahu. Bagaimana setelah kita mati? Tidak tahu pasti. Namun berdasarkan telaah Kitab Suci yang dilakukan para ahli dan berdasarkan tradisi gereja, kita dapat sedikit mengetahui apa yang kira-kira akan terjadi setelah kita mati. Tema edisi kali ini memang hal-hal yang berhubungan dengan arwah.

Sebagai orang Katolik kita mengimani adanya kebangkitan badan dan tempat penantian. Sedikit hal tentang kebangkitan badan dan api penyucian bisa disimak pada WI edisi kali ini. Tulisan tentang api penyucian berupa cuplikan sebagian wawancara dengan Maria Simma. Dengan mengetahui sedikit hal tentang kebangkitan badan dan api penyucian tersebut kita dapat mempersiapkan diri lebih baik dan lebih mantap menghadapi kematian.

Masih tentang kematian ada tulisan berupa dua cerita pendek yang bertutur tentang saat seseorang kehilangan orang yang dikasihinya. Kita dapat mengambil hikmah dari dua cerpen dengan kisah bersambungan dan dikirim oleh ketua \textit{kepala suku} kita ini. Renungan lain tentang penilaian kita terhadap pihak lain yang terkadang cuma melihat kulitnya, tentang pencarian Tuhan yang sebetulnya selalu ada di dekat kita, dan bersyukur yang harusnya menjadi hal yang selalu mengirinya langkah kita.  

Di bagian terakhir tetap memuat kelanjutan cuplikan tentang isi Kompendium Katekese Gereja Katolik.

\begin{center}***\end{center} 

\vfill

\noindent{\framebox{\parbox{10cm}{\small
Warta Iman\\
Media komunikasi dan informasi umat lingkungan St. Petrus\\
Alamat Redaksi: Lingkungan St. Petrus Maguwo\\
E-mail: stpetrusmgw@gmail.com
}}}


