\chap{Kebangkitan Badan}

Kehidupan manusia di dunia ini berakhir dengan kematian. Pada saat kematian, tubuh manusia mengalami kehancuran sedangkan jiwanya melangkah menuju Allah dan emnunggu saat disatukan kembali dengan tubuh baru. Kebangkitan badan adalah kebangkitan orang-orang benar dari mati dalam tubuh yang baru, sebagai kebangkitan definitif untuk kehidupan abadi. Dengan bangkit dari mati, orang-orang benar akan mengalami eksistensinya yang baru. Sesudah kematian, tidak hanya jiwa kita yang hidup terus tetapi ``tubuh kita yang fana'' ini juga akan hidup kembali dalam kondisi yang samasekali baru (KGK 990)

Ajaran iman Katolik mengatakan bahwa setelah kematian ada kebangkitan badan dan kehidupan kekal.
Pada saat orang meninggal dunia, maka badan/tubuhnya akan terpisah dengan jiwanya. Badan yang dikubur/dikremasi – kembali menjadi debu, dan jiwa/rohnya akan kembali pada Allah. Dihadapan Allah, jiwa orang itu akan diadili (pengadilan khusus) berdasar perbuatan dan imannya untuk menentukan nasib selanjutnya.
\begin{itemize}
\item Jika perbuatan dan imannya tanpa cela, maka upahnya adalah langsung masuk surga
\item Jika berimankan Allah, pekerjaan baik, namun kadang masih jatuh dalam dosa --bertobat-dosa lagi-- bertobat lagi, maka mereka akan ditempatkan sementara dalam api penyucian untuk dimurnikan dosa-dosanya. Jika saatnya tiba, atas belas kasih Allah dan doa-doa serta kurban dari umat yang masih hidup, maka mereka juga akan masuk surga.
\item Jika imannya menentang Allah, secara sadar dan tanpa paksaan melakukan perbuatan yang sangat jahat dan tidak bertobat, maka nerakalah upahnya.
\end{itemize}

Pada akhir zaman, pada waktu kedatangan Kristus yang kedua kalinya, akan terjadi kebangkitan badan
Pada saat Kebangkitan badan, Kristus akan membangkitkan kembali tubuh semua orang yang sudah meninggal, namun dengan tubuh baru dan menyatukannya kembali dengan jiwa/roh masing-masing. Kita tidak tahu persisnya kapan, namun secara definitif hal itu terjadi pada hari kiamat atau akhir zaman. Meskipun demikian, di pihak lain, kita sebenarnya telah bangkit bersama Kristus dalam arti tertentu. Oleh Roh Kudus, kehidupan Kristen di dunia ini sudah merupakan keikutsertaan pada kematian dan kebangkitan Kristus:

\begin{quote}
\textit{Karena dengan Dia kamu dikuburkan dalam baptisan, dan di dalam Dia kamu turut dibangkitkan juga oleh kepercayaanmu kepada kerja kuasa Allah, yang telah membangkitkan Dia dari orang mati. \ldots
Karena itu, kalau kamu dibangkitkan bersama dengan Kristus, carilah perkara yang di atas, di mana Kristus ada, duduk di sebelah kanan Allah.
}(Kol 2:12; 3:1)
\end{quote}   

\noindent{Dari surat St. Paulus kepada jemaat di Korintus tertulis, }

\begin{quote}
\textit{Tetapi mungkin ada orang yang bertanya: 'Bagaimanakah orang mati dibangkitkan? Dan dengan tubuh apakah mereka akan datang kembali?' Hai orang bodoh! Apa yang engkau sendiri taburkan, tidak akan tumbuh dan hidup, kalau ia tidak mati dahulu. Demikianlah pula halnya dengan kebangkitan orang mati. Ditaburkan dalam kebinasaan, dibangkitkan dalam ketidakbinasaan. Ditaburkan dalam kehinaan, dibangkitkan dalam kemuliaan. Ditaburkan dalam kelemahan, dibangkitkan dalam kekuatan. Yang ditaburkan adalah tubuh alamiah, yang dibangkitkan adalah tubuh rohaniah.} (1 Kor 15:35-36, 42-44).
\end{quote}

Tubuh kaum beriman akan diubah serupa dengan Tubuh Kristus yang bangkit. Menurut teologi, tubuh yang mulia dan sempurna ini memiliki karakteristik: identitas, keutuhan dan keabadian, dengan empat ``kualitas transenden''
\begin{enumerate} 
\item ``\textbf{tak dapat rusak}'', atau bebas dari kerusakan fisik, kematian, penyakit, dan rasa sakit; 
\item ``\textbf{semarak}'' atau bebas dari cacat dan dikaruniai keindahan dan cahaya; 
\item ``\textbf{leluasa}'' di mana jiwa menggerakkan tubuh dan adanya kebebasan gerak; 
\item ``\textbf{halus}'', di mana tubuh sepenuhnya dirohanikan di bawah kuasa jiwa. Katekismus mengajarkan, ``Sesudah pengadilan umum, semua orang yang benar, yang dimuliakan dengan jiwa dan badannya, akan memerintah bersama Kristus sampai selama-lamanya.'' (KGK no. 1042).
\end{enumerate}

Bagi tubuh dari jiwa-jiwa yang dikutuk di neraka
walaupun mereka akan memiliki identitas, keutuhan dan keabadian, tetapi tidak memiliki keempat kualitas transenden. Mereka memiliki kondisi yang memungkinkan mereka menderita hukuman abadi di neraka, tetapi tidak memiliki kemuliaan Kristus yang dikaruniakan bagi mereka yang ada di surga.

Pada saat kedatangan Kristus yang kedua kalinya, akan terjadi pengadilan terakhir. Tidak ada yang tahu kapan saat itu terjadi. Hanya Bapa yang tahu kapan dan bagaimana semua akan berlangsung. Allah Bapa melalui PutraNya Yesus Kristus akan membuat penilaian ata seluruh sejarah kehidupan manusia dan ciptaanNya. Pengadilan terakhir akan membuktikan keadilan Allah yang mengatasi segala ketidakadilan seluruh makhluk ciptaanNya dan akan membuktikan bahwa cintaNya lebih besar daripada kematian (KGK 1040)

Tidak ada yang tersembunyi di mata Allah, dan di hadapan Kristus sebagai hakim, semuanya akan dibuka dan akan terlihat nyata hubungan setiap manusia dengan Allah yang sebenarnya. Pengadilan terakhir akan membuka sampai ke akibat-akibat yang paling jauh, kebaikan apa yang dilakukan atau tidak dilakukan oleh setiap orang selama hidup di dunia ini. (KGK 1039)

Kepercayaan mengenai adanya pengadilan terakhir mengajak semua orang agar bertobat. Selain itu, kita juga diingatkan bahwa kehidupan di dunia ini adalah kesempatan penuh rahmat untuk melakukan pertobatan. Tidak ada orang yang tahu kapan kita mati dan juga tidak ada yang tahu kapan akhir zaman tiba. Ajaran tentang pengadilan terakhir mengajak orang untuk membangun pertobatan sejak sekarang, sebelum semuanya terlambat. Adanya pengadilan terakhir dapat saja menimbulkan ketakutan akan Allah, namun dapat saja menjadi kerinduan dari orang-orang benar dan saleh. Semua orang benar akan memperoleh apa yang selama ini dirindukannya, yaitu saat persatuan dengan Allah dalam kehidupan abadi. Bagi orang benar, saat itu menjadi saat rahmat yang akan membuat orang merasakan kebahagiaan karena diperkenankan bersatu dengan Allah Tritunggal dan para kudus dalam kehidupan abadi.
\sumber{PAS}