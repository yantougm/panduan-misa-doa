\chap{Datanglah Kerajaan-Mu}
\small
``Mbak Wid, bagaimana keadaan bapak?'', tanya Johanes Sukendro teman kuliah suamiku Agung Laksono disaat aku sedang bertugas sebagai perawat di bagian ICU di sebuah rumah sakit swasta di Jogjakarta. 

``Belum ada peningkatan mas, penyakit bapak menurut dokter sudah berat.'', jawabku turut merasa prihatin. ``Doakan Mas Jo, semoga Tuhan berkenan memberi  kesembuhan kepada bapak. Dokter dan kami perawat akan berusaha semampu kami.'', ujarku selanjutnya. 

``Terima kasih mbak Wid, kami tetap berdoa memohon kesembuhan untuk bapak.'', ujar Jo sambil menyalamiku kemudian Jo pergi menunggu diluar.

Aku pergi menghampiri ranjang Pak Widi (nama bapaknya Jo), aku melihat ada kegelisahan pada diri Pak Widi. Beliau tidak bisa memejamkan mata seperti hari-hari lalu. Melihat kegelisahan yang beliau alami, aku mencoba mendekat dan menanyakannya. Ketika aku datang dan menyentuhnya, Pak Widi memandangku dalam-dalam. 

``Suster \ldots mengapa suster menjadi perawat?'', tanya Pak Widi tiba-tiba. Aku sendiri terkejut dengan pertanyaan itu. Semula aku akan menanyakan tentang kegelisahannya, tiba-tiba disodori pertanyaan diluar dugaanku. Aku tidak yakin dengan pertanyaan itu. Jangan-jangan Pak Widi ini bicara diluar kendali.

``Bagaimana Pak?',' tanyaku ulang.

``Suster \ldots Mengapa Suster menjadi perawat?'', Pak Widi mengulangi pertanyaannya. Yakin akan kesungguhan pertanyaan itu, maka aku mencoba mengisahkan pengalamanku, selain itu aku ingin menghiburnya, mungkin dengan mendengarkan ceritaku kegelisahannya bisa berkurang dan beliau bisa tidur tenang, ternyata tidak demikian.

``Pak \ldots mengapa bapak belum juga tidur. Bapak memikirkan apa?'', tanyaku lembut.
Pak Widi hanya diam saja. Matanya masih terbuka lebar dan menatap padaku. Aku menjadi bingung. Apa yang harus aku lakukan?

``Pak \ldots ini sudah jam dua pagi, bapak istirahat ya!'', ajakku.

Pak Widi tetap diam. Sementara bunyi monitor pasien lain berbunyi. Aku tidak beranjak, kulihat teman perawat lain menanganinya. Monitor dan erangan pasien memang menjadi perhatian khusus bagi kami semua yang bertugas pada waktu itu.
Kembali perhatianku kuarahkan pada Pak Widi.

``Pak, lebih baik bapak berdoa, supaya bapak bisa tenang dan bapak bisa tidur dengan tenang. Saya akan berdoa untuk bapak menurut cara yang saya imani. Bapak juga demikian.'' ujarku memohon.

Tiba-tiba tangan Pak Widi yang kekar itu meraih tanganku, katanya, ``Suster \ldots Berdoalah disamping saya.''

``Baik, Pak. Saya akan berdoa untuk bapak. Bapak sendiri berdoa menurut cara yang bapak anut.'' 

Aku memilih berdoa Bapa Kami. Baru beberapa kata kumulai, Pak Widi berkata, ``Suster \dots keras sedikit. Saya ingin dengar.''
Kuawali lagi doa Bapa Kami. ``Bapa kami yang ada di surga, dimuliakanlah nama-Mu. Datanglah kerajaan-Mu.''

Tiba-tiba Pak Widi menirukan yang kuucapkan, ``datanglah kerajaan-Mu''

``Tit \ldots tit \ldots Tit \ldots '' suara monitor pernapasan dan jantung itu mendadak berhenti besamaan dengan suara Pak Widi menirukan ucapanku, ``datanglah kerajaan-Mu.''

Aku cepat-cepat bangkit dan memanggil teman-teman. Aku dan teman-teman pun kemudian sibuk untuk menangani Pak Widi. Sementara ada teman yang lain memanggil dokter jaga untuk membantu menyelamatkannya. Setelah sepuluh menit kami berjuang untuk menyelamatkan Pak Widi, ternyata kami tidak berhasil. Tepat pukul 02.30 WIB Pak Widi dinyatakan meninggal dunia.

Meski kulihat wajahnya begitu tenang, namun aku merasa kehilangan. Pak Widi walaupun baru tiga hari aku mengenalnya, aku merawatnya dengan penuh kasih, apalagi beliau adalah bapak dari Jo sahabat suamiku, aku merasa seperti merawat orangtuaku sendiri. Tanpa kusadari airmataku meleleh jatuh kepipiku. Aku merasa bingung dan merasa bersalah kepada Mas Jo, karena sama sekali belum peka dengan tanda-tanda akan kematian.

Aku segera menghubungi suamiku dan menghubungi Mas Jo, mengabarkan kepergian Pak Widi ke kerajaan Allah dengan perasaan merasa bersalah. Suamiku dan Mas Jo segera datang ke rumah sakit.

``Maafkan saya, Mas Jo \ldots saya sungguh merasa bersalah.'', ujarku dengan diiringi lelehan air mata.

``Sudah kehendak Allah yang Mahakuasa, Mbak Wid. Terima kasih mbak Wid  mau merawat dan mendampingi  bapak. Tidak perlu merasa bersalah.'', kata Jo tegar. 

Kemudian aku menceritakan proses kejadiannya hingga Pak Widi pergi ke kerajaan Allah kepada Mas Jo dan suamiku.

``Syukur kepada Allah. Sungguh rahmat yang luar biasa bagi bapak. Hatiku tenang akan kepergian bapak.'', ujar Jo dengan rasa penuh syukur. 

\begin{flushright}
\textit{Medio Oktober’2011 \\Bravo Sierra}
\end{flushright}
\normalsize