\chap{Kerahiman Ilahi}

 ``Gung, kamu kan tahu kalo bapakku masih seorang muslim walaupun ibuku sudah katolik. Coba kamu lihat isi dompet milik bapakku'', kata Jo sambil menyodorkan sebuah dompet lusuh kepadaku. Dengan rasa heran aku menerima dompet itu dan membukanya.    
 
``Isinya ada uang dua ratus ribu rupiah, KTP, SIM C dan foto bu Widi.'', ujarku heran.

``Coba kamu rogoh selipan dibelakang KTP, Gung.'', kata Jo. Aku merogoh selipan dibelakang KTP ternyata ada gambar Tuhan Yesus dan gambar Bunda Maria dan sebuah kertas lipatan. Aku membuka kertas lipatan itu dan membacanya:
\begin{quote}
\textit{
KepadaMu ya Tuhan, kuangkat jiwaku; Allahku, kepadaMu  aku percaya; janganlah kiranya aku mendapat malu; janganlah musuh-musuhku bersuka ria atas aku. Ya, semua orang yang menantikan Engkau takkan mendapat malu; yang mendapatkan malu adalah mereka yang berbuat khianat dengan tidak ada alasannya. Bawalah aku berjalan dalam kebenaranMu dan ajarlah aku, sebab Engkaulah Allah yang menyelamatkan aku, Engkau kunanti-nantikan sepanjang hari. Ingatlah segala rahmatMu dan kasih setiaMu, ya Tuhan, sebab itu sudah ada sejak purbakala. Dosa-dosaku pada waktu muda dan pelanggaran-pelanggaranku janganlah Kau ingat, tetapi ingatlah kepadaku sesuai dengan kasih setiaMu, oleh karena kebaikanMu, ya Tuhan. Tuhan itu baik dan benar; sebab itu Ia menunjukkan jalan kepada orang yang sesat. Ia membimbing orang-orang yang rendah hati menurut hukum, dan Ia mengajarkan jalanNya kepada orang-orang yang rendah hati. Segala jalan Tuhan adalah kasih setia dan kebenaran bagi orang yang berpegang pada perjanjianNya dan peringatan-peringatanNya. Oleh karena namaMu, ya Tuhan, ampunilah kesalahanku, sebab besar kesalahan itu. Siapakah orang yang takut akan Tuhan? Kepadanya Tuhan menunjukkan jalan yang harus dipilihnya.}
\end{quote}



``Syukur kepada Allah. Ini adalah ayat-ayat Mazmur.'', ujarku sambil menatap wajah Jo yang penuh senyum gembira.

``Tapi Gung, walaupun ibuku sudah mengetahui hal ini, ibuku tetap memintaku untuk mengadakan upacara doa pelepasan dan pemakaman jenazah bapak dengan ritual secara agama Islam. Menurutmu bagaimana?'', tanya Jo. 


``Seharusnya begitu Jo, karena masyarakat di kampung ini mengetahui Pak Widi adalah seorang muslim. Janganlah kamu khawatir, Tuhan pasti mencari domba-dombaNya yang hilang.'', kataku sambil menepuk bahu sahabatku.

Para pelayat mulai berdatangan,  upacara doa pelepasan jenazah Pak Widi dilakukan secara agama Islam, hingga sampai di pemakaman.  Karena profesiku seorang kuli tinta tanpa sadar aku sering mengambil foto dengan camera HP ku. Ketika  jenazah Pak Widi dimasukkan ke liang makam, aku mencoba mengambil gambar (foto) nya, tapi entah mengapa kok tiba-tiba pencetan/tombol pengambil foto macet, tidak bisa di pencet seperti macet. Berulang kali aku mencoba memencet-mencet tombol pengambil  foto sampai akhirnya aku berhasil memencet mengambil foto jenazah Pak Widi di liang makam. Sampai acara pemakaman selesai cukup banyak foto yang berhasil aku ambil.

Sesampai di rumah, karena kebiasaanku sebagai kuli tinta, aku melihat kembali hasil foto-foto  yang berhasil  aku ambil. Hanya ada satu yang aneh yaitu foto jenazah Pak Widi di liang makam tidak mau keluar gambarnya. 

Esok harinya aku mencetak foto-foto pemakaman Pak Widi melalui komputerku. ``Ya Tuhan!!!'', seruku seorang diri. Foto jenazah Pak Widi di liang makam tercetak. Tampak samar-samar seperti  ada 2 orang malaikat kecil disamping kepala Pak Widi dengan penuh suka cita.  Kuambil semua hasil cetak  foto-foto itu lalu kumasukkan ke dalam tas punggungku. Bergegas kupacu sepeda motorku menuju kerumah Jo.

`` Jo, coba kamu lihat foto ini.'', ujarku sambil menyodorkan foto jenazah Pak Widi di liang makam. 

``Ada apa Gung?  Ini kan foto jenazah bapak di liang makam.'', kata Jo keheranan. ``Coba kamu lihat baik-baik foto itu. Di samping kepala bapak tampak samar-samar ada 2 malaikat kecil'', kataku sambil menunjukkan letaknya. Lalu aku ceritakan kejadian pengambilan foto itu terasa sangat aneh karena aku merasa HP ku tidak rusak.

Waktu terus berputar, 40 hari berlalu setelah Pak Widi berpulang kehadirat Tuhan. Jo mengajakku mengunjungi makam Pak Widi bapaknya. Sesampai di pemakaman  ada seorang pemuda sedang berdoa di sebuah makam tepat di samping makam Pak Widi. Kami berdua berjalan menuju makam Pak Widi dan menyapa pemuda itu.

``Selamat pagi Mas.'' Sapaku ramah. 

``Selamat pagi pak.'' kata pemuda itu. Kami lalu saling berkenalan dan saling bercerita tentang orang-orang yang dikasihi telah menghadap sang khalik.

``Bapak seorang katolik? Boleh saya turut mendoakan arwah beliau?'' tanya pemuda itu ternyata melihat kalung salib bercorpus yang dipakai Jo.

``Benar Romo. Terima kasih Romo berkenan mendoakan arwah Pak Widi.'', ujarku cepat tanggap karena aku melihat tanda salib pada krah baju yang dipakai pemuda itu. 

\begin{quote}
\textit{
Allah Bapa pemilik bumi dan Surga,\\
Engkau telah memanggil salah seorang putra-Mu, Bapak Widiatmoko,
 agar duduk di sisi-Mu, bersama para kudus. \\
Bersanding dengan Putra-Mu terkasih,\\ Tuhan Yesus Kristus.\\
 Ampunilah segala dosa dan kesalahannya, bebaskan dia dari hukuman dan denda surgawi, sebab dia sudah mengimani Putra-Mu sebagai Juru selamat sejati.\\
 Bunda Maria, Bunda Segala Bangsa, hantarkan putramu ini untuk menghadap Putramu, Tuhan Yesus Kristus, yang bersama Allah Bapa dan Roh Kudus, hidup dan bertahta, kini  dan sepanjang segala abad. \\Amin.}
\end{quote}

\begin{flushright}
\textit{Medio Oktober 2011 \\Bravo Sierra}
\end{flushright}
\normalsize