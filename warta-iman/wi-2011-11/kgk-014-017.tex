\newpage
\chap{Kompendium Katekese Gereja Katolik}
\setcounter{kgkcounter}{13}
{\normalsize

\kgk{Apa hubungan antara Tradisi dan Kitab Suci?}
    Tradisi dan Kitab Suci berhubungan erat dan saling melengkapi. Masing-           
masing menghadirkan misteri Kristus dan berbuah di dalam Gereja. Kedua hal ini       
mengalir dari satu sumber ilahi yang sama, dan bersama-sama membentuk khazanah
iman yang suci dan dari sinilah Gereja mendapatkan kepastian tentang wahyu.

\kgk{Kepada siapa iman ini dipercayakan?}
Para Rasul mempercayakan khazanah iman ini kepada seluruh Gereja.
Berkat makna iman yang adikodrati inilah umat Allah secara keseluruhan, dengan
bimbingan Roh Kudus dan dituntun oleh Kuasa Mengajar Gereja, tidak pernah
berhenti untuk menerima, meresapkan lebih dalam, dan menghayati anugerah
wahyu ilahi ini secara lebih penuh.

\kgk{Kepada siapa diberikan tugas untuk menafsirkan khazanah iman ini
secara autentik?}
Tugas untuk memberikan tafsir autentik terhadap khazanah iman ini diper-
cayakan kepada otoritas Kuasa Mengajar Gereja, yaitu pengganti Petrus, Uskup
Roma, dan para Uskup yang ada dalam kesatuan dengannya. Dalam pelayanan
Sabda Allah, pengajaran resmi ini mempunyai karisma kebenaran, dan kepadanya
juga diberi tugas untuk merumuskan dogma yang merupakan rumusan kebenaran
yang terdapat dalam wahyu Ilahi. Otoritas pengajaran resmi ini juga diperluas pada
kebenaran-kebenaran lain yang mempunyai hubungan erat dengan wahyu.

\kgk{Apa hubungan antara Kitab Suci, Tradisi, dan Kuasa Mengajar?}
Kitab Suci, Tradisi, dan Kuasa Mengajar berhubungan erat satu sama lain
sedemikian sehingga yang satu tidak dapat ada tanpa yang lain. Dengan bekerja
sama, masing-masing dengan caranya sendiri, ketiga hal tersebut memberikan
sumbangan secara efektif bagi keselamatan jiwa-jiwa di bawah naungan karya
Roh Kudus.

\flushright{(\dots \emph{bersambung} \dots)}
}