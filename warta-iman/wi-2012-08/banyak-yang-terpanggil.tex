\chap{Banyak yang Terpanggil, Tetapi Sedikit yang Terpilih}

“Dari 44 orang seangkatan yang masuk ke Seminari, akhirnya cuma 6 orang yang ditahbiskan menjadi Imam”.
Sedih dan prihatin mendengar khotbah seorang romo berkaitan dengan jumlah panggilan menjadi imam dewasa ini.
Jujur saja, agak ngeri bila membayangkan suatu saat nanti gereja kita benar2 kehabisan gembala / imam. Bayangkan saja, penambahan jumlah imam yang se ordo dengan imam tersebut, mungkin baru akan terrealisir sekitar 3 atau 4 tahun mendatang. Karena memang dalam kurun waktu 2 sampai dengan 3 tahun mendatang dipastikan tidak akan ada tahbisan imam baru dari ordo tersebut.
Berbagai upaya untuk mengantisipasi hal itu, memang sudah dari dulu di lakukan oleh gereja, baik melalui promosi2 ke beberapa sekolah maupun gereja, juga melalui misa2 dengan tema ‘panggilan’. Namun upaya tersebut ternyata masih belum cukup mampu menjawab tantangan yang dihadapi gereja saat ini, khususnya dalam mendapatkan ‘panggilan-panggilan’ baru.

Kondisi semacam itu sepertinya sudah bisa ditebak bakal terjadi bila melihat realitas yang ada diseputar gereja. Gereja bagaikan berjalan “sendirian” dalam menjala umat yang bersedia mengabdi sebagai imam. Umat terkesan tidak begitu perduli dengan semakin minimnya jumlah “panggilan-panggilan baru”.
Dalam berbagai doa bersama, memang tidak jarang beberapa umat menyuarakan doanya dgn nyaring dan lantang memohon agar jumlah ‘panggilan’ menjadi semakin bertambah. Sekalipun jauh didalam hati kecilnya doa itu sering kali diakhiri dengan kata2 dalam hati, ……”ASAL JANGAN ANAKKU, YA TUHAN”

Masalah krisis imam, bukan saja masalah gereja, tapi adalah masalah kita semua sebagai umat Katolik. Karena pada akhirnya kita jualah yang akan merasakan akibatnya.
Bertitik tolak dari hal diatas, adalah sudah selayaknya kita mulai meningkatkan kepedulian tentang krisis Imam yang sedang melanda gereja kita. Kita mesti lebih meningkatkan dukungan terhadap upaya-upaya gereja dalam menebar benih2 panggilan diantara umat. Kita harus selalu menanamkan SEMANGAT PELAYANAN dengan contoh2 yang nyata, baik dikalangan umat maupun didalam keluarga kita sendiri. Sebab bagaimana mungkin kita mengharapkan mereka bisa berkembang menjadi pribadi2 yang secara total mau menyerahkan seluruh hidupnya menjadi pelayan demi kerajaan Allah, kalau didalam diri mereka tidak pernah ditanamkan SEMANGAT UNTUK MELAYANI. Jangan matikan benih panggilan yang muncul dalam keluarga kita. Justru mesti dipupuk dan dibina agar benih panggilan tersebut tumbuh dengan subur.
Benih ‘panggilan’ seringkali datang dalam keadaan yang samar2. Dan justru di seminarilah para seminaris, hari demi hari hari, bulan demi bulan, tahun demi tahun, mencoba menghayati, memahami, memupuk, memperjelas serta memantapkan panggilannya.
Sebagai umat Katolik, kita mesti meng imani bahwa benih panggilan adalah sebuah karunia Allah yang harus disyukuri. Betapa bahagia dan bangga menjadi sebuah keluarga yang dipilih oleh Allah untuk menerima rahmat adikodrati, yang belum tentu diterimakan kepada keluarga lain, sekalipun mereka memintanya.

Sebagai ‘gereja’ yang sudah dewasa, sudah saatnya kita juga bersikap dewasa dalam iman. Kita harus dukung para imam kita dengan doa dan usaha, bahkan kalau perlu dgn airmata dalam upayanya mempertahankan dan semakin memantabkan ‘panggilan’nya. Jangan biarkan mereka bekerja sendirian sehingga mereka tidak merasa ditinggalkan oleh umatnya. Hentikan rumor2 yang tidak ada dasarnya, berita2 miring yang belum tentu kebenarannya ataupun kritik2 yang tidak membangun/asal2an terhadap para imam kita. Karena hal2 spt itu sama sekali tidak ada manfaatnya bagi kehidupan dan perkembangan gereja.
Seorang imam, bukanlah malaikat, tapi tidak lebih dari seorang manusia biasa juga, yang bisa salah, bisa marah, bisa kecewa, bisa sedih, bisa gagal, dll. Dukungan moril, tenaga dan doa dari umat, tidak saja akan membantu mereka mampu bertahan serta mengatasinya bila hal2 diatas menimpanya, melainkan justru dapat lebih meningkatkan kinerjanya dengan penuh suka cita sebagai seorang gembala yang baik. Sehingga sekalipun mereka hidup dalam KESENDIRIANnya, mereka tetap mampu memberikan “kasih” dan “senyuman” kepada setiap orang.
Mari kita senantiasa berdoa bagi pahlawan2 iman ini agar tetap setia pada panggilannya sampai karya Allah tergenapi di dunia ini.
Sekian dan Tuhan memberkati

\sumber{A. Eddy Irwanto}