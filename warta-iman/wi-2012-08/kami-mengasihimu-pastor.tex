\chap{Kami mengasihimu, pastor!}

\section*{Pendahuluan}

Ada suatu percakapan antara dua orang ibu, Tina dan Suti. Tina bertanya kepada Suti “\textit{Anak kamu kalau sudah besar ingin menjadi apa?}” Suti menjawab, "\textit{Anakku ingin menjadi dokter bedah. Bagaimana dengan anak kamu yang selalu juara, Tina?}” Kemudian Tina menjawab “Anakku ingin menjadi pastor.” Suti terdiam, dan perlahan-lahan berkata “\textit{Ehm \ldots sayang juga ya, pintar-pintar kok mau jadi pastor.}”

Disinikah kita melihat, seolah-olah kalau yang bagus dan baik, jangan menjadi pastor. Padahal kita melihat di Alkitab bahwa hanya yang terbaik sajalah yang dipersembahkan kepada Allah. Kita melihat bagaimana pemilihan kurban bakaran selalu memilih kurban yang terbaik (Im 14:10). Minyak yang dipakai di bait Allah, juga minyak yang terbaik (Bil 18:12). Hanya yang terbaiklah yang dapat kita persembahkan kepada Tuhan, termasuk imam.

Kalau kita renungkan, kita dapat mengatakan bahwa keberhasilan suatu paroki dalam membina umatnya dapat diukur dari berapa banyak kaum muda yang menjawab panggilan menjadi pastor dari paroki yang bersangkutan. Semakin baik kehidupan spiritual paroki tersebut, maka akan semakin banyak kaum muda yang terpanggil menjadi pastor, karena keinginan untuk menjadi pastor dimulai dari keluarga dan juga dari lingkungan gereja. Jadi hal pertama yang perlu kita renungkan adalah: \textbf{berapakah yang menjadi pastor dari parokiku?} Kalau jawabannya tidak ada, maka perlu dipikirkan bagaimana untuk menggalakkan panggilan, sehingga putera-puteri yang terbaik dari paroki masing-masing dapat menjadi pastor atau suster.

\section*{Bukan engkau yang memilih-Ku, namun Aku yang memilihmu}

Namun yang terbaik menurut ukuran kita, bukanlah yang terbaik untuk ukuran Tuhan. Jadi, kalau mau ditanya siapa yang layak untuk menjadi pastor? Jawabnya adalah “tidak ada yang layak.” Namun, di tengah ketidaklayakan inilah, Tuhan memilih mereka, sama seperti Tuhan memilih Daud (1 Sam 16:6-13). Nabi Samuel berfikir dan ingin mengambil keputusan berdasarkan penilaian panca indera. Namun dikatakan bahwa Tuhan melihat hati. Dan karena inilah, Tuhan memilih Daud, seorang yang berkenan di hati-Nya(1 Sam 13:14).

\section*{Imam dengan segala suka dukanya}

Mungkin kita sering melihat ada beberapa imam yang sudah ditahbiskan yang tidak memberikan contoh yang baik kepada umatnya. Dan kita sering mengatakan bahwa mereka juga manusia biasa, yang tidak lepas dari dosa. Hal ini memang ada benarnya, sama seperti para rasul yang dipilih oleh Yesus sendiri, mereka juga manusia biasa, sederhana, namun dipilih oleh Tuhan secara khusus menjadi rasul. Apakah semua rasul-Nya setia? Tidak, karena Yudas terbukti menghianati Yesus. Demikian juga dengan para imam jabatan (pastor), yang dipilih oleh Tuhan secara khusus, ada dari mereka yang karena kelemahan mereka tidak dapat berpartisipasi secara baik dalam imamat Kristus. Namun apakah semuanya atau banyak imam yang demikian? Tentu tidak! Bahkan kita dapat melihat betapa banyak imam yang hidupnya tulus dan sungguh menjadi cerminan kasih Kristus. Mereka adalah tanda kehadiran Kristus yang menyertai Gereja-Nya dan pengorbanan mereka sungguh menjadi teladan bagi kita untuk juga mau berkorban dan menyerahkan diri kepada sesama.

Mari kita renungkan sejenak, kalau melihat takaran dunia, apa yang menarik dari kehidupan para imam? Mereka tidak boleh menikah, diberi uang saku ala kadarnya, harus menuruti perintah atasan. Kalau mereka sudah hidup baik dan menyesuaikan diri dengan orang-orang di parokinya, maka atasannya memindahkan mereka, bahkan kadang ke tempat yang terpencil, yang tidak ada listrik dan transportasi yang mencukupi. Kalau mereka tinggal di dalam komunitas, mereka tidak dapat memilih teman satu rumahnya, sedangkan kita minimal masih dapat memilih teman hidup. Kalau mereka jarang ngobrol dengan umat, dibilang pastornya menjaga jarak, namun kalau pastornya akrab dengan umat, dibilang bahwa pastornya cari perhatian atau malah digosipin dekat dengan seseorang. Kalau ada yang sakit parah, maka pastor harus bergegas memberikan sakramen perminyakan orang sakit, tidak peduli jam berapa. Bukankah serba susah untuk menjadi pastor? Kadang saya pikir-pikir, pelayanan ini jauh lebih sulit daripada orang yang bekerja di kantor. Apa rahasianya, sehingga mereka dapat melakukan hal yang demikian? Ya, karena rahmat dan kasih Allah! Dan memang, tanpa mengandalkan rahmat kasih Allah itu, sungguh sangat sulit untuk menjadi pastor. Tetapi bersama Allah, lihatlah, betapa indah dan ajaibnya buah hasil kerja mereka: banyak orang dapat menyadari akan kehadiran Allah yang hidup. Banyak orang tergerak untuk mengenal dan mengasihi Allah, yang mengantar mereka kepada keselamatan kekal! Manusia biasa tak akan sanggup melakukan hal ini, sebab urusan mengubah hati itu hanya pekerjaan Tuhan, namun berbahagialah para pastor yang dipakai Allah untuk menjadi alat-Nya yang istimewa untuk pekerjaan Tuhan ini.

Syukur kepada Tuhan, di tengah-tengah tantangan yang besar ini, banyak kaum muda yang menjawab panggilan Tuhan ini dengan besar hati. Memang, para imam hanya dapat bertahan kalau mereka benar-benar menyadari akan hakikat mereka sebagai orang-orang pilihan Tuhan. Sama seperti sakramen perkawinan yang hanya dapat bertahan jika suami istri mempunyai komunikasi yang baik, demikian juga dengan Sakramen Imamat, para pastor akan bertahan dalam berkat imamatnya, kalau mereka mereka mempunyai komunikasi yang baik dengan Tuhan. Tanpa bertekun dalam doa, dan berani memberikan dirinya untuk orang lain, maka pastor tidak akan dapat memenuhi pelayanannya sesuai dengan yang Tuhan percayakan kepada mereka.

\section*{Imam bersama dan Imam tertahbis}

Mari sekarang kita melihat hakikat dari Sakramen Imamat. Seperti yang kita tahu, bahwa dengan Sakramen Baptis, kita semua menjadi imam, nabi, dan raja. Walaupun panggilan sebagai imam belaku untuk semua yang sudah dibaptis, namun Tuhan menunjuk orang-orang pilihan-Nya untuk menjadi \textbf{imam tertahbis} (imamat jabatan).

\subsection*{Dalam Perjanjian Lama}

Kel 19:5-6, menyatakan bahwa Tuhan memerintahkan Musa untuk memberitahukan kepada seluruh umat Israel, bahwa kalau mereka berpegang pada perintah Tuhan, mereka akan menjadi umat kesayangan, kerajaan imam dan bangsa yang kudus. Di samping mengangkat Israel sebagai kerajaan imam, Perjanjian Lama juga mengatakan bahwa suku Lewi dipersiapkan secara khusus sebagai imam (Bil 3:5-13). Secara umum memang bangsa Israel dipersiapkan Tuhan menjadi imam dan bangsa yang kudus, namun secara khusus, Tuhan juga menunjuk suku Lewi untuk menjadi imam dan menjalankan tugas yang berhubungan dengan korban dan persembahan. Suku Lewi yang ditunjuk secara khusus oleh Tuhan untuk menjadi imam (imamat jabatan) melayani umat yang lain atau imam secara umum (imamat bersama). Hal yang sama dapat diterapkan di dalam ajaran Gereja Katolik. Gereja Katolik mengenal adanya dua imamat: (1) Imamat jabatan dan (2) imamat bersama. Dimana imamat jabatan melayani imamat bersama.[2]

\subsection*{Dalam Perjanjian Baru}

Yesus tidak pernah melarang perantaraan imam sejauh hal tersebut berpartisipasi dalam karya keselamatan Yesus. Pada saat Yesus menyembuhkan sepuluh orang kusta, Yesus menyuruh mereka untuk memperlihatkan diri mereka kepada para imam (Luk 17:12-14) agar para imam dapat menyatakan mereka tahir. Rasul Petrus menegaskan bahwa semua umat Allah adalah bangsa terpilih, imamat yang rajani, bangsa yang kudus, kepunyaan Allah sendiri (1 Pet 2:9). Pemilihan bangsa Israel sebagai bangsa pilihan untuk mendatangkan keselamatan pada bangsa-bangsa lain membuktikan bahwa Tuhan menggunakan ‘perantara’ untuk melaksanakan rencana-Nya.

\section*{Yesus adalah Imam yang sejati dan selamanya}

Dari konsep imam di Perjanjian Lama dan Perjanjian baru, kita melihat bahwa imam di Perjanjian Lama adalah merupakan persiapan untuk imam yang lebih sempurna di Perjanjian Baru, yang dipenuhi di dalam diri Kristus, imam menurut peraturan Melkisedek, yang sempurna, yang berlangsung untuk selamanya (Ibr 5:1-10). Dengan pengorbanan Kristus di kayu salib, maka Yesus telah melengkapi semuanya, baik sebelum kedatangan-Nya, pada waktu kedatangan-Nya, dan setelah kedatangan-Nya. Pertanyaannya, apakah dengan mengakui Yesus sebagai Imam satu-satunya menutup kemungkinan adanya imam yang lain? Sama seperti penjelasan tentang imam jabatan di dalam Perjanjian Baru dan Perjanjian lama, maka \textbf{imam-imam jabatan hanyalah menjadikan Imam yang abadi, yaitu Kristus untuk hadir kembali tanpa menghilangkan keunikan imamat Kristus. Imam-imam yang ditahbiskan hanyalah menjadi pelayan dari Imam satu-satunya, yaitu Kristus.}

Kita dapat melihat bahwa Yesus adalah satu-satunya perantara antara Allah dengan manusia. Jadi doa-doa yang kita panjatkan didaraskan dalam nama Yesus. Namun bukankah kita sering meminta seseorang yang kita anggap punya hubungan baik dengan Tuhan untuk juga mendoakan permasalahan kita? Apakah dengan demikian kita menghilangkan keagungan Yesus yang menjadi satu-satunya Perantara kita dengan Allah? Tentu saja tidak, karena semua orang yang mendoakan kita turut berpartisipasi dalam karya keselamatan Kristus. \textbf{Jadi sampai tahap ini, kita menyetujui bahwa imam yang tertahbis tidak menghilangkan keagungan dan kebenaran bahwa Kristus adalah satu-satunya Imam Agung.}

Dan berdasarkan pembuktian di atas dalam Perjanjian Baru dan Perjanjian Lama, kita tahu bahwa Tuhan telah menjadikan seluruh umat Allah menjadi imam. Namun \textbf{Yesus menunjuk secara khusus imam yang ditahbiskan untuk melanjutkan karya-Nya di dunia ini sampai akhir jaman, dan juga untuk melayani imam bersama.}

\section*{Apakah Sakramen Imamat?}

\subsection*{Imamat jabatan}

Katekismus, 1536 menyebutkan bahwa Tahbisan adalah suatu Sakramen, \textbf{di mana perutusan yang dipercayakan Kristus kepada Rasul-rasul-Nya, dilanjutkan sampai akhir zaman}. Dari sini jelas, bahwa konsep Gereja bukan hanya tak kelihatan, namun juga kelihatan. Namun demikian, hal yang tak terlihat (spiritual) dan terlihat, tak dapat dipisahkan, sama seperti tubuh dan jiwa manusia tak terpisahkan. Dalam hal struktur yang terlihat (Sakramen) ini, terdapat bagian, yaitu: \textbf{episkopat, presbiterat, dan diakonat.}

Episkopat atau uskup adalah penerus dari para rasul, yang diutus oleh Yesus sendiri. Sama seperti Yesus menunjuk rasul Petrus, sebagai kepala dari para murid, maka uskup Roma menjadi penerus dari rasul Petrus menjadi kepala dari seluruh Gereja di seluruh dunia. Sedang para uskup adalah pemimpin dari gereja lokal, yang dibentuk menurut gambaran Gereja universal (semesta) dan dalam kesatuan dengan Bapa Paus. Dan pembantu dari uskup adalah para pastor (presbiterat), yang biasanya membawahi paroki. Sedangkan diakonat diperbantukan untuk membantu pelayanan para pastor dan uskup. Diakonat dan presbiterat ditahbiskan oleh uskup, sedangkan uskup ditahbiskan oleh para uskup yang lain dengan persetujuan dari uskup Roma, atau Paus. \textbf{Dari sinilah, kita melihat bahwa seluruh tahbisan di Gereja Katolik saling terkait dan dapat ditelusuri sampai kepada para Rasul, yang diutus oleh Yesus sendiri}. Karena inilah maka salah satu ciri Gereja Katolik adalah apostolik (berasal dari para Rasul).

\subsection*{Imamat bersama}

\textbf{Pada saat kita dibaptis, maka kita juga dipanggil untuk menjadi imam, nabi dan raja.} Melalui baptisan, kita juga harus menjadi imam, yaitu dengan menjadi saksi Kristus yang baik, hidup menurut iman, pengharapan, dan kasih. Melalui kesaksian hidup kita, maka kita akan menjadi pancaran terang kasih Kristus, sehingga secara tidak langsung, kita berpartisipasi untuk membawa umat yang lain kepada Sang Imam Agung, yaitu Kristus. Dan cara kedua untuk menjalankan imamat bersama adalah dengan mengikuti perayaan Ekaristi. Dimana dengan persiapan, dan partisipasi aktif, kita semua menyatukan persembahan kita, suka duka kita, dan kehidupan kita dalam kurban Ekaristi.

\section*{Imam Jabatan dan Imam bersama berjalan berdampingan untuk membangun Gereja}

Yang menjadi masalah adalah kalau ada orang yang mencoba mencampuradukkan kedua jenis imamat ini. Imam jabatan semakin ingin menyatu dengan umat, dan mengaburkan identitasnya. Dan imam bersama begitu antusias untuk berpartisipasi di pelayanan, sehingga juga mengerjakan pekerjaan-pekerjaan imam jabatan. Padahal, baik imam jabatan maupun imam bersama mempunyai identitas sendiri-sendiri dan keduanya mempunyai tujuan untuk membangun jemaat Allah. Imam bersama terjun ke masyarakat dan menjadi garam di tengah-tengah masyarakat yang nilai-nilainya belum tentu sesuai dengan nilai-nilai kristiani. Sedangkan imam jabatan membangun dan melayani imam bersama, sehingga imam bersama akan semakin dikuatkan untuk menjadi saksi yang hidup di tengah masyarakat.

\section*{Imam Jabatan yang melanjutkan tiga misi Kristus: Imam, Raja, dan Nabi}

\textbf{Sebagai Imam}, para imam melanjutkan karya Kristus dengan merayakan sakramen dan memimpin umat di dalam liturgi, terutama di dalam liturgi Ekaristi. Di sinilah peran imam menjadi begitu jelas, yang mewakili Kristus (persona Christi) untuk menghadirkan kembali kurban Paskah Kristus. Mereka memberikan sakramen Baptis, Penguatan, Pengakuan Dosa, Sakramen Perminyakan, dan memberikan penguburan kepada yang meninggal. Dalam kesehariannya, mereka juga berdoa brevier, doa yang menjadi doa Gereja.

\textbf{Sebagai Nabi}, para imam melaksanakannya dengan berkotbah, mengajar di sekolah atau persiapan Pembaptisan. Secara prinsip seorang pastor harus menyampaikan kebenaran, yang bersumber pada Kitab Suci, Tradisi Suci, dan Magisterium. Imam yang menyampaikan ajaran yang dianggapnya benar namun tidak berdasarkan tiga pilar kebenaran di atas, sebenarnya tidak menjalankan fungsinya sebagai nabi. Karena sebagai nabi, dia hanyalah meneruskan Kebenaran kepada umat, bukan mengarang kebenaran, berdasarkan pendapat pribadi, atau berdasarkan suara umat, karena kebenaran tidak tergantung dari suara terbanyak.

\textbf{Sebagai Raja}, para pastor melaksanakannya dengan pelayanannya di bidang kepemimpinan umat, baik di paroki atau komunitas yang dipercayakan kepada mereka. Mereka bekerja sama dengan dewan paroki, sehingga kegiatan paroki dapat berjalan dengan baik.

\section*{Uskup, Imam, dan diakon menurut Bapa Gereja}

Mungkin ada yang berfikir bahwa uskup, imam dan diakon hanyalah karangan Gereja Katolik semata. Namun kalau kita melihat sejarah dan belajar dari para bapa Gereja, maka kita akan melihat, bahwa sebenarnya semua itu berakar pada tradisi.

\begin{enumerate}
\item \textbf{Ignasius dari Antiokia (AD. 110)}: “Sekarang, sungguh merupakan kehormatan bagiku untuk bertatap muka denganmu secara pribadi dengan uskup (bishop) yang diberkati Tuhan, Damas; dan juga bertemu secara pribadi dengan para imam (presbyters), Bassus dan Apollonius l dan teman satu pelayanan, diakon, Zotion. ..” (Letter to the Magnesians 2).
\item \textbf{Ignasius dari Antiokia (AD. 110)}: “Perhatikanlah untuk melakukan segala sesuatu selaras dengan Tuhan, dengan uskup menempati posisi Tuhan dan dengan para imam di posisi para murid, dan dengan para diakon, yang paling dekat denganku, yang dipercaya dengan pelayanan Kristus, yang berasal dengan Allah Bapa sejak permulaan dan yang pada akhirnya telah dinyatakan (ibid, 6:1).
\item \textbf{Ignasius dari Antiokia (AD. 110)}: “Perhatikanlah, untuk menyelaraskan diri dalam perintah-perintah Allah dan para murid, sehingga di dalam setiap perbuatanmu, kamu dapat berhasil, baik di dalam tubuh maupun jiwa, baik dalam iman dan kasih, dalam Putera, Bapa dan dalam Roh Kudus, di awal dan di akhir, bersama dengan yang terhormat uskupmu; dan dengan para imam, yang terjalin secara baik dalam mahkota rohani; dan dengan para diakon, putera-putera Allah. Tunduklah kepada uskup dan satu sama lain seperti Yesus Kristus taat kepada Allah Bapa, dan para murid kepada Kristus dan Allah Bapa \ldots (ibid, 13:1-2).
\item \textbf{Ignasius dari Antiokia (AD. 110)}: “Memang, engkau harus tunduk kepada uskup seperti engkau tunduk kepada Yesus Kristus, sudahlah jelas bagiku bahwa engkau hidup bukan dengan cara manusia namun di dalam Kristus, yang telah wafat untuk kita… Dengan demikian adalah penting …bahwa engkau tidak melakukan segala sesuatu tanpa persetujuan uskup, dan engkau juga harus tunduk kepada para imam, seperti kepada para murid Kristus…Adalah penting juga bahwa para diakon, pemberi sakramen-sakramen Kristus, dalam segala sesuatu menyenangkan semua orang. .. (Letter to the Trallians 2:1-3).
\item \textbf{Ignasius dari Antiokia (AD. 110)}: ” Setiap orang yang menghormati para diakon menghormati Kristus, dan juga yang menghormati uskup menghormati Allah Bapa, dan menghormati para imam sebagai perwakilan Allah dan persekutuan para murid. Tanpa hal tersebut, maka tidak dapat dikatakan sebuah Gereja. (ibid., 3:1-2).
\item \textbf{Ignasius dari Antiokia (AD. 110)}: “Dia yang ada di dalam tempat kudus adalah kudus; tetapi dia yang berada di luar tempat kudus adalah tidak kudus. Dengan kata lain, seseorang yang bertindak tanpa uskup dan imam dan para diakon tidak mempunyai hati nurani yang bersih” (ibid., 7:2).
\item \textbf{Ignasius dari Antiokia (AD. 110)}: “…saya berbicara dengan suara yang keras, suara dari Tuhan: “Perhatikanlah uskup dan imam dan para diakon“. Sebagian orang mengira bahwa saya mengatakan hal ini karena saya tahu adanya perpecahan di antara beberapa orang; namun Dia, yang menjadi alasan mengapa saya dirantai, menjadi saksi bahwa saya tidak mengetahuinya dari manusia; melainkan dari Roh yang membuatku mengatakan hal ini, ‘Jangan melakukan sesuatu tanpa uskup, jagalah badanmu sebagai bait Allah, cintailah persatuan, jauhkanlah dari perpecahan, turutilah Kristus, seperti Dia telah menuruti Allah Bapa” (Letter to the Philadelphians 7:1-2).
\item \textbf{Clemens dari Alexandria (AD. 191)}: “Banyak nasehat-nasehat untuk orang-orang tertentu telah ditulis di dalam Kitab Suci: sebagian untuk para imam, sebagian untuk para uskup dan para diakon; … (The Instructor of Children 3:12:97:2).
\item \textbf{Clemens dari Alexandria (AD. 208)}: “Di dalam Gereja, gradasi dari para uskup, para imam, dan para diakon terjadi sebagai suatu gambaran, menurut pendapatku, dari kemuliaan malaikat dan dimana susunan tersebut, seperti yang dikatakan di dalam Alkitab, menantikan orang-orang yang telah mengikuti langkah-langkah dari para murid dan yang telah hidup di dalam kepenuhan kebenaran menurut Kitab Suci” (Miscellanies 6:13:107:2).
\item \textbf{Origen (AD.234)}: “Tidak hanya perzinahan, namun juga perkawinan membuat kita tidak pantas untuk penghormatan ekklesiastikal; karena tidak juga seorang uskup, juga imam, atau seorang diakon, …” (Homilies on Luke, number 17).
\item \textbf{Konsili Elvira (AD. 300)}: “Para uskup, para imam, dan para diakon tidak dapat meninggalkan tempat mereka untuk keperluan dagang, dan mereka juga tidak dapat bepergian ke daerah-daerah, atau jual-beli untuk keuntungan mereka sendiri” (canon 18).
\item \textbf{Konsili Nicea I (AD. 325)}: “Sinode yang kudus dan besar telah mengetahui bahwa di beberapa daerah dan kota, para diakon memegang Ekaristi untuk para imam, dimana tidak ada dalam kanonik atau kebiasaan yang memperbolehkan bahwa mereka tidak mempunyai hak untuk memberikan Ekaristi atau Tubuh Kristus kepada mereka yang melakukan persembahan (dalam hal ini imam). Dan juga menjadi perhatian, bahwa beberapa diakon sekarang menyentuh Ekaristi sebelum para uskup. Biarlah praktek-praktek seperti itu harus dihentikan, dan biarlah para diakon bertindak sesuai dengan wewenangnya, mengetahui bahwa mereka adalah para pelayan uskup dan lebih rendah dari para imam, dan biarlah mereka menerima Ekaristi sesuai dengan urutan mereka, setelah para imam, dan baik uskup atau imam memberikannya kepada mereka.” (Canon 18).
\end{enumerate}

\section*{Kenapa imam tidak menikah}

Dari penjabaran di atas, kita melihat bahwa uskup, imam, dan diakon merupakan suatu tradisi dari jemaat awal yang terus berlaku sampai sekarang. Hal lain yang menjadi pertanyaan banyak orang adalah mengapa imam tidak diperbolehkan untuk menikah? Apakah ini hanya merupakan karangan Gereja Katolik semata? Mari kita melihat bukti-bukti bahwa kaul kemurnian mempunyai dasar yang kuat:

\begin{enumerate}
\item    Para rasul telah menjalankan kaul kemurnian sebelum penderitaan Yesus, seperti yang dikemukakan oleh St. Petrus “Kami ini telah meninggalkan segala sesuatu dan mengikut Engkau; jadi apakah yang akan kami peroleh?” (Mat 19:27). Dan Yesus menjawab “Dan setiap orang yang karena nama-Ku meninggalkan rumahnya, saudaranya laki-laki atau saudaranya perempuan, bapa atau ibunya, atau \textbf{istri} (\textit{istri termasuk dalam terjemahan Douay Rheims, Vulgate and King James Bible}) anak-anak atau ladangnya, akan menerima kembali seratus kali lipat dan akan memperoleh hidup yang kekal (Mat 19:29). Meninggalkan segalanya dan istri disini, ditafsirkan sebagai tindakan untuk tidak melakukan lagi hubungan badan. Kalau kita mempelajari riwayat Mahatma Gandhi, beliau juga pada umur tertentu tidak menggunakan haknya sebagai suami demi untuk mencapai tujuan yang lebih tinggi. Jadi, hal ini bukanlah sesuatu yang aneh.
\item Di dalam Gereja perdana, karena terbatasnya kandidat yang belum menikah untuk diakon, imam, dan uskup yang, maka mereka dapat menikah sebelum ditahbiskan (lih. 1 Tim 3:1-4), namun mereka dituntut untuk mempraktekkan kaul kemurnian setelah ordinasi.
\item Dokumen pertama yang menyatakan secara explisit tentang hal ini adalah \textbf{Konsili Elvira} di Spanyol tahun 306 dan Carthage tahun 390, serta dekrit dari Paus Siricius dan Innocent, sekitar akhir abad ke-4 dan awal abad ke-5. Semuanya itu menunjukkan bahwa hidup selibat setelah ordinasi bukanlah inovasi semata, namun merupakan hal yang telah dijalankan oleh para murid, bapa gereja, dan menjadi bagian dari tradisi. \textbf{Paus Siricius} mengatakan bahwa peraturan untuk hidup selibat dimaksudkan untuk memberikan segenap jiwa dan raga untuk Tuhan dalam kaul kesucian mulai dari hari ordinasi. \textbf{Konsili Carthage} menekankan hidup selibat untuk meneruskan ajaran dan praktek hidup selibat seperti yang telah dijalankan oleh para rasul.
\item Gereja Timur tidak lagi mempraktekan tradisi apostolik ini karena perubahan yang dilakukan di Konsili Trullo (sekitar abad ke-7), namun disebutkan bahwa hanya imam yang tidak menikah yang dapat ditahbiskan menjadi uskup, dan seorang iman tidak dapat menikah setelah dia ditahbiskan.
\item Yang menjadi motif dari Konsili Trullo adalah begitu banyak penyimpangan, seperti simoni, penyimpangan kehidupan seksual para imam, atau masih menggunakan hubungan suami-istri walaupun sudah ditahbiskan. Menanggapi hal itu, \textbf{Gereja Latin} dibawah kepemimpinan St. Gregory VII mengambil jalan untuk \textbf{menjalankan peraturan secara ketat}, sebaliknya Gereja Timur mengambil cara untuk memperlunak peraturan tersebut. Cara yang sungguh patut dipuji dari St. Gregory VII membuahkan hasil dengan meletakkan pondasi yang kokoh, sehingga membuat Gereja berkembang pesat di abad 12-13.
\item Alasan yang utama dari kaul ketaatan adalah seorang imam secara sakramental \textbf{mewakili Kristus sebagai mempelai pria dari Gereja}, sehingga tidaklah pantas bahwa dia sendiri mempunyai istri bagi dirinya sendiri.
\item Jalan yang ‘sulit’ yang ditempuh oleh Gereja Katolik menambahkan kepadanya “\textbf{motive of credibility}” sebagai Gereja yang sejati. Sebuah doktrin yang bertentangan dengan kecenderungan alami tidak dapat diharapkan untuk bertahan selama 2000 tahun tanpa bantuan dari yang Ilahi.
\end{enumerate}

Dari segi kepraktisan, kita dapat melihat bahwa dengan tidak menikah maka seorang imam dapat mencurahkan segenap hati, jiwa, dan pikirannya untuk melayani Tuhan dan sesama. Rasul Paulus sendiri memberikan nasehat ” Aku ingin, supaya kamu hidup tanpa kekuatiran. Orang yang tidak beristeri memusatkan perhatiannya pada perkara Tuhan, bagaimana Tuhan berkenan kepadanya” (1 Kor 7:32). Dengan hidup selibat, seorang imam hanya memikirkan apa yang terbaik bagi Tuhan dan umat yang dipercayakan kepadanya.

\section*{Siapakah yang berani menjawab panggilan suci ini?}

Pada kesempatan ini, saya ingin memberikan tantangan kepada kaum muda. Kalau engkau ingin memberikan dirimu secara khusus kepada Tuhan, mempunyai hati untuk melayani sesama, mengasihi Tuhan dan Gereja-Nya, pertimbangkanlah untuk menjadi imam. Menjadi imam adalah suatu berkat yang istimewa; sebab imam menjadi gambaran nyata atas kasih Kristus yang hidup bagi Gereja dan dunia ini. Yesus sendiri menjanjikan kelimpahan berkat bagi mereka yang menjawab panggilan-Nya ini, dan jika Yesus sendiri yang menjanjikannya, pasti Ia akan memenuhinya.

Saya juga ingin membagikan cerita tentang pemain sepakbola yang terkenal, yaitu Chase Hilgenbrinck. Dia yang sedang mempunyai karir yang hebat dan cerah, kemudian memutuskan untuk meninggalkan karirnya untuk menjadi pastor. Diperlukan suatu keberanian untuk menjawab panggilan Tuhan. Namun kita percaya bahwa berkat dari Tuhan tercurah dengan melimpah bagi orang yang mau menjawab panggilan-Nya. Siapakah yang mau menjadi Chase-chase yang lain? Siapakah yang mau menjawab seruan Tuhan “\textbf{Siapakah yang akan Kuutus, dan siapakah yang mau pergi untuk Aku?}” Siapa yang akan menjawab bersama nabi Yesaya “\textbf{Ini aku, utuslah aku!}” (Yes 6:8)

Dan saya juga ingin mengundang seluruh pembaca untuk bersama-sama berdoa setiap hari untuk kekudusan para imam. Sebutlah satu-persatu imam yang kita kenal, dan mintalah Bunda Maria menuntun para imam agar mereka dapat semakin menyerupai Putera-Nya. Biarlah para imam dapat menjadi imam yang kudus, sehingga mereka dapat menjadi pancaran kasih Kristus.

\sumber{Stefanus Tay\\
http://katolisitas.org}