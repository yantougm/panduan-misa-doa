\chap{Ordo-Ordo dalam Gereja Katolik}

Ketika seseorang masuk seminari dan telah menetapkan pilihannya untuk menjadi seorang romo, maka dia diwajibkan untuk memilih ordo sesuai dengan spritualitas pribadinya itu. Ketikk elak dia mewartakan karya penyelamatan ALLAH dia sudah memiliki semangat spiritualitasnya sendiri. Seperti seorang yang berbakat seni, tentu dia akan lebih nyaman bila dia bersekolah dan bekerja sebagai seniman dibanding dokter bukan? oleh karena itu, ordo tersebut dimaksudkan sebagai pilihan jalan. Yang mana kesemuanya itu sebenarnya sesuai dengan jalan Kristus sendiri. Baik ordo yang bersifat misioner, kontemplatif dan lain-lain, masing-masing memuliakan Tuhan dengan cara dan jalannya sendiri-sendiri.

\section*{Pengertian / Latar Belakang}

Tarekat atau ordo adalah satu kongregrasi dalam Gereja Katolik Roma dimana para anggotanya hanya terdiri dari rohaniwan dan rohaniwati, baik imam, maupun biarawan dan biarawati. Para anggotanya mengikrarkan kaul, baik sementara maupun kekal: selibat adalah yang terutama, kemudian dan ketaatan, baik terhadap atasan mereka Yang biasanya disebut Jendral Overste, maupun kepada Uskup, Kardinal, dan Paus sebagai otoritas Gereja Katolik Roma. Mereka hidup dalam komunitas sosial sesuai dengan tata-cara dan konstitusi masing-masing kongregasi, yang telah disetujui oleh otoritas Gereja Katolik. Selain itu ada juga institusi sekuler (kaum awam) yang memiliki kongregasi yang terpisah.

Sasaran yang ingin dicapai maupun cara-cara untuk mencapainya dari masing-masing kongregasi, dinyatakan dalam peraturan dan konstitusi masing-masing kongregasi yang bersangkutan. Suatu kongregasi religius lokal yang berada dalam batas-batas suatu ke uskupan, dimana peraturan dan konstitusinya disetujui oleh uskup setempat. Kongregasi-kongregasi yang tersebar di berbagai keuskupan atau bersifat multinasional, peraturan dan konstitusinya memerlukan persetujuan oleh otoritas tertinggi gereja dari Vatikan atau yang biasa dikenal dengan istilah Tahta Suci. Yurisdiksi umum atas segenap kongregasi religius berada di tangan Kongregasi itu sendiri dibawah pengawasan otoritas Vatikan. Aspek hukum yang menyangkut semua konggregasi religius tercantum dalam Kitab Hukum Gereja (Iuris Codex Canonici) atau Canon nomor 573 sampai dengan 709 dalam Buku 2, Bagian 3, dari Kitab Hukum (Kanon) Gereja.

Ordo dalam Gereja katolik sangat banyak, jadi mungkin agak susah jelasin beda2 nya satu2. kelima ordo yang terbesar adalah Benedictines (OSB), Franciscans (OFM) Dominicans (OP) and Jesuits (SJ) and Agustinians (OSA). ordo di dirikan oleh para imam, uskup dalam gereja katolik, dan umumnya mereka itu sudah diangkat jadi santo dan santa. Setiap ordo punya visi dan misi yang berbeda sesuai dengan spiritualitas pendirinya. dan Misalnya ordo Jesuit lebih mengarah pada pengejaran intektualitas. kalau Dominican lebih pada preaching, Fransiscan untuk pelayanan orang miskin sesuai dengan spritualitas St. Fransiskus Assisi.

Semua ordo2 ini superior generalnya di Roma dan superior bertangung jawab mengurus jaringan/pelayanan ordo nya di seluruh dunia. kadang superior memerintah imam A di Belanda segera ke Papua, maka dengan tulus iklas akan di ikuti.

Tempat tugas pelayanan setiap ordo berbeda beda. Misalnya, Ordo OFM bertugas di jayapura, dan OSA di Sorong dan OSC di Agatst sedangkan MSC di Merauke. begitu juga untuk daerah2 lain. Kalau ada ordo kekurangan imam mereka di satu daerah, maka ordo tersebut bisa meminta bantuan imam dari ordo lain tapi harus ada ijin dari superior ordo setempat.

Ordo-ordo ini juga umumnya memiliki sekolah tinggi Filsafat dan teologi tersendiri. Tapi ada yang belum punya. Seperti di Papua, ordo OSC dan OSA belum mempunyai STFT. Studi mereka bergabung di STFT milik OFM di Jayapura, yaitu STFT Fajar Timur. Sedang MSC mempunyai STFT di Pineleng Manado.

Ordo Jesuit, (SJ) biasanya ordo menjadi pionir pertama di daerah misi, mencakup biarawan dan imam, tapi tidak ada biarawati
Ordo SVD, biasa juga ordo pionir di daerah misi
Ordo SCJ,
Ordo Carmelit (O Carm)
Ordo Fransiskan (OFM Cap)
Ordo Fransiskan (OFM Conv)
Ordo Benedictin (OSB), konsentrasi pada hidup rohani dan latihan yang ketat
Ordo Salib Suci (OSC), konsentrasi pada doa dan kontemplasi
Ordo keluarga kudus (MSF)
Ordo SSCC
Ordo CICM
Ordo CM
ordo MSC
Ordo Oblat Maria Immaculata (OMI)
Ordo Trapist, konsentrasi pada doa dan kontemplasi
Ordo Dominican (OP)
Ordo SDB
Ordo Xaverian (SX)
