\newpage
\chap{Kompendium Katekese Gereja Katolik}
\setcounter{kgkcounter}{49}
\small
\kgk{Apa artinya mengatakan bahwa Allah itu mahakuasa?}
      Allah mewahyukan diri-Nya sebagai ”Dia yang kuat, Dia yang kuasa” (Mzm     268-278
24:8), sebagai Dia ”yang bagi-Nya tidak ada yang mustahil” (Luk 1:37). Kemahakuasaan-Nya itu universal, gaib, dan yang menunjukkan Diri-Nya di dalam
penciptaan dunia dari ketiadaan dan penciptaan manusia dari cinta, tetapi ter-
utama menunjukkan Diri-Nya dalam Penjelmaan dan Kebangkitan Putra-Nya,
dalam anugerah pengangkatan anak dan dalam pengampunan dosa-dosa. Karena
hal inilah, Gereja mengalamatkan doa-doanya kepada ”Allah yang mahakuasa dan
kekal” (”Omnipotens sempiterne Deus ...”).

\kgk{Apa pentingnya mengatakan, ”Pada awal mula, Allah menciptakan
    langit dan bumi” (Kej 1:1)?}
     Maknanya, penciptaan itu dasar dari semua rencana penyelamatan Allah.       
Tindakan penciptaan itu menunjukkan kekuasaan dan cinta bijaksana Allah,        
merupakan langkah pertama menuju kepada perjanjian antara Allah dengan
umat-Nya. Penciptaan merupakan permulaan sejarah keselamatan yang memuncak dalam diri Kristus, dan jawaban pertama terhadap pertanyaan dasar kita mengenai asal dan tujuan akhir.

          \kgk{Siapa yang menciptakan dunia?}
Bapa, Putra, dan Roh Kudus adalah prinsip penciptaan yang satu dan tak
terpisahkan walaupun karya penciptaan dunia secara khusus dikenakan kepada
          Allah Bapa.

\kgk{Mengapa dunia diciptakan?}
Dunia diciptakan bagi kemuliaan Allah yang ingin menunjukkan dan
mengomunikasikan kebaikan, kebenaran, dan keindahan-Nya. Tujuan akhir penciptaan, Allah menjadi ”semua di dalam semua” dalam Diri Kristus (1Kor 15:28)
          untuk kemuliaan-Nya dan kebahagiaan kita.


\flushright{(\dots \emph{bersambung} \dots)}
\normalsize