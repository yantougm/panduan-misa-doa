\chap{Ordo dalam Gereja Katolik}

Tarekat atau ordo adalah satu kongregrasi dalam Gereja Katolik Roma dimana para anggotanya dari rohaniwan/i, yg terdiri dari biarawan/i di mana mereka mengikat diri dalam kaul (baik sementara maupaun kekal). Selibat dan ketaatan adalah yg paling utama, ketaatan baik terhadap atasan mereka (General Overste), maupun kepada Uskup dan Paus sebagai otoritas GK. Anggota ordo hidup dlm komunitas sosial sesuai dengan tata cara konstitusi dari masing2 kongregasi, atas persetujuan Magisterium Gereja Katolik.
Selain hal tersebut di atas, ada juga institusi sekuler (kaum awam) yang memiliki kongregasi terpisah. (kita pasti pernah mendengar Karmelit Awam...itu salah satu contohnya)

\section*{Ordo}
Komunitas (-komunitas) dengan kaul agung dalam arti luas dengan biara-biara independen (misalnya Benediktin, Trapis); atau dengan biara-biara yang tunduk kepada satu pembesar umum (misalnya Serikat Jesus, Ursulin uni Roma).

\section*{Konggregasi}
Komunitas-komunitas dengan kaul sederhana, tetap atau sementara di bawah satu pembesar umum (misal Serikat Sabda Allah - SVD, Suster Carolus Borromeus - CB)

\section*{Ordo/Konggregasi Klerikal}
Sebagian atau anggota-anggota ditahbiskan imam.
Awam: semua anggota tidak ditahbiskan imam.

\section*{Aneka Ordo Pria}
\textit{Ordo Rahib} yang anggota-anggotanya terikat pada satu biara dan menjalankan hidup kontemplatif (a.l. Ordo Benediktin dengan Ordo Sistersien dan Trapis, Ordo Kartusian).

\textit{Ordo Kanonik Reguler} menghimpun imam-imam yang membentuk dewan katedral atau dewan gereja yang hidup menurut aturan tentang dan terikat doa ofisi bersama (antara lain Ordo Kanonik S. Agustinus, Ordo Salib Suci)

\textit{Ordo Kesatria atau Hospital} melindungi milik apapun, maka dahulu mencari rejeki dengan mengemis (antara lain Ordo Fransiskan dan Klaris serta Kapusin, ordo Dominikan, Karmelit)

\textit{Ordo Kleris} dengan regula, yang melepaskan doa ofisi bersama, hidup dalam biara dan pakaian biarawan, supaya lebih bebas berkarya pada saja dan di mana pun diperlukan, baik secara sendiri maupun bersama (antara lain Serikat Jesus).

\section*{Ordo Wanita}
\noindent{Ordo Rubiah -- Benediktin (termasuk Trapis)\\
Ordo Kedua Pengemis -- Klaris, Kapusines, Karmelites\\
Ordo Biarawati aktif -- Ursulin}

\section*{Para Imam}

Dalam Gereja Katolik imam merupakan jabatan tetap,
melalui proses pendidikan  dan pemberkatan  (tahbisan,
penerimaan Sakramen Tahbisan). Yang bisa menjadi imam
hanyalah laki-laki.  Imam juga tidak
diperbolehkan untuk menikah (hidup selibat), seumur hidup.
Begitu seorang imam menikah, maka jabatan imamatnya harus
dilepaskan. 

Yang bisa memimpin ibadat misa dan memberikan sakramen
hanyalah imam yang sah. Dalam keadaan imam tidak
ada, maka ibadat misa ditiadakan dan diganti dengan ibadat
sabda yang bisa dipimpin oleh bukan imam. 

Di Jawa para imam biasa disapa dengan panggilan
``\textit{romo}'', yang artinya bapak. Ini merupakan terjemahan
harafiah dari sapaan imam di Eropa yang dipanggil \textit{pater}, atau
\textit{father}, yang juga bermakna bapak. Para imam yang mengelola
sebuah paroki, lazim disebut pastor, yang berasal dan bahasa
Latin dengan arti gembaIa. Imam kepala di sebuah paroki,
disebut sebagai pastor kepala. Namun sekarang semua imam,
meskipun tidak mengelola sebuah paroki, juga disapa dengan
panggilan pastor. Bahkan di beberapa negara, pendeta Kristen/
Protestan pun disapa dengan panggilan pastor. 


\section*{10 Tarekat Pria Terbesar di Dunia}

\begin{tabular}{llrr}
\hline
&\textbf{Tarekat}&\textbf{Anggota}&\textbf{Imam}\\
\hline
1.	&SJ (Jesuit) 		&21.490	&15.105\\
2.	&OFM (Fransiskan)	&17.335	&11.417\\
3. 	&SDB (Salesian)		&17.192	&11.224\\
4.	&OFMCap (Kapusin)	&11.303	&7.303\\
5.	&OSB (Benediktin)	&7.926	&4.565\\
6.	&OP (Dominikan)		&6.375	&4.768\\
7.	&SVD (Sabda Ilahi)	&5.962	&3.769\\
8.	&CSsR (Redemptoris)	&5.773 	&4.262\\
9.	&OMI (Oblat)		&4.831	&3.508\\
10.	&OFMConv. (Konventual)	&4.499	&2.783\\
\hline
\end{tabular}

\section*{10 Tarekat Pria Terbesar di Indonesia}

\begin{tabular}{llrr}
\hline
&\textbf{Tarekat}&\textbf{Anggota}&\textbf{Imam}\\
\hline
1.	&SVD 	&947	&440\\
2.	&SJ 	&338	&238\\
3.	&OFMCap	&335	&178\\
4.	&MSC	&322	&183\\
5.	&OCarm 	&201	&93\\
6.	&OFM 	&194	&75\\
7.	&MSF	&181	&100\\
8.	&SCJ	&177	&103\\
9.	&OSC	&135	&85\\
10.	&CM		&123 	&76\\
\#	&Praja	&2.502	&1.177\\ \hline
\end{tabular}

Di seluruh Indonesia, ada 26 ordo (konggregasi, tarekat) imam Katolik, dan imam projo (diosesan). Masing-masing ordo seakan-akan punya wilayah tersendiri. Misalnya SVD lebih banyak berkarya di NTT, OFMCap di Kalbar dan Sumut, Jesuit di Jateng (minus Purwokerto), DIY, dan DKI Jakarta. Imam Projo, yang langsung berada di bawah uskup dan tidak menjadi anggota ordo, menyebar di keuskupan masing-masing, terutama keuskupan di Jawa. Imam yang anggota tarekat maupun iman projo, tidak hanya bertugas di paroki (gereja), atau di biara masing-masing, melainkan juga berkarya di luar tarekat/keuskupan. Di antara mereka ada yang mengelola kegiatan gereja (sekolah, perguruan tinggi, penerbitan), ada pula yang berkarya di instansi pemerintah. Misalnya pastor tentara, yang bertugas di kesatuan/angkatan.

\section*{Imam Katolik di Indonesia}

\begin{enumerate}
\item Imam praja/diosesan (Pr)
\item Ordo Karmel (O Carm; Karmelit)
\item Ordo Saudara-saudara Dina (OFM; Fransiskan)
\item Ordo Salib Suci (OSC)
\item Ordo S. Agustinus (OSA)
\item Ordo Saudara-saudara Dina Konventual (OFM Conv.)
\item Ordo Saudara-saudara Dina Kapusin (OFM Cap.)
\item Serikat Jesus (SJ, Jesuit)
\item Ordo Karmel Tak Berkasut (OCD)
\item Ordo Trapis (OCSO)
\item Kongregasi Misi (CM, Lazaris)
\item Serikat Imam Misi Luar Negeri (Paris-MEP)
\item Serikat Maria Monfortan (SMM)
\item Kongregasi Sengsara Yesus (CP; Pasionis)
\item Kongregasi Redemptoris (CSsR)
\item Kongregasi Pater-pater Hati Kudus Yesus dan Maria(SSCC; Picpus)
\item Kongregasi S.  Perawan  Maria Yang Terkandung TakBernoda (OMI; Oblat)                      
\item Kongregasi Misionaris Hati Kudus (MSC)
\item Serikat Salesian Don Bosco (SDB)
\item Kongregasi Hati Maria Tak Bernoda (CICm; Scheutis)
\item Misionaris Mill Hill (MHM)
\item Serikat Sabda Allah (SVD)
\item Kongregasi Imam-imam Hati Kudus Yesus (SCJ)
\item Kongregasi Misionaris Keluarga Kudus (MSF)
\item Serikat S. Fransiskus Xaverius (SX; Xaverian)
\item Misionaris Maknoll (MM)
\item Kongregasi Murid-murid Tuhan (CDD)
\end{enumerate}
