\chap{Sakramen Imamat dan Hidup Menjadi Imam}
Sakramen Imamat sering disebut dengan Sakramen Tahbisan = Sakramen Wisuda. Dengan Tahbisan, seseorang menjadi Pemimpin dalam Gereja. Dulu pernah salah dipahami bahwa Sakramen Imamat/Tahbisan ini hanya untuk Pemimpin Ekaristi – memberi absolusi dalam Sakramen Tobat saja. Dengan Sakramen Imamat seseorang diangkat / diwisuda untuk menggembalakan Gereja dengan Sabda \& Roh Allah

Yang bisa menerima Sakramen Imamat adalah
semua orang pria/laki-laki yang tidak terikat pernikahan, yang beragama / iman Katolik
    Dewasa dalam kepribadian -- iman
    Telah mendapatkan pendidikan cukup di Seminari Tinggi

Ordo dalam arti luas: Ordo berarti lembaga religius, sehingga mencakup baik Ordo dengan kaul agung maupun konggregasi \& serikat hidup kerasulan dengan kaul sederhana. Dalam arti sempit: Ordo berarti lembaga religius / persekutuan yang anggotanya pria atau wanita. Imam atau awam mengikrarkan ketiga nasihat Injil sebagai kaul kekal yang publik serta meriah (atau agung) \& hidup dalam persaudaraan. Tujuannya membuktikan diri kepada Tuhan. Para Imam/Biarawan/Biarawati mengucapkan 3 kaul:
\begin{enumerate}
\item Kaul Ketaatan
\item Kaul Kemiskinan
\item Kaul Kemurnian
\end{enumerate}

Waktu yang diperlukan seseorang menjalani sekolah Seminari sampai akhirnya bisa disebut Frater (mulai dari proses pendaftaran sampai masuk Seminari / tahapan menjadi seorang Frater)
antara 7-8 tahun dengan pentahapan

\begin{itemize}
\item Tahap Postulat: 1 thn
\item Tahap Novis: 1 thn
\item Tahap Filosofan (studi Filsafat): 3 thn
\item Tahap Toper (praktek pastoral di Paroki): 1-2 thn
\item Tahap Teologan (studi Teologi): 2 thn
\item Tahap Diakonat: +/- 6 bulan di Paroki lagi(Diakon sudah termasuk Klerus, artinya bukan awam, karena sudah ditahbiskan sebagai Diakon)
\end{itemize}

Mulai tahap Novislat s/d Skolastikat disebut Frater. Mulai tingkat 5 diperbolehkan mengucapkan kaul kekal. Sebelumnya hanya kaul sementara (1 thn, 2 thn – 3 thn). Setelah kaul kekal lalu mendapatkan tahbisan Diakon, lalu Pastoral di Paroki 6 bulan disusul Tahbisan.

Liburan diatur menurut ketentuan Keuskupan masing-masing ada yang
\begin{itemize}
\item 3 thn mendapatkan libur/cuti 1 bulan
\item 2 thn mendapatkan libur/cuti 2 bulan
\item 1 thn mendapatkan libur/cuti 1 bulan
\item Libur Semester sesuai kalendar akademik masing-masing seminari.
\end{itemize}

Kegiatan yang dilakukan oleh Imam sehari-harinya adalah
memimpin/menggembalakan Jemaat di Paroki

\begin{itemize}
\item melayani Sakramen (7 Sakramen)
\item Doa Pribadi
\item Ada yang jadi dosen/ketua yayasan
\item Bidang sosial
\end{itemize}

Lalu, seberapa pentingnya kah DOA dalam kehidupan sehari-hari seorang Imam?
Romo: Doa menjadi tiang penyangga utama seorang Imam, khususnya dalam Perayaan Ekaristi, tapi juga doa-doa harian pribadi – renungan Kitab Suci. Juga bagi kaum awam sangat penting.

Saat menjadi frater mempunyai tugas pokok: studi, ada tugas tambahan: mengajar agama di sekolah, memberi Retreat -- Rekoleksi kaum muda, latihan-latihan pastoral  semuanya dalam bimbingan Pastor Rektor Seminari / staff.
Frater juga melakukan rekreasi dan kerja tangan.

Pemenuhan/perolehan Sakramen Imamat merupakan puncak kerinduan seorang Calon Imam. Tapi menjadi imam adalah panggilan Tuhan (banyak yang dipanggil, sedikit yang dipilih). Orang harus selalu membuka hati bagi panggilan Tuhan, bersedia meninggalkan kepentingan-kepentingan pribadi untuk bisa mengabdi / melayani Tuhan / Gereja / umatNya.

Para Uskup memakai cincin sebagai tanda bahwa dia Uskup. Uskup juga Tahbisan, justru ini tahbisan tertinggi (tingkat 1). Imam/Pastor adalah tahbisan di bawahnya (tingkat II). 

\begin{quote}\emph{
LG 21: Sebab dengan tahbisan Uskup diterimakan kepenuhan Sakramen Imamat yang biasanya disebut Imamat Tertinggi atau keseluruhan pelayanan suci. Adapun para imam biasa kendatipun tidak menerima puncak imamat dan dalam melaksanakan kuasa mereka tergantung ari para Uskup, namun mereka sama-sama imam seperti para Uskup \& berdasarkan Sakramen Tahbisan merekapun dikhususkan untuk mewartakan Injil serta menggembalakan umat beriman \& untuk merayakan Ibadah Ilahi sebagai Imam sejati perjanjian baru (1628). Akhirnya masih ada para Diakon yang juga ditumpangi tangan tapi bukan untuk Imamat melainkan untuk pelayanan. (1629).}
\end{quote}

Jadi, ada 3 macam Sakramen Tahbisan:
\begin{enumerate}
\item Tahbisan Uskup
\item Tahbisan Imam
\item Tahbisan Diakon
\end{enumerate}

\sumber{nara sumber: Pastor Josef Kristianto}