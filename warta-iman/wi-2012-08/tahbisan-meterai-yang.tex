\chap{Tahbisan: Meterai yang Tak Terhapuskan}

Dalam artikel “Keabsahan Sakramen dan Keadaan Rahmat” dijelaskan bahwa sakramen tetap sah meskipun imam dalam keadaan dosa berat. Bagaimana dengan imam yang meninggalkan imamatnya? Apakah ia masih dapat melayani sakramen?
~ seorang pembaca di Winchester

Dalam menanggapi pertanyaan di atas, pertama-tama patutlah kita ingat bahwa Sakramen Tahbisan, sama seperti Sakramen Baptis dan Sakramen Penguatan, diterimakan hanya satu kali seumur hidup. Masing-masing dari ketiga sakramen tersebut menerakan “tanda rohani yang tidak terhapuskan” pada diri penerima. Karenanya, ketiga sakramen ini tidak dapat diterima ulang dan berlangsung selamanya. Tanda rohani ini tidak dapat hilang karena dosa berat, meskipun rahmat pengudusan dapat hilang karena dosa. (Patut dicatat bahwa Sakramen Tahbisan diterimakan dalam tiga tingkatan - diakonat (diakon), presbiterat (imam), dan episkopat (uskup). Tahbisan Uskup merupakan kepenuhan Sakramen Tahbisan.)

Sebagai contoh, andai seorang yang telah dibaptis Katolik memutuskan untuk meninggalkan Gereja, mengingkari imannya dan menjadi seorang Muslim. Duapuluh tahun kemudian, ia memutuskan untuk kembali ke dalam pelukan Gereja Katolik. Orang ini tidak akan menerima ulang Sakramen Baptis ataupun Sakramen Krisma, sebab tanda rohani Sakramen Baptis dan Sakramen Krisma tidak hilang melainkan tinggal tetap. Yang harus dilakukannya ialah mengakukan dosanya, menerima absolusi, dan kemudian menyatakan Pengakuan Iman (lih Katekismus Gereja Katolik No 1581-1582).

Sebab itu, apabila seorang ditahbiskan sebagai imam, ia menerima tanda rohani yang sakral ini untuk bertindak atas nama Kristus dan bertindak sebagai alat-Nya bagi GerejaNya. Ia juga menerima wewenang dari Uskup Diosis atau otoritas legitim lainnya untuk melaksanakan karya pastoralnya.

Jadi, bagaimana jika seorang imam meninggalkan imamatnya? Karena Sakramen Tahbisan menerakan tanda rohani yang tak terhapuskan, sekali meterai itu diterima secara sah, meterai tak akan pernah dapat dibatalkan demi alasan apapun. Tentu saja seorang klerus - diakon, imam, atau uskup - dapat dibebaskan dari status klerikalnya dan memperoleh dispensasi dalam kaul selibat dari otoritas yang berwenang. Ia kehilangan hak-hak khas status klerikal dan juga tidak terikat lagi oleh kewajiban-kewajiban status klerikal, tetapi, walaupun demikian, ia tetap seorang klerus. Biasanya, praktek ini disebut “laicisasi”, artinya “kembali ke status awam” (Kitab Hukum Kanonik, No. 290-293).

Meskipun ia telah kembali ke status awam dan tidak lagi bertindak sebagai diakon, imam, ataupun uskup, ia tetap memiliki tanda sakramental Sakramen Tahbisan. Secara teknis, seandainya ia melayani suatu sakramen sesuai dengan norma-norma Gereja, maka sakramen yang dirayakannya adalah sungguh sah. Tetapi, sakramen tersebut menjadi tidak halal, artinya klerus melanggar hukum Gereja dan melakukan tindakan tercela atas pelanggaran ini sebab ia tidak lagi memiliki wewenang untuk bertindak sebagai seorang imam.

Kitab Hukum Kanonik membuat suatu pengecualian dalam keadaan gawat darurat: “Imam manapun, meski tidak memiliki kewenangan untuk menerima pengakuan, dapat mengampuni secara sah dan halal peniten manapun yang berada dalam bahaya mati dari segala hukuman dan dosa, meskipun hadir juga seorang imam lain yang telah mendapat persetujuan” (No. 976). Dengan ini Gereja mengakui tanda rohani yang tak terhapuskan yang diterima imam pada saat ia ditahbiskan - walau sekarang ia telah kembali ke status awam.

Sebagi contoh, andai seseorang terluka parah dalam suatu kecelakaan dan mendekati ajalnya. Tak ada imam yang dapat dihubungi untuk mendengarkan pengakuan dosanya. Seorang mantan imam - mungkin telah meninggalkan status klerikalnya selama bertahun-tahun - dapat secara sah mendengarkan pengakuan dosa orang yang sedang menghadapi ajal itu dan secara sah menyampaikan absolusi atas segala dosanya. Bahkan andaipun imam ini telah meninggalkan imamatnya tanpa ijin yang layak dan dalam keadaan dosa berat, ia tetap dapat secara sah menyampaikan absolusi kepada orang yang sedang menghadapi ajalnya.

Patut diingat pula bahwa jika imam yang telah kembali ke status awam memutuskan untuk kembali aktif dalam karya pastoral, ia tidak akan ditahbiskan lagi. Melainkan, ia harus mohon ijin dari Bapa Suci dan memenuhi segala ketentuan yang diwajibkan uskup atau otoritas Gereja (lihat Kitab Hukum Kanonik, No. 293).

(Sebagai tambahan, bahkan jika seorang imam meninggalkan karya pastoral aktifnya tanpa ijin yang pantas dan tanpa proses laicisasi, ia juga masih memiliki tanda sakramental Sakramen Tahbisan. Ia pun dapat secara sah menyampaikan absolusi atas dosa dalam keadaan gawat darurat).

Sementara jawaban di atas merupakan tanggapan atas pertanyaan teknis sehubungan dengan keabsahan sakramen, jawaban ini menyinggung juga masalah yang menyakitkan, yaitu para imam yang meninggalkan imamatnya. Marilah pada saat ini kita semua berdoa, teristimewa bagi para imam paroki kita yang bekerja untuk melaksanakan karya Allah, agar dengan rahmat Tuhan mereka boleh memancarkan pribadi Kristus sendiri, atas nama siapa mereka bertindak dalam merayakan sakramen-sakramen. Kita pun patut berdoa bagi para imam yang telah meninggalkan karya pastoral aktif mereka, agar, jika mungkin, mereka kembali pada panggilan mereka dan melaksanakan anugerah agung imamat yang mereka terima.

\sumber{Romo William P. Saunders\\
diterjemahkan oleh YESAYA: www.indocell.net/yesaya atas ijin The Arlington Catholic Herald.}