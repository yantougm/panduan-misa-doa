\newpage

\chap{Dari Redaksi}

\indent{Berkah Dalem,}

Masa adven adalah masa persiapan dan penantian perayaan Natal. Hari kelahiran Tuhan Yesus kita peringati setiap tahun. Makna adven yang lain adalah masa mempersiapkan diri kita untuk kedatangan Tuhan Yesus yang kedua kalinya. Dia akan datang sebagai Raja Agung pada akhir zaman. 

Seringkali kita melupakan makna adven yang kedua ini. Kita biasanya disibukkan oleh pekerjaan persiapan perayaan Natal. Baik dengan latihan koor, menghias gereja, mempersiapkan pesta Natal di lingkungan atau di kantor. Tindakan-tindakan itu tidak salah tetapi perlu dilengkapi agar makna kedua dari adven ikut terangkat. Kadang kita tidak tertarik untuk membicarakan kedatangan Kristus yang kedua karena hal ini berhubungan dengan akhir zaman, yang secara tidak langsung berhubungan dengan hal kematian. Pembicaraan tentang kematian memang banyak yang tidak berselera untuk membahasnya.

Kenyataan tersebut di atas menyadarkan kita bahwa manusia adalah benar-benar makhluk duniawi. Manusia memang lebih suka mengisi hidupnya dengan hal-hal yang yang bersifat masa kini dengan segala kekawatiran, kecemasan dan rencana-rencana manusiawi untuk mengantisipasi segala kekawatiran dan kecemasan yang timbul dengan pemikiran yang seringkali duniawi juga. Manusia jarang menyisakan ruang di dalam hatinya untuk memberi perhatian atau memikirkan hidupnya pada Tuhan, dalam rahasia akhir yang berada di tangan Tuhan.

Warta Iman kali ini, yang bertema Adven dan Natal, menyajikan tulisan tentang adven, Maria sebagai Ibu Yesus, dan Betlehem tempat kelahiran Yesus. Di samping itu ada renungan tentang penerimaan sesama apa adanya. Seperti biasa sebagai penutup adalah kutipan Kompendium Gereja Katolik.

Selamat Natal dan Tahun Baru 2012.
\begin{center}***\end{center} 

\vfill

\noindent{\framebox{\parbox{10cm}{\small
Warta Iman\\
Media komunikasi dan informasi umat lingkungan St. Petrus\\
Alamat Redaksi: Lingkungan St. Petrus Maguwo\\
E-mail: stpetrusmgw@gmail.com
}}}


