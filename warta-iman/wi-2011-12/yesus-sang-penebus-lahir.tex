\chap{Yesus Sang Penebus Lahir dari Santa Perawan Maria}

\begin{center}\emph{Oleh : Sr. Maria Yoanita, P.Karm}\end{center}

\section*{Kerinduan manusia akan Tuhan}

``'Sesungguhnya, anak dara itu akan mengandung dan melahirkan seorang anak laki-laki, dan mereka akan menamakan Dia Imanuel' -- yang berarti: Allah menyertai kita'' (Mat. 1:23)

Tidak ada seorang manusia pun yang datang ke dunia ini tanpa melalui seorang ibu. Manusia datang ke dunia, lahir melalui rahim seorang ibu. Semua manusia mempunyai ibu, demikian juga Yesus, Ia lahir melalui seorang ibu, yakni Maria. Maria adalah ibu Yesus dan ibu rohani bagi kita semua. 

Kita bisa membayangkan bagaimana hati Maria berdebar-debar gembira, penuh kerinduan dan harapan menantikan kelahiran Putranya. Sesungguhnya kerinduan Maria akan kelahiran Putranya ini menyiratkan kerinduan seluruh umat manusia akan kedatangan Sang Penebus, suatu kerinduan akan Tuhan. Hati Maria penuh rasa takwa, bila ia ingat bahwa sudah tiba saatnya kerinduan seluruh umat manusia ini dikabulkan dan janji Allah ditepati melalui dirinya.

Kiranya tidaklah salah bahwa para Karmelit pertama mengenakan nama ``Para Saudara dari Santa Perawan Maria dari Gunung Karmel'' karena kerinduan akan Tuhan yang dimiliki Maria juga mengalir dalam diri mereka bagaikan sebuah sungai yang makin lama makin deras dan menguasai seluruh hati dan hidup mereka. Kerinduan ini membakar dan menjiwai para Karmelit, sehingga mereka rela meninggalkan segalanya demi mencapai persatuan cintakasih yang mesra dengan Tuhan. Dalam sikap hati yang penuh takwa mereka senantiasa tinggal di hadirat-Nya dalam doa dan karya mereka.

\section*{Maria Bunda Allah}

Devosi para karmelit terhadap Maria bukanlah devosi yang lahir dari luapan emosi sesaat, bukan pula sekedar kekaguman akan pribadi dan keutamaan-keutamaan Maria, walaupun kekaguman itu ada. Dasar utama devosi terhadap Maria adalah fakta bahwa Maria sungguh-sungguh Bunda Tuhan kita Yesus Kristus.

Anak yang dilahirkan oleh Maria di kandang Betlehem sung-guh-sungguh adalah Allah yang Mahabesar, namun mau menjadi bayi yang mungil, kecil, lemah, tak berdaya, yang hidupnya bergantung seluruhnya kepada kasih dan perhatian dari orangtua yang melahirkan dan membesarkan-Nya. Yesus adalah Allah, maka Maria yang melahirkan-Nya adalah Bunda Allah. Konsili Efesus menyatakan bahwa Maria berhak digelari ``Theotokos'' , Bunda Allah.

Bidaah Nestorianisme tidak dapat memahami dan menerima gelar Maria ini. Mereka tidak mengakui bahwa Sabda menjadi daging, sebagaimana dikatakan Kitab Suci (bdk. Yoh. 1:14). Menurut mereka, Allah sekedar mempersatukan diri-Nya dengan seorang manusia yang sudah ada. Dan, Maria hanya melahirkan manusia Kristus ini.

Pendapat Nestorius ini dibantah oleh St. Sirilus dari Alexandria. Dalam suratnya yang kedua kepada Nestorius, St. Sirilus menulis: ``Tidak ada seorang manusia biasa yang lahir dari St. Perawan Maria lebih dahulu, dan kemudian Sang Sabda turun atas dia. Tidak demikian. Sang Sabda sudah bersatu dengan daging dalam kandungan Maria, ketika Sabda itu lahir menurut daging. Maka dapat dikatakan, bahwa Sabda menganggap kelahiran daging-Nya sebagai kelahiran-Nya sendiri.... Karena itu para bapa konsili tidak ragu-ragu menyebut St. Perawan sebagai Bunda Allah.'' Tulisan Sirilus ini didukung penuh oleh para bapa Konsili Efesus.

Martin Luther tidak pernah meragukan ajaran Konsili Efesus itu. Ia mengatakan, ``Konsili ini tidak menentukan pokok iman yang baru, hanya mempertahankan iman yang lama, melawan kecongkakan Nestorius yang baru. Pokok ini, bahwa Maria itu Bunda Allah, telah diterima oleh Gereja sejak semula. Ini bukan penemuan baru yang dibuat oleh konsili, melainkan sudah tercatat dalam Injil, atau Kitab Suci.''

Banyak ucapan dalam Kitab Suci yang membuktikan bahwa Maria itu Bunda Allah, antara lain: Injil Lukas menceritakan tentang Malaikat Gabriel yang memberi kabar kepada perawan Maria bahwa Putera dari Yang Mahatinggi akan lahir daripadanya (bdk. Luk. 1:32); Elisabeth berkata kepada Maria, "Siapakah aku ini, sampai ibu Tuhanku datang mengunjungi aku?'' (Luk. 1:43); para malaikat pada malam Natal mewartakan, ``Hari ini telah lahir bagimu Juruselamat, yaitu Kristus, Tuhan" (Luk. 2:11); demikian pula St. Paulus mengatakan, ``Allah mengutus Anak-Nya, yang lahir dari seorang perempuan'' (Gal. 4:4).

Seumur hidupnya Maria mencintai dan mengabdi Putranya. Bukti nyata adalah kesetiaannya sampai di kaki salib Putranya. Sampai sekarang Maria tetap mencintai dan mengabdi Putranya. Pernahkah seseorang mendekat padanya dan tidak dibawanya lebih dekat kepada Putranya? Devosi terhadap Maria bagaikan sebuah pintu yang membuka aliran rahmat Tuhan dengan lebih deras agar kita dapat semakin dekat pada-Nya. Contoh konkrit tentang hal ini dapat kita lihat dalam hidup St. Theresia Lisieux.

\section*{Devosi St. Theresia Lisieux kepada Bunda Maria}

\qti{Mengapa Aku Cinta Padamu, O Maria\\
~\\
Saya tahu bahwa di Nazareth, O Bunda penuh rahmat\\
Hidupmu sangat miskin, engkau tak merindukan sesuatu yang lain\\
Tak ada keterserapan, tak ada mukjizat, tak ada ekstase\\
Untuk memperindah hidupmu, O Ratu pilihan\\
Jumlah orang-orang kecil di dunia ini sangat besar\\
Tanpa gemetar mereka dapat mengarahkan matanya kepadamu\\
Lewat jalan yang biasalah, Bunda yang tak ada bandingnya\\
Engkau telah berjalan untuk memandu mereka ke surga\\
(St. Theresia dari Lisieux)}


Bagi St. Theresia Lisieux, Bunda Maria yang menjalani hidup yang sangat sederhana dan tak menyolok sama sekali itu sangat memesona dirinya. St. Theresia menemukan dalam diri Bunda Maria suatu teladan yang hidup. Dia mengagumi bagaimana Bunda Maria menanggapi kabar gembira dari malaikat, dan bagaimana ia selalu setia dalam setiap perkara hidupnya serta menyimpan segala perkara itu dalam hatinya. Dengan melihat hidup Maria yang sangat sederhana namun sangat surgawi, St. Theresia dituntun untuk menemukan ``Jalan Kecil''nya yang sangat terkenal itu. Bagi St. Theresia, Maria adalah teladan bagi jiwa-jiwa yang tak terbilang jumlahnya. Ini juga panggilan St. Theresia yang luar biasa, yaitu membawa orang-orang kepada Allah melalui jalan-jalan yang biasa dan belajar dari Maria tentang rahasia kekudusan yang tinggi dalam kehidupan yang sangat biasa.

\section*{Bunda Maria Teladan Setiap Jiwa}

Melihat kesederhanaan dan ke-"serba-biasa"-an hidup Maria, maka Maria bukanlah model khusus untuk para Karmel religius ataupun kaum berjubah saja. Ia merupakan teladan bagi setiap jiwa, termasuk bagi kaum awam, terlebih bagi para Karmel awam, apa pun pekerjaannya. Misalnya, bagi para ibu rumah tangga. Para ibu rumah tangga di seluruh dunia dapat meneladan Bunda Maria yang suci. Bunda Maria, seperti para ibu rumah tangga pada umumnya, melakukan pekerjaan-pekerjaan rumah tangga yang biasa. Kita dapat membayangkan bahwa Maria melakukan pekerjaan-pekerjaan harian seperti memasak, mencuci, membersihkan rumah, menimba air, dan mengurusi keperluan Yesus, Puteranya, serta Yusuf, suaminya. Maria dalam ketersembunyian dan dalam kesederhanaannya di Nazareth, melakukan hal-hal yang biasa dengan cinta yang luar biasa. 

Teladan Maria dapat "diterjemahkan" ke dalam "bahasa" untuk para bapak: rajin bekerja, tidak korupsi waktu/uang/dll, memerhatikan, mencintai dan melayani anak dan istri, dll. Teladan Maria juga dapat "dibahasakan" untuk para pelajar: rajin belajar, tidak menyontek/bolos, mengerjakan tugas dengan baik, dll. Singkatnya teladan Maria dapat "dibahasakan" untuk semua orang karena kesederhanaan dan keserba-biasaannya. Inti teladannya adalah dengan setia melakukan semua tugas sehari-hari kita demi cinta kepada Yesus.

Allah memilih Maria yang rendah hati menjadi ibu Tuhan. Maria menjadi besar di hadapan Allah. Walaupun dia tak pernah berkotbah, tak pernah membuat mujizat, namun Maria menjadi ratu para malaikat, ratu para rasul, ratu para kudus. Maria menjadi besar di mata Allah karena iman dan kasihnya yang besar kepada Allah dan karena kerendahan hatinya yang mendalam. 

Allah memilih Maria yang kecil dan rendah hati untuk turut serta dalam karya penyelamatan-Nya. Melalui kesediaan Maria, Yesus Kristus Sang Juruselamat lahir ke dunia untuk menebus dosa-dosa manusia. Maria dipilih oleh Allah membawakan sukacita yang besar bagi dunia. Kita semua juga dapat meneladan Maria dengan memersembahkan hidup kita, tugas dan pekerjaan harian kita yang biasa-biasa saja dengan cinta yang luar biasa kepada Allah demi karya keselamatan-Nya. Allah mau memakai kita-orang-orang biasa-bagi tujuan-Nya yang luar biasa, yakni keselamatan umat-Nya.

\section*{Natal Tak Dapat Dipisahkan dari Misteri Kesengsaraan, Wafat, dan Kebangkitan-Nya}

\qti{Firman itu telah menjadi manusia, dan diam di antara kita, dan kita telah melihat kemuliaan-Nya, yaitu kemuliaan yang diberikan kepada-Nya sebagai Anak Tunggal Bapa, penuh kasih karunia dan kebenaran (Yoh. 1:14)}

Seorang pelukis Kristen melukis sebuah kartu Natal. Seperti biasanya, pada kartu itu tampak Maria, Yosef, dan bayi Yesus. Namun, yang menarik adalah bayi Yesus tidak dilukiskan dalam sebuah palungan, melainkan dalam sebuah peti mati. Dengan itu si pelukis mau mengatakan bahwa Yesus datang untuk mati bagi kita. St. Agustinus melukiskan dengan indah sekali perjalanan hidup Yesus yang adalah Firman yang menjadi Manusia:

\qti{Firman Bapa, yang telah menjadikan zaman-zaman, telah menjadi manusia dan demi kita menetapkan hari kelahiran-Nya dalam waktu. Dalam kejadian-Nya sebagai manusia diinginkan-Nya satu hari bagi diri-Nya, padahal satu hari pun tak akan berlalu tanpa perintah Ilahi-Nya. Sementara Ia tinggal bersama Bapa-Nya, Ia mendahului segala zaman, dan Ia pada hari ini keluar dari ibu-Nya untuk memasuki peredaran tahun. Ia menjadi manusia, Ia yang menjadikan manusia. Maka Dia yang mengendalikan bintang-bintang menyusu pada buah dada ibu-Nya; Dia yang adalah roti kelaparan dan Dia yang adalah mata air kehausan; Dia yang merupakan terang menjadi temaram; Dia yang adalah jalan letih karena perjalanan; Dia yang adalah kebenaran kena tuduhan dengan saksi-saksi palsu; Dia yang menjadi hakim atas orang yang hidup dan orang mati dihukum oleh hakim yang fana; Dia yang adalah keadilan dinyatakan bersalah oleh orang-orang yang tidak adil. Pengajar didera dengan cemeti; buah anggur dimahkotai dengan duri-duri; fundamen digantungkan pada salib; kekuatan menjadi lemah; kesembuhan terlukai; bahkan hidup mengalami kematian. Dengan menanggung alih-alih kita, kehinaan-kehinaan ini dan yang serupa, yang tidak layak Ia tanggung, kita Ia bebaskan, sekalipun kita tidak layak dibebaskan. Sebab Dia yang telah menanggung segala keburukan itu demi kita tidak patut menerima keburukan apa-apa, sedangkan kita yang melalui Dia menerima segala kebaikan, tidak patut menerima kebaikan apa-apa. (Akan tetapi kehinaan-Nya menjadi kemuliaan kita dan salib-Nya adalah kemenangan kita; palang salib-Nya menjadi tanda kejayaan kita dan kematian-Nya adalah kehidupan kita). Karena itulah, Dia yang sebelum segala abad menjadi Anak Bapa, yang hari-hari-Nya tidak berawal pada hari-hari zaman akhir sudi menjadi anak manusia. Dia yang lahir dari Bapa dan tidak dijadikan oleh Bapa, terjadi di dalam seorang ibu yang telah dijadikan oleh-Nya, supaya di sini, pada saat tertentu, Ia terbit dari seorang perempuan itu yang sama sekali tidak mungkin ada, kapan pun, dan di mana pun, kecuali oleh Dia \emph{(St. Agustinus, Bagai Terang di Hati)}}.

Dalam tulisan St. Agustinus di atas kita bisa melihat ungkap-an-ungkapannya yang paradoksal. Sebenarnya St. Agustinus menjajarkan apa yang tampak di mata dunia dan apa yang tampak di mata iman. Misalnya: "Dia yang adalah jalan letih karena perjalanan". Di mata dunia yang tampak hanyalah Yesus yang dalam perjalanannya merasa letih, namun dengan mata iman kita bisa melihat bahwa Yesus adalah "jalan". Dalam seluruh perjalanan hidup kita, kita diajak untuk senantiasa mampu memandang lebih dalam daripada apa yang tampak di mata dunia. Terlebih bagi para Karmelit, yang secara istimewa mendapat rahmat panggilan kontemplatif. Karmelit dipanggil untuk senantiasa mampu memandang Tuhan dalam segala sesuatu dan untuk senantiasa ingat akan tujuan akhir hidupnya ... untuk apa ia datang ke dunia.

\section*{Menemukan Damai yang Sejati dalam Tuhan}

Sebelum Tuhan Yesus lahir di dunia, Nabi Yesaya telah bernubuat tentang Dia dan menyebutnya sebagai Raja Damai: "seorang anak telah lahir untuk kita, seorang putera telah diberikan untuk kita; lambang pemerintahan ada di atas bahunya, dan namanya disebutkan orang: Penasihat Ajaib, Allah yang Perkasa, Bapa yang Kekal, Raja Damai" (Yes. 9:5). Ketika Tuhan Yesus lahir di dunia ini para malaikat mewartakan damai di bumi bagi orang yang berkenan kepada Allah, karena memang itulah kerinduan hati Allah sendiri. Dan sesudah bangkit dari antara orang mati Tuhan Yesus menyapa murid-murid-Nya dengan kata-kata: "Damai sejahtera bagi kamu" (Luk. 24:36; Yoh. 20:19.21.26). 

Kiranya jelas bahwa Tuhan sungguh merindukan damai bagi kita. Namun, damai yang dijanjikan Tuhan tidak seperti yang dimengerti oleh dunia ini. Damai-Nya jauh lebih luhur dan lebih mulia. "Damai sejahtera Kutinggalkan bagimu. Damai sejahtera-Ku Kuberikan kepadamu, dan apa yang Kuberikan tidak seperti yang diberikan oleh dunia kepadamu. Janganlah gelisah dan gentar hatimu" (Yoh. 14:27).

Yesuslah Sang Raja Damai. Ia datang ke dunia pertama-tama adalah untuk mendamaikan manusia dengan Allah. Oleh karena itu, damai sejati baru bisa lahir dalam hati manusia jika ia memiliki relasi yang baik dengan Allah. Relasi yang baik dengan Allah akan mendorong manusia untuk mempunyai relasi yang baik dengan sesama pula. Seseorang hanya dapat membawakan damai yang sejati bila ia sendiri telah menemukan damai tersebut dan ia tidak akan menemukannya selain di dalam Yesus. Oleh karena itu, dengan mata kontemplatifnya yang senantiasa terarah memandang Yesus, seorang Karmelit bisa tetap dipenuhi damai-Nya walaupun ia berada di tengah badai kehidupan.

Dalam suatu lomba melukis para peserta diminta melukiskan "damai". Semua peserta merenung sejenak dan mulai melukis. Yah, ada macam-macam yang mereka lukis untuk mengungkapkan damai. Ada yang melukis sebuah rumah yang indah, gunung yang indah, laut yang tenang, dua tangan yang sedang berjabatan, seekor burung merpati, keluarga yang rukun, dll. Namun, ada satu lukisan yang menarik: lukisan sebuah air terjun yang sangat deras dan di balik air terjun itu ada ranting pohon yang mencuat di mana di atasnya tidur dengan nyenyaknya seekor anak burung. Terlihat suatu perbedaan antara damai yang diungkapkan lukisan ini dengan yang diungkapkan melalui lukisan-lukisan lain. Lukisan-lukisan lain tadi menggambarkan damai yang "dari (keadaan) luar", misalnya: laut yang tenang, gunung yang indah, dll. Sedangkan lukisan si anak burung ini menggambarkan damai yang "dari dalam": si anak burung ini dapat tidur nyenyak walau dia tak berdaya menyuruh air terjun supaya jangan terjun/ berisik. 

Seringkali situasi kita sama seperti si anak burung ini: kita tidak berdaya menyuruh suami/istri supaya jangan selingkuh, rekan kerja supaya jangan menipu, dll. Dengan mata imannya, seorang kontemplatif mampu memandang lebih dalam daripada apa yang tampak. Ia mampu "memandang" Tuhan dan jalan-jalan-Nya dalam segalanya. Walaupun ia tidak mengerti, namun ia percaya akan kasih dan penyertaan Tuhan dalam segalanya, sehingga hatinya diliputi damai-Nya. Hal ini bisa kita lihat dengan jelas dalam diri para kudus, khususnya para martir. St. Laurentius ketika dipanggang, dia masih bisa bergurau bahwa sisi tubuhnya yang lain belum matang. Itu diucapkannya bukan dengan kemarahan untuk mengolok-olok musuh. Ia tidak membenci musuh-musuhnya walaupun ia merasakan panasnya api pemanggangan. 

Damai seperti ini hanya bisa diberikan oleh Tuhan. Dunia-entah itu uang, narkoba, jabatan, jimat, kenikmatan daging, dll-tidak akan bisa memberi damai yang seperti ini. Dari sini dapat dimengerti pula mengapa para kontemplatif berani tinggal di padang gurun yang secara mata duniawi "tak ada hiburan apa-apa". Mereka menemukan Tuhan-Sumber Damai-dalam keheningan hatinya.

Semoga hati Anda senantiasa dipenuhi oleh damai yang tak terperikan yang berasal dari Tuhan sendiri, juga di tengah segala situasi, kondisi, kesukaran, dan percobaan hidup. Dan, semoga iman Anda semakin diteguhkan bahwa Dia yang telah memilih Anda dan yang telah mengasihi Anda lebih dahulu adalah setia dan tak akan meninggalkan Anda, seperti janji-Nya: "Sebab biarpun gunung-gunung beranjak dan bukit-bukit bergoyang, tetapi kasih setia-Ku tidak akan beranjak dari padamu dan perjanjian damai-Ku tidak akan bergoyang, firman TUHAN, yang mengasihani engkau" (Yes. 54:10) dan bahwa "Allah turut bekerja dalam segala sesuatu untuk mendatangkan kebaikan bagi mereka yang mengasihi Dia" (Rm. 8:28). Itulah yang kami harapkan dan doakan bagi Anda dalam masa Natal ini. 

\section*{Sharing}

Kehidupan Bunda Maria yang penuh kesederhanaan telah menyentuh hati St. Theresia Lisieux. Bagaimana dengan Anda pribadi dalam menjalani kehidupan sehari-hari? Sudahkah Anda juga mengikuti teladan Bunda kita dalam kesederhanaannya? 

Yesus yang lahir dari rahim Bunda Maria juga akan lahir dalam hati kita masing-masing. Siapkah Anda untuk menyambut kelahiran Yesus dalam hatimu? Bagaimana cara dan usaha Anda untuk menjadikan hati sebagai tempat yang layak bagi Dia? 

\sumber{Sumber: Majalah Rohani Vacare Deo\\
(http://www.holytrinitycarmel.com)}
