\chap{Aku Mengasihi Engkau Apa Adanya...}
 
``\textit{MAMA ..., MAMA ..., tuh si SiuSiu tuh mukanya ngeliatin Albert kayak gitu,}'' kata Albert anak kami yang baru berusia 4 tahun dengan wajah tanpa salah dan meminta dukungan penuh dari mamanya. ``\textit{Yach ngga pa pa lah Bert, memang mukanya kayak gitu, mo diapain?}'' Jawab saya sebagai Papanya. 

Dari kejadian itu saya menjadi berpikir lebih jauh lagi ke dalam, betapa Tuhan itu, sangat baik kepada kita, Dia menerima kita apa adanya, dengan segala kelemahan kita, bahkan dengan segala dosa yang kita perbuat.
Itu dibuktikan-Nya, dengan kenyataan, bahwa Dia telah mengirimkan Putera-Nya yang tunggal Tuhan kita Yesus Kristus sebagai penebus dosa manusia. 

Seperti ada tertulis di Yoh. 3:16 
\qti{Karena begitu besar kasih Allah akan dunia ini, sehingga Ia telah mengaruniakan Putera-Nya yang tunggal, supaya setiap orang yang percaya kepada-Nya tidak binasa, melainkan beroleh hidup yang kekal.}

Dalam kehidupan pelayanan saya, dalam pengalaman mendengarkan orang-orang yang 'curhat', orang yang meminta pendapat saya, banyak sekali saya melihat dan mendengar betapa manusia yang satu selalu mengharapkan, agar manusia yang lain, menjadi seorang pribadi seperti pribadi yang dia sendiri harapkan, pribadi menurut gambarannya sendiri, menjadi suatu pribadi yang diimpikannya.

Contohnya seorang istri mengharapkan suaminya selalu berpakaian seperti yang diimpikannya, menghabiskan waktu selalu bersama dengannya, bekerja menghasilkan uang yang cukup untuk keluarganya, menjadi ini dan itu, juga tidak boleh ini dan itu, bla...bla... bla.... Seorang suami mengharapkan istrinya melayani dia sebaik mungkin, pintar mengurus anak, masak yang enak untuk keluarga, pintar menjaga kecantikannya, menjadi seperti ini dan itu, juga tidak boleh ini dan itu, bla...bla... bla....Para orang tua mengharapkan anak-anak yang selalu patuh dan menurut terhadap setiap perkataan dan peraturan mereka. Dua orang yang bersahabat mengharapkan temannya melakukan semuanya, sesuai dengan keinginannya, dsb.

Tidak menerima seseorang apa adanya, kecenderungan pemaksaan kehendak pribadi sendiri, sering mengakibatkan kehidupan rumah tangga menjadi seperti medan perperangan. Sang suami tidak mau menerima sang istri, sang istripun demikian. Pada saat masa pacaran, pasangannya disanjung bagaikan dewa dewi dari kahyangan, disayang melebihi setiap barang berharga di dunia, dibawa dan diimpikan setiap saat. Tapi sayangnya semua hanya tinggal kenangan, dengan bertambahnya usia perkawinan suatu pasangan, dengan diselingi satu dua konflik kecil, pandangan terhadap pasangannya mulai berubah. Sang suami yang telah bertobat dari kesalahan yang dulu pernah dilakukannya dan telah berjanji tidak mau melakukan kesalahan yang sama lagi, tidaklah cukup memuaskan kemauan dari sang istri, yang mungkin dulu merasa tertindas dan tertekan, sekarang menginginkan suatu perubahan yang lebih baik dan lebih baik lagi. Sang istri yang dengan bertambahnya usia, mungkin menjadi tidak semenarik pada waktu masa pacaran dulu, selalu diharapkan tampil menarik, tidak boleh sakit-sakitan, tidak boleh terlalu cerewet, dsb. Terkadang ada beberapa suami (istri) mulai melirik ‘rumput tetangga sebelah’, bahkan lebih lagi, ada yang mulai koleksi wil (pil), dan sebagainya, dan berbagai macam kejadian yang cukup membingungkan untuk didengar.

Apakah semua yang selalu kita pikirkan cocok untuk seseorang, akan dan harus cocok untuk orang tersebut? Apakah kita juga telah memikirkan, bila kita adalah orang yang kita 'paksa', akankah kita mau untuk diatur oleh orang lain, menjadi siapakah kita nantinya? Menjadi seperti apakah penampilan kita? Apakah kita akan tetap menjadi kita? Apakah saya akan tetap menjadi saya?

Tuhan menciptakan kita beraneka-macam, ada yang berkulit putih, hitam, kuning, coklat, dsb. Ada yang berhidung mancung, bermata sipit, berjidat lebar dan beragam bentuk lainnya. Dia menerima kita apa adanya, mengapakah kita tidak sanggup menerima 'kita' yang lain seperti apa adanya, sementara kita adalah ciptaan-Nya? Apakah kita harus selalu mengatakan, ``Saya tidak mau lagi ketemu 'orang' itu, karena mukanya cemberut terus, bicaranya marah-marah terus, bahasanya kasar, dsb...'' Apakah seseorang yang ingin berbicara dengan kita, harus tampil dengan sejuta senyuman seperti gadis-gadis cantik, atau pria-pria ganteng di iklan pasta gigi? Terkadang kita hanya perlu menerima seseorang apa adanya, hanya menikmati keperbedaan yang Tuhan berikan kepada kita manusia, menikmati si A yang suka humor, si B yang suka marah-marah, si C yang sensi banget, si D si tukang ngatur, si E yang kalau berbicara tidak bisa diam, dan seterusnya...

Saudara saudariku terkasih dalam Yesus Kristus Tuhan, sebentar lagi kita merayakan Natal, kedatangan Anak Allah ke dunia, untuk menebus dosa kita umat manusia, kedatangan dari OUR SAVIOR, OUR LORD, kedatangan dari Yesus Kristus Sang Mesias, Gembala Agung kita. Mungkin pada kesempatan ini kita bisa pergunakan untuk melihat ke diri kita yang terdalam, ke perjalanan kehidupan kita selama ini, bagaimanakah sikap kita terhadap orang-orang lain, terhadap sesama kita?

Dia yang menebus dosa kita menerima kita apa adanya, dia tidak memilih-milih kita ini siapa, apakah kita pantas untuk ditebusnya. Pada saat pelayanan-Nya, Yesus tidak membeda-bedakan, apakah seseorang itu seorang pemungut cukai, seorang pelacur, atau orang-orang yang dianggap berdosa, juga sampah masyarakat pada saat itu, bahkan kaum kafir, malah Dia datang untuk menyelamatkan mereka. Yesus datang untuk menebus kita semua. 

Di dalam kitab Yesaya 44:22 ada tertulis 
\qti{Aku telah menghapus segala dosa pemberontakanmu seperti kabut diterbangkan angin dan segala dosamu seperti awan yang tertiup. Kembalilah kepada-Ku, sebab Aku telah menebus engkau!}

Akhirnya memang semuanya kembali tergantung kepada kita, apakah kita yang memiliki kehendak bebas ini, mau diselamatkan oleh-Nya, apakah kita mau selalu bertobat dari setiap dosa pemberontakan yang telah kita lakukan terhadap Dia. Marilah kita melakukan perintah-Nya yaitu mengasihi sesama kita, seperti perintah-Nya yang kedua ``Kasihilah sesamamu manusia seperti dirimu sendiri'', tanpa melupakan perintah-Nya yang pertama ``Kasihilah Tuhan, Allahmu, dengan segenap hatimu dan dengan segenap jiwamu dan dengan segenap akalbudimu.''(Mat.22:37). 

Tuhan Yesus memberkati kita semua.

\sumber{Sebastianus S.T, koordinator PDDB}