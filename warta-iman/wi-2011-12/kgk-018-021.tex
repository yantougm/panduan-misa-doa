\newpage
\chap{Kompendium Katekese Gereja Katolik}
\setcounter{kgkcounter}{17}
{\normalsize
\section*{KITAB SUCI}

\kgk{Mengapa Kitab Suci mengajarkan kebenaran?}
Karena Allah sendirilah pengarang Kitab Suci. Atas alasan inilah Kitab Suci
disebut ``terinspirasikan'', dan tanpa kesalahan mengajarkan kebenaran yang per-
          lu untuk keselamatan kita. Roh Kudus menginspirasikan para pengarang yang
          menuliskan apa yang Dia inginkan untuk mengajar kita. Tetapi, iman Kristen
          bukanlah ``agama Kitab'', tetapi agama Sabda Allah – ``bukan kata-kata yang tertulis
          dan bisu, melainkan Sabda yang menjadi manusia dan hidup'' (Santo Bernardus
          dari Clairvaux).

\kgk{Bagaimana Kitab Suci seharusnya dibaca?}
         Kitab Suci harus dibaca dan ditafsirkan dengan bantuan Roh Kudus dan di
  bawah tuntunan Kuasa Mengajar Gereja menurut kriteria: 1) harus dibaca dengan
          memperhatikan isi dan kesatuan dari keseluruhan Kitab Suci, 2) harus dibaca
          dalam Tradisi yang hidup dalam Gereja, 3) harus dibaca dengan memperhatikan
          analogi iman, yaitu harmoni batin yang ada di antara kebenaran-kebenaran iman
          itu sendiri.

\kgk{Apa kanon Kitab Suci itu?}
Kanon Kitab Suci ialah daftar lengkap dari tulisan-tulisan suci yang diakui oleh
Gereja melalui Tradisi Apostolik. Kanon ini terdiri dari 46 kitab Perjanjian Lama
          dan 27 kitab Perjanjian Baru.

\kgk{Apa pentingnya Perjanjian Lama bagi umat Kristen?}
Para pengikut Kristus menghormati Perjanjian Lama sebagai Sabda Allah
          yang benar. Seluruh Kitab Perjanjian Lama itu diinspirasikan secara ilahi dan
          mempunyai nilai tetap. Kitab-kitab itu memberikan kesaksian tentang pedagogi
          ilahi cinta Allah yang menyelamatkan. Kitab-kitab itu ditulis terutama untuk
          mempersiapkan kedatangan Kristus sang Penyelamat alam semesta.


\flushright{(\dots \emph{bersambung} \dots)}
}