\newpage
\chap{Kompendium Katekese Gereja Katolik}
\setcounter{kgkcounter}{37}
\normalsize
\kgk{Dengan nama apa Allah mewahyukan Diri-Nya?}
Allah mewahyukan Diri-Nya kepada Musa sebagai Allah yang hidup, ”Allah
Abraham, Allah Iskak, Allah Yakub” (Kel 3:6). Allah juga mewahyukan kepada Musa
nama-Nya yang gaib ”Aku adalah Aku (YHWH)”. Sudah sejak zaman Perjanjian
Lama, Nama Allah yang tak terkatakan ini diganti dengan gelar ilahi Tuhan. Jadi, manakala Yesus disebut Tuhan di dalam Perjanjian Baru, Ia tampil sebagai benar-benar Allah.

\kgk{Apa Allah itu satu-satunya yang ”ada”?}
Karena makhluk menerima segalanya dari Allah, mereka ada dan kepunyaan
mereka dari Allah. Hanya Allah dalam Diri-Nya sendiri merupakan kepenuhan dari
yang ada dan dari setiap kesempurnaan. Allah itu ”Dia yang ada” tanpa awal dan
tanpa akhir. Yesus mewahyukan bahwa Ia juga menyandang nama ilahi ”Aku ada”
 (Yoh 8:28).

\small

\kgk{Mengapa pewahyuan Nama Allah itu penting?}
Dalam mewahyukan nama-Nya, Allah memberitahukan kekayaan yang
 ada di dalam misteri ada-Nya yang tak terkatakan. Hanya Dia sendirilah yang
 dari kekal sampai kekal. Dia mengatasi dunia dan sejarah. Dialah yang membuat
 langit dan bumi. Dia adalah Allah yang setia yang selalu dekat dengan umat-Nya
 untuk menyelamatkan mereka. Dialah kekudusan tertinggi, ”penuh dengan belas
 kasihan” (Ef 2:4), selalu siap untuk mengampuni. Dialah yang spiritual, transenden,
 mahakuasa, personal, dan sempurna. Dia adalah kebenaran dan cinta.

\section*{Seksi Dua: Pengakuan Iman Kristen}

\kgk{Apa artinya bahwa Allah adalah Kebenaran?}
 Allah adalah Kebenaran, dengan demikian Dia tidak dapat menipu ataupun 
ditipu. Dia adalah ”terang, dan di dalam-Nya tidak ada kegelapan” (1Yoh 1:5). Putra 
Allah yang kekal, penjelmaan kebijaksanaan, diutus ke dunia untuk ”memberikan
kesaksian akan Kebenaran” (Yoh 18:37).

\flushright{(\dots \emph{bersambung} \dots)}
\normalsize