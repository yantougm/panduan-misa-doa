\chap{Apa yang harus kuketahui tentang Liturgi?} 

\section*{Pendahuluan}

Saya pernah mendengar bahwa ada orang-orang yang mengatakan liturgi di Gereja Katolik itu ‘membosankan’. Katanya lagu-lagunya itu-itu saja, kurang bersemangat dan kurang berkesan. Apa iya, demikian halnya? Sebelum berkomentar, mari kita lihat dulu apa sebenarnya arti liturgi di dalam Gereja Katolik. Lalu, setelah itu baru kita tilik kembali komentar itu. Sebab, jangan-jangan masalahnya bukan pada liturgi-nya tetapi pada diri si penerima. Ibaratnya, “kesalahan bukan pada stasiun pemancar radio, tetapi pada antena Anda.” Walaupun demikian, mari kita lihat juga apa yang perlu kita lakukan supaya kita dapat menghayati liturgi dan menjadikannya bagian dari diri kita, supaya kita tidak sampai bosan. Ini adalah bentuk “perbaikan antena” sehingga radio kita dapat menangkap sinyal dengan lebih baik.

\section*{Pengertian liturgi}

Telah kita ketahui bahwa sakramen adalah penghadiran Misteri Kristus (lihat artikel: Sakramen: Apa pentingnya dalam kehidupan iman kita?). Di dalam liturgi, Gereja merayakan Misteri Paskah Kristus yaitu sengsara, wafat, kebangkitan dan kenaikan Yesus ke surga- yang membawa kita kepada Keselamatan. Dengan merayakan Misteri Kristus ini, kita memperingati dan merayakan bagaimana Allah Bapa telah memenuhi janji dan menyingkapkan rencana keselamatan-Nya dengan menyerahkan Yesus Putera-Nya oleh kuasa Roh Kudus untuk menyelamatkan dunia. Jadi sumber dan tujuan liturgi adalah Allah sendiri.

Liturgi pada awalnya berarti “karya publik”. Dalam sejarah perkembangan Gereja, liturgi diartikan sebagai keikutsertaan umat dalam karya keselamatan Allah. Di dalam liturgi, Kristus melanjutkan karya Keselamatan di dalam, dengan dan melalui Gereja-Nya. Pada jaman Gereja awal seperti dijabarkan di dalam surat rasul Paulus, para pengikut Kristus beribadah bersama di dalam liturgi (dikatakan sebagai “korban dan ibadah iman” di dalam Flp 2:17). Termasuk di sini adalah pewartaan Injil “(Rom 15:16); dan pelayanan kasih (2 Kor 9:12). Maka, dalam Perjanjian Baru, kata ‘liturgi’ mencakup tiga hal, yaitu ibadat, pewartaan dan pelayanan kasih yang merupakan partisipasi Gereja dalam meneruskan tugas Kristus sebagai Imam, Nabi dan Raja.

Secara khusus, liturgi merupakan wujud pelaksanaan tugas Kristus sebagai Imam Agung. Dalam hal ini, liturgi merupakan penyembahan Kristus kepada Allah Bapa, namun dalam melakukan penyembahan ini, Kristus melibatkan TubuhNya, yaitu Gereja; sehingga liturgi merupakan karya bersama antara Kristus (Sang Kepala) dan Gereja (Tubuh Kristus). Konsili Vatikan II mengajarkan pengertian tentang liturgi sebagai berikut:

“Maka, benarlah bahwa liturgi dipandang sebagai pelaksanaan tugas imamat Yesus Kristus. Di dalam liturgi, dengan tanda-tanda lahiriah,  pengudusan manusia dilambangkan dan dihasilkan dengan cara yang layak bagi masing-masing tanda ini; di dalam Liturgi, seluruh ibadat publik dilaksanakan oleh Tubuh Mistik Yesus Kristus, yakni Kepala beserta para anggota-Nya.
Oleh karena itu setiap perayaan liturgis sebagai karya Kristus sang Imam serta Tubuh-Nya yakni Gereja, merupakan kegiatan suci yang sangat istimewa. Tidak ada tindakan Gereja lainnya yang menandingi daya dampaknya dengan dasar yang sama serta dalam tingkatan yang sama.”

Oleh karena itu tidak ada kegiatan Gereja yang lebih tinggi nilainya daripada liturgi karena di dalam liturgi terwujudlah persatuan yang begitu erat antara Kristus dengan Gereja sebagai ‘Mempelai’-Nya dan Tubuh-Nya sendiri.

Paus Pius XII dalam surat ensikliknya tentang Liturgi Suci, Mediator Dei, menjabarkankan definisi liturgi sebagai berikut:

\begin{quote}
\emph{“Liturgi adalah ibadat publik yang dilakukan oleh Penebus kita sebagai Kepala Gereja kepada Allah Bapa dan juga ibadat yang dilakukan oleh komunitas umat beriman kepada Pendirinya [Kristus], dan melalui Dia kepada Bapa. Singkatnya, liturgi adalah ibadat penyembahan yang dilaksanakan oleh Tubuh Mistik Kristus secara keseluruhan, yaitu Kepala dan anggota-anggotanya.”}
\end{quote}

Atau, dengan kata lain, definisi liturgi adalah seperti yang dirumuskan oleh Rm. Emanuel Martasudjita, Pr. dalam bukunya Liturgi, yaitu: “\textit{Liturgi adalah perayaan misteri karya keselamatan Allah di dalam Kristus, yang dilaksanakan oleh Yesus Kristus, Sang Imam Agung, bersama Gereja-Nya di dalam ikatan Roh Kudus.}”
Allah Bapa: Sumber dan Tujuan Liturgi

Alkitab mengatakan, “Terpujilah Allah Bapa Tuhan kita Yesus Kristus yang dalam Kristus telah mengaruniakan kepada kita segala berkat rohani di dalam sorga. Sebab di dalam Dia Allah telah memilih kita sebelum dunia dijadikan, supaya kita kudus dan tak bercacat di hadapan-Nya. Dalam kasih Ia telah menentukan kita dari semula oleh Yesus Kristus untuk menjadi anak-anak-Nya sesuai dengan kerelaan kehendak-Nya supaya terpujilah kasih karunia-Nya yang mulia, yang dikaruniakan-Nya kepada kita di dalam Dia yang dikasihi-Nya” (Ef 1:3-6). Dari sini kita mengetahui bahwa Allah Bapalah yang memberikan rahmat sorgawi kepada kita, melalui Kristus dan di dalam Kristus. Dan karena rahmat itu diberikan di dalam sakramen melalui liturgi, maka sumber liturgi adalah Allah Bapa, dan tujuan liturgi adalah kemuliaan Allah.

\section*{Kristus Bekerja di dalam Liturgi}

Karena Kristus telah bangkit mengalahkan maut, maka, Ia yang telah duduk di sisi kanan Allah Bapa, pada saat yang sama dapat terus mencurahkan Roh Kudus-Nya kepada Tubuh-Nya, yaitu Gereja-Nya, melalui sakramen-sakramen. Karena Yesus sendiri yang bertindak dengan kuasa Roh Kudus-Nya, maka kita tidak perlu meragukan efeknya, karena pasti Kristus mencapai maksud-Nya.

Puncak karya Kristus adalah Misteri Paska-Nya, maka Misteri Paska inilah yang dihadirkan di dalam liturgi Gereja. Jadi Misteri Paska yang sungguh-sungguh telah terjadi di masa lampau dihadirkan kembali oleh kuasa Roh Kudus. Karena Kristus telah menang atas kuasa dosa dan maut, maka Misteri Paska-Nya tidak berlalu begitu saja ditelan waktu, namun dapat dihadirkan kembali oleh kuasa Ilahi, yang mengatasi segala tempat dan waktu. Hal ini dilakukan Allah karena besar kasih-Nya kepada kita, sehingga kita yang tidak hidup pada masa Yesus hidup di dunia dapat pula mengambil bagian di dalam kejadian Misteri Paska Kristus dan menerima buah penebusan-Nya.

Kristus selalu hadir di dalam Gereja, terutama di dalam perayaan liturgi. Pada perayaan Ekaristi/ Misa kudus, Kristus tidak hanya hadir di dalam diri imam-Nya, namun juga di dalam wujud hosti kudus (lihat artikel: Sudahkah kita pahami arti Ekaristi?). Liturgi di dunia menjadi gambaran liturgi surgawi di mana Yesus duduk di sisi kanan Allah Bapa, dan kita semua sebagai anggota Gereja memuliakan Allah bersama seluruh isi surga.

\section*{Roh Kudus dan Gereja di dalam Liturgi}

Jika Roh Kudus bekerja di dalam diri seseorang, maka Ia akan menggerakkan hati orang tersebut untuk bekerjasama dengan Allah. Kita dapat melihat hal ini pada teladan Bunda Maria dan para Rasul. Demikian halnya liturgi menjadi hasil kerjasama Roh Kudus dengan kita sebagai anggota Gereja. Kerjasama Roh Kudus dan Gereja ini menghadirkan Kristus dan karya keselamatan-Nya di dalam liturgi, sehingga liturgi bukan sekedar ‘kenangan’ akan Misteri Kristus, melainkan adalah kehadiran Misteri Kristus yang satu-satunya itu.

Peran Roh Kudus dinyatakan pada saat pembacaan Sabda Allah, karena Roh Kudus menjadikan Sabda itu dapat diterima dan dilaksanakan di dalam hidup umat. Kemudian Roh Kudus memberikan pengertian rohani terhadap Sabda Tuhan itu, yang menghidupkan perkataan doa, tindakan dan tanda-tanda lahiriah yang dipergunakan dalam liturgi, dan dengan demikian Roh Kudus menghidupkan hubungan antara umat (beserta para imam) dengan Kristus. Selanjutnya peran Roh Kudus nyata saat konsekrasi, yaitu saat roti dan anggur diubah menjadi Tubuh dan Darah Kristus. Di sinilah puncak perayaan Ekaristi terjadi, saat Kristus berkenan menghadirkan Diri di tengah Gereja-Nya.

Oleh karena itu Sang Pelaku yang utama dalam liturgi adalah Kristus, dan kita sebagai anggota Gereja mengambil bagian di dalam karya keselamatan Allah yang dilakukan oleh Kristus itu. Dengan demikian bukan kita pribadi yang dapat menentukan segala sesuatunya dalam liturgi menurut kehendak sendiri, melainkan kita sepantasnya mengikuti apa yang telah ditetapkan oleh Tuhan Yesus dalam perayaan tersebut, sebagaimana yang telah dilakukan oleh para rasul dan diteruskan dengan setia oleh para penerus mereka.

\section*{Kristus mengajak kita ikut serta mengambil bagian dalam Misteri Keselamatan-Nya}

Yesus mengajak kita semua ikut mengambil bagian dalam karya keselamatan-Nya, terutama dalam Misteri Paska-Nya yang dihadirkan kembali di dalam Liturgi. Karena kuasa kasih dan kebangkitan-Nya, Kristus memberikan kita kesempatan yang sama dengan orang-orang yang hidup pada zaman Ia hidup di dunia 2000 tahun yang lalu, yaitu menyaksikan dan ikut mengambil bagian dalam peristiwa yang mendatangkan keselamatan kita, yaitu wafatNya di salib, kebangkitan-Nya dan kenaikan-Nya ke surga. Secara khusus penghadiran Misteri Paska ini nyata dalam Ekaristi, yang merupakan penghadiran kurban Kristus yang sama dan satu-satunya itu oleh kuasa Roh Kudus. Kuasa Roh Kudus yang dulu menghadirkan Yesus dalam rahim Maria, kini hadir untuk menghadirkan Yesus di altar. Kuasa Roh Kudus yang dulu hadir pada hari Pentakosta kini hadir di dalam setiap perayaan Ekaristi, untuk mengubah kita menjadi seperti para rasul, dipenuhi kasih dan semangat yang berkobar untuk ikut serta melakukan pekerjaan-pekerjaan Allah di dunia ini.

Jika kita menghayati kebenaran ini, kita seharusnya tidak bosan dan mengantuk dalam mengikuti misa. Sebab jika demikian, kita seumpama mereka yang hidup di jaman Yesus, hadir di bawah kaki salib Yesus, tetapi malah melamun dan tidak mempunyai perhatian akan apa yang sedang terjadi di hadapan mata mereka. Sungguh tragis, bukan? Memang Misteri Paska itu tidak hadir persis secara fisik seperti 2000 tahun lalu, namun secara rohani, Misteri Kristus yang sama dan satu-satunya itu hadir dan membawa efek yang sama seperti pada 2000 tahun yang lalu. Betapa dalamnya makna dari misteri ini, namun kita perlu menilik ke dalam hati kita yang terdalam untuk melihatnya dengan mata rohani dan menghayatinya dengan sikap tunduk dan kagum.

\section*{Bagaimana sikap kita di dalam liturgi}

Bayangkan jika Anda secara pribadi diundang pesta oleh Bapak Presiden. Tentu Anda akan mempersiapkan diri sebaik-baiknya bukan? Anda akan berpakaian yang sopan, bersikap yang pantas, mempersiapkan apa yang akan Anda bicarakan, dan Anda akan datang tidak terlambat, jika perlu siap sebelum waktunya. Mari kita memeriksa diri, sudahkah kita bersikap demikian di dalam ‘pertemuan’ kita dengan Tuhan di dalam liturgi. Karena Tuhan jauh lebih mulia dan lebih penting daripada Bapak Presiden, seharusnya persiapan kita jauh lebih baik daripada persiapan bertemu dengan Presiden.
\\{~}\\
\begin{enumerate}[label=\textbf{Langkah \arabic*}]
\item Mempersiapkan diri sebelum mengikuti liturgi dan mengarahkan hati sewaktu mengikuti liturgi
\\{~}\\
Untuk menyadari kedalaman arti misteri ini, kita harus mempersiapkan diri dengan sungguh-sungguh sebelum mengambil bagian di dalam liturgi. Persiapan ini dapat berbentuk: membaca dan merenungkan ayat kitab suci pada hari itu, hening di sepanjang jalan menuju ke gereja, datang di gereja lebih awal, berpuasa ( 1 jam sebelum menyambut Ekaristi dan terutama berpuasa sebelum menerima sakramen Pembaptisan dan Penguatan), memeriksa batin, mengaku dosa dalam sakramen Tobat sebelum menerima Ekaristi.
\\{~}\\
Lalu, sewaktu mengikuti liturgi, kitapun harus senantiasa mengarahkan sikap hati yang benar. Jika terjadi ‘pelanturan’, segeralah kita kembali mengarahkan hati kepada Tuhan. Kita harus mengarahkan akal budi kita untuk menerima dengan iman bahwa Yesus sendirilah yang bekerja melalui liturgi, dan bahwa Roh KudusNya menghidupkan kata-kata doa dan teks Sabda Tuhan yang diucapkan di dalam liturgi, sehingga menguduskan tanda-tanda lahiriah yang dipergunakan di dalam liturgi untuk mendatangkan rahmat Tuhan.
\\{~}\\
Sikap hati ini dapat diwujudkan pula dengan berpakaian yang sopan, tidak ‘ngobrol’ pada saat mengikuti liturgi, dan tidak menyalakan hp/ mengangkat telpon di gereja. Sebab jika demikian dapat dipastikan bahwa hati kita tidak sepenuhnya terarah pada Tuhan.
\\{~}\\
\item Bersikap aktif: jangan hanya menerima tetapi juga memberi kepada Tuhan
\\{~}\\
St. Thomas Aquinas mengajarkan bahwa penyembahan yang sempurna itu mencakup dua hal, yaitu menerima dan memberikan berkat-berkat ilahi. Di dalam liturgi, penyembahan kita kepada Tuhan mencapai puncaknya, saat kita kita turut memberikan/ mempersembahkan diri kita kepada Tuhan dan pada saat kita menerima buah dari penebusan Kristus melalui Misteri Paska-Nya. Puncak liturgi adalah Ekaristi, di mana di dalam Misteri Paska yang dihadirkan kembali itu, Kristus menjadi Imam Agung, dan sekaligus Kurban penebus dosa.\\{~}\\
Dalam liturgi Ekaristi, kita sebagai anggota Tubuh Kristus seharusnya tidak hanya ‘menonton’ atau sekedar menerima, tetapi ikut mengambil bagian dalam peran Kristus sebagai Imam Agung dan Kurban tersebut. Caranya adalah dengan turut mempersembahkan diri kita, beserta segala ucapan syukur, suka duka, pergumulan, dan pengharapan, untuk kita persatukan dengan kurban Kristus. Setiap kali menghadiri misa, kita bawa segala kurban persembahan diri kita untuk diangkat ke hadirat Tuhan, terutama pada saat konsekrasi, yaitu saat kurban roti dan anggur diubah menjadi Tubuh dan Darah Yesus. Dengan demikian kurban kita akan menjadi satu dengan kurban Yesus. Oleh karena itu, liturgi menjadi penyembahan yang sempurna karena Kristus yang adalah satu-satunya Imam Agung dan Kurban yang sempurna, menyempurnakan segala penyembahan kita. Bersama Yesus di dalam liturgi kita akan sungguh dapat menyembah Allah Bapa di dalam roh dan kebenaran (Yoh 4:24), karena di dalam liturgi kuasa Roh Kudus bekerja menghadirkan Kristus yang adalah Kebenaran itu sendiri.
\\{~}\\
Hal kehadiran Yesus tidak hanya terjadi dalam Ekaristi, tetapi juga di dalam liturgi yang lain, yaitu Pembaptisan, Penguatan, Pengakuan Dosa, Perkawinan, Tahbisan suci, dan Pengurapan orang sakit. Dalam liturgi tersebut, kita harus berusaha untuk aktif berpartisipasi agar dapat sungguh menghayati maknanya. Partisipasi aktif ini bukan saja dari segi ikut menyanyi, atau membaca segala doa yang tertulis, melainkan terutama partisipasi dari segi mengangkat hati dan jiwa untuk menyembah dan memuji Tuhan, dan meresapkan segala perkataan yang diucapkan di dalam hati.
\\{~}\\
\item Jangan memusatkan perhatian pada diri sendiri tetapi pada Kristus
\\{~}\\
Jadi, agar dapat menghayati liturgi, kita harus memusatkan perhatian kita kepada Kristus, dan pada apa yang telah dilakukanNya bagi kita, yaitu: oleh kasihNya yang tak terbatas, Kristus tidak menyayangkan nyawa-Nya dan mau wafat bagi kita untuk menghapus dosa-dosa kita. Kita bayangkan Yesus sendiri yang hadir di dalam liturgi dan berbicara sendiri kepada kita. Dengan berfokus pada Kristus, kita akan memperoleh kekuatan baru, sebab segala pergumulan kita akan nampak tak sebanding dengan penderitaan-Nya. Kitapun akan dikuatkan di dalam pengharapan karena percaya bahwa Roh Kudus yang sama, yang telah membangkitkan Yesus dari kubur akan dapat pula membangkitkan kita dari pengaruh dosa dan segala kesulitan kita.
\\{~}\\
Jika kita memusatkan hati dan pikiran pada Kristus, maka kita tidak akan terlalu terpengaruh jika musik atau penyanyi di gereja kurang sempurna, khotbah kurang bersemangat, kurang keakraban ataupun hawa panas dan banyak nyamuk. Walaupun tentu saja, idealnya semua hal itu sedapat mungkin diperbaiki. Kita bahkan dapat mempersembahkan kesetiaan kita disamping segala ketidak sempurnaan itu- sebagai kurban yang murni bagi Tuhan. Langkah berikutnya adalah, apa yang dapat kita lakukan untuk turut membantu memperbaiki kondisi tersebut. Inilah salah satu cara menghasilkan ‘buah’ dari penerimaan rahmat Tuhan yang kita terima melalui liturgi.
\end{enumerate}

\section*{Liturgi adalah sumber kehidupan}

Jadi sebagai karya Kristus, liturgi menjadi kegiatan Gereja di mana Kristus hadir dan membagikan rahmat-Nya, yang menjadi sumber kehidupan rohani kita. Walaupun demikian, liturgi harus didahului oleh pewartaan Injil, iman dan pertobatan, sebab tanpa ketiga hal tersebut akan sangat sulit bagi kita untuk menghayati perayaan liturgi, apalagi menghasilkan buahnya dalam kehidupan sehari-hari. Ibaratnya tak kenal maka tak sayang, maka jika kita ingin menghayati liturgi, maka sudah selayaknya kita mengetahui makna liturgi, menerimanya dengan iman dan menanggapinya dengan pertobatan.

Liturgi yang bersumber pada Allah menjadi sumber dan puncak kegiatan Gereja. Bersumber pada liturgi ini, Gereja menimba kekuatan untuk melaksanakan pembaharuan di dalam Roh, misi perutusan, dan menjaga persatuan umat. Maka jika kita mengalami ‘kemacetan ataupun percekcokan’ di dalam kegiatan paroki, petunjuk praktis untuk memeriksa adalah: Sudah cukupkah keterlibatan anggota dalam Ekaristi -tiap minggu atau jika mungkin setiap hari? Adakah kedisiplinan anggota untuk mengaku dosa di dalam Sakramen Tobat secara teratur, misalnya sebulan sekali? Walaupun demikian, kehidupan rohani kita tidak terbatas hanya dari keikutsertaan dalam liturgi, tetapi juga dari kehidupan doa yang benar (doa pribadi (Mat 6:6) dan doa tanpa henti (1Tes 5:17)).

\section*{Kesimpulan}

Seperti telah diuraikan di atas: liturgi merupakan partisipasi kita di dalam doa Kristus kepada Allah Bapa oleh kuasa Roh Kudus. Liturgi terutama Ekaristi yang menghadirkan Misteri Paska Kristus merupakan peringatan akan karya Allah Tritunggal untuk mendatangkan keselamatan bagi dunia. Maka liturgi merupakan puncak kegiatan Gereja, dan sumber di mana kuasa Gereja dicurahkan, yaitu kehidupan baru di dalam Roh, keikutsertaan di dalam misi perutusan Gereja dan pelayanan terhadap kesatuan Gereja. Jadi bagi kita umat beriman, terutama yang ikut ambil bagian di dalam karya kerasulan awam, keikutsertaan di dalam liturgi merupakan sesuatu yang utama. Tidak bisa kita melayani umat, jika kita sendiri tidak diisi dan diperbaharui oleh rahmat Tuhan sendiri. Prinsipnya, “kita tidak bisa memberi, jika kita tidak terlebih dahulu menerima” rahmat yang dari Allah.

Rahmat Allah ini secara nyata kita terima melalui liturgi. Dalam hal ini, Ekaristi memegang peranan penting karena di dalamnya rahmat yang diberikan adalah Kristus sendiri. Kini tinggal giliran kita untuk memeriksa diri dan mempersiapkan hati untuk menerima berkat rahmat itu. Jika kita mempunyai sikap hati yang benar dan berpartisipasi aktif di dalam liturgi, maka Tuhan sendiri akan memberkati dan menjadikan kita anggota TubuhNya yang menghasilkan buah bagi kemuliaan nama-Nya. Menimba bekal rohani melalui liturgi merupakan salah satu cara yang paling nyata untuk menjawab undangan Tuhan Yesus, “Tinggallah di dalam Aku dan Aku di dalam kamu…. Barang siapa tinggal di dalam Aku dan Aku di dalam dia, ia berbuah banyak, sebab di luar Aku kamu tidak dapat berbuat apa-apa” (Yoh 15:4-5).
