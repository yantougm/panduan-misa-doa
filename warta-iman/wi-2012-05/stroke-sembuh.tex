\chap{\textit{Cerpen}\\Stroke sembuh: Terpujilah kristus!}
\small

	Tidak ada angin tidak ada hujan, suatu sore datang Mas Martin ke rumah saya. Nama lengkapnya Martinus Hendro Winarto. Sudah lama sekali kami tidak saling bertemu. Mungkin ada sekitar 15 tahun. Dulu sering bersama-sama nonton teater, diskusi soal sastra, jurnalistik dan kebudayaan. Tetapi kedatangannya sore itu tidak ada kaitannya dengan seni dan budaya. Ia datang minta tolong kepada saya agar mau membantu menyembuhkan ayahnya yang sedang sakit karena serangan stroke.

	“Lho, piye to \ldots kok mas Martin datang pada saya? Ya mbok ke dokter atau rumah sakit! Saya ini kan bukan paranormal atau penyembuh alternatif!” kata saya benar-benar heran.

	“Jangan merendah to, Mas Agung. Rendah hati memang bagus, tetapi jauh lebih bagus kalau kamu mau menolongku.” Kata mas Martin.

	Saya hanya menggeleng-gelengkan kepala. “Wee, lha edan tenan iki!” kata saya dalam hati. Mas Martin lalu cerita bahwa sudah tiga bulan ayahnya tidak berdaya karena stroke. Pengobatan ke dokter di rumah sakit sudah rutin dilakukan tetapi hasil kesembuhannya belum maksimal. Ayahnya masih bisa jalan pelan-pelan, namun harus dibimbing. Padahal umurnya baru 56 tahun. Untunglah, semangat hidup pak Paulus Pujo Winarto, ayahnya itu, masih tinggi. Pernah ia minta dibawa ke Sendangsono, meski harus didorong dengan kursi roda hingga di depan Bunda Maria. Sembilan kali dia dibawa ke Sendangsono dan pada ziarah yang ke Sembilan pak Paulus dimandikan dengan air sendang.

	“Tetapi belum sembuh juga. Karena itu mewakili keluarga, saya datang kemari minta pertolongan padamu.” Kata mas Martin sungguh-sungguh.

	Meski saya yakinkan berkali-kali jika saya ini tidak bisa menyembuhkan penyakit, bahkan saya ini termasuk kategori manusia tidak sehat, tetapi mas Martin tetap memaksa. “Apapun yang kamu katakan akan kami turuti. Yang penting ayah saya bisa sembuh!” begitu kata mas Martin sebelum pamit pulang. 

\section*{Bisikan Bunda}

	Siapa tidak pusing tujuh keliling jika tiba-tiba mendapat “cobaan” seperti itu? Sumpah mati saya belum pernah mempelajari metode penyembuhan model apapun. Saya sendiri termasuk orang yang sakit-sakitan. Ini buah dari hobi melamun, duduk-duduk sambil membaca dan agak alergi dengan olah raga. Lalu tiba-tiba diminta tolong untuk menyembuhkan orang sakit stroke.

	Bunda Maria, bagaimana anakmu ini? Bukan paranormal, bukan dukun, bukan akupunturis, bukan ahli prana, bukan dokter, tidak punya kesaktian, lalu harus menyembuhkan orang sakit.

	Tuhan Yesus, Engkau sering membuat mujizat. Menyembuhkan orang lumpuh, orang buta, perempuan yang pendarahan, orang kerasukan setan, bahkan menghidupkan orang mati. Engkau bisa sebab Engkau adalah Putra Allah. Lalu saya ini apa? Pendosa yang ringkih. Hamba yang tidak setia. Sahabat yang sering berkhianat.

	Santo Petrus berani berbuat seperti Diri-Mu, menyembuhkan orang lumpuh sebab dia adalah rasul-Mu, yang pernah hidup -Mu, pernah menjamah jubah-Mu, makan bersama dan duduk berdekatan dengan-Mu. Karena itu dia pasti ‘ketularan’ kesaktian-Mu.

	Semalam suntuk saya tidak bisa tidur. Esoknya dengan sepeda motor butut saya lari ke Sendangsono. Seharian penuh saya duduk didepan patung Bunda Maria. Segala kebingungan saya tumpahkan di depan Bunda Maria.

	Pukul tiga sore, antara tidur dan terjaga, saya seperti mendengar bisikan halus: \textit{Bawalah kemari. Biarkan dia berjalan sendiri.} Dalam perjalanan pulang bisikan itu terus mengiang-ngiang di telinga saya. Dan bahkan saya yakin itu bukan bisikan roh halus yang tinggal di pohon angsana dan pohon beringin di Sendangsono. Itu adalah bisikan Bunda Maria!

\section*{Mulailah berjalan sendiri}

	“Mas Agung, apa yang harus kami lakukan?” Tanya mas Martin lima hari kemudian. “katakan apa saja, asal kami sanggup melakukan, pasti kami lakukan. Yang penting ayah kami bisa sembuh!” lanjutnya serius. Ia datang bersama dengan isteri dan adiknya.

	“Mas Martin, saya hanya seorang prodiakon Gereja. Pekerjaanku hanya seorang wartawan. Saya tidak mampu menyembuhkan ayah panjenengan. Dokter yang ahlinya saja belum mampu menyembuhkan apalagi saya yang ‘gebleg’ ini, kecuali \ldots. Hanya satu orang yang mampu menyembuhkan ayah panjengan, yaitu Yesus Kristus!” kata saya tegas. 

	“Justru karena itu, saya datang kepadamu. Karena kamu adalah orang yang ‘dekat’ dengan Kristus. Melalui kamulah, semoga Kristus berbelas kasih mau menyembuhkan ayahku. Please \ldots.” Kata Martin penuh keyakinan.

	Bisikan Bunda Maria kembali mengiang-ngiang ditelingaku. Rasanya saya yakin itu adalah bisikan Bunda Maria. Entah sadar atau tidak, aku mengangguk-angguk, tiba-tiba saya teringat bacaan Kitab Suci yang semalam saya baca: “\textit{Siapakah yang membuat lidah manusia, siapakah yang membuat orang bisu atau tuli, membuat orang melihat atau buta; bukankah Aku, yakni Tuhan? Oleh sebab itu, pergilah, aku akan menyertai lidahmu dan mengajar engkau, apa yang harus kau katakan.}” (Kel 4: 11-12)

	“Baiklah kalau kamu memaksa, tetapi ingat bahwa saya tak punya kemampuan keparanormalan. Nasihat ini jika bisa dicoba, tetapi jika dianggap berat, ya jangan dilaksanakan.” Kata saya kemudian. Mas martin, isteri dan asiknya mengangguk-angguk. 

	“Ajaklah ayah Mas martin berjalan sendiri di jalan menuju sendangsono. Mungkin bisa dimulai dari Kalibawang menuju Gereja Promasan, atau langsung dari Gereja Promasan menuju Sendangsono. Yakinkan kepada beliau bahwa Bunda Maria telah menunggu disana. Tentu harus setahap demi setahap, harus telaten. Mungkin seminggu tiga kali, yang penting, ayah panjenengan yakin dan mantap.”

	Kembali entah sadar atau tidak, seperti refleks,  saya mengambil kertas dan menulis: “\textit{Karena Engkaulah pelitaku, ya Tuhan, dan Tuhan menyinari kegelapanku. Karena dengan Engkau, aku berani menghadapi gerombolan, dengan Allahku, aku berani melompati tembok.}”  

	“Kutipan ayat dari Samuel 22: 29 ini bisa diucapkan sertiap kali beliau mau mulai langkah-langkah penyembuhan.” Kataku sambil memberikan kertas itu kepada mas Martin. Mas Martin mengangguk-angguk.

	“Yakinkan juga kepada ayah mas Martin bahwa Tuhan senantiasa menyertai setiap langkahnya,” lanjut saya.

	“Tidak ada jamu atau ramuan yang harus diminum?” Tanya Felix, adik mas Martin.

	“Obat dari dokter tetap  terus diminum. Ajaklah mengikuti Misa Kudus setiap pagi. Terlebih sebelum berangkat ke Sendangsono.” Jawab saya mantap.

	“Hanya itu?” Tanya mas Martin.

	“Semakin bertekunlah dalam doa. Jangan menuntut sesuatu kepada Tuhan, namun berpasrahlah selalu.” Jawabku. Aneh. Waktu mengatakan semua itu tidak ada keraguan sedikitpun pada lidah saya.

Mereka mengangguk-angguk lalu pamit pulang.
“Semoga iman kalian akan Tuhan yang maha kasih dan maha pemurah menyembuhkan si sakit,” kata saya dalam hati.

\section*{Ada cahaya gaib}
	Tiga hari kemudian saya harus pergi keluar kota dalam waktu yang cukup lama karena tuntutan tugas pekerjaan. Setiap malam saya berdoa untuk kesembuhan ayah mas Martin. Saya hanya mampu membayangkan, bagaimana orangtua yang terserang stroke itu menggerak-gerakan kakinya untuk melangkah selangkah demi selangkah.

	Tiga minggu kemudian Mas martin menelpon saya, “Banyak kemajuan. Malah perkembangannya pesat!” kata mas Martin dengan nada gembira. “sekarang ayah sudah bisa berjalan sendiri, tanpa dipapah, tanpa tongkat penyangga. Dua hari sekali beliau kami ajak ke mulut jalan menuju Sendangsono. Seperti katamu, kami katakan bahwa Bunda Maria sudah menunggu ayah. Dia bersemangat sekali, meski pada hari pertama dia sempat terkencing-kencing. Ha\ldots ha \ldots”

	“Puji Tuhan! Dia memang maha pemurah, maha pengasih dan maha penyayang. Bulan depan saya sudah pulang. Kita bisa cerita panjang lebar!” ujarku dengan gembira.

	“kami sangat menunggu kehadiranmu,” balas Mas Martin akhirnya.

	Ketika saya kembali kerumah setelah tugas diluar kota selesai, esoknya mas Martin langsung mengajak saya ke Sendangsono. Kami langung menuju rumah penduduk  yang tak jauh dari Gereja Promasan. Disitu pak Paulus Pujo Winarto, ayah mas martin, menginap. Begitu ketemu, orangtua itu langsung memeluk saya erat-erat. Meski baru sekali itu ketemu, dia tampak menangis karena terharu.

	“Stroke saya sembuh, Nak,” katanya setelah berhenti menangis. “Saya sudah bisa berjalan sendiri. Meski belum sampai di sendangsono, namun saya sudah bisa jalan salib sendiri sampai di pemberhentian yang ke VIII,” tutur pak Pujo bangga.

	“Puji Tuhan!” seruku gembira. “Tuhan benar-benar memberi kekuatan kepada bapak.” Sambut saya.

	“Ada peristiwa aneh, nak. Jika malam hari saya berlatih jalan kaki sendiri, diatas sana seperti ada cahaya. Warnanya indah, kebiru-biruan. Cahaya itu saya perkirakan memancar persis diatas lokasi sendangsono. Karena itu saya jadi tambah semangat. Mungkinkah cahaya itu berasal dari Bunda Maria sendiri?” Tanya pak Pujo.

	“Mungkin cahaya itu adalah kasih Allah yang memancar lewat telapak tangan Bunda Maria. Cahaya itu memanggil pak Pujo untuk datang kesana.” Jawab saya.

	“Ya,ya. Maka saya mantap sekali jika membaca ayat dari Samuel 22: 29. Ayat itu saya anggap seperti mantra. Saya baca berulang-ulang, saya jadikan doa penguat dan pembakar semangat. Karena itulah saya sekarang jauh merasa lebih sehat. Rasanya stroke itu sudah pergi dari tubuh saya.” Tutur pak Pujo gembira.

	“Puji Tuhan! Alleluya!” sahut saya turut merasakan kegembiraan.

	Dua minggu kemudian, pak Pujo benar-benar bebas dari stroke. Dia sudah mampu doa jalan salib sendiri dari Gereja Promasan sampai di sendangsono. Meski harus istirahat beberapa kali. Di depan Bunda maria, lelaki itu menangis sejadi-jadinya. Tangis kegembiraan. Kegembiraan hati orang yang beriman kepada Tuhan Yesus, Putra Allah Bapa, dan sangat percaya dengan kasih dan pertolongan Bunda Maria. Setelah tangisnya reda, pak Pujo lalu berdoa Rosario. Doa itu didaraskan sampai sepuluh kali. Usai berdoa, dengan wajah berbinar, ia memeluk orang-orang yang ada disekitarnya. Isterinya, empat orang anaknya, dua menantu, tiga cucunya dan saya sendiri.

	“Bunda Maria, cahaya gaibmu telah menuntunku sampai disini. Terima kasih Bunda, engkau telah pula memintakan kepada Puteramu, Tuhan Yesus Kristus, untuk kesembuhan diriku,” ucap pak Pujo sambil bersujud.

	Diam-diam saya menarik lengan mas Martin lalu saya ajak menjauh dari sana.  “Mas Agung, terima kasih, karena bantuanmulah ayahku sembuh,” ujar mas Martin sambil menjabat tanganku. 

	“Oke \ldots. Hanya satu pintaku, tolong jangan ceritakan kepada siapa-siapa akan kejadian ini. Aku bukan apa-apa. Ayahmu bisa sembuh dari stroke karena ayahmu benar-benar yakin dan percaya akan rahmat dan belas kasih Tuhan Yesus.  Aku tidak ingin ada ‘pasien’ lagi yang datang padaku minta penyembuhan. Bisa gila aku, Martin! Please\ldots.!” Pintaku kepada mas Martin.

“Hahaha \ldots. Gak janji deh” ujar mas Martin sambil berlari lalu menjulurkan lidahnya kepadaku  karena kuacungkan tinjuku kearahnya,  lalu kami bergabung kembali dengan keluarganya yang sedang bersyukur kepada Tuhan dengan penuh kegembiraan.
	
\sumber{Medio, April’12\\Bravo Sierra}





  
