\newpage
\section*{Sepenggal Kisah dari OMK Ling. St.Petrus}
\subsection*{\emph{Oleh Yacintha Ferry Kurnia}}

Beberapa bulan yang lalu lingkungan St.Petrus baru saja merayakan pesta nama pelindung, untuk memeriahkan acara tersebut OMK St. Petrus, yang memang saat itu sedang memiliki program untuk melestarikan musik tradisional Jawa, yaitu Gamelan, diberi kesempatan untuk tampil perdana memainkan gamelan mengiringi misa pada perayaan pesta nama di gereja Bunda Maria Maguwo. 

	Begitu banyak kisah dibalik penampilan perdana mereka. Mulai dari mereka telat datang latihan hingga masalah baju surjan yang kegedean. Namun itu semua tidak menghalangi kami untuk berkarya dan menunjukkan yang terbaik. Awalnya ketua mudika kami sempat sedikit putus asa karena hanya sedikit yang ingin berpartisipasi bermain gamelan ini. Namun setelah latihan hari pertama dengan hanya beberapa orang namun sudah cukup membuat orangtua kami begitu terpana. Diantara kami hanya 3 orang saja yang memang sudah menguasai gamelan dan yang lainnya semua mulai dari nol. Bagi sebagian orang mungkin cukup mengejutkan.  Namun itulah kami, he he $\ldots$ $\ddot\smile$

	Setelah melalui proses latihan yang bisa dibilang begitu singkat, akhirnya tibalah saat kami menunjukkan hasil kerja keras kami berlatih selama beberapa minggu. Cukup grogi untuk kami, namun semuanya itu sirna setelah kami mulai memainkan lagu pertama dan cukup sukses hingga acara misa itu selesai. Sungguh pengalaman yang sangat berharga dan membanggakan bagi kami.

	Dari sedikit cerita tentang usaha kaum muda yang mau dan rela meluangkan waktunya untuk berlatih  hal atau sesuatu dalam bidang musik, yang bisa dibilang untuk kalangan muda di jaman sekarang adalah sesuatu yang membosankan, karena mereka lebih memilih musik yang cenderung keras, dapat dikatakan bahwa gamelan merupakan sebuah permainan alat musik yang dapat menyatukan semua unsur  terutama unsur kebersamaan, kerja sama dan toleransi. Gamelan memang tidak bisa dimainkan seorang diri dan nada alat yang satu dengan yang lain memang harus selaras. Sama juga seperti keberagaman yang ada di negeri kita Indonesia ini, dengan semboyan “Bhineka Tunggal Ika” diharapkan kaum muda saat ini lebih bisa menghargai budaya satu sama lain tanpa mengejek atau menjadikan bahan ejekan yang mengandung SARA. Kaum muda juga diharapkan bisa melestarikan kebudayaan bangsa kita ini. Bermain gamelan seperti di atas juga merupakan salah satu cara melestarikan budaya lho .. Apalagi budaya di Indonesia ini bisa dibilang cukup beragam. Banyak suku di Indonesia yang mempunyai kebudayaannya masing-masing.

	Dalam rangka memperingati hari kemerdekaan RI pada 17 Agustus nanti, diharapkan kaum muda dapat berperan aktif dalam menyemarakkan peringatan tersebut. Cara paling sederhana adalah bagi teman-teman yang masih sekolah dapat mengikuti upacara bendera di sekolah atau bahkan terlibat dalam kelompok PASKIBRAKA. Peran kaum muda saat ini sangat dibutuhkan. Kaum muda merupakan penerus bangsa dan menjadi tumpuan untuk kemajuan bangsa ini. Intinya buat kaum muda, mari bangkitlah dan berjuang dalam menghadapi jaman yang serba modern ini. Jadilah penerus bangsa! Kalau bukan kita siapa lagi kawan .. 
	
DIRGAHAYU INDONESIAKU!!! 