\newpage
\section*{KASIH ITU PEMBEBASAN}
 
{\small Manusia dalam abad modern ini sungguh terasa lebih mementingkan diri sendiri, demi “uang”, “harga diri” “jabatan” dan sebagainya. Nilai-nilai dasar kemanusiaan tidak lagi menjadi patokan untuk membuat sebuah pertimbangan. Kasus korupsi, ketidak adilan hukum menjadi berita sehari-hari baik di media massa maupun media elektronik. Kasus Korupsi dari Gayus Tambunan hingga Nazarudin, kasus ketidak adilan hukum dari Prita dan sebagainya membuat manusia  bangsa dan negeri ini semakin meninggalkan hati nurani mereka.

Dalam berbagai ketidak pastian akan penyelesaian akan segala hal tersebut, Dia mengundang kita untuk merayakan perjamuan Tuhan. Allah Tahu bahwa ada pengkhianatan, tapi Dia tidak membalas pengkhianatan dengan kekerasan, Dia membalas dengan cinta dan kasih. Cinta yang begitu besar yang ada di dalam diri Allah inilah yang membuat Dia melakukan tindakan yang sulit diterima oleh akal budi. 


Simbol perjamuan adalah penerimaan antara yang satu dengan yang lain dimana idak ada saling curiga, permusuhan, pemaksaan. Semua dilebur dalam bahasa kasih. Kasih adalah sebuah tindakan nyata seperti halnya Yesus sendiri merelakan harga diri mau menjadi hamba dengan tindakan membasuh kaki para muridnya. Itu bukan hanya sebuah pengakuan bahwa dosa sudah diampuni, tetapi lebih dalam lagi adalah bahwa engkau yang menyebut diri sebagai murid Yesus harus melakukan hal yang sama. 

 Menjadi murid Yesus harus berani menyangkal dirinya. Siapa yang berani mengikuti Yesus, dia  harus berani membagi kasih itu kepada sesamanya. Orang yang membiarkan sesamanya kelaparan adalah melawan kasih. Orang yang membiarkan ketidak adilan terus berjalan adalah melawan kasih. Tindakan Allah mencintai adalah sebuah pemberian. Tindakan Yesus memberikan nyawa-Nya sebagai tebusan bagi  manusia berdosa bukan dengan bahasa pemaksaan dan kekerasan tetapi dengan kasih. Tindakan kasih itu adalah tindakan orang-orang yang memiliki hati nurani yang sehat, yakni  orang-orang yang tidak membiarkan sesamanya menderita karena adanya suatu ketidak adilan. Tindakan Kasih N
Itulah yang dilakukan oleh Yesus dalam perjamuan. “Inilah tubuh-Ku yang dikorbankan bagimu, inilah piala perjanjian baru yang diikat dalam darah-Ku sebagai pembaharuan dalam kasih. Pembaharuan bukan sekedar bualan dengan kata-kata, tetapi suatu tindakan moral. Bertindak atas nama moral berarti  membangun kesejahteraan bangsa ini.}

\vspace{0.5cm}

{\flushleft Maguwo, medio Juli 2011\\
Bravo Sierra}