\newpage
\section*{KEMERDEKAAN MANUSIA DARI BEBAN HUKUM}
	Dalam Perjanjian Lama, Hari Tuhan adalah Sabat. Dalam Kel 20:8 dikatakan, “Ingatlah hari Sabat, supaya kau kuduskan”. Dalam Ul 5: 13-14, dijelaskan bagaimana orang mengkhususkan hari Sabat itu, yakni dengan berhenti bekerja. Hari Sabat berarti kebebasan sehari setelah manusia bekerja selama 6 hari. Dalam ayat 15, Hari Sabat dirayakan sebagai kemenangan atas pembebasan dari Mesir, ketika Tuhan membebaskan manusia yang tertindas. Hari Tuhan adalah hari pembebasan manusia. Dalam agama Yahudi, hari Sabat menjadi lambang agama dan tanda pengenal bangsa. Khususnya ketika tanah mereka direbut bangsa lain dan kenisah diporak-porandakan. Hari Sabat menjadi lambing keyahudian yang paling utama, tanda kehadiran Tuhan. Hari sabat menjadi hari keramat, dibebani dengan peraturan yang ditambah-tambah, sampai Sabat sendiri kehilangan artinya. Dari sana lahirlah peraturan dan ketetapan-ketetapan seperti sampai berapa jauh orang boleh berjalan pada hari Sabat, hanya kerja apa saja yang boleh dilakukan, kapan boleh membantu orang sakit, kapan ada dispensasi dari peraturan Sabat, dan apa hukuman kalau melanggar peraturan. Sabat menjadi beban dan bukan lagi hari perayaan pembebasan.

Dalam Injil sering kali diceritakan bahwa orang Farisi bertengkar dengan Yesus mengenai hari sabat. Sering kali perbuatan-perbuatan Yesus dianggap melanggar peraturan Sabat. Diantara perbuatan-perbuatan yang dilakukan Yesus terdapat tindakan tertentu yang mengungkapkan sikap dan pandangan Yesus mengenai hukum Taurat. Yesus memaklumkan bahwa Allah adalah segala hukum, peraturan, dan perintah yang harus diabdikan pada tujuan pemerdekaan manusia. Maksud terdalam setiap hukum adalah membebaskan manusia dari segala sesuatu yang dapat menghalangi manusia berbuat baik. Begitu pula tujuan hukum Taurat. Sikap Yesus mengenai hukum Taurat dapat diringkaskan dengan mengatakan bahwa Yesus selalu memandang hukum Taurat dalam hukum kasih.

	Orang yang tidak perduli dengan maksud dan tujuan suatu hukum, yang penting adalah peraturan-peraturan dalam hukum harus ditepati, sangat bersikap legalistis. Pemenuhan hukum secara lahiriah dipentingkan sedemikian rupa sehingga semangat hukum sering kali dikorbankan. Ketika semisalnya kaum Farisi mau menerapkan hukum Sabat dengan cara merugikan perkembangan manusia, dimana manusia tidak diperkenankan berbuat baik pada hari Sabat. Yesus mengajukan protes demi tercapainya tujuan hukum Sabat yakni kesejahteraan manusia jiwa raganya. 

	Menurut keyakinan orang Yahudi sendiri, hukum Sabat adalah kurnia Allah demi kesejahteraan manusia (lih. Ul 5:12-15; Kel 20:8-11; Kej 2:3). Akan tetapi sejak pembuangan Babilonia (587-538 SM),  hukum Sabat oleh para rabi cenderung ditambah dengan larangan-larangan yang sangat rumit. Memetik bulir gandum sewaktu melewati ladang yang terbuka tidak dianggap sebagai pencurian. Kitab Ulangan yang bersemangat perikemanusiaan mengizinkan perbuatan tersebut. Tetapi hukum seperti yang ditafsirkan para rabi melarang orang menyiapkan makanan pada hari Sabat dan karenanya melarang orang menuai dan menumbuk gandum. Dengan demikian para rabi menulis hukum mereka sendiri yang bertentangan dengan semangat manusiawi kitab Ulangan. Hukum ini semakin menjadi beban dan bukan lagi bantuan guna mencapai kepenuhan hidup manusia. Maka Yesus mengajukan protes. Ia mempertahankan maksud Allah dengan hukum Sabat itu. Yang dikritik Yesus bukanlah hukum Sabat sebagai pernyataan kehendak Allah, melainkan cara hukum itu ditafsirkan dan diterapkan. Mula-mula hukum Sabat  itu hukum sosial yang bermaksud memberikan kesempatan kepada manusia beristirahat, berpesta, dan bergembira setelah enam hari bekerja. Istirahat dan pesta itu memungkinkan manusia selalu ingat siapakah sebenarnya manusia itu dan untuk apakah hidup. Hidup bukan untuk binasa dalam pekerjaan dan penderitaan, tetapi untuk bangkit kembali dan hidup lagi sebagai manusia bebas dalam kegembiraan. Sebenarnya hukum Sabat mengatakan kepada kita bahwa masa depan kita bukan kebinasaan, tetapi pesta. Dan pesta itu sudah boleh mulai kita rayakan sekarang dalam hidup di dunia ini., dalam perjalanan kita menuju Sabat yang kekal.

	Cara yang baik mempergunakan hari Sabat ialah menolong sesama (Mrk 3:1-5). Hari Sabat bukan untuk mengabaikan kesempatan berbuat baik. Pandangan Yesus tentang hukum Taurat adalah pandangan yang bersifat memerdekakan sesuai dengan maksud asli hukum Taurat itu sendiri.  


\emph{Sumber: Buku Informasi dan Referensi  Iman Katolik, Konferensi Waligereja Indonesia, Penerbit Kanisius bekerja sama dengan Penerbit Obor, cetakan ke 12 tahun 2007.}