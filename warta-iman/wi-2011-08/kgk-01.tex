\newpage
\section*{Kompendium Katekese Gereja Katolik}
\setcounter{kgkcounter}{0}
{\small
\kgk{Apa rencana Allah untuk manusia?}
Allah, yang sempurna dan penuh bahagia, berencana membagikan kebaik-
an-Nya dengan menciptakan manusia agar manusia ikut ambil bagian dalam
kebahagiaan-Nya. Dalam kepenuhan waktu, ketika saatnya tiba, Allah Bapa
mengutus Putra-Nya sebagai Penebus dan Penyelamat manusia, yang sudah
jatuh ke dalam dosa, memanggil semuanya ke dalam Gereja-Nya, dan melalui
karya Roh Kudus, mengangkat mereka sebagai anak-anak-Nya dan pewaris
kebahagiaan abadi.

\kgk{Mengapa manusia mempunyai kerinduan akan Allah?}
Allah, dalam menciptakan manusia menurut citra-Nya, telah mengukirkan
dalam hati manusia kerinduan untuk melihat Dia. Bahkan walaupun kerinduan
ini diabaikan, Allah tidak pernah berhenti menarik manusia kepada Diri-
Nya karena hanya dalam Dialah manusia dapat menemukan kepenuhan akan
kebenaran yang tidak pernah berhenti dicarinya dan hidup dalam kebahagiaan.
Karena itu, menurut kodrat dan panggilannya, manusia adalah makhluk religius
yang mampu masuk ke dalam persekutuan dengan Allah. Hubungan akrab dan
mesra dengan Allah mengaruniakan martabat kepada manusia.

\kgk{Bagaimana mungkin manusia mengenal Allah hanya melalui terang akal budinya?}
Dengan bertolak dari ciptaan, yaitu dari dunia dan pribadi manusia, hanya
melalui akal budinya manusia dapat mengenal Allah secara pasti sebagai asal
dan tujuan alam semesta, sebagai kebaikan tertinggi, dan sebagai kebenaran dan
keindahan yang tak terbatas.

\kgk{Apakah terang akal budi saja sudah memadai untuk mengenal misteri
Allah?}
Jika hanya melalui terang akal budi saja, manusia mengalami banyak
kesulitan untuk mengenal Allah. Dengan kekuatannya sendiri, manusia sungguh-
sungguh tidak mampu masuk ke dalam kehidupan intim misteri ilahi. Karena
itu, manusia membutuhkan pencerahan melalui wahyu; tidak hanya untuk
hal-hal yang melampaui pemahamannya, tetapi juga untuk kebenaran religius
dan moral, yang sebenarnya tidak melampaui daya tangkap akal budi manusia.
Bahkan dalam kondisi saat ini, kebenaran-kebenaran tadi dapat dipahami dengan
mudah oleh semua manusia, secara pasti, dan tanpa kesalahan.

\flushright{(\dots \emph{bersambung} \dots)}
}