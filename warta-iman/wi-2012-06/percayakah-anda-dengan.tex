\chap{Percayakah  Anda dengan Mukjizat Ekaristi?}
\small
    Barangkali kita masih ingat kabar tentang hosti suci berubah menjadi tetes noda darah di lantai ketika hosti jatuh dari tangan seorang umat yang menyambutnya di Paroki Kota Baru.  Kabar itu segera menyebar ke kalangan umat dan menimbulkan pertanyaan:  benarkah demikian ? Tentu bagi orang yang tidak percaya tidak mungkin bisa diberi penjelasan. Sebaliknya bagi orang yang percaya tidak diperlukan lagi penjelasan.

    Di bawah ini saya ingin menyampaikan kisah-kisah mukjizat ekaristi ``Miracles of the Eucharisti''; The Eucharisti Apostles of The Divine Mercy; www.thedivinemercy.org sebagaimana diterjemahkan YESAYA : www.indocell.net/yesaya.Sebetulnya ada 21 kisah, akan tapi tentu tidak semua bisa di coba di sini atau disajikan. Saya akan mengambil beberapa saja. Mudah-mudahan bisa berguna bagi peneguhan kehidupan rohani kita khususnya terhadap Ekaristi.
\normalsize
 

\renewcommand{\thesection}{\arabic{section}.}

\section{ALATRI, tahun 1228}

Seorang pemudi, yang tertarik pada seorang pemuda, diminta untuk membawa sekeping Hosti yang telah dikonsekrir agar dapat dibuatkan ramuan cinta. Sang pemudi menerima Komuni dan berjalan pulang ke rumah, tetapi karena merasa bersalah ia menyembunyikan Kristus di suatu pojok rumah.

Beberapa hari kemudian, ia datang dan mendapati bahwa Hosti telah berubah warna seperti daging. Imam paroki segera diberitahu dan ia membawa Hosti kepada Uskup. Bapa Uskup menulis surat kepada Paus Gregorius IX yang isinya:
\small
\qti{``Kita patut menyampaikan puji syukur sedalam-dalamnya kepada Dia yang, sementara senantiasa menyelenggarakan segala karya-Nya dengan cara-cara yang mengagumkan, pada kesempatan-kesempatan tertentu juga mengadakan mukjizat-mukjizat dan melakukan hal-hal menakjubkan agar para pendosa menyesali dosa-dosa mereka, mempertobatkan yang jahat, dan mematahkan kuasa bidaah sesat dengan memperteguh iman Gereja Katolik, menopang pengharapan-pengharapannya serta mendorong amal kasihnya.     
\\{~}\\
Oleh sebab itu, saudaraku terkasih, dengan surat Apostolik ini, kami menyarankan agar engkau memberikan penitensi yang lebih ringan kepada gadis tersebut, yang menurut pendapat kami, dalam melakukan dosa yang teramat serius itu, lebih terdorong oleh kelemahan daripada kejahatan, terutama dengan mempertimbangkan kenyataan bahwa ia sungguh menyesal setulus hati ketika mengakukan dosanya. Namun demikian, terhadap wanita yang menghasutnya, yang dengan kejahatannya mendorong si gadis untuk melakukan dosa sakrilegi, perlu dikenakan hukuman disipliner yang menurutmu lebih pantas; juga memerintahkannya untuk mengunjungi semua Uskup di wilayah terdekat, guna mengakukan dosa-dosanya kepada mereka dan mohon pengampunan dengan ketaatan yang tulus \ldots''
}
\normalsize

Mukjizat Hosti dipertontonkan dua kali setahun, yaitu pada hari Minggu pertama sesudah Paskah dan hari Minggu pertama sesudah Pentakosta.

Pada tahun 1960, Uskup Facchini dari  Alatri membuka segel tempat Hosti disimpan dan mengeluarkannya. Uskup menyatakan bahwa Hosti tetap dalam keadaan sama seperti saat pertama diketemukan, yaitu, sekerat daging yang tampak sedikit kecoklatan.

Pada tahun 1978, perayaan-perayaan istimewa diselenggarakan untuk memperingati 750 tahun terjadinya mukjizat.

\textit{``Akulah roti hidup. Nenek moyangmu telah makan manna di padang gurun dan mereka telah mati. Inilah roti yang turun dari sorga: Barangsiapa makan dari padanya, ia tidak akan mati.'' (Yoh 6:48-50)}
 

\section{ORVIETO dan BOLSENA, tahun 1263}

Mukjizat ini terjadi pada masa suatu ajaran sesat yang disebut Berengarianisme merajalela di Eropa. Bidaah ini menyangkal Kehadiran Nyata Kristus dalam Ekaristi. Pada tahun 1263, seorang imam bernama Petrus dari Prague sedang dalam perjalanan ziarah ke Roma untuk berdoa di makam pelindungnya, St Petrus, sebab ia menghadapi masalah yang amat serius. Ia merasakan kebimbangan yang besar mengenai Kehadiran Nyata Yesus dalam Ekaristi Kudus. Ia berdoa agar santo pelindungnya memohonkan rahmat baginya guna menyelamatkan imannya yang goyah. Dalam perjalanan, ia singgah untuk bermalam di suatu kota kecil bernama Bolsena, sekitar 70 mil sebelah utara Roma.

Keesokan harinya, Pastor Petrus merayakan Misa Kudus di Gereja St Kristina. Sementara ia mengucapkan kata-kata konsekrasi, “Inilah TubuhKu,” roti di tangannya berubah rupa menjadi Daging dan mulai mencucurkan darah dengan derasnya. Darah jatuh menetes ke korporal. Pastor Petrus amat terperanjat; ia tidak tahu apa yang harus diperbuatnya. Maka, ia membungkus Hosti Kudus dalam Korporal lalu pergi meninggalkan altar. Sementara ia berjalan pergi, tetesan-tetesan Darah jatuh ke atas lantai pualam di altar.

Paus Urbanus IV sedang berada di kota Orvieto, yang tak jauh dari sana. Pastor Petrus segera menemui paus guna menceritakan apa yang telah terjadi. Paus segera mengutus seorang uskup ke Gereja St Kristina guna menyelidiki peristiwa tersebut dan mengambil Korporal.

Segera sesudah Paus menerima Korporal dari Uskup, ia pergi ke balkon Istana Kepausan dan dengan hormat mempertontonkan mukjizat Korporal kepada orang banyak. Bapa Suci menyatakan bahwa mukjizat Ekaristi telah terjadi guna mengusir bidaah Berengarianisme. Pada saat yang sama, seorang pengikut St. Yuliana dari Liège menghubungi paus untuk sekali lagi memohon demi ditetapkannya Hari Raya Corpus Christi. Setahun kemudian, pada tahun 1264, Paus Urbanus IV memaklumkan Hari Raya agung ini kepada seluruh Gereja. (Mukjizat Korporal disimpan hingga kini di Katedral Orvieto. Lantai pualam bernoda Darah disimpan di Gereja St Kristina di Bolsena).  

\section{CASCIA, sekitar tahun 1300}
Cascia adalah sebuah kota kecil di pegunungan di lembah Umbrian, Italia. Itulah kota kediaman St. Rita dari Cascia. Jenazah St. Rita yang hingga kini masih utuh dibaringkan di Basilika Utama. Di bawahnya, di Basilika Kecil, disimpan Mukjizat Ekaristi dan jenazah Beato Simone Fidati, seorang imam yang terlibat langsung dalam mukjizat tersebut.

Pada masa terjadinya mukjizat, seorang imam tak lagi memiliki rasa hormat terhadap Ekaristi. Ketika diminta untuk mengantarkan Sakramen Mahakudus kepada seorang petani yang sedang sakit, ia mengambil sekeping Hosti yang telah dikonsekrasikan, menempatkannya dengan sembarangan di antara halaman-halaman buku breviary, lalu berangkat. Ketika ia membuka bukunya, ia mendapati bahwa Hosti telah berubah warna merah darah segar dan darah meresap ke kedua halaman buku di mana Hosti diselipkan.

Imam tersebut kemudian mohon nasehat Beato Simone Fidati, seorang imam yang kudus dan dihormati pada masa itu. Pastor Fidati menerima pengakuan sang imam dan memberinya absolusi. Beato Fidati mengambil kedua halaman dari breviary itu; satu ditempatkannya di tabernakel di Perugia dan satunya lagi ditempatkannya di Cascia. Mukjizat Ekaristi ini diperingati secara istimewa di Cascia setiap tahun pada Hari Raya Tubuh dan Darah Kristus.

Orang-orang yang melihat ke halaman yang ternoda darah itu dapat melihat gambar Kristus tertera di sana.

Ya Kristus, berilah kami rahmat agar dapat melihat Engkau dalam Ekaristi dan mengenali-Mu pada saat pemecahan roti.

 

\section{BAGNO DI ROMAGNA, tahun 1412}
Mukjizat Ekaristi ini terjadi di sebuah kota kecil di Italia bernama Bagno di Romagna ketika seorang imam merayakan Misa dengan dihantui keragu-raguan yang besar akan Kehadiran Nyata Kristus dalam Ekaristi. Setelah mengkonsekrasikan anggur, imam melihat ke dalam piala dan amat terkejut melihat bahwa anggur telah berubah menjadi darah. Darah mulai meluap keluar dari piala dan membasahi korporal. Terguncang oleh peristiwa adikodrati ini, imam segera berdoa mohon pengampunan. Kelak, ia bahkan digelari Venerabilis karena kesalehan hidupnya setelah terjadinya mukjizat.

Pada tahun 1912, ulang tahun ke-500 mukjizat Ekaristi, suatu perayaan besar diselenggarakan. Pada tahun 1958, dilakukan penelitian ilmiah yang hasilnya menguatkan bahwa darah di korporal adalah darah manusia dan masih mengandung karakteristik darah setelah hampir 600 tahun sesudah mukjizat terjadi.

Mungkin mukjizat Darah yang meluap hendak menunjukkan kepada kita bahwa Yesus sungguh hadir dalam Ekaristi. Mari merenungkan bagaimana seharusnya kita berubah setelah menyambut Yesus dengan mengijinkan-Nya tinggal dalam kita dan mengisi kita dengan kuasa Roh Kudus.

\section{MIDDLEBURG ~ LOUVAIN, tahun 1374}
Pada tahun 1374, seorang pemuda dengan dosa berat dalam jiwanya pergi menyambut Komuni Kudus. Ketika Hosti ditempatkan di atas lidahnya, Hosti berubah menjadi Daging sehingga ia tak dapat menelannya. Darah menetes dari bibirnya dan membasahi kain pada rel komuni. Imam bertindak cepat dengan mengambil Hosti Kudus serta menempatkannya dalam sebuah piala di altar.

Berita mengenai mukjizat ini tersebar keseluruh penjuru Belgia dan mukjizat Hosti dipindahkan 700 mil jauhnya ke Cologne. Sebuah ostensorium berhias indah dibuat. Sebagian Hosti dan sepotong kain dengan noda darah kemudian dibawa ke Louvain di mana telah dipersiapkan sebuah wadah reliqui yang indah.

Bagian mukjizat Ekaristi yang disimpan di Louvain berwarna agak kecoklatan dan dapat dikenali dengan mudah sebagai daging. Reliqui disimpan dalam sebuah wadah reliqui yang dibuat pada tahun 1803. Dokumen-dokumen penting dan hasil penelitian terhadap reliqui disimpan dalam perpustakaan Gereja St. Jacques.

\section{BOLOGNA, tahun 1333}
Mukjizat ini terjadi pada tahun 1333 di Bologna, Italia karena seorang gadis remaja saleh yang berumur sebelas tahun memiliki kerinduan yang berkobar-kobar untuk menyambut Kristus dalam Ekaristi.

Imelda Lambertini dilahirkan dalam sebuah keluarga kaya. Ayahnya adalah Count Eagno Lambertini. Imelda bergabung dalam Biara Dominikan ketika usianya baru sembilan tahun. Ia disayangi oleh para biarawati lainnya. Dalam usia yang masih sangat muda, Imelda memiliki cinta yang menyala-nyala kepada Yesus dalam Ekaristi dan karenanya sungguh rindu menyambut-Nya dalam Komuni Kudus. Tetapi, hal itu tidak mungkin baginya karena usianya belum cukup untuk dapat menerima Komuni.

Tuhan mengaruniakan kepadanya suatu anugerah istimewa pada Pesta Kenaikan Yesus ke Surga pada tahun 1333. Sementara ia berdoa, sebuah Hosti tampak melayang-layang di udara di hadapannya. Imam segera dipanggil dan ia memberikan kepada Imelda Komuni Kudusnya. Imelda mengalami ekstasi dan tidak pernah bangun kembali. Ia wafat saat menyambut Komuni Kudusnya yang Pertama!

Devosi kepada Beata Imelda pun dimulai dan pada awal tahun 1900-an suatu komunitas Dominikan dibentuk dengan nama Suster-suster Dominikan dari Beata Imelda. Para biarawati ini berjuang keras menyebarluaskan cinta dan devosi kepada Ekaristi serta menggalakkan Adorasi Abadi. Jenasah Beata Imelda yang tetap utuh hingga kini dibaringkan di Gereja San Sigismondo dekat Universitas Bologna. Paus St. Pius X memaklumkan Imelda sebagai Pelindung Para Penerima Komuni Pertama.

Ya Kristus, biarkan kami mati setiap hari bagi-Mu dan menyambut Engkau dalam Ekaristi seakan-akan itulah komuni kami yang terakhir. Jadikan kami pula seperti anak-anak kecil, dengan cinta yang polos dan kepercayaan penuh akan cinta dan belas kasihan-Mu.

\section{SANTAREM, tahun 1247}

Seorang wanita yang suaminya tidak setia, meminta nasehat dari seorang wanita tenung. Wanita sihir itu berjanji akan mengubah perilaku suaminya jika si wanita membawakan baginya sekeping Hosti yang telah dikonsekrasikan. Ia juga menasehati si wanita untuk berpura-pura sakit agar dapat menerima Komuni Kudus dalam minggu itu dan segera memberikan Hosti kepadanya. Si wanita tahu bahwa hal itu dosa. Ia pergi menerima Komuni, tetapi tidak menyantap Tubuh Kristus. Ia meninggalkan Misa dan dalam perjalanan menuju tempat wanita tenung, Hosti mulai mengeluarkan darah. Beberapa orang yang melihat kejadian tersebut menyangka bahwa ia mengalami pendarahan. Rasa takut menguasai dirinya dan ia pulang ke rumah, menempatkan Hosti dalam sebuah peti, membungkusnya dengan saputangan, lalu menutupinya dengan linen yang bersih.

Tengah malam, ia dan suaminya terbangun oleh suatu sinar cemerlang yang berasal dari peti, yang menjadikan ruangan mereka terang-benderang. Para malaikat telah membuka peti dan membebaskan Tuhan. Wanita itu menceritakan kepada suaminya apa yang telah terjadi dan bahwa dalam peti terdapat sekeping Hosti yang telah dikonsekrasikan. Berdua mereka melewatkan sepanjang malam dengan berlutut dalam sembah sujud. Seorang imam dipanggil. Imam membawa Hosti Kudus kembali ke gereja dan menyegelnya dalam sebuah segel lilin.

Sembilan belas tahun kemudian, seorang imam membuka tabernakel dan memperhatikan bahwa segel telah terbuka sementara Hosti tersimpan dalam sebuah piksis kristal. Mukjizat ini, 750 tahun kemudian, yaitu pada tahun 1997, diperingati dengan berbagai perayaan meriah di Santarem.

Kita mungkin bertanya mengapa Tuhan mengadakan mukizat-mukjizat ini bagi kita. Mungkin untuk menyatakan betapa Ia sungguh hadir dalam Ekaristi dan betapa Ia sungguh mengasihi kita. Ia menghendaki agar kita semua, termasuk juga domba-domba yang hilang, bergabung kembali dalam kawanan. Ia mengasihi kita, bagaimana pun berdosanya kita. Ia adalah Allah Kasih dan Belas Kasihan. Dan Ia menghendaki agar kita membagikan Kasih dan Belas Kasihan itu kepada sesame

\section{AUGSBURG, tahun 1194}
Mukjizat ini terjadi ketika seorang wanita ingin menyimpan Hosti yang telah dikonsekrasikan dalam rumahnya. Suatu pagi, ia menerima Ekaristi, tetapi tidak menyantapnya. Ia membawa pulang Hosti dan menempatkannya dalam segel, menjadikannya suatu reliqui sederhana. Ia menyimpan Tubuh Kristus di rumahnya selama lima tahun, tetapi lama-kelamaan timbul perasaan bersalah hingga akhirnya ia mengatakannya kepada pastor paroki.

Pastor Berthold, imam setempat, terperanjat ketika membuka segel reliqui. Dialah yang pertama melihat bahwa Hosti telah berubah menjadi sesuatu yang tampak seperti daging dengan lapisan-lapisan merah yang nampak jelas. Imam mendiskusikan masalah ini panjang lebar dan memutuskan bahwa mereka akan dapat mengidentifikasikannya dengan lebih baik jika daging dibagi menjadi dua bagian. Mereka keheranan ketika mendapati bahwa daging tidak dapat dibagi karena disatukan oleh pembuluh-pembuluh darah yang seperti benang. Diyakini kemudian bahwa daging tersebut adalah daging Tuhan kita Yesus Kristus.

Uskup Udalskolk dengan seksama meneliti mukjizat tersebut dan memerintahkan agar mukjizat Hosti ditempatkan kembali ke dalam segel reliquinya semula untuk dipindahkan ke katedral.

Mukjizat Hosti dan segelnya kemudian ditempatkan dalam suatu wadah kristal dan disimpan dalam kaca. Hosti tetap dalam keadaan semula hingga hampir 800 tahun.

Setiap tahun pada tanggal 11 Mei, pada perayaan Fest des Wunderbarlichen, yaitu Pesta Mukjizat Harta yang Mengagumkan, Hosti dihormati dengan perayaan Misa yang khidmat dan pakaian liturgi khusus.   

Ya Kristus, berilah kami rahmat untuk memahami lebih baik serta membagikan kebenaran akan Kehadiran-Mu yang Nyata dalam Ekaristi. \textit{``Kuduskanlah mereka dalam kebenaran; firman-Mu adalah kebenaran.'' (Yoh 17:17)}.

\section*{Penutup}

Setelah kita membaca kisah-kisah di atas (bila ingin lebih lengkap bisa mengakses pada website seperti di atas), semoga kita semua menjadi bagian dari orang yang percaya dan karenanya iman kita menjadi semakin diteguhkan. Kristus sungguh hadir dalam rupa roti dan anggur, hadir di dalam hati umatNya dan menjadi tanda ``\textit{ubi Christus ibi ecclesia}'' di mana Kristus berada di situ Gereja berada.

\sumber{Yogyakarta, Hari Raya Pentakosta, 27  Mei 2012\\(AS)}

\renewcommand{\thesection}{\Alph{section}}
