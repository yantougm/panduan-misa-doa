\chap{Cara Menyambut Komuni}

Sebenarnya, jika kita mempelajari tradisi umum penerimaan Ekaristi, kita akan menemukan dua cara, langsung ke mulut atau di tangan. Memang, pada sampai sebelum Vatikan II, penerimaan Ekaristi dilakukan langsung ke mulut, namun pada tahun 1969, Roma mengeluarkan surat Instruksi yang memperbolehkan dua cara penerimaan Ekaristi, yaitu di tangan dan di mulut, dengan memberikan anjuran agar sedapat mungkin dipertahankan tradisi pemberian Ekaristi langsung ke mulut, walau tidak menutup kemungkinan pemberian ke tangan, jika itu diputuskan oleh konferensi para uskup di tempat yang bersangkutan, asal tetap menjaga penghayatan dan penghormatan yang layak kepada Ekaristi. Selengkapnya, silakan baca dokumen Instruksi resmi yang dikeluarkan oleh Paus Paulus VI, “\textit{Memoriale Domini“, the Instruction on the Manner of Administering Holy Communion, } \underline{The Congregation for Divine Worship on 29 Mei 1969}

Penerimaan Komuni langsung ke mulut dapat menghindari tercecernya serpihan-serpihan hosti yang kita percayai mengandung seluruh Kristus. Maka jika Komuni diberikan di tangan, maka perhatian khusus harus diberikan agar tidak ada serpihan hosti yang tercecer. Dan tak kalah penting, adalah tetap dengan hormat menerima Tubuh Kristus dengan sikap batin yang baik.

Dari segi kepraktisan, penerimaan komuni lewat mulut akan menghindari kemungkinan orang-orang yang ingin mendapatkan hosti untuk maksud-maksud yang tidak baik.

Penerimaan hosti (Sang Sabda yang menjadi daging) langsung ke mulut juga sesuai dengan ‘penerimaan Sabda Tuhan’ yang diberikan kepada nabi Yehezkiel(lih. Yeh 2:1,8,9,3: 1-3).

Mereka yang mendukung penerimaan Komuni di tangan, biasanya mengutip St. Sirilus/ Cyril (dari Yerusalem yang mengatakan, “Umat menerima Komuni dengan tangan kanan mendukung tangan kiri, dengan telapak yang membentuk cekungan; dan pada saat Tubuh Kristus diberikan, umat menjawab, Amen.” Atau juga St Basil (330-379) yang memperbolehkan pemberian Komuni di tangan pada jaman penindasan, yaitu pada saat tidak ada diakon/ imam yang dapat memberikan komuni.

Sebagai anggota Gereja Katolik di Indonesia, maka kita menghormati keputusan konferensi para uskup Indonesia, yang memperbolehkan penerimaan Ekaristi di tangan; walaupun sesungguhnya, kita dapat saja tetap menerima Ekaristi langsung di mulut, seperti yang dianjurkan oleh Bapa Paus. 

Di atas semua itu, perlu kita ingat, bahwa yang terpenting adalah penghayatan kita akan apa yang kita sambut; yaitu Kristus sendiri dalam rupa hosti. Cara penerimaan Komuni merupakan hal disiplin, yang jangan sampai mengaburkan makna Ekaristi itu sendiri. 

Seperti sudah pernah dibahas sebelumnya, maka terdapat dua cara menerima komuni, yaitu dengan tangan atau langsung di mulut/ di lidah.

Berikut ini adalah cara menerima komuni yang benar:

\subsubsection*{Dengan mulut/lidah}
\begin{enumerate}
\item        Berjalanlah ke hadapan Pastor/petugas Prodiakon dengan tangan terkatup.
\item        Sesaat sebelum giliran Anda menyambut Hosti, Anda maju dan tundukkanlah kepala anda dengan hormat untuk menghormati Kristus yang hadir dalam rupa Hosti kudus.
\item        Ketika Pastor/Prodiakon mengangkat hosti dan mengatakan “Tubuh Kristus”, pandanglah Hosti itu katakanlah “Amin” (artinya, Saya percaya)
\item        Bukalah mulut Anda dengan posisi lidah yang pantas agar Pastor/ petugas Prodiakon dapat meletakkan Hosti pada lidah Anda.
\item        Sambil Anda kembali ke tempat duduk Anda, Anda dapat mengunyah Hosti itu, ataupun membiarkan Hosti itu hancur di mulut Anda.
\end{enumerate}

\subsubsection*{Dengan Tangan}
\begin{enumerate}
\item        Berjalanlah ke hadapan Pastor/ petugas Prodiakon dengan tangan terkatup.
\item        Sesaat sebelum giliran Anda menyambut Hosti, Anda maju dan tundukkanlah kepala Anda dengan hormat untuk menghormati Kristus yang hadir dalam rupa Hosti kudus.
\item        Letakkan telapak tangan, satu di atas yang lain, dengan terbuka menghadap ke atas. Tangan yang dipakai untuk mengambil Hosti diletakkan di bawah telapak tangan yang lain.
\item        Arahkan telapak tangan Anda dengan jelas, sehingga Pastor/ Prodiakon dapat melihat bahwa Anda akan menerima Hosti dengan tangan.
\item        Ketika Pastor/ Prodiakon mengangkat hosti dan mengatakan “Tubuh Kristus”, pandanglah Hosti itu katakanlah “Amin” (artinya, Saya percaya)
\item        Setelah Hosti diberikan di telapak tangan yang teratas, ambillah Hosti tersebut dengan telapak tangan yang di bawah, dan segera letakkan hosti tersebut di mulut Anda. (Jangan membawa hosti tersebut ke bangku Anda/kemanapun)
\item        Sekembalinya Anda ke tempat duduk Anda, Anda dapat mengunyah Hosti itu, ataupun membiarkan Hosti itu hancur di mulut Anda.
\item        Pastikan Anda memakan serpihan Hosti (jika ada) yang mungkin jatuh di telapak tangan Anda.
\end{enumerate}

Menurut tata cara di atas, tidak ada ketentuan apakah tangan kiri atau tangan kanan yang di atas/ di bawah. Bagi kita orang Timur, memang jika kita menyambut dengan tangan, maka tangan yang mengambil Komuni ke dalam mulut adalah tangan kanan, tetapi ini tidak berarti bahwa harus demikian, karena orang yang kidal mungkin lebih dapat menggunakan tangan kiri.

Yang jelas jika sudah menyambut dengan tangan, jangan mengambil Hosti dengan lidah, karena resiko Hosti jatuh lebih besar. Kecuali jika Anda melihat ada serpihan Hosti di tangan, maka Anda harus mengambilnya dengan lidah Anda, untuk Anda makan. Sebab kita percaya serpihan Hosti itu juga adalah Kristus.

Jika ingin menyambut Hosti dengan mulut/lidah, silakan menyambutnya dengan cara yang benar.