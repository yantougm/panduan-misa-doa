\chap{Sekilas Tentang Hari Raya Tubuh dan Darah Kristus}
 
Pesta \textit{Corpus Christi} (nama lengkapnya: \textit{Corpus et Sanguis Christi}), atau Hari Raya Tubuh dan Darah Kristus (dinamakan demikian kini), bisa ditelusuri asal-usulnya kembali ke abad ke-13.
Akan tetapi pesta ini merayakan sesuatu yang jauh lebih tua: institusi Sakramen Perjamuan Kudus di malam perjamuan terakhir.

Pada 1246, Uskup Robert de Thorete dari keuskupan Belgina dari Liège, atas saran biarawati Juliana dari Mont St Cornillon (juga di Belgia), mengadakan sinode dan melembagakan perayaan pesta ini.
Dari Liège, perayaan itu mulai menyebar, dan, pada September 8, 1264, Paus Urbanus IV menerbitkan buku kepausan “\textit{Transiturus de hoc mundo},” yang meresmikan Pesta Corpus Christi sebagai perayaan universal Gereja, yang akan dirayakan pada Kamis setelah Minggu Trinitas (minggu pertama setelah Pentakosta)

Atas permintaan Paus Urbanus IV, St Thomas Aquinas menulis ofisi (doa resmi Gereja) untuk pesta ini.
Ofisi ini secara luas dianggap salah satu yang paling indah di dalam \textit{Brevir} tradisional Roma (\textit{brevir} tradisional = buku doa resmi, Liturgi Jam Kudus), dan merupakan referensi himne Ekaristi terkenal “\textit{Pange Lingua Gloriosi}” (PS 502) dan “\textit{Tantum Ergo Sacramentum}” (PS 558 dan 559)

Selama berabad-abad setelah perayaan ini diperluas mencapai keseluruhan Gereja universal, perayaan itu juga dirayakan dengan prosesi ekaristi, di mana Hosti Kudus diarak ke penjuru kota, disertai dengan pujian dan litani. Sementara itu, umat beriman memuliakan Tubuh Kristus saat prosesi melewati mereka.

Beberapa kanon dari Kitab Hukum Kanonik yang memuat hari raya ini antara lain

\begin{description}
\item[Kan. 944 § 1] 	\textit{Jika menurut penilaian Uskup diosesan dapat dilaksanakan, sebagai kesaksian publik penghormatan terhadap Ekaristi mahakudus, hendaklah diselenggarakan prosesi lewat jalan-jalan umum, terutama pada hari raya Tubuh dan Darah Kristus.}
\item[Kan. 944 § 2] 	\textit{Uskup diosesan bertugas menetapkan peraturan-peraturan mengenai prosesi, dengannya dijamin partisipasi serta kepantasannya.}
\item[Kan. 1246 § 1] 	\textit{Hari Minggu, menurut tradisi apostolik, adalah hari dirayakannya misteri paskah, maka harus dipertahankan sebagai hari raya wajib primordial di seluruh Gereja. Begitu pula harus dipertahankan sebagai hari-hari wajib: hari Kelahiran Tuhan kita Yesus Kristus, Penampakan Tuhan, Kenaikan Tuhan, Tubuh dan Darah Kristus, Santa Perawan Maria Bunda Allah, Santa Perawan Maria dikandung tanpa noda dan Santa Perawan Maria diangkat ke surga, Santo Yusuf, Rasul Santo Petrus dan Paulus, dan akhirnya hari raya Semua Orang Kudus.}
\item[Kan. 1246 § 2] 	\textit{Namun Konferensi para Uskup dengan persetujuan sebelum- nya dari Takhta Apostolik, dapat menghapus beberapa dari antara hari- hari raya wajib itu atau memindahkan hari raya itu ke hari Minggu.}
\end{description}

Berdasar Kan 1246 point 2, di Indonesia Hari Raya Tubuh dan Darah Tuhan dirayakan pada Minggu kedua setelah Pentakosta. Untuk tahun 2012 ini jatuh pada tanggal 10 Juni 2012.




Secara tradisonal, pada awalnya sebutan yang tepat untuk Hari Raya Tubuh dan Darah Kristus adalah \textit{Sollemnitas Sanctissimi Corporis Christi}) yang kemudian dalam penggunaan populer digunakan frasa “\textit{Corpus Christi}”. Pada awalnya memang tidak ada kata “Darah” walaupun dalam teks Misa dan Ibadat Harian (\textit{brevir}) ada rujukan mengenai kata “Darah”



Perubahan yang terjadi adalah konsekuensi perubahan terhadap \textit{Festum Sanguinis Christi} (Pesta Darah Mulia). Pesta Darah Mulia adalah salah satu Pesta “devosional” terhadap kemanusiaan Kristus. (Dalam Gereja Katolik ada tiga tingkatan hari-hari istimewa, yaitu Hari Raya/\textit{Solemnitas}, Pesta/\textit{Festum}, dan Peringatan/\textit{Memoraria}). Pesta ini merupakan bagian dari “Pesta-pesta Sengsara” yang diadakan di hari-hari Jumat dalam Masa Prapaska di banyak tempat. 

 

Pada 1849, Paus Pius IX menyatakan hari Minggu pertama bulan Juli sebagai Pesta Darah Mulia dan wajib dirayakan secara universal. Namun demikian beliau tidak menghapuskan hari-hari Jumat “Pesta sengsara” yang masih dipraktikan pada berbagai penanggalan gerejawi lokal.

 

Ketika Paus Pius X melakukan pembaruan penanggalan liturgi, Pesta Darah Mulia dipindahkan menjadi tanggal 1 Juli, dan sejalan dengan kerangka liturgis yang ditetapkan pada hari itu, maka banyak keuskupan dan ordo tidak mempraktikan lagi “Pesta-pesta Sengsara”.Namun pesta-pesta ini tetap dipertahankan seperti yang tertulis pada appendiks buku pedoman misa (\textit{missal}) dengan judul “\textit{Pro Aliquibus Locis}” (di banyak tempat).

 

Pada 1961, semua pesta-pesta sengsara termasuk Pesta Darah Mulia yang tercantum dalam \textit{appendix}, dihapuskan, kecuali apabila ada permintaan dengan alasan yang masuk akal oleh ordo/kongregasi atau Keuskupan yang memiliki keterkaitan istimewa dengan pesta-pesta tersebut, misalnya kongregasi yang kemudian dikenal di Indonesia dengan nama Kongregasi Suster-suster Amalkasih Darah Mulia (ADM).

 

Kebijakan gerejawi berubah pada masa kepemimpinan Paus Yohanes XXIII. Beliau adalah seorang yang berdevosi pada Darah Mulia. Beliau menambahkan frasa “Terpujilah darahNya yang mahaindah” (PS No.205), mempromulgasikan (mengumumkan secara resmi) Litani Darah Mulia yang disertai dengan indulgensi, dan mempromosikan devosi terhadap Darah Mulia melalui ensiklik “\textit{Inde a Primis}”.

Pada tahun 1960-an ada perubahan penanggalan liturgi Gereja universal. Diputuskan bahwa pesta-pesta devosional harus dipindahkan atau paling tidak diturunkan tingkatannya. Pesta Darah Mulia yang dirayakan pada 1 Juli juga turut dihapuskan, walaupun tidak lama setelah keputusan ini dikeluarkan, banyak petisi dari para Uskup yang meminta agar Pesta Darah Mulia tetap dilestarikan. Namun demikian Konsili menolak petisi-petisi tersebut dan memutuskan untuk menambahkan kata “Darah” sehingga Hari Raya yang kita rayakan secara resmi hari ini dinamakan “\textbf{Hari Raya Tubuh dan Darah Kristus}” (\textit{Sollemnitas Sanctissimi Corporis et Sanguinis Christi}) atau boleh juga disebut “\textit{Corpus Sanguinisque Christi}”. Walaupun demikian, di banyak tempat, secara tradisional umat Katolik sudah telanjur terbiasa dengan penyebutan “\textit{Corpus Christi}” dan kita pun saat ini tetap boleh menyebut Hari Raya ini sebagai “\textit{Corpus Christi}” karena toh kita mengimani bahwa Hosti yang kita terima (apabila komuni hanya diterimakan dengan satu rupa), tidak pernah hanya Tubuh Kristus saja, melainkan sekaligus adalah Tubuh, Darah, Jiwa dan Keallahan Kristus, pendek kata SELURUH KRISTUS YANG TELAH WAFAT DAN BANGKIT, DAN KINI BERTAKHTA DI SISI BAPA. Hal ini sesuai juga dengan teks Kitab Suci, \textit{Jadi barangsiapa dengan cara yang tidak layak makan roti ATAU minum cawan Tuhan, ia berdosa terhadap Tubuh DAN Darah Tuhan..}(1 Kor 11:27)  .

 

Walaupun kita tidak lagi mempraktikkan Pesta Darah Mulia dalam penanggalan liturgi Gereja Universal, namun secara tradisional, Gereja Katolik mendedikasikan BULAN JULI demi penghormatan pada Darah Mulia Kristus.

\sumber{http://www.misa1962.org\\
http://www.facebook.com/notes/gereja-katolik}