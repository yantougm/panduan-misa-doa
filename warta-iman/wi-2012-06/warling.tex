\chap{Warta Lingkungan}

\subsection*{Bulan Rosario}
Bulan Mei seperti biasa dijadikan waktu untuk devosi kepada Bunda Maria. Di lingkungan St. Petrus dilaksanakan doa rosario setiap hari selama bulan Mei. Inilah salah satu cara berdevosi kepada Bunda Maria. Cukup menggembirakan bahwa yang hadir dalam acara ini rata-rata 40 orang.

Bulan Mei juga KAS menetapkan sebagai Bulan Katekese Liturgi (BKL). Oleh karena itu dalam doa Rosario juga disampaikan renungan BKL. Renungan ini berisi sentilan-sentilan halus terhadap umat Katolik tentang pemaknaan liturgi.

Karena dalam bulan Mei juga ada hari raya Kenaikan Tuhan dan Pentakosta maka di antara kedua hari raya itu diadakan Novena Roh Kudus selama sembilan hari. Pelaksanaannya dilakukan bersamaan dengan doa rosario. 

Hari terakhir doa rosario cukup istimewa karena dilaksanakan di rumah keluarga Yosef Laba yang merupakan salah satu warga St. Petrus tetapi sekarang berdomisili di Temanggal. Pak Yosef Laba merupakan anggota koor dengan suara bas yang aktif dan masih dinantikan suaranya dalam tugas koor St. Petrus.

\subsection*{Kunjungan Romo Kardinal}
Julius Kardinal Darmaatmadja SJ memimpin Misa 80 tahun Paroki Marganingsih Kalasan, Sleman, DI Yogyakarta, Minggu, 6/5. Ekaristi berlangsung di halaman samping gereja, dihadiri sekitar 2.500 orang. Mereka membawa tikar dan koran bekas sebagai alas duduk, dan payung untuk melindungi diri dari terik matahari.

“Paroki Marganingsih sudah berkembang menjadi paroki yang besar. Saya pernah melayani umat di sini selama setengah tahun. Waktu itu, saya menggantikan sementara Pastor Waskito yang sedang cuti. Jumlah umat belum sampai lima ribu orang, sekarang sudah lebih dari sepuluh ribu,” demikian Kardinal dalam khotbahnya.

Lebih lanjut Kardinal mengungkapkan, “Dulu, paroki ini masih stasi. Romo F. X Starter SJ yang pertama berkarya di paroki ini. Baru kemudian ditugaskan seorang imam Indonesia, Romo Kusumo SJ. Merekalah yang mengawali pembangunan paroki ini dengan susah payah.” Menurut Kardinal, saat ini yang pantas dibawa Gereja ke tengah masyarakat adalah kehidupan moral yang baik, kejujuran, keadilan, dan keutamaan berbagi kasih.

“Yang diajarkan Gereja harus menjadi ciri-ciri kita. Semoga persaudaraan kristiani menular dan akhirnya menjadi rahmat bagi semua orang di sekitar kita,” harap Kardinal.

\sumber{H. Bambang S. Majalah Hidup}
