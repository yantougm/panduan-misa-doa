
\chap{Komuni Kudus dan Hidup Kita}


        Tema \textit{Warta Iman} (WI) kita bulan ini adalah KOMUNI. Istilah atau kata ini tentu  tidak asing di telinga kita. Barangkali makna dan pengaruhnya dalam hidup yang harus lebih kita dalami. Sebab untuk itulah  komuni lalu menjadi sebuah ``kata yang hidup'', sebuah kata yang memiliki realita konkrit dalam hidup  sehari-hari.

       Saya kira banyak dari kita paham bahwa komuni adalah salah satu  inti pokok dari liturgi ekaristi disamping doa syukur agung. Maksudnya ialah kalau ada suatu  liturgi tanpa melibatkan doa syukur agung dan komuni, maka liturgi itu tidak bisa disebut liturgi ekaristi. Kendati demikian agar inti pokok itu terlaksana dalam liturgi ekaristi maka persembahan atau persiapan roti dan anggur diperlukan dan ini menjadi sangat penting dan syarat mutlak, meskipun tidak menjadi bagian inti liturgi ekaristi. Sebab tidak mungkin ada komuni tanpa  roti dan anggur. Ketiga hal itu lalu secara bersama-sama  merupakan liturgi ekaristi.

      Namun demikian liturgi ekaristi sebagai inti perayaan ekaristi atau misa tidak berhenti di situ. Dalam tata ekaristi, liturgi sabda menjadi bagian penting dan diperlukan. Sebab di sana umat dipersiapkan untuk mengenang dan merenungkan karya keselamatan Allah dalam hidup, pewartaan dan karya Yesus dan terutama wafat dan kebangkitanNya. Dengan demikian umat kemudian bisa mengucap puji syukur kepada Allah sebagaimana itu menjadi inti pokok dari doa syukur agung. Oleh karena itu liturgi sabda benar-benar menjadi persiapan penting untuk liturgi ekaristi .

\section*{Pengaruh Komuni Dalam Hidup Kita}
      Dalam liturgi ekaristi ada tiga bagian pokok, tetapi dua di antaranya adalah inti liturgi ekaristi. Bagian pokok itu adalah persembahan, doa syukur agung dan komuni. Dua yang terakhir disebut inti liturgi ekaristi. Kendati persembahan tidak menjadi bagian dari inti liturgi ekaristi, namun persembahan menjadi syarat penting bahkan mutlak untuk terlaksananya liturgi ekaristi, sebab di dalam liturgi ekaristi dibutuhkan  roti dan anggur sebagai lambang yang pada hakekatnya mengandung makna dan nilai dari pemberian umat selain pemberian uang untuk keperluan Gereja dan kaum miskin. Persembahan lalu dimengerti sebagai pemberian umat bukan pertama-tama kepada Allah melainkan kepada Gereja yang kemudian oleh Gereja dipersembahkan atau dihunjukkan kepada Tuhan  di dalam liturgi ekaristi. Dalam arti inilah maka dalam perayaan ekaristi atau misa umat diperbolehkan  persembahan berupa hasil bumi misalnya padi, sayuran dan buah-buahan dls. Namun roti dan anggur tetap menjadi bahan penting untuk terlaksananya liturgi ekaristi. 

     Doa syukur agung dan komuni merupakan pusat dan puncak dari liturgi ekaristi (\textit{fons et culmen}). Atau dengan kata lain inilah inti liturgi ekaristi sebab di dalam doa syukur agung dan komuni itu iman Gereja diungkapkan secara resmi dalam doa puji syukur dan dalam doa pengudusan. Intinya adalah Gereja mengajak kita / umat bersyukur atas kebaikan Tuhan bukan saja kebaikan masa lampau tetapi terutama masa kini. Kita diajak bersyukur atas kekuatan Tuhan dan daya illahiNya, bersyukur atas keagungan dan kasihNya yang dinyatakan kepada kita. Bersyukur atas perlindungan dan jaminan Tuhan atas hidup kita dan atas kasihNya kepada kita yang tiada tara. Karena itu pantaslah kita  mengenang tanda kasih Allah itu teristimewa kasih Allah yang terbesar yaitu wafat Kristus (\textit{anamnese}) yang tentu juga atas kebangkitanNya yang mulia.

      Seluruh isi doa puji syukur sebagai ungkapan resmi iman Gereja, menjadikan kita dipersatukan dengan Kristus dan Allah oleh dan karena iman itu ketika kita mengenang wafat dan kebangkitanNya dan sekaligus dengan itu kita diperkenankan mengambil bagian di dalam pengharapan akan kebangkitan dan kehidupan abadi. Dan berkat iman Gereja juga dan oleh karya Roh Kudus, melalui doa pengudusan atas roti dan anggur kita dapat berjumpa dengan Kristus. Roti dan anggur berubah bukan sekedar menjadi tanda tetapi sungguh-sungguh Kristus hadir. Barangsiapa menyambut roti dan anggur itu dengan menyantapnya. ia menyambut kehadiran Kristus. Kristus hadir dalam hati orang yang menyambutNya Kristus hadir di dalam Gereja. Kristus hadir dalam setiap perayaan Gereja

      Maka komuni menjadi bagian inti liturgi yang penting. Sebab persatuan kita dengan Kristus terjadi ketika kita menyambut roti dan anggur yang telah dikuduskan dengan menyantapnya.  Melalui komuni sesuai dengan artinya (\textit{Communio}-kesatuan) maka kita dipersatukan dengan Kristus. Kita tinggal di dalam Kristus dan Kristus juga tinggal di dalam kita. Persatuan demikian menurut St. Yohanes akan menghasilkan buah. `` Barangsiapa tinggal di dalam Aku dan Aku tinggal di dalam dia, ia akan berbuah banyak'' (Yoh. 15: 4 -5) Maka setelah sekian kali -- mungkin ratusan kali -- kita menyambut komuni, sungguhkah kita telah menghasilkan buah ?

        Menurut Bapa Suci Benediktus XVI – dalam buku \textit{Paus Benediktus XVI, Sepuluh Gagasan Yang Mengubah Dunia}, Kanisius 2007 - ekaristi bila disikapi dengan sungguh-sungguh merupakan kekuatan yang luar biasa, kekuatan yang mampu mengubah dunia. Pada level pribadi kekuatan yang mampu mengubah dunia itu ialah iman ekaristi yaitu iman yang mendorong kita menjalani kehidupan menurut model Kristus. Apa yang telah kita santap (komuni) menjadikan hidup kita selaras dengan model Kristus itu (St. Augustinus). Dengan kata lain setiap orang yang dipimpin Roh karena persatuan kita dengan Kristus maka orang itu akan menghasilkan buah. Dan buah dari roh  menurut St. Paulus adalah kasih, sukacita, kebaikan, damai sejahtera, kesabaran, kemurahan, kesetiaan, kelemahlembutan, penguasaan diri (Gal. 5: 22) Dan kasih sendiri adalah sabar, murah hati, tidak sombong, tidak egois, tidak dendam, tidak iri hati (cemburu) (1 Kor. 13:4-5) Inilah beberapa  bagian dari hidup menurut model Kristus  

        Tentu bagi kita iman ekaristi bahwa persatuan kita dengan Kristus terjadi melalui komuni kudus selayaknya terus menerus kita renungkan: apakah kita memang telah dan sedang mengalami perubahan atau pembaharuan hidup karena sering menyambut komuni kudus? Katakanlah dalam kehidupan sehari-hari,  yang semula kita suka marah-marah misalnya kini menjadi lebih sabar, lebih mampu mengontrol emosi. Mampu mengusai diri dari kecenderungan-kecenderungan berbuat buruk, tidak lagi gemar membicarakan keburukan orang lain ``\textit{nggrenengi}'', menjadi bijaksana dalam bicara terutama menyangkut orang lain dan keburukannya. Tindakan-tindakan kebaikan yang didasari kasih makin menonjol.  Dalam kehidupan religius secara pribadi,  makin kentara mulai mempunyai kebiasaan melakukan doa pribadi atau perenungan sabda Tuhan (membaca Kitab Suci), rindu untuk selalu ikut perayaan ekaristi atau kegiatan olah rohani lainnya. Tentu kita masih bisa menggali lebih dalam.

         Dalam suatu riset yang diterbitkan oleh seorang professor dari Universitas Virginia Bradford Wilcox dan polling yang dilakukan oleh Gallup sebuah organisasi riset di Amerika menyimpulkan bahwa orang yang rajin mengikuti kebaktian terutama para pria Amerika hidupnya akan lebih bahagia. Pernikahannya bila pria itu sudah menikah, ia lebih bertanggung jawab terhadap pendidikan anak dan tidak suka melakukan kekerasan fisik terhadap anak maupun isteri. Orang juga memiliki emosi positip dari pada orang yang tak pernah datang ke Gereja yaitu bisa menikmati hidup, segar dan murah senyum serta suka melakukan sesuatu yang menarik dalam kebaikan.(http://www.jawaban.com)

Riset ini sebetulnya bukan sesuatu yang baru berkaitan dengan iman ekaristi, namun toh bisa memberikan gambaran yang sama bahwa kebiasaan menyambut tubuh dan darah Kristus (komuni) secara personal seharusnya makin mengubah hidup kita, hidup yang diperbaharui.

       Pada level sosial, iman ekaristi, persatuan kita dengan Kristus mendorong kita melakukan usaha-usaha membangun suatu dunia atau masyarakat di mana kasih Kristus yang selalu diperbaharui pada waktu ekaristi,  menjadi landasan bagi masyarakat yang dibangun itu sekalipun  terpaksa berlawanan dengan ideologi, keuntungan atau kehendak buta akan kekuasaan. Bapa Paus mengingatkan kita betapa pentingnya kasih yang ditimba dari kasih Kristus menjadi landasan dalam usaha-usaha membangun kehidupan bersama termasuk keluarga dan linkungan kita St. Petrus.  Bila hal itu terus bergerak dari kehidupan bersama yang satu ke kehidupan bersama yang lain, maka dunia akan sungguh berubah  Menurut St. Yohanes ``Inilah kemenangan yang mengalahkan dunia, iman kita `` (1 Yoh. 5:4) Tentu saja usaha-usaha demikian tidak mudah, sebab harus mengadapi berbagai tantang yang tidak ingin dunia berubah.

       Bagaimana dengan kehidupan bersama kita terutama keluarga dan lingkungan St. Petrus? Tentu saja bila kita sebagai pribadi senantiasa menyambut komuni, kita percaya  kasih Kristus akan makin kentara dalam usaha-usaha yang kita lakukan demi dan untuk keluarga dan lingkungan kita.yang lebih baik. Ledakan kebaikan dari pemecahan nuklir inti terdalam makhluk yaitu kemenangan kasih atas kebencian dan kematian, sedikit demi sedikit akan memicu rangkaian perubahan lain  menuju dunia yang lebih baik – demikian Bapa Paus menggambarkan kekuatan kasih Kristus yang senantiasa disegarkan dalam perayaan ekaristi bagi kehidupan bersama.

\section*{Hidup Subur}
      Akhirnya kita tahu bahwa buah-buah kehidupan itu muncul dari hidup yang subur. Hidup yang berasal dari penyerahan diri kita kepada Allah, sebagai jawaban atas undangan Kristus kepada kita ``Tinggallah di dalam Aku, dan Aku di dalam kamu'' (Yoh. 15:4) Inilah kehidupan rohani kita. Akan tetapi kehidupan rohani itu tidak dengan sendirinya menghasilkan buah bila tidak dipupuk dengan kasih mesra. Itulah sebabnya mengapa St. Yohanes berbicara mengenai iman dan kasih itu memiliki kaitan erat bahkan azasi. Karena iman kepada Allah dalam Roh Yesus, kita menerima Allah sebagai Bapa (1 Yoh.4:26) dan karena itu kita harus melakukan perintah-perintahNya. Perintah itu ialah mengasihi Allah (Mrk. 12:30). Akan tetapi kasih kepada Allah itu membutuhkan ungkapan dan ungkapan itu nyata dalam kasih kita kepada sesama.

\small
       Bila Yohanes Pembaptis berbicara keras tentang pohon  yang tidak berbuah ditebang dan dibuang ke dalam api  (Mat. 3:10) itu mengingatkan kita akan arti penting hidup rohani kita. Sebab hidup rohani yang tidak berbuah adalah hidup rohani yang mandul. Tentu bagi kita sangat tidak nyaman atau enak disebut sebagai orang mandul. Barangkali yang lebih menyakitkan kalau kita disebut orang katolik yang tidak menghasilkan buah kakatolikkannya, sekalipun setiap hari kita menyambut komuni. Untuk itu marilah merenungkan prinsip-prinsip dasar hidup seorang yang memiliki hidup subur sebagai penutup tulisan ini. Ada 3 segi kehidupan yang subur  menurut Henri JM. Nouwen dalam bukunya ``\textit{Tanda-Tanda Kehidupan}'':

\begin{enumerate}
\item Hidup yang subur adalah hidup yang tidak menutup diri dengan segala pertahanan diri, melainkan hidup yang terbuka terhadap kelemahan sekalipun rawan terhadap luka hidup. Yesus Kristus menjadi pola hidup subur yaitu hidup ``terbuka mudah kena''(\textit{vulnerability}) bahwa sekalipun Allah, Ia tidak mengganggap kesetaraanNya itu sebagai milik yang harus dipertahankan, tetapi justru mengosongkan diriNya sendiri dan menjadi sama dengan manusia (Flp. 2: 6 – 7)
\item Hidup yang subur adalah hidup yang penuh syukur. Syukur yang mengalir dari anugerah Tuhan yang Ia berikan dalam kasih dengan cuma-cuma kepada kita sebagai anugerah dan kita bagirasakan kepada saudara-saudara kita. Dalam kisah pergandaan roti (Yoh. 6 : 5-15) kita melihat bagaimana Yesus memberi contoh menghayati hidup subur. Setelah mengucapkan syukur atas 5 roti dan 2 ikan, Ia menyuruh para murid membagikannya kepada orang banyak. Membagikan kebaikan karena kebaikan hati Allah adalah ciri hidup subur.
\item Hidup subur adalah hidup yang dipelihara dan diperhatikan. Buah kehidupan tidak  muncul dengan sendirinya tanpa ada pemeliharaan dan perhatiaan.Yesus memberikan  contoh kepada kita bagaimana Ia memberikan perhatian kepada orang-orang yang Ia jumpai tanpa memaksa atau mengusai, melainkan melalui sabda dan tindakanNya Ia menawarkan kesempatan kepada mereka untuk mencari arah hidup dan keputusan-keputusan yang baru.
\end{enumerate}

\sumber{Yogyakarta, 17 Mei 2012\\
Hari Kenaikan Tuhan Yesus Ke Surga\\
(AS)}
\normalsize