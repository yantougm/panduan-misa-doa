\chap{Nilai Sopan Santun}

\small
	Jono, pemuda yang baik hati, jatuh cinta setengah mati kepada Rini. Cintanya tak bertepuk tangan sebelah. Jalinan cintanya pun berjalan mulus. Setelah Jono merasa mantap, ia pun mengantarkan Rini ke rumahnya yang jauh, di kampung, sekaligus untuk melamar kepada orangtuanya. Namun, sama sekali tak terduga, lamaran Jono ditolak. Kabarnya, gara-gara pakaian Jono yang kurang sopan. Bahkan, menurut orangtua Rini, pakaiannya itu memuakkan. Sayang sekali. Jono yang sebenarnya anak baik, gara-gara dia tidak suka memperhatikan soal pakaiannya, cintanya kandas. Kelihatannya masalah kecil, nyatanya menjadi persoalan besar.
	
	Karwan, yang terbukti di pengadilan bahwa dia membunuh orang, menjadi terkejut ketika dijatuhi hukuman penjara hanya 5 tahun. Karwan malah menolak, "Bapak hakim, tolong saya dihukum lebih berat lagi. Saya telah berdosa besar, kenapa hanya dihukum 5 tahun? Sebaiknya saya dihukum seumur hidup atau dihukum mati saja." Tetapi, bapak hakim menjawab dengan ramah, "Saudara terdakwa, perkataan saudara itu memang benar. Membunuh itu dosa besar. Tetapi, ada pertimbangan lain yang meringankan hukuman saudara, yaitu saudara masih muda, belum pernah ditahan/dihukum, jujur mengakui kesalahan dan menunjukkan penyesalan yang dalam. Disamping itu, saudara sungguh-sungguh berlaku sopan selam persidangan, Terima dan jalanilah hukuman saudara itu sampai selesai dengan sikap tawakal dan tabah." Karwan menangis terharu dalam pelukan saudara-saudaranya.

	Sopan santun sering kali diremehkan oleh manusia zaman sekarang ini. Dianggap tidak efisien, pemborosan waktu, dan penuh basa-basi saja. Padahal, dimanapun kita berada, ada sopan santunnya sendiri. Di jalan raya, di persidangan, di sekolah, di kantor, di ruang makan, waktu pesta, cara menulis surat, cara menelepon, cara menyambut tamu, cara berpakaian, dan sebagainya. Pendek kata, hidup tak pernah lepas dari sopan santun. Dari kedua kasus diatas nampak betapa tinggi nilai sebuah sopan santun itu.  Bagaimanapun, penampilan seseorang dapat menjadi salah satu petunjuk sejauh mana kepribadiannya. Di samping itu, dengan berlaku sopan sesungguhnya kita menunjukkan juga rasa hormat kita kepada orang lain. Karenanya, dengan sopan santun, kita menghormati orang lain sebagai lebih penting, dan kita sendiri diterimanya sebagai insan yang lebih berbudaya, punya toleransi dengan sesama, dan memahami citra kehidupan. Yang pasti adalah sopan santun merupakan bahasa hati yang indah bagi manusia. Sayang bila orang tidak memanfaatkannya.

\vspace{0.5cm}

{\noindent \emph{Bacaan: Mat 22:11	Ketika raja itu masuk untuk bertemu dengan tamu-tamu itu, ia melihat seorang yang tidak berpakaian pesta.}}
\normalsize