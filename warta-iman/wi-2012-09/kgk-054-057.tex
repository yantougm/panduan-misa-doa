\newpage
\chap{Kompendium Katekese Gereja Katolik}
\setcounter{kgkcounter}{53}
\small
\kgk{Bagaimana Allah menciptakan semesta alam?}
Dari kehendak bebas-Nya, Allah menciptakan semesta alam dalam ke-bijaksanaan dan cinta. Dunia diciptakan bukan karena kebutuhan, atau takdir
          buta, ataupun kebetulan. Allah menciptakan dari ketiadaan (ex nihilo) (2Mak
          7:28) sebuah dunia yang teratur dan baik. Ia jauh mengatasi ciptaan-Nya. Allah
          memelihara ciptaan-Nya dalam keberadaan dan menopangnya, memberinya
          kemampuan untuk bertindak, membimbingnya menuju kepenuhannya melalui
          Putra-Nya dan Roh Kudus.

\kgk{Apa penyelenggaraan ilahi itu?}
Penyelenggaraan ilahi terletak pada kesediaan Allah untuk membimbing
makhluk-makhluk ciptaan-Nya menuju tujuan akhir mereka. Allah adalah Tuan
          yang berkuasa atas rencana-Nya. Tetapi untuk melaksanakannya, Allah juga
          berkehendak untuk bekerja sama dengan makhluk-makhluk ciptaan-Nya. Allah
          menganugerahkan kepada makhluk-makhluk ciptaan-Nya martabat untuk dapat
          bertindak dari kebebasan mereka sendiri dan saling memimpin satu sama lain.

\kgk{Bagaimana kita bekerja sama dengan penyelenggaraan ilahi?}
    Dengan tetap menghormati kebebasan kita, Allah meminta kita untuk                
bekerja sama dengan-Nya dan memberikan kepada kita kemampuan untuk           
melaksanakannya melalui semua tindakan, doa, dan penderitaan kita. Jadi, Allah
membangkitkan dalam diri kita ”kemauan maupun pekerjaan menurut kerelaan-Nya” (Flp 2:13).

\kgk{Jika Allah itu mahakuasa dan mahabaik, mengapa ada kejahatan?}
     Terhadap pertanyaan ini, yang menyedihkan dan sekaligus juga misterius,         
hanya keseluruhan iman Kristenlah yang dapat memberikan jawaban. Allah sama sekali bukanlah penyebab kejahatan, baik langsung maupun tidak langsung.
Dia menerangi misteri kejahatan di dalam Putra-Nya, Yesus Kristus, yang wafat
dan bangkit untuk mengalahkan kejahatan moral itu, yaitu dosa manusia, yang
menjadi akar dari semua kejahatan lain.


\flushright{(\dots \emph{bersambung} \dots)}
\normalsize