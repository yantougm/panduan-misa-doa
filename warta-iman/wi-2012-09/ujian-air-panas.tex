\chap{Ujian Air Panas}

Saudara - saudaraku, anggaplah sebagai kebahagiaan, apabila jatuh kedalam berbagai macam pencobaan, sebab kamu tahu, bahwa ujian terhadap imanmu itu menghasilkan ketekunan.

Ketika kemalangan datang menenggelamkan hidup, apa reaksi anda? Mungkin anda pernah mendengar ilustrasi berikut: kentang, telur, dan bubuk kopi baru saja mengalami kemalangan yang sama. Yaitu, sama – sama dimasukkan dalam air mendidih. Yang berbeda adalah reaksi mereka masing – masing. Kentang mula – mula keras, kuat dan tidak mau tunduk, tapi setelah melewati waktu yang cukup lama, ia menjadi lunak dan lemah. Telur yang awalnya mudah pecah dan rapuh, akhirnya ia menjadi keras dan padat. Lain halnyanya dengan bubuk kopi, semula ia tidak menarik, tapi ketika ia dimasukkan ke air panas, ia justru mampu mengubah air panas di sekelilingnya menjadi kopi yang harum dan memikat.

Masuk ke kelompok mana anda dalam ilustrasi tersebut? Apa persoalan anda hari ini? Pengkhianatan, sakit penyakit, kegagalan. Selamat! Itu berarti anda masuk dalam panasnya air mendidih. Seperti ilustrasi di atas, hasil akhir ada di tangan anda, karena hal itu bergantung pada cara anda bereaksi terhadap masalah dan penyelesaiannya. Maju dan keluar sebagai pemenang atau mundur sebagai pecundang akan menerima upahnya. Dan semua ujian itu akan menghasilkan ketekunan.

Respon positif dalam menghadapi masalah dapat menjadikan anda lebih dari seorang pemenang.

\sumber{www.renungan-harian-kita.blogspot.com}