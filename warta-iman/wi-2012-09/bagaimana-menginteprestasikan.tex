\chap{Perkenalan dengan Kitab Suci}
Orang mengatakan bahwa tak kenal maka tak sayang. Maka agar kita dapat mengasihi Tuhan, kita perlu mengenal Dia. Sekarang pertanyaannya, bagaimana caranya kita mengenal Allah?  Agaknya Tuhan memahami bahwa manusia akan mempunyai pertanyaan semacam ini dalam hatinya, sehingga Allah-lah yang pertama- tama melakukan inisiatif: Ia mewahyukan Diri-Nya, melalui alam semesta, melalui suara hati nurani dan yang secara khusus, melalui Wahyu umum yang diberikan kepada Gereja.
\section*{Apakah Kitab Suci itu?}
Kita sering mendengar bahwa Kitab Suci adalah "Wahyu Allah". Wahyu atau pernyataan Allah tentang diri-Nya ini tidak terlepas dari kenyataan bahwa manusia diciptakan menurut gambar dan rupa Allah (Kej 1: 26). Artinya kita adalah mahluk rohani yang diciptakan menurut gambaran Allah, yang dilengkapi oleh akal budi dan kehendak bebas, sehingga kita dapat mengetahui, memilih dan mengasihi. Dengan demikian, kita manusia dapat menyimpulkan bahwa Tuhan Sang Pencipta itu ada, dengan melihat segala ciptaan-Nya yang ada di sekitar kita. Keberadaan Tuhan juga diketahui dengan memperhatikan suara hati nurani dalam setiap orang, di mana Tuhan menuliskan hukum-hukum-Nya untuk menyatakan hal yang benar dan yang salah; inilah wahyu yang universal. Selanjutnya, Tuhan juga secara khusus memberikan wahyu yang merupakan pernyataan akan diri-Nya dan kehendak-Nya bagi manusia untuk mencapai tujuan akhir yang direncanakan-Nya. Wahyu ini disampaikan kepada manusia sejak awal sejarah manusia sampai sekarang; yaitu  melalui para nabi, yang mencapai puncaknya di dalam Yesus Kristus Putera-Nya, dan wahyu ini yang kemudian dilanjutkan oleh para rasul Kristus.

Wahyu umum ini adalah wahyu Allah yang khusus diberikan kepada umat manusia agar manusia dapat mengenal siapa diri-Nya dan rencana-Nya untuk menyelamatkan manusia. Wahyu umum ini bermula dari wahyu yang diberikan kepada para nabi, dan berakhir dengan wafatnya rasul Kristus yang terakhir. Wahyu umum ini terdiri dari dua jenis, yang tergantung dari cara penyampaiannya; yaitu Kitab Suci (tertulis) dan Tradisi Suci (lisan). Maka kita ketahui ketiga hal ini:
\begin{enumerate}
\item Kitab Suci adalah Wahyu ilahi yang disampaikan secara tertulis di bawah inspirasi Roh Kudus.
\item Tradisi Suci adalah Wahyu ilahi yang tidak tertulis, namun yang diturunkan oleh para rasul sejak awal oleh inspirasi Roh Kudus, sesuai dengan yang mereka terima dari Yesus dan yang kemudian diturunkan kepada para penerus mereka.
\item Maka kita ketahui sekarang bahwa untuk menerima wahyu Allah secara lengkap, kita tidak hanya perlu Kitab Suci, namun juga Tradisi Suci, dan pihak wewenang mengajar Gereja (Magisterium) yang dapat secara benar mengartikan wahyu ilahi tersebut. Ketiga hal ini disebut sebagai pilar iman, yang ditujukan untuk menjaga dan mengartikan wahyu publik dari Allah ini di dalam kemurniannya.
\end{enumerate}

\section*{Kitab Suci yang adalah Sabda Allah itu terdiri atas Perjanjian Lama dan Perjanjian Baru}
Pernahkah anda membaca novel, namun hanya membaca bagian akhirnya saja? Walaupun mungkin anda dapat menangkap bagian yang terpenting dari kisah tersebut, namun tentu, kisah tersebut akan lebih dapat dipahami, jika anda membaca buku tersebut mulai dari bagian awal. Demikianlah halnya dengan Kitab Suci yang merupakan Sabda Allah yang tertulis tentang rencana keselamatan Allah yang dimulai sejak awal mula penciptaan dunia, sampai penggenapannya di dalam diri Kristus.

Oleh karena itu, untuk mempelajari Kitab Suci, kita perlu melihat kaitan antara Perjanjian Lama (sebelum kedatangan Kristus) dan Perjanjian Baru (saat dan setelah kedatangan Kristus), dan antara ayat yang satu dengan ayat yang lainnya, untuk mendapat pengertian yang menyeluruh dan pemahaman yang benar akan Sabda Allah itu. Perjanjian Lama (PL) yang melatar-belakangi Perjanjian Baru (PB), merupakan satu kesatuan dengan Perjanjian Baru. Sebab "Perjanjian Baru terselubung dalam Perjanjian Lama, sedangkan Perjanjian Lama tersingkap dalam Perjanjian Baru.". Sama seperti suatu kisah tidak akan lengkap jika hanya dibaca awalnya saja, atau akhirnya saja, tanpa memperhatikan kaitannya, demikian juga Kitab Suci hanya akan dapat kita pahami secara menyeluruh dalam kesatuan antara PL dan PB, dan dalam kaitan satu ayat dengan ayat yang lain.

\section*{Penulisan Kitab Suci melibatkan akal budi para penulisnya}
Kitab Suci merupakan Sabda Allah yang disampaikan melalui tulisan penulis kitab yang ditunjuk oleh Allah untuk menuliskan hanya yang diinginkan oleh Tuhan. Namun demikian, ini melibatkan juga kemampuan sang penulis tersebut dalam hal gaya bahasa, cara penyusunan, latar belakang budayanya, dst. Maka jika kita ingin memahami Kitab Suci, kita perlu mengetahui makna yang disampaikan oleh para pengarang kitab dan apakah yang ingin disampaikan oleh Allah melalui tulisannya. Karena Kitab Suci bersumber pada Allah yang satu, maka kita harus melihat keseluruhan Kitab Suci, walaupun ditulis oleh orang yang berbeda- beda, sebagai satu kesatuan yang saling melengkapi. Inilah yang menjadi dasar bagaimana kita memperoleh pengertian yang mendalam tentang Kitab Suci, dan dengan cara demikianlah jemaat awal mengartikan Kitab Suci.

\section*{Kitab Suci itu diberikan kepada Gereja sebagai pedoman}
Rasul Paulus memberikan alasan kepada kita untuk mempelajari Kitab Suci yaitu, "Segala tulisan yang diilhamkan Allah memang bermanfaat untuk mengajar, untuk menyatakan kesalahan, untuk memperbaiki kelakuan dan untuk mendidik orang dalam kebenaran" (2 Tim 3:16) agar kita yang menjadi umat-Nya diperlengkapi untuk setiap perbuatan baik. Dengan demikian, Kitab Suci mendapatkan tempat yang begitu tinggi di dalam Gereja Katolik, yang dapat kita lihat dari dokumen-dokumen Gereja yang senantiasa mempunyai sumber dari Kitab Suci disamping Tradisi Suci, dan kita dapat melihat secara jelas dalam liturgi Gereja. Paus Benediktus XVI dalam pengajaran apostoliknya, Verbum Domini, mengajarkan bahwa Kitab Suci mendapatkan tempat di dalam sakramen-sakramen, liturgi, brevier, buku-buku doa dan pemberkatan, lagu-lagu, dll.


\chap{Prinsip untuk menginterpretasikan Kitab Suci}
\section*{Cara umum:}
Konsili Vatikan II mengajarkan tiga cara umum untuk menafsirkan Kitab Suci sesuai dengan Roh Kudus yang mengilhaminya:
\begin{enumerate}
\item Memperhatikan isi dan kesatuan seluruh Kitab Suci

Kita harus mengartikan ayat tertentu dalam Kitab Suci dalam kaitannya dengan pesan Kitab Suci secara keseluruhan. Mengartikan satu paragraf atau bahkan satu kalimat saja namun tidak memperhatikan kaitannya dengan ayat yang lain, dapat berakibat fatal. Contohnya, seorang atheis mengutip Mzm 14:1, dan berkata "Tidak ada Allah". Tetapi sebenarnya, keseluruhan kalimat itu berkata, "Orang bebal berkata dalam hatinya: "Tidak ada Allah". Maka arti yang disampaikan dalam Kitab Suci tentu sangat berbeda dengan pengertian orang atheis tersebut.

\item Membaca Kitab Suci dalam terang Tradisi hidup seluruh Gereja

Banyak ahli Kitab Suci di jaman modern yang tidak mengindahkan interpretasi yang berakar dari tradisi Gereja. Mereka berpikir seolah-olah baru pada saat mereka menginterpretasikan Kitab Suci, Roh Kudus memberikan pengertian yang paling "asli", sedang interpretasi pada abad- abad yang lalu itu keliru. Sikap ini tentunya tidak mencerminkan kerendahan hati. Gereja mengajarkan bahwa kita harus menginterpretasikan Kitab Suci sesuai dengan Tradisi hidup seluruh Gereja, sebab "Kitab suci lebih dulu ditulis di dalam hati Gereja daripada di atas pergamen (kertas dari kulit)". Di dalam Tradisi Suci inilah Roh Kudus menyatakan kenangan yang hidup tentang Sabda Allah dan interpretasi spiritual dari Kitab Suci. Tradisi Suci tercermin dari tulisan Para Bapa Gereja, dan ajaran- ajaran definitif yang ditetapkan oleh Magisterium, seperti yang dihasilkan dalam Konsili-konsili, Bapa Paus maupun yang dijabarkan dalam doktrin Gereja.

\item Memperhatikan "analogi iman"

Analogi iman maksudnya adalah bahwa wahyu Allah berisi kebenaran- kebenaran yang konsisten dan tidak bertentangan satu sama lain. Gereja Katolik percaya bahwa Roh Kudus yang meng-inspirasikan Kitab Suci adalah Roh Kudus yang sama, yang membimbing dan menjaga wewenang mengajar Gereja (Magisterium), yang juga bekerja dalam Tradisi Suci Gereja. Maka tidak mungkin ajaran Gereja Katolik bertentangan dengan Kitab Suci, karena Roh Kudus tidak mungkin bertentangan dengan diri-Nya sendiri. Juga, karena Gereja menjaga kemurnian ajaran dalam Kitab Suci, maka untuk meng-interpretasikan Kitab Suci, kita harus melihat kaitannya dengan ajaran/ doktrin Gereja.

Analogi iman yang berdasarkan ajaran Gereja berperan sebagai "penjaga" yang membantu kita agar kita tidak sampai salah jalan dalam meng-interpretasikan Kitab Suci. Ibaratnya, seperti pagar yang membatasi rumah kita dengan dunia luar yang penuh dengan anjing galak. Di dalam halaman rumah, kita tetap dapat beraktivitas, anak-anak dapat bermain dengan bebas, namun aman dari bahaya. Maka dengan berpegang pada ajaran Gereja, kita tetap mempunyai kebebasan dalam menginterpretasikan ayat-ayat Kitab Suci, namun kita dapat yakin bahwa interpretasi kita tidak salah, ataupun tidak bertentangan dengan kebenaran yang diwahyukan. Keyakinan ini merupakan karunia yang diberikan kepada kita, jika kita setia berpegang pada pengajaran Gereja yang disampaikan oleh Magisterium (Wewenang mengajar Gereja). Magisterium inilah yang bertugas menginterpretasikan Sabda Allah dengan otentik, baik yang tertulis (Kitab Suci) maupun yang lisan (Tradisi Suci), dengan wewenang yang dilakukan dalam nama Tuhan Yesus agar Sabda itu dapat diteruskan sesuai dengan yang diterima oleh para rasul.
\end{enumerate}

\section*{Menghindari dualisme hermenetik sekular/ ‘secularized hermeneutic’ dan interpretasi fundamentalis/"fundamentalist interpretation"}
Paus Benediktus XVI dalam ekshortasi apostoliknya, Verbum Domini menekankan dua kesalahan ekstrem yang tidak boleh dilakukan oleh umat Katolik dalam menginterpretasikan Kitab Suci. Dua kesalahan ini adalah hanya berdasarkan metoda sekular dan metoda fundamentalisme.
\begin{enumerate}
\item Metoda historical-criticism yang terpisah dari teologi

Metode sekular yang terpisah dari teologi ini menekankan sisi sejarah dari Kitab Suci, sehingga Kitab Suci hanya dilihat sebagai buku dari masa lampau yang tidak mempunyai kaitan dengan saat ini. Pendekatan yang ilmiah dengan mengesampingkan sisi-sisi Ilahi dari Kitab Suci membuat metode ini kehilangan apa yang menjadi dasar untuk mengerti Kitab Suci – yaitu iman – yang pada akhirnya menolak campur tangan Tuhan dalam sejarah manusia. Inilah sebabnya, Paus Yohanes Paulus II – dalam ensiklik Fides et Ratio – dan Paus Benediktus XVI menekankan harmoni antara iman dan akal budi. Menyandarkan metoda ilmiah dalam menginterpretasikan Kitab Suci tanpa dibarengi iman, mereduksi sisi Ilahi dari Kitab Suci menjadi buku sejarah atau hanya menjadi buku literatur biasa. Sebagai contoh: Ketika di Kitab Suci dikatakan bahwa Yesus memberi makan lima ribu orang (Mt 14:15-21; Mk 6:34-44), maka metoda ini cenderung untuk mereduksi sisi Ilahi dari Kristus dan kemudian menjelaskannya dengan sesuatu yang lebih ilmiah, seperti orang-orang yang ada berkumpul mengeluarkan bekal masing-masing dan kemudian saling berbagi.

\item Metode fundamentalisme

Metode fundamentalisme mengambil kata demi kata di Kitab Suci dan menganggap kata demi kata dalam Kitab Suci adalah didikte oleh Tuhan tanpa melihat bahwa penulisan Kitab Suci senantiasa di dalam konteks sejarah pada waktu tulisan tersebut dibuat. Penolakan akan sisi sejarah dan juga penolakan akan Gereja sebagai pemberi interpretasi Kitab Suci yang otentik membuat metode ini menjadi sangat subyektif yang mengarah bahwa interpretasi pribadinya sendiri adalah yang paling benar. Metoda ini juga dapat menyebabkan kegagalan untuk melihat Sabda Allah dalam konteks keseluruhan. Sebagai contoh, ketika Yesus mengatakan "Tetapi tentang hari dan saat itu tidak seorangpun yang tahu, malaikat-malaikat di sorga tidak, dan Anakpun tidak, hanya Bapa sendiri." (Mt 24:36), maka metoda ini mempunyai tendensi untuk mengatakan bahwa Yesus memang tidak mengetahui kapan akhir dunia terjadi, tanpa melihat kompleksitas dari kodrat Yesus yang sungguh Allah dan sungguh manusia. Karena mereka juga tidak mau melihat komentar Kitab Suci dari Bapa Gereja dan Magisterium Gereja, maka mereka akan mengeraskan hati bahwa interpretasi merekalah yang paling benar.
\end{enumerate}


