\chap{Bagaimana Injil Ditulis?}

\section*{Dokumen-dokumen}
Semua penerbitan Perjanjian Baru dan khususnya Injil, entah dalam bahasa Inggris, Spanyol, atau bahasa-bahasa lain, adalah hasil terjemahan dari teks-teks asli yang ditulis rlulam bahasa Yunani. Naskah-naskah kuno yang berisi teks-teks ini disalin berkali-kali, sampai akhirnya setiap teks dirapikan berkat penemuan mesin cetak; hal ini terjadi kira-kira tahun 456 di mana Guttenberg mencetak Kitab Suci yang pertama.

Orang-orang yang menyalin teks-teks ini tidak dapat menghindari beberapa kesalahan. Dengan membandingkan berbagai teks, pengelompokannya menurut perbedaan dan asal-usulnya, dan mengritiknya, kita dapat menentukan mana-mana saja teks asli yang diakui Gereja Katolik sebagai pernyataan iman para rasul dan sabda Allah. Pertanyannya: siapa yang menulis injil-injil yang pertama ini dan apa yang menjadi sumber mereka?

Beberapa naskah Perjanjian Baru yang hangus dari abad ke-4 telah diselamatkan. Naskah-naskah ini dikukuhkan oleh beberapa dokumen lainnya yang jauh lebih tua yang berisi paragraf-paragraf atau kadang-kadang kitab-kitab lengkap Perjanjian Baru. Selanjutnya, para penulis Kristen dari abad ke-2 dan ke-3 sering kali mengutip teks-teks suci yang sudah dikomentari. Injil Yohanes diperkiraan ditulis antara tahun 90 sampai 100, dan penggalan-penggalannya ditemukan di Mesir, sangat jauh dari tempat asalnya. Penggalan-penggalan itu tertanggal antara tahun 120-130.

Selanjutnya, kita akan memberi perhatian khusus pada Injil-injil, meskipun Injil-injil bukan merupakan tulisan Perjanjian Baru yang paling kuno saat ketiga Injil pertama ditulis pada tahun 50-70, Paulus sudah mengirim surat-surat aslinya.

\section*{Penulis-Penulis Injil}
Menarik untuk diperhatikan bahwa para ahli sejarah Gereja yang pertama sudah menyebutkan secara khusus orang-orang yang dianggap oleh tradisi sebagai penulis ketiga Injil sinoptik.

Pada tahun 110, Papias dari Herapolis (dekat Efesus) menulis: "Markus, penerjemah Petrus, menulis dengan tepat, meskipun tidak dalam susunan yang teratur, semua yang diingatnya mengenai perkataan-perkataan dan perbuatan-perbuatan Tuhan. Ia menemani Petrus yang mengajar sesuai dengan kebutuhan waktu, bukan dalam bentuk suatu komposisi, dan ia tidak melakukan kesalahan dalam memasukkan beberapa hal yang diingatnya. Matius mengumpulkan perkataan-perkataan Tuhan dalam bahasa Ibrani dan selanjutnya setiap orang menerjemahkannya sesuai dengan kemampuannya".

Pada tahun 185 Santo Ireneus, uskup dan martir, menulis, "Matius mewartakan injil di antara orang-orang Ibrani dan dalam bahasa mereka, sementara Petrus dan Paulus pergi keluar untuk menginjili Roma dan mendirikan gereja. Setelah mereka meninggalkan Roma, Markus, seorang murid dan penerjemah Petrus, menuliskan ajaran Petrus. Lukas, teman Paulus juga menulis sebuah kitab mengenai injil yang diajarkan oleh Paulus".

Sumber-sumber kuno yang mana kita masih bisa menambahnya, diteliti dengan seksama oleh banyak ahli kitab suci modern, dan akhirnya mereka diterima sekali lagi sebagai informasi benilai sejarah.

Selanjutnya, adalah suatu kesalahan untuk berpikir bahwa injil-injil telah ditulis dalam suatu kepingan oleh orang-orang seperti Matius, Markus atau Lukas yang pada suatu waktu tertentu memutuskan untuk mencatat dengan menuliskan pelayanan aktif dan ajaran Yesus.

\section*{Dari Tradisi Lisan ke Injil}
Kita tahu bahwa Yesus telah wafat dan Ia wafat tanpa menuliskan apapun. Yesus telah mengabdikan sebagian terbesar waktunya untuk membentuk kedua belas rasul yang telah dipilih-Nya. Mereka tinggal bersama Dia, sebagaimana kebiasaan dari murid-murid guru Yahudi. Yesus meminta mereka mempelajari ajaran-Nya dengan hati ketimbang melipat gandakan pengajaran-pengajaran-Nya, Yesus mengulangi kebenaran-kebenaran esensial dalam banyak cara. Kita tidak dapat meragukan itu bahwa setelah hari Pentakosta, perhatian mereka adalah memberi bentuk kepada perintah-perintah Yesus ini, yang kemudian menjadi katekismus Gereja Perdana. Pada awalnya para rasul memberi kesaksian tentang apa yang telah mereka lihat dan dengar. Lambat laun timbul kebutuhan untuk memiliki suatu catatan tertulis mengenai kesaksian mereka untuk menjaga ingatan: kita sendiri sering kali melakukan ini dalam suatu pertemuan, ketika sharing dari para peserta dicatat untuk kepentingan orang-orang yang tidak hadir.

Komunitas-komunitas Kristen, di Palestina berbicara dalam bahasa Aram atau Ibrani sesuai dengan daerah atau lingkungan. Dengan demikian tulisan yang pertama dimunculkan dalam kedua bahasa ini. Lambat laun teks-teks yang berkaitan dengan apa yang diucapkan dan diperbuat Yesus dikelompokkan kembali; dalam hal ini komunitas-komunitas Kristen pertama meneruskan dari kesaksian lisan menjadi teks tertulis: dan itu adalah injil.

Pada waktu itu komunitas-komunitas Kristen yang berbahasa Yunani telah menjadi kelompok mayoritas dan teks-teks kuno diterjemahkan ke daJam bahasa itu (Yunani).

\section*{Injii Yohanes}
Ketiga penulis injil yang pertama tidak hanya berbeda dalam fokus mereka tetapi juga agak berbeda dalam menyajikan perbuatan-perbuatan dan perkataan-perkataan Yesus; sebetulnya masing-masing mempunyai teologi-nya sendiri, caranya mengenal Yesus, dari pandarig an 'yang mendalam' serta kesaksian pribadi inilah yang akhirnya menentukan perbedaan.

Dalam Injil Yohanes kita menemukan, bagian-bagian suaru injil kuno yang sesederhana lnjil Markus, dengan lebih banyak perbuatan daripada perkataan Yesus, yang mungkin telah disampaikan kepada komunitas umat Kristen di Samaria, dan yang ditulis dalam bahasa Aram. Inilah dasar dimanu Yohanes mengembangkan uraian yang panjang mengenai Yesus yang memperlihatkan bahwa keselamatan mengubah umat manusia dan memperbarui ciptaan.

\section*{Apakah kita dapat mempercayai perkataan Injil?}
Sebagian besar dari kita mungkin sudah menanyakan ini: mengapa kita mempunyai empat kesaksian ketimbang satu, dan apa nilainya masing-masing, Mengikuti apa yang baru saja kita katakan akan lebih mudah mengerti apa yang menyusul:
\begin{itemize}
\item Tidak semua perbuatan dan perkataan Yesus ditemukan dalam injil.
\item Berkenaan dengan kata-Kata Yesus, setiap penulis injil mengungkapkannya dalam caranya sendiri dan menerapkannya untuk pemahaman yang lebih baik bagi para pembacanya.
\item Kejadian-kejadian tidak selalu diceritakan berurutan sesuai dengan terjadinya; dan hal-hal yang dikatakan Yesus pada kesempatan yang berbeda dapat digabungkan dalam nas yang sama.
\end{itemize}

Ini tidak mengatakan bahwa kita tidak dapat mempercayai kesaksian para penulis injil. Kita tidak diberi sebuah "foto", sebuah rekaman kata-kata Yesus, tetapi sebaliknya empat pandangan yang berbeda yang saling melengkapi. Mengapa khawatir jika terdapat pertentangan-pertentangan tertentu dalam perincian-perincian, Jika di pintu gerbang Yeriko terdapat satu atau dua orang buta, apa bedanya dalam pesan pokok yang disampaikan?.

\section*{Tempat Khusus Injil-injil dalam Literatur}
Injil-injil adalah karya kekecualian khusus diantara karya-karya tulisan sepanjang zaman. Perbandingan-perbandingan tulisan-tulisan lain dari zamannya, Kristen atau bukan, memperlihatkan suatu kontras luar biasa -- dalam injil, kesederhanaan keinginan untuk bersabar, sementara teks-teks yang lain, lebih mengutamakan apa-apa yang hebat, kompleks dan tidak berakar rumput. Seorang filsuf modern -- bukan orang beriman -- bertanya mengapa tidak dapat lebih banyak mukjizat di dalam Injil. Injil-injil memiliki jaminan keasliannya sendiri. Mempertimbangkan apa yang dikatakan dalam paragraf terdahulu, kritik modern belum dapat menemukan kekeliruan di dalam Injil meskipun ia telah memeriksanya dengan teliti dengan menggunakan kaca pembesar selama lebih dari satu abad. Apalagi: injil menyapa kita dengan suatu perasaan penuh arti setiap kali kita mampu membuka diri kita kepadanya.

\section*{Mereka Meragukannya}
Masih banyak orang yang mempertanyakan kesaksian Injil. Kadang-kadang disebabkan karena mereka mengira mereka melihat pertentangan-pertentangan di dalam Injil; tetapi lebih sering karena tampaknya tidak mungkin bagi mereka untuk menerima mukjizat-mukjizat. Bahkan di antara orang-orang beriman yang mempelajari Injil, beberapa orang memiliki sejumlah keraguan berkaitan dengan nilai historis segala sesuatu yang dapat disebut mukjizat dalam arti harfiah.

Hal ini mungkin disebabkan oleh kenyataan bahwa mereka telah dididik dalam budaya "ilmiah" yang hanya mengandalkan pengertian manusia untuk mengatasi setip masalah. dalam suatu dunia yang melindungi dirinya dengan asuransi, sedikit yang diharapkan dari Allah dan Allah tidak melipatgandakan mukjizat.

Mereka berpikir sebagai berikut: Jika sekarang saya tidak dapat melihat apa pun yang sama dengan apa yang terjadi dalam Injil, bagaimana saya bisa percaya bahwa hal-hal semacam itu juga terjadi kemudian? Segala sesuatu barangkali berbeda jika mereka melibatkan diri di dalam komunitas-komunitas Kristen yang miskin atau tertindas. DI sana mereka mungkin dapat menyaksikan campur tangan Allah yang terus-menerus demi kebaikan orang-orang yang hanya dapat berharap kepada-Nya saja. Sesungguhnya, dalam kumuniras-komuniras ini dikatakan: Jika hari itu Allah mengerjakan mukjizat-mukjizat Itu, mengapa Ia tidak mungkin mengerjakannya pada zaman Yesus seturut kehendak-Nya?

Sesungguhnya, tidak mungkin mempelajari Injil secara parsial, sebagaimana yang bisa lakukan terhadap buku-buku lainnya karena lnjil mengajukan pertanyaan mengenai keseluruhan hidup kita dan bukann hnaya mengenai gagasan-gagasan tertentu. Jika kita mengambil bagian dalam iman pararasul, kita seharusnya tidak mengalami kesulitan untuk menerima Kitab Suci seraya menyadari pertanyaan-pertanyaan yang kritis. Tetapi, jika kita tidak memenuhi persyaratan-persyaratan yang akan memungkinkan kita untuk melihat Allah, kita merasa risih sampai menemukan alasan-alasan untuk mengurangi kesaksian Injil kepada sesuatu yang tampak masuk akal; yang berarti bahwa aJasan tersebut tidak akan mempertanyakan pendirian kita dalam hidup. Itulah sebabnya banyak orang, meskipun mereka mengagumi Injil dan menolak untuk menganggapnya sebagai suatu kebohongan, mencari ribuan aJasan unruk menyangkal apa yang tampaknya mengguncangkan mereka; kesaksian lnjil tentang Allah menjadi manusia; seorang Allah yang bergaul di antara manusia dan yang membangkitkan orang dari mati.

\section*{Beberapa Penolakan}
Oleh karena itu, mereka secara khusus berpegang teguh pada dua argumen pokok:
\begin{enumerate}
\item Mereka mengatakan bahwa Injil ditulis bertahun-tahun setelah kematian Yesus di mana telah terbentuk sebuah gambaran yang suci tentang Yesus. Dengan demikian, mereka tidak menyatakan realitas Yesus yang sesungguhnya kepada kita, melainkan iman Gereja ~ada abad pertama. (Bandingkan dengan apa yang telah kita katakan mengenai waktu penulisan Injil.
\item Mereka juga mengatakan bahwa Injil adalah tulisan-tulisan yang dimaksudkan sebagai katekismus dan pengajaran untuk orang-orang Kristen: fakta-fakta yang bertujuan untuk mendukung apa yang mereka ajarkan. Karena itu, tidaklah penting apakah Yesus berjalan di atas air atau tidak; episode Itu ditulis untuk menunjukkan bahwa Yesus memiliki kekuasaan Ilahi.
\end{enumerate}

\section*{Tetapi, bagaimana dengan para rasul?}
Mereka telah menjadi saksi-saksi Yesus,dan peranan mereka adalah untuk tetap menjadi saksi-saksi resmi-Nya di dalam Gereja. Mereka tahu persis apa sesungguhnya yang telah terjadi; apakah mereka harus tetap diam semen tara orang-orang membelokkan sejarah Yesus? Jaminan Injil ditemukan dalam struktur hakiki Gereja Katolik, yang tidak pemah merupakan suatu kelompok umat beriman yang secara spontan dikendalikan oleh antusiasme atau oportunisme.

Injil berasal dari tradisi para rasul dan Gereja tetap memelihara tradisi karena Gereja mengakuinya di dalam Injil. Dalam tahun-tahun itu dan dalam abad benkutnya, Injil-injil yang lain ditulis: Injil Petrus, Injil Thomas, Injil Nikodemus, Injll Pertama Yakobus. Akan tetapi, Gereja tidak menerima Injil-injil ini karena peristiwa-peristiwa fantastis yang tertulis di dalamnya, atau karena orientasi teologis yang tidak sesuai dengan ajaran yang diterima dari para rasul. 