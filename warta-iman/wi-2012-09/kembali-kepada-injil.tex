\chap{Kembali kepada Injil}

Kembali "kepada" Injil. Sepanjang sejarahnya Gereja telah menjalani pembaharuan-pembaharuan dan mempunyai pembaharu-pembaharu. Dan pembaharuan selalu merupakan suatu gerak kembali kepada Injii.

Tetapi dewasa ini, kembali- kepada lnjil sudah mendapat arti khusus. Selama berabad-abad (apa yang disebut Paulus dan Lukas sebagai "zaman bangsa-bangsa"), Gereja menjadi pendidik masyarakat di mana ia berada. Bersama mereka ia mengalami pengalaman-pengalaman umat Israel. Ia mewartakan Injil, yaitu misteri persatuan Allah, kasih, dan misteri salib, dan ia mengambil bagian dalam perjuangan bangsa-bangsa untuk menjadi dewasa, lewat banyak penderitaan, perjuangan dan ketidak-tahuan.

Dewasa ini kita mulai menengok kembali jalan yang telah ditempuh. Injil diperuntukkan bagi semua orang, dan itulah pemyataan manusia-dengan-Allah. Namun kemajuan sekarang berarti mengambil langkah-langkah besar menuju perjumpaan semua kebudayaan dan semua kenyataan manusiawi. Sudah saatnya bagi Gereja masuk sepenuhnya ke dalam Perjanjian Baru. Mulai sekarang Gereja tidak lagi menjadi pengajar bangsa-bangsa, orang-orang Kristianiakan menjadi ragi dalam adonan. Struktur-struktur Gereja yang besar yang cenderung makin besar akan menjadi kurang penting. Usaha meneari Allah lewat sabdaNya akan terlaksana pertama-tama dalam kehidupah religius orang beriman. .

Maka lebih tepat bagi kita berbicara tentang "kembalinya Injil." Dewasa ini Injil nampak sebagai kunci sejarah kita. Selama berabad-abad orang-orang Kristen telah melihat dalam iman dan agama sarana untuk menyelamatkan jiwa-jiwa dan melayani Allah, tetapi mereka memiliki kunci untuk memahami sejarah aktual dalam realitasnya sehari-hari yang keras. Sekarang inilah saatnya kata-kata injil mulai mempunyai makna dalam konteks masalah-masalah global: tidaklah kebetulan bahwa di negara-negara kawasan Timur jutaan orang mencari dalam Injil rahasia keunggulan yang rupanya dimiliki bangsa-bangsa barat.

Betapapun dosa dan ketidak-tahuan orang-orang barat, Kristus yang bangkit telah berkarya di tengah-tengah manusia lewat mereka. Injil tidak hanya kata-kata (dan bukan suatu agama),tetapi lebih merupakan suatu pembukaan, suatu keadaan berahmat dalam diri manusia yang menyadari bahwa ia berhadapan dengan Allah lewat salib Kristus. Tinggal sedikit saja benteng-benteng pertahanan dalam bangsa manusia yang bertahan terhadap kekuatan-kekutan baru. Kecemasan-kecemasan budaya, meskipun tersesat, hampir selalu membuka pintu-pintu bagi Injil. Musik Mozart telah' memberikan suatu kesadaran Kristen yang baru kepada lebih banyak orang daripada yang dapat dibuat oleh misionaris-misionaris. Emansipasi wanita telah membawa bangsa-bangsa dan jutaan orang kepada pertobatan sejati.

Sudah saatnya bagi seiiap orang Kristen, setiap komunitas Kristen menyadari diri dituntun kembali kepada zaman Yesus dan rasul-rasul, Terbebas dari struktur-strukrur religius yang mendukung dan pada saat yang sama memenjarakan leluhur-leluhur kita, sudah saatnya bagi kita untuk mewartakan Kabar Baik kepada dunia. Kita telah melihat bahwa banyak harapan yang dijanjikan oleh ilmu, komunisme dan intelek yang menganggap diri sebagai penguasa tunggal temyata tak dapat terpenuhi. Bangsa manusia, yang sekarang menguasai banyak unsur dari masa depannya, mulai menghadapi pertanyaan-pertanyaan dasariah yang besar: hidup, tetapi mengapa? Kita telah memasuki abad evangelisasi yang agung: Tuhan datang! 