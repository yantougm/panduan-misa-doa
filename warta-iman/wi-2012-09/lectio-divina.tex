\chap{Lectio Divina}
Walaupun pendekatan pengetahuan akan Tradisi Suci adalah sangat penting, namun pendekatan penghayatan Sabda dalam relasi pribadi dengan Tuhan juga tak kalah penting. Oleh karena itu, Bapa Gereja dan konsili-konsili Gereja mengajarkan kita untuk senantiasa berdoa dalam membaca Kitab Suci. Lebih lanjut, \textit{Verbum Domini} menekankan untuk membaca Kitab Suci dalam dialog dengan Tuhan. St. Agustinus mengatakan "\textit{Your prayer is the word you speak to God. When you read the Bible, God speaks to you; when you pray, you speak to God}". Dengan demikian, orang yang membawa Sabda Allah dan merenungkannya di dalam doa, maka menjadi suatu percakapan yang begitu intim dan personal antara seseorang dengan Tuhan, yang pada akhirnya akan membaca seseorang pada kesempurnaan kehidupan Kristiani. Namun, dokumen yang sama mengingatkan agar kita tidak boleh terjebak pada interpretasi pribadi, namun harus teks-teks di dalam Kitab Suci harus senantiasa dimengerti dalam kesatuan dengan Gereja.

\section{Pengertian}
Tradisi Gereja Katolik mengenal apa yang disebut sebagai "\textit{lectio divina}" untuk membantu kita umat beriman untuk sampai kepada persahabatan yang mendalam dengan Tuhan. Caranya ialah dengan mendengarkan Tuhan berbicara kepada kita melalui sabda-Nya. "\textit{Lectio}" sendiri adalah kata Latin yang artinya "bacaan". Maka "\textit{lectio divina}" berarti bacaan ilahi atau bacaan rohani. Bacaan ilahi/ rohani ini terutama diperoleh dari Kitab Suci. Maka, \textit{lectio divina} adalah cara berdoa dengan membaca dan merenungkan Kitab Suci untuk mencapai persatuan dengan Tuhan Allah Tritunggal. Di samping itu, dengan berdoa sambil merenungkan Sabda-Nya, kita dapat semakin memahami dan meresapkan Sabda Tuhan dan misteri kasih Allah yang dinyatakan melalui Kristus Putera-Nya. Melalui \textit{lectio divina}, kita diajak untuk membaca, merenungkan, mendengarkan, dan akhirnya berdoa ataupun menyanyikan pujian yang berdasarkan sabda Tuhan, di dalam hati kita. 

Penghayatan sabda Tuhan ini akan membawa kita kepada kesadaran akan kehadiran Allah yang membimbing kita dalam segala kegiatan kita sepanjang hari. Jika kita rajin dan tekun melaksanakannya, kita akan mengalami eratnya persahabatan kita dengan Allah. Suatu pengalaman yang begitu indah tak terlukiskan!

\section{Empat hal dalam proses \textit{lectio divina}}
Meskipun terjemahan bebas dari kata \textit{lectio} adalah bacaan, proses yang terjadi dalam \textit{lectio divina} bukan hanya sekedar membaca. Proses \textit{lectio divina} ini menyangkut empat hal, yaitu: \textit{lectio}, \textit{meditatio}, \textit{oratio} dan \textit{contemplatio}.
compassio
\begin{enumerate}[label=\alph*.]
\item \textit{Lectio}
\\
Membaca di sini bukan sekedar membaca tulisan, melainkan juga membuka keseluruhan diri kita terhadap Sabda yang menyelamatkan. Kita membiarkan Kristus, Sang Sabda, untuk berbicara kepada kita, dan menguatkan kita, sebab maksud kita membaca bukan sekedar untuk pengetahuan tetapi untuk perubahan dan perbaikan diri kita. Maka saat kita sudah menentukan bacaan yang akan kita renungkan (misalnya bacaan Injil hari itu, atau bacaan dari Ibadat Harian), kita dapat membacanya dengan kesadaran bahwa ayat-ayat tersebut sungguh ditujukan oleh Tuhan kepada kita.
\item \textit{Meditatio} \\
Meditatio adalah pengulangan dari kata-kata ataupun frasa dari perikop yang kita baca, yang menarik perhatian kita. Ini bukan pelatihan pemikiran intelektual di mana kita menelaah teksnya, tetapi kita menyerahkan diri kita kepada pimpinan Allah, pada saat kita mengulangi dan merenungkan kata-kata atau frasa tersebut di dalam hati. Dengan pengulangan tersebut, Sabda itu akan menembus batin kita sampai kita dapat menjadi satu dengan teks itu. Kita mengingatnya sebagai sapaan Allah kepada kita.
\item \textit{Oratio}\\
Doa adalah tanggapan hati kita terhadap sapaan Tuhan. Setelah dipenuhi oleh Sabda yang menyelamatkan, maka kita memberi tanggapan. Maka seperti kata St. Cyprian, "Melalui Kitab Suci, Tuhan berbicara kepada kita, dan melalui doa kita berbicara kepada Tuhan." Maka dalam \textit{lectio divina} ini, kita mengalami komunikasi dua arah, sebab kita berdoa dengan merenungkan Sabda-Nya, dan kemudian kita menanggapinya, baik dengan ungkapan syukur, jika kita menemukan pertolongan dan peneguhan; pertobatan, jika kita menemukan teguran; ataupun pujian kepada Tuhan, jika kita menemukan pernyataan kebaikan dan kebesaran-Nya.
\item \textit{Contemplatio}\\
Saat kita dengan setia melakukan tahapan-tahapan ini, akan ada saatnya kita mengalami kedekatan dengan Allah, di mana kita berada dalam hadirat Allah yang memang selalu hadir dalam hidup kita. Kesadaran kontemplatif akan kehadiran Allah yang tak terputus ini adalah sebuah karunia dari Tuhan. Ini bukan hasil dari usaha kita ataupun penghargaan atas usaha kita. St. Teresa menggambarkan keadaan ini sebagai  doa persatuan dengan Allah/ prayer of union di mana kita "memberikan diri kita secara total kepada Allah, menyerahkan sepenuhnya kehendak kita kepada kehendak-Nya."
\end{enumerate}

Ke-empat fase ini membuat kelengkapan \textit{lectio divina}. Jika lectio diumpamakan sebagai fase perkenalan, maka meditatio adalah pertemanan, oratio persahabatan dan contemplatio sebagai persatuan.

\section{Bagaimana caranya memulai \textit{lectio divina}}
Karena maksud dari \textit{lectio divina} adalah untuk menerapkan Sabda Allah dalam kehidupan kita, dan dengan demikian hidup kita diubah dan dipimpin olehnya, maka langkah-langkah \textit{lectio divina} adalah sebagai berikut:
\begin{enumerate}
\item    Ambillah sikap doa, bawalah diri kita dalam hadirat Allah. Resapkanlah kehadiran Tuhan di dalam hati kita. Mohonlah agar Tuhan sendiri memimpin dan mengubah hidup kita melalui bacaan Kitab Suci hari itu.
\item     Mohonlah kepada Roh Kudus untuk membantu kita memahami perikop itu dengan pengertian yang benar.
\item Bacalah perikop Kitab Suci tersebut secara perlahan dan dengan seksama, jika mungkin ulangi lagi sampai beberapa kali.
\item Renungkan untuk beberapa menit, akan satu kata atau ayat atau hal-hal yang disampaikan dalam perikop tersebut dan tanyakanlah kepada diri kita sendiri, "Apakah yang diajarkan oleh Allah melalui perikop ini kepadaku?"
\item Tutuplah doa dengan satu atau lebih resolusi/keputusan praktis yang akan kita lakukan, dengan menerapkan pokok-pokok ajaran yang disampaikan dalam perikop tersebut di dalam hidup dan keadaan kita sekarang ini.
\item Resapkanlah kehadiran Tuhan di sepanjang aktivitas kita sehari itu. Dengan kesadaran ini kita dapat selalu mengarahkan dan mempersembahkan segala sesuatu yang kita lakukan hari itu demi kemuliaan nama-Nya.
\end{enumerate}

\section{Contoh merenungkan Kitab Suci dengan \textit{lectio divina}}
Mari kita melihat bacaan Injil dari Mat 18:21-19:2. Di dalam perikop tersebut diceritakan perumpamaan tentang pengampunan. Pada saat kita merenungkan perikop ini, maka kita dapat bertanya pada diri sendiri, apakah yang Tuhan inginkan agar kita terapkan dalam kehidupan kita sehari hari?

Maka kita bisa membayangkan salah satu tokoh dalam perikop itu, misalnya, kita menjadi hamba itu yang berhutang sepuluh ribu talenta. Namun oleh belas kasihan raja --yaitu Tuhan--, maka hutang hamba itu dihapuskan. Namun kemudian kita berjumpa dengan orang yang telah menyakiti hati kita, dan kita merasa sulit untuk mengampuni. Dengan demikian, kita bersikap seperti hamba itu, yang walaupun sudah diampuni dan dihapuskan hutangnya, namun tidak dapat/ sukar mengampuni orang lain. Mari dengan jujur melihat, apakah kita pun pernah atau sering bersikap seperti hamba yang tidak berbelas kasihan ini?  Siapakah kiranya orang yang Tuhan inginkan agar kita ampuni? Tanyakanlah kepada Tuhan dalam hati, "Tuhan, tunjukkanlah kepadaku, adakah aku pernah bersikap demikian? Siapakah yang harus kuampuni \ldots ?

Sambil terus merenungkan ayat demi ayat dalam perikop tersebut, bercakap-cakaplah dengan Tuhan dalam keheningan batin. Mungkin Tuhan ingin mengingatkan kita akan ayat ini, "Bukankah engkaupun harus mengasihani kawanmu seperti Aku telah mengasihani engkau?" (Mat 18:33). Jika ayat itu yang sungguh berbicara pada kita hari ini, maka kita mengingatnya dan mengulanginya kembali dalam hati, sebagai perkataan Tuhan yang ditujukan kepada kita. Dan semakin kita merenungkannya, semakin hiduplah perkataan itu di batin kita, dan bahkan kita dapat mendapat dorongan untuk menerapkannya.

Atau jika pada saat ini kita masih terluka atas perlakuan seseorang kepada kita, maka, kitapun dapat membawanya ke hadapan Kristus. Kita dapat pula menyatakan kepada-Nya, betapa kita ingin mengampuni, namun rasa sakit masih begitu mendalam dan nyata dalam hati kita. Maka, mungkin ayat yang berbicara adalah beberapa ayat sesudahnya yang berkata, "Orang banyak berbondong-bondong mengikuti Dia dan Iapun menyembuhkan mereka di sana." (Mat 19:2). Kita dapat membayangkan bahwa kita berada di antara orang yang berbondong-bondong itu, dan memohon agar Ia menyembuhkan luka-luka batin kita. Biarlah ayat Mat 19:2 meresap dalam hati kita, dan kita ulangi berkali-kali sepanjang hari, "\dots dan Tuhan Yesus-pun menyembuhkan luka-luka batinku di sana." Biarkan jamahan Tuhan yang menyembuhkan banyak orang pada 2000 tahun yang lalu menyembuhkan kita juga pada saat ini. Dengan kita mengalami kesembuhan batin, maka sedikit demi sedikit Tuhan membantu kita untuk mengampuni, sebab kekuatan kasih-Nya memampukan kita melakukan sesuatu yang di luar batas kemampuan kita sebagai manusia.

Memang, pada akhirnya, \textit{lectio divina} ini tidak akan banyak berguna jika kita berhenti pada meditatio/ permenungan, tapi tanpa langkah selanjutnya. Kita harus menanggapi apa yang Tuhan sampaikan lewat sabda-Nya, dan membuat keputusan tentang apakah yang akan kita lakukan selanjutnya, setelah menerima pengajaran-Nya. Maka langkah berikut, kita dapat mengadakan percakapan/ oratio yang akrab dengan Tuhan Yesus, entah berupa ucapan syukur, pertobatan, atau permohonan, yang semua dilakukan atas dasar kesadaran kita akan besarnya kasih Tuhan kepada kita. 

Kesadaran akan kasih Kristus inilah yang sedikit demi sedikit mengubah kita, dan mendorong kita untuk memperbaiki diri, supaya dapat mengikuti teladan-Nya, untuk mengasihi orang-orang di sekitar kita, terutama anggota keluarga kita sendiri: suami, istri, orang tua, dan anak-anak. Kasih-Nya ini pula yang membangkitkan di dalam hati kita rasa syukur, atas pengampunan dan pertolongan-Nya pada kita. Dengan memandang kepada Yesus, kita dapat melihat dengan jujur ke dalam diri kita sendiri, untuk menemukan hal-hal yang masih harus kita perbaiki, agar kita dapat hidup sesuai dengan panggilan kita sebagai murid- murid-Nya.

Jika melalui lectio divina akhirnya kita mampu mengalahkan kehendak diri sendiri untuk mengikuti kehendak Allah, maka  kita perlu sungguh bersyukur. Sebab sesungguhnya, ini adalah karya Roh Kudus yang nyata dalam hidup kita. Perubahan hati, atau pertobatan terus menerus yang menghantar kita lebih dekat kepada Tuhan dengan sendirinya mempersiapkan kita untuk bersatu dengan-Nya dalam contemplatio. Dalam contemplatio ini, hanya ada Allah saja di dalam hati dan pikiran kita. Kerajaan-Nya memenuhi hati kita, sehingga kehendak-Nya sepenuhnya menjadi kehendak kita. "Jadilah padaku ya Tuhan, menurut kehendak-Mu…." Dan dalam keheningan dan kedalaman batin kita masuk dalam persatuan dengan Dia.

Jika arti doa yang sesungguhnya adalah "turun dengan pikiran kita menuju ke dalam hati, dan di sana kita berdiri di hadapan wajah Tuhan, yang selalu hadir, selalu memandang kita, di dalam diri kita,". Pandangan kepada Yesus ini adalah suatu bentuk penyangkalan diri, di mana kita tidak lagi menghendaki sesuatu yang lain daripada kehendak Allah. Dengan pandangan ini kita mempercayakan seluruh diri kita ke dalam tangan-Nya, dan kita semakin terdorong untuk mengasihi dan mengikuti Dia yang terlebih dahulu mengasihi kita.

\section{Apa buah-buah dari \textit{lectio divina}?}
Buah-buah dari \textit{lectio divina} adalah \textit{compassio} dan \textit{operatio}. Dengan persatuan kita dengan Tuhan, maka kita membuka diri juga untuk lebih memperhatikan dan mengasihi sesama dan ciptaan Tuhan yang lain. Kita juga didorong untuk melakukan tindakan nyata untuk membantu sesama yang membutuhkan pertolongan, ataupun untuk selalu mengusahakan perdamaian dengan semua orang. Dengan demikian perbuatan kita menjadi kesatuan dengan doa kita, atau dengan perkataan lain kita memiliki perpaduan sikap Maria dan Martha (lih. Luk 10:38-42).

\section{Mari, memulai perjalanan iman dengan \textit{lectio divina}}

Jika kita membaca pengalaman para orang kudus, kita mengetahui bahwa banyak dari mereka menerapkan \textit{lectio divina} dalam kehidupan rohani mereka. Diakui bahwa perjalanan menuju contemplatio bukan sesuatu yang mudah, karena memerlukan disiplin dan kesetiaan kita untuk menyediakan waktu untuk berdoa. Namun demikian, sesungguhnya setiap orang dapat mulai menerapkan \textit{lectio divina} ini dalam kehidupan sehari-hari.
Banyak orang keliru jika berpikir bahwa membaca dan merenungkan Kitab Suci secara pribadi hanya dapat dilakukan oleh orang-orang tertentu yang tingkat pendidikan yang tinggi tentang Kitab Suci. Kenyataannya, sebagian besar perikop Kitab Suci tidak sulit di-interpretasikan. Bahkan perikop yang mengandung ayat yang sulit sekalipun, akan tetap berguna untuk direnungkan. Maka sesungguhnya, tidak ada alasan bagi kita untuk malas membaca dan merenungkan Kitab Suci. Kita dapat menggunakan ayat-ayat Kitab Suci untuk berdoa dan untuk menjadi penuntun sikap kita sehari-hari. Membaca atau menghafalkan ayat- ayat Kitab Suci adalah sesuatu yang baik, tetapi alangkah lebih baik jika kita meresapkannya dan membiarkan hidup kita terus menerus diubah olehnya. Tentu, ke arah yang lebih baik, agar kita semakin dapat mengikuti teladan Kristus Tuhan kita.

\section{Kesimpulan}
Permenungan akan makna Sabda Tuhan dapat dilakukan dengan cara \textit{lectio divina}, yang terdiri dari empat tahapan yaitu membaca Kitab Suci (\textit{lectio}), merenungkannya (\textit{meditatio}), berdoa (\textit{oratio}), menghayati persatuan dengan Allah (\textit{contemplatio}). Selanjutnya, buah yang ditunjukkannya adalah belas kasihan dan perbuatan baik (\textit{compassio dan operatio}) yang membawa kepada pertumbuhan hidup rohani menuju kekudusan dalam persahabatan yang erat dengan Allah dan sesama.

"O, Tuhan Yesus, tambahkanlah di dalam hatiku, kasih kepada-Mu; sehingga aku dapat setia menginginkan persahabatan dengan Engkau melalui doa dan Sabda."