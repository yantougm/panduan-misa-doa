\section*{\Large \begin{center} Ssssttt.. Ada Gosip!!\end{center}}

\vspace{-0.45cm}

\begin{center}
\textit{Oleh Novka Kuaranita}
\end{center}

\vspace{0.25cm}

\small
\textit{``Mas, mana ikan bakar saya?! Lama banget nggak dateng-dateng!''
Seorang pelayan tergopoh-gopoh mendatangi meja ibu yang meneriakinya tadi, lalu berkata, “Sebentar, Bu.. Sedang dibuatkan. Sebentar lagi datang..''
“Ah, kelamaan. Saya ganti aja. Biar cepet. Saya nggak punya banyak waktu.'' Ibu itu kemudian meneliti daftar menu. Katanya kemudian, ``Ini aja, Mas. Soto ayam kampung. Nggak pake lama, ya!''
Pelayan itu mencatat lalu melangkah mundur dengan wajah kesal namun tak berdaya. Beberapa menit kemudian, ia datang sambil membawa semangkok soto. Ia membungkuk dan menaruh soto itu di meja ibu yang mengganti pesanannya tadi.
Si ibu mengaduk-aduk isi mangkok itu, lalu berkomentar, ``Dikit amat ayamnya. Saya tadi pesen soto ayam, ya… bukan soto kubis.''
}

Dari meja lain, kami menonton ``drama dadakan'' tersebut sambil menunggu adegan selanjutnya.

\dots

Teriakan awal ibu yang memanggil pelayan rumah makan tersebut membuat kami menyetel kuping dengan pendengaran supertajam, berusaha merekam apa yang terjadi. Konflik selalu menarik perhatian orang, tak terkecuali kami. Pertengkaran selalu memancing rasa ingin tahu, dan kami pun terpancing.

Kami merasa mendapat topik baru yang seru untuk dibicarakan. Otak dan mulut kami pun tak mau diam, mereka urun berkomentar. Misalnya, ‘kok bisa ibu itu marah-marah di tempat umum kayak gini.. nggak punya urat malu apa?’ atau ‘ya ampun, kasian banget ya anak-anaknya harus liat ibunya marah-marah kayak gitu… untung nggak jadi anaknya’ atau ‘kok bisa suaminya mau sama perempuan yang lebih pedes daripada cabe rawit’ atau ‘ngidam apa dulu orang tuanya, bisa-bisanya punya anak kayak gitu.’ Sementara asyik mengunyah makanan, mulut kami pun larut dalam keasyikan yang lain: ngomongin ibu yang duduk di meja seberang.

Sesampainya di rumah, saya mengingat kembali peristiwa di rumah makan tadi siang. Ada beberapa tanda tanya yang menggantung di kepala saya. Kali ini tanda tanya-tanda tanya itu tidak mengarah kepada ibu tersebut, tapi menyerbu diri saya sendiri. 
‘Kenapa saya bisa begitu asyiknya ngomongin orang seperti anak kecil yang mendapat mainan baru?’ 
‘Apa untungnya bagi saya ngrasani orang di belakangnya kayak gitu?’
‘Apakah dengan ngomongin orang lain saya jadi merasa begitu hebat?’
dan, pertanyaan terakhir yang paling menohok:
‘Apa iya saya lebih baik daripada ibu yang saya cela habis-habisan dalam pikiran saya tadi siang?’

Lalu saya merasa malu. Memang benar, untuk ukuran umum, yang dilakukan ibu tersebut—marah-marah di tempat makan—memang tidak sopan. Setidaknya, ada cara yang lain untuk menyatakan ketidakpuasan. Namun, yang lebih mengganjal hati saya adalah pertanyaan yang lebih besar: apa iya saya sudah lebih baik daripada ibu itu? Jangan-jangan perbuatan saya pun tidak berkenan dan mengganggu orang lain… Kalau begitu, apa hak saya mengadili orang lain dalam pikiran saya?

Ngrasani orang lain, menurut saya, bisa jadi positif dan negatif.
Positif ketika itu jadi kritik yang membangun, entah bagi diri sendiri ataupun bagi orang lain. 
Bagi diri sendiri, ngrasani orang berguna ketika kita memutuskan untuk mengambil pelajaran dengan tidak meniru apa yang menurut kita tidak baik. Tentang ibu-ibu di rumah makan tadi, misalnya, jadi positif ketika saya memutuskan saya tidak akan mempermalukan diri saya sendiri dan orang lain dengan berteriak-teriak di tempat umum atau ketika saya belajar cara lain yang lebih menyenangkan untuk menyampaikan ketidaknyamanan saya.

Kalau kita ingin unek-unek itu bermanfaat bagi orang lain, ya suarakanlah kepada orang yang bersangkutan secara terbuka dan kalau bisa dengan cara yang tidak menyinggungnya,
bukan dengan bersuara di belakang. Yang biasanya terjadi adalah unek-unek itu disampaikan bukan dengan orang yang bersangkutan. Unek-unek tentang seseorang disimpan untuk diceritakan lagi kepada orang lain. 

Dengan adanya ``stok gosip'' di kepala kita, kita merasa punya bahan obrolan yang menyenangkan dengan orang lain. Memang asyik sekali, kalau ketemu orang, kita bisa dengan senyum-senyum kecil dan suara bisik-bisik berkata, ``Ssssttt.. tau nggak sih.. si itu ininya gini, lo….'' Lalu cerita pun bergulir lebih gurih dengan bumbu di sana-sini. Nah, kalau suara-suara ini sudah menggaung di belakang, efek ngrasaninya jadi  negatif. Suara-suara di belakang tidak akan mengubah apa pun kecuali membuat gap baru antara orang yang ngrasani dan dirasani. Bukannya membuat keadaan menjadi lebih baik, hal itu justru akan menambah sebuah masalah baru.

Maka, sekarang, saya berharap otak dan mulut saya lebih hati-hati dalam berkomentar, apalagi jika itu menyangkut orang lain. Semoga, ketika otak dan mulut saya mulai larut dalam keasyikannya menggosip, hati saya selalu mengingatkan saya dengan pertanyaan: Apa iya kamu sudah lebih baik daripada dia yang kamu rasani?


\begin{flushright}\textit{Tangerang, 25 September 2011} \end{flushright}
\normalsize



