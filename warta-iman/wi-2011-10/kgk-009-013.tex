\newpage
\section*{Kompendium Katekese Gereja Katolik}
\setcounter{kgkcounter}{8}
{\small

\kgk{Bagaimana wahyu Allah yang penuh dan definitif itu terlaksana?}
Tahap wahyu Allah yang penuh dan definitif terlaksana dalam Sabda-Nya
yang menjadi daging, Yesus Kristus, pengantara dan kepenuhan wahyu. Sebagai
Putra Tunggal Allah yang menjadi manusia, Dialah Sabda Bapa yang sempurna
dan definitif. Dalam pengutusan Sang Putra dan pemberian Roh Kudus, sekarang
wahyu Allah menjadi lengkap secara penuh, namun iman Gereja harus sedikit
demi sedikit memahami maknanya yang lengkap selama berabad-abad.

\kgk{Apa nilai wahyu-wahyu pribadi?}
            Walaupun tidak termasuk dalam khazanah iman, wahyu-wahyu pribadi
         dapat membantu manusia untuk menghidupi imannya sejauh membawa kita
         kepada Kristus. Kuasa Mengajar Gereja yang mempunyai tugas untuk menilai
         wahyu-wahyu pribadi semacam itu tidak dapat menerima mereka yang meng-
         klaim bahwa wahyu pribadi itu melebihi atau mengoreksi wahyu definitif, yaitu
         Kristus.

                  \begin{center}\textbf{PEWARISAN WAHYU ILAHI}\end{center}

\kgk{Mengapa dan dengan cara bagaimana wahyu ilahi itu diwariskan?}
            Allah menghendaki agar manusia diselamatkan dan sampai pada pengetahuan
         akan kebenaran (1Tim 2:4), yaitu Yesus Kristus. Karena alasan inilah, Kristus harus
         diwartakan kepada semua menurut perintah-Nya, ``Pergilah dan ajarlah segala
         bangsa'' (Mat 28:19). Dan, ini diwariskan oleh Tradisi Apostolik.

\kgk{Apa Tradisi Apostolik itu?}
        Tradisi Apostolik adalah pewarisan pesan Kristus, yang diturunkan sejak
      awal Kekristenan melalui khotbah, kesaksian, institusi, ibadah, dan tulisan-tulisan
         yang diilhami. Para Rasul mewariskan apa yang sudah mereka terima dari Kristus
         dan belajar dari Roh Kudus kemudian terus berlanjut kepada pengganti-pengganti

\kgk{Bagaimana terjadinya Tradisi Apostolik?}
            Tradisi Apostolik terjadi dalam dua cara: melalui pewarisan langsung Sabda
         Allah (yang disebut sebagai Tradisi) dan melalui Kitab Suci yang merupakan
         pewartaan keselamatan yang sama dalam bentuk tulisan.
         mereka, para Uskup, dan melalui mereka kepada semua generasi sampai akhir dunia.


\flushright{(\dots \emph{bersambung} \dots)}
}