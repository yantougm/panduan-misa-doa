\chapter*{\begin{center}Bagaimana Kita Menghormati Bunda Maria\end{center}}			
\begin{center}\textit{Tim Carmelia}\end{center}	   

\subsection*{PENGANTAR}
Sebagai orang Katolik, kita harus mengenal bagaimana peranaan Bunda Maria dalam Gereja, karena Maria adalah Bunda Gereja. Kita tidak dapat melihat kedudukan Bunda Maria dengan perasaan kita, tetapi kita harus mengacu kepada tafsiran Gereja dan tafsiran Kitab Suci. Orang Katolik menghormati Bunda Maria, fakta ini menimbulkan pertanyaan bagi sebagian orang, batu sandungan, kadang-kadang menjadi bahan tuduhan dari saudara-saudari kita yang berkepercayaan lain. Dari kritikan ini mengakibatkan begitu banyak orang Katolik yang tidak tertarik lagi untuk menghormati Bunda Maria bahkan meninggalkan Gereja Katolik.
Tuduhan-tuduhan yang sering kita dengar antara lain: orang Katolik menghormati Bunda Maria secara berlebihan, atau orang Katolik menyembah Maria, atau orang Katolik menyembah patung. Menghadapi pertanyaan- pertanyaan seperti ini, kita tidak bisa hanya membiarkannya berlalu begitu saja atau melarikan diri, tetapi kita harus berusaha menjawabnya sambil merenung apakah penghormatan kita kepada Bunda Maria sudah benar atau salah.

\subsection*{SYARAT-SYARAT PENGHORMATAN KEPADA BUNDA MARIA}
Dalam penghormatan kepada Bunda Maria, kita berpedoman pada empat sifat, yaitu:
\begin{enumerate}
\item Penghormatan kepada Bunda Maria harus berdasarkan Kitab Suci.

Dalam Kitab Suci memang tidak ada satu ayat pun yang menyuruh kita untuk menghormati Bunda Maria. Akan tetapi dalam Injil Lukas 1:26-38 dan ayat paralelnya, kita menemukan dasar mengapa kita menghormati Bunda Maria. Dasar-dasar dalam menghormati Bunda Maria antara lain:
\begin{itemize}
\item Bunda Maria terlibat aktif dalam karya penebusan

Seperti dalam dalam Injil Lukas, ketika Malaikat Gabriel menyampaikan pesan Allah kepada Bunda Maria bahwa ia akan melahirkan seorang laki-laki. ”Sesungguhnya engkau akan mengandung dan melahirkan seorang anak laki-laki dan hendaklah engkau menamakan Dia Yesus. Ia akan menjadi besar dan akan disebut Anak Allah yang Mahatinggi. Dan Tuhan Allah akan mengaruniakan kepada-Nya takhta Daud, bapa leluhurnya” (Luk 1:31-32). Bunda Maria dengan iman yang penuh, pasrah kepada Allah dan hanya menjawab: “Sesungguhnya aku ini hamba Tuhan terjadilah padaku menurut perkataanmu” (Luk 1:38). Di sinilah Bunda Maria menerima tugas untuk mengambil bagian dalam karya keselamatan. Seluruh hidup Bunda Maria diabdikan kepada karya penebusan. Apa yang dibuat oleh Bunda Maria? Bunda Maria mengandung dan kemudian melahirkan Sang Penebus, dan sebagai akibatnya ia harus mengungsi ke Mesir. Kemudian Bunda Maria harus membesarkan anaknya itu yaitu Tuhan Yesus, dengan segala kebutuhan-Nya. Dengan demikian Bunda Maria terlibat penuh dalam karya penebusan.
\item Bunda Maria merupakan seorang kudus yang besar

Dalam Gereja katolik kita menghormati orang-orang kudus karena mereka merupakan karya tangan Tuhan yang penuh dan kaya akan rahmat. Bunda Maria adalah yang terkudus dari para kudus Bahkan sejak dalam kandungan Santa Anna, ia sudah dipersiapkan oleh Allah sebagai ibu penyelamat. Maka pantaslah kita menghormatinya sebagai karya tangan Allah yang istimewa. Dikatakan oleh malaikat: “Salam hai engkau yang dikarunia, Tuhan menyertai engkau.” Suci artinya dikaruniai oleh Tuhan. Orang dijadikan suci bukan karena karya manusia, doa manusia, kepandaian, tetapi pertama-tama oleh karena karunia Tuhan. Dari dalam diri manusia dibutuhkan suatu jawaban yang serius akan rahmat Tuhan yang istimewa ini, karena rahmat membutuhkan kerja sama dengan usaha manusia, tetapi yang menjadi penggerak utama adalah rahmat Tuhan. Ketika malaikat mengatakan “salam hai engkau yang dikaruniai Tuhan”, di sinilah malaikat mengakui bahwa Maria adalah orang yang kudus.
\end{itemize}


\item Penghormatan kepada Bunda Maria harus sesuai dengan liturgi/ibadat.

Pusat ibadat/liturgi dalam Gereja Katolik hanyalah satu, yaitu Allah sendiri melalui Putra-Nya, Yesus Kristus. Segala penghormatan kita harus sampai kepada Allah, menghormati Bunda Maria tidak hanya sampai pada Maria itu sendiri, agar tidak mengambil arti atau mengambil kesimpulan singkat bahwa kita menjadikan Bunda Maria sebagai Allah. Kita harus menghormati Bunda Maria supaya ia menghantar kita kepada Allah, permohonan kita dapat sampai kepada Allah. Hanya Allah saja yang Mahakuasa, di luar Allah tidak ada yang mahakuasa. Puncak dari ibadat/liturgi dalam Gereja Katolik adalah Ekaristi. Dalam ajaran Gereja Katolik, tidak pernah Ekaristi diperalatkan atau diganti demi dan untuk menghormati Maria. Misalnya: jangan sampai kita yang percaya akan keagungan Ekaristi, kita mengikuti misa, tetapi sepanjang misa kita berdoa rosario. Jikalau dalam perayaan Ekaristi, pusatnya hanyalah Tuhan Yesus, jangan sampai kita berkata dan berbangga bahwa sepanjang Ekaristi kita dengan tekun berdoa rosario, hal itu salah; atau sesudah menerima komuni kudus kita menyanyikan lagu-lagu Maria, misalnya menyanyikan lagi Ave Maria. Itu berarti kita tidak menyadari Yesus yang sudah hadir di dalam hati kita dan juga mengurangi nilai Ekaristi. Hal seperti inilah yang membuat kita dianggap menyembah Bunda Maria.

\item Penghormatan kepada Bunda Maria harus Ekuimenis artinya tidak menghambat persatuan umat Katolik dengan Allah.

Sebagai orang Katolik kita harus berusaha supaya penghormatan kepada Bunda Maria membantu atau membawa kita untuk bersatu dengan Allah. Akan tetapi kita tidak boleh meninggalkan Maria demi ekuimeni. Ekuimeni artinya apa? Ada orang mengatakan ekuimeni itu menyesuaikan diri. Lalu siapa yang harus menyesuaikan diri.... orang Katolik! Tidak perlu memakai Maria dalam arti sebagai perantara, misa kudus, pengakuan dosa, ini bukan ekuimeni. Ekuimeni artinya bersama-sama menampilkan diri seperti keyakinan kita masing-masing supaya dengan saling mengenal, kita mencari iman yang sejati untuk sampai kepada Allah, yang harus satu ialah imannya dan tidak boleh melepaskan pokoknya, yaitu Ekaristi. Menghormati Bunda Maria juga merupakan pokok iman sehingga jikalau kita melepaskan Bunda Maria atau tidak menghormati Bunda Maria berarti bukan ekuimeni dan bahkan dapat dikatakan bahwa kita tidak setia kepada Tuhan Yesus. Mengapa...? Karena Ia telah menyerahkan Bunda Maria kepada Yohanes murid-Nya dan kita juga merupakan murid Kristus. Akan tetapi, di sini jangan sampai terkesan bahwa kita menyamakan Bunda Maria dengan Yesus. Maria tetap sebagai ciptaan dan Yesus sebagai pencipta. Jikalau ada orang yang setiap hari berdoa tiga kali rosario, tetapi pada hari Minggu tidak pergi mengikuti perayaan Ekaristi, maka itu penghormatan yang salah dan sesat. Bunda Maria hanya mau dan menghendaki untuk membawa kita kepada Yesus dan Bunda Maria tidak pernah membuat atau mendirikan kerajaannya sendiri di dunia ini.

\item Penghormatan kepada Bunda Maria harus antropologis.

Antropos artinya manusia. Antropologis artinya disesuaikan dengan perkembangan manusia. Ada orang yang menanyakan apakah mungkin penghormatan kepada Bunda Maria masih relevan atau sesuai dengan perkembangan wanita modern sekarang? Bunda Maria itukan wanita yang kolot, tinggal di desa, dan tidak memiliki pengetahuan. Sedangkan wanita modern sekarang tinggalnaya di kantor, di sekolah, dan lain-lain. Di sini bukan dilihat dari itu semua, tetapi kita melihat sifat dan pribadi dari Bunda Maria itu sendiri. Ia adalah seorang hamba Allah yang setia dan sebagai teladan dalam iman kepada Allah. Dalam diri Bunda Maria kita juga menemukan suatu sifat kewanitaan yang sungguh-sungguh setia dan seorang ibu yang penuh perhatian serta sabar dalam menanggung penderitaan. Untuk menjadi contoh wanita modern, Bunda Maria tidak bisa dihilangkan begitu saja. Banyak wanita modern sekarang yang ingin demokrasi dalam keluarganya, tidak mau hanya sebagai pendengar dan pelaksana, tetapi ingin menentukan jalannya keluarga itu, bahkan ikut berperan dalam pertanggungjawaban perkembangan keluarga. Memang ini merupakan sesuatu yang sangat baik. Lalu apakah kebebasan seperti itu terdapat pula dalam diri Bunda Maria? Bunda Maria bahkan menerima tanggung jawab yang sangat besar yang melebihi karya dan tanggung jawab wanita sekarang, yaitu ingin menjadi ibu Sang Mesias yang datang menyelamatkan manusia. Bukankah ini merupakan suatu tugas yang sangat berat yang harus dilaksanakan Bunda Maria. Ketika Bunda Maria dan Santo Yusuf mempersembahkan Yesus di Bait Allah, Simeon bernubuat bahwa sebilah pedang akan menembus jiwanya. Bukankah ini merupakan sesuatu tugas yang berat? Akan tetapi Bunda Maria Menyimpan semuannya itu di dalam hatinya. Bunda Maria merupakan salah satu tokoh atau pribadi yang patut diteladani atau dihormati oleh setiap orang khususnya orang Kristen.
\end{enumerate}

\subsection*{PENUTUP}
Bunda Maria telah mengambil bagian dalam karya keselamatan melalui imannya. Anak Allah sebagai Juru Selamat diterimanya terlebih dahulu dalam hatinya kemudian dalam rahimnya sebagaimana ketika ia menjawab “ya” kepada kabar keselamatan dari Allah melalui Malaikat Gabriel. Ini bukan berarti Allah menghendaki agar pelaksanaan karya keselamatam tergantung pada manusia melainkan bahwa menurut rencana Allah bahwa manusia pada gilirannya berkat rahmat Ilahi akan mengimani keselamatan. Dalam “ya” yang penuh kepercayaan itu, Bunda Maria menerima keselamatan bagi semua umat manusia. Oleh karena itu jangan sampai kita menjadikan Bunda Maria sebagai tukang pos, dalam arti tolong sampaikan permohonan saya kepada Tuhan Yesus. Banyak orang mengormati Bunda Maria dengan membanjirinya melalui doa permohonan dan permintaan yang lainnya. Jikalau doa permohonan kita ingin dikabulkan, kita harus kembali atau melihat dalam Injil Yohanes tentang perkawinan di Kana. Bunda Maria mengatakan kepada pelayan-pelayan supaya lakukan apa yang diperintah oleh Tuhan Yesus. Sehingga hal yang sama juga pada kita, jikalau doa permohonan kita ingin dikabulkan maka kita harus menjalani apa yang diperintah oleh Tuhan Yesus kepada kita, yaitu melalui ajaran-ajaran-Nya baik itu melalui ajaran resmi Gereja maupun apa yang ditulis oleh para rasul dalam Kitab Suci.

\begin{flushright}\textit{sumber:http://www.carmelia.net/}\end{flushright}
