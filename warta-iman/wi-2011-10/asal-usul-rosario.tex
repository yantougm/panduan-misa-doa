\chapter*{Asal-Usul Rosario}

\begin{center}\textit{P. William P. Saunders}\end{center}

Asal-usul rosario agak ``kabur''. Penggunaan ``manik-manik'' dan pendarasan doa yang diulang-ulang untuk membantu orang dalam meditasi berasal dari masa-masa awal Gereja dan telah ada bahkan pada masa-masa sebelum kekristenan. Didapati bukti-bukti dari Abad Pertengahan bahwa untaian manik-manik dipergunakan untuk membantu orang menghitung jumlah Bapa Kami atau Salam Maria yang didaraskan. Sesungguhnya, untaian manik-manik ini kemudian dikenal sebagai ``Paternosters,'' bahasa Latin untuk ``Bapa kami''. Sebagai contoh, pada abad ke-12, guna membantu agar mereka yang kurang terpelajar dapat berpartisipasi lebih baik dalam liturgi, pendarasan 150 Bapa Kami dipakai untuk menggantikan 150 Mazmur, dan dikenal sebagai ``brevir orang-orang sederhana''.

Struktur rosario perlahan-lahan berkembang antara abad ke-12 dan abad ke-15. Pada akhirnya 50 Salam Maria (atau lebih) didaraskan dan dihubungkan dengan ayat-ayat Mazmur atau ayat-ayat lain mengenangkan ``sukacita Maria'' dalam hidup Yesus dan Maria. Dominikus dari Prussia, seorang biarawan Carthusian, pada tahun 1409 mempopulerkan praktek mempertalikan 50 ayat mengenai hidup Yesus dan Maria dengan 50 Salam Maria. Pada masa ini, bentuk doa ini dikenal sebagai rosarium (``kebun mawar''), sesungguhnya suatu istilah umum, yang berarti bunga rampai, yang dipergunakan untuk menyebut suatu kumpulan bahan yang serupa, misalnya suatu bunga rampai kisah-kisah dengan subyek atau tema yang sama. Pada akhirnya, ditambahkan juga ``dukacita Maria'' dan ``sukacita surgawi'', sehingga jumlah Salam Maria menjadi 150. Dan akhirnya, ke-150 Salam Maria digabungkan dengan ke-150 Bapa Kami; satu Salam Maria sesudah satu Bapa Kami.

Pada awal abad ke-15, Henry Kalkar (wafat 1408), seorang biarawan Carthusian lainnya, membagi ke-150 Salam Maria ke dalam kelompok-kelompok; satu kelompok berisi 10 Salam Maria dengan diawali satu Bapa Kami. Pada abad ke-16, struktur lima misteri rosario didasarkan pada tiga rangkaian peristiwa - Peristiwa GEMBIRA (1. Maria menerima kabar gembira dari Malaikat Gabriel; 2. Maria mengunjungi Elisabet, saudarinya; 3. Yesus dilahirkan di Betlehem; 4. Yesus dipersembahkan dalam Bait Allah; 5. Yesus diketemukan dalam Bait Allah), Peristiwa SEDIH (1. Yesus berdoa kepada BapaNya di surga dalam sakrat maut; 2. Yesus didera; 3. Yesus dimahkotai duri; 4. Yesus memanggul salib-Nya; 5. Yesus wafat disalib) dan Peristiwa MULIA (1. Yesus bangkit dari kematian; 2. Yesus naik ke surga; 3. Roh Kudus turun atas para Rasul; 4. Maria diangkat ke surga; 5. Maria dimahkotai di surga). Pada tahun 2002, Bapa Suci Paus Yohanes Paulus II menetapkan peristiwa CAHAYA (1. Yesus dibaptis di Sungai Yordan; 2. Yesus menyatakan DiriNya dalam perjamuan nikah di Kana; 3. Yesus mewartakan Kerajaan Allah serta menyerukan pertobatan; 4. Yesus dipermuliakan; 5. Yesus menetapkan Ekaristi). Juga, setelah penampakan Bunda Maria di Fatima pada tahun 1917, doa yang diajarkan Bunda Maria kepada anak-anak secara umum ditambahkan pada akhir setiap misteri, ``Ya Yesus yang baik, ampunilah dosa-dosa kami, selamatkanlah kami dari api neraka. Hantarlah jiwa-jiwa ke surga, teristimewa jiwa-jiwa yang amat membutuhkan kerahiman-Mu.''

Menurut tradisi, St Dominikus (wafat 1221) menyusun rosario seperti yang kita kenal sekarang. Tergerak oleh suatu penampakan Bunda Maria, ia mewartakan penggunaan rosario dalam karya misionarisnya di antara kaum Albigensia, suatu kelompok bidaah yang fanatik. Albigensia berasal dari nama kota Albi di Perancis selatan di mana mereka tinggal; mereka percaya bahwa semua yang jasmaniah adalah jahat dan semua yang rohaniah adalah baik. Karenanya, mereka menyangkal inkarnasi Tuhan kita; bagi mereka, Yesus, sungguh Allah yang menjadi sungguh manusia dan mengenakan kodrat manusiawi kita, sungguh tidak masuk akal. Menurut ajaran Albigensia, jiwa orang dianggap terbelenggu dalam tubuh yang jahat. Sebab itu, mereka berpantang kasih perkawinan dan prokreasi, sebab dianggap jahat membelenggu suatu jiwa lain dalam suatu raga. Tindakan religius mereka yang paling ``luhur'' disebut ``endura,'' suatu tindakan bunuh diri yang membebaskan jiwa dari raga. Mereka juga menentang otoritas manapun yang mewakili suatu kerajaan dunia ini, sebab itu mereka membantai para pejabat kerajaan dan para pejabat Gereja. Gereja mengutuk bidaah ini, dan St Dominikus berusaha mempertobatkan mereka melalui khotbah-khotbah yang logis dan kasih Kristiani sejati. Sayangnya, otoritas kerajaan tidak cukup berbelaskasih. (Sekedar tambahan, suatu siaran traveling menyiarkan di televisi suatu program traveling di Perancis selatan, dan mengunjungi kota Albi, mengatakan bahwa orang-orang di sana ``dianiaya oleh Gereja''; narator program tersebut tidak menyebutkan bahwa orang-orang ini adalah bidaah bunuh diri yang ajarannya membahayakan jiwa-jiwa umat beriman.) Namun demikian, St Dominikus mempergunakan rosario sebagai suatu sarana ampuh untuk mempertobatkan kaum Albigensia.

Sebab itu, rosario meupakan bagian dari sejarah rohani Gereja yang patut dijunjung tinggi. Rosario memampukan umat beriman untuk berpartisipasi dalam sejarah keselamatan yang hidup, mempersatukan kita secara lebih akrab dengan Juruselamat dan BundaNya yang Tesuci, dan dengan segenap Gereja. Rosario perlu menjadi bagian dari sejarah tiap-tiap individu dan tiap-tiap keluarga, sebab melalui doa rosario ikatan kasih diperteguh.

\textit{
* Fr. Saunders is pastor of Our Lady of Hope Parish in Potomac Falls and a professor of catechetics and theology at Christendom's Notre Dame Graduate School in Alexandria.\\
sumber : ``Straight Answers: The Origins of the Rosary'' by Fr. William P. Saunders; Arlington Catholic Herald, Inc; Copyright ©2005 Arlington Catholic Herald. All rights reserved; www.catholicherald.com\\
Diperkenankan mengutip / menyebarluaskan artikel di atas dengan mencantumkan: ``diterjemahkan oleh YESAYA: www.indocell.net/yesaya atas ijin The Arlington Catholic Herald.''}