\chapter*{\begin{center}Salam Maria dan Rosario\end{center}}

\vspace{-0.7cm}

\begin{center}\textit{oleh: P. Victor Hoagland, C.P.}\end{center}

Kita mengatakan Bapa ``kami'' dalam doa Bapa Kami. Dengan mengatakan ``kami'', kita menyatakan bahwa doa bukanlah suatu tindakan yang kita lakukan seorang diri. Kita berdoa bersama dengan yang lain.

Bersama siapakah kita berdoa? Kita berdoa bersama Yesus Kristus. Yesus tidak hanya mengajar kita bagaimana harus berdoa, tetapi Ia juga berdoa bersama kita serta mempersatukan doa-doa kita dengan doa-Nya sendiri. Oleh karena kita berdoa bersama Dia, doa-doa kita seringkali diakhiri dengan kata-kata sebagai berikut: ``dengan pengantaraan Yesus Kristus, Putra Tunggal Allah, Tuhan kita, yang hidup dan berkuasa untuk selama-lamanya.'' Kita berdoa bersama Yesus Kristus.

Bapa ``kami'' berarti kita berdoa bersama yang lain juga; sebagai contoh, mereka semua yang telah dibaptis dalam nama Kristus. Doa Bapa Kami hendaknya senantiasa mengingatkan umat Kristiani akan persatuan mereka satu dengan yang lain, meskipun sayangnya, perbedaan-perbedaan masih memisahkan gereja-gereja Kristen. Kita, umat Kristen, percaya bahwa kita bersatu dalam doa; kita dapat berdoa bersama yang lain serta saling mendoakan satu sama lain. Doa merupakan sumber hidup yang mempersatukan kita semua.

Dalam Gereja Katolik, keyakinan bahwa kita bersatu dalam doa dengan yang lain diungkapkan dalam doa kepada Bunda Maria, Bunda Yesus, dan kepada para kudus. ``Kita percaya akan persekutuan para kudus'' yang berdoa bersama kita dan bagi kita, dalam persatuan dengan Yesus Kristus.

Doa yang indah bagi Bunda Maria dalam tradisi Katolik adalah doa Salam Maria. Bagian pertama dari doa tersebut berkembang dalam abad pertengahan ketika Maria, Bunda Yesus, menjadi perhatian umat Kristiani sebagai saksi terbesar atas hidup, wafat serta kebangkitan Kristus. Bagian awal doa merupakan salam Malaikat Gabriel di Nazaret, menurut Injil Lukas:

\begin{quote}\large 
\textit{Salam Maria,\\
penuh rahmat,\\
Tuhan sertamu,
}\end{quote}

Dengan perkataan tersebut, malaikat Tuhan menyatakan belas kasih Ilahi. Tuhan akan menyertai Maria. Maria akan melahirkan Yesus Kristus ke dunia.

Bagian selanjutnya, adalah salam yang disampaikan kepada Maria oleh Elisabet, sepupunya, seperti ditulis dalam Injil St. Lukas:

\begin{quote}\large 
\textit{terpujilah engkau di antara wanita,\\
dan terpujilah buah tubuhmu, Yesus.
}\end{quote}

Dan akhirnya, pada abad ke-15, bagian doa selanjutnya ditambahkan:

\begin{quote}\large 
\textit{Santa Maria, bunda Allah,\\
doakanlah kami yang berdosa ini\\
sekarang dan waktu kami mati.
}\end{quote}
Bagian doa tersebut memohon kepada Maria, yang penuh rahmat serta dekat dengan Putra-nya, untuk mendoakan kita orang berdosa, sekarang dan saat ajal menjelang. Bersama dengan murid kepada siapa Yesus mempercayakan ibunda-Nya di Kalvari dengan mengatakan ``Inilah ibumu!'', kita mengakui Bunda Maria sebagai bunda kita. Bunda Maria akan senantiasa mendekatkan kita pada Kristus. Sejak dari permulaan Bunda Maria mengenal-Nya; ia menjadi saksi atas hidup, wafat dan kebangkitan Kristus; tidakkah Bunda Maria akan membantu kita untuk lebih mengenal Putra-nya dan misteri hidup-Nya? Kita mengandalkan belas kasih Bunda Maria kepada kita seperti yang ia lakukan bagi pasangan pengantin di Kana, di Galilea. Kita mempercayakan segala kebutuhan kita kepada Bunda Maria.

Pada akhir abad ke-16, kebiasaan mendaraskan 150 Salam Maria dalam suatu rangkaian doa atau perpuluhan menjadi populer di kalangan umat Kristiani. Dalam doa-doa tersebut, peristiwa-peristiwa  hidup, wafat dan kebangkitan Yesus direnungkan. Praktek doa itu sekarang dikenal sebagai Doa Rosario.  

Struktur rosario perlahan-lahan berkembang antara abad ke-12 dan abad ke-15. Pada akhirnya 50 Salam Maria (atau lebih) didaraskan dan dihubungkan dengan ayat-ayat Mazmur atau ayat-ayat lain mengenangkan ``sukacita Maria'' dalam hidup Yesus dan Maria. Dominikus dari Prussia, seorang biarawan Carthusian, pada tahun 1409 mempopulerkan praktek mempertalikan 50 ayat mengenai hidup Yesus dan Maria dengan 50 Salam Maria. Pada masa ini, bentuk doa ini dikenal sebagai rosarium (``kebun mawar''), sesungguhnya suatu istilah umum, yang berarti bunga rampai, yang dipergunakan untuk menyebut suatu kumpulan bahan yang serupa, misalnya suatu bunga rampai kisah-kisah dengan subyek atau tema yang sama. Pada akhirnya, ditambahkan juga ``dukacita Maria'' dan ``sukacita surgawi'', sehingga jumlah Salam Maria menjadi 150. Dan akhirnya, ke-150 Salam Maria digabungkan dengan ke-150 Bapa Kami; satu Salam Maria sesudah satu Bapa Kami.

Pada awal abad ke-15, Henry Kalkar (wafat 1408), seorang biarawan Carthusian lainnya, membagi ke-150 Salam Maria ke dalam kelompok-kelompok; satu kelompok berisi 10 Salam Maria dengan diawali satu Bapa Kami. Pada abad ke-16, struktur lima misteri rosario didasarkan pada tiga rangkaian peristiwa - Peristiwa GEMBIRA (1. Maria menerima kabar gembira dari Malaikat Gabriel; 2. Maria mengunjungi Elisabet, saudarinya; 3. Yesus dilahirkan di Betlehem; 4. Yesus dipersembahkan dalam Bait Allah; 5. Yesus diketemukan dalam Bait Allah), Peristiwa SEDIH (1. Yesus berdoa kepada BapaNya di surga dalam sakrat maut; 2. Yesus didera; 3. Yesus dimahkotai duri; 4. Yesus memanggul salib-Nya; 5. Yesus wafat disalib) dan Peristiwa MULIA (1. Yesus bangkit dari kematian; 2. Yesus naik ke surga; 3. Roh Kudus turun atas para Rasul; 4. Maria diangkat ke surga; 5. Maria dimahkotai di surga). Pada tahun 2002, Bapa Suci Paus Yohanes Paulus II menetapkan peristiwa CAHAYA (1. Yesus dibaptis di Sungai Yordan; 2. Yesus menyatakan DiriNya dalam perjamuan nikah di Kana; 3. Yesus mewartakan Kerajaan Allah serta menyerukan pertobatan; 4. Yesus dipermuliakan; 5. Yesus menetapkan Ekaristi). Juga, setelah penampakan Bunda Maria di Fatima pada tahun 1917, doa yang diajarkan Bunda Maria kepada anak-anak secara umum ditambahkan pada akhir setiap misteri, ``Ya Yesus yang baik, ampunilah dosa-dosa kami, selamatkanlah kami dari api neraka. Hantarlah jiwa-jiwa ke surga, teristimewa jiwa-jiwa yang amat membutuhkan kerahiman-Mu.''

Menurut tradisi, St Dominikus (wafat 1221) menyusun rosario seperti yang kita kenal sekarang. Tergerak oleh suatu penampakan Bunda Maria, ia mewartakan penggunaan rosario dalam karya misionarisnya di antara kaum Albigensia, suatu kelompok bidaah yang fanatik. Albigensia berasal dari nama kota Albi di Perancis selatan di mana mereka tinggal; mereka percaya bahwa semua yang jasmaniah adalah jahat dan semua yang rohaniah adalah baik. Karenanya, mereka menyangkal inkarnasi Tuhan kita; bagi mereka, Yesus, sungguh Allah yang menjadi sungguh manusia dan mengenakan kodrat manusiawi kita, sungguh tidak masuk akal. Menurut ajaran Albigensia, jiwa orang dianggap terbelenggu dalam tubuh yang jahat. Sebab itu, mereka berpantang kasih perkawinan dan prokreasi, sebab dianggap jahat membelenggu suatu jiwa lain dalam suatu raga. Tindakan religius mereka yang paling ``luhur'' disebut ``endura,'' suatu tindakan bunuh diri yang membebaskan jiwa dari raga. Mereka juga menentang otoritas manapun yang mewakili suatu kerajaan dunia ini, sebab itu mereka membantai para pejabat kerajaan dan para pejabat Gereja. Gereja mengutuk bidaah ini, dan St Dominikus berusaha mempertobatkan mereka melalui khotbah-khotbah yang logis dan kasih Kristiani sejati. Sayangnya, otoritas kerajaan tidak cukup berbelaskasih. (Sekedar tambahan, suatu siaran traveling menyiarkan di televisi suatu program traveling di Perancis selatan, dan mengunjungi kota Albi, mengatakan bahwa orang-orang di sana ``dianiaya oleh Gereja''; narator program tersebut tidak menyebutkan bahwa orang-orang ini adalah bidaah bunuh diri yang ajarannya membahayakan jiwa-jiwa umat beriman.) Namun demikian, St Dominikus mempergunakan rosario sebagai suatu sarana ampuh untuk mempertobatkan kaum Albigensia.

Bunda Maria senantiasa menjadi teladan iman dan pelindung orang-orang Kristen yang percaya. Ketika Malaikat Gabriel datang kepadanya, ia percaya akan warta yang disampaikan malaikat dan tetap teguh pada imannya tanpa ragu sedikit pun meskipun harus melewati pencobaan gelap Kalvari. Bunda Maria mendampingi kita juga yang adalah saudara dan saudari Putra-nya, sepanjang ziarah kita di dunia yang penuh dengan kesulitan dan mara bahaya.

Selama berabad-abad telah banyak umat Kristiani mengakui bahwa Salam Maria dan Rosario merupakan sumber rahmat rohani. Doa rosario adalah doa yang sederhana sekaligus mendalam. Rosario dapat dilakukan siapa saja, pengulangan kata-katanya mendatangkan kedamaian bagi jiwa. Renungan akan kisah hidup Yesus dalam peristiwa-peristiwa gembira, cahaya, sedih, maupun mulia dimaksudkan agar diamalkan dalam hidup kita sendiri. Melalui peristiwa-peristiwa tersebut, kita berharap untuk ``meneladani apa yang diteladankan dan memperoleh apa yang dijanjikan''.

\scriptsize
\begin{flushleft}
\textit{sumber : \\
``The Hail Mary and the Rosary'' by Fr. Victor Hoagland, C.P.; \\
Copyright 1997-1999 - The Passionist Missionaries; www.cptryon.org/prayer
\\
Diperkenankan mengutip / menyebarluaskan artikel di atas dengan mencantumkan:\\ ``diterjemahkan oleh YESAYA: www.indocell.net/yesaya atas ijin Fr. Victor Hoagland, CP.''}
\end{flushleft}
\normalsize