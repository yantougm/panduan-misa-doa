\newpage
\section*{\Large \begin{center}Sendangsono\end{center}}
	Suatu hari saya bertemu dengan teman lama Johanes Sukendro, teman semasa masih kuliah dulu. 
	
	``Apa kabar Jo demikian panggilan akrabnya, lama tak berjumpa, istrimu masih mantan pacarmu dulu Martha anak managemen itu?'' tanyaku bertubi-tubi karena lama tak pernah bertemu dengan sahabat lama semasa kuliah dan kami berpencar setelah lulus tanpa dapat berkomunikasi selama 20 tahun. 
	
	``Aku, Martha istriku dan tiga anakku baik-baik dan sehat-sehat saja, kamu sendiri bagaimana Gung? Istrimu sekarang berapa dan siapa saja namanya, dan anakmu berapa orang?'' jawab Jo dilanjutkan dengan pertanyaan yang beruntun juga kepadaku. Jabat tangan kami belum terlepas dan kami berangkulan  untuk melepas kerinduan kami. Kemudian kami saling bercerita keadaan keluarga kami sambil bernostalgia sewaktu masih kuliah dulu.
	
	Johanes Sukendro setelah lulus sempat menganggur akhirnya diterima bekerja di sebuah bank swasta di Surabaya hingga jabatan akhir  sebagai \textit{branch manager}, tapi sayang sekali akhirnya Jo ``dipensiunkan'' karena bank tersebut dilikuidasi oleh pemerintah. Sekarang Jo membuka usaha tambak di Situbondo. Yang mengelola masih mantan anak buahnya sewaktu bekerja di bank. Di Yogya Jo mempunyai rumah besar dengan 15 kamar, sebagian untuk usaha kost mahasiswi. Mobilnya ada tiga. Namun hari-hari terakhir ini Jo mengaku menjadi orang miskin. 
	
	``Hidup semakin susah'' gerutunya kepadaku. ``Aku kemarin baru pulang menengok usaha tambakku di Situbondo. Dari laporan anak buahku, usaha tambakku mengalami kerugian besar karena ada virus yang menyerang tambak udang windu. Gila, aku bisa gila Gung!'' kata Jo sambil menendang gundukan pasir. ``Dua ratus juta melayang dalam sekejap. Penghasilanku tinggal dari usaha kost. Berapa besar sih hasil usaha kost? Paling cuma cukup buat bayar listrik dan telepon. Apa kami tidak makan?'' keluh Jo dengan wajah murung.
	 
``Coba, menurutmu Gung, Tuhan itu adil atau tidak?'' Tanya Jo kepadaku.

``Sangat adil,'' jawabku pelan.

``Sangat adil? Usaha tambakku adalah satu-satunya penghasilan utamaku untuk menghidupi anak istriku. Mengapa Tuhan membiarkan usaha tambakku hancur mengalami kerugian?'' tukas Jo.    

``Tuhan adalah penguasa jagad raya. Tetapi Dia tidak mengurusi tiga petak tambak, Jo.'' ujarku.

``Lalu dimana keadilan-Nya?'' protes Jo dengan nada tinggi.
Saya tarik  tangan Jo dan mengajaknya duduk sambil memandangi hijaunya sawah.

``Jo, kamu asli mana?'' Tanya saya. 

``Dari Sragen. Kan kamu sudah tahu asal usulku.'' Ujar Jo heran.

``Ya, sebuah desa di Sragen. Jauh dari kota. Ketika SD berangkat sekolah tidak pakai sepatu, karena bapaknya tidak mampu membelikan sepatu. Baru di SMP dibelikan sepatu. Di SMA pagi sekolah, sorenya membantu saudara jual makanan dan minuman. Jadi tak sempat menikmati masa remaja. Saat kuliah pun sama. Di sela-sela waktu luang jadi loper Koran dan majalah, kadang-kadang jadi kernet, ikut cuci mobil. Begitu lulus sarjana tidak langsung dapat pekerjaan. Tiap minggu jalan salib dan ziarah ke sendangsono. Memohon-mohon kepada Bunda Maria agar mau menjadi perantara doa, supaya Tuhan Yesus memberi jalan agar Jo segera memperoleh pekerjaan. Berapa lama Jo ziarah dan jalan salib ke sendangsono waktu itu?

``Kurang lebih Sembilan bulan,'' jawab Jo pelan.

``Setiap minggu selama Sembilan bulan. Dan Tuhan mengabulkan doa Jo. Malah memberi kelimpahan, Jo bekerja dengan jabatan kepala cabang di sebuah bank dengan gaji besar. Sekarang punya tiga rumah, tiga mobil, tabungan yang tidak sedikit. Coba renungkan sebentar Jo, apakah Tuhan tidak adil?'' tanyaku.

Laki-laki itu diam. Matanya memandang ke bentangan sawah hijau yang menyejukkan.

``Lalu, adilkah Jo terhadap Tuhan?'' Tanya saya kemudian.

``Adil yang bagaimana, Gung?'' Jo malah bertanya kepadaku.

``Pernahkah Jo mengorbankan waktu untuk Tuhan? Bukankah selama ini Jo sangat sibuk dengan mencari uang. Ke gereja hanya kadang-kadang, doa bersama umat di wilayah atau lingkungan tak pernah ada waktu, juga untuk membantu sesama yang membutuhkan, tidak pernah membantu saat pembangunan gereja. Lalu kapan Jo akan punya waktu untuk Tuhan dan gereja? Gereja tidak hanya butuh kolekte saja, tetapi hati, pikiran dan tenaga kita pun dibutuhkan.''  

Laki-laki itu diam. Ketika saya lirik beberapa saat kemudian, matanya tampak berkaca-kaca.

``Ya, ya Gung, aku jadi sangat malu,'' ucap Jo dengan suara serak.

\section*{Ziarah ke Sendangsono}
	Dua bulan kemudian saya baru bertemu dengan Jo. Selama itu pula ia mengaku sedang menelusuri kembali jalan hidup yang selama ini ia jalani. Dan Jo beserta dengan keluarganya mengajak saya ziarah ke sendangsono.  Jo mengajak Martha istrinya dan tiga anaknya yang semuanya sudah menjadi sarjana. Di perjalanan Jo ``buka kartu'' di depan istri dan anaknya. 
	
``Bapakmu ini dulu orang susah. Bisa menjadi sarjana ekonomi karena kerja keras. Bisa bekerja di bank karena aku menangis-nangis di depan Bunda Maria. Setiap minggu bapakmu sowan Bunda Maria di Sendangsono. Tetapi ketika hidup kita membaik, apa yang dulu tidak kita punyai sekarang kita miliki secara berlebih, bapakmu sering melupakan Bunda Maria. Bahkan juga melupakan Tuhan Yesus, karena jarang ke gereja. Bahkan aku juga tidak sempat mengajari kalian anak-anakku bagaimana berdoa dengan baik. Nah, waktunya belum terlambat, mumpung aku masih diberi kesehatan, sedikit kekayaan, kini apa yang kumiliki akan kupersembahkan bagi kemuliaan Tuhan. Kalian setuju?'' ucap Jo dengan penuh semangat 45.

``Setuju!'' jawab istri dan tiga anaknya serentak. ``setuju sekali!'' sambung mereka.

``Bagus, bapak berterima kasih kepada kalian. Tadi bapak ragu-ragu, jangan-jangan karena pengaruh  kehidupan metropolitan kalian lalu tidak mendukung niatku.'' Ujar Jo kepada istri dan anak-anaknya.

``Kami semua mendukung niat Bapak!'' kata mereka serempak kompak.

Jo mengangguk-angguk. Ketika saya lirik, matanya tampak berkaca-kaca. Di depan Bunda Maria, Jo malah menangis. Ia tak kuasa menahan air matanya. Begitupun Martha, istrinya, wanita setengah baya itu juga tak kuasa menahan air matanya.

Di depan bunda Maria, Bunda Segala Bangsa, Jo yang didampingi istri dan ketiga anaknya, saya minta agar mereka merenungi betapa besar rahmat dan karunia yang telah mereka terima dari Allah lewat Bunda Maria. Upaya dan doa yang dulu dilakukan Jo tidak sia-sia. Bunda Maria mendengar keluh kesah hatinya, lalu menyampaikannya kepada Putra yang dikasihi, Tuhan Yesus Kristus.

	Allah Bapa yang maha kasih, kami sekeluarga bersimpuh di depan wanita pilihan-Mu, Bunda Maria, Bunda Segala Bangsa Engkau telah member rahmat dan karunia serta mendampingi diriku selama ini. Dulu aku sering melupakan-Mu, ya Tuhan, demi mengejar harta dunia. Aku bersalah, aku berdosa, namun Engkau tetap mengampuni aku orang bedosa ini. Karena kebodohanku, maka bimbinglah aku. Karena kelemahanku, maka kuatkanlah aku, ya Tuhan. Dengan bimbingan Bunda Maria pula, Perkenankanlah aku dan segenap keluargaku menjadi hamba-Mu. Semua ini kami mohonkan kehadapan-Mu dengan perantaraan Kristus, Tuhan dan pengantara kami. Amin.

	Selesai berdoa Jo mendekatiku dan memelukku, ``Dua buah mobilku sudah kujual, setengah hasil penjualan mobil aku sumbangkan untuk pembangunan gereja dan panti asuhan, yang setengah lagi kujadikan modal untuk peternakan, Thanks sobat, kamu adalah utusan Allah untuk keluargaku!'' bisik Jo kepadaku. Aku bengong gak mengerti bagaimana bisa begitu cepatnya Jo mengerti akan hal ini?  Aku hanya terpaku sambil menggaruk-garuk rambut kepalaku yang tidak gatal, memandangi Jo melangkah dengan wajah berseri.

\begin{flushright}
\textit{Medio, Sept’11\\
Bravo Sierra   
}\end{flushright}