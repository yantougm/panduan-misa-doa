\newpage
\section*{\centering\Large KRITIS, KRITIK, KREATIF, DAN KEGIATAN DALAM PERJALANAN GEREJA BUNDA MARIA}

\small
	Benih-benih umat Katolik di Maguwoharjo mulai bertunas sekitar tahun 70-an. Saat itu beberapa keluarga Katolik tinggal di Maguwoharjo dan berkelompok membentuk Kring di bawah Paroki Kalasan. Meskipun masih sedikit namun tidak mengurangi semangat mereka untuk senantiasa mengikuti Misa kudus setiap hari Minggu di Gereja Paroki, yakni Gereja Marganingsih Kalasan meskipun harus mengendarai sepeda dengan jarak yang relatif cukup jauh.

	Masih sedikitnya jumlah umat Katolik di Maguwoharjo berdampak pada pendidikan iman anak di sekolah. Seorang Bapak A tersentak saat mendengar anaknya bernyanyi :``Ayo, ayo, ayo/ Piye, piye, piye/ Allah Pangeranku/Muhammad Nabiku/ Islam agamaku''. Timbul pikiran kritisnya, dari mana anakku belajar lagu itu? Ternyata di sekolah anak itu mengikuti pelajaran agama Islam, karena memang tidak ada guru agama Katolik yang ditugaskan mengajar di sekolah itu oleh pemerintah karena sangat sedikitnya murid yang beragama Katolik. Bapak itu kemudian datang ke sekolah dan menkonfirmasi kondisi tersebut dengan bertanya kepada Kepala Sekolah, mengapa sekolah tidak menyediakan guru agama Katolik honorer? Kepala Sekolah minta maaf atas kondisi tersebut dan bercerita bahwa sebenarnya sekolah sudah berupaya mencari guru agama Katolik, namun 2 guru agama Katolik, yakni Bu Endang dan Pak Dalikin yang pernah mengajar di sekolah itu hanya bertahan beberapa bulan saja.

	Melihat kondisi tersebut seorang Bapak, sebut saja Bapak X, yang sudah lama berkecimpung di organisasi kemasyarakatan lantas berpikir kreatif bagaimana cara menyediakan seorang guru agama di sekolah tersebut. Bapak X berpandangan bahwa iman anaknya dan anak-anak Katolik lain di sekolah tersebut akan Yesus harus sudah mulai ditumbuhkan sejak kecil. Setelah diskusi lama dengan sang istri ternyata istri Bapak X sendiri bersedia menjadi guru agama Katolik. Sejak saat itu istri Bapak X menjadi guru agama Katolik, tidak hanya di SD Maguwoharjo namun diminta juga mengajar di SD Triharjo dan SD Sorogenen. Benih-benih iman Katolik mulai ditabur Tuhan melalui ibu guru yang baik ini. Karya tangan Tuhan tidak berhenti hanya sampai anak-anak SD saja, namun meluas pula ke beberapa orangtua di Sambilegi, Karang Ploso, Dewan dan Susteran. Inilah momentum yang menandai berkembangnya umat Katolik di Maguwoharjo.

	Jumlah umat yang semakin banyak ini memunculkan beberapa ide kreatif beberapa pengurus Kring untuk mengusahakan adanya Misa hari Minggu di Maguwoharjo yakni di Kapel Dominikus, sehingga umat tidak harus pergi jauh ke Gereja Kalasan. Status Kring Maguwo juga ditingkatkan menjadi Wilayah Maguwo. Lama kelamaan Kapel Susteran Dominikus tidak muat lagi, sehingga Misa hari Minggu dipindahkan ke aula SPG Sanjaya milik Susteran Dominikus.

	Tahun 1984 Suster Kepala menginformasikan bahwa aula tidak bisa digunakan lagi sebagai tempat ibadah karena akan diubah fungsinya sebagai Laboratorium yang sangat dibutuhkan sekolah. Suster Kepala juga berpendapat bahwa sudah saatnya umat Katolik Maguwoharjo mulai memikirkan pembangunan gereja di wilayah Maguwoharjo mengingat jumlah umat yang semakin banyak.

	Puji Tuhan, kebutuhan umat akan gereja terjawab dengan adanya hibah tanah dari salah seorang umat. Maka pada tanggal 10 Februari dibentuklah Panitia Pembangunan Gereja. Panitia bekerja berdasarkan ``cengkir'' yaitu ``kencenging pikir'' dan ``tebu'' yaitu ``antebing kalbu''. Panitia berusaha menghimpun dana dari umat dan juga usaha-usaha lain dengan semboyan ``sethithik ora ditampik, akeh luwih pekoleh''. Kerja keras umat dan Panitia Pembangunan dengan disertai doa permohonan tiada henti kepada Bunda Maria membuahkan mukjijat. Sebuah bangunan gereja meskipun sederhana berhasil didirikan ditengah keterbatasan umat. 

	Tanggal 2 Juni 1988 Uskup Mgr. Yulius Darmoadmadja, SJ berkenan memberkati  gereja baru tersebut dengan nama pelindung Bunda Maria dan meresmikan penggunaannya bersama Bapak Bupati Sleman waktu itu yaitu Drs. H. Arifin Ilyas. Status Wilayah yang masih disandang juga ditingkatkan menjadi Stasi Maguwo. Mulai saat itu setiap hari Minggu pagi umat Katolik Maguwoharjo dengan sukacita mengadakan Misa di gereja Bunda Maria. Panitia Pembangunan masih melanjutkan pembangunan beberapa bangunan lain termasuk joglo di bagian belakang.  Setelah dirasa cukup, maka pada tanggal 13 September 1994 Panitia Pembangunan Gereja secara resmi dibubarkan melalui surat keputusan nomor 02-L/IX/1994. Saat acara tersebut Ketua panitia menyerahkan Laporan Pertanggung jawaban dan sisa uang sebesar Rp. 600.000 yang direncanakan sebagai modal untuk melanjutkan pembangunan 2 kamar pengakuan dosa disamping gereja dan menara lonceng.

	Beberapa tahun kemudian gereja Bunda Maria kembali melanjutkan pembangunannya berupa Gazebo dan gedung pertemuan di bagian belakang dan perluasan sisi kanan gereja. Kiranya pembangunan akan terus berlanjut mengingat jumlah umat yang semakin berkembang. 

	Refleksi yang dapat Penulis ambil dari perjalanan sejarah perkembangan dan pembangunan Stasi Maguwo adalah bahwa sikap kritis, kritik, kreatif dan kegiatan merupakan 4 pilar yang senantiasa perlu dikembangkan. Kritis untuk menemukan hal-hal yang tidak baik, berani mengkritik secara santun temuan tersebut namun juga kreatif mengajukan solusi secara positif dan ikut berperan aktif berkegiatan nyata, tidak berhenti pada wacana semata.

\begin{flushright}\textit{AR} \end{flushright}
\normalsize
	