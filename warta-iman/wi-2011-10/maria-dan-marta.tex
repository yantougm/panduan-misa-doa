\newpage
\section*{\begin{center}Maria dan Marta\end{center}}

Banyak kisah ``aneh'' di dalam Alkitab. Maksudnya, kisah yang menggelitik keinginan untuk bertanya. Salah satunya adalah kisah singgahnya Yesus di rumah Maria dan Marta dari kampung Betania. Mengapa ``aneh''? Sebab anjuran untuk melayani sangat ditekankan oleh Injil Lukas 10:38-42. Sekarang tiba-tiba ketika ada orang melayani, Yesus malah menegurnya. Mengapa?


Marta, dalam kisah ini disebutkan ``sibuk sekali melayani''. Ia melayani sedemikian rupa, sehingga tidak bisa melihat pentingnya apa yang dilakukan oleh Maria, yaitu ``duduk dekat kaki Tuhan dan terus mendengarkan perkataan-Nya''. Ia tidak mengerti tindakan Maria. Sementara itu, ia melayani sambil menggerutu dan mengasihani diri. Padahal apa yang dilakukan Maria adalah bagian utama dari tindakan melayani. Hati yang menyembah dan rindu mendengar suara Tuhan ibarat mata air dari sebuah tindakan pelayanan. Tanpa itu, melayani hanya akan menjadi sederet ``kesibukan'' dan kegelisahan yang serba ``menyusahkan diri dengan banyak perkara''.

Marta tidak sendiri. Sebagai pelayan di pelbagai aktivitas kristiani, kita pun kerap begitu sibuk dan kehilangan sukacita. Sebagai gantinya, kita terus mengeluh, mengasihani diri, dan mencela sesama pelayan. Satu hal yang harus kita ingat: kita bukan melayani ``sesuatu'', melainkan ``Seorang Pribadi'', yaitu Yesus. Tanpa hubungan kasih yang hangat secara pribadi dengan Yesus, pelayanan akan menjadi beban. Marta tidak keliru karena melayani. Ia keliru karena melupakan nilai penting tindakan Maria. Masihkah kita melayani karena mengasihi Yesus?

\begin{center}\textbf{\textit{PELAYANAN BUKAN PILIHAN ANTARA \\TINDAKAN MARIA ATAU MARTA, \\
MELAINKAN KOMBINASI ANTARA KEDUANYA}}
\end{center}

\begin{flushright}\textit{Pipi Agus Dha}\end{flushright}