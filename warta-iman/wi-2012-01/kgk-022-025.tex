\newpage
\chap{Kompendium Katekese Gereja Katolik}
\setcounter{kgkcounter}{21}
{\normalsize
\section*{KITAB SUCI}

\kgk{Apa pentingnya Perjanjian Baru bagi umat Kristen?}
Perjanjian Baru, yang berpusat pada Yesus Kristus, menyatakan kebenaran
terakhir wahyu ilahi kepada kita. Dalam Perjanjian Baru, keempat Injil menurut
Matius, Markus, Lukas, dan Yohanes merupakan inti dari seluruh Kitab Suci
karena merupakan saksi utama hidup dan ajaran Yesus. Dengan demikian,
keempatnya mempunyai tempat yang unik di dalam Gereja

\kgk{Bagaimana kesatuan Perjanjian Lama dan Perjanjian Baru?}
     Kitab Suci adalah satu sejauh Sabda Allah itu satu. Rencana penyelamatan              
Allah itu satu, dan inspirasi ilahi dari kedua Perjanjian itu juga satu. Perjanjian Lama   
mempersiapkan yang Baru dan Perjanjian Baru menyempurnakan yang Lama,
keduanya saling menerangkan satu sama lain.

\kgk{Apa peranan Kitab Suci di dalam kehidupan Gereja?}
     Kitab Suci memberikan dukungan dan kekuatan bagi kehidupan Gereja. Bagi               
Putra-Putri Gereja, Kitab Suci merupakan suatu peneguhan iman, makanan jiwa,               
dan sumber hidup spiritual. Kitab Suci adalah jiwa teologi dan khotbah pastoral.
Para pemazmur berkata bahwa Kitab Suci ”pelita bagi kakiku dan cahaya bagi
langkahku” (Mzm 119:105). Karena itu, Gereja menganjurkan semua umat beriman
untuk sering membaca Kitab Suci karena ”tidak mengenal Kitab Suci berarti tidak
mengenal Kristus” (Santo Hieronimus).

\begin{center}\textbf{JAWABAN MANUSIA KEPADA ALLAH: AKU PERCAYA} \end{center}

\kgk{Bagaimana manusia menjawab Allah yang mewahyukan Diri-Nya?}
     Dengan bantuan rahmat ilahi, kita menjawab Allah dengan ketaatan iman, yang
berarti penyerahan diri kita kepada Allah secara penuh dan menerima kebenaran-
Nya sebagaimana dijamin oleh Dia, sang Kebenaran sejati.


\flushright{(\dots \emph{bersambung} \dots)}
}