\chap{Sakramen Pembaptisan dan Sejarah Perubahan Ritusnya}

    Seperti sebuah karya seni yang telah diperbaiki, difernis sampai beberapa kali dan kadang kala sedikit disalahgunakan, ritus Sakramen Pembaptisan dalam Gereja Kristen juga telah mengalami perubahan selama berabad-abad dengan alasan dan latar belakang tertentu. Namun biarpun demikian, perubahan ritus itu tetap tidak mengubah hakekat Pembaptisan sebagai Sakramen tanda kelahiran baru seseorang ke dalam Kerajaan Allah di dalam Yesus Kristus. Tapi dewasa ini Gereja dihimbau untuk kembali ke ritus pembaptisan yang dipraktekkan oleh Gereja Kristen pada abad-abad pertama yaitu pembaptisan dengan dengan "mencelupkan" seorang catechumen ke dalam kolam air. Karena ritus pembaptisan seperti ini dirasa lebih efektif mengungkapkan peristiwa "kelahiran baru" seseorang ke dalam Karajaan Allah yang dimasukinya melalui Sakramen Permandian. Tapi demi alasan praktis, Gereja tetap diijinkan untuk memakai ritus pembaptisan sederhana dengan "menuangkan sedikit air pada kepala". Untuk lebih mamahami hal ini, mari kita kembali menengok sejarah Gereja seputar ritus sakramen permandian.

    ***

    Dalam bab pertama Injil Markus, seperti telah diramalkan Nabi Yesaya, Yohanes Pembaptis tampil di padang gurung sambil memaklumkan sebuah "pembaptisan pertobatan" demi pengampunan atas dosa. Orang banyak berkerumun mendengar kothbah Yohanes Pembaptis yang berkobar-kobar. Mereka berbondong-bon-dong ke Sungai Yordan dan memberi diri mereka untuk disucikan (Kata pembaptisan sendiri berasal dari kata Yunani "bapizo" yang berarti "membasuh atau mencelupkan atau menenggelamkan").

    Markus menceriterakan bahwa Yesus juga kemdian datang kepada Yohanes dan dibaptis. Ketika Ia keluar dari air, Ia melihat surga terbuka dan Roh Kudus dalam rupa seekor burung merpati turun ke atasNya. Dan Ia mendengar suara dari atas yang mengatakan:"Engkaulah PuteraKu yang terkasih; kepadamu aku berkenan."

    Penginjil Markus kemudian melanjutkan ceriteranya dengan mengatakan bahwa segera setelah peristiwa permandian di Sungai Yordan, Yesus dibawa Roh Allah ke padang gurun and tinggal di sana untuk berdoa dan berpuasa 40 hari lamanya, sambil digodai iblis. Setelah berdoa dan berpuasa, Yesus mulai menjalankan perutusanNya di depan umum.

    Lima belas bab kemudian, pada bagian akhir dari Injilnya, Penginjil Markus kembali mencatat kata-kata terakhir Yesus kepada kesebelas rasul (dikurangi Yudas Iskariot yang telah mengkianati Yesus): "Pergilah ke seluruh bumi dan wartakanlah Injil \dots ~Barangsiapa percaya dan dibaptis akan diselamatkan."

    Kalau kita meneliti Kitab Suci Perjanjian Baru (PB) maka kita akan menemukan kenyataan bahwa Kitab Suci PB tidak menceriterakan "bagaimana para Rasul membaptis". Tapi ahli sejarah Gereja berpendapat bahwa kemungkinan besar seorang calon permandian berdiri di air sungai atau di sebuah kolam umum, dan kemudian air dituangkan ke atas kepalanya, sambil ditanyakan kepadanya: Apakah saudara (saudari) percaya kepada Allah Bapa? Apakah saudara percaya akan Allah Putera, yaitu Yesus Kristus? Apakah saudara percaya akan Allah Roh Kudus? Setiap kali calon menjawab "ya" atas masing-masing pertanyaan itu, ia ditenggelamkan (dicelupkan ) ke dalam air sebanyak tiga kali juga.

    Tentang hal ini, Yustinus Martir (100-165 AD) menulis begini:

\small
\begin{quote}\textit{
    "Calon permandian berdoa dan berpuasa.\\
    Komunitas beriman berdoa dan berpuasa dengan dia.\\
    Calon permandian masuk ke dalam air.\\
    Petugas Gereja mengajukan kepadanya tiga pertanyaan Trinitaris.\\
    Calon sekarang diperkenalkan kepada komunitas umat beriman.\\
    Doa umat kemudian disampaikan oleh semua untuk yang baru saja dibaptis.\\
    Ciuman tanda kasih dan damai diberikan kepadanya oleh semua umat beriman.\\
    Lalu Ekaristi kudus dirayakan."}
\end{quote}
\normalsize

    Setengah abad kemudian, pujangga Gereja Tertulianus menjelaskan lebih detail lagi. Ia mulai menyebut adanya "pengurapan" minyak suci, "tanda salib" dan "penumpangan tangan" atas calon permandian.

    Untuk orang-orang yang hidup pada tiga abad yang pertama sesudah Yesus, langkah-langkah yang harus ditempuh sebelum dibaptis tidak terlalu gampang. sering mereka diarahkan kepada kemartiran.

    Sebelum Kaisar Romawi Konstantinus mengumumkan pada tahun 313 bahwa Gereja Kristen bukan lagi sebuah agama ilegal, maka setiap orang, baik laki-laki, perempuan maupun anak-anak yang menggabungkan diri menjadi orang Kristen dipandang sebagai sebagai penjahat dan dihukum dengan sangat keji. Ingat sejarah Gereja. Selama tiga abad pertama orang-orang Kristen dianiaya dan dibunuh oleh pemerintahan kafir Romawi. Orang Romawi pada masa itu mempunyai agama sendiri dengan pusat kultus penyembahan kepada dewa-dewi. Orang Kristen yang tidak menyembah dewa-dewi sembahan kaisar dianggap kafir, kriminal, melawan kaisar dan mereka dihukum dengan sangat keji seperti digantung hidup-hidup dikayu salib, dibakar hidup-hidup, digoreng dan direbus hidup-hidup, dilempar hidup-hidup ke dalam kandang singa yang sengaja tidak diberi makan berhari –hari supaya mereka lapar betul dan makan orang Kristen.

    Kemungkian besar Gereja waktu itu menyusun sebuah proses perkenalan kepada orang yang baru bergabung ke dalam komunitas umat beriman. Gereja (umat beriman) butuh waktu untuk mengenal dan percaya kesungguhan hati setiap calon permandian sebelum mereka dipermandikan (sama seperti si calon permadian juga butuh waktu untum memperlajari lebih tentang Gereja yang merupakan agama "di bawah tanah" pada masa itu).

    Ada suatu alasan mengapa calon permandian membutuhkan sponsor (wali permandian, bapa-ibu permandian), yaitu seorang anggota komunitas beriman yang menjamin si calon permandian. Sponsorlah yang bertugas pergi menghadap uskup dan membuktikan kepadanya bahwa calon permandian merupakan seorang yang sungguh baik. Lalu, selama bertahun-tahun sponsor bekerja, berdoa dan berdoa bersama anak/orang didikannya sampai pada hari pembaptisan tiba.

    Pada waktu itu, masa katekumen (dari bahasa Yunani yang berarti "instruction" atau pelajaran) terdiri atas dua bagian.

    Bagian pertama adalah sebuah "masa persiapan rohani" yang berlangsung selama kurang lebih tiga tahun. Bagian kedua adalah masa persiapan akhir menjelang permbaptisan. Bagian ini dimulai pada masa Puasa dan kegiatannya terdiri atas doa-doa yang rutin, puasa, dan penelitian kelayakan sang calon permandian oleh uskup.

    Kemudian si calon dibawa ke depan uskup dan para imam, sementara sang sponsor ditanyai. jika sponsor bisa menjamin bahwa sang calon tidak mempunyai tabiat buruk yang serius (seperti mabuk, tiak menghormati orangtua dan lain-lain) uskup kemudian mencatat nama calon ke dalam buku baptis.

    Calon tidak diijinkan untuk mengambil bagian secara penuh dalam perayaan misa kudus. Setelah Liturgi Sabda (sesudah homili) seorang calon permandian diminta untuk meninggalkan Gereja atau tempat berlangsungnya perayaan misa kudus. Para calon hanya diijinkan untuk mendengar Credo dan doa Bapa Kami dan menghafalnya secara diam-diam.

    Puncak dari upacara itu dimulai pada Hari Kamis Suci dengan sebuah wadah pemandian sebagai sarana penyucian rohani. Calon kemudian berdoa dan berpuasa keras pada Hari Jumat Agung dan Sabtu Suci.

    Pada malam hari Sabtu Suci, calon permandian laki-laki dan wanita ditempatkan di ruangan yang terpisah dan gelap. Di ruangan yang gelap ini, setiap calon berdiri sambil menghadapkan wajah ke arah Barat (Barat dianggap simbol kegelapan dan setan, karena matahari terbit di Timur). Seorang diakon akan meminta para calon untuk merentangkan lengan mereka dan menghembuskan nafas untuk mengeluarkan semua roh yang tidak baik dari dalam tubuh, sambil berkata: "Saya melepaskan diri dari kau, setan, dari kungkunganmu, dari segala pernyembahan terhadapmu dan semua malaikatmu yang jahat." Lalu sesudah itu, sambil memutarkan badan ke arah Timur, para calon berseru: "Sekarang saya menyerahkan diriku kepadaMu, O Yesus Kristus." Berdasarkan ini, bertobat kemudian harafiah berarti "memutar haluan" (\textit{turning around}).

    Sampai di sini, para calon kemudian menurunkan tangan dan lengan mereka, dan uskup lalu mengurapi kepala mereka ma-sing-masing dengan minyak. Ini adalah lambang meterai Kristus. Sekarang secara rohani mereka ditandai, sama seperti seorang gembala menandai (mencap) kawanan ternaknya.

    Sesudah itu setiap kelompok akan pergi ke ruang lain dan menanggalkan pakaian mereka. Peristiwa "penanggalan pakaian" ini melambakan "penanggalan manusia lama dari seseorang" (\textit{taking off the old self}) dan kembali ke keadaan murni taman Eden sebelum manusia pertama jatuh ke dalam dosa dan lebih dari itu ada kepercayaan orang pada masa itu bahwa roh-roh jahat sering melekat pada pakaian seseorang seperti kutu busuk.

    Lalu dalam keadaan telanjang dan terpisan menurut jenis kelamin, para calon dihantar ke tempat permandian. Setiap calon masuk ke dalam air yang dalamnya sampai setinggi dada dan uskup akan berlutut di samping kolam air. uskup lalu dengan halus menekan kepada calon ke dalam air sampai tiga kali, sambil mempermandikan (menuangkan air) mereka satu persatu di dalam nama Bapa, Putera dan Roh Kudus.

    Setelah orang Kristen yang baru dipermandikan itu keluar dari air dan setelah tubuh mereka dilap, mereka diberi pakaian baru berbentuk kain linen putih yang mereka pakai sampai minggu berikut. Setiap anggota baru dari komunitas umat beriman dibagikan sebuah lilin bernyala dan ciuman tanda kasih dan damai.

    Setelah semua calon dibaptis, mereka merayakan Ekaristi dengan seluruh komunitas umat beriman. Untuk pertama kali, orang yang baru dibaptis mengambil bagian secara penuh dalam seluruh misa dan menerima Komuni Kudus.

    Kemudian hari, aspek kerahasiaan dan kesedian calon untuk mengorbankan hidupnya untuk mati demi Kristus menjadi pudar setelah Gereja Kristen diterima sebagai agama resmi Kekaisaran Roma pada awal abad IV. Lebih dari itu, sejak Gereja Kristen diakui sebagai agama resmi dari negara, menggabungan diri ke dalam Gereja merupakan suatu kebijakan politis.

    \begin{center}***\end{center}

    Penting untuk diingat bahwa doktrin tentang Sakramen Pembaptisan kemudian berkembangan seturut perkembangan jaman. Tidak terlalu mudah, misalnya, untuk menentukan apa yang harus dibuat dengan orang-orang yang melakukan dosa berat setelah pembaptisan atau dengan orang-orang yang menyangkap iman mereka, lalu kemudian bertobat lagi dan minta diterima lagi ke dalam komunitas umat beriman.

    Salah satu dari masalah-masalah itu adalah masalah peranan pembaptisan bayi. Para ahli Kitab Suci mengandaikan bahwa ketika "seluruh rumahtangga" dipermandikan, permandian itu termasuk anak-anak, bahkan yang paling kecil sekalipun (bayi). Tapi sekali lagi, oleh karena perkembangan refleksi iman/teologi, seperti penjelasan St. Agustinus tentang Dosa Asal pada abad V, yang akhirnya membuat permandian bayi menjadi amat populer dan dominan. Pada poin ini, Pembaptisan tidak lagi dilihat terutama sebagai awal dari kehidupan moral, tapi lebih ditekankan sebagai jaminan untuk diterima di dalam kerajaan surga setelah kematian.

    \begin{center}***\end{center}

    Pada awal Abad Pertengahan, ketika seluruh suku di Eropa utara bertobat dan seluruh suku (sering jumlahnya sampai ribuan) harus dibaptis secara serempak jikalau kepala suku atau raja mau masuk Kristen. Dalam keadaan seperti itu, sebuah ritus (tata upacara) yang lebih sederhana, praktis dan cepat, amat dibutuhkan. Sampai pada akhir abad VIII, upacara permandian yang sebelumnya panjang dan berlangsung selama berminggu-minggu telah dibuat sangat singkat. Anak-anak menerima upacara pengusiran roh jahat selama tiga kali pada minggu-minggu sebelum Paska dan Sabtu Suci. Setelah air pembaptisan dan bejana pembaptisan (bukan lagi kolam) diberkati, anak-anak kecil dicelupkan kepadanya ke dalam bejana air itu sampai tiga kali. Sesudah itu para imam mengurapi kepala mereka dengan minyak, uskup menumpangkan tangan ke atas mereka dan mengurapi mereka sekali lagi dengan minyak suci, dan mereka diberi Komuni Kudus dalam perayaan misa Kudus.

    Ritus kemudian terus dibuat semakin singkat ketika kebiasaan bayi menerima komuni suci pada waktu permandian dihapus oleh Konsili Trente pada tahun 1562.

    Dan karena Pembaptisan sekarang dilihat sebagai kunci untuk diterima dalam kerajaan surga, Gereja kemudian menawarkan sebuah ritus darurat yang pendek untuk bayi-bayi yang berada dalam bahaya kematian. Sebelum awal abad XI sejumlah uskup mengingatkan bahwa bayi kemungkian besar selalu berada dalam bahaya kematian yang tiba-tiba dan karena itu mereka mendorong para orangtua untuk tidak menunggu sampai perayaaan besar pada Hari Sabtu suci untuk mempermandikan bayi-bayi mereka.

    Sebelum abad XIV, perayaan pembaptisan pada hari Sabtu Paska benar-benar sudah hilang, kecuali upacara pemberkatan bejana dan air, dan ritus permandian yang lama dipersingkat lagi dan hanya dibuat sebagai upacara kecil waktu imam masuk Gedung Gereja.

    Sejak masa ini pembaptisan hanya disaksikan oleh anggota keluarga inti dan sejumlah kecil kaum kerabat, daripada disaksikan oleh seluruh komunitas umat beriman (seperti sebelumnya). Ketimbang mencelupkan bayi-bayi ke dalam kolam air, para imam hanya menuangkan sedikit air ke atas kepala anak-anak.

    Seiring dengan perjalanan sejarah, dan ritus pendek permandian yang semula disusun khusus hanya untuk bayi-bayi, yang berada dalam bahaya kematian, menjadi begitu universal, ritus permandian Gereja perdana (abad I sampai III) semakin lama semakin dilupakan. Tapi kemudian pada akhir tahun 1950-an para ahli sejarah Gereja mulai meneliti dan studi kembali mengenai ritus-ritus Gereja abad pertama. Hasilnya adalah bahwa pada tahun 1969, sebuah ritus Pembaptisan untuk anak-anak (bayi), yang telah direvisi, diterbitkan. Sama seperti ritus Gereja perdana, ritus yang disempurnakan ini menekankan aspek kommunal dari perayaan sakramen-sakramen. Upacara pembaptisan dianjurkan untuk dibuat dalam rangkaian perayaan misa (seperti sakramen perkawinan). Ritusnya diperpanjang juga. Sekarang orangtua diharapkan menghadiri pembinaan (pendalaman) iman setiap kali anak mereka mau dipermandikan. Dan penekanan teologis bergeser dari Pembaptisan sebagai "jaminan masuk surga" ke upacara permulaan masuk ke dalam kehidupan moral. Pada tahun 1980, sebuah dokumen dari Vatikan menegaskan bahwa jika orangtua tidak menjamin dan tidak mau memastikan bahwa anak (bayi) mereka akan dibesarkan dalam iman Katolik, maka Pambaptisan sebaiknya ditunda sampai tidak ada batas waktu.

    Setelah beberapa dekade berlalu, kita perlahan-lahan kembali kepada simbol-simbol, drama dan spiritutalitas komunal umat beriman yang merupakan ciri khas pembaptisan pada abad-abad pertama sejarah Gereja yang dirasa lebih efektif mengungkapkan makna permandian sebagai tanda kelahiran baru ke dalam Kerajaan Allah di dalam Yesus Kristus, sambil tetap mengakui keabsahan pembaptisan dengan ritus yang sederhana dan singkat. Karena biar bagaimanapun bentuk, panjang atau pendeknya ritus Sakramen Permandian, hakekatnya tetapi sama dan sah sebagai tanda kelahiran baru.

\sumber{Romo Alex Jebadu, SVD http://www.imankatolik.or.id}
