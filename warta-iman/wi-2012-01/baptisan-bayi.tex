\chap{Baptisan Bayi}

    Banyak sekali saudara/i kita dari Gereja Protestan yang tidak dapat menerima praktek babtisan bayi. Alasan yang sering diajukan antara lain: Babtisan memerlukan pertobatan dan iman (anak kecil dan bayi tidak bisa) juga yang sering juga diajukan adalah tidak adanya dasar alkitab bagi babtisan bayi.
        Perlu kita ketahui bahwa babtisan bayi lebih merupakan Tradisi Apostolik, dan kita ketahui bahwa dasar Iman Katolik tidak hanya Alkitab tetapi juga Tradisi Apostolik dan Magisterium. Jika kita ingin mencari babtisan bayi dalam kita suci hal itu sulit didapat karena dalam Kitab Suci tidak diungkapkan secara eksplisit mengenai babtisan bayi tetapi tidak ada larangan agar anak-anak (bayi) dibabtis. Kita tahu bahwa babtisan itu melahirbarukan dan menghapus dosa asal. Oleh karena itulah Gereja tidak melarang bayi dibabtis. Lalu bagaimana dengan iman anak? Jawaban yang mudah adalah bahwa perkembangan iman anak adalah tanggung jawab orang tua karena itu janji mereka ketika menikah untuk membesarkan anak-anak dalam iman katolik (tidak mungkin ada orang tua yang mau anaknya berbeda iman dengannya).
        Sekarang kita mencoba mereview Kitab Suci. dalam Kis 2:38-39 dikatakan "Jawab Petrus kepada mereka: 'Bertobatlah dan hendaklah kamu masing-masing memberi dirimu dibaptis dalam nama Yesus Kristus untuk pengampunan dosamu, maka kamu akan menerima karunia Roh Kudus. Sebab bagi kamulah janji itu dan bagi anak-anakmu dan bagi orang yang masih jauh, yaitu sebanyak yang akan dipanggil oleh Tuhan Allah kita.' " disini jelas sekali ungkapan Petrus bahwa kita perlu bertobat dan dibabtis yang akhirnya kita mendapat buah dari babtisan itu yaitu menerima Karunia Roh Kudus (ayat 38) dan janji itu berlaku pula untuk anak-anak (bayi juga termasuk anak-anak) (ayat 39) tentunya juga dengan melakukan hal yang sama yaitu dibabtis. Bila kita melihat dalam Perjanjian Lama dimana kita tahu bahwa bayi harus disunat (padahal mereka tidak tahu apa-apa soal iman) lihat pada Kej 17:12, Im 2:21, Luk 2:21 lalu pada Kolose 2:11-12 "Dalam Dia kamu telah disunat, bukan dengan sunat yang dilakukan oleh manusia, tetapi dengan sunat Kristus, yang terdiri dari penanggalan akan tubuh yang berdosa,  karena dengan Dia kamu dikuburkan dalam baptisan, dan di dalam Dia kamu turut dibangkitkan juga oleh kepercayaanmu kepada kerja kuasa Allah, yang telah membangkitkan Dia dari orang mati." disini jelas bahwa Paulus mempararelkan antara Sunat (Ayat 11) dengan Babtisan (ayat 11b-12) kita tahu bahwa hukum sunat berlaku juga untuk anak (bayi) berarti babtispun demikian. lalu dalam Kis16:15,33 dikatakan "ia dibaptis bersama-sama dengan seisi rumahnya" (ayat 15) dan "Seketika itu juga ia dan keluarganya memberi diri dibaptis." (ayat 33) dari kedua ayat ini tidak tertutup kemungkinan akan adanya bayi dan ikut dibabtis karena pada ayat itu maupun sebelum atau sesudahnya tidak ada kata "kecuali bayi atau anak-anak". Pada abad ke II sudah ditemukan babtisan bayi seperti St. Polikarpus, misalnya, dibunuh sebagai martir pada tahun 155 M. Pada saat penguasa Romawi memaksa Polikarpus untuk menyangkal Yesus Kristus dan untuk menyembah kaisar Roma, ia berseru demikian, "Delapan puluh enam tahun saya menjadi hamba-Nya, dan Ia tidak pernah berbuat yang tidak baik kepadaku, bagaimana mungkin saya dapat menghojat Rajaku yang telah menebusku?" Kesaksian ini berarti bahwa Polikarpus dibaptis sejak ia masih bayi atau kanak-kanak, yakni sekitar tahun 70-an. Hal ini tidak benar hanya jika Polikarpus sudah mencapai usia yang amat tinggi pada tahun 155 M itu, sehingga 86 tahun sebelumnya ia sudah dewasa dan baru dibaptis waktu itu.
 

\sumber{Thomas Rudy}
