\chap{Sebuah Hati untuk yang Terkecil dan Terpencil}
	Disaat aku bertugas meliput peresmian Gereja St. Maria Ratu Rosario di Pulau Bunyu,  yang terletak di bagian paling utara Keuskupan Samarinda, aku dikejutkan oleh seseorang yang memanggil namaku, padahal aku merasa tidak punya kenalan di Pulau Bunyu.
	
``Mas Agung Laksono… ???  Masih ingat sama saya?'' sapa `orang itu' sambil menjabat tanganku.

``Maaf, saya lupa \dots siapa ya??? Ujarku penuh kebingungan.

``Masih ingat dengan tattoo ini???'' kata orang itu sambil membuka kancing bajunya dan memperlihatkan tattoo ``tengkorak diatas dua tulang yang disilang'' didada kirinya.

``Barjo \dots ???!!!,'' teriakku kaget terbengong melihat sosok `orang itu' tersenyum sambil mengangguk-angguk.

``Apa kabarnya Mas, sudah hampir 25 tahun kita tidak pernah bertemu.'' Ujar Barjo sambil memeluk aku yang masih  ``bengong'' seakan tidak percaya masih bisa bertemu dengan Barjo Panjul ``si preman tengik'' , teman kecilku main layang-layang dan mandi di Kali Code ketika masih tinggal  di daerah Tukangan Yogya. 

Orang tua Barjo bercerai dan mereka meninggalkan Barjo hidup bersama dengan neneknya. Setelah neneknya meninggal Barjo berusia remaja sudah mulai mengenal dunia kejahatan dengan ``bimbingan'' para preman senior didaerah tempat tinggalku. Dari belajar mencopet di kawasan Malioboro, mencuri, merampok dan sebagainya.
 
Di usia remaja kami bertemu lagi, Barjo sudah malang melintang sebagai jagoan di kawasan Stasiun Tugu Yogyakarta, juga dikawasan Malioboro. ``Kalau kamu kecopetan di stasiun atau di Malioboro, panggil saja aku. Jangan lapor polisi. Nanti dompetmu pasti kembali.'' Pesannya.  Barjo dikenal pemabuk, tukang kepruk, pelindung para PSK (pekerja seks komersial), \textit{bodyguard} para boss, provokator bayaran, \textit{debt collector}, apa saja yang penting ada bayarannya . 

``Ba \dots ba \dots baik Mas Barjo, bagaimana Mas Barjo bisa sampai di Pulau Bunyu yang terpencil ini?'' tanyaku dengan penuh keheranan.

``Panjang ceritanya Mas Agung, mari kita cari tempat yang nyaman untuk bercerita.'' Ujar Barjo sambil mengajakku ke pendopo di samping gereja.

``Masih ingat saat Mas Agung menolong saya dari ancaman ``Petrus''?'' Tanya Barjo kepadaku.

	Aku terbayang kembali di saat tahun 1982-1983, Yogyakarta digemparkan adanya ``Petrus'' alias penembak misterius. Para jagoan alias preman yang selama ini meresahkan masyarakat banyak yang tumbang diterjang peluru ``petrus''.

	Disuatu malam aku pulang dari Gereja Antonius Kotabaru Yogyakarta berjalan kaki, ketika akan memasuki emplasemen stasiun kereta api Lempuyangan, tiba-tiba dari arah gerbong barang meloncat seseorang dan langsung menubrukku. Aku hampir menjerit, tetapi orang itu sudah membekap mulutku.

``Tolonglah aku. Sungguh, hanya kamu yang bisa kupercaya,'' kata orang itu sambil menyeretku masuk ke gerbong barang yang kosong. ``Aku Barjo Panjul! Masih ingat?''

``Bangsat!'' umpatku dalam hati. Nafasku terengah-engah.

``tolonglah aku. Terserah bangaimana caramu, yang penting aku bisa selamat. Aku tidak ingin mati seperti teman-temanku. Sungguh, aku janji, kalau aku selamat, aku akan bertobat. Tolonglah aku Gung!'' rintih Barjo Panjul.

``Kamu takut mati?'' tanyaku.

``Ya.'' 

``Mengapa?''

``Pokoknya aku tidak mau mati konyol. Di antara semua teman-temanku, hanya kamu yang paling baik padaku. Walaupun aku bejat, kamu masih mau berteman denganku, sedangkan yang lain muak dan takut padaku.''  

Di malam itu aku memikirkan mencari tempat yang aman bagi Barjo Panjul. Resikonya berat. Karena siapapun yang melindungi gali atau preman, bisa ikut disikat aparat. Saya tahu resiko itu. Tapi menyelamatkan nyawa seorang teman aku kira jauh lebih baik daripada hanya berpikir tentang resiko. Malam itu aku `menyulap' penampilan Barjo Panjul, rambutnya aku potong dan kumisnya aku cukur agar tidak mudah dikenali. Sebuah tattoo tengkorak diatas dua tulang yang disilang didada kiri badan Barjo tidak bisa aku hapus. Lalu aku `menyimpan'  Barjo di tempat aman.

Tiga hari kemudian, aku berusaha membujuk Barjo agar menyerahkan diri kepada aparat keamanan. Aku berusaha menjelaskan bahwa nanti kamu di sana akan diperlakukan dengan semestinya. Mulanya Barjo tidak mau. Namun akhirnya Barjo mau setelah aku beri penjelasan panjang lebar bahwa di sana pasti aman. ``Daripada mati sebagai buronan, lebih baik mati di kamar tahanan!'' hiburku. ``Karena nanti banyak yang akan membelamu jika hal itu terjadi.'' 

Aku mengantarkan Barjo ke Koramil (Komando Rayon Militer) terdekat. Sebelum dimasukkan ke dalam sel tahanan, aku memberikan secarik kertas berisi tulisan:

\begin{quote}
\textit{``Sendengkanlah telinga-Mu, ya Tuhan, jawablah aku, \\sebab sengsara dan miskin aku. \\
Peliharalah nyawaku, sebab aku orang yang Kau kasihi,\\ selamatkanlah  hamba-Mu yang percaya kepada-Mu. \\
Engkau adalah Allahku, kasihanilah aku ya Tuhan, \\sebab kepada-Mu-lah, ya Tuhan, aku berseru sepanjang hari.\\ Buatlah jiwa hamba-Mu bersukacita, \\sebab kepada-Mu-lah, ya Tuhan, kuangkat jiwaku. \\Sebab Engkau, ya Tuhan,\\ baik dan suka mengampuni dan berlimpah kasih setia bagi semua orang yang berseru kepada-Mu.''
}
\end{quote}

``Bacalah ini selama kamu berada di dalam sel tahanan. Semoga Tuhan senantiasa mendengar doa-doamu, pesanku. Barjo Panjul mengangguk-angguk. Dan dia tidak bisa menahan air matanya untuk jatuh. Lelaki sangar dan banyak ditakuti orang itu ternyata bisa menangis!.

Bagaimana nasib Barjo selanjutnya, saya tidak tahu karena tuntutan pekerjaan aku harus sering keluar kota .  Di pulau Bunyu yang terpencil inilah aku bertemu dengan Barjo lagi, Barjo sekarang tidak sesangar  seperti dahulu lagi, tapi penuh wibawa.

``Setelah 6 bulan aku di tahanan Koramil, aku di ``kirim'' di pulau Bunyu ini bersama dengan para anggota koramil untuk membangun pedesaan di sini selama 1 tahun. Aku dibebaskan dari hukuman dengan syarat aku harus membantu kehidupan penduduk disini selama 3 tahun,'' Barjo mulai bercerita.

``Rakyat di pulau Bunyu sangat miskin dan terbelakang, karena kebijakan pemerintah setempat tidak pernah membagi hasil tambang minyak dan methanol pulau ini kepada sekitar 12 ribu penduduk  di sini. Jiwaku memberontak akan ketidak adilan ini. Aku mengajak tokoh masyarakat dan para pemuka agama untuk bermusyawarah mencari solusinya. Hasilnya terbentuklah Badan Komunikasi Aspirasi Masyarakat Bunyu (BKAMB) yang mewakili rakyat pulau Bunyu ini untuk menegosiasikan dengan pemerintah daerah Balongan. Aku diberi amanah sebagai ketua umum.''

``BKAMB ini juga bertujuan mencarikan jalan untuk keberlangsungan usaha kelompok-kelompok petani, petambak, nelayan, pengusaha dan sebagainya dengan sistem Koperasi. Walau aku dulu seorang gali bajingan tengik, pengalaman aku mengelola para tukang parkir dan para preman di Malioboro yang merupakan daerah kekuasaanku dulu dengan sistem hampir mirip dengan koperasi, ternyata dapat berguna untuk aku terapkan disini. Demikian juga para pemuda penganggur disini kami rekrut melalui para tokoh agama (surau dan gereja) agar mereka semua bisa bekerja. '' 

``Hasilnya cukup nyata, negosiasi BKAMB dengan pemda Balongan memutuskan 10\% hasil eksploitasi tambang di Bunyu menjadi hak rakyat Bunyu. Dari sinilah masyarakat Bunyu bisa mendirikan sekolah, rumah sakit, pengaspalan jalan , pembangunan masjid juga gereja yang akan diresmikan ini.'' Jelas Barjo.

Tiba-tiba Barjo mengambil dompetnya di kantong belakang celananya dan mengambil secarik kertas dan memberikan padaku. ``Terima kasih atas pertolongan Mas Agung, kutipan Mazmur$^{*)}$ ini masih kusimpan dengan baik. Aku membacanya tiap malam. Hatiku kupersembahkan untuk yang terkecil dan terpencil disini. Selama 25 tahun hatiku terus bergumul akan hari pembaptisanku hari ini bersamaan dengan peresmian gereja ini. Ketidaklayakan dan ketidakpantasan diriku menjadi murid Kristus selalu membayangiku. Aku manusia bejat, memandang-Nya pun sungguh diriku tidak layak. Penyesalanku membuatku tidak mampu mengampuni diriku sendiri. Semoga Tuhan berkenan mengampuni diriku.''  Ujar Barjo perlahan dengan mata berkaca.

Ada seorang perempuan cantik setengah baya dengan menggendong bayi beserta  dua orang pemuda datang mendekati kami berdua di pendopo. Mereka tersenyum dan menganggukkan kepala kepadaku dan kubalas anggukkan dan senyuman mereka.
``Maaf, kami mengganggu sebentar. Ayo Pak, acara peresmian gereja sudah akan dimulai. Kami dari tadi bingung mencari keberadaan Bapak.'' Kata perempuan cantik itu.

``Bu serta anak-anak, kenalkan inilah Mas Agung Laksono, penolong  Bapak yang pernah Bapak ceritakan kepada kalian.'' Kata Barjo memperkenalkan diriku.

``Mas Agung, ini isteriku Yuniarsih, anakku yang pertama  Agung Wicaksana, anakku yang kedua Agung Satria Muda dan anak bayi perempuanku Kasih Setianingrum.  Maaf Mas, nama kedua anakku saya ambil dari nama panjenengan sebagai rasa ungkapan terima kasih dan penghormatan kami. Siapa sangka hari ini Tuhan masih berkenan mempertemukan kami dengan panjenengan.'' Kata Barjo

``Ah, Mas Barjo ini bisa saja. Bukan saya yang menolong Mas Barjo tetapi Tuhanlah yang berkehendak. Ayo kita ke gereja, misa sudah dimulai.'' Ujarku saat mulai  terdengar lagu pujian dari gereja.

Hari ini aku menyaksikan upacara pembaptisan Barjo dan peresmian gereja St. Maria Ratu Rosario di Pulau Bunyu Keuskupan Samarinda. Setelah acara selesai, Barjo sekeluarga memintaku menginap di rumahnya dan malam itu Barjo mengadakan pesta kecil juga mengundang warga disekitarnya. 

Setelah pesta usai sekitar jam 10 malam, aku dan Barjo dan keluarganya berkumpul. 

``Mas Agung, ada peristiwa yang membuat iman kami dikuatkan.'' Kata Barjo membuka pembicaraan.\\
Kami menikah pada tahun 1988, pada saat itu usia istriku 21 tahun. Pada tahun 2006 kemarin, istriku pada usia 39 tahun hamil lagi. Pada bulan Mei istriku  yang hamil tua jatuh sakit dan harus dibawa ke rumah sakit di Tarakan. Pada malam harinya anak dalam kandungan dinyatakan meninggal dunia oleh pihak rumah sakit. Beberapa jam kemudian istri saya juga dinyatakan meninggal juga. ``Tolong disiapkan segala keperluannya besok pagi,'' kata Barjo mengutip ucapan suster rumah sakit.

Dalam kepanikan saat itu, aku merasa tidak yakin bahwa istriku meninggal, aku syok, aku memasuki kamar operasi dan kulihat wajah istriku sudah ditutup dengan kain. Aku berlutut disamping jenazah istriku. ``Ya Tuhan, kenapa istriku dan anakku yang Kau ambil??? Akulah yang berdosa, ambillah aku!!! `` gugatku kepada Tuhan. Aku berdoa dengan cara yang aku yakini, baik dengan cara kejawen seperti yang pernah aku pelajari dari guru spiritualku diwaktu aku masih menjadi preman, yaitu untuk mengembalikan roh istri saya maupun segala macam doa lain. Entah berapa lama aku menatapi jenazah istriku seakan tidak percaya Tuhan begitu tega mengambil orang yang paling kusayangi. Tiba-tiba muncul sinar yang begitu terang seperti sinar matahari. Saking tidak kuatnya, aku ngumpet dibawah kolong tempat tidur istriku. Tidak seperti biasa seolah-olah ada suara yang membisikkan kepadaku bahwa untuk kesembuhan istriku `berdoalah seperti yang diajarkan oleh BapaMu'.  

Aku teringat saat anakku Agung Wicaksana saat kelas 4 SD. Dia membawa lilin dan Injil sambil berkata,'' Bu, saya mau belajar agama karena besok mau ulangan.'' Dia mengucapkan Doa Bapa Kami sampai dua kali. Saya ingat inilah doa yang harus saya ucapkan seperti dalam bisikan tadi. Akhirnya doa itu saya tirukan, belum sampai Amin, istri saya tiba-tiba bangun. Hidup kembali. Anak saya yang dalam kandungan pun keluar tanpa operasi atau bantuan orang lain. Seluruh perawat  dan dokter  di rumah sakit itu bingung dan terheran-heran setelah saya mengabarkan bahwa istri dan anakku masih hidup dan istriku sudah melahirkan. Sejak itulah saya mengatakan pada istri saya, 'Kamu hidup kembali melalui Doa Bapa Kami. Mulai hari ini kamulah orang pertama yang melihat diriku berubah'.

\vspace{0.5cm}
\noindent{\textit{*) Mazmur 86:1-5}}

\sumber{Medio Desember  '11\\
Bravo Sierra}

