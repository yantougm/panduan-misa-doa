\chap{Baptisan dan Hidup Baru}

Banyak orang Protestan mengatakan dan percaya bahwa baptis hanya sebuah simbol. Dalam Gereja Katolik baptisan tidak hanya sebagai simbol tetapi adalah sebuah sakramen. Baptisan (menurut Gereja Katolik) membuat kita lahir baru. Dasar kitab suci dari ajaran tentang baptis ini cukup banyak antara lain:

Yohanes 3:5 "Aku berkata kepadamu, sesungguhnya jika seorang tidak dilahirkan dari air dan Roh, ia tidak dapat masuk ke dalam Kerajaan Allah" pada ayat ini Yesus menekankan pentingnya baptis sebagai jalan untuk masuk dalam Kerajaan Allah.

Dalam Kis 2:38 St. Petrus mengatakan "Bertobatlah dan hendaklah kamu masing-masing memberi dirimu dibaptis dalam nama Yesus Kristus untuk pengampunan dosamu, maka kamu akan menerima karunia Roh Kudus." St. Petrus menekankan perlunya baptis untuk pengampunan dosa dan syarat untuk menerima karunia Roh Kudus.

St. Paulus dalam Titus 3:5 "Pada waktu itu Dia telah menyelamatkan kita, bukan karena perbuatan baik yang telah kita lakukan, tetapi karena rahmat-Nya oleh permandian kelahiran kembali dan oleh pembaharuan yang dikerjakan oleh Roh Kudus" lalu dalam Kis 22:16 "Dan sekarang, mengapa engkau masih ragu-ragu? Bangunlah, berilah dirimu dibaptis dan dosa-dosamu disucikan sambil berseru kepada nama Tuhan!"

Dari beberapa ayat diatas jelaslah bahwa Sakramen baptis bukan hanya sebuah lambang atau simbol (jikalau itu simbol untuk apa Para Rasul menekankan perlunya baptis?) tetapi baptisan memang membuat kita lahir baru, karena baptisan itu berhubungan erat sekali dengan Roh Kudus yang membuat kita lahir baru. Bila kita perhatikan Yohanes 3:5 "Aku berkata kepadamu, sesungguhnya jika seorang tidak dilahirkan dari air dan Roh, ia tidak dapat masuk ke dalam Kerajaan Allah" kata "air dan Roh" (Baptisan dan Roh Kudus) memiliki suatu hubungan erat yang tidak dapat dipisahkan. Hubungan yang erat antara baptisan dan Roh Kudus yang tak terpisahkan inilah yang membuat kita memperoleh hidup baru pada saat kita dibaptis. Karena hubungan yang erat antara Roh Kudus dan baptisan sehingga ketika Paulus berbicara mengenai baptisan ia tidak menyebut Roh Kudus "Atau tidak tahukah kamu, bahwa kita semua yang telah dibaptis dalam Kristus, telah dibaptis dalam kematian-Nya? Dengan demikian kita telah dikuburkan bersama-sama dengan Dia oleh baptisan dalam kematian, supaya, sama seperti Kristus telah dibangkitkan dari antara orang mati oleh kemuliaan Bapa, demikian juga kita akan hidup dalam hidup yang baru." (Roma 6:3-4).

Baptisan bukan perbuatan manusiawi belaka tetapi baptis adalah tanda dan sarana Rahmat Allah (yaitu kelahiran/hidup baru) dimana Allah berkarya melalui para pelayan (Imam, Diakon, dll) yang membaptis. Jadi baptisan adalah karya Allah sendiri yang mencurahkan Roh Kudus-Nya. Baptisan tidak dapat dibedakan/ dipisahkan dari Iman kepada Yesus dan dari Pencurahan Roh Kudus. Baptisan merupakan perwujudan iman seseorang kepada Yesus dan Iman itu berhubungan dengan pencurahan Roh Kudus lihatlah pada 1 Kor 12:3 "Karena itu aku mau meyakinkan kamu, bahwa tidak ada seorangpun yang berkata-kata oleh Roh Allah, dapat berkata: "Terkutuklah Yesus!" dan tidak ada seorangpun, yang dapat mengaku: "Yesus adalah Tuhan", selain oleh Roh Kudus."

Dari uraian diatas jelaslah bahwa baptis bukan hanya sebuah simbol tetapi benar-benar membuat kita lahir baru karena peranan dari Roh Kudus yang membuat kita lahir baru didalam pembaptisan. Oleh karena hal itulah St. Petrus menegaskan perlunya baptisan bagi keselamatan "Juga kamu sekarang diselamatkan oleh kiasannya (kiasannya=air bah {lihat ayat sebelumnya untuk lebih jelas}), yaitu baptisan--maksudnya bukan untuk membersihkan kenajisan jasmani, melainkan untuk memohonkan hati nurani yang baik kepada Allah--oleh kebangkitan Yesus Kristus" (1 Pet 3:21)

\section*{Baptis cara Selam}
Setelah berbicara banyak mengenai Hakekat baptis yang membuat kita lahir baru, Ada baiknya kita membahas mengenai masalah baptis selam. Baptis selam dalam gereja Pantekosta  dimutlakkan dan mereka terkadang (tidak semuanya) mengatakan baptisan selain cara selam tidak sah.

secara umum terkadang mereka mengajukan bukti dari Matius 3:16 "Yesus segera keluar dari air" kata "keluar dari air" menurut mereka berarti Yesus sebelumnya berada didalam air. Menurut kami kata "keluar dari air" tidak menunjukkan berapa banyak bagian tubuh yang terendam (yang menarik bahwa lukisan Kristen kuno tentang  pembaptisan Yesus pada Katakombe, dll pada jaman yang dekat dengan jaman para rasul digambarkan bahwa Yesus masuk ke air hanya sebatas lutut). 

Yang agak Rumit disini adalah pembahasan dari kata Babtizo (kata Yunani untuk membaptis) yang menurut mereka berarti "menenggelamkan sesuatu dalam air". Sebenarnya kata Babtizo memiliki beberapa arti yaitu "menenggelamkan" dan arti yang lain "mencelupkan". Ada hal yang menarik bahwa Lukas 11:38 "Orang Farisi itu melihat hal itu dan ia heran, karena Yesus tidak mencuci tangan-Nya sebelum makan" kata "Mencuci" dalam Lukas 11:38 dalam bahasa Yunani babtizo tetapi dalam hal ini tentu bukan kata menenggelamkan, tetapi mungkin hanya mencelupkan (dalam tradisi yahudi ada tempayan yang digunakan untuk pembasuhan sebelum besantap) tetapi rasanya tidak etis dan tidak higienis jika seseorang mencelupkan tangannya (yang kotor) kedalam tempayan itu (sementara tempayan itu digunakan untuk pembasuhan tidak hanya untuk satu orang) tentunya orang akan mengambil gayung dan mengambil air dari tempayan itu lalu mengucurkannya ke tangan. Jadi jelaslah penggunaan kata "babtizo" sangat fleksibel tidak hanya menenggelamkan, oleh karena itu tradisi baptis Kristen sangat fleksibel (tidak hanya dengan diselam saja) berikut kesaksian dari Dokumen 12 Rasul (berasal dari abad II M) mengatakan bahwa jika tidak ada air yang cukup untuk membaptis maka pembaptisan dengan pengucuran airpun adalah sah. Hal itu juga ditegaskan dalam dokumen Didache (sekitar tahun 100 Masehi) yang berisi hal yang sama dan juga pernyataan St. Agustinus, sekedar pengetahuan Bahwa gereja Protestan yang termasuk aliran utama (Lutheran, Calvinis) menggunakan cara baptis mirip seperti Katolik. Kita tahu kitab suci tidak memberikan petunjuk yang jelas dengan cara apakah Para Rasul membaptis (apakah dengan cara selam, dibasuh, atau dengan cara lain) tetapi kesaksian Yustinus Martir bahwa pembaptisan dilakukan dengan cara "masuk ke air" dan menurut banyak sejarah Gereja, pembaptisan dilakukan dengan cara menenggelamkan orang dan ini merupakan cara baptis pada Gereja perdana yang akhirnya berevolusi menjadi Ritus yang lebih sederhana (untuk lebih jelasnya lihat: Sakramen baptis dan sejarah Perubahan Ritusnya). Cara (Ritus) apapun yang digunakan untuk pembaptisan tetap tidak mengubah hakekat sakramen baptis, bahkan menurut informasi di beberapa Gereja Katolik di Amerika terdapat "kolam" untuk membaptis. 

\section*{Keselamatan tanpa baptis?}
Dalam Dokumen Konsili Vatikan II Lumen Gentium No 16 dikatakan bahwa "Mereka yang tanpa bersalah tidak mengenal Injil Kristus serta Gereja-Nya, tetapi dengan tulus hati mencari Allah, dan berkat pengaruh rahmat berusaha melaksanakan kehendak-Nya yang mereka kenal melalui suara hati dengan perbuatan nyata, dapat memperoleh keselamatan kekal". Sekilas ajaran itu bertentangan dengan 1 Tim 2:5 "Karena Allah itu esa dan esa pula Dia yang menjadi pengantara antara Allah dan manusia, yaitu manusia Kristus Yesus" ajaran Konsili Vatikan II menegaskan bahwa mereka "yang tanpa bersalah tidak mengenal Injil Kristus" bisa selamat didasarkan karena Yesus menjadi tebusan bagi semua orang (lihat Mat 20:28; Mrk 10:45; 1Tim 2:6). Ajaran Konsili Vatikan II juga tidak bertentangan dengan Mrk 16:15 dan Yoh 3:18 karena menurut pendapat kami (yang bisa saja salah) Mrk 16:15 dan Yoh 3:18 tidak perlu ditafsirkan secara harafiah dalam arti yang ketat kedua  ayat itu menekankan tentang perlunya iman dan baptisan agar orang dapat selamat, namun bagi "yang tanpa bersalah tidak mengenal Injil Kristus" masakah mereka juga harus dihukum? Ajaran Konsili Vatikan II ini tidak mengakui bahwa semua agama itu sama, Gereja merasa perlu untuk mewartakan Injil dan kita wajib memperkenalkan Kristus yang adalah jalan kebenaran dan Kehidupan dan Gereja sendiri memiliki kepenuhan sarana-sarana keselamatan (lihat Redemptoris Missio 55, Ensiklik Evangelii Nuntiandi, dokumen Konsili Vatikan II Digitatis Humanae 14, Ad Gentes 6 dan 7, dll) oleh karena itulah Gereja tidak pernah melupakan Perutusan agung yang diberikan Yesus "Karena itu pergilah, jadikanlah semua bangsa murid-Ku dan baptislah mereka dalam nama Bapa dan Anak dan Roh Kudus, dan ajarlah mereka melakukan segala sesuatu yang telah Kuperintahkan kepadamu. Dan ketahuilah, Aku menyertai kamu senantiasa sampai kepada akhir zaman." (M at 28:19-20) dengan tujuan memperkenalkan Kristus yang adalah jalan yang pasti dan singkat menuju Rumah Bapa.

\sumber{Thomas Rudy http://www.imankatolik.or.id}