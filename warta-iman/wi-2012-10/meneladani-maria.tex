\chap{Meneladani Maria}

“Salam, hai engkau yang dikaruniai, Tuhan menyertai engkau”  Maria adalah seorang gadis desa, yang sederhana dan dipilih Allah untuk menunjukan karya kemuliaan dan kerahiman Allah kepada manusia. Mengapa Allah memilih Maria? Hanya Allah yang tahu. Tetapi dari telaah Kitab Suci: Allah selalu memilih/memakai yang lemah/sederhana/tersisih untuk melakukan karya-karya besar supaya hanya kemuliaan Tuhan yang nampak. Tuhan juga memilih kita orang berdosa untuk diselamatkanNya.


Maria sebagai manusia biasa juga ber-reaksi kaget/ terkejut/ ragu-ragu/ takut, pada mulanya tidak percaya ketika mendapat kinjungan dari Malaikat / utusan Tuhan sendiri, tetapi maria tidak berhenti dengan ketakutan dan kerahu-raguannya. Maria berusaha mencari jawaban dari si pembawa berita yaitu Malaikat Tuhan yang bernama Malaikat Agung Gabriel. Apakah kita pun mencari jawaban dari Tuhan ketika mengalami keraguan/bimbang dalam memutuskan sesuatu ?

“Tidak ada yang mustahil bagi Allah”  Secara manusia hamil tanpa berhubungan sex adalah tidak mungkin. Ini adalah karya Roh Kudus yang ingin menunjukan bahwa Allah sanggup melakukan segala sesuatu. Kita harus melihat ini dengan iman. Iman berarti kepercayaan dan penyerahan diri secara total kepada kehendak Allah. Maria adalah Musa baru dalam teladan iman dan penyerahan diri kepada Allah (Bdk. Abraham siap mengorbankan anak tunggalnya karena perintah Allah - Kej 22:1-11)

Maria menjadi teladan setiap orang beriman, terutama orang katolik, dalam iman dan kepercayaannya kepada Allah. Pengabdiaan Maria kepada Allah terungkap lewat sikap yang siap menerima tugas dan perintah Allah sekalipun akan berakibat penghinaan, penolakan dan bahaya kematian dari masyarakat sekitarnya pada waktu itu. “Sesungguhnya aku ini adalah hamba Tuhan; jadilah padaku menurut perkataanmu itu” (Lukas 1:38). Rencana penyelamatan Allah tidak akan terlaksana jika Maria tidak mengatakan “YA” kepada kehendak Allah. Kita diselamatkan oleh Yesus yang ada di dunia ini karena jawaban “YA” (FIAT) Maria itu.

“Malaikat itu meninggalkan Maria”  Allah membiarkan dan membebaskan Maria untuk melaksanakan tugas yang telah diutuskan kepadanya. Maria di hadapkan pada ketidakpastian dan ketidakjelasan panggilan Allah yang penuh dengan resiko. Kita pun diberi kebebasan oleh Allah untuk melakukan segala sesuatu sesuai dengan kehendakNYA atau kehendakku dengan segala resiko masing-masing. Apakah kita selalu memilih yang baik dan benar apapun resikonya?

Sama seperti Maria, Allah pun memilih kita bukan karena kita pandai/ kaya/ hebat/ cantik, tapi Allah selalu melihat hati; ketulusan/ ketaatan/ kerendahan hati/ kepasrahan kepada Allah yang ditunjukan Maria itulah yang selalu menjadi teladan bagi orang katolik. Mengapa kita yang dipilih? Hanya Tuhan yang tahu. Itulah tanda cinta dan kasih setia Tuhan pada kita manusia berdosa. Sudahkah kita bersyukur untuk itu?

Dengan iman dan penyerahan diri Maria kepada Allah, selain menjadi teladan kita, Maria, karena kedekatannya dengan Allah juga menjadi perantara orang katolik untuk menyampaikan segala permohanan kita kepada Yesus. Yesus dan Maria menjadi Adam dan Hawa baru untuk menebus dosa para leluhur kita di taman Firdaus. Yesus adalah sumber Rahmat dan Maria adalah perantara rahmat yang membuat kita lahir kembali dan mendapat tempat di surga.

Pribadi Maria dalam kehidupan orang katolik mendapat tempat yang istimewa. Penghormatan (devosi) Gereja Katolik kepada Maria dilakukan dengan beberapa hal, misalnya memperingati hari-hari penting Bunda Maria, berdoa Salam Maria dan Rosario, Novena Tiga Salam Maria, puji-pujian kepada Maria dan berziarah ke Gua Maria. Gereja menetapkan bulan Mei dan Oktober sebagai bulan penghormatan (devosi) kepada Bunda Maria. Dalam bulan-bulan ini akan banyak kegiatan di Gereja dan lingkungan/wilayah untuk menghormati Bunda Maria.

Doa Salam Maria bersumber dari ucapan salam dari Malaikat Utusan Allah kepada Maria (Salam Maria, penuh Rahmat, Tuhan besertamu) dan ungkapan Elisabeth ketika bertemu Maria (Terpujilah Engkau antara wanita dan terpujilah buah Tubuhmu: Yesus).

Keberadaan gua maria, patung maria bukan untuk menggeser atau menyamakan posisi Yesus. Patung atau gambar bukan dewa yang disembah. Orang katolik tidak menyembah berhala, patung dll. Itu hanyalah sarana untuk lebih mudah berkonsentrasi dalam berdoa, seperti halnya kita melihat foto. Tujuan utama adalah satu yaitu Tuhan Yesus sendiri. Maria adalah pengantara untuk menyampaikan segala permohonan, keinginan, menyatukan, mendekatkan kita kepada Putranya Yesus.