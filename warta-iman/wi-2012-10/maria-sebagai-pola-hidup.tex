\chap{Maria Sebagai Pola Hidup Orang Beriman\\
 \textit{P. Josep Susanto Pr}}

\section*{Inkarnasi}

\small
Misteri inkarnasi (natal) bukan cuma soal Sabda yang menjadi daging (manusia) atau  Allah yang mendatangi manusia untuk menebus umat manusia, melainkan misteri itu bisa kita lihat dan renungkan lebih dalam lagi yaitu Allah yang berinisiatif untuk bersatu dengan ciptaanNya secara definitif supaya ciptaanNya bersatu dengan Dia. Di sinilah terjadi peristiwa iman-wahyu : Proses Membuka diri (bdk. Yoh 1:1)
\begin{quote}
\textit{Pada mulanya adalah Firman; Firman itu bersama-sama dengan Allah dan Firman itu adalah Allah}
\end{quote}

Dalam peristiwa inkarnasi, Kristus, Sang Putra Allah, menjadi senasib dengan manusia, membuat ke-AllahanNya yang tidak terbatas menjadi seolah-olah terbelanggu dan berdiam dalam kemanusiaanNya, dengan segala kelemahan dan ketebatasanNya. Dalam hal ini Santo Agustinus mengatakan : Dia yang adalah Allah Putra, mengosongkan diriNya, dan mengambil rupa seorang hamba, namun Ia tidak kehilangan ke-AllahanNya.
	
Inkarnasi, Allah yang menjadi manusia, adalah suatu rencara besar Allah bagi manusia (Luk 2:10-14). Rencana ini tidak pernah bisa dilepaskan dari sejarah keselamatan manusia sejak manusia pertama kali jatuh ke dalam dosa.
(bdk. Ibr 1:1-2).
\begin{quote}
\textit{Setelah pada jaman dahulu Allah berulang kali dan dalam pelbagai cara berbicara kepada nenek moyang kita dengan perantaraan nabi-nabi, maka pada zaman akhir ini Ia telah berbicara kepada kita dengan perantaraan AnakNya, yang telah Ia tetapkan sebagai yang berhak menerima segala yang ada.}
\end{quote}

\section*{Maria diikutsertakan dalam Rencana Allah}

Setiap manusia dilibatkan oleh Allah dalam rencana keselamatan. Seperti kita ketahui cerita dan riwayat para nabi (Yesaya, Yeremia, Samson, Daud, Ishkak, dll). Dalam hal ini Maria mendapatkan peran yang sangat penting demi terjadinya kehendak Allah tersebut. Maria dipilih oleh Allah menjadi ibu Mesias.

Selama ini yang sering kita dengar dan renungkan mungkin hanya Maria akan mengandung seorang bayi, dan bayi itu anak Allah. Tetapi kalau kita memahaminya dari sudut pandang orang Yahudi tentunya tugas ini menjadi dua kali lebih berat dari pada yang kita bayangkan. Orang Yahudi sangat kental tradisi keagamaanNya. Orang Yahudi pada jaman Yesus bahkan sampai sekarang, menanti-nantikan hadirnya seorang Mesias, Juru Selamat. Sebagaimana janji Allah kepada Daud, Raja Israel, Allah akan membangkitkan seseorang dari keturuan Daud (dibaca : akan lahir dari keturunan Daud), dan akan menjadi raja bagi bangsa Israel, dan kerajaanNya akan jaya selamanya dan tidak akan pernah berakhir.

Sebagai seorang Yahudi yang saleh, Maria tentunya juga hidup dalam pengharapan yang sama dengan harapan orang-orang sebangsanya. Dalam peristiwa Maria menerima kabar Gembira dari Malaikat sebenarnya Maria mengalami dua keterkejutan sekaligus kebahagian.
\normalsize

Keterkejutan yang pertama adalah ternyata Mesias seorang tokoh yang sudah ditunggu-tunggu oleh seluruh bangsanya akan muncul. Harapan besar itu segera terwujud. Segala penderitaan Israel karena penjajahan dan morat marit pemerintahan pada waktu itu akan segera berakhir dengan hadirnya seorang mesias.

Keterkejutan yang kedua adalah ternyata mesias ini akan hadir ke dunia melalui rahimnya sendiri. Seperti yang sudah sering kita dengar, hal ini tentu tidak mudah bagi Maria yang nota bene masih sangat muda waktu itu.

Sisi yang kita mau kupas dalam kesempatan ini adalah pergulatan Maria ketika ia tahu bahwa yang dikandungnya adalah seorang tokoh besar yang sudah dinanti-nantikan dan diharap-harap oleh banyak orang. Orang Israel mengira Mesias akan lahir dari rahim seorang perempuan dari kalangan kerajaan, yang jelas-jelas keturunan Raja Daud. Sementara Maria hanya perempuan biasa. Bisa kita bayangkan sejak semula Maria sudah sadar akan penolakan bangsaNya terhadap Putranya ini. Akan terjadi tegangan antara pengharapan orang-orang sebangsanya dengan kenyataan apa yang dikehendaki oleh Yesus. Dan sepertinya penolakan terhadap Yesus Sebagai Sang mesias sungguh-sungguh terjadi.

Seperti kita tahu mendidik anak bukanlah pekerjaan yang mudah. Anak adalah titipan Tuhan di mana seorang ibu mempunyai tugas untuk mendidik dan membesarkannya dengan penuh kasih sayang dan pengorbanan yang tidak sedikit.

Dalam kasus Maria, hal ini menjadi sangat tidak mudah. Bisa dibayangkan bagimana sikap dan kelakuan Yesus setelah besar kalau Maria salah mendidiknya. Tugas Maria tidak berhenti ketika ia mengandung dan melahirkan Yesus. Tugas Maria seperti dikatakan dalam Injil adalah mendidik Yesus, memperkenalkan Yesus pada bait Allah, memperkenalkan Yesus pada tradisi bangsanya, mendampingi Yesus, bahkan sampai menemani di kaki Salib.

Senjata Maria adalah Doa. Menghadapi tugas yang tidak mudah itu Maria mempunyai kekuatan yaitu kesadaran bahwa Allah pasti akan mendampingi dia dalam segala hal. Doa Maria sudah sering kita doakan seperti Ibadat Sore yaitu dalam Kidung Maria.
(Luk 1:46-55) Dalam Kidung Maria sebetulnya terkandung iman Maria yang sangat dalam, yaitu Allah yang selalu setia pada perjanjianNya. Hidup kita ini didasari oleh perjanjian antara Allah dan manusia, di mana Allah akan memberikan berkat melimpah kepada manusia yang senantiasa percaya dan berpegang teguh padaNya. Dan dalam perjanjian itu Allah selalu setia sedangkan manusia seringkali lupa dan tidak setia dengan perjanjian tersebut. Namun Allah sebagai pihak yang dirugikan berkali-kali memperbaharui perjanjianNya dengan manusia dengan harapan manusia akan berubah. Di atas ketidaksetiaan manusia, Allah tetap setia.

\section*{Relevansi kita sekarang:}

Sebagai biarawati ataupun imam, kita sering mendapat penugasan-penugasan dari para pemimpin kita. Mungkin beberapa penugasan adalah hal yang tidak mudah ataupun yang tidak pernah kita bayangkan sebelumnya. Bahkan dalam pengalaman ada penugasan-penugasan yang sepertinya dalam hati kita mau nya kita tolak entah karena kita tidak suka, tugas itu terlalu berat, atau pun alasan lainnya.

Dalam situasi ini kita mungkin bisa mencontoh pengalaman Maria, yaitu melihat tugas itu dalam kacamata seluruh sejarah keselamatan Allah bagi manusia. Artinya setiap manusia menjadi rekan kerja Allah dalam mewujudkan keselamatan bagi dirinya dan sesamanya. Mungkin mendengar hal itu kita menjadi minder dan kecil hati dengan bertanya : Siapakah diri kita, kok bisa menyelamatkan seluruh dunia?

Dalam hal ini kita perlu ingat, Kerajaan Allah justru bermula dari hal-hal kecil dan sederhana yang hampir tidak diperhatikan dan diperhitungkan orang lain. Lagi-lagi Maria bisa menjadi contoh bagi diri kita untuk berani mengatakan :  \textbf{“Terjadilah padaku menurut kehendakMu.”}

\sumber{http://www.imankatolik.or.id}