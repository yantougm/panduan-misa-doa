\newpage
\chap{Kompendium Katekese Gereja Katolik}
\setcounter{kgkcounter}{57}
\small
\kgk{Mengapa Allah mengizinkan kejahatan ada?}
      Iman memberikan kepastian kepada kita bahwa Allah tidak akan mengizinkan       
kejahatan jika Dia tidak menyebabkan suatu kebaikan yang datang dari kejahatan       
itu. Hal ini dilaksanakan oleh Allah dengan cara yang menakjubkan dalam wafat
dan kebangkitan Kristus. Kenyataannya, dari kejahatan moral yang paling besar dari
semuanya (pembunuhan Putra-Nya), Dia membawa kebaikan yang paling besar
dari semuanya (kemuliaan Kristus dan penebusan kita).

                                    \section*{Surga dan Bumi}

\kgk{Apa yang diciptakan Allah?}
     Kitab Suci mengatakan, ”Pada awal mula, Allah menciptakan langit dan bumi”      
(Kej 1:1). Gereja dalam pengakuan imannya menyatakan bahwa Allah adalah
Pencipta segala sesuatu, yang kelihatan dan tak kelihatan, semua makhluk spiritual
dan yang bertubuh, yaitu para malaikat dan dunia yang kelihatan, khususnya,
manusia.

\kgk{Siapa para malaikat itu?}
Malaikat-Malaikat adalah makhluk murni spiritual, bukan makhluk bertubuh,
tak kelihatan, tak dapat mati, dan berpribadi, dianugerahi akal dan kehendak.

          Mereka mengontemplasikan dan bertatap muka dengan Allah terus-menerus, dan
          mereka memuliakan-Nya. Mereka mengabdi-Nya dan menjadi pembawa pesan
          dalam melaksanakan misi penyelamatan-Nya bagi semua.

\kgk{Dengan cara bagaimana para malaikat hadir dalam kehidupan Gereja?}
Gereja bergabung dengan para malaikat dalam menyembah Allah, meminta
pertolongan mereka dan memperingati mereka dalam liturgi.


\flushright{(\dots \emph{bersambung} \dots)}
\normalsize