\chap{Kuasa Doa Satu Salam Maria}

\begin{quote}
\textit{Salam Maria, penuh rahmat, Tuhan sertamu,
terpujilah engkau di antara wanita, dan terpujilah buah tubuhmu, Yesus.
\\{~}\\
Santa Maria, Bunda Allah, doakanlah kami yang berdosa ini,
sekarang dan waktu kami mati. Amin.}
\end{quote}

Jutaan umat Katolik biasa mendaraskan Salam Maria. Sebagian mendaraskannya dengan begitu cepat, bahkan tanpa memikirkan kata-kata yang mereka ucapkan. Pernyataan-pernyataan berikut ini semoga dapat membantu kita mendaraskannya dengan lebih khusuk.
 
Satu Salam Maria yang didaraskan dengan baik memenuhi hati Bunda Maria dengan sukacita dan memperolehkan bagi diri kita sendiri rahmat-rahmat luar biasa yang tak terkatakan, yang ingin dilimpahkan Bunda Maria kepada kita. Satu Salam Maria yang didaraskan dengan baik memperolehkan bagi kita jauh lebih banyak rahmat daripada seribu Salam Maria yang didaraskan secara asal.
 
Doa Salam Maria bagaikan suatu tambang emas di mana kita senantiasa dapat menggali darinya tanpa ia pernah menjadi habis. Sulitkah mendaraskan Salam Maria dengan baik? Yang kita perlukan hanyalah belajar memahami nilai dan artinya.
 
St. Hieronimus mengatakan bahwa “kebenaran yang terkandung dalam Salam Maria begitu agung dan luhur, begitu mengagumkan, hingga tak ada manusia atau pun malaikat yang dapat memahami sepenuhnya.”
 
St. Thomas Aquinas, Pujangga Gereja yang terkemuka, 'yang paling bijaksana di antara para kudus dan yang paling kudus di antara para bijaksana', seperti dinyatakan oleh Paus Leo XIII, berkhotbah selama 40 hari lamanya di Roma hanya tentang Salam Maria, membuat para pendengarnya terpesona serta penuh sukacita.
 
Pastor F. Suarez, seorang imam Yesuit yang terpelajar dan kudus, ketika sedang menghadapi ajal menyatakan bahwa dengan senang hati ia akan menyerahkan seluruh dari banyak buku berbobot yang ia tulis, juga seluruh karya sepanjang hidupnya, demi mendapatkan ganjaran dan jasa dari satu doa Salam Maria yang didaraskan dengan khusuk dan tulus.


St. Mechtilda, yang sangat mengasihi Bunda Maria, suatu hari sedang berusaha keras untuk menggubah sebuah doa yang indah untuk menghormati Bunda Maria. Bunda Maria menampakkan diri kepadanya, dengan tulisan emas di dadanya: “Salam Maria penuh rahmat.” Santa Perawan berkata kepadanya, “Berhentilah, anakku terkasih, dari usahamu itu, oleh sebab tidaklah mungkin engkau dapat menggubah suatu doa yang dapat memberiku sukacita dan kebahagiaan seperti Salam Maria.”


Seorang pria memperoleh sukacita luar biasa dengan mendaraskan Salam Maria secara perlahan-lahan. Santa Perawan menampakkan diri kepadanya dengan tersenyum dan mengatakan kepadanya hari serta jam bilamana ia akan meninggal, serta memperolehkan baginya kematian yang paling kudus dan bahagia. Setelah kematiannya, sekuntum bunga bakung putih yang indah tumbuh dari mulutnya. Pada daun-daun bunganya tertulis “Salam Maria”.
 
Cesarius menceritakan kisah serupa. Seorang biarawan yang rendah hati dan kudus tinggal di sebuah biara. Daya tangkap dan daya ingatnya begitu lemah hingga ia hanya dapat menghafalkan satu doa saja, yaitu “Salam Maria”. Setelah kematiannya, sebatang pohon tumbuh di atas kuburnya dan pada semua daun-daunnya tertulis: “Salam Maria”.
 
Kisah-kisah indah berikut ini menunjukkan kepada kita betapa tinggi nilai devosi kepada Bunda Maria dan betapa besar kuasa doa Salam Maria yang didaraskan dengan khusuk.
 
Setiap kali kita mengucapkan Salam Maria, kita mengulangi kata-kata yang sama yang diucapkan Malaikat Agung St. Gabriel pada waktu menyampaikan salam kepada Maria pada Hari Kabar Sukacita, yaitu ketika ia menjadi Bunda Putra Allah.
 
Begitu banyak rahmat dan sukacita yang memenuhi jiwa Maria saat itu.
 
Sekarang, pada saat kita mendaraskan Salam Maria, kita mempersembahkan sekali lagi segala rahmat dan sukacita tersebut kepada Bunda Maria dan ia menerimanya dengan bahagia yang mendalam.
 
Sebagai balasnya, ia membagikan sukacitanya itu kepada kita.
 
Suatu ketika, Yesus meminta St. Fransiskus Asisi untuk memberi-Nya sesuatu. Orang kudus itu menjawab, “Tuhan terkasih, aku tak dapat memberi-Mu apa-apa lagi, sebab aku telah memberikan segalanya untuk-Mu, yaitu segenap cintaku.”
 
Yesus tersenyum dan berkata, “Fransiskus, berikan pada-Ku segenap cintamu itu lagi dan lagi, setiap kali, cintamu itu mendatangkan kesukaan yang sama bagi-Ku.”
 
Demikian juga dengan Bunda kita terkasih. Setiap kali kita mendaraskan Salam Maria, Bunda Maria menerima dari kita segala sukacita dan kebahagiaan yang sama seperti yang ia terima dari perkataan St. Gabriel.
 
Allah yang Mahakuasa telah menganugerahkan kepada Bunda-Nya yang Terberkati segala kemuliaan, keagungan, dan kekudusan yang diperlukan untuk menjadikannya Bunda-Nya Sendiri yang paling sempurna.
 
Namun demikian, Ia juga menganugerahkan kepada Bunda-Nya segala pesona, cinta, kelemah-lembutan serta kasih sayang yang diperlukan untuk menjadikannya Bunda kita yang paling terkasih. Bunda Maria adalah sungguh-sungguh dan benar-benar Bunda kita.
 
Seperti anak-anak lari kepada ibunya ketika menghadapi bahaya untuk minta perlindungan, demikian juga patutlah kita lari segera dengan keyakinan tak terbatas kepada Maria.
 
St. Bernardus dan banyak para kudus lainnya mengatakan bahwa tak pernah sekali pun terdengar pernah terjadi di suatu waktu atau pun tempat bahwa Bunda Maria menolak mendengarkan doa anak-anaknya yang di bumi.
 
Mengapakah kita tidak menyadari kebenaran yang sangat menghibur hati kita ini? Mengapakah kita menolak cinta dan penghiburan yang ditawarkan oleh Bunda Allah yang Manis kepada kita?
 
Adakah sikap acuh kita yang mengerikan, yang menjauhkan kita dari pertolongan dan penghiburan yang sedemikian itu?
 
Mengasihi dan mengandalkan Maria berarti berbahagia di dunia sekarang ini dan berbahagia kelak di Surga.


Dr. Hugh Lammer adalah seorang Protestan fanatik, dengan prasangka-prasangka kuat menentang Gereja Katolik. Suatu hari ia menemukan suatu penjelasan tentang Salam Maria dan membacanya. Ia begitu terpesona olehnya hingga mulai mendaraskannya setiap hari. Tanpa disadarinya, segala antipati dan kebenciannya terhadap Gereja Katolik mulai lenyap. Ia menjadi seorang Katolik, seorang imam yang kudus dan profesor Teologi Katolik di Breslau.


Seorang imam diminta datang ke sisi pembaringan seorang yang sedang menghadapi ajal dalam keputusasaan oleh karena dosa-dosanya. Namun demikian, orang itu bersikukuh menolak mengakukan dosa-dosanya. Sebagai usahanya yang terakhir, imam meminta si sakit agar setidak-tidaknya ia mendaraskan Salam Maria. Sesudah mendoakan Salam Maria, pria malang itu mengakukan dosanya dengan tulus dan meninggal dengan kudus.
 
Di Inggris, seorang imam paroki diminta untuk pergi menemui seorang wanita Protestan yang sedang sakit parah dan rindu menjadi seorang Katolik. Ketika ditanya apakah ia pernah pergi ke Gereja Katolik, atau apakah ia pernah belajar dari umat Katolik, atau apakah ia membaca buku-buku Katolik, ia menjawab, “Tidak, tidak pernah.” Sejauh yang dapat diingatnya ialah - ketika masih kanak-kanak - ia belajar dari seorang gadis kecil tetangga yang Katolik doa Salam Maria, yang kemudian dilakukannya setiap malam. Wanita itu kemudian dibaptis dan sebelum meninggal boleh menikmati kebahagian menyaksikan suami dan anak-anaknya dibaptis juga.


St. Gertrude mengatakan dalam bukunya, “Wahyu” bahwa ketika kita mengucap syukur kepada Tuhan atas rahmat-rahmat yang Ia berikan kepada seorang kudus tertentu, kita juga memperoleh bagian besar atas rahmat-rahmat tersebut.


\small
Jika demikian, rahmat-rahmat apakah yang tidak akan kita peroleh jika kita mendaraskan Salam Maria sementara kita mengucap syukur kepada-Nya atas segala rahmat tak terkatakan yang telah Ia anugerahkan kepada Bunda-Nya Maria?


\sumber{With Ecclesiastical Approval}
 
“. . . Satu Ave Maria (Salam Maria) yang didaraskan tanpa perasaan mendalam, tetapi dengan kehendak yang tulus dalam masa kekeringan, jauh lebih bernilai di hadapanku daripada satu Rosario penuh yang didaraskan di tengah penghiburan.”
 
\sumber{Bunda Maria kepada Sr. Benigna Consolata Ferrero}

 
“Seorang imam Yesuit yang kudus dan terpelajar, Pastor Suarez, memahami dengan begitu mendalam nilai Salam Malaikat (Salam Maria) hingga ia mengatakan bahwa ia akan dengan senang hati menyerahkan segala ilmu yang diperolehnya demi memperoleh ganjaran dan jasa satu Salam Maria yang didaraskan dengan pantas.”
 
\sumber{St. Louis De Montfort, Rahasia Rosario, hal. 48}
\normalsize