\newpage
\chap{Kompendium Katekese Gereja Katolik}
\setcounter{kgkcounter}{29}
{\normalsize
\section*{KAMI PERCAYA}

\kgk{Mengapa iman itu tindakan pribadi dan sekaligus gerejawi?}
     Iman adalah tindakan pribadi sejauh menjadi jawaban bebas pribadi manusiawi kepada Allah yang mewahyukan Diri-Nya. Tetapi, sekaligus merupakan tindakan gerejawi yang mengungkapkan dirinya dalam pengakuan ``Kami percaya''.
Kenyataannya, Gerejalah yang percaya, dan dengan rahmat Roh Kudus, Gereja
mendahului, memunculkan, dan memperkembangkan iman setiap orang Kristen.
Karena alasan inilah Gereja adalah Bunda dan Guru.

\kgk{Mengapa rumusan iman itu penting?}
     Rumusan iman itu penting karena dengannya orang beriman dapat mengungkapkan, menghayati, merayakan, dan saling berbagi kebenaran-kebenaran iman
bersama dengan orang beriman lainnya melalui satu bahasa yang sama.

\kgk{Mengapa iman Gereja itu hanya satu?}
       Gereja, walaupun terdiri dari banyak orang dari macam-macam bahasa,            budaya, dan ritus, mengakui satu iman dalam kesatuan suara; iman yang diterima        dari satu Allah dan diwariskan oleh satu Tradisi Apostolik. Gereja hanya mengakui
satu Allah, Bapa, Putra, dan Roh Kudus, dan menunjuk pada satu jalan keselamatan.
Karena itu, kita percaya dengan satu hati dan satu jiwa semua yang terdapat dalam
Sabda Allah, diwariskan langsung atau ditulis, dan diakui oleh Gereja sebagai wahyu
ilahi.

\kgk{Apa simbol-simbol iman itu?}
     Simbol-simbol iman adalah rumusan-rumusan yang diformulasikan, disebut 
juga ”pengakuan iman” atau ”syahadat”. Sejak awal mula berdirinya, Gereja merumuskan pengakuan iman ini secara sintetis dan mewariskannya dalam bahasa
yang normatif dan umum bagi semua umat beriman.


\flushright{(\dots \emph{bersambung} \dots)}
}