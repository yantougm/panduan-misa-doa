\chap{Mengapa Disebut ``Rabu Abu''?}

Rabu Abu adalah hari pertama Masa Prapaska, yang menandai bahwa kita memasuki masa tobat 40 hari sebelum Paska. Angka ``40″ selalu mempunyai makna rohani sebagai lamanya persiapan. Misalnya, Musa berpuasa 40 hari lamanya sebelum menerima Sepuluh Perintah Allah (lih. Kel 34:28), demikian pula Nabi Elia (lih. 1 raj 19:8). Tuhan Yesus sendiri juga berpuasa selama 40 hari 40 malam di padang gurun sebelum memulai pewartaan-Nya (lih. Mat 4:2).

\section{Mengapa hari Rabu?}

Nah, Gereja Katolik menerapkan puasa ini selama 6 hari dalam seminggu (hari Minggu tidak dihitung, karena hari Minggu dianggap sebagai peringatan Kebangkitan Yesus), maka masa Puasa berlangsung selama 6 minggu ditambah 4 hari, sehingga genap 40 hari. Dengan demikian, hari pertama puasa jatuh pada hari Rabu. (Paskah terjadi hari Minggu, dikurangi 36 hari (6 minggu), lalu dikurangi lagi 4 hari, dihitung mundur, jatuh pada hari Rabu).

Jadi penentuan awal masa Prapaska pada hari Rabu disebabkan karena penghitungan 40 hari sebelum hari Minggu Paska, tanpa menghitung hari Minggu.

\section{Mengapa Rabu ``Abu''?}

Abu adalah tanda pertobatan. Kitab Suci mengisahkan abu sebagai tanda pertobatan, misalnya pada pertobatan Niniwe (lih. Yun 3:6). Di atas semua itu, kita diingatkan bahwa kita ini diciptakan dari debu tanah (Lih. Kej 2:7), dan suatu saat nanti kita akan mati dan kembali menjadi debu. Olah karena itu, pada saat menerima abu di gereja, kita mendengar ucapan dari Romo, ``Bertobatlah, dan percayalah kepada Injil'' atau, ``Kamu adalah debu dan akan kembali menjadi debu'' (\textit{you are dust, and to dust you shall return}).''

\section{Tradisi Ambrosian}

Namun demikian, ada tradisi Ambrosian yang diterapkan di beberapa keuskupan di Italia, yang menghitung Masa Prapaskah selama 6 minggu, termasuk hari Minggunya, di mana kemudian hari Jumat Agung dan Sabtu Sucinya tidak diadakan perayaan Ekaristi, demi merayakan dengan lebih khidmat Perayaan Paskah. 

Semua paroki di bawah Keuskupan Agung Milan memakai ritus Katolik Ambrosian. Keuskupan Agung Milan sering pula disebut Keskupan Ambrosian. Keuskupan Ambrosian ini mengepalai paroki di propinsi Milano, Varese, Lecco, sebagian paroki di propinsi Como dan beberapa paroki di Bergamo dan Pavia. Dinamakan Ambrosian karena mengambil nama dari pendiri ritus ini dan Santo Pelindungnya yaitu Santo Ambrosius, yang sangat dihormati dan dicintai dan merupakan salah satu Santo Pelindung terpenting di Negara Italia.

Ritus Ambrosian sendiri adalah ritus Liturgi dari Gereja Katolik Milanese, yang berbeda dengan ritus Gereja Katolik Roma. Ritus Ambrosian berasal dari tradisi yang kuat dan masuk ke dalam liturgia Milanese. Ritus Ambrosian ini sebelum pengesahannya, telah menderita dari berbagai kritik akan keberadaannya, meskipun para pengikut Santo Ambrosius ini menyatakan setia terhadap Gereja Roma. Ritus Ambrosian menerima pengesahan dan pengakuannya dari Gereja Katolik Roma pada Konsili di Trento, di mana salah satu tokoh dari Konsili Trento adalah Santo Carlo Borromeo (St. Carolus Borromeus), seorang Santo Milanese.

Permulaan PraPaskah dan pemberian abu pada hari Minggu ini membedakan masa dari karnival ``baru'' (Roma) yang diakhiri dengan ``Selasa gemuk'' (Mardi Gras) dan karnival ``kuno'' (Ambrosian) yang berakhir beberapa hari kemudian.

Dalam ritus Roma, hari Minggu tidak dianggap sebagai hari bertobat/penitensi, dan oleh karenanya masa PraPaskah menjadi lebih panjang dan dimulai lebih awal. Sedangkan dalam ritus Ambrosian, hari Minggu dianggap sebagai hari penitensi.

Berbeda pula dalam konsepsi Jumat Agung: bagi ritus Ambrosian, Jumat Agung adalah hari libur Ekaristi, di mana tidak dapat dirayakan Misa, demi menjalankan hidup dengan cara radikal sengsara Kristus, sama halnya dengan Hari Sabtu Suci, demi merayakan dengan lebih khidmat Perayaan Paskah.

Pada hari-hari Minggu masa PraPaskah, sebagaimana tradisi Ambrosian, digarisbawahi pembaptisan, yang mempersiapkan dan membawa katekumen kepada Pembaptisan pada Hari Paskah, dan membimbing umat yang dibaptis untuk menemukan kembali arti dari Sakramen ini, yang mana dalam Kristus yang wafat dan bangkit menjadi anak-anak Allah.

Jadi dalam ritus Roma masa PraPaskah adalah 6 minggu (ditambah beberapa hari yang mana hari Minggu tidak dihitung) dan Masa Adven adalah 4 minggu, sementara dalam ritus Ambrosian semua hari dalam seminggu dihitung sebagai Masa PraPaskah dan baik Masa PraPaskah dan Adven adalah 6 minggu.

Sebuah elemen fundamental dari ritus Ambrosian juga terbentuk dalam lagu ``ambrosian''. Adalah Santo Ambrosius sendiri yang pada pertama kalinya dalam liturgia Gereja, pada tahun 386 After Christ memperkenalkan penggunaan lagu-lagu yang bukan berasal dari Mazmur.

Inovasinya ini dengan cepat tersebar ke dalam gereja-gereja dari ritus lainnya. Seperti ritus Gregorian, ritus Ambrosian juga dimodifikasi dalam perjalanan abad dari ``penemuan'' nya oleh santo Ambrosius, bahkan sampai saat ini dinyatakan sebagai organ musik barat paling antik. Dan untuk memelihara warisan ini telah didirikan institusi PIAMS (Lembaga Musik Kudus Ambrosian) yang bekerjasama dengan Lembaga Musik Kudus Kepausan di Roma. Ritus Ambrosian yang antik dan khidmat ini telah ikut memperkaya Liturgia Gereja Katolik.

\sumber{Stefanus Tay, Ingrid Tay, dan Caecilia Triastuti\\
http://katolisitas.org}