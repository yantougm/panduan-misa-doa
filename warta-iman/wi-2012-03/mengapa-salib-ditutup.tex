\chap{Mengapa salib ditutup dengan kain ungu?}

Berikut ini adalah terjemahan dari ``\textit{Saint Joseph Catholic Manual}'' (copyright 1956)

\section*{Masa Sengsara Yesus}

Masa Sengsara Yesus dimulai pada Minggu ke- 5 Masa Prapaska, yang dikenal sebagai Minggu Sengsara, dan dari hari itu sampai Paska, Gereja masuk lebih dalam lagi ke dalam Kisah Sengsara Tuhan Yesus dan membawa sengsara-Nya lebih dan lebih dalam lagi ke hadapan umat-Nya. Liturgi mengesampingkan semua lambang suka cita dan menampilkan dalam kata dan perbuatan, kesedihan dan penitensi yang harus mengisi setiap jiwa orang Kristen pada saat merenungkan peristiwa-peristiwa akhir dalam kehidupan Penyelamat kita di dunia ini.

Sebelum Vespers pada hari Sabtu sebelum Minggu Sengsara, \textit{crucifix} (salib Yesus), patung-patung dan gambar-gambar di altar dan di sekitar gereja ditutup dengan kain ungu polos, kecuali gambar-gambar Jalan Salib. Salib Tuhan Yesus ditutupi kain ungu sampai hari Jumat Agung, sedangkan patung-patung dan gambar-gambar lainnya tetap ditutup sampai pada saat Gloria pada Sabtu Suci. Patung-patung dan gambar-gambar para malaikat dan santa-santo ditutup, untuk menunjukkan bahwa Gereja membungkus dirinya sendiri dan berkabung saat Tuhannya sedang mempersiapkan Diri untuk mengalami kesengsaraan dan kematian untuk menebus dunia. Dengan semua tanda-tanda lahiriah dan upacara Masa Sengsara, umat beriman diingatkan bagaimana Tuhan dalam keilahian-Nya di sepanjang masa sengsara-Nya, dan dengan penglihatan dan pendengaran, para pendosa diingatkan agar bertobat dan menarik diri semakin jauh dari kesenangan-kesenangan duniawi, dengan mendevosikan diri semakin dalam kepada doa- doa Masa Prapaska dan merenungkan kisah sengsara Kristus yang telah wafat demi kasih-Nya kepada mereka.
