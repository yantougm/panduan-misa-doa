\chap{Empat Prinsip Hidup}
\small
\begin{wrapfigure}{r}{3cm}
\includegraphics[scale=0.35]{gambar/child.png}
\end{wrapfigure}
Setiap hari kita berinteraksi dengan begitu banyak orang. Di keluarga, di tempat kerja, di sekelah, atau dimanapun kita akan membuat hubungan dengan begitu banyak orang. Tidak jarang, tiba-tiba timbul persoalan atau juga konflik dalam hubungan kita dengan orang lain. Tapi, itulah hidup! Namun, bagaimana kita menyikapi koflik tersebut?
Apakah kita percaya bahwa Tuhan bisa memakai orang - orang di sekitar kita, bahkan yang sedang berkonflik dengan kita, untuk membentuk karakter dalam hidup kita?

Jika kita ingin memaknai hidup dengan cara seperti itu, kita perlu 4 prinsip hidup berikut dalam berhubungan dengan orang lain:
\begin{enumerate}
\item Jagalah hati. Firman Tuhan berkata: ``Jagalah hatimu dengan segala kewaspadaan, karena dari situlah terpancar kehidupan.'' Ketika kita menerima kata - kata atau perlakuan yang menyakitkan, Jagalah Hati. Jika kita bisa menjaga kondisi hati kita untuk tidak mudah terpengaruh emosi dan tindakan orang lain, kita akan mampu melepaskan pengampunan dan lepas dari belenggu sakit hati.
\item Jagalah perkataan. Kita tidak hanya perlu memperhatikan apa yang kita ucapkan, tetapi juga cara mengucapkannya. Ada kalanya hanya karena salah ucap, atau nada suara dan ungkapan sinis bisa memancing sebuah pertengkaran. Hindarilah perkataan - perkataan yang tajam dan tidak perlu.
\item Jangan ungkit kegagalan masa lalu. Ingat, mungkin kita sedang bicara dengan orang yang perncah gagal di masa lalu. Daripada mengungkit-ungkit masa lalu yang bisa menimbulkan kesalahpagaman, lebih baik membicarakan hal-hal yang sekarang.
\item Jangan menunjukkan perkataan yang sombong. Tidak perlu memuji diri karena sebuah perbuatan yang pernah kita lakukan. Sikap rendah hati adalah kuncu dalam menjalin komunikasi yang positif. Belajarlah untuk bersuka cita ketika orang lain menerima pujian, sekalipun saat itu kita pun layak menerimanya.
\end{enumerate}

\begin{center}
\textbf{Belajarlah setiap hari untuk mendatangkan damai sejahtera bagi setiap orang.}\\
`` \ldots \textit{Sehingga kamu hidup sebagai orang - orang yang sopan di mata orang luar dan tidak bergantung pada mereka.} (1 Tesalonika 4 : 12)
\end{center}

\sumber{Kiriman Aditya Bimantara\\
dari ``Empat Prinsip Hidup.'' Renungan Harian Kita Maret 2010 \\  
http://www.renungan-harian-kita.blogspot.com.} 
\normalsize