\chap{Warta Lingkungan}

\subsection*{Doa rutin}
Kamis minggu pertama bulan Februari diisi dengan doa rutin lingkungan. Ibu Nanik sebagai tuan rumah doa lingkungan menyediakan hidangan khas \textit{thiwul}, makanan tempo dulu tapi dikemas modern. Sebelum menikmati \textit{thiwul} Ibu MOS Padmini memimpin ibadat doa rutin dengan mengambil bacaan dari kalender liturgi, dari Injil Lukas bab 2 yaitu tentang Yesus diserahkan ke bait Allah. Dari beberapa \textit{sharing} oleh umat muncul tentang kewajiban sebagai warga negara dan kewajiban sebagai umat Allah. Kedua kewajiban itu perlu dipatuhi agar kehidupan bermasyarakat dan beragama bosa berjalan seiring dan saling menguatkan.

\subsection*{Misa pemberkatan rumah}
Tgl 11 Februari lingkungan St. Petrus mendapat tambahan anggota baru dengan ditandai pemberkatan rumah di rumah Bapak Valentinus Dalyono. Keluarga Bapak V. Dalyono beranggotakan Bapak dan Ibu Dalyono dengan 2 anak Lili dan Fael. Ibu Dalyono sementara ini masih bekerja di Jakarta. Misa pemberkatan dipimpin oleh Rm Antonius Juned Triatmoko dari Gereja Paroki Wedi. Bacaan misa tentang kisah Zakeus. Dalam homilinya Romo menekankan perlunya mencontoh Zakeus yang memikirkan orang lain, tidak hanya mementingkan diri sendiri. Berkat kedatangan Yesus, Zakeus berubah total dari pemungut cukai yang banyak hartanya menjadi orang yang mau berbagi hartanya bagi orang miskin.
Romo juga mengatakan bahwa mental kita adalah mental korupsi, yaitu menggunakan barang milik publik atau negara untuk kepentingan diri sendiri atau golongan. Romo menyinggung masalah penggunaan jalan umum untuk parkir dan keperluan pribadi lain, yang mengindikasikan mental korupsi kita.

\subsection*{Ibadat arwah}
Kamis Minggu ke-3 diisi dengan Ekaristi Peringatan 2 tahun meninggalnya Ibu Monika Suarti, ibu dari Ibu Th. Adriani (istri Bapak M. Daryanto) bertempat di rumah Bapak M. Daryanto. Romo Kalictus Ninda SVD berkenan memimpin ekaristi. Bacaan Kitab Suci diambil dari Yehezkiel 37:1-6 dan Markus 4:30-34. Perumpamaan tentang biji sesawi yang kecil bila ditaburkan di tanah ternyata dapat tumbuh besar dan dapat digunakan sebagai naungan oleh burung-burung.

\subsection*{Pendaftaran Krisma}
Telah diumumkan di gereja bahwa sakramen Krisma untuk paroki Marganingsih Kalasan akan dilangsungkan bulan September 2012. Calon penerima sakramen Krisma dapat mendaftarkan diri ke Ibu Munarti, Bapak Neo Suradi, atau kepada ketua lingkungan, dengan menyerahkan fotokopi surat baptis.
