\chap{Mengapa kita berpantang dan berpuasa?}

\section*{Kamu mau pantang apa dalam Masa Prapaska ini?}
Ya, seharusnya pertanyaan ini timbul di hati kita sebelum kita memulai masa Prapaska, jika kita ingin membuat Masa Prapaska ini suatu kesempatan kita untuk bertumbuh secara rohani. Inilah kesempatan bagi kita untuk merenungkan, hal apa yang paling kita sukai, yang dapat kita `korbankan' demi menyatakan kasih kepada Tuhan, yang lebih dahulu mengasihi kita. Hal yang disukai bisa berbeda antara orang yang satu dengan yang lain, dan karena itu, yang paling dapat merasakan efeknya adalah orang yang bersangkutan. Ada keluarga teman saya yang senengnya menonton TV, kemudian mereka memutuskan untuk mengurangi nonton TV sehingga hanya 1 kali seminggu, hari Sabtu. Waktu yang tadinya dipakai untuk nonton TV dipergunakan untuk berkumpul dan berdoa bersama. Tahun lalu, di samping pantang daging,  suami saya memilih pantang kopi, dan saya pantang sambal. Minggu pertama sangat berat buat suami saya, yang sudah bertahun-tahun terbiasa minum kopi minimal 3 gelas sehari. Awalnya, kepalanya pusing dan selalu mengantuk, namun toh akhirnya bisa juga. Lalu saya, dengan pantang sambal maka makan apapun rasanya kurang pas di lidah saya. Tapi hal ini mengajarkan saya supaya tidak lekas komplain. Sebab ini bukan apa-apa jika dibandingkan dengan pengorbanan Yesus di kayu salib.

Memang, kita dapat menemukan banyak jenis pantang, dan mungkin pula kita dapat memilih yang  sedikit lebih sulit, yang melibatkan penguasaan diri. Contohnya, pantang membicarakan kekurangan orang lain, pantang membicarakan kelebihan diri sendiri,  pantang mengeluh/ komplain, pantang berprasangka negatif atau pantang marah bagi orang yang lekas emosi. Selanjutnya kita diajak untuk lebih mengarahkan hati kepada Tuhan dan berusaha menyenangkan hati-Nya dengan pikiran dan perbuatan kita.  Ini adalah contoh yang paling sederhana dari ucapan, ``Aku mau mati terhadap diri sendiri dan hidup bagi Tuhan'' (lih. Rom 6:8). Jadi pantang dan puasa bukan sekedar tidak makan daging atau tidak jajan, tetapi selebihnya tak ada yang berubah dalam hubungan kita dengan Tuhan. Kita diundang untuk melihat ke dalam diri kita, untuk melihat kebiasaan apakah yang selama ini menghalangi kita untuk lebih dekat kepada Tuhan. Mari, pada masa Prapaska ini, kita membuat suatu usaha nyata untuk mengambil `penghalang' tersebut dalam hidup kita. Dan dengan demikian, kita dapat mengalami hubungan yang lebih baik dengan Tuhan.

\section*{Buat apa berpantang dan berpuasa}
Setiap masa Prapaska, kita diajak oleh Gereja untuk bersama-sama berpantang dan berpuasa. Puasa dan pantang yang disyaratkan oleh Gereja Katolik sebenarnya tidak berat, sehingga sesungguhnya tidak ada alasan bagi kita untuk tidak melakukannya. Namun, meskipun kita melakukannya, tahukah kita arti pantang dan puasa tersebut bagi kita umat Katolik?

Bagi kita orang Katolik, puasa dan pantang artinya adalah tanda pertobatan, tanda penyangkalan diri, dan tanda kita mempersatukan sedikit pengorbanan kita dengan pengorbanan Yesus di kayu salib sebagai silih dosa kita dan demi mendoakan keselamatan dunia. Jika pantang dan puasa dilakukan dengan hati tulus maka keduanya dapat menghantar kita bertumbuh dalam kekudusan. Kekudusan ini yang dapat berbicara lebih lantang dari pada khotbah yang berapi-api sekalipun, dan dengan kekudusan inilah kita mengambil bagian dalam karya keselamatan Allah. Allah begitu mengasihi dan menghargai kita, sehingga kita diajak oleh-Nya untuk mengambil bagian dalam karya keselamatan ini. Caranya, dengan bertobat, berdoa dan melakukan perbuatan kasih, dan sesungguhnya inilah yang bersama-sama kita lakukan dalam kesatuan dengan Gereja pada masa Prapaska.

Jangan kita lupa bahwa  masa puasa selama 40 hari ini adalah karena mengikuti teladan Yesus, yang juga berpuasa selama 40 hari 40 malam, sebelum memulai tugas karya penyelamatan-Nya (lih. Mat 4: 1-11; Luk 4:1-13). Yesus berpuasa di padang gurun dan pada saat berpuasa itu Ia digoda oleh Iblis. Yesus mengalahkan godaan tersebut dengan bersandar pada Sabda Tuhan yang tertulis dalam Kitab Suci. Maka, kitapun hendaknya bersandar pada Sabda Tuhan untuk mengalahkan godaan pada saat kita berpuasa. Dengan doa dan merenungkan Sabda Tuhan, kita akan semakin menghayati makna puasa dan pantang pada Masa Prapaska ini.

\section*{Puasa dan pantang tak terlepas dari doa}
Jadi puasa dan pantang bagi kita tak pernah terlepas dari doa. Dalam masa Prapaska, puasa, pantang dan doa disertai juga dengan perbuatan amal kasih bersama-sama dengan anggota Gereja yang lain. Dengan demikian, pantang dan puasa bagi kita orang Katolik merupakan latihan rohani yang mendekatkan diri pada Tuhan dan sesama, dan bukan untuk hal lain, seperti semata-mata `menyiksa badan', diet/ supaya kurus, menghemat, dll. Janganlah kita lupa, tujuan utama puasa dan pantang adalah supaya kita dapat lebih menghayati kasih Tuhan yang kita terima dan kasih kepada Tuhan. Kita diajak untuk merenungkan sengsara Kristus demi menyelamatkan kita, dan selanjutnya kita diajak untuk menyatakan kasih kita kepada Kristus, dengan mendekatkan diri kepada-Nya dan sesama.

\section*{Dengan puasa kita mengambil bagian dalam karya keselamatan Allah}

Dengan mendekatkan dan menyatukan diri dengan Tuhan, maka kehendak-Nya menjadi kehendak kita. Dan karena kehendak Tuhan yang terutama adalah keselamatan dunia, maka melalui puasa dan pantang, kita diundang Tuhan untuk mengambil bagian dalam karya penyelamatan dunia, yaitu dengan berdoa dan menyatukan pengorbanan kita dengan pengorbanan Yesus di kayu salib. Kita pun dapat mendoakan keselamatan dunia dengan mulai mendoakan bagi keselamatan orang-orang yang terdekat dengan kita: orang tua, suami/ istri, anak-anak, saudara, teman, dan juga kepada para imam dan pemimpin Gereja. Kemudian kita dapat pula berdoa bagi para pemimpin negara, para umat beriman, ataupun mereka yang belum mengenal Kristus.

\section*{Puasa dan Pantang menurut Ketentuan Gereja Katolik}

Berikut ini mari kita lihat ketentuan tobat dengan puasa dan pantang, menurut Kitab Hukum Gereja Katolik:
\begin{enumerate}
\item \textbf{Kan. 1249}:  Semua orang beriman kristiani wajib menurut cara masing-masing melakukan tobat demi hukum ilahi; tetapi agar mereka semua bersatu dalam suatu pelaksanaan tobat bersama, ditentukan hari-hari tobat, dimana umat beriman kristiani secara khusus meluangkan waktu untuk doa, menjalankan karya kesalehan dan amal-kasih, menyangkal diri sendiri dengan melaksanakan kewajiban-kewajibannya secara lebih setia dan terutama dengan berpuasa dan berpantang, menurut norma kanon-kanon berikut.

\item \textbf{Kan. 1250}: Hari dan waktu tobat dalam seluruh Gereja ialah setiap hari Jumat sepanjang tahun, dan juga masa prapaskah.

\item \textbf{Kan. 1251}: Pantang makan daging atau makanan lain menurut ketentuan Konferensi para Uskup hendaknya dilakukan setiap hari Jumat sepanjang tahun, kecuali hari Jumat itu kebetulan jatuh pada salah satu hari yang terhitung hari raya; sedangkan pantang dan puasa hendaknya dilakukan pada hari Rabu Abu dan pada hari Jumat Agung, memperingati Sengsara dan Wafat Tuhan Kita Yesus Kristus.

\item \textbf{Kan. 1252}: Peraturan pantang mengikat mereka yang telah berumur genap empat belas tahun; sedangkan peraturan puasa mengikat semua yang berusia dewasa sampai awal tahun ke enampuluh; namun para gembala jiwa dan orangtua hendaknya berusaha agar juga mereka, yang karena usianya masih kurang tidak terikat wajib puasa dan pantang, dibina ke arah cita-rasa tobat yang sejati.

\item \textbf{Kan. 1253}: Konferensi para Uskup dapat menentukan dengan lebih rinci pelaksanaan puasa dan pantang; dan juga dapat mengganti-kan seluruhnya atau sebagian wajib puasa dan pantang itu dengan bentuk-bentuk tobat lain, terutama dengan karya amal-kasih serta latihan-latihan rohani.
\end{enumerate}

Memang sesuai dari yang kita ketahui, ketentuan dari Konferensi para Uskup di Indonesia menetapkan selanjutnya:

\begin{enumerate}
\item Hari Puasa dilangsungkan pada hari Rabu Abu dan Jumat Agung. Hari Pantang dilangsungkan pada hari Rabu Abu dan tujuh Jumat selama Masa Prapaska sampai dengan Jumat Agung.

\item Yang wajib berpuasa ialah semua orang Katolik yang berusia 18 tahun sampai awal tahun ke-60. Yang wajib berpantang ialah semua orang Katolik yang berusia genap 14 tahun ke atas.

\item Puasa (dalam arti yuridis) berarti makan kenyang hanya sekali sehari. Pantang (dalam arti yuridis) berarti memilih pantang daging, atau ikan atau garam, atau jajan atau rokok. Bila dikehendaki masih bisa menambah sendiri puasa dan pantang secara pribadi, tanpa dibebani dengan dosa bila melanggarnya.
\end{enumerate}

\noindent{Maka penerapannya adalah sebagai berikut:}
\small
\begin{enumerate}
\item Kita berpantang setiap hari Jumat sepanjang tahun (contoh: pantang daging, pantang rokok dll) kecuali jika hari Jumat itu jatuh pada hari raya, seperti dalam oktaf masa Natal dan oktaf masa Paskah. Penetapan pantang setiap Jumat ini adalah karena Gereja menentukan hari Jumat sepanjang tahun (kecuali yang jatuh di hari raya) adalah hari tobat. Namun, jika kita mau melakukan yang lebih, silakan berpantang, setiap hari selama Masa Prapaska.

\item Jika kita berpantang, pilihlah makanan/ minuman yang paling kita sukai. Pantang daging adalah contohnya, atau yang lebih sukar mungkin pantang garam. Tapi ini bisa juga berarti pantang minum kopi bagi orang yang suka sekali kopi, dan pantang sambal bagi mereka yang sangat suka sambal, pantang rokok bagi mereka yang merokok, pantang jajan bagi mereka yang suka jajan. Jadi jika kita pada dasarnya tidak suka jajan, jangan memilih pantang jajan, sebab itu tidak ada artinya.

\item Pantang tidak terbatas hanya makanan, namun pantang makanan dapat dianggap sebagai hal yang paling mendasar dan dapat dilakukan oleh semua orang. Namun jika satu dan lain hal tidak dapat dilakukan, terdapat pilihan lain, seperti pantang kebiasaan yang paling mengikat, seperti pantang nonton TV, pantang 'shopping', pantang ke bioskop, pantang `gossip', pantang main `game', pantang buka Internet, dll. Jika memungkinkan tentu kita dapat melakukan gabungan antara pantang makanan/ minuman dan pantang kebiasaan ini.

\item Puasa minimal dalam setahun adalah Hari Rabu Abu dan Jumat Agung, namun bagi yang dapat melakukan lebih, silakan juga berpuasa dalam ketujuh hari Jumat dalam masa Prapaska, (atau bahkan setiap hari dalam masa Prapaska).

\item Waktu berpuasa, kita makan kenyang satu kali, dapat dipilih sendiri pagi, siang atau malam. Harap dibedakan makan kenyang dengan makan sekenyang-kenyangnya. Karena maksud berpantang juga adalah untuk melatih pengendalian diri, maka jika kita berbuka puasa/ pada saat makan kenyang, kita juga tetap makan seperti biasa, tidak berlebihan. Juga makan kenyang satu kali sehari bukan berarti kita boleh makan snack/ cemilan berkali-kali sehari. Ingatlah tolok ukurnya adalah pengendalian diri dan keinginan untuk turut merasakan sedikit penderitaan Yesus, dan mempersatukan pengorbanan kita dengan pengorbanan Yesus di kayu salib demi keselamatan dunia.

\item Maka pada saat kita berpuasa, kita dapat mendoakan untuk pertobatan seseorang, atau mohon pengampunan atas dosa kita. D oa-doa seperti inilah yang sebaiknya mendahului puasa, kita ucapkan di tengah-tengah kita berpuasa, terutama saat kita merasa haus/ lapar, dan doa ini pula yang menutup puasa kita/ sesaat sebelum kita makan. Di sela-sela kesibukan sehari-hari kita dapat mengucapkan doa sederhana, ``Ampunilah aku, ya Tuhan. Aku mengasihi-Mu, Tuhan Yesus. Mohon selamatkanlah \ldots.'' (sebutkan nama orang yang kita kasihi)

\item Karena yang ditetapkan di sini adalah syarat minimal, maka kita sendiri boleh menambahkannya sesuai dengan kekuatan kita. Jadi boleh saja kita berpuasa dari pagi sampai siang, atau sampai sore, atau bagi yang memang dapat melakukannya, sampai satu hari penuh. Juga tidak menjadi masalah, puasa sama sekali tidak makan dan minum atau minum sedikit air. Diperlukan kebijaksanaan sendiri (prudence) untuk memutuskan hal ini, yaitu seberapa banyak kita mau menyatakan kasih kita kepada Yesus dengan berpuasa, dan seberapa jauh itu memungkinkan dengan kondisi tubuh kita. Walaupun tentu, jika kita terlalu banyak `excuse' ya berarti kita perlu mempertanyakan kembali, sejauh mana kita mengasihi Yesus dan mau sedikit berkorban demi mendoakan keselamatan dunia.
\end{enumerate}
\normalsize
\section*{Tidak terbatas Pantang dan Puasa dan derma/amal}

Dalam masa Prapaska ini, dapat pula kita melakukan sesuatu yang baik yang belum secara konsisten kita lakukan. Misal, bangun lebih pagi setiap hari untuk berdoa, misal dari yang biasanya 5 menit, usahakan jadi 10 menit; atau dari yang biasanya 10 menit, usahakan jadi 20 menit, atau yang 30 menit jadi 1 jam. Memulai hari dengan berdoa dan merenungkan Sabda Tuhan adalah sesuatu yang perlu kita usahakan setiap hari.

Mengikuti Misa Harian (di samping Misa hari Minggu, tentu saja) adalah sesuatu yang dapat pula kita lakukan, jika itu memang memungkinkan dalam situasi kita. Jangan terlalu cepat mengatakan tidak mungkin, jika belum pernah mencoba. Apalagi jika kita tidak mencobanya karena malas bangun pagi. Mengikuti Misa dan menyambut Kristus dalam Ekaristi adalah bukti yang nyata bahwa kita sungguh menghargai apa yang telah dilakukan-Nya bagi kita di kayu salib demi keselamatan kita. Kita dapat pula meluangkan waktu untuk doa Adorasi, di hadapan Sakramen Maha Kudus, jika memang ada kapel Adorasi di paroki/ di kota tempat kita tinggal. Atau kita dapat mulai berdoa Rosario setiap hari. Atau mulai dengan setia meluangkan waktu untuk mempelajari Kitab Suci dan Katekismus Gereja Katolik. Atau mengikuti Ibadat Jalan Salib di gereja, atau jika tidak mungkin, melakukannya bersama dengan keluarga di rumah.

Dalam relasi kita dengan sesama, juga tidak terbatas dengan `asal sudah nyumbang, maka sudah beres'. Dengan merenungkan sengsara Tuhan Yesus, maka kita diajak untuk lebih peka terhadap sikap kita terhadap sesama yang kurang beruntung. Misalnya, yang paling dekat adalah pembantu rumah tangga dan supir. Pernahkah kita memberi kesempatan pada mereka untuk beristirahat, misalnya memberi mereka libur? Libur di sini tidak termasuk hanya pada libur Lebaran, dst, tetapi libur/ istirahat agar mereka juga dapat berekreasi dan melepas lelah. Atau apakah kita menjalin persahabatan dengan sesama anggota Paroki yang berkekurangan?

Wah, banyak sekali sesungguhnya yang dapat kita lakukan, jika kita sungguh ingin bertumbuh di dalam iman. Namun seungguhnya, mulailah saja dengan langkah kecil dan sederhana. St. Theresia dari Liseux pernah mengatakan tipsnya, yaitu, ``Lakukanlah perbuatan-perbuatan yang kecil dan sederhana, namun dengan kasih yang besar.''

\section*{Penutup}

Maka untuk menjawab pertanyaan awal, ``Mau pantang apa aku pada Masa Prapaska ini?'', kita perlu kembali melihat ke dalam hati kita masing-masing. Pasti jika kita mau jujur, akan selalu ada yang dapat kita lakukan. Mengurangi nonton TV, mengurangi ngemil/ jajan, mengurangi nonton bioskop, tidak main game di internet, dll hanya contoh saja, namun itu belum lengkap, jika kita tidak menggunakan waktu tersebut, untuk hal-hal lain yang lebih mendukung perbuatan kasih kita kepada Tuhan dan sesama.

Ya, dengan Rabu Abu, kita diingatkan bahwa hidup kita di dunia ini hanyalah sementara, maka mari kita mempersiapkan diri bagi kehidupan kita yang sesungguhnya di surga kelak. Kita hanya dapat masuk surga dan memandang Tuhan hanya jika kita memiliki kekudusan itu (lih. Ibr 12:14), maka sudah saatnya kita bertanya pada diri sendiri: sudahkah aku hidup kudus? Masa pertobatan adalah masa rahmat yang Tuhan berikan pada kita, untuk mengatur kembali fokus kehidupan kita. Apakah yang menjadi pusat kegiatanku sehari-hari: aku atau Tuhan? Jika kita masih banyak menemukan `aku' sebagai pusatnya, mungkin sudah saatnya kita mulai mengubahnya \ldots

\sumber{Ingrid Listiati\\http://www.katolisitas.org}
