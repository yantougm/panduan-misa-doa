\chap{Ketika Garam Kehilangan Asinnya}
\small
\begin{wrapfigure}{r}{3cm}
\includegraphics[scale=0.45]{gambar/gandhi.png}
\end{wrapfigure}
``Kamu adalah garam dunia, jika garam itu menjadi tawar, dengan apakah ia diasinkan? Tidak ada lagi gunanya selain dibuang dan diinjak orang'' (Mat. 5:13).
Mahadma Gandhi pernah sangat kecewa kepada orang Kristen di india karena sikap mereka yang sangat arogan dan suka membeda - bedakan. ``Aku bersimpati kepada ajaran Yesusmu, akan tetapi sangat muak dengan cara hidupmu'', ucap tokoh negeri Sungai Gangga yang dikenal seantero dunia ini. 

Mengapa banyak orang kecewa terhadap gereja?

Kecewa terhadap pengikut Kristus yang seharusnya menjadi batu lompatan untuk mengenal Kristus?

Mengapa mereka tidak dapat melanjutkan ketertarikan mereka kepada Kristus saat mereka melihat cara hidup orang-orang yang menamakan diri sebagai pengikut Kristus?

Pembaca Warta Iman terkasih, kali ini firman Tuhan mengingatkan kembali inti keberadaan kita di atas muka bumi ini. ``Kamu adalah garam dunia. Jika garam Kehilangan asinnya tidak ada gunanya lagi kecuali dibuang dan diinjak-injak orang'', demikian pesan Tuhan Yesus.

Pada zaman itu, garam yang dipakai untuk memasak menempel pada suatu media yang bisa berupa bunga karang atau lainnya. Setiap kali garam dipakai dimasukkan ke dalam tempat yang telah disediakan sebagai wadah memasak. Ketika garam yang menempel di bunga karang itu habis, maka media tersebut akan dibuang. Mengapa Demikian?? Karena manfaatnya telah tiada. Tinggal sisa-sisa yang tidak ada gunanya lagi!!!

Sebagai orang yang percaya kepada Tuhan Yesus. Seberapa besarkah tekad kita untuk menjadi sarana-Nya sehingga olehnya banyak orang yang datang kepada Dia? Sudahkah kita memahami rencan-Nya bagi dunia ini? Yakni agar segenap lutut berlutut dan mengaku bahwa Kristus adalah Tuhan bagi kemuliaan Allah Bapa!

\begin{center}
``Tuhan ajari aku agar mengerti maksudMu. Bimbing aku supaya berhasil 
menggarami dunia ini.''
\end{center}

\sumber{Kiriman Aditya Bimantara\\
dari ``Ketika Garam Kehilangan Asinnya'' Renungan Harian Kita Maret 2010 \\  
http://www.renungan-harian-kita.blogspot.com.} 
\normalsize
