\chap{Pesan Bapa Suci Benektus XVI
untuk Masa Prapaskah 2012}

``Dan  marilah kita saling memperhatikan supaya kita saling mendorong dalam kasih dan dalam pekerjaan baik'' (Ibr 10:24).

\small
Saudara dan saudari yang terkasih,

Masa Prapaskah sekali lagi memberikan kepada kita sebuah kesempatan untuk merenungkan inti terdalam dari kehidupan seorang Kristen, yaitu: perbuatan amal kasih. Ini adalah waktu yang tepat untuk memperbaharui perjalanan iman kita, baik sebagai seorang individu maupun sebagai bagian dari komunitas, dengan bimbingan Sabda Tuhan dan sakramen-sakramen Gereja. Perjalanan ini adalah perjalanan yang ditandai dengan doa dan berbagi, hening dan berpuasa, sebagai antisipasi menyambut sukacita Paskah.

Tahun ini saya ingin mengajukan beberapa pemikiran dalam terang ayat-ayat Kitab Suci yang diambil dari Surat kepada umat Ibrani: ``Dan marilah kita kita saling memperhatikan supaya kita saling mendorong dalam kasih dan dalam pekerjaan baik''. Kata-kata ini adalah bagian dari perikop di mana sang penulis surat yang kudus menghimbau kita untuk menaruh kepercayaan di dalam Yesus Kristus sebagai Imam Agung yang telah memenangkan pengampunan Allah bagi kita dan membuka jalan kepada Tuhan. Mengimani Kristus membuat kita mampu menghasilkan buah di dalam hidup yang ditopang oleh tiga kebaijkan teologis: hal itu berarti menghampiri Tuhan ``dengan hati tulus dan penuh iman'' (ay.22), tetap ``teguh dalam harapan yang kita nyatakan'' (ay.23) dan senantiasa berusaha untuk menjalani hidup yang dibangun di atas ``cinta kasih dan pekerjaan-pekerjaan baik'' (ay.24), bersama dengan saudara dan saudari kita. Sang penulis surat tersebut menyatakan bahwa untuk mempertahankan hidup yang dibentuk oleh Injil, adalah penting untuk berpartisipasi secara aktif dalam liturgi dan doa bersama komunitas, dengan mengingat akan tujuan eskatologis untuk bersatu secara penuh dengan Tuhan (ay.25). Di sini saya ingin membuat refleksi atas ayat 24, yang memberikan pengajaran yang ringkas, bernilai, dan tepat di segala zaman, atas tiga aspek hidup Kristiani, yaitu: kepedulian kepada sesama, kasih timbal balik, dan kekudusan pribadi.

\section{``Dan  marilah kita saling memperhatikan..'' : tanggung jawab terhadap para saudara dan saudari kita.}

Aspek pertama adalah sebuah undangan untuk ``peduli'' : kata kerja bahasa Yunani yang dipakai di sini adalah katanoein, yang artinya adalah untuk memeriksa (menyelidiki), untuk menaruh perhatian, untuk mengamati dengan seksama dan percaya akan sesuatu. Kita menjumpai kata ini di dalam Injil ketika Yesus mengundang para murid untuk ``memperhatikan'' burung-burung gagak, yang tanpa bekerja keras, berada di tengah perhatian dan pemeliharaan Penyelenggaraan Ilahi (bdk. Luk 12:24) dan untuk ``memeriksa'' balok di dalam mata kita sendiri sebelum mengeluarkan selumbar dari mata saudara kita (bdk. Luk 6:41). Di dalam ayat yang lain dari Surat kepada orang-orang Ibrani, kita menemukan ajakan untuk ``mengarahkan pikiranmu kepada Yesus'' (3:1), Rasul dan Imam Besar dari iman kita. Maka kata kerja yang mengantar pengajaran kita mengatakan kepada kita untuk memperhatikan sesama, pertama-tama kepada Yesus, untuk saling memperhatikan satu sama lain, dan tidak tinggal dalam keterasingan serta sikap acuh tak acuh kepada keadaan sesama kita. Namun demikian, terlalu sering sikap yang kita tunjukkan justru sebaliknya: yaitu pengabaian dan keacuhan yang lahir dari keegoisan yang disamarkan sebagai tindakan menghargai ``privasi''. Saat ini pun, suara Tuhan meminta kita semua untuk saling memperhatikan satu sama lain. Bahkan hari ini, Tuhan meminta kita untuk menjadi ``penjaga'' saudara dan saudari kita (Kej 4:9), untuk membangun suatu relasi yang didasarkan atas kepedulian satu sama lain dan perhatian kepada kesejahteraan integral jasmani dan rohani dari sesama kita. Perintah yang utama untuk mengasihi satu sama lain menuntut kita untuk mengenali tanggung jawab kita kepada sesama yang, sebagaimana halnya kita sendiri, adalah ciptaan dan anak-anak Tuhan sendiri. Menjadi saudara dan saudari dalam kemanusiaan dan, dalam banyak hal,  juga dalam iman, selayaknya menolong kita untuk mengenali di dalam diri sesama kita, sebuah kebalikan dari diri kita (alter ego), yang dicintai tanpa batas oleh Tuhan. Jika kita menanamkan pada diri kita cara ini yang memandang sesama sebagai saudara dan saudari kita, maka solidaritas, keadilan, belas kasihan dan bela rasa akan secara alamiah berkembang di dalam hati kita. Sang Pelayan Tuhan Paus Paulus VI pernah menyatakan bahwa dunia saat ini menderita terutama karena kurangnya persaudaraan: ``Kebudayaan umat manusia sedang sangat sakit. Penyebabnya bukanlah karena berkurangnya sumber-sumber daya alam, dan bukan juga karena kontrol monopoli dari segelintir orang: melainkan lebih karena melemahnya ikatan persaudaraan di antara pribadi-pribadi dan di antara bangsa-bangsa (Populorum Progressio, 66).

Kepedulian kepada sesama berkaitan juga dengan menginginkan segala yang baik untuk mereka dari setiap sudut pandang: baik fisik, moral, maupun spiritual. Budaya kontemporer nampaknya telah kehilangan naluri untuk membedakan yang baik dari yang jahat, namun disadari tetap ada suatu kebutuhan yang nyata untuk menyatakan kembali bahwa kebaikan itu ada dan akan mengatasi [yang jahat], karena Allah ``baik dan berbuat baik'' (Mzm 119:68). Kebaikan adalah segala sesuatu yang bersifat memberi, melindungi, dan menjunjung tinggi kehidupan, persaudaraan, dan persekututuan. Maka tanggung jawab kepada sesama berarti menginginkan dan mengusahakan kebaikan sesama, dalam harapan bahwa mereka pun menjadi mudah menerima kebaikan dan tuntutan- tuntutannya. Peduli kepada sesama berarti menjadi peka akan kebutuhan-kebutuhan mereka. Injil Suci memperingatkan kita akan bahaya bahwa hati kita dapat menjadi keras karena ``ketidaksadaran spiritual'', yang membuat kita tidak peka dan mati rasa terhadap penderitaan sesama. Penulis Injil Lukas mengaitkan dua perumpaan Yesus dengan membuat contoh. Di dalam perumpamaan tentang Orang Samaria yang Baik Hati, sang imam dan sang orang Lewi lewat begitu saja,  tidak peduli akan keberadaan seseorang yang dirampok dan dipukuli oleh para perampok (bdk. Luk 10:30-32). Dalam kisah perumpamaan Orang Kaya dan Lazarus yang Miskin, si orang kaya tidak peduli pada kemiskinan Lazarus, yang sedang kelaparan hingga sekarat di depan pintu rumahnya yang ada di depan matanya (bdk. Luk 16:19). Kedua perumpamaan tersebut menunjukkan contoh-contoh kebalikan dari  ``menjadi peduli'', yaitu sikap menaruh perhatian kepada sesama dengan penuh cinta dan belas kasihan. Apa yang menghalangi pandangan kemanusiaan dan penuh cinta kepada saudara dan saudari kita ini? Seringkali, penyebabnya adalah kepemilikan kekayaan materi dan perasaan berkecukupan akan segala sesuatu, namun bisa juga penyebabnya adalah kecenderungan untuk meletakkan segala kepentingan/ keinginan, dan masalah kita sendiri di atas semua yang lain. Kita tak pernah boleh gagal untuk ``menunjukkan belas kasihan'' kepada mereka yang menderita. Hati kita tak pernah boleh terlalu terbungkus rapat oleh urusan-urusan dan masalah-masalah kita sehingga hati kita tak mampu mendengar jeritan kaum miskin. Kerendahan hati dan pengalaman pribadi akan penderitaan dapat membangkitkan dalam diri kita, suatu naluri belas kasihan dan empati. ``Orang benar mengetahui hak orang lemah, tetapi orang fasik tidak memahaminya'' (Ams 29:7). Kita kemudian dapat memahami sikap dari ``mereka yang meratap'' (Mat 5:5), mereka yang mampu melihat melampaui diri sendiri dan merasakan belas kasihan terhadap penderitaan orang lain. Menjangkau orang lain dan membuka hati kita kepada kebutuhan-kebutuhan mereka dapat menjadi sebuah kesempatan bagi  keselamatan dan keadaan terberkati.

``Menjadi peduli satu sama lain'' juga mengikutsertakan sikap menaruh perhatian pada kesejahteraan jasmani dan rohani satu sama lain. Di sini saya ingin menyebutkan sebuah aspek hidup Kristiani, yang saya percaya telah cukup terlupakan selama ini: koreksi persaudaraan dalam pandangan keselamatan abadi. Dewasa ini, secara umum, kita menjadi sangat peka kepada gagasan perbuatan amal kasih dan kepedulian kepada kesejahteraan fisik dan materi dari sesama, namun hampir sepenuhnya diam mengenai tanggung jawab spiritual kita kepada saudara dan saudari kita. Hal ini tidak menjadi persoalan di dalam jemaat Gereja perdana atau di dalam komunitas yang telah sangat dewasa dalam iman, [yaitu] mereka yang peduli tidak hanya terhadap kesehatan fisik sesama mereka, tetapi juga terhadap kesehatan spiritual dan kehidupan kekal mereka. Kitab Suci berkata kepada kita: ``Janganlah mengecam seorang pencemooh, supaya engkau jangan dibencinya, kecamlah orang bijak, maka engkau akan dikasihinya'' (Ams 9:8). Kristus sendiri memerintahkan kita untuk menasehati saudara kita yang berbuat dosa (bdk. Mat 18:15). Kata yang dipergunakan untuk mengekpresikan koreksi persaudaraan – elenchein – adalah sama seperti yang biasa digunakan untuk menunjukkan misi kenabian dari orang-orang Kristen untuk menentang generasi yang mengikuti kejahatan (bdk. Ef 5:11). Tradisi Gereja juga memasukkan ``memberi nasehat kepada para pendosa'' di antara karya-karya karitatif rohani (belas kasihan secara rohani). Adalah penting untuk mengembalikan dimensi ini dari perbuatan amal kasih Kristiani. Kita tidak boleh tinggal diam dalam menghadapi kejahatan. Saya berpikir tentang semua umat Kristen itu yang,  karena pertimbangan manusiawi atau semata-mata karena pertimbangan kenyamanan pribadi, memilih berkompromi dengan mentalitas yang umum, daripada mengingatkan saudara dan saudarinya terhadap cara berpikir dan bertindak yang bertentangan dengan kebenaran dan yang tidak mengikuti jalan kebaikan. Menasehati secara Kristiani, tidak pernah dimotivasi oleh semangat menuduh atau menuntut balas, melainkan selalu digerakkan oleh cinta dan belas kasihan, dan tumbuh dari kepedulian yang tulus, demi kebaikan orang lain. Sebagaimana Rasul Paulus mengatakan:''Saudara-saudara, kalaupun seorang kedapatan melakukan suatu pelanggaran, maka kamu yang rohani, harus memimpin orang itu ke jalan yang benar dalam roh lemah lembut, sambil menjaga dirimu sendiri, supaya kamu juga jangan kena pencobaan.'' (Gal 6:1). Di dalam dunia yang diliputi oleh semangat individualisme, adalah esensial untuk menemukan kembali pentingnya koreksi persaudaraan, sehingga bersama-sama kita dapat berjalan menuju kekudusan. Kitab Suci mengatakan pada kita bahwa  bahkan ``tujuh kali orang benar jatuh'' (Ams 24:16); semua dari kita adalah lemah dan tak sempurna (bdk. 1 Yoh 1:8). Maka, adalah suatu bentuk pelayanan yang amat berarti, untuk membantu sesama kita, dan mengizinkan mereka membantu kita, sehingga kita dapat terbuka terhadap seluruh kebenaran mengenai diri kita, memperbaiki diri kita dan berjalan dengan lebih setia di jalan Tuhan. Selalu akan ada kebutuhan terhadap sebuah pandangan yang penuh kasih dan mengingatkan, yang mengetahui dan memahami, yang membedakan secara bijak dan mengampuni (bdk. Luk 22:61), sebagaimana yang Tuhan telah kerjakan dan masih akan terus mengerjakannya di dalam diri kita masing- masing.

\section{``Saling memperhatikan satu sama lain'': sebuah karunia kasih timbal balik''}

Panggilan untuk ``menjaga'' sesama kita adalah berkebalikan dengan mentalitas yang, dengan mengurangi nilai hidup hanya kepada dimensi duniawinya saja, gagal untuk melihatnya dalam perspektif eskatologis dan menerima sembarang pilihan moral apapun atas nama kebebasan pribadi. Masyarakat seperti masyarakat kita dapat menjadi buta terhadap penderitaan fisik dan tuntutan spiritual dan moral kehidupan. Hal ini tak boleh terjadi dalam komunitas Kristiani! Rasul Paulus mendorong kita untuk mengejar ``apa yang mendatangkan damai sejahtera dan yang berguna untuk saling membangun'' (Rom 14:19) demi kebaikan sesama, ``untuk mendukung satu sama lain'' (Rom 15:2), mencari bukan keuntungan pribadi melainkan lebih kepada ``kebaikan setiap orang yang lain, sehingga mereka dapat diselamatkan'' (1Kor 10:33). Koreksi yang saling membangun, dukungan dalam semangat kerendahan hati, dan perbuatan amal kasih harus menjadi bagian dari kehidupan komunitas Kristiani.

Murid-murid Tuhan, dipersatukan dengan Dia melalui Ekaristi, hidup dalam persaudaraan yang menyatukan mereka satu dengan yang lain sebagai anggota-anggota dari satu tubuh. Hal ini berarti bahwa sesama adalah bagian dari diriku, dan bahwa hidupnya, keselamatannya, berkaitan dengan hidup dan keselamatanku sendiri. Di sini kita menyentuh aspek yang mendasar dari persekutuan: keberadaan kita berkaitan erat dengan keberadaan orang lain, baik dalam suka maupun duka. Baik dosa-dosa kita maupun perbuatan-perbuatan kasih kita, sama-sama mempunyai dimensi sosial. Hubungan kasih timbal balik ini nampak di dalam Gereja, tubuh mistik Kristus: komunitas tersebut senantiasa melakukan pertobatan, dan memohon pengampunan atas dosa-dosa anggotanya, namun juga tak pernah gagal untuk bersukacita dalam teladan-teladan kebajikan dan perbuatan amal kasih yang hadir di tengah-tengahnya. Sebagaimana St. Paulus berkata: ``supaya anggota-anggota yang berbeda itu saling memperhatikan (1 Kor 12:25), sebab kita semua adalah anggota dari satu tubuh. Perbuatan amal kasih kepada saudara dan saudari kita – sebagaimana dinyatakan dalam pemberian derma, sebuah perbuatan yang diiringi dengan doa dan puasa, adalah perbuatan yang menjadi ciri khas masa Prapaskah – berakar dari kepemilikan bersama. Umat Kristiani juga dapat menyatakan keanggotaannya di dalam satu tubuh yang adalah Gereja melalui kepedulian yang konkrit bagi mereka yang paling miskin dari yang miskin. Kepedulian kepada satu sama lain juga berarti mengakui kebaikan yang sedang dikerjakan Tuhan dalam diri sesama dan menaikkan ucapan syukur atas keajaiban rahmat di mana Allah Yang Maha Besar di dalam segala kebaikan-Nya terus menerus menggenapinya di dalam diri anak-anak-Nya. Ketika umat Kristen memandang bahwa Roh Kudus sedang terus bekerja di dalam diri sesama, mereka tidak dapat berbuat yang lain selain bersukacita dan memuliakan Allah Bapa di surga (bdk. Mat 5:16).

\section{``Supaya kita saling mendorong dalam kasih dan dalam pekerjaan baik'': berjalan bersama dalam kekudusan.}

Kata-kata dari Surat kepada orang Ibrani ini (10:24) mendorong kita untuk merefleksikan panggilan universal kepada kekudusan, sebuah perjalanan yang terus menerus dari kehidupan spiritual sebagaimana kita mengusahakan untuk memperoleh karunia-karunia spiritual yang lebih utama dan kepada perbuatan amal kasih yang lebih bermakna dan berhasil guna (bdk. 1 Kor 12:31-13:13). Menjadi peduli satu sama lain selayaknya menggerakkan kita kepada kasih yang bertambah dan lebih efektif di mana, ``seperti cahaya fajar, yang kian bertambah terang sampai rembang tengah hari'' (Ams 4:18), membuat kita hidup setiap hari sebagai antisipasi akan datangnya hidup kekal yang menantikan kita di dalam Tuhan. Waktu yang dikaruniakan kepada kita dalam hidup ini adalah berharga untuk menilai secara bijaksana dan menampilkan perbuatan-perbuatan yang baik dalam cinta kasih kepada Tuhan. Dengan cara ini, Gereja sendiri senantiasa tumbuh kepada kedewasaan penuh di dalam Kristus (bdk. Ef 4:13). Ajakan kita untuk mendorong satu sama lain untuk meraih kepenuhan cinta dan perbuatan baik berada di dalam prospek pertumbuhan yang dinamis ini.

Sayangnya, senantiasa ada godaan untuk menjadi suam-suam kuku, untuk memadamkan Roh, untuk menolak menanamkan berbagai talenta yang telah kita terima, demi kebaikan kita sendiri dan kebaikan sesama kita (lih. Mat 25:25–). Semua dari kita telah menerima kekayaan spiritual atau material yang dimaksudkan untuk digunakan bagi kepenuhan rencana Allah, demi kebaikan Gereja dan demi keselamatan kita sendiri (bdk. Luk 12:21b; 1 Tim 6:18). Pakar-pakar rohani mengingatkan kita, bahwa dalam kehidupan beriman, mereka yang tidak bertumbuh akan dengan sendirinya mengalami kemunduran. Saudara dan saudari yang terkasih, marilah kita menerima undangan ini, hari ini, seperti tak ada waktu lain yang lebih baik, untuk menuju ke ``standar yang tinggi dari kehidupan Kristiani'' (Novo Millennio Ineunte, 31). Kebijaksanaan Gereja dalam mengenali dan memproklamasikan orang-orang Kristen tertentu yang luar biasa sebagai Yang Terberkati dan para Santo/a juga dimaksudkan untuk menginspirasi sesama agar mencontoh kebajikan mereka. Santo Paulus menghimbau kita untuk ``saling mendahului dalam memberi hormat'' (Rom 12:10).

Dalam dunia yang menuntut dari umat Kristen sebuah kesaksian yang diperbaharui akan cinta dan kesetiaan kepada Tuhan, kiranya kita semua merasakan kebutuhan yang mendesak untuk saling mendahului dalam berbuat amal kasih, pelayanan dan pekerjaan-pekerjaan baik (bdk. Ibr 6:10). Permohonan ini terutama ditekankan dalam bulan yang suci ini sebagai persiapan Paskah. Sebagaimana saya menaikkan harapan-harapan yang baik dalam doa-doa saya demi masa Prapaskah yang penuh berkat dan menghasilkan banyak buah, saya mempercayakan Anda semua dalam perantaraan doa Bunda Maria Tetap Perawan dan dengan penuh kehangatan saya memberikan Berkat Apostolik saya.

\sumber{Dari Vatikan, 3 November 2011\\
Bapa Paus Benediktus XVI}
\normalsize