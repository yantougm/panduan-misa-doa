\chap{Sakramen Perkawinan: sebuah usaha untuk memahami secara ``baru''}

\begin{center} \textit{Blasius Slamet Lasmunadi, Pr} \end{center}

\small
    Usaha memahami Sakramen Perkawinan secara ``baru'' ini saya gulirkan
    dalam beberapa tulisan. Mengapa ``secara baru'', karena gagasan ini
    saya tuangkan dalam rangka memperdalam gagasan retret para imam
    Keuskupan Purwokerto (10-14 Nov 2008) di Purwokerto bersama Fr Fio
    Mascarenhas dari India, yang menekan pentingnya relasi umat beriman
    dengan Tritunggal Mahakudus.
    	
    Saya menempatkan status diri dalam tulisan ini sebagai seorang
    imam dan sebagai  seorang anak dalam keluargaku. Karena itu tulisan
    saya ini barangkali banyak bernuansa teologis daripada praksis.
    Justru karena kurang praksis, terbukalah kesempatan untuk Anda
    semua, untuk mengkritik tulisan ini atau memberi komentar apapun,
    juga yang ``nakal'' sekalipun dipersilakan.
    Tulisan ini dibagi menjadi 2 bagian: (A) Suami isteri sebagai mitra
    kerja Allah dan (B) Peran suami isteri sebagai imam, nabi dan raja.
\normalsize

\section{Suami isteri sebagai ``mitra rekan kerja Allah''}
    Dalam Perayaan Sakramen Pernikahan, kita sering mendengarkan Sabda
    Tuhan yang diwartakan seperti ini, ``\textit{Sebab pada awal dunia, Allah
    menjadikan mereka laki-laki dan perempuan,sebab itu laki-laki akan
    meninggalkan ayahnya dan ibunya dan bersatu dengan
    isterinya,sehingga keduanya itu menjadi satu daging. Demikianlah
    mereka bukan lagi dua, melainkan satu. Karena itu, apa yang telah
    dipersatukan Allah, tidak boleh diceraikan manusia (Mrk 10:6-9)}''
    
    Sabda Tuhan yang menegaskan kebersatuan suami isteri itu dan sifat
    monogaminya, juga ditegaskan dalam Kitab Hukum Kanonik 1983, kan. 1056:\\
    ``\textit{Ciri-ciri hakiki perkawinan ialah unitas (kesatuan=monogami) dan
    indissolubilitas/ (sifat tak-dapat-diputuskan), yang dalam
    perkawinan kristiani memperoleh kekukuhan khusus atas dasar sakramen.}''
    
    Sifat monogami dan sifat tak dapat diputuskan itu tentulah
    dimengerti oleh para calon suami isteri sebelum mereka mengucapkan
    kesepakatan janji nikah. Janji nikah yang diucapkan pria dan wanita
    yanga dibaptis, dan diucapkan di hadapan Allah dan Gereja, mereka
    berdua telah ``saling menerimakan sakramen perkawinan''. Kesepakatan
    nikah pria dan wanita yang dibuat  dengan tahu, sadar dan bebas dari
    segala paksaan apapun, adalah keputusan untuk ``menjadi mitra Allah''
    dalam karya keselamatan-Nya.

    Suami isteri menjadi ``mitra Allah'' dengan hidup dalam persekutuan
    sebagai ``Gereja keluarga''. Apakah artinya ``persekutuan'' bagi suami
    isteri? Artinya, saat mengucapkan janji nikah di hadapan Allah dan
    Gereja, suami isteri saling ``menukar'' hidup dan pribadinya. Suami
    menyatakan ``engkau isteriku, seluruh dirimu kugantikan dengan
    diriku''. Demikian juga isteri bersedia ``engkau suamiku, seluruh
    dirimu kugantikan dengan diriku''. Maka dengan pertukaran itu, suami
    dapat memandang dan memperlakukan isterinya, sebagai ``dirinya
    sendiri'' , sebaliknya begitu. Dengan kata lain, suami isteri saling
    mengarahkan jerih payahnya untuk hidup pasangannya, bukan hidup
    dirinya sendiri. Itulah ``mengasihi sesama seperti dirinya sendiri''
    dalam keluarga. Kasih antar sesama itu dapat menjadi ``tanda kasih''
    yang hidup dari kesetiaan kasih Allah kepada manusia.
    
    Allah Bapa tidak menyesal menciptakan manusia, meskipun Adam dan
    Hawa, akhirnya jatuh dalam dosa asal. Keturunan merekapun satu per
    satu, bergantian, turun temurun, dari generasi satu ke generasi yang
    lain, mewarisi dosa asal. Maka setelah melalui sejarah yang
    berliku-liku, Allah mengutus Putera-Nya yang tunggal, Yesus untuk
    hidup dan tinggal bersama manusia.

    Kristus itulah yang melaksanakan tugas untuk menebus dosa manusia,
    dengan hidup dan wafat-Nya di kayu salib. Tugas itu dilaksanakan
    dengan sempurna oleh Kristus sehingga Allah tidak segan untuk
    meninggikan ``Dia di atas segala nama'', dengan membangkitkan-Nya pada
    hari ketiga. Kebangkitan itu menganugerahkan kebebasan sebagai
    anak-anak Allah. Akan tetapi kebebasan itu tidak serta merta
    ditanggapi manusia untuk hidup di dalam Roh, mengasihi Allah dan
    sesama, malahan kerap kali kebebasan itu disalahgunakan untuk
    ``bekerjasama dengan kuasa kegelapan dosa'', yakni hidup menurut
    daging. Karena itulah, Yesus mengutus Roh-Nya sendiri setelah 50
    hari kebangkitan-Nya agar manusia mampu memenangkan pertempuran
    antara kehendak untuk hidup dalam Roh dan kecenderungan hidup dalam
    kegelapan dosa.
    
    Dengan lain kata, keputusan pria dan wanita untuk hidup menikah,
    adalah buah Roh Kudus, yakni menggunakan kebebasannya sebagai anak
    Allah untuk mewujudkan panggilan dasarnya sebagai citra dan anak-Nya
    untuk mencintai seperti Allah mencintai manusia. Panggilan dasar itu
    diwujudkan dalam hidup pernikahan. Maka sakramen pernikahan
    memperbaharui buah buah sakramen pembaptisan. Buah sakramen
    pembaptisan, tidak hanya mendapat anugerah kebebasan sebagai anak
    Allah, melainkan juga memberi daya kekuatan untuk menggulirkan
    kebebasan itu dalam tiga perannya: sebagai imam, nabi dan raja.

     

\section{Tiga peran Suami Isteri dalam kemitraan dengan Allah}

    Suami isteri kristiani sebagai orang yang dibaptis telah
    dipercaya menjadi anak-Nya sekaligus ahli waris. Karena itu mereka
    dipanggil untuk menjadi \textbf{imam}, \textbf{nabi} dan \textbf{raja}.
     
    \textbf{Sebagai imam}, suami isteri dipanggil untuk membangun relasi yang
    intim dengan Allah. Relasi itu dibangun dengan ``merayakan iman'' dan
    ``mewujudkan iman dalam tindakan kasih.'' Tugas merayakan iman adalah
    kesediaan untuk berdoa: berbicara dengan Tuhan dalam berbagai macam
    kesempatan. Termasuk juga, yang harus dibuat, belajar minta Roh
    Kudus kepada Allah Bapa karena Roh Kudus tidak otomatis
    dianugerahkan kepada kita, melainkan Ia akan hadir dan terlibat
    dalam hidup kita.

\begin{quote}
    \textit{Jadi jika kamu yang jahat tahu memberi pemberian yang baik kepada
    anak-anakmu, apalagi Bapamu yang di sorga! Ia akan memberikan Roh
    Kudus kepada mereka yang meminta kepada-Nya (Luk 11, 13)}
\end{quote}


    Roh Kudus sudah hadir di tengah tengah kita, namun bagaimana kita
    mampu mengalami karya Roh itu kalau tidak membuka diri. Ibarat
    bagaikan orang yang mencari sinar matahari di pagi hari sampai
    siang, padahal dia terus menerus tinggal di gua dan tidak pernah mau
    keluar dari gua itu. Maka, penting dan mendesak, jangan ragu-ragu
    untuk meminta Roh Kudus kepada Bapa agar terlibat membantu
    memberikan pencerahan di saat banyak kesulitan.

    Sikap hidup ``yang melibatkan Roh'' itu pasti akhirnya menantang
    suami isteri untuk  melepaskan diri dari ketergantungan terhadap 
    fasilitas yang nampaknya dapat diandalkan. Dengan lain kata,
    melibatkan Roh dalam hidup bersama, berarti  jerih payah apapun
    suami isteri dapat menjadi korban persembahan bagi Tuhan kalau
    dilaksanakan demi kepentingan terwujudnya buah-buah Roh, ``\textit{kasih,
    sukacita, damai sejahtera, kesabaran, kemurahan, kebaikan,
    kesetiaan, kelemahlembutan, penguasaan diri.' (Gal 5:22-23)}''
    
    Sebaliknya jerih payah suami isteri, bahkan yang kelihatan luhur
    sekalipun tidak akan menjadi "kurban persembahan bagi Allah" kalau
    dilaksanakan demi "kepentingan sendiri" atau demi kepentingan
    "daging". Karena hidup dalam daging, ``\textit{percabulan, kecemaran, hawa
    nafsu,  penyembahan berhala, sihir, perseteruan, perselisihan, iri
    hati, amarah, kepentingan diri sendiri, percideraan, roh
    pemecah,kedengkian, kemabukan, pesta pora dan sebagainya (Gal 5: 21)}''

    Dengan kata lain \textbf{peran sebagai imam menuntut peran sebagai "raja"},
    yang memiliki sikap "proaktif untuk melayani sesamanya". Mereka
    tidak akan berbangga kalau menjadi pribadi yang suka disapa, atau
    jadi pribadi yang ditakuti pasangan hidup atau anaknya sendiri. 
    Allah sendiri menganugerahkan Roh sebagai anak Allah bukan roh
    perbudakan yang membuat kita takut."\textit{Semua orang, yang dipimpin Roh
    Allah, adalah anak Allah.Sebab kamu tidak menerima roh perbudakan
    yang membuat kamu menjadi takut lagi, tetapi kamu telah menerima Roh
    yang menjadikan kamu anak Allah. Oleh Roh itu kita berseru: 'ya
    Abba, ya Bapa!' (Rm 8:14-15)}"  Dengan keyakinan Santo Paulus ini,
    suami isteri dipanggil untuk menampilkan hidup sebagai anak Allah.
    Hidup sebagai anak Allah selalu terarah pada kepentingan Bapa, dan
    bukan kepentingan harga diri sendiri. Maka, suami isteri mesti
    belajar untuk dinilai dan dikritik oleh pasangannya. Kalau keliru,
    belajar cepat meminta maaf, tidak malahan membela diri dan
    berargumentasi bahwa dirinya benar. Kalaupun benar pendapatnya,
    lebih baik mengatakan, "Terima kasih atas kritikanmu! Iya, bisa jadi
    saya keliru, meski sekarang saya yakin pendapatku ini benar!"
    Keterbukaan seperti itulah, yang meningkatkan kualitas pribadi yang
    siap untuk diubah oleh Roh Kudus.  Dengan semangat itu, suami isteri
    dapat mewujudkan sakramen perkawinan: sebagai tanda kehadiran cinta
    Tuhan yang nyata, yakni,
\small
\begin{enumerate}[label=(\roman*) ]
\item menjadi tanda cinta Allah Bapa Sang Pencipta dan pemelihara
    hidup melalui prokreasi, merawat dan mendidik anak sampai mandiri,

\item menjadi tanda kasih Yesus yang menebus dosa manusia dengan
    mengampuni satu sama lain, tidak menghakimi, namun belajar untuk
    mengubah kelemahan pasangan menjadi kesempatan untuk berefleksi dan

\item belajar untuk menjadi tanda kehadiran Roh Kudus yang menyertai
    kita sepanjang hidup, dengan belajar mendengarkan dan berkanjang
    bersama: tidak saling melempar kesalahan, tidak saling melempar
    tanggung jawab, melainkan belajar setia, yakni sehati seperasaan
    dalam suka dan duka.
\end{enumerate}
\normalsize
    Ketiga tindakan itu berwarna Trinitaris, maka ketiga tindakan itu
    tidak terpisahkan. Tidak cukup pasutri hanya prokreasi dan mendidik
    anak tanpa pengampunan dan solider antara suami isteri dan antar
    orang tua dan anak. Dengan cara hidup macam seperti suami isteri
    menjadi "tanda cinta yang hidup dari kesetiaan Allah kepada manusia".
    Akan tetapi bagaimana penghayatan itu sampai pada kenyataan kalau
    suami isteri kurang membuka diri kepada sabda Allah. \textbf{Karena itu
    peran sebagai imam dan raja, mesti didukung dengan peran sebagai
    nabi, yang bersedia mendengarkan sabda Allah dan melaksanakan dalam
    hidup setiap hari.} Sabda Tuhan itu adalah roh dan kehidupan. Maka
    suami isteri ditantang untuk hidup dari Sabda agar mereka memiliki
    "roh dan kehidupan" karena ``\textit{Perkataan-perkataan yang Kukatakan
    kepadamu adalah roh dan hidup.}'' (Yoh 6:63). itulah sebabnya Petrus
    pun setia mengikuti kemana Yesus pergi karena "Perkataan-Mu adalah
    perkataan hidup yang kekal"/ (Yoh 6:68). Kata-kata Yesus itu sendiri
    meneguhkan kita semua, agar tidak lagi ragu-ragu untuk setia
    mendengarkan Sabda Tuhan agar kita mengenal siapa Kristus, dan
    terlebih agar kita memiliki roh dan hidup!! Maka suami isteri
    ditantang untuk sungguh berperan sebagai "nabi": menjadi tanda
    kehadiran Allah yang bersabda bagi pasangannya, anak-anaknya dan
    saudara-saudarinya.
    
    Dengan penghayatan begitulah, pasutri membawa hidupnya dalam
    persembaan di altar dalam ekaristi. Hidupnya dengan segala kerapuhan
    dan kelemahan dipersembahkan bersama kurban Kristus, agar saat
    komuni terjadilah "pertukaran ilahi": Kristus hadir dalam diri suami
    isteri  untuk menerima hati mereka dengan segala keletihan dan rasa
    lesu serta beban berat, dan menggantikannya dengan  Tubuh dan
    Darah-Nya, agar setelah ekaristi, hidup mereka dalam keluarga
    sungguh menampilkan hidup Kristus yang setia pada Gereja-Nya. 
    Karena itu Kristus yang setia pada Gereja-Nya membutuhkan suami
    isteri untuk bekerjasama, agar kesetiaan Kristus tampak bagi dunia. 
    Di situlah tugas suami isteri, "menampakkan" kesetiaan kasih Kristus
    bagi dunia.
    
    Dengan "menampakkan kesetiaan" itu dalam hidup bersama yang diwarnai
    kasih, suami isteri menjadi tanda "pertukaran ilahi" antara Kristus
    dengan manusia. Itulah "pertukaran" yang menjadi ciri khas
    "persekutuan suami isteri monogami dan tidak terceraikan". Semoga
    makin banyak pasutri kristiani yang menjadi tanda kasih Allah yang
    hidup bagi dunia.

    Salam hangat untuk pasutri dan keluarga kristiani di manapun berada.

    \sumber{Blasius Slamet Lasmunadi, Pr\\http://www.imankatolik.or.id/}
\normalsize