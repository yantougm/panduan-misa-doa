\chap{Ciri-ciri Perkawinan Katolik}

Sebagai penggambaran persatuan ilahi antara Kristus dengan Gereja-Nya, Perkawinan Katolik mempunyai tiga ciri yang khas, yaitu ikatan 
\begin{enumerate}
\item yang \textbf{terus berlangsung seumur hidup}, 
\item \textbf{monogami}, yaitu satu suami, dan satu istri, 
\item yang \textbf{tidak terceraikan}. 
\end{enumerate}
Sifat terakhir inilah yang menjadi ciri utama perkawinan Katolik. Di dalam ikatan Perkawinan ini, suami dan istri yang telah dibaptis menyatakan kesepakatan mereka, untuk saling memberi dan saling menerima, dan \textbf{Allah sendiri memeteraikan kesepakatan ini}. Perjanjian suami istri ini digabungkan dengan perjanjian Allah dengan manusia, dan karena itu cinta kasih suami istri diangkat ke dalam cinta kasih Ilahi. Atas dasar inilah, maka Perkawinan Katolik yang sudah diresmikan dan dilaksanakan tidak dapat diceraikan. Ikatan perkawinan yang diperoleh dari keputusan bebas suami istri, dan telah dilaksanakan, tidak dapat ditarik kembali. \textbf{Gereja tidak berkuasa untuk mengubah penetapan kebijaksanaan Allah ini}.

Karena janji penyertaan Allah ini, dari ikatan perkawinan tercurahlah juga berkat-berkat Tuhan yang juga menjadi persyaratan perkawinan, yaitu berkat untuk menjadikan perkawinan tak terceraikan, berkat kesetiaan untuk saling memberikan diri seutuhnya, dan berkat keterbukaan terhadap kesuburan akan kelahiran keturunan. \textbf{Kristus-lah sumber rahmat dan berkat ini}. Yesus sendiri, melalui sakramen Perkawinan, menyambut pasangan suami istri. Ia tinggal bersama-sama mereka untuk memberi kekuatan di saat-saat yang sulit, untuk memanggul salib, bangun setelah jatuh, saling mengasihi dan mengampuni.

Maka, apa yang dianggap mustahil oleh dunia, yaitu setia seumur hidup kepada seorang manusia, menjadi mungkin di dalam Perkawinan yang mengikutsertakan Allah sebagai pemersatu. Ini merupakan kesaksian Kabar Gembira yang terpenting akan \textbf{kasih Allah yang tetap kepada manusia}, dan bahwa para suami dan istri mengambil bagian di dalam kasih ini. Betapa kita sendiri menyaksikan bahwa mereka yang mengandalkan Tuhan dalam perjuangan untuk saling setia di tengah kesulitan dan cobaan, sungguh menerima penyertaan dan pertolonganNya pada waktunya. Hanya kita patut bertanya, sudahkah kita mengandalkan Dia?