\chap{Prosedur Pernikahan Gereja Katolik}
\setlist{noitemsep}

\section{Tahap pertama}
\begin{enumerate}
\item   Pendaftaran pernikahan di Gereja melalui sekretariat pada paroki masing-masing pada hari kerja (hari kerja dan waktu buka seketariat disesuaikan masing-masing paroki
\item    Membawa surat pengantar dari lingkungan calon mempelai (baik pria dan wanitanya). Dalam hal ini surat pengantar untuk mengikuti KPP (Kursus Persiapan Perkawinan)
\item    Membawa fotokopi surat baptis yang diperbaharui :
\begin{enumerate}
    \item    Katolik dengan non Katolik - salah satu calon mempelai yang beragama Katolik
    \item    Katolik dengan Katolik – kedua calon mempelai wajib melampirkannya
\end{enumerate}

   Surat baptis yang diperbaharui berlaku 6 bulan samapai dengan hari H (pernikahannya)

\item    Membawa pasfoto 3x4 masing-masing 3 lembar
\item    Menyelesaikann biaya administrasi KPP (Kursus Persiapan Pernikahan), besar biaya disesuaikan paroki masing-masing. Hal-hal yang berkaitan dengan pendaftaran KPP, bisa ditanyakan di sekretariat masing-masing paroki.
\end{enumerate}

\section{Tahan Kedua}
\begin{enumerate}
\item   Selesaikan prosedur Tahap Pertama
\item    Mengisi formulir dan menyerahkan berkas-berkas pernikahan,
    yaitu:
\begin{enumerate}
\item        Surat pengantar dari lingkungan masing--masing
\item        Sertifikat Kursus Persiapan Pernikahan yg asli dan fotokopinya
\item        Surat baptis asli yang telah diperbaharui
\item        Foto berwarna berdampingan ukuran 4x6 sebanyak 3 lembar
\item        Fotokopi KTP saksi pernikahan 2 (dua) orang yang Katolik
\end{enumerate}
\item    Kedua calon mempelai datang ke Romo ybs untuk melakukan pendaftaran penyelidikan kanonik (harus datang sendiri, tidak dapat diwakilkan)
\item    Bagi calon mempelai yang belum Katolik danlatau bukan Katolik, harap menghadirkan 2 (dua) orang saksi pada saat penyelidikan kanonik untuk menjelaskan status pihak yang bukan Katolik. Saksi adalah orang yang benar-benar mengenal pribadi calon mempelai yang bukan Katolik dan bukan anggota
    keluarga kandungnya.
\item    Apabila kedua calon mempelai dari luar Paroki/Gereja dimana domisili calon mempelai harap membawa surat delegasi/pelimpahan pemberkatan pernikahan dari Pastor/Romo setempat (tempat penyelidikan kanonik)
\end{enumerate}

\section{Pernikahan Catatan Sipil}
\begin{enumerate}
\item    Datang ke sekretariat Gereja sebulan sebelumnya untuk pengurusan pernikahan catatan sipil dengan membawa: (Bila catatan sipil dilakukan di Gereja setelah pernikahan)
\begin{enumerate}
\item        Surat pengantar dari Kelurahan untuk pendaftaran perkawinan
\item        Fotokopi KTP dan Kartu Keluarga kelurahan kedua belah pihak
\item        Fotokopi Akta Kelahiran kedua mempelai
\item        Fotokopi SKBRI (WNI). Jika tidak ada, bawa SKBRI/WNI orang tua
\item        Untuk umat keturunan - fotokopi surat ganti nama (Bila tidak ada, lampirkan surat ganti nama dari orangtua)  
\item        Pas foto berdampingan ukuran. 4 x 6 sebanyak 6 lembar
\end{enumerate}
\item    Akan dibuatkan pengumuman ke kantor Catatan Sipil sesuai KTP yang bersangkutan dari calon mempelai. (kebijakan ini tergantung catatan sipil setempat)
\item    Pada hari "H", Akta Kelahiran asli kedua mempelai dan Surat Pemberkatan Nikah Gereja diserahkan kepada petugas Catatan Sipil
\item    Pencatatan pernikahan sipil bisa diurus oleh mempelai sendiri atau oleh pihak Gereja.
\end{enumerate}
