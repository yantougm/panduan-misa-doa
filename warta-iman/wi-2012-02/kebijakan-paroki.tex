\chap{Kebijakan Paroki Tentang Pernikahan Pada Masa Khusus}
    Pada prinsipnya gereja dilarang merayakan misa ritual pada hari Minggu selama masa khusus. Aturan ini tercantum dalam \textit{Misale Romanum} terbaru art. 372. beberapa hal yang harus diperhatikan melalui pernyataan di atas adalah:

\begin{enumerate}
\item    Misa ritual adalah perayaan yang berkaitan dengan sakramen (mis: pernikahan) atau sakramentali (pemberkatan rumah).
\item  Masa khusus meliputi:
\begin{enumerate}
\item    Adven
\item    Rabu Abu
\item    Prapaskah
\item    Pekan Suci (Minggu Palma - Kamis Putih - Jumat Agung -  Sabtu Suci -Malam Paskah - Minggu Paskah)
\item     Minggu Palma
\item Kamis Putih

\item Jumat Agung
\item Sabtu Suci
\item Malam Paskah
\item Paskah
\item OktafPaskah
\item Peringatan arwah semua orang beriman (setiap tgl. 02 November)          
\end{enumerate}
\end{enumerate}


    Berdasarkan makna dan suasana masa khusus dari dua dokumen liturgi, yaitu: \textit{Misale Romanum} dan \textit{Litterae Circurales De Festis Paschalibus Praeparandis et Celebrands}, biasanya ada kebijakan (tergantung paroki setempat) berkaitan dengan perayaan upacara pernikahan, sbb:
\begin{enumerate}
\item    Dalam masa Adven dan Prapaskah masih diijinkan untuk melangsungkan upacara pernikahan dengan memperhatikan kesederhanaan. Ukuran kesederhanaannya adalah:
\begin{enumerate}
\item \textbf{Masa Adven}

    Gereja

\begin{itemize}
\item        Hiasan bunga diijinkan hanya di sekitar altar.
\item        Tidak menggunakan karpet di lorong.
\item        Tidak ada hiasan bunga di sepanjang lorong menuju altar.
\item        Tidak ada hiasan bunga di pintu masuk gereja.
\item        Warna liturgi mengikuti masa yang berlaku
\end{itemize}

    Imam dan mempelai

\begin{itemize}
\item        Kasula imam berwarna putih.
\item        Mempelai diperkenankan membawa bunga tangan.
\item        Diperkenankan mempersembahkan bunga di patung Maria.
\end{itemize}

\item \textbf{Masa prapaskah}

    Gereja

\begin{itemize}
\item        Hiasan bunga TIDAK DIIJINKAN sama sekali dan diganti
\item        dengan dedaunan secukupnya di sekitar altar.
\item        Tidak menggunakan karpet di lorong
\item        Tidak ada hiasan bunga di sepanjang lorong menuju altar
\item        Tidak ada hiasan bunga di pintu masuk gereja
\item        Warna liturgi mengikuti masa yang berlaku
\item        Organ/alat musik lainnya hanya bersifat mengiringi lagu (tidak ada instrumental)
\item        Lagu-Iagu juga tidak sebanyak masa liturgi umum (dikonsultasikan dengan imam)
\end{itemize}

    Imam dan mempelai
\begin{itemize}
\item        Kasula imam berwarna putih
\item        Mempelai diperkenankan membawa bunga tangan
\item        Diperkenankan mempersembahkan bunga di patung Maria
\end{itemize}
\end{enumerate}

\item Dalam upacara Rabu abu, pekan suci, oktaf paskah, dan peringatan arwah semua orang beriman 2 November TIDAK DIIJINKAN untuk melangsungkan upacara pernikahan.

\item Kebijakan ini akan berubah (bersifat tentatif) setelah dokumen khusus tentang pernikahan dari KWI mendapat pengesahan dari Vatikan dan diberlakukan di Keuskupan-keuskupan di Indonesia.
\end{enumerate}