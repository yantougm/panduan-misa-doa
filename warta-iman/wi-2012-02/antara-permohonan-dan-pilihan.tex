\chap{\textit{Cerpen:}\\Antara Permohonan dan Pilihan}

	Jika anda sering menonton celotehan selebriti di televisi, maka anda akan menemukan kisah-kisah lucu dan paradoksal. Terutama menyangkut hidup rumah tangga mereka. Pada saat diwawancarai menjelang pernikahan, betapa keduanya saling memberi pujian setinggi gunung. Dengan yakinnya mereka mengatakan bahwa Tuhan telah memberikan jodoh yang baik, cocok, penuh pengertian dan setia. Kitapun akan mengamini. Tetapi tidak sampai setahun setelah hari pernikahan mereka, pasangan itu tahu-tahu hadir lagi di Pengadilan Agama untuk mengurus  perceraian. Saat dikerubuti wartawan, masing-masing mengaku bahwa mereka sudah tidak ada kecocokan lagi. Perceraian adalah jalan terbaik daripada masing-masing menderita batin.

	Hal itu pula yang akan dilakukan oleh teman saya Thomas Joko Prayitno. Ia berniat akan menceraikan Yohanna istrinya. Alasannya mirip para selebriti yang mau cerai, tidak ada keharmonisan lagi. Hatiku turut merasakan prihatin. Joko dan Yohanna adalah teman karibku semasa kuliah dulu hingga sekarang kami tetap akrab.

``Kamu serius Jok?'' Tanya saya.

``Ya, setelah aku renung-renungkan, lebih baik kami cerai.'' Jawab Joko mantap. ``Seumpama piring, keluargaku adalah piring yang retak. Jadi tidak mungkin disatukan lagi.'' Lanjut Joko sambil menghela nafas panjang dan menggeleng-gelengkan kepalanya.

Saya tersenyum. ``Perumpamaan itu keliru besar. Hanya orang `\textit{p\'{e}thuk}'  yang mau menyamakan keluarganya sama dengan sebuah piring. ''

Mulut Joko ternganga. Ia kaget. ``Kamu menghina aku, Gung?'' bentak Joko emosi.

``Tenang kawan, sabar, sorry aku tidak bermaksud menghina kamu. Izinkan aku memberi penjelasan. Di dalam sebuah piring itu tidak ada Allah. Piring itu benda mati. Begitu pecah di buang ke keranjang sampah. Masih ingat dalam Injil ada Sabda yesus yang bunyinya begini: tetapi tentang kebangkitan orang-orang mati tidaklah kamu baca apa yang difirmankan Allah, ketika Ia bersabda: Akulah Allah Abraham, Allah Ishak dan Allah Yakub? Ia bukanlah Allah orang mati, melainkan Allah orang hidup. Jadi Allah kita tidak berkarya di benda mati seperti piring. Tetapi Dia berkarya di dalam diri kita, di dalam keluarga kita. Mengapa? Karena kita masih hidup, berarti masih punya jiwa, masih punya hati.'' Jelasku.

``Tetapi bagaimana jika dua hati sudah tidak bisa disatukan lagi?'' protesnya.

``Kamu masih ingat saat menerima Sakramen perkawinan di altar?'' tanyaku.

``Ya. Yang sudah dipersatukan oleh Allah tidak bisa diceraikan oleh manusia. Bukankah itu yang ingin kamu katakan? Ayat dari Kitab Suci itu memang bagus tujuannya, yakni hanya ada perkawinan sekali dalam hidup manusia. Tetapi bagaimana jika kami sudah tidak bisa harmonis lagi, tak ada kecocokan lagi. Setiap hari yang ada hanya cekcok mulut?'' tegas Joko.
Joko sedang terbakar emosinya. Seseorang dalam kondisi kejiwaan seperti itu tidak butuh nasehat. Justru nasehat bisa berubah menjadi minyak yang ditumpahkan ke bara api. Api akan membesar dan membakar semuanya.

``Hari Sabtu besok mau ikut saya?'' tanyaku setelah kami diam agak lama.

``Kemana?''

``Sudah lama saya ingin ke Sendangsono. Sudah lama saya tidak sowan Bunda Maria. Kangen rasanya.''
Joko mengangguk-angguk. Dan ia setuju mau ikut. Kami berangkat Sabtu pagi.

\section*{Jalan Salib penuh kenangan}

Hari itu saya dan Joko memulai jalan salib dari pemberhentian ke-1 di Gereja Promasan. Ajakannya untuk jalan salib rute pendek di Sendangsono saya tolak. Sebab itu jalan salibnya orang malas. Jalan salib yang tidak memberi inspirasi pada tubuh untuk sejenak ikut merasakan penderitaan Tuhan Yesus.
Menjelang, pemberhentian ke-VII kami berhenti. ``Masih ingat, apa yang kamu lakukan disini 18 tahun yang lalu?'' tanyaku pada Joko.

Joko diam. Terbayang kembali dalam ingatannya saat itu kondisi jalan untuk jalan salib di Sendangsono belum disemen seperti sekarang, masih dari tanah lempung, kalau hujan tanahnya akan ambles jika diinjak.  Malam itu Yohanna hampir jatuh, karena tanah yang diinjaknya ambles cukup dalam sehingga Yohanna tidak bisa lagi menjaga keseimbangan badannya hampir jatuh, lalu aku cepat-cepat merengkuh tangannya dan memeluknya erat-erat.

``Jok, masih ingat apa yang terjadi disini 18 tahun yang lalu?'' tanyaku lagi kepada Joko.

``Saat itu Yohanna hampir jatuh, lalu aku cepat-cepat merengkuh tangannya kemudian kupeluk erat-erat.'' Jawab Joko pelan.

Jalan salib kami lanjutkan lagi. Persis di pemberhentian ke-XII saya bertanya lagi, ``Masih ingat apa yang terjadi disini 18 tahun yang lalu?''
Joko mengangguk.

``Ada peristiwa apa?''

``Yohanna jatuh pingsan dalam pelukanku. Dia lalu `kubopong' dan kutidurkan di belakang warung itu. Setelah siuman, pelan-pelan dia `kupapah' sampai di Sendang. Meski tampak lemah, tapi dia kuat sampai disana.'' Kata Joko pelan.

Aku tersenyum. Doa jalan salib kami lanjutkan sampai di Sendang, kami bersimpuh di depan patung Bunda Maria. Selesai berdoa saya bertanya pada Joko, ``Apa yang kamu doakan 18 tahun yang lalu disini?''
Joko tidak segera menjawab. Ia mengambil nafas dalam-dalam, lalu berkata lirih, ``Aku minta Bunda Maria agar beliau mau menjadi perantara permohonanku kepada Tuhan Yesus.''

``Apa yang kamu mohon pada Tuhan Yesus?''

``Agar Dia mengijinkan Yohanna menjadi jodohku, menjadi isteriku. Menjadi ibu dari anak-anakku.''

``Dan Tuhan Yesus meluluskan permohonanmu, tidak?''

``Ya \ldots ya \ldots tetapi \ldots'' Joko tidak melanjutkan kalimatnya. Ada rombongan peziarah lain datang memenuhi pelataran di depan patung Bunda Maria.

\section*{Doa Peneguhan}

Seminggu kemudian, tanpa sepengetahuan Joko, Yohanna yang sudah kuanggap adik sendiri itu saya ajak ke Sendangsono. Kepada Yohanna saya ajukan pertanyaan yang sama seperti halnya Joko pada saat jalan salib. Dan jawabannya pun hampir sama mirip dengan jawaban suaminya Joko.

``Jadi Joko itu jodoh yang kamu minta kepada Tuhan Yesus dengan perantaraan  Bunda Maria?'' tanyaku pada Yohanna setelah selesai berdoa di depan patung Bunda Maria.
Yohanna mengangguk. 

Dua minggu kemudian keduanya saya ajak bersama-sama ke Sendangsono. Kami doa jalan salib bertiga, juga mulai dari pemberhentian ke-1 di Gereja Promasan. Menjelang pemberhentian ke-VII saya minta Yohanna untuk pura-pura jatuh seperti kejadian 18 tahun yang lalu. Dan Joko harus siap-siap menolongnya. Adegan itu berjalan dengan mulus. Yohanna jatuh ke dalam pelukan suaminya.

Di pemberhentian ke XII kembali saya minta Yohanna untuk pura-pura pingsan. Tetapi dia tidak mau. ``Malu!'' jawabnya. Namun saya terus membujuknya, sedikit mengancam bahwa saya akan pulang sendiri jika dia tetap menolak. Akhirnya yohanna menyerah. Dia pura-pura pingsan, lalu Joko membopong dan menidurkannya di belakang warung. Setelah itu Joko merangkul isterinya menuju ke pemberhentian ke-XV, lalu berdoa di depan patung Bunda Maria.

``Sekarang berdoalah seperti doa kalian 18 tahun yang lalu.'' Kata sayakemudian. ``Jangan berbohong. Sebab Bunda Maria dan Tuhan Yesus melihat kalian. Silahkan.'' Kata saya sambil mundur lalu duduk dibawah pohon beringin.
Saya tidak tahu apa yang mereka doakan. Sebab saya sendiri berdoa, memohon kepada Tuhan Yesus lewat perantaraan Bunda Maria, agar pasangan suami-isteri ini mengurungkan niatnya untuk berpisah. Bahtera oleng itu sudah biasa, namun jangan sampai menabrak karang dan hancur berkeping-keping.

Dua puluh menit kemudian saya dekati mereka berdua, saya sodorkan selembar kertas yang berisi doa tulisan tangan. Doa itu saya buat malam menjelang kami ziarah bersama. ``Tolong bacalah doa ini bersama-sama. Pelan-pelan saja. Bunda Maria pasti mendengar, begitu juga Tuhan Yesus.'' Saya lalu mundur dan duduk dibelakang keduanya sekitar jarak tiga meter.

Inilah doanya:
\begin{quote}\textit{
Allah Bapa yang maha rahim, Kami berdua duduk bersimpuh di depan Bunda Maria, kami berdua berpasrah diri depan Putra-Mu, Tuhan kami Yesus Kristus. 18 tahun yang lalu ya Tuhan, kami mengetuk kemurahan hati-Mu, dengan perantaraan bunda Maria, agar Engkau berkenan menyatukan dua hati kami menjadi suami-isteri dalam rumah tangga Kristiani. Engkau maha pemurah, maha mendengar, sebab kami pun boleh menerima Sakramen Perkawinan. Disaksikan para malaikat dan orang kudus, kami berjanji di depan –Mu untuk sehidup semati, dalam untung dan malang, tak terceraikan oleh manusia dan hukum mana pun, kecuali oleh maut. 
\bigskip\\ 
Namun ternyata dalam perjalanan mengayuh biduk, kami diterpa badai, diamuk ombak, haruskah kami karam di telaga kehidupan yang keras? Jika hal itu terjadi ya Tuhan, berarti kami tidak menghormati Dikau, tidak menghargai pemberian-Mu yang tak ternilai harganya, yakni pasangan hidup kami. 
\bigskip\\ 
Karena itu ya Tuhan, di depan Bunda Maria, kami berjanji untuk saling menjaga, tetap saling mencintai pasangan hidup kami, dengan kelebihan dan kekurangannya. Kami sadar bahwa tidak ada yang sempurna di dunia ini, kecuali kasih-Mu yang Engkau curahkan tanpa batas kepada kami. Bunda Maria berkatilah anak-anakmu ini. Mohonkan kepada Tuhan Yesus semangat memperbaharui hidup iman dan hidup cinta kasih kami. Amin.}
\end{quote}

Sebulan kemudian saya tidak bertemu dengan Joko dan Yohanna karena tugas pekerjaan harus keluar kota. Joko dan Yohanna memberi kabar bahwa mereka telah meninggalkan rumah orangtuanya. Keduanya sepakat untuk hidup mandiri, mengontrak rumah sederhana dan kecil. Kehidupan baru dimulai kembali, semoga Tuhan selalu berkenan hadir dalam keluarga mereka juga dalam  keluarga kita semua. Amin.

\sumber{Medio Januari '12\\Bravo Sierra}
