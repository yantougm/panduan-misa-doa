\newpage
\chap{Kompendium Katekese Gereja Katolik}
\setcounter{kgkcounter}{25}
{\normalsize

\kgk{Siapa saksi-saksi utama ketaatan iman dalam Kitab Suci?}
     Ada banyak saksi-saksi macam itu, secara khusus kita melihat dua. Yang 
pertama, Abraham, ketika mengalami ujian, dia tetap ``percaya kepada Allah''
(Rom 4:3) dan selalu taat kepada panggilan-Nya. Karena itulah Abraham disebut
          ``Bapa kaum beriman'' (Rom 4:11.18). Contoh yang kedua, Santa Perawan Maria
          yang seluruh hidupnya menjadi kesaksian sempurna ketaatan iman: ``Terjadilah
          padaku menurut perkataanmu'' (Luk 1:38).

\kgk{Apa artinya percaya kepada Allah bagi  seseorang dalam praktek
              hidupnya?}
Artinya, setia kepada Allah, mempercayakan hidup kepada-Nya, dan mengamini semua kebenaran yang diwahyukan Allah karena Allah adalah Kebenaran.
          Ini berarti percaya kepada satu Allah dalam tiga Pribadi, yaitu Bapa, Putra, dan Roh
          Kudus.

\kgk{Apa ciri-ciri iman?}
Iman adalah keutamaan adikodrati yang mutlak perlu bagi keselamatan. Iman
adalah anugerah cuma-cuma dari Allah dan tersedia bagi semua orang yang dengan
          rendah hati mencarinya. Tindakan iman adalah tindakan manusiawi, yaitu tindakan
          dari intelek manusia -- terdorong oleh kehendak yang digerakkan oleh Allah -- yang
          dengan bebas mengamini kebenaran ilahi. Iman juga pasti karena mempunyai dasar
          pada Sabda Allah, iman bekerja ``oleh kasih'' (Gal 5:6); dan iman berkembang terus-menerus dengan mendengarkan Sabda Allah dan doa. Dengan iman, bahkan
          sekarang ini juga, orang mencecap kegembiraan surga.

\kgk{Mengapa tidak ada kontradiksi antara iman dan ilmu?}
Walaupun iman itu mengatasi akal budi, tidak pernah ada kontradiksi antara
          iman dan ilmu karena kedua-duanya berasal dari Allah. Allah sendirilah yang
          memberikan, baik terang akal budi maupun terang iman kepada kita.

\flushright{(\dots \emph{bersambung} \dots)}
}