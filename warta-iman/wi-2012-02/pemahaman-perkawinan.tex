\chap{Pemahaman Perkawinan Menurut Gereja Katolik}


\section*{Ajaran Gereja Katolik Tentang Perkawinan}

Ada begitu banyak pasangan calon mempelai yang sudah lama ber­pacaran, namun seringkali mereka belum mempergunakan kesempatan pacaran itu untuk dapat mempersiapkan diri dalam membangun keluarga katolik. Salah satu hal yang sangat penting namun seringkali terlupakan adalah kurangnya/ tidak pernah dilaksanakan pengolahan pengalaman hidup untuk melangsungkan suatu pernikahan sesuai ajaran Gereja Katolik. Oleh karena itu pentinglah, dalam membaca uraian di bawah ini, pembaca menggali pengalaman pribadi, khususnya ketika mempersiapkan perkawinannya. Rumusan ini bisa membantu un­tuk menilai diri sendiri, apakah memang sudah siap (minimal) secara mental dan rohani untuk melangsungkan perkawinan.

\noindent{Perkawinan adalah}
\begin{quote}
\begin{enumerate}[label=\Alph*.]
\item PERSEKUTUAN HIDUP  
\item ANTARA SEORANG PRIA DAN SEORANG WANITA  
\item YANG TERJADI KARENA PERSETUJUAN PRIBADI  
\item YANG TAK DAPAT DITA­RIK KEMBALI 
\item DAN HARUS DIARAHKAN 
\item KEPADA SALING MENCINTAI SEBAGAI SUAMI ISTERI 
\item DAN KEPADA PEMBANGUNAN KELUARGA 
\item DAN OLEH KARENANYA MENUNTUT KESETIAAN YANG SEMPURNA 
\item DAN TIDAK MUNGKIN DIBATALKAN LAGI OLEH SIAPAPUN, 
\item KECUALI OLEH KEMATIAN.
\end{enumerate}
\end{quote}

\section{Persekutuan hidup}
Apa yang pertama-tama kelihatan pada perkawinan Katolik? Jawabnya adalah: Hidup bersama. Namun, hidup bersama itu masih berane­karagam isinya. Dalam perkawinan Katolik, hidup bersama itu mewujudkan persekutuan. Jadi, hidup bersama yang bersekutu. Bersekutu mengisyaratkan adanya semacam kontrak, semacam ikatan tertentu dengan sekutunya. Bersekutu mengandaikan juga kesediaan pribadi untuk melaksanakan persekutuan itu, dan untuk menjaga persekutuan itu. Ada kesediaan pribadi untuk mengikatkan diri kepada sekutunya, dan ada kesediaan pribadi untuk memperkembangkan ikatannya itu supaya menjadi semakin erat.

Ikatan ini tidak mengurangi kebebasannya. Justru ikatan itu mengisi kebebasan orang yang bersangkutan. Pertama-tama karena para calon mempelai memilih sendiri untuk bersekutu, dan bebas un­tuk memilih mau bersekutu dengan siapa, memilih untuk terikat dengan menggunakan kebebasan sepenuhnya; tetapi juga karena kebebasan itu hanya dapat terlaksana dalam melaksanakan pilihannya untuk bersekutu ini. Dengan kata lain boleh dikatakan bahwa persekutuan itu membuat orang sungguh-sungguh bebas karena dapat memperkem­bangkan kreatifitas dalam memelihara dan mengembangkan persekutuan itu; bukan dengan menghadapkan diri pada pilihan-pilihan yang baru lagi. Persekutuan yang dibangun itu menjadi tugas kehidupan yang harus dihayatinya.


\section{Seorang pria dengan seorang wanita}
Penekanan pertama di sini adalah seorang dengan seorang: arti­nya orang seutuhnya dengan orang seutuhnya. Ini menggambarkan penerimaan terhadap satu pribadi seutuhnya. Yang diterima untuk bersekutu adalah pribadi, bukan kecantikan, kegantengan, kekayaan atau kepandaiannya saja. Ada beberapa catatan untuk penerimaan satu pribadi ini: Pertama, menerima pribadi itu berarti menerima juga seluruh latar belakang dan menerima seluruh masa depannya. Artinya, saya tidak dapat menerima pribadi itu hanya sebagai satu pribadi yang berdiri sendiri. Selalu, saya harus menerima juga orang tuanya, kakak dan adiknya, saudara-saudaranya, teman-teman­nya, bahkan juga bahwa dia pernah berpacaran atau bertunangan dengan si ini atau si itu. Lebih jauh lagi, saya juga harus meneri­ma segala sesuatu yang terjadi padanya di masa mendatang: syukur kalau ia menjadi semakin baik, tetapi juga kalau ia menjadi sema­kin buruk karena penyakit, karena ketuaan, karena halangan-halangan; saya masih tetap harus menerimanya. Yang ke dua, menerima pribadi berarti menerima dia apa adanya, dengan segala kelebihan dan kekurangannya. Kalau dipikir secara matematis: yang berseku­tu itu satu dengan satu; bukan $\frac{3}{4} + \frac{1}{2}$, atau $1 + \frac{6}{7}$; lebih-le­bih lagi, bukan 1 dengan $1\frac{1}{2}$, $1\frac{1}{4}$, atau $1\frac{3}{4}$, apalagi dengan 2, 3, dan seterusnya.

Dengan ungkapan lain lagi: Saya seutuhnya, mau mencintai dia seutuhnya/apa adanya. Ini berarti, saya mau menerima dia seutuh­nya, apa adanya; tetapi juga sekaligus saya mau menyerahkan diri seutuhnya kepadanya saja. Yang lain sudah tidak mendapat tempat lagi di hati saya, di pikiran saya. Hanya dia saja. Bahkan, anak-anakpun tidak boleh melebihi dia di hadapan saya, dalam pelayanan saya.
Penekanan ke dua pada seorang pria dengan seorang wanita.Yang ini kiranya cukup jelas. Hanya yang sungguh-sungguh pria dan yang sungguh-sungguh wanita yang dapat melaksanakan perkawinan secara katolik.


\section{Persekutuan pribadi}
Hidup bersekutu itu terjadi karena setuju secara pribadi. Yang harus setuju adalah yang akan menikah. Dan persetujuan itu dilakukan secara pribadi, tidak tergantung pada siapapun, bahkan juga pada pasangannya. Maka, rumusannya yang tepat adalah: “Saya setu­ju untuk melangsungkan pernikahan ini, tidak peduli orang lain setuju atau tidak, bahkan tidak peduli juga pasangan saya setuju atau tidak”.
“Lalu bagaimana kalau pasangan saya kurang atau bahkan tidak setuju?. Dia hanya pura-pura setuju”. Kalau demikian, bukankah pihak yang setuju dapat dirugikan? Ya, inilah resiko cinta sejati. Cinta sejati di sini berarti saya setuju untuk mengikatkan diri dengan pasangan, saya setuju untuk menyerahkan diri kepada pasangan, saya setuju untuk menjaminkan diri pada pasangan; juga kalau akhirnya persetujuan saya ini tidak ditanggapi dengan baik/sesuai dengan kehendak saya. Yang menjadi dasar pemahaman ini adalah karena setiap mempelai membawa cinta Kristus sendiri. Kristuspun tanpa syarat mengasihi kita, Kristus tanpa syarat menerima kita dan memberikan DiriNya bagi kita.

\section{Persetujuan pribadi yang tak dapat ditarik kembali}
Persetujuan pribadi untuk bersekutu itu nilainya sama dengan sumpah/janji dan bersifat mengikat seumur hidup. Sebab persetujuan itu mengikutsertakan seluruh kehendak, pikiran, kemauan, pera­saan. Pokoknya seluruh kepribadian. Maka dinyatakan bahwa perse­tujuan itu tidak dapat ditarik kembali. Sebab, penarikan kembali pertama-tama berarti pengingkaran terhadap diri sendiri, penging­karan terhadap kebebasannya sendiri, pengingkaran terhadap cita-cita dan kehendaknya sendiri. Tetapi, kemudian, juga berarti bah­wa pribadinya sudah tidak menjadi utuh kembali.

\section{Dan yang diarahkan}
Sebenarnya, pengalaman untuk membuat dan memelihara dan memper kembangkan persetujuan pribadi untuk bersekutu itu sudah harus dipupuk sejak masa pacaran Maka, ada banyak yang merasa bahwa persetujuan semacam itu sudah tidak perlu dipertanyakan lagi. Pokok­nya sudah beres, begitu. Semua sudah siap. Namun, kenyataannya persetujuan yang terjadi pada masa pacaran belumlah memenuhi sya­rat perkawinan. Dan benarlah, persetujuan yang dibangun pada masa pacaran baiklah persetujuan sebagai pacar. Persetujuan yang dibangun pada masa tunangan, baiklah persetujuan sebagai tunangan. Baru, setelah menikah, persetujuan itu boleh menjadi persetujuan sebagai suami-isteri. Maka, Kita lihat, misalnya adanya pembatasan-pembatasan dalam berpacaran, menunjukkan bahwa persetujuan itu belum bisa dilaksanakan sepenuhnya. Secara lebih positif dapat dikatakan bahwa persetujuan semasa pacaran lebih diarahkan untuk dapat melaksanakan janji pada saat perkawinan. Supaya janji pada saat perkawinan sungguh berisi dan memberi jaminan bagi masa de­pan baik pribadi maupun pasangannya. Tiga kata ini juga dapat diartikan penegasan terhadap perkawi­nan sebagai awal dari kehidupan baru bagi kedua mempelai. Bagai­manapun oleh perubahan situasi manusia masih dapat berubah. Pene­gasan ini membantu para suami/isteri untuk melaksanakan isi persetujuan itu.

\section{Saling mencintai sebagai suami isteri}
Pengalaman menunjukkan bahwa calon mempelai biasanya bingung dengan ungkapan ini. Mereka merasa sudah saling mencintai, kok masih ditanya soal ini. Masalahnya, sering tidak disadari bahwa cinta itu bermacam-macam. Ada cinta sebagai saudara, ada cinta sebagai sahabat, ada cinta karena belas kasihan, demikian pula ada cinta suami isteri. Tentu saja, yang namanya cinta sejati tidak pernah dapat berbeda-beda. Yesus menunjuk cinta sejati itu seba­gai orang yang mengorbankan nyawaNya bagi yang dicintaiNya. Dan Yesus memberi teladan dengan hidupNya sendiri yang rela sengsa­ra, bahkan sampai wafat untuk kita semua yang dicintaiNya. Na­mun, perwujudan cinta sejati itu ternyata bisa beranekaragam. Kekhasan dari cinta suami isteri adalah adanya keterikatan isti­mewa yang membuat mereka dapat menyerahkan diri seutuhnya bagi pasangannya. Dalam hal ini kiranya cinta suami isteri dapat diseja­jarkan dengan cinta yang diwujudkan dalam suatu kaul biara atau janji seorang imam. Bedanya, kalau kaul biara atau janji seorang imam tertuju kepada Tuhan di dalam umatNya; dalam perkawinan cinta itu tertuju kepada Tuhan di dalam pasangannya. Yang mau dituju adalah membangun suasana saling mencintai sebagai suami/isteri. Maka, tidak hanya membabi buta dengan cintanya sendiri. “Pokoknya saya sudah mencintai”. Ini tidak cukup. Perjuangan seorang suami/isteri adalah di samping memelihara dan memperkembangkan cintanya, juga mengusahakan supaya pasangannya da­pat ikut mengembangkan cintanya sebagai suami/isteri.

\section{Pembangunan keluarga}
Hidup dalam persekutuan sebagai suami-isteri mau tidak mau mewujudkan suatu keluarga. Harus siap untuk menerima kedatangan anak-anak, harus siap untuk tampil sebagai keluarga, baik di hadapan saudara-saudara, di hadapan orang tua maupun di hadapan masya­rakat pada umumnya. Maka, membangun hidup sebagai suami-isteri membawa juga kewajiban untuk mampu menghadapi siapapun sebagai satu kesatuan dengan pasangannya. Mampu bekerjasama menerima, meme­lihara dan mendewasakan anak, mampu bekerjasama menerima atau da­tang bertamu kepada keluarga-keluarga lain, mampu ikut serta mem­bangun Gereja. Semuanya dilaksanakan dalam suasana kekeluargaan.

\section{Kesetiaan yang sempurna}
Setia dalam hal apa? Empat hal yang sudah diuraikan di atas, yakni persekutuan hidup antara seorang pria dan seorang wanita, memelihara dan memperkembangkan persetujuan pribadi, membangun sa ling mencintai sebagai suami isteri, membangun hidup berkeluarga yang sehat. Tidak melaksanakan salah satunya berarti sudah tidak setia. Apalagi kalau kemudian mengalihkan perhatiannya kepada se­suatu yang lain: membangun persekutuan yang lain, membuat perse­tujuan pribadi yang lain, membangun hubungan saling mencintai sebagai suami isteri dengan orang lain, membangun suasana kekeluargaan dengan orang lain (juga saudara): Ini dosanya besar sekali
Satu pedoman untuk kesetiaan yang sempurna adalah Kristus sen­diri. Ia setia kepada tugas perutusanNya, Ia setia kepada Bapa­Nya, Ia setia kepada manusia, kendati manusia tidak setia kepadaNya.

\section{Tak dapat dipisahkan oleh siapapun}
Persekutuan perkawinan terjadi oleh dua pihak, yakni oleh sua­mi dan isteri. Maka, tidak ada instansi atau siapapun yang akan dapat memutuskan persetujuan pribadi itu. Bahkan suami isteri itu sendiripun tidak dapat memutuskannya, sebab persekutuan itu dibangun atas dasar kehendak Tuhan sendiri. Dan Tuhanlah yang merestuinya. Maka, pemutusan persekutuan perkawinan bisa dipandang sebagai pemotongan kehidupan pribadi suami/isteri. Ini bisa be­rarti pembunuhan, karena pribadi itu dihancurkan.

\section{Kecuali oleh kematian}
Pengecualian ini didengar tidak enak. Namun, nyatanya, misteri kematian tidak terhindarkan. Karena kematian yang wajar, persetu­juan pribadi itu menjadi batal, karena pribadi yang satu sudah tidak mampu lagi secara manusiawi melaksanakan persetujuannya. 

\sumber{http://www.imankatolik.or.id}