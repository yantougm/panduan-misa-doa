\chap{Halangan yang Menggagalkan Perkawinan}

\setcounter{section}{0}
\section{Kurangnya umur (bdk. kan 1083):}

Syarat umur yang dituntut oleh kodeks 1983 adalah laki-laki berumur 16 tahun dan perempuan berumur 14 tahun dan bukan kematangan badaniah. Tetapi hukum kodrati menuntut kemampuan menggunakan akalbudi dan mengadakan penilaian secukupnya dan ``\textit{corpus suo tempore habile ad matrimonium}''. Hukum sipil sering mempunyai tuntutan umur lebih tinggi untuk perkawinan dari pada yang dituntut hukum Gereja. Jika salah satu pihak belum mencapai umur yang ditentukan hukum sipil, Ordinaris wilayah harus diminta nasehatnya dan izinnya diperlukan sebelum perkawinan itu bisa dilaksanakan secara sah (bdk kan. 1071, \S1, no.3). Izin semacam itu juga harus diperoleh dari Ordinaris wilayah dalam kasus di mana orang tua calon mempelai yang belum cukup umur itu tidak mengetahui atau secara masuk akal tidak menyetujui perkawinan itu (bdk. kan 1071, \S1, no.6).

\section{Impotensi (bdk kan. 1084):}

Impotensi itu adalah halangan yang menggagalkan, demi hukum kodrati, dalam perkawinan. Sebab impotensi itu mencegah suami dan istri mewujudkan kepenuhan persatuan hetero seksual dari seluruh hidup, badan dan jiwa yang menjadi ciri khas perkawinan. Yang membuat khas persatuan hidup suami istri adalah penyempurnaan hubungan itu lewat tindakan mengadakan hubungan seksual dalam cara yang wajar. Impotensi yang menggagalkan perkawinan, haruslah sudah ada sebelum perkawinan dan bersifat tetap. Pada waktu perkawinan sudah ada, bersifat tetap maksudnya impotensi itu terus menerus dan bukan berkala, serta tidak dapat diobati kecuali dengan operasi tidak berbahaya. Impotensi ada dua jenis: bersifat absolut dan relatif. Impotensi absolut jika laki-laki atau perempuan sama sekali impotens. Impotensi relatif jika laki-laki atau perempuan tertentu ini tidak dapat melaksanakan hubungan seksual. Dalam hal absolut orang itu tidak dapat menikah sama sekali, dalam impotensi relatif pasangan tertentu juga tidak dapat menikah secara sah.

\section{Adanya ikatan perkawinan (bdk. kan 1085):}

Ikatan perkawinan terdahulu menjadi halangan yang menggagalkan karena hukum ilahi. Kan 1085, \S1: menghilangkan ungkapan “kecuali dalam hal privilegi iman” (Jika dibandingkan dengan kodeks 1917). Ungkapan ini berarti jika seorang yang dibaptis menggunakan privilegi iman walau masih terikat oleh ikatan perkawinan terdahulu, dia bisa melaksanakan perkawinan secara sah dan ketika perkawinan baru itu dilaksanakan ikatan perkawinan lama diputuskan.

\section{Disparitas cultus (bdk. kan 1086):}

Perkawinan antara dua orang yang diantaranya satu telah dibaptis dalam Gereja Katolik atau diterima di dalamnya dan tidak meninggalkannya dengan tindakan formal, sedangkan yang lain tidak dibaptis, adalah tidak sah. Perlu dicermati ungkapan “meninggalkan Gereja secara formal” berarti melakukan suatu tindakan yang jelas menunjukkan etikat untuk tidak menjadi anggota Gereja lagi. Tindakan itu seperti menjadi warga Gereja bukan Katolik atau agama Kristen, membuat suatu pernyataan di hadapan negara bahwa dia bukan lagi Katolik. Namun demikian janganlah disamakan tindakan itu dengan orang yang tidak pergi ke Gereja Katolik lagi tidak berarti meninggalkan Gereja. Ada dua alasan tentang norma ini: pertama karena tujuan halangan ini adalah untuk menjaga iman katolik, tidak ada alasan mengapa orang yang sudah meninggalkan Gereja harus diikat dengan halangan itu. Kedua, Gereja tidak mau membatasi hak orang untuk menikah.

Perkawinan yang melibatkan disparitas cultus (beda agama) ini, sesungguhnya tetap dapat dianggap sah, asalkan: 1) sebelumnya pasangan memohon dispensasi kepada pihak Ordinaris wilayah/ keuskupan di mana perkawinan akan diteguhkan. Dengan dispensasi ini, maka perkawinan pasangan yang satu Katolik dan yang lainnya bukan Katolik dan bukan Kristen tersebut tetap dapat dikatakan sah dan tak terceraikan; setelah pihak yang Katolik berjanji untuk tetap setia dalam iman Katolik dan mendidik anak-anak secara Katolik; dan janji ini harus diketahui oleh pihak yang non-Katolik (lih. kan 1125). 2) Atau, jika pada saat sebelum menikah pasangan tidak mengetahui bahwa harus memohon dispensasi ke pihak Ordinaris, maka sesudah menikah, pasangan dapat melakukan \textit{Convalidatio} (lih. kann. 1156-1160) di hadapan imam, agar kemudian perkawinan menjadi sah di mata Gereja Katolik.

\section{Tahbisan suci (bdk. kan. 1087):}

Adalah tidak sahlah perkawinan yang dicoba dilangsungkan oleh mereka yang telah menerima tahbisan suci.
Kaul kemurnian dalam suatu tarekat religius (bdk. kan. 1088):

Kaul kekal kemurnian secara publik yang dilaksanakan dalam suatu tarekat religius dapat menggagalkan perkawinan yang mereka lakukan.

\section{Penculikan dan penahanan (bdk. kan. 1089):}

Antara laki-laki dan perempuan yang diculik atau sekurang-kurangnya ditahan dengan maksud untuk dinikahi, tidak dapat ada perkawinan, kecuali bila kemudian setelah perempuan itu dipisahkan dari penculiknya serta berada di tempat yang aman dan merdeka, dengan kemauannya sendiri memilih perkawinan itu. Bahkan jika perempuan sepakat menikah, perkawinan itu tetap tidak sah, bukan karena kesepakatannya tetapi karena keadaannya yakni diculik dan tidak dipisahkan dari si penculik atau ditahan bertentangan dengan kehendaknya.

\section{Kejahatan (bdk. kan. 1090):}

Tidak sahlah perkawinan yang dicoba dilangsungkan oleh orang yang dengan maksud untuk menikahi orang tertentu melakukan pembunuhan terhadap pasangan orang itu atau terhadap pasangannya sendiri.

\section{Persaudaraan (konsanguinitas (bdk. kan. 1091):}

Alasan untuk halangan ini adalah bahwa perkawinan antara mereka yang berhubungan dalam tingkat ke satu  garis lurus bertentangan dengan hukum kodrati. Hukum Gereja merang perkawinan di tingkat lain dalam garis menyamping, sebab melakukan perkawinan di antara mereka yang mempunyai hubungan darah itu bertentangan dengan kebahagiaan sosial dan moral suami-isteri itu sendiri dan kesehatan fisik dan mental anak-anak mereka.

\section{Hubungan semenda (bdk. kan. 1092):}

Hubungan semenda dalam garis lurus menggagalkan perkawinan dalam tingkat manapun. Kesemendaan adalah hubungan yang timbul akibat dari perkawinan sah entah hanya ratum atau ratum consummatum. Kesemendaan yang timbul dari perkawinan sah antara dia orang tidak dibaptis akan menjadi halangan pada hukum Gereja bagi pihak yang mempunyai hubungan kesemendaan setelah pembaptisan dari salah satu atau kedua orang itu. Menurut hukum Gereja hubungan kesemendaan muncul hanya antara suami dengan saudara-saaudari dari isteri dan antara isteri dengan saudara-saaudara suami. Saudara-saudara suami tidak mempunyai kesemendaan dengan saudara-saudara isteri dan sebaliknya. Menurut kodeks baru 1983 hubungan kesemendaan yang membuat perkawinan tidak sah hanya dalam garis lurus dalam semua tingkat.

\section{Halangan kelayakan publik (bdk. kan. 1093):}

Halangan ini muncul dari perkawinan tidak sah yakni perkawinan yang dilaksanakan menurut tata peneguhan yang dituntut hukum, tetapi menjadi tidak sah karena alasan tertentu, misalanya cacat dalam tata peneguhan. Halangan ini muncul juga dari konkubinat yang diketahui publik. Konkubinat adalah seorang laki-laki dan perempuan hidup bersama tanpa perkawinan atau sekurang-kurangnya memiliki hubungan tetap untuk melakukan persetubuhan kendati tidak hidup bersama dalam satu rumah. Konkubinat dikatakan publik kalau dengan mudah diketahui banyak orang.

\section{Adopsi (bdk. kan. 1094):}

Tidak dapat menikah satu sama lain  dengan sah mereka yang mempunyai pertalian hukum yang timbul dari adopsi dalam garis lurus atau garis menyamping tingkat kedua. Menurut norma ini pihak yang mengadopsi dihalangi untuk menikah dengan anak yang diadopsi, dan anak yang diadopsi dihalangi untuk menikah dengan anak-anak yang dilahirkan dari orang tua yang mengadopsi dia. Alasannya karena adopsi mereka menjadi saudara-saudari se keturunan.