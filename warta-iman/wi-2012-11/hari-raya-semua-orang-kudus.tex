\chap{Hari Raya Semua Orang Kudus dan Peringatan Arwah Semua Orang Beriman\\
\small \emph{Romo William P. Saunders} \normalsize}

\kutipan{Dapatkah dijelaskan asal mula Hari Raya Semua Orang Kudus dan Peringatan Arwah Semua Orang Beriman? Apakah kedua perayaan tersebut ada hubungannya dengan paham kekafiran dan perayaan Halloween?}
\sumber{seorang pembaca di Springfield}


Keduanya, Hari Raya Semua Orang Kudus dan Peringatan Arwah Semua Orang Beriman, berkembang dalam kehidupan Gereja, terlepas dari paham kekafiran dan perayaan Halloween.

Marilah pertama-tama kita membahas Hari Raya Semua Orang Kudus. Asal mula yang tepat dari perayaan ini tidak diketahui dengan pasti, walau, sesudah disahkannya kekristenan pada tahun 313 M, suatu peringatan umum demi menghormati para kudus, khususnya para martir, muncul di berbagai wilayah di segenap penjuru Gereja. Sebagai contoh di Timur, kota Edessa merayakan pesta ini pada tanggal 13 Mei; Siria merayakannya pada hari Jumat sesudah Paskah; kota Antiokhia merayakannya pada hari Minggu pertama sesudah Pentakosta. Baik St Efrem (wafat 373) dan St Yohanes Krisostomus (wafat 407) menegaskan akan adanya perayaan ini dalam khotbah mereka. Di Barat, suatu peringatan demi menghormati semua orang kudus juga dirayakan pada hari Minggu pertama sesudah Pentakosta. Alasan utama menetapkan suatu pesta umum ini adalah karena kerinduan untuk menghormati sejumlah besar martir, teristimewa yang wafat dalam masa penganiayaan oleh Kaisar Diocletion (284-305), yaitu masa penganiayaan yang paling luas, keji dan bengis. Singkatnya, tidak akan ada cukup hari dalam satu tahun apabila masing-masing martir dirayakan tersendiri, lagipula kebanyakan dari para martir ini wafat dalam kelompok. Sebab itu, suatu pesta umum bagi semua orang kudus, dianggap paling tepat.

Pada tahun 609, Kaisar Phocas memberikan Pantheon di Roma (= kuil yang dipersembahkan bagi semua dewa) kepada Paus Bonifasius IV, yang mempersembahkannya kembali pada tanggal 13 Mei di bawah nama St Maria ad Martyres (atau St Maria dan Semua Martir). Apakah Bapa Suci dengan sengaja memilih tanggal 13 Mei karena tanggal perayaan yang populer ini telah ditetapkan di Timur atau apakah hal ini sekedar kebetulan belaka, tak seorang pun tahu pasti.

Penetapan tanggal 1 November sebagai Hari Raya Semua Orang Kudus berkembang seturut berjalannya waktu. Paus Gregorius III (731-741) mempersembahkan suatu oratorium di Basilika St Petrus yang asli demi menghormati semua orang kudus pada tanggal 1 November (setidaknya demikian menurut beberapa catatan), maka kemudian tanggal ini menjadi tanggal resmi untuk merayakan Hari Raya Semua Orang Kudus di Roma. St. Beda (wafat 735) mencatat HR Semua Orang Kudus dirayakan pada tanggal 1 November di Inggris, dan perayaan serupa juga ada di Salzburg, Austria. Ado dari Vienne (wafat 875) menceritakan bagaimana Paus Gregorius IV meminta Raja Louis yang Saleh (778-840) untuk memaklumkan tanggal 1 November sebagai HR Semua Orang Kudus di seluruh wilayah Kekaisaran Romawi yang Kudus. Buku Doa Misa dari abad ke-9 dan ke-10 juga menempatkan HR Semua Orang Kudus dalam penanggalan liturgi pada tanggal 1 November.

Menurut seorang sejarahwan Gereja perdana, John Beleth (wafat 1165), Paus Gregorius IV (827-844) secara resmi memaklumkan tanggal 1 November sebagai HR Semua Orang Kudus, memindahkannya dari tanggal 13 Mei. Tetapi, Sicard dari Cremona (wafat 1215) mencatat bahwa Paus Gregorius VII (1073-85) akhirnya menghapus tanggal 13 Mei dan mengamanatkan 1 November sebagai tanggal perayaan HR Semua Orang Kudus. Secara keseluruhan dapat kita lihat bahwa Gereja menetapkan perayaan liturgis demi menghormati para kudus ini sama sekali terlepas dari pengaruh kekafiran.

Sekarang, kita membahas hubungannya dengan perayaan Halloween. Tanggal 1 November menandai Samhain, yaitu dimulainya musim dingin bangsa Celtic. (Bangsa Celtic hidup sekitar 2000 tahun yang lalu di Inggris, Scotlandia, Wales, Irlandia dan Perancis utara.) Samhain, yang namanya dipakai sebagai nama perayaan, adalah dewa kematian bangsa Celtic, namanya secara harafiah berarti “akhir musim panas”. Karena musim dingin adalah masa-masa dingin, kegelapan dan kematian, kaum Celtic segera menghubungkannya dengan kematian manusia. Malam menjelang Samhain, yaitu tanggal 31 Oktober, adalah saat kurban kafir bangsa Celtic, dan Samhain mengijinkan jiwa-jiwa orang mati untuk kembali ke rumah-rumah duniawi mereka pada malam ini. Setan-setan, hantu, roh dan tukang sihir datang untuk mencelakai manusia, teristimewa orang-orang yang pernah menyakiti mereka semasa mereka masih hidup. Kucing, juga, dianggap keramat sebab dianggap dulunya mereka adalah manusia yang dikutuk sebagai hukuman atas perbuatan-perbuatan jahat mereka semasa di dunia.

Guna melindungi diri dari roh-roh jahat yang bergentayangan pada malam Samhain, orang-orang memadamkan perapian mereka, dan para Druids (para imam dan guru rohani bangsa Celtic) mendirikan suatu api unggun tahun baru yang sangat besar terbuat dari dahan-dahan pohon oak yang keramat. Druids mempersembahkan kurban-kurban bakaran - hasil bumi, hewan, bahkan manusia - dan menyampaikan ramalan mengenai tahun yang akan datang dengan memeriksa sisa-sisa kurban bakaran. Orang-orang terkadang mengenakan kostum dari kepala dan kulit binatang. Dari api unggun yang baru ini, perapian rumah para penduduk sekali lagi dinyalakan.

Kelompok-kelompok etnis yang berbeda masing-masing memiliki adat mereka sendiri yang berbaur dengan perayaan. Di Irlandia, orang mengadakan suatu arak-arakan demi menghormati dewa Muck Olla. Mereka mengikuti sang pemimpin yang mengenakan jubah putih dengan topeng dari kepala binatang dan minta sedekah makanan. (Irlandia juga merupakan asal dari dongeng `jack-o-lantern': seorang bernama Jack yang tak dapat masuk ke surga karena kikir, namun ia juga tak dapat masuk ke neraka karena ia sering melontarkan lelucon untuk mengolok-olok iblis; jadi ia dihukum untuk berjalan mengelilingi dunia dengan lenteranya hingga tiba Hari Penghakiman.)

Orang-orang Scotlandia berjalan menyusuri padang dan desa-desa dengan membawa suluh dan menyalakan api unggun guna menghalau tukang sihir dan roh-roh jahat.

Di Wales, setiap orang meletakkan suatu batu yang telah ditandai pada api unggun yang sangat besar. Jika batu miliknya tak dapat diketemukan kembali keesokan paginya, maka pastilah orang itu akan mati dalam tahun itu.

Di samping tradisi Celtic yang telah ada, penjajah Romawi yang berkuasa atas Inggris pada tahun 43 M membawa serta dua perayaan kafir lainnya: Feralia yang dirayakan di penghujung bulan Oktober demi menghormati mereka yang telah meninggal dunia; dan suatu perayaan pada musim gugur demi menghormati Pomona, dewi buah-buahan dan pepohonan, kemungkinan, melalui perayaan ini, buah apel kemudian dihubungkan dengan perayaan Halloween. Unsur-unsur perayaan Romawi ini dipadukan dengan perayaan Samhain bangsa Celtic.

Dengan tersebar luasnya kekristenan dan dengan ditetapkannya HR Semua Orang Kudus, sebagian dari tradisi-tradisi kafir ini tetap tinggal dalam wilayah yang masyarakatnya berbahasa Inggris, dalam perayaan All Hallows Eve (atau Halloween, All Saints Eve, Malam menjelang HR Semua Orang Kudus), kemungkinan pertama-tama memang berasal dari takhayul, tetapi kemudian, lebih pada unsur sukaria tanpa ada hubungan dengan kekafiran. Oleh sebab itulah, anak-anak kecil (dan juga sebagian orang dewasa) masih mengenakan berbagai macam kostum dan malam itu berpura-pura menjadi setan, tukang sihir, drakula, monster, ninja, bajak laut, dan lain sebagainya, tanpa lagi memikirkan kekafiran. Dengan demikian, HR Semua Orang Kudus jelas muncul dari devosi Kristiani yang sejati, terlepas dari paham kekafiran.

Sejalan dengan Hari Raya Semua Orang Kudus, berkembang pula Peringatan Arwah Semua Orang Beriman. Gereja tak henti-hentinya mendorong umat beriman untuk mempersembahkan doa-doa dan Misa Kudus bagi jiwa-jiwa umat beriman yang telah meninggal dunia, yang masih berada di purgatorium. Pada saat kematian mereka, jiwa-jiwa ini belum bersih sepenuhnya dari dosa-dosa ringan atau belum melunasi hutang dosa di masa lalu, dan oleh sebab itu belum dapat menikmati kebahagiaan surgawi. Umat beriman di dunia dapat menolong jiwa-jiwa di api penyucian ini agar dapat segera menikmati kebahagiaan surgawi melalui doa-doa, perbuatan-perbuatan baik dan mempersembahkan Misa Kudus bagi jiwa-jiwa menderita ini.

Pada masa-masa Gereja awali, nama-nama umat beriman yang telah meninggal dunia ditempelkan di Gereja sehingga komunitas akan mengenangkan mereka dalam doa. Pada abad keenam, biara-biara Benediktin mengadakan peringatan khidmad akan para anggota yang telah meninggal dunia, pada hari-hari sesudah Pentakosta. Di Spanyol, St Isidorus (wafat 636) menegaskan adanya perayaan pada hari Sabtu sebelum Minggu Sexagesima (Minggu kedua sebelum Masa Prapaskah, kedelapan sebelum Paskah, dalam penanggalan kuno). Di Jerman, Widukind, Abbas Corvey (wafat 980) mencatat adanya suatu upacara khusus pada tanggal 1 Oktober bagi umat beriman yang telah meninggal dunia. St Odilo, Abbas Cluny (wafat 1048) mengamanatkan kepada seluruh biara Cluniac agar doa-doa khusus dipanjatkan dan Ofisi bagi Yang Meninggal dimadahkan demi segenap jiwa-jiwa di purgatorium, pada tanggal 2 November, sehari sesudah HR Semua Orang Kudus. Benediktin dan Kartusian menerapkan devosi yang sama, dan segera saja tanggal 2 November dirayakan sebagai Peringatan Arwah Semua Orang Beriman di segenap Gereja.

Tradisi-tradisi lainnya muncul dalam perjalanan waktu sehubungan dengan perayaan Peringatan Arwah Semua Orang Beriman. Pada abad ke-15, Dominikan menetapkan suatu tradisi di mana setiap imam mempersembahkan tiga Misa Kudus pada Peringatan Arwah Semua Orang Beriman. Paus Benediktus XIV pada tahun 1748 menyetujui praktek ini dan devosi ini dengan cepat menyebar ke seluruh Spanyol, Portugis dan Amerika Latin. Dalam masa Perang Dunia I, Paus Benediktus XV, menyadari akan banyaknya mereka yang tewas akibat perang dan begitu banyak Misa yang tak dapat dipersembahkan karena hancurnya gereja-gereja, memberikan hak istimewa kepada segenap imam untuk mempersembahkan tiga Misa Kudus pada Peringatan Arwah Semua Orang Beriman: satu untuk intensi khusus, satu untuk arwah semua orang beriman, dan satu untuk intensi Bapa Suci.

Tradisi-tradisi lainnya berkembang sehubungan dengan Peringatan Arwah Semua Orang Beriman. Di Meksiko, sanak-saudara merangkai karangan-karangan bunga dan dedaunan, juga membuat salib-salib dari bunga-bunga segar maupun bunga-bunga kertas beraneka warna guna diletakkan pada makam sanak-saudara yang telah meninggal, di pagi hari Peringatan Arwah Semua Orang Beriman. Keluarga akan menghabiskan sepanjang hari itu di pemakaman. Imam akan mengunjungi makam, menyampaikan khotbah dan mempersembahkan doa-doa bagi mereka yang meninggal, serta memberkati makam-makam satu per satu. Permen “Tengkorak” dibagikan kepada anak-anak.

Praktek serupa didapati pula di Louisiana. Sanak-saudara membersihkan serta melabur batu-batu nisan, mempersiapkan karangan-karangan bunga dan dedaunan, juga salib-salib dari bunga-bunga segar maupun bunga-bunga kertas untuk menghiasi makam. Pada siang hari Peringatan Arwah Semua Orang Beriman, imam berarak sekeliling makam, memberkati makam-makam dan mendaraskan rosario. Lilin-lilin dinyalakan dekat kubur pada senja hari; satu untuk setiap anggota keluarga yang telah meninggal dunia. Pada Peringatan Arwah Semua Orang Beriman, biasanya Misa dirayakan di pemakaman. Dua contoh praktek kebudayaan ini berpusat pada pentingnya mengenangkan mereka yang telah meninggal dunia serta mendoakan jiwa-jiwa mereka.

Namun demikian, pada Abad Pertengahan, suatu kepercayaan takhayul, mungkin pengaruh dari paham kafir bangsa Celtic, mengatakan bahwa jiwa-jiwa di api penyucian menampakkan diri pada Peringatan Arwah Semua Orang Beriman sebagai tukang sihir, kodok, hantu, dll kepada mereka yang telah berbuat salah terhadap mereka semasa mereka masih hidup di dunia. Oleh sebab itu, beberapa kelompok etnis juga mempersiapkan makanan sesaji guna menjamu dan menenangkan roh-roh pada hari itu. Praktek-praktek semacam ini kemungkinan merupakan sisa-sisa perayaan Samhain bangsa Celtic seperti yang dibicarakan di atas. Pada masa sekarang, makanan sajian seperti itu tidak lagi ada hubungannya dengan kekafiran melainkan lebih sebagai wujud penitensi.  

Oleh sebab itu, baik Hari Raya Semua Orang Kudus maupun Peringatan Arwah Semua Orang Beriman, berasal dari kepercayaan kristiani dan muncul dalam kehidupan Gereja melalui spiritualitas yang sehat. Segala praktek seputar kedua perayaan religius ini, yang berasal dari paham kafir - seperti Halloween - telah lama kehilangan makna kekafirannya.


* Fr. Saunders is is pastor of Our Lady of Hope Church in Potomac Falls.
sumber : “Straight Answers: All Saints and All Souls Day” by Fr. William P. Saunders; Arlington Catholic Herald, Inc; Copyright ©2004 Arlington Catholic Herald. All rights reserved; www.catholicherald.com
Diperkenankan mengutip / menyebarluaskan artikel di atas dengan mencantumkan: “diterjemahkan oleh YESAYA: www.indocell.net/yesaya atas ijin The Arlington Catholic Herald.”
	
	                                                                                                                                                                                                                                                                                                            	