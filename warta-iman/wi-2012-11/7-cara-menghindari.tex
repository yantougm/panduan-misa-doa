\chap{7 Cara Menghindari Api Penyucian\\
 \small \emph{P. Josep Susanto Pr.} \normalsize}
	

Ketika kita masih muda dan sehat kita belum banyak memberi perhatian terhadap hidup setelah mati. Pikiran tentang sorga, neraka, Api penyucian belum mendominasi benak kita. Namun ketika orang sudah pensiun, mulai sakit-sakitan, atau tenaga sudah mulai loyo barulah mereka memberi perhatian serius akan nasibnya sesudah mati. Pikiran tentang surga atau Api penyucian akan mewarnai hari-hari mereka.

Masalah mulai datang ketika kita
berhadapan dengan fakta dalam kehidupan sehari-hari, kematian tidak pandang tua ataupun muda, bisa terjadi kapan saja, sehat atau sakit, laki-laki atau perempuan. Kutipan Kitab Suci ini bisa menyadarkan kita : “Apa gunanya seorang memperoleh seluruh dunia tetapi kehilangan nyawanya? Dan apakah yang dapat diberikannya sebagai ganti nyawanya? (Mat 16:26).  Karena itu “Carilah dahulu Kerajaan Allah, maka semuanya itu akan ditambahkan juga kepadamu” (Luk 12:31).

     Menurut ajaran Gereja, ada dua macam hukuman di Api Penyucian. Pertama, hukuman karena “perasaan kehilangan Tuhan”. Sedangkan yang kedua hukuman karena “sesal batin yang tak kunjung henti”. Hukuman karena kehilangan Tuhan diakibatkan oleh hilangnya kesempatan berjumpa dengan Allah, Sumber segala kebaikan, Tujuan dan Akhir hidup manusia. Sedangkan hukuman karena sesal batin yang tak kunjung henti menyangkut penderitaan yang dapat dirasakan sebagaimana halnya dengan hukuman fisik yang dapat kita rasakan.

     Setiap orang menerima hukuman sebanding dengan perbuatan yang dilakukan oleh seseorang selama hidup. Masing-masing orang harus mempertanggungjawabkan dosanya dihadapan Allah. Siksaan yang dijalani di dalam Api penyucian tidak bisa dibandingkan dengan penderitaan yang paling menyiksa yang ada di dunia ini. Hal itu digambarkan oleh Santo Thomas Aquinas dengan kalimat : Hukuman yang paling ringan di Api Punyucian beratnya melampaui semua penderitaan yang ada di dunia ini. Sejalan dengan itu Thomas A Kempis mengatakan : Satu jam penyiksaan lebih mengerikan dari pada seratus tahun menjalankan penebusan dosa yang keras yang kita lakukan di dunia ini. Hal itu dikarenakan, yang disiksa adalah “jiwa”, bukan sekedar “fisik”.

     Kiranya hal yang mengerikan yang terhadi di Api Penyucian memang berguna bagi kita untuk meningkatkan kualitas hidup kita lebih baik lagi di dunia ini, namun kita juga tidak boleh hidup dalam ketakuan saja. Karena hasil dari semua penyucian itu adalah kedamaian dan kebahagiaan abadi, sehingga kita bersatu kembali (pulang kembali) kepada Allah, Pemilik hidup kita. Jika pintu surga dan pintu Api penyucian dibuka, orang-orang yang meninggal akan segera memilih pintu Api Penyucian dari pada mereka menghadap Tuhan dengan noda dosa karena mereka melihat dirinya sendiri tidak pantas. Meraka akan menyucikan diri dengan kemauan mereka sendiri dan dengan penuh kasih, karena Api yang menyucikan mereka adalah api kasih yang berasal dari Allah sendiri.

\section*{Beberapa Kesaksian}

\subsection*{Santa Catharina dari Genoa (1447-1510)}

     Jiwa-jiwa akan mengalami kesengsaraan yang begitu dasyat sehingga tidak ada lidah yang dapat melukiskannya, atau sulit untuk dapat dimengerti sedikitpun sebelum Tuhan membuka rahasiaNya dengan rahmat istimewa.

\subsection*{Pastor Nieremberg (wafat 1658)} 

     Pada hari raya semua orang kudus, seorang gadis yang masih suci melihat penampakan di depan matanya, seorang wanita yang ia kenal, yang telah meninggal beberapa waktu sebelumnya. Sosok wanita itu berpakaian putih dengan kerudung di kepalanya, memegang rosario. Wanita yang telah meninggal itu sewaktu maish hidup pernah berjanji mempersembahkan Misa 3 kali di depan altar Santa Perawan Maria. Namun ia tidak melaksanakannya dalam hidup, sehingga hutang itu ditambahkan pada hukumannya. Sekarang ia meminta tolong gadis itu untuk mempersembahkan tiga kali Misa Kudus untuk dirinya. Gadis itu bersedia. Ketika Misa kudus yang ketiga sudah dirayakan arwah itu menampalkan diri lagi, mengungkapkan kegembiraan dan terima kasihnya. Arwah wanita itu selalu menampakan diri di depan Sakramen maha Kudus, melakuakan Pujian dan Penyembahan. Arwah itu terus menerus menampakan dirinya sampai saatnya tiba bagi dia naik ke surga diiring oleh malaikat pelindungnya. Awrah itu menyatakan bahwa dia sudah tidak menderita lagi atas hukuman “perasaan kehilangan Tuhan”, namun ia menambahkan bahwa kehilangan Tuhan itu menyebabkan dia menderita sengsara yang tak tertahankan.

\subsection*{Kesaksian seorang Bocah 11 tahun, Blasio Massei}           

     Pada suatu ketika seorang bocah 11 tahun menginggal dunia bernama Blasio Massei. Orang tuanya lalu berdevosi kepada Santo Bernardus dari Siena yang baru saja diangkat santo oleh Gereja Katolik. Orang tua Blassio mempersembahkan jiwa anaknya melalui santo ini. Ketika tubuh Blasio dibawa ke kubur, Blasio bangkit seolah-olah habis tidur nyenayak dan mengatakan bahwa Santo Bernardus telah menjadikannya hidup kembali, supaya menceritakan keajaiban yang telah diperintahkan santo.

     Blassio bercerita pada waktu dia meninggal, Santo Bernardus menampakan diri kepadanya. Santo itu menuntun tangannya dan berkata : Jangan takut, tetapi perhatikanlah apa yang kuperintahkan kepadamu, maka kamu harus mengingatnya dan setelah itu dapat menceritakannya.” Mereka mengunjungi Neraka, Api penyucian lalu sorga. Di neraka, Blassio melihat kengerian-kengerian yang tak terkatakan dan berbagai macam siksaan yang diberikan kepada mereka yang congkak, tamak, jahat, dan pendosa-pendosa lain.

     Di api penyucian, ia melihat siksaan yang paling menakutkan, ada berbagai macam hukuman menurut dosa yang dilakukan. Ia melihat ada arwah-arwah yang memohon kepadanya agar memberitahukan kepada sanak saudaranya, mereka merindukan dia permohonan dan perbuatan baik yang mereka lakukan.

     Di sorga, ia sangat terpesona. Ia melihat malaikat yang banyak sekali jumlahnya mengelilingi takhta Allah dan Perawan Maria yang Kudus.

\section*{7 Cara Menghindari Api Penyucian}
\begin{enumerate}
\item         Beriman Teguh pada Allah dan KerahimanNya
\item         Menghilangkan penyebabnya yaitu Dosa
\item         Devosi kepada Bunda Perawan Maria
\item         Membuat silih dengan amal kasih
\item         Rela menderita
\item         Menerima Sakramen-sakramen terutama Ekaristi dan Perminyakan orang sakit
\item         Menerima kematian dengan pasrah.
\end{enumerate}

\sumber{http://www.imankatolik.or.id}