\chap{Paranormal dan Penyihir: Apa Kata Kitab Suci?\\
\small \emph{Romo William P. Saunders} \normalsize}

Belakangan ini, saya melihat banyak iklan di televisi yang menawarkan jasa paranormal. Ada juga tayangan TV mengenai Sabrina, si gadis penyihir. Bagaimanakah kita, sebagai umat Kristiani, menyikapi masalah ini?
~ seorang pembaca di McLean

Sebagai orang Katolik, kita ingat bahwa perintah Allah yang pertama mengatakan, “Akulah TUHAN, Allahmu. Jangan ada padamu allah lain di hadapan-Ku.” Ketika ditanya hukum manakah yang terutama, Tuhan kita Yesus Kristus, dengan mengulang perintah yang ada dalam Kitab Ulangan, mengatakan, “Kasihilah Tuhan, Allahmu, dengan segenap hatimu dan dengan segenap jiwamu dan dengan segenap akal budimu” (Mat 22:37). Sementara Tuhan, menurut kehendak-Nya, dapat mewahyukan masa depan kepada para nabi atau para kudus, kita sebagai pribadi wajib senantiasa percaya akan penyelenggaraan ilahi-Nya. St Paulus mengingatkan kita, “Kita tahu sekarang, bahwa Allah turut bekerja dalam segala sesuatu untuk mendatangkan kebaikan bagi mereka yang mengasihi Dia, yaitu bagi mereka yang terpanggil sesuai dengan rencana Allah” (Rm 8:28). Mungkin terkadang kita juga memiliki rasa ingin tahu tentang apa yang akan terjadi di masa mendatang, namun demikian kita mengandalkan hidup kita pada Tuhan, percaya penuh akan kasih sayang dan pemeliharaan-Nya.

Berusaha mengetahui masa depan dengan membaca tangan, kartu ramal, atau bentuk-bentuk ramalan lainnya, atau berusaha mengendalikan masa depan melalui black magic, ilmu gaib atau sihir, merupakan pelanggaran terhadap perintah Allah yang pertama. Kitab Suci banyak mengutuk praktek-praktek ini: dalam Perjanjian Lama kita dapati, “Seorang ahli sihir perempuan janganlah engkau biarkan hidup” (Kel 22:18), “Siapa yang mempersembahkan korban kepada allah kecuali kepada TUHAN sendiri, haruslah ia ditumpas.” (Kel 22:20). “Apabila seorang laki-laki atau perempuan dirasuk arwah atau roh peramal, pastilah mereka dihukum mati, yakni mereka harus dilontari dengan batu dan darah mereka tertimpa kepada mereka sendiri” (Im 20:27), dan “Di antaramu janganlah didapati seorangpun yang mempersembahkan anaknya laki-laki atau anaknya perempuan sebagai korban dalam api, ataupun seorang yang menjadi petenung, seorang peramal, seorang penelaah, seorang penyihir, seorang pemantera, ataupun seorang yang bertanya kepada arwah atau kepada roh peramal atau yang meminta petunjuk kepada orang-orang mati. Sebab setiap orang yang melakukan hal-hal ini adalah kekejian bagi TUHAN….” (Ul 18:10-12).

Perjanjian Baru juga membicarakan masalah ini: Dalam Kisah Para Rasul, di Filipi St Paulus bertemu dengan seorang hamba perempuan “yang mempunyai roh tenung” yang mendapatkan penghasilan besar dengan tenungan-tenungannya. St Paulus membebaskannya dari roh jahat itu (Kis 16:16 dst). Dalam ayat-ayat lain, kita dapati kutukan-kutukan terhadap sihir dan praktek-praktek gaib pada umumnya. St Paulus mengutuk seorang tukang sihir (Gal 5:20). Dalam Kisah Para Rasul, St Paulus mencela Elimas, tukang sihir, dan menyebutnya sebagai “anak Iblis, engkau musuh segala kebenaran” (Kis 13:8 dst). St Petrus mengecam Simon, si tukang sihir, yang bermaksud membeli kuasa Roh Kudus guna menjadikan diri lebih berkuasa (Kis 8:9 dst). Dalam Kitab Wahyu, Yesus memaklumkan, “Tetapi orang-orang penakut, orang-orang yang tidak percaya, orang-orang keji, orang-orang pembunuh, orang-orang sundal, tukang-tukang sihir, penyembah-penyembah berhala dan semua pendusta, mereka akan mendapat bagian mereka di dalam lautan yang menyala-nyala oleh api dan belerang; inilah kematian yang kedua” (Why 21:8).

Katekismus Gereja Katolik dalam menjelaskan perintah Allah yang pertama, mengulang kutukan terhadap praktek ramalan, “Segala macam ramalan harus ditolak: mempergunakan setan dan roh jahat, pemanggilan arwah atau tindakan-tindakan lain, yang tentangnya orang berpendapat tanpa alasan, seakan-akan mereka dapat `membuka tabir' masa depan. Di balik horoskop, astrologi, membaca tangan, penafsiran pratanda dan orakel (petunjuk gaib), paranormal dan menanyai medium, terselubung kehendak supaya berkuasa atas waktu, sejarah dan akhirnya atas manusia; demikian pula keinginan menarik perhatian kekuatan-kekuatan gaib. Ini bertentangan dengan penghormatan dalam rasa takwa yang penuh kasih, yang hanya kita berikan kepada Allah” (No. 2116). Segala praktek dengan mempergunakan kuasa-kuasa gaib dikutuk karena bertentangan dengan agama yang benar dan biasanya dianggap sebagai dosa berat. Segala bentuk permohonan kepada setan jelas merupakan dosa berat. (Perlu diperhatikan bahwa sekedar membaca horoskop di koran karena iseng bukanlah dosa berat; tetapi menganggapnya serius atau sengaja membayar demi mendapatkan ramalan adalah dosa).

Perhatian khusus perlu diberikan kepada sihir, yang menyangkut baik menyingkapkan masa depan maupun berusaha mengendalikan masa depan. Memang, dalam tayangan televisi, Sabrina atau tukang sihir satunya yang lebih senior dapat saja mengarang-ngarang cerita yang menyenangkan seputar dunia sihir dan ilmu sihir. Namun demikian, sihir menyangkut mengadakan sesuatu tertentu yang di luar kuasa normal manusia, dengan bantuan kuasa-kuasa (gaib) selain dari kuasa Allah. Pada umumnya, sihir menyangkut suatu perjanjian dengan setan, atau setidaknya memohon bantuan roh-roh jahat. Dunia sihir meliputi ritus membangkitkan orang mati, membangkitkan gairah dalam diri orang, dan mendatangkan bencana atau bahkan kematian atas musuh. Setanisme, secara istimewa, menghaturkan pemujaan kepada Penguasa Kegelapan, dan bahkan merayakan “Misa Hitam,” yang meniru-niru Misa kita tetapi penuh dengan tindakan sakrilegi dan hujat. Bahkan jika orang berbicara mengenai “white magic” atau “sihir baik,” pelakunya juga memohon pada kekuatan-kekuatan yang bukan berasal dari Tuhan.

Kita percaya, seperti yang ditulis St Yohanes, bahwa “Allah adalah kasih” (1 Yoh 4:16). “Karena begitu besar kasih Allah akan dunia ini, sehingga Ia telah mengaruniakan Anak-Nya yang tunggal, supaya setiap orang yang percaya kepada-Nya tidak binasa, melainkan beroleh hidup yang kekal” (Yoh 3:16). Yesus adalah terang dunia; terang itu bercahaya di dalam kegelapan dan kegelapan itu tidak menguasainya (Yoh 1:4-5). Yesus adalah jalan dan kebenaran dan hidup (Yoh 14:6). Memohon bantuan setan atau kuasa-kuasa lain, masuk dalam dunia gelap (gaib) guna mendapatkan pertolongan, atau berusaha merampas kuasa yang hanya dimiliki Allah Sendiri, adalah menantang kuasa Allah yang Mahakuasa. Bertindak demikian adalah berpaling dari Allah dan menempatkan jiwa kita sendiri dalam bahaya.


* Fr. Saunders is dean of the Notre Dame Graduate School of Christendom College and Pastor of Queen of Apostles Parish, both in Alexandria.
sumber : “Straight Answers: Psychics and Witches: What Scripture Says” by Fr. William P. Saunders; Arlington Catholic Herald, Inc; Copyright ©1997 Arlington Catholic Herald, Inc. All rights reserved; www.catholicherald.com
Diperkenankan mengutip / menyebarluaskan artikel di atas dengan mencantumkan: “diterjemahkan oleh YESAYA: www.indocell.net/yesaya atas ijin The Arlington Catholic Herald.”
	
	                                                                                                                                                                                                                                                                                                           