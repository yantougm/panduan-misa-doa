\newpage
\chap{Kompendium Katekese Gereja Katolik}
\setcounter{kgkcounter}{45}
\small
\kgk{Apa yang diwahyukan Yesus kepada kita tentang misteri Bapa?}
        Yesus Kristus mewahyukan kepada kita bahwa Allah itu ”Bapa”, bukan hanya
          karena Dia menciptakan alam semesta dan manusia, tetapi terutama karena Dia
          secara kekal melahirkan dalam Diri-Nya, Putra-Nya, yaitu Sabda-Nya, ”cahaya
          kemuliaan Allah dan gambar wujud Allah” (Ibr 1:3).

\kgk{Siapakah Roh Kudus yang diwahyukan oleh Yesus Kristus kepada
              kita?}
Roh Kudus adalah Pribadi ketiga Tritunggal. Dia adalah Allah, satu dan
          setara dengan Bapa dan Putra. Dia ”berasal dari Bapa” (Yoh 15:26) yang adalah
          dasar tanpa sebuah dasar, dan asal dari semua kehidupan trinitaris. Dia berasal
          pula dari Sang Putra (Filioque) lewat anugerah abadi yang dibuat oleh Bapa
          kepada sang Putra. Diutus oleh Bapa dan Putra yang menjelma, Roh Kudus
          membimbing Gereja ”ke dalam seluruh kebenaran” (Yoh 16:13).

\kgk{Bagaimana Gereja mengungkapkan iman trinitarisnya?}
Gereja mengungkapkan iman trinitarisnya dengan percaya kepada keesaan
Allah yang dalam-Nya terdapat tiga Pribadi, Bapa, Putra, dan Roh Kudus. Ketiga
          Pribadi ilahi ini hanya satu Allah karena masing-masing memiliki secara setara
          kepenuhan kodrat ilahi yang satu dan tak terbagi. Mereka berbeda satu sama lain
          karena relasi yang menghubungkan mereka satu sama lain. Bapa melahirkan Putra,
          Putra dilahirkan oleh Bapa, Roh Kudus keluar dari Bapa dan Putra.

\kgk{Bagaimana ketiga Pribadi Ilahi ini bekerja?}
     Tidak terpisahkan dalam satu hakikat, ketiga Pribadi Ilahi ini juga tidak   
terpisahkan dalam aktivitas mereka. Tritunggal mempunyai satu tindakan yang      
satu dan sama. Namun di dalam tindakan yang satu ini, setiap Pribadi hadir
menurut cara adanya yang khas baginya di dalam Tritunggal.

\flushright{(\dots \emph{bersambung} \dots)}
\normalsize