\chap{Apa Perbedaan Mudika dan OMK?}

Muda-Mudi Katolik (Mudika) ialah kelompok OMK  (pemuda/i yang beragama Katolik) teritorial paroki. Mudika berkembang menjadi salah satu organisasi dalam paroki. Sejarah Mudika dimulai sejak “Pemuda Katolik” menjadi Organisasi Massa pada awal Orde Baru. OMK yang tidak mau menjadi ormas “Pemuda Katolik” kemudian membentuk kelompok teritorial paroki bernama Mudika. Pencetus nama Mudika ini ialah FX Puniman (seorang aktivis OMK 1970-an) yang juga wartawan di kota Bogor. Maka anggota Mudika ialah OMK-OMK yang tidak mau menjadi anggota Ormas “Pemuda Katolik”. Sedangkan OMK ialah individu atau sekelompok orang yang berusia muda dan beragama Katolik.

Jadi OMK \textbf{lebih luas} daripada Mudika. OMK ada di mana-mana, baik di organisasi Mudika maupun komunitas non–Mudika. Banyak pula OMK yang tidak mau menjadi anggota Mudika. Mereka lebih suka menjadi anggota kelompok kategorial seperti Persekutuan Doa Karismatik Katolik, Persekutuan Doa Legio Mariae, Komunitas OMK Peduli Sampah, Persekutuan Doa Meditatif ala Taize, dll, atau, banyak pula OMK yang hanya misa sekali seminggu.

Maka OMK dan Mudika dapat ada bersama-sama dalam satu paroki. Namun harap dicatat bahwa OMK bukan organisasi. OMK ialah individu atau komunitas orang berusia muda dan beragama Katolik.

Mudika merupakan salah satu kelompok OMK di Gereja Paroki lingkupnya teritorial. Sementara OMK adalah individu atau komunitas yang tak hanya lingkup teritorial. Persamaan keduanya: keduanya beranggota orang berusia muda beragama Katolik.

Seksi Kepemudaan di Paroki lah yang bertugas membina baik komunitas OMK  teritorial (Mudika) maupun berbagai komunitas OMK kategorial.

\section*{Sejarah Mudika}
\small
Mengenai istilah “Mudika”, maka kita harus melihat sejarahnya. Entitas ini mengalir dari sejarah perjuangan bangsa dan Gereja Katolik Indonesia. Pada masa Hindia-Belanda-Indonesia-Kemerdekaan-Orde Lama, seorang pemuda atau pemudi yang beragama Katolik belum memiliki organisasi yang beraneka ragam seperti sekarang. Hanya satu istilah untuk mereka waktu itu yaitu “Pemoeda Katolik”. Istilah Pemuda Katolik ini menunjuk segala aktivitas maupun organisasi pemuda beragama Katolik yang ada di paroki, maupun yang ada di organisasi massa yang ada di bawah partai-partai.

Pada masa Presiden Soekarno, ada “Partai Katolik”, yang salah satunya berunsur “Pemuda Katolik” sebagai organisasi \textit{onderbouw} Partai Katolik itu. Pada awal Orde Baru, Presiden Soeharto membuat kebijakan agar jumlah partai hanya dua plus satu golongan karya. Maka semua partai dan unsur-unsur di dalam tiap partai harus ikut fusi ke 2 partai dan 1 golkar tersebut. Partai Katolik berfusi (melebur) dalam Partai Demokrasi Indonesia. Pemuda Katolik berubah menjadi organisasi massa. 

Sementara itu, anggota-anggota Pemuda Katolik sebagai organisasi ada yang meninggalkan Pemuda Katolik, ada pula yang tetap di dalamnya dan melanjutkan organisasi itu sebagai ormas. Sedangkan pemuda Katolik yang bukan anggota ormas Pemuda Katolik, lalu menyebut diri “Muda-Mudi Katolik”, yaitu muda dan mudi beragama Katolik yang maunya bukan ikut organisasi massa Pemuda Katolik, namun mau berperan serta dalam pembangunan secara internal di dalam gereja paroki. FX Puniman (wartawan di Bogor) ialah penggagas istilah Mudika ini. Mudika pun menjadi organisasi di bawah paroki, internal Gereja, di samping ormas Pemuda Katolik yg bergerak di masyarakat.
\normalsize

\section*{Sejarah OMK}
Tahun demi tahun sejak 1970an sampai 2005, muncul banyak kelompok orang muda Katolik baik di paroki (berdasar kedekatan tempat tinggal/teritorial), maupun berdasar minat (kategorial). Mudika de facto telah menjadi organ yang teritorial/parokial. Jika ada kegiatan Mudika, maka yang ikut hanya orang muda Katolik yang ada dalam Mudika itu. OMK lain tak mau ikut. Karena itu, agar semangat dan jangkauan pastoral kepemudaan meluas kembali, pada pertemuan nasional Komisi Kepemudaan KWI tahun 2005, diluncurkan istilah baru yaitu “OMK”. 

Jangan sampai OMK ini menjadi disempitkan lagi hanya menjadi organisasi teritorial saja atau organisasi kategorial saja. Maka jika para pembina tidak memahami asal sejarah ini, bahaya itu bisa terjadi lagi, yaitu bahwa pastoral orang muda Katolik tidak menjangkau semua OMK, selain hanya para aktivisnya saja, seperti di zaman organisasi bernama Mudika. Maka penjelasan sejarah ini penting. 

OMK bukan organisasi. Kalau mau membentuk kelompok OMK, namailah kelompok OMK itu dengan nama Santo Santa atau nama-nama lain, tetapi jangan bernama “OMK Paroki Y”, yang nanti akan sama saja dengan “Mudika paroki Y”. Hanya mengganti istilah “Mudika” dengan “OMK” merupakan tindakan yang keluar dari mulut singa masuk ke mulut buaya, tidak mengubah keadaan. Mudika biarlah menjadi Mudika, tak usah diganti menjadi OMK, karena Mudika sudah menjadi entitas OMK teritorial paroki. Namun OMK ialah pribadi-pribadi orang muda katolik. Akan lebih menggerakkan jika sekelompok OMK bernama misalnya “\textit{Sint Paul Youth Community}” yang sebenarnya anggotanya ialah OMK pencinta studi Tradisi Katolik. 

Sedangkan Mudika (di banyak keuskupan tetap ada) menjadi salah satu organ OMK teritorial. Jika di paroki Anda sudah tak ada lagi istilah Mudika, maka semoga baik OMK teritorial maupun OMK kategorial tetaplah OMK yang dinamik, yang menyapa semua pribadi OMK yang ada di teritori paroki dan keuskupan. Di Paroki, kelompok-kelompok OMK yang tidak tunggal itu, kini dipayungi oleh Dewan Paroki seksi Kepemudaan sebagai pemersatu, komunikator, koordinator, animator, motivator.

\sumber{http://www.katolisitas.org}


 