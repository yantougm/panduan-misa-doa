\chap{Orang Muda Katolik (OMK) dan Liturgi}

\section*{Prasangka}
\small
Dalam praktek, banyak kali muncul masalah pada relasi antara OMK dan liturgi (perayaan iman, ibadat). Di antara liturgi dan OMK seolah ada hubungan ”enggan tapi rindu”. Di balik tema “liturgi dan orang muda”, masih bercokol  prasangka laten baik terhadap Orang Muda Katolik (OMK), maupun terhadap Liturgi Gereja Katolik Roma. OMK seolah-olah suka hura-hura, semaunya sendiri, tidak bisa diatur dalam berliturgi. Sebaliknya, liturgi sering dipandang sebagai aturan sakral dan baku, seakan-akan jauh dari gelora kerinduan orang muda. Terhadap OMK, Tim Liturgi Paroki biasanya mengenakan frasa ”OMK yang pragmatis, maunya serba lain”. Seakan-akan OMK diperlawankan dengan liturgi yang tak memberi ruang kebebasan ungkapan iman. Dari pihak OMK, ada pula prasangka, bahwa liturgi itu serba  kaku.

Prasangka ini  bisa dipahami, karena sifat umum orang muda yang masih dalam masa pertumbuhan yang pesat. Mereka sedang berkembang dalam dimensi psikologis, intelektual, seksual-hormonal, emosi, peran sosial dan iman. OMK memang sedang  mengalami transformasi menuju kepribadian yang integral. Rentang masa muda yang panjang (usia 13-35 th) adalah masa distingtif,  saat mencari, mempertanyakan, belajar dan mengambil keputusan. Kita yang pernah menjalani masa muda tentu merasakan bahwa saat itu merupakan saat yang sukar, menantang sekaligus menggairahkan karena penemuan-penemuan baru. Sering kali kita ingin sesuatu yang ”lain dari pada yang lain” pada masa muda. Sedangkan di pihak lain, Liturgi Gereja Katolik Roma, sudah berkembang dalam 20 abad dan sering dipandang sebagai peraturan yang kaku alih-alih sebagai perayaan yang membebaskan. Padahal, potret berliturgi oleh OMK tak selamanya demikian.

Prasangka dan kecurigaan  yang digeneralisasi begitu saja terhadap OMK itu tentu tidak akan memecahkan persoalan yang sering kali muncul dalam praktek penghayatan OMK terhadap liturgi. Tidak bijaksana,  generalisasi mengenai OMK yang ”pragmatis dan maunya serba lain” itu. Liturgi Gereja pun tidak sepantasnya diperlawankan dengan gejolak dan selera orang muda. Kenyataannya, bahwa banyak orang terpanggil menjadi kudus pada masa muda, dan panggilan kekudusan itu banyak yang bermula dari penghayatan liturgi. Kita pun tahu, Ekaristi Kaum Muda (EKM) baik yang diselenggarakan oleh paroki, maupun oleh Panitia \textit{World Youth Day} yang mendatangkan Sri Paus sebagai pemimpin liturgi, selalu dipenuhi OMK dengan kerinduan mendalam. Bahkan, kelompok misa bahasa Latin yang terkesan ”penuh aturan ketat” ada yang digerakkan  oleh orang muda.

\section*{Memerlukan Dukungan}

Seperti pada umumnya orang Katolik Indonesia, tua maupun muda, penghayatan OMK akan liturgi sebenarnya tergantung pada pengetahuan dan pengalaman mereka akan liturgi itu sendiri. Bahwa praktek liturgi OMK kadang-kadang membuat para penanggungjawab liturgi mengerutkan kening, bagi saya lumrah saja dalam konteks pembelajaran. Gelegak kreativitas masa muda sekaligus tingkat pengetahuan dan pengalaman OMK akan liturgi haruslah bisa dipahami dan didukung. Tak usahlah daya kreatif mereka dalam ber-liturgi dihakimi dengan sewenang-wenang seperti yang sering terdengar dari keluhan mereka. Gara-gara  maunya kreatif, mereka ”dikecam secara liturgis”.

Sepanjang pengalaman para pendamping, tak ada OMK yang menjadi buruk karena mau kreatif dalam merencanakan dan mengolah liturgi. Justeru sebaliknya, para aktivis kelompok-kelompok OMK yang mau proaktif , mau belajar, mau secara jujur mengusulkan berbagai kreasi dalam liturgi, dan karenanya  berani mencari dan melakukan yang benar, berani mengakui  kesalahan bila terjadi dan berani memperbaikinya) terbukti menjadi aktivis  dengan penghayatan liturgi yang nyata dalam perilaku. Lagipula, jika OMK membuat kesalahan dalam ber-liturgi, ternyata kesalahan itu tidaklah fatal, normal saja. Kesalahan mereka pun kadang-kadang karena pengaruh kelompok kategorial  yang lebih senior. Justeru kelompok-kelompok kategorial yang beranggotakan orang-orang tidak muda lagi lah yang sering bikin kesalahan fatal, dan keras kepala, bukan? Sebaliknya, biasanya dengan taat OMK mau belajar dari kesalahan. Mereka tetap gembira dan kreatif, asalkan pendamping dengan empati mau  setia mendampingi, menjelaskan makna simbol dan hakikat liturgi yang kaya makna itu kepada mereka, memetakan posisi kelompok dalam lebensrauung Gereja lokal, dsb. Saya yakin, dalam kerja sama yang baik dengan pendamping itu dapatlah dihindarkan kesalahan-kesalahan fatal yang tidak perlu terjadi. Sebenarnyalah di antara Liturgi dan OMK ada hubungan batin yang saling mendukung. Liturgi menjadi ongoing formation  bagi OMK. Sedangkan daya kreativitas dan gelora kemudaan OMK membuat liturgi dirayakan dengan bersemangat. Liturgi tanpa keterlibatan orang muda, merupakan tanda nyata kematian Gereja.
\normalsize

\section*{Liturgi Kelompok OMK}

Ketika naskah ini diketik (Mei 2008-pen), kantor Youth Desk – FABC di Manila sedang mengolah survei mengenai penghayatan OMK akan Liturgi Ekaristi. Munculnya jajak pendapat untuk OMK mengenai Ekaristi ini didasari praduga bahwa kerinduan OMK akan liturgi berbanding lurus dengan pengetahuan dan pengalaman mereka ber-liturgi.  Tema Liturgi Ekaristi menjadi pembahasan dalam \textit{Asian Youth Day} tahun  2009 di Manila.  Mengapa tema ini diagendakan? Saya menduga, di satu sisi ada kecemasan kalau-kalau  Liturgi  ”ditinggalkan” alias ”tidak laku” lagi di kalangan OMK. Liturgi disangka tidak mampu menjawab kerinduan OMK di tengah arus percepatan globalisasi yang mengasingkan OMK. Di sisi lain ada pula kecemasan kalau-kalau  OMK di berbagai kelompok kategorial yang masih mau aktif ber-liturgi  mulai ”meninggalkan” kaidah liturgi, alias ”mengikuti maunya sendiri”. Komunitas-Komunitas OMK  lebih mementingkan ”rasa kepuasan kelompok” dalam berliturgi dibandingkan ”rasa universal” Gereja. Dua macam kecemasan itu bermuara pada dua pertanyaan atas satu kenyataan liturgi: 
\begin{enumerate}
\item Bagaimanakah liturgi menjawab kerinduan OMK akan perasaan ditemani oleh ”Yang Ilahi” di tengah arus zaman dan perubahan selera ini? \item Bagaimanakan OMK menyadari tanggungjawab dan penghayatannya akan liturgi yang bergairah karena setia pada aturan Gereja?
\end{enumerate}

Saya menemukan dua prasyarat atas jawaban pertanyaan di atas setelah mengamati beberapa komunitas OMK.  Prasyarat itu adalah 
\begin{enumerate}
\item Jika mereka mendapatkan komunitas yang digembalakan dengan semangat berbagi dan mereka dipercaya dalam kegiatan komunitas. 
\item Jika ungkapan kemudaan mereka diberi ruang dan waktu yang cukup dalam liturgi komunitas.
\end{enumerate}

Pada beberapa kelompok OMK, liturgi mereka hayati sepenuh hati. Tampaknya mereka ”puas” dan selalu rindu dengan liturgi komunitas mereka.  Sebabnya, liturgi tak mereka lepaskan  dari kehidupan komunitas kategorial mereka, dan bahwa komunitas memberi ruang dan waktu bagi karakter kemudaan mereka dalam liturgi. Ada ”gembala” (pendamping/ moderator) yang secara tetap mempercayai mereka dalam kegiatan komunitas. Sang pendamping ini (imam dan biarawati/awam) mendampingi liturgi mereka dengan tak bosan mengajarkan prinsip-prinsip  liturgi  sesuai  maksud Gereja. Beberapa kelompok OMK itu adalah: Komunitas Sant’ Egidio (SE), beberapa komunitas Persekutuan Doa Karismatik Katolik (PDKK) muda-mudi, beberapa sel Komunitas  Tritunggal Mahakudus (KTM) muda-mudi; beberapa presidium Legio Mariae (LM) muda-mudi, kelompok Imago Dei (ID), dan kelompok Doa Taize (DT). Mereka memiliki kesamaan pengalaman, bahwa perjumpaan dengan Allah dalam doa, teristimewa liturgi merupakan puncak dan sumber spiritualitas dan kegiatan komunitas.

\section*{Variasi yang Melegakan}

Kelompok-Kelompok OMK itu  biasa berkumpul untuk berdoa secara rutin.  Pertemuan doa mereka mengikuti bukanlah liturgi karena dan karenanya memiliki variasinya masing-masing.  Dalam hal ini Tata Perayaan Sabda atau Ibadat Sabda atau Doa kelompok kategorial di luar Ekaristi, adalah kegiatan non liturgis atau para liturgi atau devosi. Perayaan Sabda adalah liturgi bila dilakukan dalam Ibadat Harian dan liturgi sakramen termasuk Ekaristi yang meliputi dua bagian utama: liturgi Sabda dan Liturgi Ekaristi. Ibadat Sabda di luar liturgi atau para liturgi dan devosi tidak sangat terikat pada kaidah-kaidah liturgi. Dalam hal ini kreativitas orang muda mendapat ruang yang lebih luas dan nyaman. Sebulan atau beberapa bulan sekali mereka mengundang imam untuk merayakan ekaristi. Umumnya, mereka mengikuti aturan liturgi yang baku. KTM dan PDKK  secara berkala membuat adorasi sakramen Mahakudus. Sebaiknya Adorasi Sakramen Mahakudus dibuat sebelum atau sesudah Perayaan Ekaristi, atau sebagai unsur dari Ritus Penutup Ekaristi, dan bukan di tengah liturgi Sabda atau di tengah liturgi Ekaristi.  Beberapa variasi liturgi, khususnya dalam perayaan ekaristi dan adorasi, tampak paling ekstensif dalam PDKK dan KTM.

Lagu-lagu yang dipakai KTM dan PDKK sebagian besar bercorak mirip pop rohani. Ciri khas liturgi PDKK dan KTM adalah sangat ekspresif. Bernyanyi melambungkan pujian kepada Allah, sambil bertepuk tangan, mengangkat tangan, diiringi musik yang meriah. Di sini dikenal juga pencurahan Roh. Salah satu yang khas pula dalam PDKK dan KTM ialah peran pemimpin doa yang mengantar sesi-sesi lagu, doa dan firman. Dalam perayaan ekaristi, ada kesan bahwa peran imam  sebagai pemimpin resmi liturgi Gereja, tenggelam oleh peran pembawa acara yang juga pemimpin doa. Ambil contoh, pengantar tobat yang dibawakan imam, sering kali masih diulang oleh pemimpin doa dengan lebih panjang. Bagaimana hal ini menjadi variasi yang tidak mengganggu? Kuncinya, dialog persiapan antara imam dan pemimpin doa.  Menjadi soal jika imam tidak diajak bicara dahulu  dan celakanya merasa tidak rela karena dilangkahi dalam memimpin doa. Bisa jadi saat homili menjadi saat pelampiasan ketidakpuasan imam. Namun hal ini jarang sekali terjadi dalam kelompok doa OMK. Pemimpin doa komunitas PDKK dan KTM OMK biasanya bisa bekerja sama dengan baik dengan imam pemimpin ekaristi.

Pertemuan doa  DT mendaraskan mazmur dan doa singkat dalam nada sederhana dengan musik lembut dan dekorasi temaram dengan ikon salib Kristus di altar depan. Doa-doa  Sant’ Egidio mirip dengan DT. ”Mereka biasa berdoa sebentar di depan ikon Yesus. Ini sungguh doa inklusif. Pemimpinnya tidak menempatkan diri di depan umat (di belakang altar), tetapi di depan umat. Kalau ada imam ingin berbagi atau memberi keterangan Sabda Tuhan, barulah beliau tampil di mimbar. Hal ini wajar karena pertemuan doa mereka ini bukanlah perayaan Ekaristi.  Sejauh perlu, doa dilakukan “bersama” di depan ikon Yesus. Doa ini juga merupakan relativisasi di hadirat Tuhan. Orang, setelah seharian bekerja, tidak mensyukuri prestasinya atau mengumpat kegagalan hari itu, melainkan mempersembahkan seluruh perjuangan sepanjang hari kepada Tuhan, entah sukses, entah gagal, entah biasa-biasa saja. Sebetulnya cara pandang seperti ini biasa saja. Hanya saja, doa ini dikemas dalam suatu liturgi yang menyentuh hati: di depan ikon besar Yesus, dalam keredupan cahaya gereja, dengan koor satu suara, sederhana, dengan iringan organ, tapi mengantar pada keagungan Tuhan. Liturgi mereka tidak ada yang istimewa. Pengalaman mereka yang selalu dibawa dalam doa-doa harian dan misa di akhir pekan, itulah yang istimewa.

ID mengungkapkan doa bersama sebagai sarana berkomunikasi dengan Tuhan, untuk mengetahui kehendak-Nya, mendapatkan restu, serta mendapat kekuatan. Dasar kegiatan ID ialah saling menguatkan dalam doa (Gal. 6:2; Ef. 6:18-20). Suasana praise and  worship bersifat fun, gembira penuh syukur. Namun jika dilakukan dalam Liturgi (misalnya Ekaristi), apapun bentuk variasi penyesuaian, hendaknya tidak timbul kesan bahwa di tengah liturgi dimasukkkan acara gembira ria yang bersifat profan.  Mereka pun mencari \textit{rhema} untuk minggu itu dan mengungkapkan dalam liturgi.

LM menempatkan devosi kepada Bunda Maria. Namun yang menjadi sentralnya ialah Allah Bapa, Putra dan Roh Kudus.  Semua pusat devosi, ibadat, liturgi adalah Allah: Bapa, Putra dan Roh Kudus. Maka devosi yang khusus kepada Maria itu tidak harus mengaburkan atau menghilangkan sentralnya, tetapi justru semakin mengarahkan para devosan ke sentral: \textit{ad Iesum, ad Patrem, ad Spiritum Sanctum}, itu sebabnya semua orang yang berdoa rosario atau mempunyai devosi khusus kepada Bunda Maria menyalaminya sebagai Putri Allah Bapa, Bunda Allah Putra dan Mempelai Allah Roh Kudus. \textit{Per Mariam ad Iesum}.  Doa  mereka mengikuti tatacara dalam Buku Pegangan LM. Lagu sangat minim, kalaupun ada,  lagu penghormatan kepada Bunda Maria selalu dinyanyikan dengan khusuk. Kerendahan hati dan kesederhanaan menjadi pola doa  kelompok ini. Doa rosario dan \textit{catena legionis} diwajibkan dalam devosi mereka. Jika ada misa, doa rosario dan catena legionis wajib didoakan tetapi sebelum liturgi, sebelum Ekaristi, yang berarti di luar liturgi dan bukan di tengah liturgi. Ini sangat tepat, walaupun mungkin ada yang kurang paham lalu mendoakannya dalam liturgi. Tak banyak OMK yang terlibat dalam LM dibanding kelompok lain.

Walaupun berbeda ungkapan, namun nyatalah bahwa perasaan mereka sama-sama terangkat kepada kehadiran Yang Ilahi dan iman dimantapkan. Adanya penyesuaian dan variasi itu dirasakan melegakan  OMK dalam menghayati iman akan Allah.

\section*{Liturgi yang Tergairahkan oleh Kemudaan}

Apakah istimewanya liturgi orang muda? Sebenarnya tatacara liturgi mereka tidak ada bedanya dengan liturgi pada umunya. Yang istimewa adalah apa yang mereka rindukan dan  cara ungkapannya. Dalam situasi pertumbuhan menuju masa depan yang tidak serba jelas, di tengah zaman yang hiruk pikuk tidak pasti, liturgi menjadi wahana ungkapan mereka. Apakah liturgi bisa menjawab kerinduan mereka? Alih-alih menunggu liturgi memuaskan mereka, maka banyak kali terjadi, mereka-lah yang berinisiatif menggairahkan liturgi sesuai desakan kuat di dalam dada untuk mengungkapkan kerinduan mereka akan Allah. Sayangnya, para penanggungjawab liturgi kadang-kadang tidak (mau) menanggapi dengan sabar. Akibatnya, gairah OMK sering dikecewakan. Yang diperlukan sekarang adalah penanggungjawab liturgi yang melibatkan OMK dalam tim liturgi, agar perencanaan doa-doa dan lagu, serta variasi lain bisa menjawab kerinduan OMK. Ada aneka warna kelompok doa OMK. Sangat bagus jika mereka dilibatkan oleh para penanggungjawab liturgi di berbagai tingkat (paroki, dekenat/kevikepan, dan keuskupan) untuk menggairahkan liturgi kita. Dengan demikian, tak kan ada lagi kecurigaan dan penilaian sepihak atas OMK seperi pada alinea pertama tulisan ini. Mereka pun merasa tersapa dan pasti belajar liturgi Katolik dengan lebih baik. Ada satu lagi potret ber-liturgi OMK walaupun sangat jarang. Yakni OMK yang tergerakkan oleh liturgi sedemikian rupa, sehingga terinspirasi untuk terjun dalam perjuangan menegakkan perdamaian dan keadilan. Mereka pun perlu dilibatkan dalam perencanaan dan pelaksanaan liturgi agar liturgi benar-benar ”puncak dan sumber” hidup beriman bagi OMK.

\section*{OMK dan Musik Liturgi}

Bayangkanlah suatu perayaan Liturgi Ekaristi di gedung gereja yang besar dengan ribuan OMK. Bagaimana jika tanpa musik? Nah, musik ialah salah satu kegemaran favorit OMK, sejak era zaman batu hingga era dot com ini. Namun musik Liturgi, memiliki ketentuan liturgis. Apakah OMK masih tertarik dengan musik liturgi? Atau, apakah musik liturgi masih mampu berdaya pikat terhadap OMK?

Perayaan Liturgi  tidak melulu  pikiran (ratio) . Liturgi selalu meliputi tata gerak dan menyangkut  seluruh kekayaan cita-rasa batin yang mendorong setiap orang untuk mengungkapkannya secara lahir. Wujudnya doa, permohonan, pujian, sembah sujud, dan semacamnya. Relasi dengan Allah ialah misteri. Maka apa pun yang sulit dinyatakan dalam kata-kata, diwujudkan dalam seni yaitu musik, syair, nyanyian, lukisan, pahatan yang menembus misteri relasi Allah dan manusia. Itulah jiwa Liturgi, yaitu Allah yang selalu ingin mengkomunikasikan diri kepada manusia dan manusia yang rindu menyambut-Nya. Ungkapan itu diungkapkan oleh manusia dengan segala dimensi kemanusiaannya. Di sinilah kita tempatkan musik dan seni liturgi. Tujuan Musik Liturgi ialah kemuliaan Allah dan pengudusan manusia (Konstitusi tentang Liturgi, / \textit{Sacrosanctum Concilium}, SC, 112) dan Alkitab memuji lagu-lagu ibadat (Ef 5:19; Kol 3:16)

Yang harus diketahui ialah perbedaan antara Musik / Lagu Liturgi dan Musik / Lagu Rohani Umat.   Ketika ungkapan musik diwujudkan dalam perayaan Liturgi, maka kita mengenal istilah Musik/ Nyanyian Liturgi ; sedangkan ketika ungkapan musik diwujudkan dalam perayaan non-liturgi, maka kita mengenal istilah Nyanyian rohani umat (populer). Dalam Musik/Lagu Liturgi Ada 3 kekuatan yang terkandung sesuai hakikat perayaan Liturgi:
\begin{enumerate}
\item Dinamisme iman pribadi

\item Dinamisme misteri Allah Bapa melalui Kristus dalam Roh Kudus.

\item Dinamisme komunitas Gerejawi sebagai anggota Tubuh mistik Kristus.
\end{enumerate}

Komponen musik Liturgi adalah ungkapan komunikasi-komunal antara  saya – kita – Allah Tritunggal dalam cara yang lebih mesra dan batiniah.

Lagu/Nyanyian Rohani Umat atau Lagu Rohani Populer, tidak mengandung hakikat  tiga kekuatan dinamis terebut. Musik rohani merupakan ungkapan iman personal saja, belum eklesial/Gerejawi dan sering juga tidak memperhitungkan dimensi dinamika Allah Tritunggal. Namun SC 118 mengingatkan agar nyanyian rohani umat dikembangkan secara ahli, sehingga kaum beriman dapat bernyanyi dalam kegiatan devosional dan perayaan-perayaan ibadat menurut ketentuan rubrik.

Dalam dokumen “\textit{Musicam Sacram}” artikel 5 dikatakan sbb:

\begin{quote}
\emph{Sungguh, lewat bentuk  ini (musik-pen), doa diungkapkan secara lebih menarik, dan misteri Liturgi, yang sedari hakikatnya bersifat hirarkis dan jemaat, dinyatakan secara lebih jelas. Kesatuan hati dicapai secara lebih mendalam berkat perpaduan suara. Hati lebih mudah dibangkitkan ke arah hal-hal surgawi berkat keindahan upacara kudu \ldots”}
\end{quote}

Apabila telah dipahami betapa luhurnya misteri yang terjadi selama perayaan Liturgi maka perlulah OMK menyadari betapa luhurnya peran musik dan nyanyian dalam Liturgi sebagai sarana komunikasi dengan yang ilahi dalam kebersamaan.

\textit{Sacrosanctum Concilium} artikel 112 menyebutkan peran musik sebagai tugas pelayanan untuk mendukung ibadat kepada Allah. Paus Pius X menyebutnya sebagai umile ancella. Paus Pius XI menyebutnya sebagai \textit{serva nobilissima}. Paus Pius XII menyebutnya sebagai \textit{sacrae liturgiae quasi administra}. Dan Konsili Vatikan II menyebutnya: \textit{munus ministeriale in dominico servitio} (tugas pelayanan dalam mengabdi Tuhan).

\subsubsection*{Syarat-syarat Musik Liturgi}
\begin{enumerate}
\item    Harus merupakan musik sejati menurut seni musik.
\item    Kata-kata dan nada harus sungguh menghantar manusia kepada Allah, oleh karena itu harus berdasarkan teks Kitab Suci dan Teks Liturgi.
\item    Harus mengungkapkan daya-daya seni dan religiositas dalam dialog dan tanda-tanda simbolik lainnya selama berlangsung perayaan, di mana Allah dimuliakan dan umat beriman dikuduskan.
\end{enumerate}

Maka Musik Liturgi semakin suci, bila semakin erat hubungannya dengan upacara Ibadat, entah dengan mengungkapkan doa-doa secara lebih mengena, entah dengan memupuk kesatuan hati, entah dengan memperkaya upacara suci dengan kemeriahan yang lebih semarak. Gereja menyetujui segala bentuk kesenian yang sejati, yang memiliki sifat-sifat menurut persyaratan Liturgi, dan mengizinkan penggunaannya dalam Ibadat kepada Allah (SC 112).

\section*{Ekaristi untuk Orang Muda}

\textit{Actio Pastoralis}, instruksi Kongregasi Ibadat mengenai “Misa untuk kelompok-kelompok khusus” 15 Mei 1969 menyatakan ”Gereja sangat menganjurkan penyelenggaraan Misa untuk berbagai kelompok dalam paroki baik territorial maupun kategorial sebab mempunyai dampak lebih mendalam terhadap penghayatan hidup kristiani, saling mendukung dalam perkembangan hidup rohani dan kesaksian iman”.

Untuk itu diperlukan berbagai penyesuaian, yang dapat dibagi dalam dua kategori:
\begin{enumerate}
\item penyesuaian akomodatif
\item penyesuaian inkulturatif
\end{enumerate}

\textbf{Akomodatif}  maksudnya penyesuaian dalam hal-hal yang tidak berkaitan dengan budaya setempat misalnya: bacaan, nyanyian, cara berkomunikasi, tata-gerak ritual, dramatisasi, tarian,  dll. Misa Orang Muda perlu banyak penyesuaian sesuai dengan jiwa mereka agar sungguh berdaya-guna bagi hidup mereka, namun mengindahkan kaidah liturgi.

\textbf{Inkulturatif} maksudnya  unsur-unsur budaya setempat di mana dituntut studi yang mendalam mengenai unsur-unsur yang dapat dimanfaatkan untuk membantu kelompok-kelompok masyarakat tertentu dalam penghayatan iman mereka.

\section*{Harapan}

Semoga dengan demikian musik dan lagu liturgi menyemarakkan cita rasa batin yang terdalam dari OMK. Musik yang dipakai dalam liturgi haruslah yang menjunjung rasa hormat sembah bakti dalam memuliakan Allah, dan membuat OMK sadar dan aktif dalam liturgi.


\sumber{Yohanes Dwi Harsanto Pr\\ (Sekretaris Eksekutif komisi Kepemudaan KWI)\\
http://www.katolisitas.org} 