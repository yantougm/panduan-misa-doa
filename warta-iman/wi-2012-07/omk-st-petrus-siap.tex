\chap{OMK  St. Petrus  Maguwo\\
 Siap Menjadi Bagian Harapan Gereja}
 
 
       OMK Ya! adalah mereka yang sudah dibaptis dalam Gereja Katolik berusia antara 13 – 35 tahun, belum menikah, aktif atau tidak aktif, berada dalam lingkup teritorial paroki, (stasi, wilayah, lingkungan) ataupun masih dalam lingkup teritorial paroki tetapi berada dalam satu wadah kategorial seperti Legio Maria, Choice, Kharismatik Katolik, Misdinar atau juga di luar kedua lingkup dan wadah itu seperti Pemuda Katolik, PMKRI dan lain sebagainya. Inilah OMK yaitu orang muda katolik yang menyatu dalam satu iman tetapi berada dalam lingkup dan wadah berbeda.
 
     Belum ada data resmi untuk tahun 2012 ini berapa jumlah OMK di Indonesia. Tetapi diperkirakan  hampir 60\% dari jumlah seluruh umat Katolik yang ada di Indonesia adalah OMK. Di lingkungan St. Petrus sendiri berdasarkan buku Informasi Lingkungan yang diterbitkan  tahun 2012 ini, jumlah OMK ada 44 orang, suatu jumlah yang relatif kecil sekitar 20\% dari jumlah umat yang ada 201 orang. Akan tetapi jumlah ini bisa  dipandang cukup sebagai jumlah yang lumayan untuk ukuran suatu lingkungan atau jumlah yang cukup juga untuk suatu keberadaan organisasi yang bersifat kewilayahan/stasi. Sebagai kelompok yang besar sudah barang tentu OMK menjadi tumpuan dan harapan Gereja Katolik Indonesia di masa depan. Dalam konteks inilah OMK khususnya OMK kita Lingkungan St. Petrus harus bisa memposisikan diri untuk siap mengambil bagian dari harapan Gereja itu.
 
\section*{Harapan Gereja Terhadap OMK}
 
      Dalam Temu Raya OMK Keuskupan Agung Semarang yang mengangkat tema “\textit{Tunjukkanlah Merahmu, Buktikanlah Putihmu 100\% Katolik, 100\% Indonesia}” yang berlangsung di Kanisius Yogyakarta 21 Agustus 2011 Bp. Uskup Agung Yohannes Pujasumarto  mengharapkan OMK menjadi orang yang rela berkorban bagi nusa dan bangsa seperti telah dilakukan oleh para pahlawan kita Agustinus Adisutjipto dan Ignatius Slamet Riyadi. Sisi lain OMK hendaknya juga - ini yang utama dan pertama - meneladani Yesus, orang muda yang berdarah-darah dan mati muda untuk menyelamatkan seluruh bangsa. Dengan kata lain Bp. Uskup mau mengatakan pentingnya OMK memiliki keterlibatan penuh bagi keselamatan Republik Indonesia, namun sekaligus mengingatkan pentingya OMK  hidup dan dihidupi oleh iman akan Yesus Kristus  sebagaimana itu dinyatakan dalam tema World Youth Day di Madrid Spanyol 16 -21 Agustus  2011. “\textit{Berakar dalam Kristus dan Dibangun atas Dia, hendaklah Bertambah Teguh dalam Iman” (Kol.2:7)} (pujasumarta.multiply.com)
     
      Hal yang sama juga dikumandangkan oleh Komisi Kepemudaan KWI dengan akan diselenggarakannya Indonesian Youth Day (IYD) 2012 di Sanggau, Kalimantan Barat, 20 - 26 Oktober 2012 mendatang dengan mengusung tema “\textit{Berakar dan Dibangun dalam Yesus Kristus, Berteguh dalam Iman}” dan sub tema “\textit{OMK Makin Beriman, Makin Meng-Indonesia}”. Menurut  Ketua Umum Panitia IYD 2012 dan juga menjabat sebagai Sekretaris Komisi Kepemudaan KWI Romo Yohanes Dwi Harsanto Pr, yang akrab disapa Romo Santo, tema ini mau menegaskan pentingnya OMK memiliki semangat hidup meng-Gereja dengan penghayatan iman yang mendalam. IYD 2012 ingin membawa Kristus kepada OMK dan membawa OMK kepada Kristus sekaligus menyadarkan tugas perutusannya bahwa OMK penting bagi Indonesia dan Indonesia penting bagi OMK.  
\small\textit{(hope-komkepbanjarmasin.blogspot.com)}\normalsize
 
        Harapan besar Gereja Indonesia yang sedang dan telah menggelinding pada tingkat keuskupan dan nasional ini – tentu juga banyak paroki – sesungguhnya selain menunjukkan betapa besar perhatian Gereja terhadap OMK, juga karena  didasari keprihatinan Gereja terhadap bangsa kita yang makin rusak dalam berbagai bidang kehidupan. Melalui beberapa Surat Kegembalaan Prapaskah (SKP) dan Nota Pastoral (NP)  yang dikeluarkan KWI dalam kurun waktu 1997 hingga 2006 Gereja senantiasa mengingatkan  bangsa Indonesia bahwa  bangsa kita sedang bergerak menuju kerusakan moral yang luar biasa (SKP 1997 \& 2001). Bila hal ini terus berlangsung maka kehancuran keadaban publik tak akan terhindarkan lagi. Pada gilirannya kesejahteraan bangsa juga tidak tercapai (NP 2003). Gereja katolik sebagai Gereja yang bercirikan 100\% katolik dan 100\% Indonesia terpanggil untuk ikut serta memperbaiki bangsa dan  mencita-citakan bangsa ini segera memiliki habitus baru terutama habitus baru dalam tata ekonomi yang berkeadilan. Akan tetapi hingga sekarang situasi bangsa tidak beranjak menjadi lebih baik bahkan lebih buruk (NP 2004 \& 2006).
 
       Dalam kerangka itulah Gereja Katolik mencintai dan memandang penting peran OMK dalam    kehidupan bernegara dan berbangsa serta berharap agar OMK bukan saja menjadi putra-putri Gereja yang beriman baik, tetap juga dengan itu menjadi agen-agen transformatif yang mampu menggerakan perubahan menuju Indonesia yang lebih baik. Gereja Katolik Indonesia ingin agar Gereja ke depan  tidak kehilangan masa depannya sebagaimana telah terjadi di Gereja-Gereja Eropa.  OMK  diharapkan tidak  menjadi bagian problema seperti di Negara-negara Eropa melainkan justru menjadi bagian solutif. Kiranya kita pantas menyambut dengan baik apa yang pernah dikatakan Beato Yohanes Paulus II mengenai perlunya evangelisasi baru bagi kaum muda yang menekankan pentingnya mereka bukan hanya sebagai obyek pastoral tetapi juga sebagai  agen dan rekan dalam melakukan misi Gereja dalam berbagai karya kerasulan dalam pelayanan dan cinta (\textit{Eclessia in Asia,} 47).
 
        Kiranya jelas dengan memperhatikan tema-tema yang diusung dalam berbagai pertemuan baik pada tingkat paroki, keuskupan atau pun nanti pada tingkat nasional,  Gereja mengharapkan OMK menjadi komunitas Gereja muda terdepan yang tangguh baik dalam kehidupan beriman maupun dalam tugas perutusannya bagi dan dalam kehidupan bermasyarakat dan berbangsa terutama untuk menanggapi situasi bangsa yang saat ini sedang mengalamai carut-marut menuju kehancuran peradaban.
 
       OMK Lingkungan St. Petrus sebagai bagian dari OMK Indonesia hendaknya memahami dan menyadari betul  harapan Gereja Katolik saat ini. Partisipasi Gereja Katolik Indonesia untuk menyelamatkan bangsa dan negara dari kehancuran peradaban dan membangun Gereja masa depan akan sangat tergantung dari peran aktif OMK sendiri dalam menanggapi harapan Gereja itu. Kita percaya harapan Gereja itu telah menjadi benih yang tertanam dalam hati setiap OMK Lingkungan St. Petrus Maguwo yang siap tumbuh dan berkembang dalam setiap usaha yang telah dan hendak dilakukan.
 
\section*{Spiritualitas Kaum Muda Katolik}
 
      Salah satu ciri utama OMK yang tidak bisa tidak harus ada adalah iman akan Kristus. Ciri yang lain tidak mempunyai daya ikat sekuat iman bahkan tanpa dengannya OMK tidak bisa ada apalagi hidup. Iman menentukan eksistensi OMK. Kekatolikan yang menyatukan pribadi-pribadi OMK satu dengan yang lain adalah wujud kesetiaan pada iman itu. Jadi iman akan Kristus adalah roh yang menjiwai OMK. Roh itu juga yang menggerakkan dan menuntun OMK menggapai harapan dan cita-cita Gereja. Oleh sebab itu OMK harus terus menerus menggeluti imannya agar lebih mendalam dan tangguh dalam segala situasi.
 
       Iman adalah proses terus menerus menanggapi panggilan Allah yang  kasihNya kepada kita nyata dalam hidup Yesus Kristus dan RohNya. Iman adalah jawaban kita atas kasih itu yang mewujud bukan hanya dalam perayaan liturgi, tetapi terlebih dalam hidup sehari-hari. Persoalan-persoalan sosial termasuk persoalan bangsa yang kita hadapi adalah medan bagi perwujudan iman itu. Oleh karena itu iman sebagai bentuk tanggapan atas panggilan atau kasih Allah kepada kita harus kita hayati dalam dialog terus menerus dengan Allah melalui Kristus dan RohNya. Dalam dialog itu kita akan merasakan bahwa Allah itu ada dan nyata dalam hidup kita, bahwa Allah itu membantu dan melindungi kita serta mendorong kita untuk kuat dan berani menghadapi persoalan hidup bahkan persoalan hidup berbangsa dan bernegara dan dalam semuanya itu tugas perutusan kita sebagai orang katolik atau OMK  menjadi nyata. Inilah spiritualitas OMK yang mengalir dari iman akan Trinitas, Allah Bapa, Allah Putra dan Allah Roh Kudus.
     
      Ada dua model biblis yang bisa menjadi gambaran penghayatan spiritualitas OMK dalam menyikapi soal-soal hidup bersama terutama persoalan bangsa saat ini. Model pertama adalah pengalaman iman Abraham dalam Tuhan dan model kedua adalah perjuangan Musa demi keadilan bagi bangsanya. Kedua model ini memuncak pada perjumpaan dengan Kristus sebagaimana dikatakan dalam Injil Yohanes bahwa Yesus bertemu Abraham dalam sejarah sebab “ sebelum Abraham ada Aku ada” (Yoh. 8:58) Demikian pun Musa sebagaimana ditegaskan dalam surat St. Paulus kepada umat Ibrani “ Karena iman maka Musa setelah dewasa menolak disebut anak puteri Firaun, karena ia lebih suka menderita sengsara dengan umat Allah daripada untuk sementara menikmati kesenangan dari dosa. Ia menganggap penghinaan karena Kristus sebagai kekayaan yang lebih besar dari pada semua harta Mesir sebab pandangannya ia arahkan kepada upah” (Ibr. 11: 24-26)
     
        Dalam sejarah keselamatan, kita mengenal bagaimana Abraham meninggalkan rumah dan keluarganya karena ketaatannya kepada suara batinnya untuk berangkat ke tempat yang tidak jelas baginya. Dengan seluruh penderitaan yang ia alami selama pencarian panjang itu termasuk permintaan mengorbankan puteranya Ishak anak yang dicintai, Abraham memperlihatkan  ketaatan kepada Dia yang menuntunnya ke tempat yang ia belum kenal. Sikap lepas Abraham dalam ketaatannya kepada Tuhan ini akhirnya berujung pada suatu pembentukan bangsa. Inilah suatu perjalanan iman yang dari hari ke hari semakin mendalam hingga akhirnya Tuhan menganugerahi suatu bangsa, suatu buah anugerah iman.
 
      Berbeda dengan Abraham, Musa mengawali perjalanan imannya justru dari realitas kehidupan yaitu dari keterlibatannya kepada bangsanya. Ia tidak mau menjadi bagian dari sistem yang menindas melainkan justru menjadi bagian dari orang yang tertindas. Kesiapan dan ketaatannya kepada cinta yang membebaskan ini Musa akhirnya berjumpa dengan Tuhan ketika ia sampai ke puncak gunung Sinai. Inilah cinta yang berakhir dalam Tuhan.
     
     Dari kedua model biblis pengalaman Musa dan Abraham, kita bisa memahami bahwa penghayatan spiritualitas OMK mendapat bentuknya dalam sinergitas dari kedua model biblis itu. Apa yang didapat dari pengalaman iman (Tuhan) harus berujung pada cinta kepada bangsa Indonesia dan sebaliknya cinta kepada bangsa Indonesia harus berakhir pada cinta kepada Tuhan.   Setiap pribadi OMK  harus menjadi seorang contemplativus simul in actiones sebagaimana St. Ignasius dari Loyola. yaitu seorang yang beriman dan serentak mewujudkan imannya dalam keterlibatanya dalam soal-soal sosial termasuk hidupnya atau persoalan bangsa. Keterlibatan ini bukanlah merupakan semacam penerapan iman melainkan sungguh-sungguh merupakan bagian integral dari seluruh perutusannya sebagai kaum beriman. Seorang pribadi OMK harus siap meninggalkan “ruang Gereja” untuk menggemakan jawabannya kepada Allah dalam persoalan hidup sehari-hari termasuk persoalan bangsa.
 
\section*{Siap dalam Harapan Gereja}                                
 
     Sekarang apa yang baik dilakukan oleh OMK khususnya OMK kita St. Petrus Maguwo? Kita masih ingat apa yang diharapkan oleh Romo Yulius Blasius Fitri Gutanto Pr dalam kotbahnya ketika ia melantik pengurus OMK Lingkungan St. Petrus Maguwo periode 2011 - 2013 di Kaliurang  Wisma Omah Jawi 5 – 6 Maret 2011 bahwa setiap pribadi OMK harus menjadi pribadi beriman yaitu orang katolik sejati, bukan orang katolik KTP. Dalam arti lebih luas apa yang dikatakan Romo mempunyai relevansi jelas bahwa orang katolik sejati adalah orang katolik 100\%, tetapi  juga 100\% orang Indonesia. Kesejatian orang katolik terletak pada kesetiaan  iman dan keterlibatannya kepada persoalan hidup baik persoalan hidup masyarakat,  bangsa dan  negara.
 
      Memposisikan diri sebagai OMK sejati tidak lain adalah OMK yang siap melaksanakan tugas perutusannya seiring dengan derap langkah Gereja Katolik Indonesia saat ini. Akan tetapi kesiapan perutusan demikian tidak berarti menunggu dalam keadaan pasif, tidak berbuat sesuatu,  melainkan justru menunggu dalam keadaaan aktif, seperti digambarkan dalam perumpamaan 10 gadis yang menunggu mempelai datang, lima diantaranya bodoh dan lima bijaksana. (Mat. 25:1 – 13) Lima yang terakhir merupakan gambaran orang yang menunggu dalam keadaan aktif, karena mereka mengerti apa yang dilakukan dalam keadaan atau selama  menunggu mempelai datang, sehingga ketika mempelai datang mereka tidak kehabisan minyak.
 
      Saya yakin OMK Lingkungan St. Petrus tahu apa yang harus dilakukan saat ini  di sini di tempat ini, di tingkat wilayah, stasi, Paroki  dan secara khusus pada tingkat lingkungan kita, sebagai perwujudan diri dari menunggu dalam keadaan aktif. Berbagai program dan cara melakukan kegiatan tentu sudah dibuat atau seharusnya dibuat. Diharapkan program itu berjalan dengan baik, sebab melalui program dan cara bergiat yang berjalan menunjukkan bahwa roh itu atau iman  akan Kristus mewujud dalam pola penghayatan Abraham dan Musa. Hanya soalnya seringkali program kita tidak berjalan dengan baik atau bahkan mungkin gagal. Tidak apa!
     
       Oleh karena itu barangkali program harus dibuat berdasarkan dialog dengan realitas atau situasi atau kebutuhan umat agar lebih realistis dibanding dengan program berdasarkan ide atau gagasan sekalipun baik.dan diperlukan. Dialog dengan orangtua lingkungan misalnya mungkin penting dilakukan untuk menemukan apa yang perlu dan baik bagi lingkungan, sehingga dengan program-programnya OMK mampu menjadi motor penggerak kehidupan lingkungan atau memiliki peran yang siginifikan. Demikian kiranya berlaku juga pada tingkat wilayah, stasi, Paroki, bahkan dengan masyarakat. Model dialog perlu dipilih dan disesuaikan dengan bagaimana OMK mau melakukan dalam bentuknya. Musa menemukan Tuhan karena dialog dan terlibat dengan persoalan bangsa dan Abraham  dialog dengan imannya menemukan  masyarakat bangsa.
 
      Tulisan ini  bukan hendak membahas program OMK, melainkan mau mengingatkan OMK khususnya OMK lingkungan kita bahwa mereka sekarang berada dalam “gerbong” harapan Gereja Katolik yang sedang melaju menuju masa depan dan berpartisipasi dalam usaha-usaha menyelesaikan persoalan bangsa. Lingkungan St. Petrus kiranya  merupakan lini terdepan dari persoalan-persoalan sosial kita dan dari situlah OMK mulai. Inilah panggilan perutusan kaum muda. Inilah kesiapan OMK lingkungan St. Petrus dalam mengambil bagian dari harapan Gereja Katolik Indonesia.
 
      Dengan memahami apa yang diharapkan Gereja Katolik terhadap OMK dan  dengan penghayatan spiritualitas yang mengalir dari Trinitas dalam model penghayatan biblis Abraham dan Musa, OMK kita khususnya OMK Lingkungan St. Petrus diharapkan selalu berada dalam kesiapan tugas perutusannya melalui berbagai program lingkungan yang dibuatnya. Niscaya dengan kesadaran demikian lingkungan St. Petrus dan masyarakat sekitarnya akan menjadi ladang perwujudan iman OMK dan buahnya terasa khususnya bagi umat lingkungan.
 
\sumber{Yogyakarta, 10 Juni 2012,\\
Hari Raya Tubuh dan Darah Kristus\\(AS)}