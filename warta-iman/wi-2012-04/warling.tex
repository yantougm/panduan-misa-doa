\chap{Warta Lingkungan}

\subsection*{APP}
Bulan Maret 2012 sudah masuk dalam masa Prapaskah. Sesuai dengan tradisi, setiap masa Prapaskah diadakan Aksi Puasa Pembangunan yang kegiatannya antara lain adalah ibadat APP di lingkungan. Untuk tahun ini Keuskupan Agung Semarang (KAS) menetapkan tema APP: \textit{Umat Katolik Sejati Harus Peduli dan Berbagi}. Dalam pelaksanaan di lingkungan St. Petrus, ibadat APP banyak diisi dengan \textit{sharing} yang mengacu pada buku panduan dari KAS.
Topik APP berawal dari baptis. Kapan kita dibaptis, kesan-kesan saat dibaptis, dan relevansinya dengan hidup menggereja dan bermasyarakat.

\subsection*{Misa pemberkatan rumah dan mitoni}
Bulan ini umat St. Petrus bertambah lagi dengan satu keluarga yang secara resmi bergabung sebagai warga lingkungan. Keluarga Bapak R. Mulyadi yang bertempat tinggal di Nanggulan, mengadakan misa syukur pemberkatan rumah dan sekaligus mitoni pada tanggal 22 Maret. Cicilia Nony Prayoga yang merupakan putri dari Bapak/Ibu Mulyadi menantikan kelahiran putranya bersama dengan suami tercinta
Bernadus Budhiprayoga.

Misa dipimpin oleh Rm. Albertus Purnomo, OFM yang merupakan kenalan baik keluarga R. Mulyadi saat masih di Jakarta. Dalam homilinya Romo menekankan bahwa Musa dapat melakukan tawar-menawar dengan Tuhan karena kedekatan Musa dengan Tuhan. Kenapa Musa dekat dengan Tuhan, karena Musa sering berdoa dan menaati perintah-Nya. Oleh karena itu kalau kita ingin dekat dengan Tuhan, maka rajin-rajinlah berdoa dan senantiasa menaati perintah-Nya.

\subsection*{Pendaftaran Krisma}
Telah diumumkan di gereja bahwa sakramen Krisma untuk paroki Marganingsih Kalasan akan dilangsungkan bulan September 2012. Calon penerima sakramen Krisma dapat mendaftarkan diri ke Ibu Munarti, Bapak Neo Suradi, atau kepada ketua lingkungan, dengan menyerahkan fotokopi surat baptis.
