\chap{\textit{Cerpen:}\\Mabuk Harta}

\small
	Sore itu saya mampir ke warung sate langganan saya dulu. Letaknya ada di pinggir kota. Ternyata kini warungnya semakin besar, langganannya semakin banyak. Mobil berderet- deret di tempat parker. Itu bukti bahwa bakaran sate dan tongseng racikan Mas Yus, lengkapnya Yustinus Kamidi disukai banyak orang.
	
	``Kemana saja kamu, Gung?'' Tanya Mas Yus begitu melihat saya duduk di lincak (kursi panjang dari bambu). ``Saya kira kamu sudah almarhum!'' selorohnya. Tetapi sesaat kemudian terdengar bunyi dering handphone dan Mas Yus  merogoh saku bajunya, mengeluarkan handphone Blackberry terbaru lalu berbicara panjang lebar. Lebih dari dua puluh menit! Setelah selesai Mas Yus menghampiri saya, 
	
	``Sekarang saya bisnis mobil, jual beli rumah, tanah dan saya juga menjadi anggota suatu partai Pemenang Pemilu di Negara ini. Warung sate ini hanya bisnis sambilan. Hitung-hitung untuk menampung tenaga kerja. Ha \ldots ha \ldots ha \ldots  ``

	Mas Yus lalu cerita, entah benar entah bohong, kini ia lebih konsentrasi pada bisnis mobil, rumah, tanah, juga aktif dalam partai terkaya di negri ini. ``Mereka adalah relasi-relasi bisnis andalan saya. Penghasilanku tiap bulan tidak menentu. Kadang bisa 50 juta rupiah, paling pahit 15 juta. Ada kalanya sampai 100 juta.'' Ujarnya penuh bangga.   
	
	Tiba-tiba blackberry Mas Yus berdering lagi, ``Halo \ldots  oh Pak Rony \ldots  Aduh pak \ldots  tolonglah pak \ldots  jangan dilaporkan ya pak. Pasti saya beri `hadiah' buat Pak Rony ok?  Aduh jangan 50 juta dong pak berat \ldots  20 juta saja ok? Baik besok pagi saya transfer ke rekening Pak Rony \ldots  apa \ldots  o \ldots  di transfer ke rekening teman pak Rony saja. Ok, Ok, tolong nomor rekeningnya di SMS ke saya pak. Ok. Thanks pak. Bye!!!''

	Niat saya untuk menikmati sate dan tongseng lenyap seketika. Disaat Mas Yus berbicara dengan blackberrynya, terbayang di benakku praktek politik suap sudah ada pada zaman Yesus, saat pengkhianatan Yudas Iskariot menjual informasi keberadaan Yesus di taman Getsemani pada Imam kepala bangsa Yahudi. Terbayang kembali dimana para tokoh dan pemuka agama bangsa Yahudi memberi sejumlah uang kepada para pengawal kubur Yesus, agar tidak mengatakan kebenaran bahwa Yesus telah bangkit. Dengan uang penutup mulut itu, mereka lalu merekayasa kebangkitan Yesus: Yesus tidak bangkit, tetapi murid-murid Yesuslah yang mencuri Jenazah-Nya disaat mereka sedang tidur.

	``Wah, sorry ya Gung. '' Ujar Mas Yus dengan penuh senyum. Kemudian blackberry Mas Yus berdering kembali, dan Mas Yus terlihat seperti membaca SMS, kemudian blackberry itu dimasukkan kembali ke saku bajunya.

	``Gimana \ldots . Kamu Masih jadi prodiakon Paroki?'' Tanya Mas Yus kepadaku.

	``Ya \ldots  amanah dari teman-teman masih meminta kepadaku untuk melayani umat Gereja.'' Kataku.

	``Aku sekarang tidak dapat aktif seperti dulu. Kadang ke gereja pun tidak sempat. Kepada umat dan pengurus lingkungan sudah saya katakan, saya tidak bisa aktif lagi. Tetapi jika mereka butuh dana, silahkan ketuk pintu rumahku. Begitu juga untuk Paroki. Entah sudah berapa juta saya menyumbang. Saya ikhlas. Sebab itu demi gereja. Katanya Romo Paroki juga maklum kalau saya lalu jarang ke gereja. Maklum, bisnis lagi berkembang. Jangan sampai hanya karena ikut Misa hari Minggu puluhan juta malah melayang. Sayang, bukan?'' papar Mas Yus berapi-api.

	``Zaman sekarang ini orang harus berpikir realistis, Bung!'' ujar Mas Yus lagi. ``Jangan buang-buang waktu. Setiap Jam, setiap menit, setiap detik, jika itu bisa menghasilkan duit, kenapa harus kita buang dengan percuma. Sukses itu tidak turun dari sorga, tapi dari kerja keras dan memanfaatkan waktu secara efektif. Urusan sorga, ya kita pikirkan, namun santai-santai saja. Toh kita masih muda. Nanti kakau umur sedah mendekati enam puluh tahun, barulah kita berpikir tentang sorga secara serius.''

	``Bagaimana jika umur kita tidak sampai enam puluh tahun?'' pancing saya.

	Mas Yus tertawa. ``Itu pertanyaan orang pesimis!'' tukasnya. ``Kalu kita banyak duit, hidup kita senang, gizi tercukupi, maka penyakit tidak mudah datang. Jadinya kita bisa berumur panjang. Ha\ldots ha\ldots ha\ldots ''

	``Siapa bisa menduga datangnya maut? Bukankah Tuhan Yesus pernah mengumpamakan maut itu datang seperti pencuri?''

	``Sekali lagi itu pendapat orang yang pesimis memandang hidup ini. Bagi yang optimis, semua peristiwa di dunia ini bisa dinalar. Jika kita sehat walafiat, apa mungkin kita mati mendadak?'' tukas Mas Yus lagi. ``Duit, duit Bung! Kita butuh duit, Karena duit sekarang saya punya tiga mobil, empat rumah dan deposito yang lumayan. Dengan duit itu pula saya sekarang bisa kemana-mana, mau makan apa saja bisa saya beli. Bandingkan sepuluh tahun yang lalu, untuk berwisata ke bali saja rasanya Cuma bisa mimpi. Sekarang kalau mau tiap minggu, saya mampu. Ha\ldots ha\ldots ha\ldots ''

 	Laki-laki itu kini lebih mementingkan urusan bisnis melebihi segalanya. Waktu untuk Tuhan pun dikalahkan. Ekaristi kudus kalah oleh urusan duit!

Seiring perjalanan waktu, saya beberapa kali mencoba datang ke rumah Mas Yus untuk mengajaknya ikut pertemuan APP, tetapi Mas Yus masih lebih mementingkan bisnisnya daripada imannya. Dulu Mas Yus adalah `motor penggerak' pendalaman Kitab Suci kaum muda, pandai dalam memimpin sharing iman, hebat dalam memandu APP, sehingga saya masih terpesona dengan gaya kepemimpinannya.

\normalsize
Suatu hari Mas Yus datang ke rumahku. ``Gung, tolonglah aku. Aku di tipu oleh rekan bisnisku. Dan aku sekarang dituntut oleh partaiku, karena uang partai telah aku salah gunakan untuk kepentingan bisnisku. Mereka sekarang mengambil semua hartaku, rumah, mobil, dan sebagainya. Dan kini aku sedang menghadapi tuntutan di pengadilan dengan tuduhan korupsi.'' Tutur Mas Yus dengan mata basah berurai air mata.

``Apa yang bisa saya bantu Mas Yus.'' Ujarku penuh iba.

``Aku butuh saran dan pendapatmu, apa yang harus aku lakukan? Maaf, saat ini aku tidak bisa berpikir. Hatiku takut, galau, cemas dan stress'' ujar Mas Yus.


``Mari, Mas Yus ikut saya. Saya hanya bisa membantu secara spiritual.'' Kataku sambil mengajak Mas Yus menuju ke sebuah kamar. Kamar ini memang saya sediakan untuk merenung dan berdoa.  Di dinding kamar itu ada 5 buah gambar/poster yaitu:
\begin{enumerate}
\item Yesus sedang berdoa di Taman Getsemani. ``Masih ingat bagaimana kejiwaan/psikis Tuhan Yesus pada saat Dia akan menghadapi sakratul maut? Mohonlah Peneguhan-Nya.''
\item Yesus dihakimi dan dijatuhi hukuman mati. ``Teguhkanlah imanmu.''
\item Yesus memanggul salib. ``Panggullah salibmu.''
\item Yesus disalib. ``Taatlah dan mohonlah ampun atas dosa-dosamu.''
\item Yesus bangkit dari kubur. ``Bangkitlah, engkau di utus mewartakan kebenaran.''
\end{enumerate}

``Terima kasih Gung! Aku sekarang mengerti.'' Kata Mas Yus dengan wajah cerah dan tegar.

Saya mengangguk-angguk sambil tersenyum. Jalan pertobatan itu akhirnya datang juga, meskipun pahit rasanya. 
	

\sumber{Medio, Maret 2012\\Bravo Sierra}