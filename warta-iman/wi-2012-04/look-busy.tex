\chap{\textit{Look Busy}}

\begin{wrapfigure}[7]{r}{2.5cm}
\includegraphics[scale=0.35]{gambar/look-busy.png}
\end{wrapfigure} 

Pada saat saya sedang dalam perjalanan pulan kerumah, sebuah setiker yang ditepelkan di belakang mobil sedan yang berada di depan saya menarik perhatian saya. Stiker itu tertulis dengan kata `\textit{Look Busy}'.

Ketika saya menyerap kata-kata di stiker tersebut di dalam pikiran, saya jadi melihat segala kesibukan orang-orang di sekitar saya kenal atau tidak. Sadar atau tidak, kita sering merasa bahwa kegiatan kita begitu banyak sedangkan waktu itu terasa sedikit. Dua puluh empat (24) jam seolah tidak cukup untuk ``menampung'' segala kegiatan kita di dunia ini.

Tidak heran bila ada orang di dunia yang hidup dari satu kegiatan dan melakukan kegiatan lain. Tidur dianggap prioritas kesekian untuk dikerjakan. Waktu-waktunya pun habis untuk mengejar uang, prestis, dan segala hal yang semu.

Namun, ada kesibukan yang begitu baik untuk kita kerjakan, yakni sibuk mempersiapkan kedatangan Tuhan Yesus kedua kali. I Yohanes 2:28 menulis bahwa kita sebagai murid-murid harus tetap  tinggal di dalam Kristus supaya ketika Ia menyatakan diri-Nya kali kedua, kita beroleh keberanian percaya dan tidak usah malu terhadapnya.

Kita `tinggal' disini tidak hanya mengenai mempertahankan iman Katholik kita, tetapi juga melakukan amanat agung-Nya, yakni memberitakan Injil ke segala mahkluk di dunia dan menjadikan semua bangsa menjadi murid-Nya. Serta membatis mereka dalam nama Bapa, Putra, dan Roh Kudus.

Kini, di hadapan Anda terdapat dua pilihan: apakah Anda mau terlihat sibuk dengan mengejar segala harta dan kemewahan dunia atau justru yang kedua, Anda mau terlihat sibuk dengan menceritakan kebaikan Tuhan Yesus kepada orang-orang disekitar dan membawa mereka kepada Sang Juruselamat? Pilihan yang tidak sulit jika Anda benar-benar mengasihi-Nya.

Tuhan Yesus akan segera datang dan ini adalah tanda agar kita terus bekerja mempersiapkan jalan bagi kedatangan-Nya kelak.

\sumber{kiriman dari Aditya Bimantara\\ dari ``Look Busy'' \\Renungan Harian Kita April 2011\\  
www.renungan-harian-kita.blogspot.com.} 
\normalsize