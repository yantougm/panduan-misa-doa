\chap{\textit{Renungan:}\\Salib Katolik}

``Mengapa kamu banyak sekali menggantung Salib? Hampir setiap kamar ada Salib. Seakan kamu menyalibkan Yesus dimana-mana.'' Kata Andi kepada Beti ketika Andi berkunjung ke rumah Beti sahabat karibnya. Kebetulan Beti adalah seorang Katolik.

``Anehnya adalah mengapa Salib orang Katolik masih menggantungkan tubuh Yesus di Salib? Bukankah DIA sudah bangkit dan naik ke Surga dan duduk di sebelah kanan Allah Bapa?'' Andi melontarkan pertanyaan retoris kepada Beti. Sebenarnya Andi tahu bahwa Beti pun percaya hal yang sama.

``Orang Kristen percaya bahwa Yesus Kristus telah bangkit dan naik ke Surga. Itulah mengapa Salib orang Kristen tidak menggantungkan tubuh Yesus Kristus. Lagi pula kami tidak menggantung Salib di setiap kamar.'' Lanjut Andi lagi seolah tidak memberi waktu kepada Beti untuk menjawab.

``Lihatlah, Salib-salib itu banyak sekali bentuknya. Ada yang terbuat dari kayu, ada yang dari besi, bahkan ada yang dari aluminium. Bukankah Salib Yesus itu terbuat dari kayu?'' Celoteh Andi lagi sambil memegang Salib aluminium besar yang berdiri kokoh di sebuah meja sudut di ruang tamu. Kali ini Beti hanya tersenyum simpul menanggapi sahabatnya yang seorang Kristen itu.

Seminggu kemudian Beti berkunjung ke rumah Andi.

``Hai, ini foto pacar kamu?'' tanya Beti ketika memandang sebuah foto seorang wanita cantik berukuran besar yang diberi bingkai kayu ukiran yang mewah.

``Hus \ldots Ini ibuku!'' bisik Andi sambil meletakkan telunjuknya di depan bibirnya seolah menyuruh Beti untuk tidak berbicara keras-keras. Maklum Beti tadi bertanya dengan nada cukup tinggi.

``Ibu kamu cantik ya?'' kata Beti lagi dengan suara lebih pelan.

``Hmm \ldots Ya jelas dong. Ibu siapa dulu? Itu foto ibu ketika masih muda, mungkin waktu itu umurnya 25 tahun.'' Kata Andi dengan bangga sambil membetulkan kerah bajunya yang sebenarnya tidak perlu dibetulkan baik lipatan maupun bentuknya.

``Sekarang umurnya berapa? Sekarang ibu kamu ada dimana?'' Berondong Beti.

``Sekarang umurnya sudah 60 tahun. Kebetulan saat ini ibu sedang pergi belanja bersama ayah.'' Jawab Andi dengan sabar menghadapi pertanyaan-pertanyaan sahabatnya itu.

``Lho, kok masih dipajang? Bukankah sekarang ibu sudah berumur 60 tahun? Bukankah sekarang ibu sedang pergi belanja? Kok kamu masih memajang fotonya ketika usianya masih 25 tahun?'' Kali ini Beti lebih memberondong Andi dengan berbagai pertanyaannya.

``Emang kenapa? Apa yang salah dengan itu?'' Andi bertanya balik sambil heran mengapa sahabatnya bertanya hal-hal yang demikian.

``Lho, kamu sendiri kan yang pernah tanya padaku? Waktu itu kamu bertanya: Mengapa Salib orang Katolik masih menggantung tubuh Yesus padahal Yesus telah bangkit dan naik ke Surga?'' Beti menjawab dengan nada halus seolah berusaha mengingatkan Andi akan peristiwa 1 minggu sebelumnya di rumah Beti.

``Kalau begitu boleh dong aku sekarang bertanya hal yang sama tentang ibu kamu?'' Beti bertanya lagi dengan nada lebih lembut. Kali ini Andi telah ingat beberapa pertanyaan yang dia lontarkan kepada Beti seminggu yang lalu.

``Iya ya?'' Jawab Andi. ``Kamu hebat Beti. Aku sekarang mengerti.'' Andi lalu memeluk Beti sahabatnya.

Salib orang Katolik memang menggantungkan tubuh Yesus Kristus. Salib orang Katolik memang beragam bentuknya. Bahkan tidak hanya dari kayu, tetapi juga dari bahan lain seperti aluminium, besi, dan ada juga yang dari baja. Patungnya pun beragam bentuk dan posisi. Tetapi bukan itu esensinya. Salib bagi orang Katolik adalah sebuah media untuk mengingat kisah sengsara Yesus dalam karya keselamatan-Nya bagi umat manusia.

Justru Salib orang Katolik digantung tubuh Yesus Kristus agar kita tahu bahwa Salib itu memang benar Salib Yesus Kristus. Kita benar-benar menghormati kisah sengsara dan wafat Yesus di Salib itu. Di sanalah terbentang misteri keselamatan Allah.

Umat Katolik tidak menghormati kayu salib yang berupa 2 bilah kayu yang disusun bersilangan. Tetapi umat Katolik sangat menghormati kisah sengsara dan wafat Yesus di Salib. Itulah mengapa dalam memvisualisasikan salib, orang Katolik menggantungkan tubuh Yesus di sana. Justru formasi 2 bilah kayu pembentuk salib itu tidak akan ada artinya tanpa Yesus Kristus yang rela mati untuk menebus dosa manusia dengan disalib. Ingatkan bahwa Yesus disalib bersama 2 orang penjahat yang juga disalib di sisi kanan dan kiri-Nya? Jadi kalau ada orang yang hanya percaya kepada 2 bilah kayu bersilangan itu, kita patut bertanya padanya: ``Ini salib siapa? Jangan-jangan salah satu salib dari 2 penjahat itu.''

Kalau ada orang yang ngotot berargumentasi dengan bertanya pada orang Katolik: ``Bukankah Yesus telah bangkit dan naik ke Surga? Kita tidak perlu menggantungkan tubuh Yesus di Salib.''
Kita patut bertanya balik kepadanya: ``Bukankah Yesus telah bangkit dan naik ke Surga? Tetapi mengapa kamu masih merayakan Natal (kelahiran Yesus)? Atau mengapa kamu masih merayakan Paskah (kisah sengsara Yesus)?''

Jadi jangan berpikiran sempit ya? Juga jangan berpikiran bahwa umat Katolik hanya menghormati kisah sengsara Yesus Kristus. Kami sangat menjunjung tinggi Yesus, Sang Sabda, Sang Putra Allah. Kami juga sangat menghormati setiap bagian hidupnya sebagai manusia: sejak dikandung, dilahirkan, hidup sebagai guru, hidup untuk memberitakan Kabar Baik sambil memberikan keselamatan (jasmani dan rohani), sampai saat Dia harus menderita, wafat di kayu Salib, bangkit dan naik ke Surga. Kami juga sangat mengharapkan kedatangan-Nya yang kedua kali untuk menjadi hakim atas dunia ini.

Salib dengan tubuh Yesus itu adalah media paling ampuh bagi kami untuk mengenangkan kisah sengsara Yesus dalam karya keselamatan-Nya. Salib itu juga mengingatkan kami agar kami mampu memikul salib kami yang sebenarnya sangat ringan dan enak. Salib kami tiada artinya jika dibandingkan dengan Salib Yesus Kristus.

``Marilah kepada-Ku, semua yang letih lesu dan berbeban berat, Aku akan memberi kelegaan kepadamu. Pikullah kuk yang Kupasang dan belajarlah pada-Ku, karena Aku lemah lembut dan rendah hati dan jiwamu akan mendapat ketenangan. Sebab Kuk yang Kupasang itu enak dan beban-Ku pun ringan.'' (Matius 11:28-30).

``Barangsiapa tidak memikul salibNya dan mengikut Aku, ia tidak layak bagi-Ku.'' (Matius 10:38. Bandingkan: Matius 16:24, Markus 8:34, Lukas 9:23, Lukas 14:27)
Ia sendiri telah memikul dosa kita di dalam tubuh-Nya di kayu salib, supaya kita, yang telah mati terhadap dosa, hidup untuk kebenaran. Oleh bilur-bilur-Nya kamu telah sembuh. (I Petrus 2:24).

\sumber{Medio Maret 2012\\Bravo Sierra}