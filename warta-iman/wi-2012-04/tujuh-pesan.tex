\chap{Tujuh pesan terakhir Yesus}

\section{Pesan terakhir yang penuh makna}
Kalau seseorang yang kita kasihi meninggal, maka kita mencoba mengingat pengalaman-pengalaman bersama dengan orang tersebut, baik pengalaman suka
maupun duka. Namun, terutama kita mencoba mengingat apa yang diucapkan pada
saat-saat menjelang ajalnya, karena pesan pada saat-saat terakhir adalah
penting dan penuh makna.

Dalam tulisan ini, maka kita akan melihat tujuh pesan Yesus yang diucapkan-Nya
pada saat Dia tergantung di kayu salib, saat-saat akhir hidup-Nya. Dari pesan
terakhir ini, kita akan dapat menangkap hal-hal yang terpenting yang ingin
disampaikan-Nya kepada kita. Tujuh pesan Yesus terdiri dari:
\begin{enumerate}
\item Luk 23:34 "Ya
Bapa, ampunilah mereka, sebab mereka tidak tahu apa yang mereka perbuat."; 
\item Luk 23:43 "Aku berkata kepadamu, sesungguhnya hari ini juga engkau akan ada
bersama-sama dengan Aku di dalam Firdaus." 
\item Yoh 19:26-27 "Ibu, inilah,
anakmu!" dan "Inilah ibumu!"; 
\item Mar 15:34 "Allahku, Allahku, mengapa Engkau
meninggalkan Aku?"; 
\item Yoh 19:28 "Aku haus!"; 
\item Yoh 19:30 "Sudah selesai";
\item Luk 23:46 "Ya Bapa, ke dalam tangan-Mu Kuserahkan nyawa-Ku."
\end{enumerate}

Dari pesan ini, kita melihat bagaimana Yesus ingin membawa keselamatan bagi
semua orang dengan memberikan pengampunan kepada umat manusia, sehingga manusia
dapat bersatu dengan Allah di dalam Kerajaan Sorga, sama seperti Yesus membawa
pencuri di sebelah kanan-Nya ke Firdaus. Bagaimana cara untuk mencapai Kerajaan
Sorga? Yesus menunjukkan agar kita dapat menerima Maria sebagai bunda kita,
senantiasa berharap pada Allah dalam kesulitan, haus akan jiwa-jiwa untuk
diselamatkan, serta terus setia terhadap panggilan kita sampai akhir hayat
kita, sampai tiba saatnya kita menyerahkan nyawa kita kepada Bapa dan kemudian
memulai kehidupan baru di dalam Kerajaan Sorga.

\subsection{Luk 23:34 "Ya Bapa, ampunilah mereka, sebab mereka tidak tahu apa yang
mereka perbuat."}

Pada saat Yesus tergantung di kayu salib, di tahta-Nya yang dipandang hina oleh
banyak orang, Dia melihat dengan jelas drama kehidupan kehidupan manusia, mulai
dari serdadu yang kejam, murid-muridnya yang pengecut, kaum Farisi yang iri
hati, orang-orang yang tidak melakukan apapun ketika mereka melihat
ketidakadilan. Di kayu salib dan juga dalam permenungan-Nya di taman Getsemani,
Kristus juga melihat dosa-dosa seluruh umat manusia, mulai dari Adam dan Hawa
sampai manusia terakhir. Ini berarti Dia juga melihat semua dosa kita. Inilah
yang menyebabkan Yesus meneteskan keringat darah.

Jika kita berdoa dan melakukan perbuatan kasih di masa kini, kita
menemani dan menghibur Kristus pada saat Dia mengalami penderitaan di Taman
Getsemani. Kita mengikuti apa yang diperintahkan oleh Kristus sendiri, ketika
Dia mengatakan "Hati-Ku sangat sedih, seperti mau mati rasanya. Tinggallah di
sini dan berjaga-jagalah dengan Aku." (Mat 26:38). Jangan biarkan kita lengah
sehingga Kristus menegur kita dengan mengatakan "Tidakkah kamu sanggup berjaga-
jaga satu jam dengan Aku?" (Mat 26:40).

Bagaimana dengan pengetahuan manusia seperti kita? Kita dapat mempunyai
pengetahuan eksperimental atau kalau Tuhan menghendaki. Namun, menjadi
kodrat dari manusia untuk belajar secara bertahap. Pengetahuan manusia akan
Tuhan didapatkan secara bertahap. 

Dengan melihat kodrat manusia ini, Kristus berdoa "Ya Bapa, ampunilah mereka,
sebab mereka tidak tahu apa yang mereka perbuat." (lih. Luk 23:34). Kristus
tahu bahwa manusia memang berdosa karena dipengaruhi oleh kelemahan-
kelemahannya akibat dosa asal. Dengan demikian, apa yang diperbuat oleh manusia
bisa saja terjadi karena ketidaktahuannya. 

Namun tidak semua ketidaktahuan
mengakibatkan orang terbebas dari dosa. Ketidakketidaktahuan yang tak
terhindari (\textit{invincible ignorance}) membuat orang tidak berdosa, namun
ketidaktahuan yang disebabkan oleh ketidakpedulian orang itu sendiri (\textit{culpable
ignorance}) menyebabkan seseorang tetap bersalah. Rasul Petrus mengerti bahwa
orang-orang yang menyalibkan Yesus bertindak karena ketidaktahuan mereka,
sehingga dia mengatakan \textit{"Hai saudara-saudara, aku tahu bahwa kamu telah berbuat
demikian karena ketidaktahuan, sama seperti semua pemimpin kamu."} (Kis 3:17)

Bagaimana dengan kita yang telah menerima Kristus? Kita tidak mempunyai alasan
lagi bahwa kita tidak tahu. Oleh karena itu, tanggung jawab kita lebih berat,
karena barang siapa diberi banyak akan dituntut lebih banyak (lih. Luk 12:48).
Menyadari bahwa manusia dengan kekuatannya sendiri tidak dapat menjalankan
semua perintah Allah, Kristus menyediakan Diri-Nya sendiri untuk disalibkan,
sehingga rahmat yang berlimpah dapat mengalir kepada kita umat Allah. 

Bahkan
kesalahan-kesalahan yang dibuat umat Allah dapat dihapuskan dengan melakukan
pengakuan dosa. Dan kalau seseorang tidak mensyukuri dan menggunakan semua
kemudahan untuk mendapatkan pengampunan dosa, maka orang tersebut tidak lagi
mempunyai alasan apapun kalau sampai dia kehilangan keselamatan kekal.

\subsection{Luk 23:43 "Aku berkata kepadamu, sesungguhnya hari ini juga engkau
akan ada bersama-sama dengan Aku di dalam Firdaus."}

Seluruh
kehidupan-Nya ditujukan untuk mengemban misi ini, dan Kristus telah
melaksanakannya dengan sempurna. Bahkan sampai pada menjelang akhir wafat-Nya,
Dia tidak membuang kesempatan sedikitpun untuk menyelamatkan pencuri yang
disalibkan bersama-Nya.

Menjadi sesuatu yang umum, bahwa pada saat seseorang disalibkan, maka dia akan
menyumpahi orang yang menyalibkannya, bahwa menyumpahi dirinya, menyumpahi
Tuhan dan hari kelahirannya. 
Namun, dua pencuri yang disalibkan mendengarkan
seseorang yang disalib di tengah-tengah mereka mengatakan, "Ya Bapa, ampunilah
mereka, sebab mereka tidak tahu apa yang mereka perbuat." (Luk 23:34).

Pengampunan ini mendatangkan rahmat. Paling tidak salah satu dari pencuri ini
menyambut rahmat Allah. Bahkan ketika pencuri di sebelah kiri mengatakan
"\textit{Bukankah Engkau adalah Kristus? Selamatkanlah diri-Mu dan kami!"} (Luk 23:39),
maka pencuri di sebelah kanan Yesus menjawab 
\textit{"40 Tidakkah engkau takut, juga
tidak kepada Allah, sedang engkau menerima hukuman yang sama? 41  Kita memang
selayaknya dihukum, sebab kita menerima balasan yang setimpal dengan perbuatan
kita, tetapi orang ini tidak berbuat sesuatu yang salah."} (Luk 23:40-41)

Percakapan ini mungkin terlihat sepele. Namun, kita jangan melupakan bahwa
setiap kata yang keluar dari orang yang disalibkan adalah merupakan suatu
penderitaan, karena setiap tarikan nafas menjadi suatu siksaan. Pencuri di
sebelah kanan, yang menurut tradisi bernama Dimas, dalam keterbatasannya telah
memberikan nyawanya untuk Kristus, dan dia juga menaruh pengharapan di dalam
Kristus, sehingga dia memohon kepada Yesus \textit{"Yesus, ingatlah akan aku, apabila
Engkau datang sebagai Raja."} (Luk 23:42) Sungguh suatu ungkapan pengharapan dan
iman yang begitu sederhana dan dalam. 

Terhadap ungkapan iman dan kasih ini,
Yesus menjawab "Aku berkata kepadamu, sesungguhnya hari ini juga engkau akan
ada bersama-sama dengan Aku di dalam Firdaus." (Luk 23:43)

Mari, dalam Pekan Suci ini, kita bersama-sama merenungkan, bahwa kita yang
telah menerima baptisan sakramental, seharusnya mempunyai sikap seperti yang
ditunjukkan oleh Dimas, bahkan dituntut lebih. Mengapa? Karena kita telah
menerima rahmat Allah yang begitu istimewa dalam Sakramen Baptis, seperti: 
\begin{enumerate}[label=(\alph{enumi})]
\item rahmat pengudusan, 
\item menjadi anak-anak Allah dan dipersatukan dalam Tubuh
Mistik Kristus, 
\item menerima tiga kebajikan ilahi (iman, pengharapan dan
kasih), 
\item menerima tujuh karunia Roh Kudus seperti yang disebutkan di dalam
Yes 11:2-3 (kebijaksanaan, pengertian, nasihat, keperkasaan, pengenalan,
kesalehan, dan takut kepada Allah). 
\end{enumerate}

Dengan rahmat-rahmat ini kita dimampukan
untuk mengikuti perintah Kristus, yang menuntun kita kepada keselamatan kekal.

\subsection{Yoh 19:26-27 "Ibu, inilah, anakmu!" dan "Inilah ibumu!"}
Memandang dari kayu salib, Kristus melihat dua orang
yang dikasihi-Nya, yaitu Ibu-Nya, Bunda Maria dan murid-Nya yang terkasih,
rasul Yohanes. Dengan sisa-sisa nafas-Nya, Kristus memberikan pesan yang begitu
penting kepada kita, yaitu pesan ketika Kristus memandang Ibu-Nya dan murid-Nya
dan berkata "Ibu (RSV = \textit{Woman}), inilah, anakmu!.. dan inilah ibumu" (Yoh 19:26-
27). 

Kristus menyerahkan ibu-Nya kepada  kepada murid yang
dikasihi-Nya – tanpa nama, untuk menyatakan bahwa perintah ini ditujukan kepada
semua murid Kristus.
Sebaliknya Kristus juga menyerahkan murid-Nya untuk menjadi putera Bunda Maria.
Satu-satunya anak Maria memang tidak tergantikan, yaitu Kristus. Ini
berarti, Kristus menginginkan agar Bunda Maria turut berpartisipasi dalam karya
keselamatan Kristus dan memperlakukan seluruh umat beriman sebagai anaknya.

Suka atau tidak suka, Kristus menginginkan hal ini dan memberikan Maria sebagai
bunda bagi seluruh umat beriman. Kalau Kristus tidak berkeberatan untuk dididik
oleh Maria dan Maria dipandang baik oleh Kristus sebagai Bunda Allah, maka
siapakah kita yang memandang bahwa kita tidak perlu menghormati Bunda Maria,
bahkan ada yang menyingkirkan Bunda Maria dari kehidupannya? Apakah ada seorang
pria yang merasa bahwa pacarnya terlalu berlebihan karena dia menghormati
ibunya juga?

\subsection{Mrk 15:34 "Allahku, Allahku, mengapa Engkau meninggalkan Aku?"}
Di dalam penderitaan-Nya, Dia telah menunjukkan adanya suatu kepercayaan yang
kokoh akan rencana Allah. Perkataan Eli, Eli Lamasabakthani, merupakan
permulaan dari Mazmur 22, yang lengkapnya adalah sebagai berikut:
\begin{quote}
\scriptsize
\begin{enumerate}
     \item  Untuk pemimpin biduan. Menurut lagu: Rusa di kala fajar. Mazmur
     Daud. (22-2) Allahku, Allahku, mengapa Engkau meninggalkan aku? Aku
     berseru, tetapi Engkau tetap jauh dan tidak menolong aku.
     \item Allahku, aku berseru-seru pada waktu siang, tetapi Engkau tidak
     menjawab, dan pada waktu malam, tetapi tidak juga aku tenang.
     \item Padahal Engkaulah Yang Kudus yang bersemayam di atas puji-pujian
     orang Israel.
     \item Kepada-Mu nenek moyang kami percaya; mereka percaya, dan Engkau
     meluputkan mereka.
     \item Kepada-Mu mereka berseru-seru, dan mereka terluput; kepada-Mu
     mereka percaya, dan mereka tidak mendapat malu.
     \item Tetapi aku ini ulat dan bukan orang, cela bagi manusia, dihina oleh
     orang banyak.
     \item Semua yang melihat aku mengolok-olok aku, mereka mencibirkan
     bibirnya, menggelengkan kepalanya:
     \item "Ia menyerah kepada TUHAN; biarlah Dia yang meluputkannya, biarlah
     Dia yang melepaskannya! Bukankah Dia berkenan kepadanya?"
     \item Ya, Engkau yang mengeluarkan aku dari kandungan; Engkau yang
     membuat aku aman pada dada ibuku.
     \item Kepada-Mu aku diserahkan sejak aku lahir, sejak dalam kandungan
     ibuku Engkaulah Allahku.
     \item Janganlah jauh dari padaku, sebab kesusahan telah dekat, dan tidak
     ada yang menolong.
     \item Banyak lembu jantan mengerumuni aku; banteng-banteng dari Basan
     mengepung aku;
     \item mereka mengangakan mulutnya terhadap aku seperti singa yang
     menerkam dan mengaum.
     \item Seperti air aku tercurah, dan segala tulangku terlepas dari
     sendinya; hatiku menjadi seperti lilin, hancur luluh di dalam dadaku;
     \item kekuatanku kering seperti beling, lidahku melekat pada langit-
     langit mulutku; dan dalam debu maut Kauletakkan aku.
     \item Sebab anjing-anjing mengerumuni aku, gerombolan penjahat mengepung
     aku, mereka menusuk tangan dan kakiku.
     \item Segala tulangku dapat kuhitung; mereka menonton, mereka memandangi
     aku.
     \item Mereka membagi-bagi pakaianku di antara mereka, dan mereka
     membuang undi atas jubahku.
     \item Tetapi Engkau, TUHAN, janganlah jauh; ya kekuatanku, segeralah
     menolong aku!
     \item Lepaskanlah aku dari pedang, dan nyawaku dari cengkeraman anjing.
     \item Selamatkanlah aku dari mulut singa, dan dari tanduk banteng.
     Engkau telah menjawab aku!
     \item Aku akan memasyhurkan nama-Mu kepada saudara-saudaraku dan memuji-
     muji Engkau di tengah-tengah jemaah:
     \item kamu yang takut akan TUHAN, pujilah Dia, hai segenap anak cucu
     Yakub, muliakanlah Dia, dan gentarlah terhadap Dia, hai segenap anak
     cucu Israel!
     \item Sebab Ia tidak memandang hina ataupun merasa jijik kesengsaraan
     orang yang tertindas, dan Ia tidak menyembunyikan wajah-Nya kepada
     orang itu, dan Ia mendengar ketika orang itu berteriak minta tolong
     kepada-Nya.
     \item Karena Engkau aku memuji-muji dalam jemaah yang besar; nazarku
     akan kubayar di depan mereka yang takut akan Dia.
     \item Orang yang rendah hati akan makan dan kenyang, orang yang mencari
     TUHAN akan memuji-muji Dia; biarlah hatimu hidup untuk selamanya!
     \item Segala ujung bumi akan mengingatnya dan berbalik kepada TUHAN; dan
     segala kaum dari bangsa-bangsa akan sujud menyembah di hadapan-Nya.
     \item Sebab Tuhanlah yang empunya kerajaan, Dialah yang memerintah atas
     bangsa-bangsa.
     \item Ya, kepada-Nya akan sujud menyembah semua orang sombong di bumi,
     di hadapan-Nya akan berlutut semua orang yang turun ke dalam debu,
     dan orang yang tidak dapat menyambung hidup.
     \item Anak-anak cucu akan beribadah kepada-Nya, dan akan menceritakan
     tentang TUHAN kepada angkatan yang akan datang.
     \item Mereka akan memberitakan keadilan-Nya kepada bangsa yang akan
     lahir nanti, sebab Ia telah melakukannya.
\end{enumerate}
\end{quote}

Bagi umat Yahudi, kalau seseorang memulai kalimat pertama dari Mazmur, maka
berarti orang bermaksud untuk menyelesaikannya. Dan dalam kondisi tersalib,
sungguh tidak mungkin untuk menyelesaikan pengucapan keseluruhan Mazmur
tersebut. Ini berarti, bahwa kalimat pertama dari Mazmur 22 harus dimengerti
dalam konteks keseluruhan, yaitu untuk mempercayai dan menggantungkan segala
sesuatunya ke dalam tangan Bapa, yang pada akhirnya akan membawa kemuliaan, di
mana seluruh ujung bumi akan mengingat dan berbalik kepada Tuhan (lih. Mzm 22:
27). 

Ini adalah suatu pengajaran dari Kristus yang harus diikuti oleh seluruh
murid Kristus tentang bagaimana menaruh pengharapan di dalam Tuhan dalam
kondisi apapun. Cara dan sikap dalam menghadapi penderitaan adalah salah satu
perbedaan antara orang yang mengenal Kristus dan yang tidak mengenal Kristus.

Bahkan rasul Paulus mengatakan (Rom 5:3-5)
\begin{quote} 
\scriptsize
\begin{enumerate}
\setcounter{enumi}{2}

\item Dan bukan hanya itu saja. Kita malah bermegah
juga dalam kesengsaraan kita, karena kita tahu, bahwa kesengsaraan itu
menimbulkan ketekunan, \item  dan ketekunan menimbulkan tahan uji dan tahan uji
menimbulkan pengharapan. \item  Dan pengharapan tidak mengecewakan, karena kasih
Allah telah dicurahkan di dalam hati kita oleh Roh Kudus yang telah
dikaruniakan kepada kita." 
\end{enumerate}
\end{quote}

Kalau seseorang menjadi murid Kristus, maka dia akan mengikuti apa yang
dilakukan oleh Kristus, termasuk adalah cara menghadapi permasalahan dan
penderitaan. Karena dengan penderitaan-Nya, Kristus dapat memenangkan belenggu
dosa, maka dengan menyatukan segala penderitaan kita dengan Kristus, kita akan
memperoleh kemenangan, yaitu kemenangan yang menyelamatkan, yang mengantar kita
pada  kehidupan kekal. Kuncinya adalah menghadapi permasalahan dengan terus
bertekun dalam doa yang didasarkan iman, pengharapan dan kasih, seperti yang
dilakukan oleh Kristus.

Mungkin ada yang bertanya, kalau Yesus memang Tuhan, mengapa pada saat disalib,
Dia berdoa? Santo Thomas Aquinas
membahas tentang definisi doa, dimana dia mengatakan bahwa doa adalah membuka
keinginan kita kepada Tuhan, sehingga Dia dapat memenuhinya." Karena di
dalam Kristus (satu pribadi) ada dua kehendak, yaitu kehendak manusia dan
kehendak Tuhan, maka menjadi hal yang wajar, kalau Yesus berdoa karena Dia
mempunyai kodrat manusia. Sama seperti kita sebagai orang beriman, kita
menyatakan keinginan/ kehendak kita di hadapan Allah.

Alasan kedua adalah Yesus berdoa untuk kepentingan manusia. Yesus dapat saja
berdoa dalam hati, namun Dia ingin menunjukkan kepada kita bagaimana seharusnya
sebagai manusia kita berdoa, yaitu bahwa kita harus senantiasa tunduk kepada
kehendak Allah Bapa, meskipun di dalam situasi yang paling sulit sekalipun.

Yesus mengajarkan doa yang sempurna, yaitu doa Bapa Kami, yang terdiri dari
tujuh petisi (lih. Mt 6:9-13).
Yesus menunjukkan bahwa di dalam setiap percobaan, maka Tuhanlah yang menjadi
kekuatan dalam doa, seperti yang ditunjukkan oleh Yesus di dalam drama
penyaliban (Mt 27:46; Mk 15:34; Lk 23:46).
Yesus juga mengajarkan pentingnya untuk mengampuni orang yang bersalah kepada
kita, seperti yang ditunjukkan oleh Yesus dengan berdoa "Ya Bapa, ampunilah
mereka, sebab mereka tidak tahu apa yang mereka perbuat." (lih. Lk 23:34).
Dan masih begitu banyak contoh yang lain, yang menyebabkan pengikut Kristus
tahu bagaimana untuk berdoa, karena Tuhan sendiri – melalui Kristus – yang
menunjukkan kepada manusia bagaimana seharusnya berdoa.

Dengan demikian, maka kita dapat melihat bahwa doa Yesus di atas kayu salib
sungguh merupakan doa yang berpengharapan yang menyelamatkan dan memberikan
contoh bagi seluruh umat beriman.

\subsection{Yoh 19:28 "Aku haus!"}
Contoh apalagi yang ingin diberikan oleh Kristus sebelum dia menghembuskan
nafas-Nya yang terakhir ketika Dia mengatakan "Aku haus!"? Dikatakan di ayat
Yoh 19:28 bahwa perkataan Yesus "Aku Haus" adalah untuk memenuhi nubuat di
dalam Kitab Suci. Ini adalah pemenuhan dari Mzm 69:21 yang mengatakan "… dan
pada waktu aku haus, mereka memberi aku minum anggur asam." Dengan demikian,
pernyataan Yesus merupakan penegasan bahwa Yesus yang tersaliblah yang
dinubuatkan dalam Perjanjian Lama.

Memang dalam kodrat-Nya sebagai manusia, Yesus mengalami penderitaan dan
kehausan yang begitu sangat. Namun, kehausan dalam kapasitas yang lebih dalam
adalah kehausan untuk meyelamatkan jiwa-jiwa. Ini adalah drama pencarian Tuhan
akan manusia. Drama di mana Tuhan yang dari Sorga turun ke dunia untuk
menjangkau jiwa-jiwa yang tercerai berai. 

Kehausan ini mengingatkan kita akan
permintaan Yesus kepada wanita Samaria "Berilah Aku minum" (Yoh 4:7). Dan
percakapan ini pada akhirnya membawa keselamatan kepada wanita Samaria dan juga
orang-orang di kota tersebut. Keselamatan wanita Samaria dan orang-orang di
kota tersebut tidaklah cukup bagi Yesus, sehingga di atas kayu salib, Dia tetap
merasa kehausan, karena Dia ingin menjangkau seluruh umat manusia, ingin
menemukan dan mengantar seluruh umat manusia pada keselamatan dan pengetahuan
akan kebenaran (lih. 1Tim 2:4)

Karena Tuhan senantiasa dalam pencarian akan manusia, maka sejak dari
Perjanjian Lama dikatakan "13 apabila kamu mencari Aku, kamu akan menemukan
Aku; apabila kamu menanyakan Aku dengan segenap hati, 14  Aku akan memberi kamu
menemukan Aku" (Yer 29:13-14) Inilah sebabnya ketika seseorang menyadari bahwa
dia memerlukan Tuhan, ketika seseorang melihat penderitaan dalam kacamata iman,
ketika seseorang menerima penderitaan dengan tabah, ketika seseorang mau
menyangkal dirinya dan memikul salibnya dan mengikuti Kristus, maka Tuhanlah
yang sebenarnya menjadi penggerak utama dari semuanya itu. Dalam drama
penyaliban, terutama perkataan Yesus bahwa Dia haus, kita menyaksikan akan
drama tentang Tuhan yang sungguh mencintai manusia dengan sehabis-habisnya.
Bagaimana tanggapan manusia? Bagaimana tanggapan kita?

\subsection{Yoh 19:30 "Sudah selesai"}
Setelah prajurit memberikan bunga karang yang telah dicelupkan pada anggur
asam, lalu Yesus meminumnya dan berkata "sudah selesai" (lih. Yoh 19:30). Kita
dapat melihat adanya tiga hal yang berkaitan dengan "sudah selesai". Di dalam
Kitab Kejadian, setelah Tuhan menyelesaikan penciptaan, maka pada hari ke
tujuh, Dia mengatakan "Ketika Allah pada hari ketujuh telah menyelesaikan
pekerjaan yang dibuat-Nya itu, berhentilah Ia pada hari
ketujuh dari segala pekerjaan yang telah dibuat-Nya itu." (Kej 2:2) 

Dalam konteks inilah Yesus mengatakan "sudah selesai" untuk menyatakan
bahwa Dia telah menyelesaikan pekerjaan yang diberikan oleh Bapa dengan
sempurna, bukan dengan keputusasaan dan kegetiran, namun dengan dasar kasih
yang sempurna. Inilah yang membuat persembahan Kristus di kayu salib dapat
menyenangkan hati Bapa -- yaitu karena didasarkan kasih yang sempurna.
Ini juga yang seharusnya mendorong kita dalam perjalanan kehidupan kita. Sama
seperti Rasul Paulus, kita juga ingin berlari ke tujuan untuk memperoleh
hadiah, yaitu panggilan Sorgawi dari Allah dalam Kristus Yesus (lih. Flp 3:14).

\subsection{Luk 23:46 "Ya Bapa, ke dalam tangan-Mu Kuserahkan nyawa-Ku."}
Kata yang terakhir dari Yesus setelah mengatakan "sudah selesai" adalah "Ya
Bapa, ke dalam tangan-Mu Kuserahkan nyawa-Ku". Dalam satu kalimat ini, kita
dapat melihat hubungan yang sungguh dalam dan tak terpisahkan antara Bapa dan
Putera. Bapa begitu mencintai manusia, sehingga Dia mengutus Putera-Nya yang
tunggal untuk menebus dosa dan menyelamatkan manusia (lih. Yoh 3:16). Kristus
datang ke dunia dan senantiasa melaksanakan kehendak Bapa. Dari umur duabelas
tahun, Kristus telah mengatakan bahwa Dia harus berada di dalam rumah Bapa-Nya
(Luk 2:49). Dalam seluruh karya-Nya, Kristus senantiasa melakukan apa yang
berkenan kepada Bapa (lih. Yoh 8:29). Sampai pada akhirnya, Kristus menyerahkan
nyawaNya ke dalam tangan Bapa (lih. Luk 23:46). Dengan kebebasan-Nya, Kristus
melakukan kehendak Bapa.

Bagaimana dengan kita? Bagaimana kita menggunakan kebebasan kita? Orang sering
salah dalam mengartikan kebebasan. Orang sering mengartikan kebebasan sebagai
"kebebasan dari" dan bukan "kebebasan untuk".
Kebebasan yang lebih menekankan "kebebasan dari" merupakan ekspresi akan
keinginan yang terbebas dari hal-hal yang dianggap mengikatnya, termasuk
tanggung jawab. Orang yang menginginkan kebebasan untuk minum minuman keras
tanpa mau dibatasi jumlahnya, cepat atau lambat akan menemukan bahwa dirinya
tidak lagi bebas. Dia akan terikat akan minuman keras, dan tidak lagi mempunyai
kebebasan untuk mengatakan tidak terhadap minuman keras. Dengan demikian, kita
dapat melihat bahwa mengumbar kebebasan tanpa adanya batasan yang jelas dapat
membuat manusia menjadi tidak bebas lagi. 

Mari, dalam Pekan Suci ini, kita merenungkan sejauh mana kita telah menggunakan
kebebasan kita. Apakah kita telah menggunakan kebebasan kita dengan
bertanggungjawab berdasarkan kebenaran dan kebaikan, sehingga dapat mengarahkan
kita kepada keselamatan diri kita maupun membantu keselamatan orang-orang di
sekitar kita? Jika kita telah mati dari dosa kita -- karena Sakramen Baptis --
yang kita terima, dan membuat kita dapat bangkit bersama Kristus, maka kita
juga harus mengikuti teladan Kristus. Kita dapat menyerahkan kebebasan kita
kepada Tuhan sehingga kita dapat semakin bebas untuk melaksanakan seluruh
perintah Tuhan.

\section{Melaksanakan tujuh pesan terakhir Yesus mengantar kita kepada keselamatan}

Dari pemaparan di atas, kita dapat melihat bahwa tujuh pesan terakhir Yesus
sungguh penuh makna yang mendalam. Kalau kita terus merenungkan pesan-pesan ini
sepanjang Pekan suci ini, maka kita akan semakin menghargai pengorbanan Yesus.
Apapun kondisi kita, di Pekan suci ini, Kristus menawarkan pengampunan kepada
kita semua. Bagi yang berdosa berat, segeralah mengaku dosa dan bagi yang
berjuang dalam kekudusan, teruslah berfokus pada tujuan akhir. Yesus
menginginkan agar semua manusia dapat sampai pada tujuan akhir, yaitu Sorga.
Tidak ada kata terlambat. Sejauh kita masih hidup dan bertobat, sama seperti
pencuri yang disalibkan di sisi kanan Yesus, maka Kristus akan memberikan janji
yang sama, yaitu keselamatan kekal.
Demikian pula, Kristus menyerahkan Bunda-Nya menjadi Bunda segenap umat
beriman, agar kita dapat memohon dukungan doanya agar dapat sampai kepada
keselamatan. Tujuan akhir ini juga harus dihadapi dengan pengharapan akan
Allah, sehingga pencobaan dan penderitaan tidak menjadikan kita perputus asa.
Dalam perjalanan kita menuju Sorga, kita juga harus mempunyai semangat untuk
membawa orang-orang di sekitar kita untuk memperoleh pengetahuan akan
kebenaran. Dan ini harus kita lakukan sampai akhir hidup kita, sampai tugas
kita selesai dan sampai kita menyerahkan nyawa kita ke dalam tangan Bapa.
Dengan menjalankan pesan Kristus ini, maka kita dapat mencapai tujuan akhir
dengan selamat.
Semoga Trihari Suci membawa kita pada permenungan yang lebih mendalam akan
misteri Paskah Kristus.

\sumber{Ditulis oleh: Stefanus Tay\\
Stefanus Tay telah menyelesaikan program studi S2 di bidang teologi di Universitas Ave Maria - Institute for Pastoral Theology, Amerika Serikat.}
