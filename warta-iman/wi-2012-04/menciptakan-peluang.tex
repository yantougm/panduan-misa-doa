\chap{Menciptakan Peluang}
\small
\noindent{ Dari Kejadian 45 : 5}

\begin{wrapfigure}[7]{r}{3.5cm}
\includegraphics[scale=0.5]{gambar/like.png}
\end{wrapfigure} 
\begin{quote}
\textit{``tetapi sekarang, janganlah bersusah hati dan janganlah menyesali diri, karena kamu menjual aku kesini, sebab untuk memelihara kehidupanlah Allah menyuruh aku mendahului kamu,''}
\end{quote}

Menurut Diana Kirschner, Ph.D., psikolog dan ahli komunikasi, rasa cemburu adalah suatu bentuk pemikiran negatif yang datang dari dalam diri sendiri. Melalui penelitian, terbukti bahwa rasa cemburu bisa berujung pada sakit hati, curiga, ledakan amarah, bahkan kemunduran kualitas dalam berhubungan. Tapi, mungkinkah rasa cemburu diubah menjadi sesuatu yang positif? Inilah salah satu caranya.

\textbf{Ketika rasa cemburu mulai merayapi pikiran,} sadari bahwa hal itu merupakan tanda betapa Anda sangat menyayangi pasangan. Daripada melelahkan diri dengan pikiran-pikiran negatif, cobalah untuk mendekati pasangan Anda dan katakan betapa sayangnya Anda kepadanya.

Di kehidupan ini kadang kita tidak tidak bisa memilih. \textbf{Suka atau tidak, kita harus belajar untuk menerima segala sesuatu apa adanya.} Kenyataan di depan kita adalah fakta tak terbantahkan dan kita tidak memiliki alternatif lain. \textbf{Jika hal seperti ini terjadi, bagaimana tindakan selanjutnya? Apakah kita harus arah dengan situasi yang ada?} Marilah kita belajar dari kehidupan tokoh alkitab perjanjian lama yang bernama Yusuf.

Yusuf adalah pemuda yang harus kehilangan kemerdekaan dan harkat sebagai orang merdeka karena dijual para saudaranya yang iri kepadanya. Berbagai pengalaman berat setelah itu pun harus ia alami. Namun, Yusuf tidak sudi menyerah. Dengan pertolongan Allah, ia berhasil mengubah semua rintangan di jalan kehidupannya sebagai kesempatan. Dalam beberapa tahun, ia akhirnya diangkat oleh Firaun sebagai penguasa kedua di Mesir.

Belajar dari kehidupan Yusuf di atas, \textbf{marilah kita menjawab tantangan sepanjang hari ini}. Ubahlah paradigma tentang kekuatan terhadap tantangan menjadi kesempatan untuk meraih keberhasilan. \textbf{Allah yang ada di dalam diri kita sanggup melakukan segala perkara untuk mendatangkan kebaikan bagi kita.}

\sumber{kiriman dari Aditya Bimantara\\ dari ``Menciptakan Peluang.'' \\Renungan Harian Kita April 2011\\  
www.renungan-harian-kita.blogspot.com.} 
\normalsize