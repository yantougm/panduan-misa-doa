\newpage
\chap{Kompendium Katekese Gereja Katolik}
\setcounter{kgkcounter}{33}
\normalsize
\kgk{Apa simbol-simbol iman yang paling kuno itu?}
      Simbol-simbol iman yang paling kuno ialah pengakuan iman pembaptisan karena diberikan ``atas nama Bapa, dan Putra, dan Roh Kudus'' (Mat 28:19),
pengakuan kebenaran-kebenaran iman dalam Sakramen Pembaptisan diformu-
lasikan mengacu pada tiga Pribadi Tritunggal.

\kgk{Simbol-simbol iman apa yang paling penting?}
     Yang paling penting adalah Syahadat Para Rasul yang merupakan simbol pembaptisan kuno dari Gereja Roma dan Syahadat Nicea-Konstantinopel
yang merupakan hasil dari dua Konsili ekumenis, yaitu Nicea (325 M) dan
Konstantinopel (381 M); bahkan sampai sekarang, syahadat ini umum digunakan
oleh semua Gereja besar di Timur dan Barat.

\begin{center}
         \textbf{``AKU PERCAYA AKAN ALLAH BAPA YANG MAHAKUASA,\\
                   PENCIPTA LANGIT DAN BUMI''}
\end{center}

\kgk{Mengapa Pengakuan Iman mulai dengan kata-kata ``Aku percaya akan Allah''?}
     Pengakuan Iman mulai dengan kata-kata ini karena pernyataan ``Aku percaya akan Allah'' adalah hal yang paling penting, sumber dari semua kebenaran yang lain tentang manusia dan dunia, serta tentang seluruh kehidupan orang yang percaya kepada Allah.

\kgk{Mengapa orang mengaku percaya hanya kepada satu Allah?}
Kepercayaan akan satu Allah ini diakui karena Dia sudah mewahyukan
Diri-Nya kepada bangsa Israel sebagai Yang Satu ketika bersabda: ``Dengarlah, hai Israel: Allah itu Allah kita, Allah itu esa'' (Ul 6:4) dan ``tidak ada yang lain'' (Yes 45:22). Yesus sendiri meneguhkan bahwa ``Allah kita itu esa'' (Mrk 12:29).

          Pengakuan bahwa Yesus dan Roh Kudus adalah juga Allah dan Tuhan tidak membawa perpecahan di dalam Allah yang esa.

\flushright{(\dots \emph{bersambung} \dots)}
