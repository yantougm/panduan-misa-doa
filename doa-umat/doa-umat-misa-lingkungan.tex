\documentclass[12pt]{scrartcl}
\usepackage[a4paper,top=2cm,right=2cm,left=2cm,bottom=2cm]{geometry}
\usepackage{graphicx}
\usepackage{palatino}
\usepackage{microtype}

\makeatletter
\newcommand{\judul}[1]{%
  {\parindent \z@ \centering \normalfont
    \interlinepenalty\@M \Large \bfseries #1\par\nobreak \vskip 20\p@ }}
\newcommand{\subjudul}[1]{%
  {\parindent \z@ \normalfont
    \interlinepenalty\@M \bfseries #1\par\nobreak \vskip 20\p@ }}
\newcommand{\lagu}[1]{%
  {\parindent \z@ \normalfont
    \interlinepenalty\@M \bfseries \emph{#1}\par\nobreak \vskip 20\p@ }}

\renewenvironment{description}
               {\list{}{\labelwidth\z@ \itemindent-\leftmargin
                        \let\makelabel\descriptionlabel}}
               {\endlist}
\renewcommand*\descriptionlabel[1]{\hspace\labelsep 
                                \normalfont\bfseries #1 }
    

\makeatother

\newcommand{\BU}[1]{\begin{itemize} \item[U:] #1 \end{itemize}}
\newcommand{\BI}[1]{\begin{itemize} \item[I:] #1 \end{itemize}}
\newcommand{\BP}[1]{\begin{itemize} \item[P:] #1 \end{itemize}}
% \newcommand{\BL}[1]{\begin{itemize} \item[Wawan:] #1 \end{itemize}}
% \newcommand{\BW}[1]{\begin{itemize} \item[Novi:] #1 \end{itemize}}
% \newcommand{\BMP}[1]{\begin{itemize} \item[W+N:] #1 \end{itemize}}
% \newcommand{\BS}[1]{\begin{itemize} \item[Saksi:] #1 \end{itemize}}
\newcommand{\ultah}{25 }
\newcommand{\romo}{Anonius Dadang Hermawan Pr.}
\hyphenation{ba-gi-mu}
\hyphenation{di-se-rah-kan}
\hyphenation{me-la-lui}
\hyphenation{ka-nak}
\hyphenation{ka-re-na}
\hyphenation{ber-ka-ta}
\hyphenation{te-ta-pi}
\hyphenation{per-ka-win-an}
\hyphenation{pa-tut}
\hyphenation{me-lu-hur-kan}
\hyphenation{ber-nya-nyi}
\hyphenation{di-tum-pah-kan}
\hyphenation{pe-ngam-pun-an}
\hyphenation{ber-a-da}
\hyphenation{kau-lim-pah-kan}
\hyphenation{ke-bang-kit-an-Nya ke-la-ku-an-nya}
\hyphenation{men-da-tang-kan}
\hyphenation{me-nya-ta-kan}
\hyphenation{per-ka-ta-an}
\hyphenation{pa-sang-kan-lah}
\hyphenation{DA-RAH-KU}
\hyphenation{ke-na-ik-kan-nya}
\hyphenation{per-sem-bah-an}
\hyphenation{per-se-ku-tu-an}

\usepackage[bahasa]{babel}
\selectlanguage{bahasa}

\topmargin=-0.5in
\textheight=8in
\title{MISA \\LINGKUNGAN ST. THERESIA}
\author{oleh Romo \romo} 
\date{11 September 2019}
\begin{document}

\maketitle
\Large
\thispagestyle{empty}

\judul{LITURGI SABDA}

\subjudul{Bacaan pertama: Kolose 3:1-11}

\BP{\emph{Pembacaan dari Surat Paulus kepada Umat di Kolose.}

Karena itu, kalau kamu dibangkitkan bersama dengan Kristus, carilah perkara yang di atas, di mana Kristus ada, duduk di sebelah kanan Allah.
Pikirkanlah perkara yang di atas, bukan yang di bumi.
Sebab kamu telah mati dan hidupmu tersembunyi bersama dengan Kristus di dalam Allah.

Apabila Kristus, yang adalah hidup kita, menyatakan diri kelak, kamupun akan menyatakan diri bersama dengan Dia dalam kemuliaan.
Karena itu matikanlah dalam dirimu segala sesuatu yang duniawi, yaitu percabulan, kenajisan, hawa nafsu, nafsu jahat dan juga keserakahan, yang sama dengan penyembahan berhala,
semuanya itu mendatangkan murka Allah (atas orang-orang durhaka).
Dahulu kamu juga melakukan hal-hal itu ketika kamu hidup di dalamnya.
Tetapi sekarang, buanglah semuanya ini, yaitu marah, geram, kejahatan, fitnah dan kata-kata kotor yang keluar dari mulutmu.

Jangan lagi kamu saling mendustai, karena kamu telah menanggalkan manusia lama serta kelakuannya,
dan telah mengenakan manusia baru yang terus-menerus diperbaharui untuk memperoleh pengetahuan yang benar menurut gambar Khaliknya;
dalam hal ini tiada lagi orang Yunani atau orang Yahudi, orang bersunat atau orang tak bersunat, orang Barbar atau orang Skit, budak atau orang merdeka, tetapi Kristus adalah semua dan di dalam segala sesuatu.

Demikianlah Sabda Tuhan}
\BU{Syukur kepada Allah.}

\subjudul{Mazmur:  145:2-3,10-11,12-13ab}

\BP{\emph{Tuhan itu baik kepada semua orang}}

\begin{itemize}
	\item 
Setiap hari aku hendak memuji Engkau, dan hendak memuliakan nama-Mu untuk seterusnya dan selamanya.
Besarlah TUHAN dan sangat terpuji, dan kebesaran-Nya tidak terduga.

\item
Segala yang Kaujadikan itu akan bersyukur kepada-Mu, ya TUHAN, dan orang-orang yang Kaukasihi akan memuji Engkau.
Mereka akan mengumumkan kemuliaan kerajaan-Mu, dan akan membicarakan keperkasaan-Mu,

\item 
untuk memberitahukan keperkasaan-Mu kepada anak-anak manusia, dan kemuliaan semarak kerajaan-Mu.
Kerajaan-Mu ialah kerajaan segala abad, dan peme-rintahan-Mu tetap melalui segala keturunan. TUHAN setia dalam segala perkataan-Nya dan penuh kasih setia dalam segala perbuatan-Nya.
\end{itemize}	

\subjudul{Bacaan Injil: Lukas 6: 20 - 26}

\BI{Tuhan sertamu}

\BU{Dan sertamu juga}

\BI{Inilah Injil Yesus Kristus menurut Santo Lukas}

\BU{Dimuliakanlah Tuhan}

\BI{Pada waktu itu Yesus memandang murid-murid-Nya lalu berkata: "Berbahagialah, hai kamu yang miskin, karena kamulah yang empunya Kerajaan Allah.
	Berbahagialah, hai kamu yang sekarang ini lapar, karena kamu akan dipuaskan. Berbahagialah, hai kamu yang sekarang ini menangis, karena kamu akan tertawa.
	Berbahagialah kamu, jika karena Anak Manusia orang membenci kamu, dan jika mereka mengucilkan kamu, dan mencela kamu serta menolak namamu sebagai sesuatu yang jahat.
	Bersukacitalah pada waktu itu dan bergembiralah, sebab sesungguhnya, upahmu besar di sorga; karena secara demikian juga nenek moyang mereka telah memperlakukan para nabi.
	
	Tetapi celakalah kamu, hai kamu yang kaya, karena dalam kekayaanmu kamu telah memperoleh penghiburanmu.
	Celakalah kamu, yang sekarang ini kenyang, karena kamu akan lapar. Celakalah kamu, yang sekarang ini tertawa, karena kamu akan berdukacita dan menangis.
	Celakalah kamu, jika semua orang memuji kamu; karena secara demikian juga nenek moyang mereka telah memperlakukan nabi-nabi palsu."
	
Berbahagialah orang yang mendengarkan sabda Tu-han, dan tekun melaksanakannya.}

\BU{Sabda-Mu adalah jalan, kebenaran dan hidup kami.}

\subjudul{Homili}

\subjudul{Aku Percaya}

\subjudul{Doa Umat}
\BI{Saudara-saudari marilah kita sehati dan sejiwa berdoa kepada Allah, Bapa Tuhan kita:}
\BP{Bagi Gereja kita: 
	
	Semoga Bapa sumber cahaya menerangi Gereja agar mampu membenahi diri dengan keadilan yang mendasari cinta kasih Kristiani. Marilah kita  mohon, }

\BU{Kabulkanlah doa kami ya Tuhan.}

\BP{Bagi mereka yang berhak menghakimi di dunia ini: Semoga Allah Bapa mendampingi mereka yang berhak mengadili agar janganlah mereka mengadili berdasarkan apa yang tampak atau desas-desus saja melainkan secara adil dan benar. Kami mohon,}

\BU{Kabulkanlah doa kami ya Tuhan.}

\BP{Bagi mereka yang menderita dan tertindas: Semoga Allah Bapa memberkati dan melindungi mereka yang tak mendapat dukungan organisasi ataupun masyarakat luas, agar dapat hidup wajar dan tahu bahwa mereka masih diperhitungkan. Marilah kita mohon \dots\dots\dots,....}

\BU{Kabulkanlah doa kami ya Tuhan.}

\BP{Bagi lingkungan St. Theresia: Ya Bapa, teristimewa untuk lingkungan ini, kiranya Kasih dan karunia-Mu senantiasa menyertai lingkungan ini, dalam setiap usaha yang umat lakukan, sehingga umat-Mu juga semakin rajin dan tekun bersyukur atas karunia-Mu itu, Marilah kita mohon \dots\dots\dots}

\BU{Kabulkanlah doa kami ya Tuhan.}

\BP{Bagi kita yang berkumpul di ruangan ini: Semoga Allah Bapa yang Mahapengasih memberkati kita agar dengan rajin dan tekun, bijaksana dan berani, ikut serta membangun Kerajaan Allah di dunia. Marilah kita mohon \dots\dots\dots}

\BU{Kabulkanlah doa kami ya Tuhan.}

\BP{Bagi Ibu Kusdaryati yang sedang menderita sakit: Semoga Allah Bapa yang Mahapengasih memberkati Ibu Kus agar sembuh dari sakitnya dan dapat beraktivitas kembali bersama keluarga dan lingkungan. Marilah kita mohon \dots\dots\dots}

\BU{Kabulkanlah doa kami ya Tuhan.}

\BI{Demikianlah ya Bapa permohonan-permohonan kami ini, masih banyak lagi doa-doa yang pada kesempatan ini tidak sempat kami sampaikan, namun kami yakin dan percaya Engkau mengetahui isi hati kami, kiranya yang baik dihadapan-Mu Engkau relakan bagi kami. Demi Kristus, Tuhan dan  pengantaraan kami.}

\BU{Amin}



\end{document}