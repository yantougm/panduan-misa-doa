% Inbuilt themes in beamer
\documentclass[aspectratio=169]{beamer}
\usepackage{tikz}
\usepackage[bahasa]{babel}

% Theme choice:
\usetheme{Boadilla}
\usecolortheme{dolphin}

\setbeamertemplate{background} 
{
\tikz\node[opacity=0.3] {\includegraphics[height=\paperheight,width=\paperwidth]{background.jpg}};
}

% Title page details: 
\title[BKL: Partisipasi]{Bulan Katekese Liturgi 2022\\\textit{\textbf{\Large Partisipasi}}} 
\author{Lingkungan St. Theresia}
\date{\today}
\logo{\large BKL 2022}


\begin{document}

% Title page frame
\begin{frame}
    \titlepage 
\end{frame}

% Remove logo from the next slides
\logo{}

\begin{frame}[fragile]
\frametitle{Pembuka}
\begin{itemize}[<+->]
    \item Lagu pembuka
    \item Tanda salib
	\item Salam 
    \item Doa tobat
\end{itemize}

\begin{block}{Doa pembuka}
Tuhan Yesus
Engkau nyatakan kuasa-Mu dalam hidup kami setiap hari
Karena Engkau menghendaki supaya hidup kami jadi saksi belas kasih dan kuasa-Mu
Ampunilah kami jika Kau dapati kami hanya mencari diri sendiri
Kami mohon kuasailah hati dan pikiran kami hanya untuk rencana dan kehendak-Mu

Sebab Engkaulah Tuhan, pengantara kami,
Yang bersama dengan Bapa,
Dalam persatuan Roh Kudus,
Hidup dan berkuasa,
Allah, kini dan sepanjang masa. Amin.
\end{block}
\end{frame}


% Outline frame
\begin{frame}{Partisipasi}
    %\tableofcontents
\begin{enumerate}[<+->]
    \setcounter{enumi}{6}
	\item Umat beriman terlibat dalam liturgi
    \item Panggilan
    \item Partisipasi dalam perayaan liturgi
\end{enumerate}
\end{frame}


% Lists frame
\section{Umat beriman terlibat dalam liturgi}
\begin{frame}{Umat beriman terlibat dalam liturgi}

\begin{itemize}[<+->]
	\item hak dan kewajiban
    
    {~}
\end{itemize}

\begin{columns}
\begin{column}{0.5\textwidth}
petugas liturgi
\begin{itemize}[<+->]
	\item lektor
    \item misdinar
    \item pemazmur
    \item prodiakon
    \item organis
    \item imam
\end{itemize}
\end{column}
\begin{column}{0.5\textwidth}
penunjang liturgi dan peribadatan
\begin{itemize}[<+->]
	\item keamanan
    \item parkir
    \item tata laksana
    \item among tamu
    \item prokes
\end{itemize}
\end{column}
\end{columns}

\begin{itemize}[<+->]
	\item 
\end{itemize}


\end{frame}


% Blocks frame
\section{Panggilan}
\begin{frame}{Panggilan}
    \begin{block}{Panggilan berkeluarga}
        Hidup berkeluarga merupakan panggilan juga.
    \end{block}
    \pause  
    \begin{exampleblock}{Pastor (Imam)}
    Imam adalah seorang laki-laki yang menyerahkan seluruh hidupnya untuk tugas pengabdian kepada Tuhan Allah dan Gereja. Seorang imam memiliki martabat sebagai pemimpin umat berkat tahbisan yang diterima.
    \end{exampleblock}

    \begin{exampleblock}{Suster (Biarawati)}
    Biarawati adalah seorang perempuan yang secara sukarela meninggalkan kehidupan duniawi dan memfokuskan hidupnya untuk kehidupan agama di suatu biara atau tempat ibadah dan tergabung dalam suatu tarekat atau ordo religius.  Para suster biasanya bekerja di bidang pendidikan, kesehatan, dan pelayanan sosial. 

    \end{exampleblock}
\end{frame} 

\begin{frame}{Panggilan (lanjutan)}
    \begin{exampleblock}{Burder (Biarawan)}
    Biarawan adalah seorang laki-laki yang melakukan asketisme, memfokuskan pikiran dan raganya untuk mengabdi sepenuhnya dalamm rangka mengikuti panggilan Tuhan dan suatu ordo atau tarekat religus seperti Yesuit, Dominikan, Fransiskan, Benediktin dan sebagainya. 
    \end{exampleblock}

    \begin{block}{Kaul}
\begin{itemize}[<+->]
	\item \textbf{Kaul kemiskinan}
adalah pelepasan sukarela hak atas milik atau penggunaan milik tersebut dengan maksud untuk menyenangkan Allah. Semua harta milik dan barang-barang menjadi milik Kongregasi, atau tarekat. 
    \item \textbf{Kaul kemurnian} mewajibkan manusia lepas perkawinan dan menghindari segala sesuatu yang dilarang oleh perintah keenam dan kesembilan. 
    \item \textbf{Kaul Ketaatan} lebih tinggi daripada dua kaul yang pertama. Sebab, kaul ketaatan adalah suatu kurban, dan ia lebih penting karena ia membangun dan menjiwai tubuh religius. Dengan kaul ketaatan biarawan-biarawati berjanji pada Allah untuk taat kepada para pimpinan yang sah dalam segala sesuatu yang mereka perintahkan demi peraturan. 
\end{itemize}
    
    \end{block}
\end{frame} 

\section{Partisipasi dalam perayaan liturgi}
\begin{frame}[fragile]
\frametitle{Partisipasi dalam perayaan liturgi}
\textit{Upacara liturgi menjadi lebih agung, bila ibadat kepada Allah dirayakan dengan nyanyian meriah, bila dilayani oleh petugas-petugas liturgi, dan bila Umat ikut serta secara aktif} (SC 113)

\textbf{Syarat keagungan liturgi}
\begin{enumerate}[<+->]
	\item nyanyian yang dipilih dalam perayaan liturgi disesuaikan dengan tema perayaan dan dapat diikuti oleh umat. 
    \item semua petugasliturgi mempersiapkan diri dan melakukan tugasnya dengan baik. 
    \item umat yang hadir berpartisipasi aktif dalam setiap bagian liturgi, entah itu ikut bernyanyi, menjawab dialog dan aklamasi yang diucapkan oleh imam atau petugas liturgi yang lain, atau pun melakukan tata gerak yang sesuai.
\end{enumerate}
\end{frame}

\begin{frame}[fragile]
\frametitle{Doa mohon panggilan}
\small 
Allah Bapa Pencipta, kami senantiasa bersyukur atas GerejaMu di Keuskupan Agung Semarang yang semakin hari semakin bertumbuh subur.
Kami juga bersyukur atas para Suster, Bruder dan Romo yang berkarya di Keuskupan kami.
Namun saat ini kami merasa prihatin, karena di antara kami masih sedikit yang menanggapi panggilanMu untuk turut bekerja di ladangMu.

Kami mohon kepadaMu, ya Bapa:

Gerakkanlah hati setiap orang tua, remaja dan anak-anak untuk senantiasa terbuka dan mengusahakan panggilan menjadi imam, bruder dan suster.

Semoga setiap orang tua dapat menjadi saksi cinta kasih dan mendidik anak-anaknya sebagaimana orang tua Yesus; Yusuf dan Maria mendidik Yesus sendiri.

Semoga dengan demikian anak muda dan anak-anak memiliki kedekatan denganMu baik di dalam doa ataupun dalam kehidupannya, sehingga bila suatu saat Engkau memanggil mereka untuk menjadi imam, bruder dan suster, mereka senantiasa siap sedia.

Ya Bapa semoga selalu ada orang-orang yang Kau utus untuk menuai panenan subur dan menyebarkan kabar gembira di Indonesia dengan segala kemajemukannya.

Demi Kristus Tuhan, Saudara dan Pengantara kami yang hidup dan berkuasa sepanjang segala masa.
Amin.
\end{frame}



\end{document}