\documentclass[a4paper]{article}

\title{Mazmur 51 - Mazmur Pertobatan\\BKS 2010}
\author{Lingkungan St Petrus Maguwo}
\date{30 September 2010}
\begin{document}
\maketitle
\section*{Pengakuan dosa}
Mazmur ini terkenal dalam liturgi kita karena dipakai pada masa tobat. Terkenal dengan sebutan Miserere. Itulah judulnya dalam Latin, karena ia memohon belas kasih dan ampunan Allah. Latar belakangnya ialah memohon kepada Allah agar diampuni dan dibaharui hidupnya. Kita dapat membagi mazmur ini dalam empat bagian. Bagian I, ay 1-8. Bagian II, ay 9-15. Bagian III, ay 16-19. Bgn IV, ay 20-21. 

\subsection*{Ayat 20-21}
\begin{enumerate}
\setcounter{enumi}{19}
\item Lakukanlah kebaikan kepada Sion menurut kerelaan hati-Mu bangunkanlah tembok-tembok Yerusalem!

\item Maka Engkau akan berkenan kepada korban yang benar, korban bakaran dan korban yang terbakar seluruhnya; maka orang akan mengorbankan lembu jantan di atas mezbah-Mu.
\end{enumerate}


\subsection*{Tobat}
Tobat, adalah kata serapan Arab, menurut Kamus Bahasa Indonesia 
\begin{quote}
tobat: (kk) sadar dari dosa dan tak akan mengulanginya lagi; kembali ke jalan yang benar, kembali ke jalan agama; menyesali perbuatannya, sedangkan bertobat adalah : (kk) menyesal dan berniat hendak memperbaiki perbuatan (perilakunya); kembali kepada Tuhan atau agama (jalan) yang benar.
\end{quote}

\section*{Bagian Pertama - ayat 1-8}
Dalam Bgn I, ada beberapa hal penting. Dalam ay 3 ada kesadaran bahwa hanya Allah yang bisa mengampuni. Itu sebabnya ia berseru kepada Allah mohon belas kasih. Ia merasa jika diampuni Allah ia akan hidup. Dalam ay 4 ia mohon agar dibersihkan dari dosa. Ia mengambi perumpamaan mencuci pakaian: ia meminta agar dicuci seperti pakaian kotor, yaitu dibanting di batu atau dikucak di penggilasan. Dalam ay 5 ia melukiskan satu pengalaman gangguan suara hati. Ini muncul karena ia sadar akan dosa dan pelanggarannya dan hal itu membayang di hadapannya. Semacam dihantui rasa bersalah (guilty feeling). Dosa mengalami proses personifikasi dan itu dilawan. Dalam ay 6 ada kesadaran bahwa dosa sosial-horizontal juga adalah dosa vertikal-teologal. Itu karena dosa merusak manusia citra Allah, bait Roh. Dengan berdosa aku pantas dihukum mati. Jika aku mati, Allah tetap adil dalam perbuatanNya karena aku yang berdosa. Ay 7 melukiskan situasi eksistensial manusia, yaitu ia terarah kepada dosa. Tetapi sekaligus ia ditarik oleh Allah. Ay 8 melukiskan daya tarik Allah itu dalam dan melalui suara hati. Ini pandangan positif akan suara hati. Allah dapat berbicara dalam suara hati kita. 


\section*{Bagian Kedua - ayat 9-15}
Dalam Bgn II ada beberapa hal penting. Dalam ay 9 ia memohon agar dicuci Allah supaya jadi putih seperti salju. Ia berharap agar dari situ ia muncul dengan sukacita (ay 10). Jika hal itu terjadi, maka kemurungan akibat dosa akan berlalu. Pengalaman sukacita rohani itu tampak secara jasmani. Kini jalan pelan diganti dengan loncat kegirangan. Dalam ay 11 ia memohon penghapusan dosa, agar Allah tidak usah melihat dosanya. Atas dasar itu, dalam ay 12 ia mohon perubahan, pembaharuan, dijadikan ciptaan baru. Ia mohon agar dibua hati baru, Roh baru. Ia mohon agar diberi Roh yang teguh agar tidak mudah jatuh dalam dosa. Ini perubahan dan pembaharuan dari dalam. Tidak pada level permukaan. Itu diharapkan datang dari Allah. Dalam ay 13 ia meminta (permintaan khusus): agar Allah tidak mengambil roh-Nya dari dia, dan agar jangan sampai ia dicampakkan dari hadapan Allah. Mudah-mudahan dengan itu ia berharap akan mudah rela dan taat (ay 14). Dalam ay 15 ada niat dan janji untuk mengajarkan semuanya itu kepada semua orang. Ia mau menebarkan pengalaman akan cinta kasih Allah itu. 

\section*{Bagian Ketiga - ayat 16-19}
Dalam Bgn III, ada beberapa hal penting. Dalam ay 16 ia melukiskan dosa pembunuhan. Mungkin ini pengalaman Daud yang “membunuh” Uria demi Batsyeba. Ia minta dibebaskan dari ganggugan dosa itu agar ia bebas mewartakan pujian Allah. Niat itu dilanjutkan dalam ay 17 yang terkenal karena dipakai dalam Ibadat Harian untuk mengawali Doa-doa lima atau tujuh waktu gereja. Di sini ia melukiskan ada niat untuk mewartakan Tuhan; dalam rangka itu ia memohon kepada Allah agar sudi membuka mulutnya agar menjadi alat pewartaan itu. Ay 18 melukiskan penolakan korban silih. Allah tidak berkenan pada korban silih itu. Sejalan dengan para nabi, korban yang terbaik ialah hati yang remuk-redam, jiwa yang hancur karena dosa dan bertobat. Allah berkenan kepada hati seperti ini. Sebab korban sejati yang berkenan kepada Allah ialah hati yang bertobat (ay 19). 

\section*{Bagian Keempat - ayat 20-21}
Bagian IV, ada beberapa hal penting juga. Mazmur asli hanya sampai ay 19. Sampai ay 19 itu adalah mazmur doa pribadi (individual). Tetapi dalam ayat 20-21 ada perubahan besar, yang diduga tambahan pada masa pasca pembuangan. Apa yang tadinya merupakan doa pribadi, dengan tambahan dua ayat ini, seluruh mazmur menjadi doa atau liturgi komunal yang dipakai di Bait Allah. Yang tadinya bersikap negatif terhadap ibadat korban, tetapi kini korban seakan dipulihkan justru karena ibadat kolektif di Yerusalem.

\section*{Doa}
 Selamatkan saya, Tuhan, dari kehancuran diri.
 
\end{document}
 