\documentclass[10pt]{book}
\usepackage[a4paper,top=3cm,bottom=3cm]{geometry}

\makeatletter
\newcommand{\judul}[1]{%
  {\parindent \z@ \centering \normalfont
    \interlinepenalty\@M \Large \bfseries #1\par\nobreak \vskip 20\p@ }}
\newcommand{\subjudul}[1]{%
  {\parindent \z@ \normalfont
    \interlinepenalty\@M \bfseries #1\par\nobreak \vskip 20\p@ }}
\newcommand{\lagu}[1]{%
  {\parindent \z@ \normalfont
    \interlinepenalty\@M \bfseries \emph{#1}\par\nobreak \vskip 20\p@ }}

\renewenvironment{description}
               {\list{}{\labelwidth\z@ \itemindent-\leftmargin
                        \let\makelabel\descriptionlabel}}
               {\endlist}
\renewcommand*\descriptionlabel[1]{\hspace\labelsep 
                                \normalfont\bfseries #1 }
    

\makeatother

\newcommand{\BU}[1]{\begin{itemize} \item[U:] #1 \end{itemize}}
\newcommand{\BI}[1]{\begin{itemize} \item[I:] #1 \end{itemize}}
\newcommand{\BP}[1]{\begin{itemize} \item[P:] #1 \end{itemize}}
\newcommand{\BL}[1]{\begin{itemize} \item[Wawan:] #1 \end{itemize}}
\newcommand{\BW}[1]{\begin{itemize} \item[Novi:] #1 \end{itemize}}
\newcommand{\BMP}[1]{\begin{itemize} \item[W+N:] #1 \end{itemize}}
\newcommand{\BS}[1]{\begin{itemize} \item[Saksi:] #1 \end{itemize}}
\newcommand{\BO}[1]{\begin{itemize} \item[Orangtua:] #1 \end{itemize}}
\newcommand{\ultah}{40 }
\newcommand{\suami}{V.F. Parlan }
\newcommand{\istri}{M.G. Waljiyah }
\newcommand{\ultahdua}{15 }
\newcommand{\suamidua}{Y. Suyanto }
\newcommand{\istridua}{Th. Prima Ari S }
\newcommand{\mempelaip}{Fransiskus Xaverius Catur Ari Setiawan }
\newcommand{\mempelaiw}{Tuti Novianti }
\newcommand{\romo}{Adolfus Suratmo Pr. }
\hyphenation{ba-gi-mu}
\hyphenation{de-ngan}
\hyphenation{di-se-rah-kan}
\hyphenation{di-res-mi-kan}
\hyphenation{me-la-lui}
\hyphenation{ka-nak}
\hyphenation{ka-re-na}
\hyphenation{ber-ka-ta}
\hyphenation{te-ta-pi}
\hyphenation{per-ka-win-an}
\hyphenation{pa-tut}
\hyphenation{me-lu-hur-kan}
\hyphenation{ber-nya-nyi}
\hyphenation{di-tum-pah-kan}
\hyphenation{pe-ngam-pun-an}
\hyphenation{ber-a-da}
\hyphenation{kau-lim-pah-kan}
\hyphenation{Xa-ve-ri-us}
\hyphenation{se-ka-li}
\hyphenation{ke-sa-yang-an-Mu}
\hyphenation{ke-mu-li-a-an-Mu}
\hyphenation{ke-sa-yang-an}
\hyphenation{ke-mu-li-a-an}
\hyphenation{ke-sa-yang-an-Mu}
\hyphenation{Tu-han}

\usepackage[bahasa]{babel}
\selectlanguage{bahasa}

%\topmargin=-0.5in
%\textheight=8in

\begin{document}
\thispagestyle{empty}
\Large
\judul{RITUS PEMBUKA}

\subjudul{UCAPAN SELAMAT DATANG}

\BI{Selamat datang saudara-saudara sekalian, semoga rahmat, damai sejahtera dan kasih Allah Bapa Yang Maha Kuasa selalu beserta kita.}

\BU{Sekarang dan selama-lamanya.}


\subjudul{PENYERAHAN KEDUA MEMPELAI}

\BO{Romo \romo yang kami hormati, pada hari ini kami menyerahkan putera-puteri kami: \mempelaip dan \mempelaiw , untuk  menerima Berkat Pernikahan menurut tata cara Gereja Katolik yang kudus. Kami mohon kesediaan Romo untuk meresmikan dan meneguhkan pernikahan pasangan ini serta memohonkan berkat Tuhan bagi mereka.}

\BI{Dengan gembira, saya atas nama Gereja menerima kedua mempelai dan akan meneguhkan pernikahan mereka menurut tata cara Gereja Katolik.}


\subjudul{PEMERCIKAN AIR SUCI} 

\BI{Saya memerciki kalian berdua dengan air suci ini seperti embun surgawi. Semoga kalian diberkati oleh Tuhan, diterangi dengan sinarNya, dilimpahi dengan rahmat kasihNya dan disucikan dalam Roh Kudus, agar pantas menghadapi peristiwa suci ini.} 

\BU{Amin.}

\BI{Marilah kita bersama-sama dengan penuh hikmat menghadap Allah, sumber segala cinta kasih, untuk memohon berkatNya bagi kedua mempelai ini serta bagi kita semuanya.}

\lagu{Lagu pembukaan: Berserah Setia}

\subjudul{Tanda Salib}
\BI{Dalam nama Bapa dan Putera dan Roh Kudus}
\BU{Amin}

\subjudul{Salam Pembukaan}
\BI{Rahmat Tuhan kita Yesus Kristus, cinta kasih Allah, dan persekutuan Roh Kudus beserta kita}
\BU{Sekarang dan selama-lamanya}

\subjudul{Pengantar}
\BI{Misa syukur dan kenangan ini secara khusus dipersembahkan bagi kita semua sebagai ungkapan terima kasih atas segala kebaikan yang senantiasa mengalir di dalam keluarga ini.

Wawan dan Novi yang berbahagia, hari ini kalian berdua datang ke hadapan Allah untuk memohon agar cinta kasih kalian berdua dikuduskan sebagai tanda kehadiran kasih Tuhan di tengah keluarga yang akan kalian bangun bersama.  

		Kini bersama segenap keluarga dan saudara semua, marilah kita hening sejenak, mempersiapkan diri dan hati kita dengan memeriksa batin masing-masing.  Marilah kita mengarahkan hati kita kepada Tuhan, mohon belas kasih dan rahmat pengampunanNya guna memasuki perayaan cinta kasih yang agung ini.}

\subjudul{Tobat}
\BI{Marilah kita hening sejenak untuk mempersiapkan diri dalam perayaan syukur ini sambil menyadari bahwa kita sering melupakan kebaikan Tuhan dan enggan mewartakan dan mewujudkan kebaikan tersebut melalui pikiran, perkataan, dan perbuatan kita.}

\BI{Saya mengaku}

\BU{Kepada Allah yang Maha Kuasa dan kepada saudara sekalian bahwa saya telah berdosa dengan pikiran dan perkataan, dengan perbuatan dan kelalaian. Saya berdosa, saya berdosa, saya sungguh berdosa. Oleh sebab itu saya mohon kepada Santa Perawan Maria, kepada Para Malaikat dan orang kudus dan kepada saudara sekalian, supaya mendoakan saya kepada Allah Tuhan kita.}

\BI{Semoga Allah Yang Maha Kuasa mengasihi kita, mengampuni dosa kita dan menghantar kita ke hidup yang kekal.}

\BU{Amin}

\lagu{Tuhan Kasihanilah Kami}

\subjudul{Doa Pembukaan}

\BI{Marilah kita berdoa,
Allah yang Maha Kasih, Engkau menghendaki agar pria dan wanita membangun keluarga yang berbahagia. Kedua hambaMu ini sudah siap memasuki bahtera pernikahan. Berkatilah mereka agar selalu menyadari kesucian hidup berkeluarga dan berusaha menghayatinya dalam suka maupun duka dan anugerahkanlah kepada mereka keturunan yang dapat dibanggakan.

Bapa di surga, \ultah tahun yang lalu Engkau telah memanggil Bapak \suami dan Ibu \istri serta \ultahdua tahun sudah Bapak \suamidua dan Ibu \istridua untuk membangun cinta bersama dalam sebuah keluarga. Dengan segala kelebihan dan kekurangannya keluarga ini telah menanggapi panggilan-Mu. Berkatilah selalu cinta kasih mereka agar mereka tetap setia untuk selama-lamanya.

Semoga cinta kasih tersebut terus tumbuh dan berkembang di dalam keluarga besar ini. Semoga keluarga ini tetap terus bercermin pada segala kebaikan yang senantiasa mengalir dari orang-orang disekitar mereka. Dengan demikian keluarga ini senantiasa hidup dalam kedamaian, kerukunan, dan kesederhanaan, demi Yesus Kristus Putera-Mu, Tuhan dan Pengantara kami yang hidup bersama Dikau dalam persekutuan Roh Kudus kini dan sepanjang masa.}

\BU{Amin}

\judul{LITURGI SABDA}

\subjudul{Bacaan pertama: I Korintus 13:1-13}

\BP{\emph{Pembacaan dari Surat Pertama Rasul Paulus kepada Umat di Korintus.}

Sekalipun aku dapat berkata-kata dengan semua bahasa manusia dan bahasa malaikat, tetapi jika aku tidak mempunyai kasih, aku sama dengan gong yang berkumandang dan canang yang gemerincing.
Sekalipun aku mempunyai karunia untuk bernubuat dan aku mengetahui segala rahasia dan memiliki seluruh pengetahuan; dan sekalipun aku memiliki iman yang sempurna untuk memindahkan gunung, tetapi jika aku tidak mempunyai kasih, aku sama sekali tidak berguna.

Dan sekalipun aku membagi-bagikan segala sesuatu yang ada padaku, bahkan menyerahkan tubuhku untuk dibakar, tetapi jika aku tidak mempunyai kasih, sedikitpun tidak ada faedahnya bagiku.

Kasih itu sabar; kasih itu murah hati; ia tidak cemburu. Ia tidak memegahkan diri dan tidak sombong.
Ia tidak melakukan yang tidak sopan dan tidak mencari keuntungan diri sendiri. Ia tidak pemarah dan tidak menyimpan kesalahan orang lain.
Ia tidak bersukacita karena ketidakadilan, tetapi karena kebenaran.
Ia menutupi segala sesuatu, percaya segala sesuatu, mengharapkan segala sesuatu, sabar menanggung segala sesuatu.

Kasih tidak berkesudahan; nubuat akan berakhir; bahasa roh akan berhenti; pengetahuan akan lenyap.
Sebab pengetahuan kita tidak lengkap dan nubuat kita tidak sempurna.
Tetapi jika yang sempurna tiba, maka yang tidak sempurna itu akan lenyap.
Ketika aku kanak-kanak, aku berkata-kata seperti kanak-kanak, aku merasa seperti kanak-kanak, aku berpikir seperti kanak-kanak. Sekarang sesudah aku menjadi dewasa, aku meninggalkan sifat kanak-kanak itu.
Karena sekarang kita melihat dalam cermin suatu gambaran yang samar-samar, tetapi nanti kita akan melihat muka dengan muka. Sekarang aku hanya mengenal dengan tidak sempurna, tetapi nanti aku akan mengenal dengan sempurna, seperti aku sendiri dikenal.

Demikianlah tinggal ketiga hal ini, yaitu iman, pengharapan dan kasih, dan yang paling besar di antaranya ialah kasih.}
\BU{Syukur kepada Allah.}

\lagu{Lagu antar bacaan: Keheningan Hati}

\lagu{Bait Pengantar Injil}

\BI{Alleluia, Alleluia, Alleluia} 

\BU{Alleluia, Alleluia, Alleluia} 

\BI{Jika kita saling mengasihi, Allah tinggal di dalam kita dan cinta kasih Allah di dalam kita menjadi sempurna.} 
	
\BU{Alleluia, Alleluia, Alleluia} 

\subjudul{Bacaan Injil: Lukas 12: 22 - 31}

\BI{Tuhan sertamu}

\BU{Dan sertamu juga}

\BI{Inilah Injil Yesus Kristus menurut Santo Lukas}

\BU{Dimuliakanlah Tuhan}

\BI{Yesus berkata kepada murid-murid-Nya: "Karena itu Aku berkata kepadamu: Janganlah kuatir akan hidupmu, akan apa yang hendak kamu makan, dan janganlah kuatir pula akan tubuhmu, akan apa yang hendak kamu pakai.
Sebab hidup itu lebih penting dari pada makanan dan tubuh itu lebih penting dari pada pakaian.
Perhatikanlah burung-burung gagak yang tidak menabur dan tidak menuai dan tidak mempunyai gudang atau lumbung, namun demikian diberi makan oleh Allah. Betapa jauhnya kamu melebihi burung-burung itu!
Siapakah di antara kamu yang karena kekuatirannya dapat menambahkan sehasta pada jalan hidupnya?
Jadi, jikalau kamu tidak sanggup membuat barang yang paling kecil, mengapa kamu kuatir akan hal-hal lain?
Perhatikanlah bunga bakung, yang tidak memintal dan tidak menenun, namun Aku berkata kepadamu: Salomo dalam segala kemegahannyapun tidak berpakaian seindah salah satu dari bunga itu.
Jadi, jika rumput di ladang, yang hari ini ada dan besok dibuang ke dalam api demikian didandani Allah, terlebih lagi kamu, hai orang yang kurang percaya!
Jadi, janganlah kamu mempersoalkan apa yang akan kamu makan atau apa yang akan kamu minum dan janganlah cemas hatimu.
Semua itu dicari bangsa-bangsa di dunia yang tidak mengenal Allah. Akan tetapi Bapamu tahu, bahwa kamu memang memerlukan semuanya itu.
Tetapi carilah Kerajaan-Nya, maka semuanya itu akan ditambahkan juga kepadamu.

Berbahagialah orang yang mendengarkan sabda Tuhan, dan tekun melaksanakannya.}

\BU{Sabda-Mu adalah jalan, kebenaran dan hidup kami.}

\subjudul{Homili}

\judul{Upacara Pemberkatan Perkawinan}

\subjudul{PERNYATAAN KEDUA MEMPELAI DAN SAKSI}

\BI{Wawan dan Novi yang berbahagia, kalian datang kemari untuk merayakan Pernikahan di hadapan Gereja.  Maka atas nama Gereja yang Kudus, saya minta supaya kalian menyatakan maksud dan isi hati kalian di hadapan saya dan para saksi.}

\BMP{Romo \romo yang kami hormati, kami berdua telah bersatu hati dan saling memilih sebagai teman hidup, maka kami mohon kesediaan Pastor sebagai perantara Kristus untuk mengukuhkan dan meresmikan hubungan kami sebagai suami-istri yang sah menurut tata cara Gereja Katolik.}

\BI{Para saksi yang terhormat, apakah ada suatu hal yang menghalangi pernikahan Wawan dan Novi ini menurut tata cara Gereja Katolik?} 

\BS{Sepanjang pengetahuan kami, tidak ada satupun halangan untuk meresmikan pernikahan ini. Sebab itu kami mendukung permohonan kedua mempelai.}

\BI{Wawan dan Novi, sesudah kalian berkenalan satu sama lain, sesudah kalian mengerti kekuatan dan kelemahan cinta, sesudah kalian kurang lebih memahami kepribadian masing-masing dari segi baik dan kurang baik, sesudah kalian menjadi yakin akan kemampuan saling membahagiakan, sesudah kalian dibimbing oleh Gereja menuju Pemberkatan Pernikahan, saya minta agar kalian berdua menyatakan maksud kalian dengan jujur di hadapan Allah dan para saksi, serta umat sekalian yang hadir di sini.}


\subjudul{PERNYATAAN MEMPELAI PRIA}

\BI{\mempelaip , apakah saudara sungguh datang ke tempat ini dengan kemauan sendiri tanpa adanya paksaan untuk menyerahkan dirimu dalam ikatan pernikahan dengan \mempelaiw ?}

\BL{Ya, saya sungguh.}

\BI{Bersediakah saudara mengasihi dan menghormati  \mempelaiw menjadi istrimu seumur hidup?}

\BL{Ya, saya bersedia.}

\BI{Bersediakah saudara menjadi bapak yang baik bagi anak-anak yang dipercayakan Tuhan kepada saudara dan mendidik mereka menjadi orang Katolik yang setia?}

\BL{Ya, saya bersedia.}


\subjudul{PERNYATAAN MEMPELAI WANITA}

\BI{\mempelaiw, apakah saudara sungguh datang ke tempat ini dengan kemauan sendiri tanpa adanya paksaan untuk menyerahkan dirimu dalam ikatan pernikahan dengan \mempelaip?}

\BW{Ya, saya sungguh.}

\BI{Bersediakah saudara mengasihi dan menghormati  \mempelaip menjadi suamimu seumur hidup?}

\BW{Ya, saya bersedia.}

\BI{Bersediakah saudara menjadi ibu yang baik bagi anak-anak yang dipercayakan Tuhan kepada saudara dan mendidik mereka menjadi orang Katolik yang setia?}

\BW{Ya, saya bersedia.}


\subjudul{PERMOHONAN DOA RESTU ORANG TUA}

\BI{Sekarang tibalah saatnya kedua mempelai yang selama ini telah dibesarkan dalam keluarga, ingin mengucapkan terima kasih kepada kedua orang tua yang telah melahirkan dan membesarkan mereka. Mereka mohon doa restu untuk memulai membangun keluarga baru sendiri.}


\lagu{Lagu pengiring: Berkatilah}



\subjudul{JANJI PERNIKAHAN}

\BI{Sekarang tibalah saatnya untuk meresmikan pernikahan ini.  Saya persilakan Wawan dan Novi menumpangkan tangan di atas Kitab Suci ini. Teguhkanlah sekarang pernikahanmu dengan berjanji setia satu sama lain di hadapan Allah.}

\BL{Di hadapan Gereja Allah yang Kudus, saya, \mempelaip, menyatakan dengan tulus kepadamu, \mempelaiw, bahwa saya memilih engkau menjadi istri saya. Saya berjanji setia kepadamu dalam suka dan duka, di waktu sehat maupun sakit, dalam untung dan malang. Saya mau mencintai dan menghormati engkau seumur hidup saya sampai maut memisahkan kita.}

\BW{Di hadapan Gereja Allah yang Kudus, saya, \mempelaiw, menyatakan dengan tulus kepadamu, \mempelaip, bahwa saya memilih engkau menjadi suami saya. Saya berjanji setia kepadamu dalam suka dan duka, di waktu sehat maupun sakit, dalam untung dan malang. Saya mau mencintai dan menghormati engkau seumur hidup saya sampai maut memisahkan kita.}


\subjudul{PEMBERKATAN PERNIKAHAN}

\BI{Atas nama Gereja Allah dan di hadapan para saksi serta hadirin sekalian, saya menegaskan bahwa pernikahan yang telah diresmikan ini adalah pernikahan yang sah menurut hukum Gereja Katolik. Semoga berkat ini menjadi sumber kekuatan dan kebahagiaan bagi kalian berdua, demi nama Bapa dan Putera dan Roh Kudus.}

\BU{Amin.}

\BI{Yang telah dipersatukan Allah.}

\BU{Janganlah diceraikan oleh manusia.}


\subjudul{BERKAT UNTUK MEMPELAI}

\BI{Saudara-saudara terkasih, marilah kita berdoa untuk kedua mempelai ini dan mohon berkat Allah bagi mereka. Semoga Allah mencurahkan rahmatNya dengan murah hati.

		 \emph{... Hening sejenak ...}
		 
		Allah yang Maha Kasih, Engkau menciptakan segala sesuatu dengan kekuatan kekuasaanMu. Engkau menciptakan manusia menurut citraMu. Engkau menciptakan pria dan wanita supaya mereka dipadukan menjadi satu. Engkau mengajarkan bahwa pernikahan yang telah Engkau teguhkan tidak boleh diceraikan.

		Bapa, pandanglah Novi dengan rela agar rahmat, cinta kasih dan damaiMu tinggal di dalam hatinya.  Semoga ia menjadi istri yang setia dan ibu yang baik seperti wanita-wanita yang dipuji dalam Kitab Suci.
		Kami berdoa pula untuk Wawan, semoga ia selalu berusaha menunaikan tanggung jawabnya sebagai kepala rumah tangga, baik terhadap istri dan anak-anak yang akan  dipercayakan Tuhan kepadanya, maupun terhadap masyarakat.

		Dan kami memohon kepadaMu, ya Tuhan, semoga kedua mempelai ini tetap berpegang pada iman dan perintah-perintahMu, semoga mereka bersatu sebagai suami-istri terpandang karena peri kehidupan yang baik dan berjasa untuk sesama.
		Kuatkanlah mereka dengan semangat Injil, sehingga mereka menjadi saksi Kristus bagi semua orang. Semoga mereka subur dan berketurunan, menjadi orang tua yang patut dicontoh dan berbahagia melihat anak cucunya.  Semoga mereka dikaruniai umur yang panjang hingga akhirnya memasuki kehidupan bahagia dalam kerajaan surga. Demi Kristus Tuhan dan Pengantara kami.}

\BU{Amin.}

\subjudul{Pemberkatan dan pemakaian cincin}

\lagu{Lagu pengiring: Cincin Kami}

\BI{Ya Allah, Curahkanlah berkat-Mu atas cincin-cincin ini (\Large{$\dagger)$}}

\BI{Wawan, kenakanlah cincin ini pada jari manis istrimu sebagai lambang cinta kasih dan kesetiaan yang abadi.} 

\BL{Novi, terimalah cincin ini sebagai lambang kesetiaan dan cinta kasihku kepadamu.}  
		
\BI{Novi, kenakanlah cincin ini pada jari manis suamimu sebagai lambang cinta kasih dan kesetiaan yang abadi.} 

\BW{Wawan, terimalah cincin ini sebagai lambang kesetiaan dan cinta kasihku kepadamu.}  
		
\BI{Bapak \suami dan Ibu \istri serta Bapak \suamidua dan Ibu \istridua telah mengawali semuanya dengan kasih. Kenakanlah cincin ini satu sama lain sebagai tanda peringatan bahwa kasih itu abadi. Dan abadilah kasih itu. Kasih itu pula yang akan membimbing dan menghantar Bapak dan Ibu serta seluruh keluarga menuju rumah abadi.}

\BU{Amin.}

\subjudul{PEMBERKATAN KITAB SUCI, SALIB DAN ROSARIO}

\BI{Ya Tuhan, berkatilah Kitab Suci, Salib dan Rosario ini agar selalu menjadi tanda kehadiranMu serta Bunda Maria di tengah keluarga baru ini dan dapat memberikan dorongan untuk selalu siap memberikan pengorbanan demi kebahagiaan sesama.}

\BU{Amin.}

\BI{Wawan dan Novi, sebagai wakil dari orang tua kalian, saya menyerahkan Salib Tuhan dan Kitab Suci ini, sebagai lambang cinta kasih dan penyertaan Allah dalam keluargamu. Bacalah selalu Kitab Suci ini dan pergunakanlah sebagai pegangan hidup kalian. Berdoalah di hadapan Salib Tuhan ini, saat engkau bahagia maupun dalam percobaan yang melanda bahtera keluargamu. Yakinlah dalam setiap doamu, maka Tuhan akan melimpahkan berkatNya. Terimalah pula rosario Bunda Maria, agar kehadiran dan perlindungannya menyejukkan suasana hidup dalam keluargamu.}

\BMP{Terima kasih.}


\subjudul{Aku Percaya}

\subjudul{Doa Umat}

\BI{Dengan dilandasi rasa syukur atas kelimpahan rahmat yang telah diterima oleh Saudara \mempelaip dan \mempelaiw, keluarga Bapak \suami dan Ibu \istri, serta keluarga Bapak \suamidua dan Ibu \istridua marilah kita mengucapkan puji syukur kepada Tuhan}

\BP{Semoga kedua mempelai senantiasa menghayati hidup pernikahan dalam cinta kasih dan damai, selalu setia dan saling menolong sampai akhir hidup mereka sehingga rahmat dan kebaikan Kristus bersinar dari rumah tangga mereka. 
Marilah kita mohon...}

\BU{Kabulkanlah doa kami, ya Tuhan.}

\BP{Semoga cinta kasih mereka diberkati oleh Tuhan dengan karunia yang berlimpah, sehingga anak-anak yang dianugerahkan kepada mereka sungguh-sungguh menggembirakan hati kedua orangtuanya dan menjadi pewaris gerejaMu di dunia ini.
Marilah kita mohon...}

\BU{Kabulkanlah doa kami, ya Tuhan.}

\BMP{Semoga pernikahan ini memberi kebahagiaan bagi orang tua dan kaum kerabat kami sehingga terjalin ikatan persaudaraan yang erat dan saling membantu.
Marilah kita mohon...}

\BU{Kabulkanlah doa kami, ya Tuhan.}

\BP{Bapa Maha Pengasih, terima kasih atas rahmat yang telah Kau limpahkan kepada keluarga Bapak \suami dan Ibu \istri 
dalam \ultah tahun perkawinan serta Bapak \suamidua dan Ibu \istridua dalam \ultahdua tahun perkawinan semoga kamipun senantiasa mensyukuri segala rahmat yang Kauberikan kepada kami, sekalipun itu tampaknya sederhana dan kecil.

Marilah kita berseru:}

\BU{Syukur kepada-Mu ya Tuhan.}

\BP{Bapa Maha Pengasih, perjalanan \ultah dan \ultahdua tahun membina hidup rumah tangga sungguh merupakan perjuangan hidup yang patut diteladani. Engkau memberi keluarga ini kebahagiaan dan kesejahteraan, sebagai buah-buah perjuangannya selama ini. Marilah kita berseru:}

\BU{Syukur kepada-Mu ya Tuhan.}

\BP{Kami berdoa pula untuk sanak saudara kami yang hari-hari ini tidak dapat menikmati kebahagiaan hidup karena tertimpa musibah, bencana dan masih bergulat dengan persoalan hidup mereka. Semoga kebahagiaan kami boleh kami tularkan kepada mereka sebagai perwujudan cinta kasih-Mu sendiri kepada manusia. Marilah kita berdoa:}

\BU{Dengarkanlah doa kami, ya Tuhan.}

\BP{Bagi kami - umat yang berkumpul di sini, ya Tuhan. Semoga kami tak segan-segan menolong dan membantu sesama kami membagikan rahmat yang telah Kauanugerahkan kepada kami. Marilah kita berdoa:}

\BU{Dengarkanlah doa kami, ya Tuhan.}

\judul{LITURGI EKARISTI}

\subjudul{Persiapan Persembahan}

\lagu{Lagu persembahan: Kupersembahkan}

\BI{Kami memuji Engkau, Ya Bapa, Allah semesta alam. Sebab dari kemurahan-Mu kami menerima roti dan anggur yang kami persembahkan ini. Inilah hasil dari bumi dan usaha manusia yang bagi kami akan menjadi santapan rohani.}

\BU{Terpujilah Allah selama-lamanya}

\BI{Berdoalah saudara-saudara, supaya persembahan kita ini diterima oleh Allah, Bapa Yang Mahakuasa.}

\BU{Semoga persembahan ini diterima demi kemuliaan Tuhan dan keselamatan kita serta seluruh umat Allah yang kudus.}

\subjudul{Doa Persiapan Persembahan:}

\BI{Marilah berdoa,

Allah Bapa di surga, terimalah ucapan puji dan syukur keluarga ini sebagai persembahan yang harum mewangi. Sebab hanya puji dan syukur yang mereka berikan atas segala kebaikan yang senantiasa Kau limpahkan kepada keluaraga ini melalui orang-orang di sekitar keluarga ini. Dengan pengantaraan Kristus, Tuhan kami.}

\BU{Amin.}

\subjudul{Prefasi}

\BI{Tuhan sertamu}
\BU{Dan sertamu juga}
\BI{Marilah mengarahkan hati kepada Tuhan}
\BU{Sudah kami arahkan}
\BI{Marilah bersyukur kepada Tuhan Allah kita}
\BU{Sudah layak dan sepantasnya}
\BI{Sungguh layak dan sepantasnya, ya Bapa yang Kudus, Allah yang kekal dan kuasa, bahwa di manapun juga kami senantiasa bersyukur kepada-Mu dengan pengantaraan Kristus, Tuhan kami. Sebab kami yang seharusnya binasa karena dosa telah ditebus oleh kemenangan Kristus atas maut, dan karena kemurahan serta kebaikan-Mu kami dipanggil untuk hidup bersama Dia, Tuhan dan pengantara kami. Maka, kami selalu meluhurkan Dikau bersama seluruh laskar surgawi yang tak henti-hentinya memuji keagungan-Mu sambil bersukaria dan bernyanyi:}

\lagu{Kudus}

\subjudul{Doa Syukur Agung II}

\BI{Sunguh kuduslah Engkau ya Bapa. Segala ciptaan patut memuji Engkau. Sebab dengan pengantaraan Putra-Mu, Tuhan kami Yesus Kristus, dan dengan daya kekuatan Roh Kudus Engkau menghidupkan dan menguduskan segala sesuatu. Tak henti-hentinya Engkau menghimpun umat-Mu, sehingga dari terbitnya matahari sampai terbenamnya di seluruh bumi dipersembahkan kurban yang murni untuk memuliakan nama-Mu. Maka kami mohon ya Bapa sudilah menguduskan persembahan ini dengan Roh-Mu agar bagi kami menjadi tubuh ($\dagger$) dan darah Putra-Mu terkasih Tuhan kami, Yesus Kristus yang menghendaki kami merayakan misteri ini.

Sebab pada malam ia dikhianati, Yesus mengambil roti, Ia mengucap syukur dan memuji Dikau, memecahkan-mecahkan roti itu dan memberikannya kepada murid-murid-Nya seraya berkata:

TERIMALAH DAN MAKANLAH! INILAH TUBUHKU, YANG DISERAHKAN BAGIMU!

Demikian pula sesudah perjamuan, Yesus mengambil piala. Sekali lagi Ia mengucap syukur dan memuji Dikau, lalu memberikan piala itu kepada murid-murid-Nya seraya berkata:

TERIMALAH DAN MINUMLAH INILAH PIALA DARAHKU, DARAH PERJANJIAN BARU DAN KEKAL, YANG DITUMPAHKAN BAGIMU DAN BAGI SEMUA ORANG DEMI PENGAMPUNAN DOSA. LAKUKANLAH INI UNTUK MENGENANGKAN DAKU!}

\lagu{Anamnese}

\BI{Bapa, kami mengenangkan sengsara Putra-Mu, yang menyelamatkan, kebangkitan-Nya yang mengagumkan, dan kenaikkan-Nya ke surga. Sambil mengharapkan kedatangan-Nya kembali, dengan penuh syukur kami mempersembahkan kepada-Mu kurban yang hidup dan kudus ini. Kami mohon, pandanglah persembahan Gereja-Mu ini dan indahkanlah kurban yang telah mendamaikan kami dengan Dikau ini.

Semoga kami disempurnakan oleh-Nya menjadi suatu persembahan abadi bagi-Mu, agar kami pantas mewarisi kebahagiaan surgawi bersama para pilihan-Mu, terutama bersama Santa Perawan Maria, Bunda Allah, para rasul-Mu yang kudus, dan para martir-Mu yang jaya, dan bersama semua orang kudus yang selalu mendampingi dan menolong kami.

Ya Bapa, semoga berkat korban yang mendamaikan ini, damai sejahtera dan keselamatan semakin dirasakan oleh dunia.

Kuatkanlah Iman dan cinta kasih Gereja-Mu yang kini masih berziarah di bumi ini bersama hamba-Mu, Paus Benedictus XVI dan Uskup kami Ignatius Soeharyo, serta semua uskup, para imam, diakon serta semua pelayan umat, dan seluruh umat kesayangan -Mu. Dengarkanlah doa-doa umat-Mu yang berhimpun di sini. Demi kerahiman dan kasih setia-Mu, ya Bapa, persatukanlah semua anak-Mu di mana pun mereka berada.

Sudilah pula menganugerahkan kebahagiaan abadi kepada semua yang telah berpulang ke hadirat-Mu, saudarasaudara kami seiman, dan semua orang lain yang hidupnya berkenan pada-Mu. Pada waktu itu, Engkau mengahpus setiap tetes air mata kami, karena dengan memandang Engkau, ya Bapa, kami akan serupa dengan Dikau sepanjang masa dan tak henti-hentinya memuji Dikau.

Kami berharap, agar bersama mereka kami pun menikmati kemuliaan -Mu selama-lamanya, dengan pengantsraan Kristus, Tuhan kami. Sebab melalui Dialah Engkau melimpahkan segala yang baik kepada dunia.

Dengan pengantaraan Kristus, bersama Dia dan dalam Dia, bagi-Mu Allah Bapa Yang Maha Kuasa, dalam persekutuan dengan Roh Kudus, segala hormat dan kemuliaan, sepanjang segala mas. Amin.}

\subjudul{Bapa Kami}

\subjudul{Doa Damai}

\BI{Ya Bapa, dengan penuh kasih Engkau telah menciptakan dan menyertai keluarga-keluarga supaya menjadi sarana dan tanda bagi karya keselamatan-Mu.

Berkatilah kami semua serta seluruh keluarga kami dengan kasih sayang, kegembiraan, kerukunan, dan kedamaian.}

\BU{Tuhan Yesus Kristus, jangan memperhitungkan dosa kami, tetapi perhatikanlah iman gereja-Mu, dan restuilah kami supaya hidup bersatu dengan rukun sesuai dengan kehendak-Mu. Sebab Engkaulah pengantara kami, kini dan sepanjang masa.

Amin.}

\subjudul{Salam Damai}

\BI{Semoga Damai Tuhan kita Yesus Kristus beserta kita}

\BU{Sekarang dan selama-lamanya.}

\lagu{Anak Domba Allah}

\subjudul{Persiapan Komuni}

\BI{Inilah Anak domba Allah yang menghapus dosa dunia. Berbahagialah kita yang diundang ke perjamuan-Nya}
\BU{Ya Tuhan, saya tidak pantas Engkau datang pada saya. Tetapi bersabdalah saja, maka saya akan sembuh.}

\lagu{Lagu Komuni: Kau yang Terindah, Ave Maria}

\subjudul{Doa Sesudah Komuni}

\BI{Marilah kita berdoa,

Allah Bapa Yang Mahabaik, semoga segala kebaikan yang Kaulimpahkan kepada keluarga ini melalui orang-orang di sekitar keluarga ini semakin meneguhkan perjalanan keluarga ini menuju rumah-Mu yang abadi. Demikian pula kami semua yang hadir di tempat ini. Bukalah mata telinga hati kami supaya kami semakin menyadari bahwa Engkai sungguh Allah Yang Baik Hati. Bimbinglah kami selalu supaya kami mampu mewartakan kebaikan tersebut dalam kehidupan kami sehari-hari. Demi Kristus, Tuhan dan Pengantara kami. Kini, selalu, dan sepanjang segala abad.}

\BU{Amin}

\judul{RITUS PENUTUP}

\subjudul{Ucapan Terima Kasih}

\BP{Saya atas nama kedua mempelai serta keluarga Bapak/Ibu Murtama dan Bapak/Ibu \suami ingin mengungkapkan terima kasih yang sedalam-dalamnya kepada Romo \romo, Bapak-bapak, Ibu-ibu, dan saudara-saudara sekalian yang hadir dalam upacara pernikahan ini. Tidak lupa juga kami ucapkan terima kasih kepada semua pihak yang mendukunng persiapan upacara ini: petugas administrasi gereja, Bapak Suraja dan koornya, para saksi, dan semua pihak yang tidak dapat disebutkan satu persatu. Terima kasih dan Tuhan memberkati.}

\subjudul{Berkat}

\BI{Saudara sekalian, marilah kita mengakhiri misa syukur berkat perkawinan Saudara \mempelaip dan \mempelaiw, serta \ultah dan \ultahdua tahun perkawinan keluarga Bapak \suami dan Ibu \istri, Bapak \suamidua dan Ibu \istridua, dengan mohon berkat dari Tuhan.}
\BI{Tuhan sertamu}
\BU{Dan sertamu juga}
\BI{Semoga keluarga ini dan kita semua yang hadir di sini senantiasa dibimbing dan dilindungi dengan limpahan berkat dari Allah Yang Maha Baik. ($\dagger$) Atas Nama Bapa, Putra, dan Roh Kudus}
\BU{Amin}
\BI{Dengan demikian misa syukur berkat perkawinan dan ulang tahun perkawinan telah selesai.}
\BU{Syukur kepada Allah.}
\BI{Marilah pergi, kita diutus}
\BU{Amin}

\subjudul{Penyerahan kepada Bunda Maria}

\BMP{\emph{Bunda Maria Perawan Murni yang diberkati}

Engkau selalu menunjukkan belas kasih dan cinta kepada yang membutuhkan pertolongan-Mu. Terima salam kami di hari yang sangat membahagiakan ini, karena Putera-Mu telah mempersatukan kami dalam ikatan perkawinan kudus sebagai keluarga baru yang mendasarkan pada keyakinan kepada Kristus Yesus. Restuilah dan dampingilah kami berdua di sepanjang jalan kehidupan yang akan kami lalui bersama. Semoga kami berhasil membangun keluarga beriman yang penuh kebahagiaan dan kesejahteraan.

\emph{Bunda Maria Pelindung kami}

Semoga di bawah naungan berkat-Mu, kami mampu menjadikan diri kami sebagai keluarga yang rukun, bersatu padu penuh semangat cinta-kasih-Mu. Sudilah menolong, meneguhkan dan melindungi kami di setiap pencobaan hidup bersama ini. Kami percaya bahwa dengan meneladani hidup keluarga kudus dari Nazareth, kami semakin berkenan di hadapan Allah Bapa dan sesama kami.

Akhirnya ya Bunda, Bunda, kami mohon doamu untuk orang tua kami yang telah mengasuh, mengasihi dan memberi pelajaran kehidupan kepada kami, sejak kami dalam kandungan. Kami sangat berterima kasih atas orangtua kami yang baik dan penuh cinta serta bijaksana terhadap kami anak-anaknya.
Lindungilah dan berkatilah mereka selalu agar senantiasa dikaruniai kebahagiaan dan kedamaian.

Kami berdoa pula bagi Romo \romo, putera altar, para saksi, handai taulan dan seluruh saudara kami yang telah mengikuti perayaan Ekaristi ini dan mendoakan kami. Amin.

\emph{Salam Maria 3x}}


  
\lagu{Lagu penutup: The Wedding}

\judul{Ucapan Terima Kasih}
\begin{center}
Dengan penuh rasa syukur kepada Allah, terima kasih yang tulus kami haturkan kepada:

Romo \romo \vspace{0.5cm}

Saksi-saksi:

Bapak Y. Ngatijan dan Bapak A. Widodo\vspace{0.5cm}

Dekorasi :
Oom Mur, Mas Wahyu, Dik Ning\vspace{0.5cm}

Putra Altar\vspace{0.5cm}

Koor Yustinus\vspace{0.5cm}

Semua pihak yang telah membantu terselenggaranya Perayaan Ekaristi Pemberkatan Pernikahan ini.\vspace{0.5cm}

Segenap kerabat dan handai taulan yang berkenan menghadiri Perayaan Ekaristi Pemberkatan Pernikahan ini.
\end{center}


\end{document}