\documentclass[12pt,twoside,anypage]{scrbook}
\usepackage[a5paper,vmargin={2cm,2.5cm},hmargin={2cm,2cm}]{geometry}
\usepackage{graphicx}
\usepackage{marvosym}
\usepackage{palatino}
\usepackage{fancyhdr}
\usepackage{microtype}
\usepackage{xspace}
\usepackage[bahasa]{babel}
\selectlanguage{bahasa}

\renewcommand{\footrulewidth}{0.5pt}
\lhead[\fancyplain{}{\thepage}]%
      {\fancyplain{}{\rightmark}}
\rhead[\fancyplain{}{\leftmark}]%
      {\fancyplain{}{\thepage}}
\pagestyle{fancy}
\lfoot[\emph{\scriptsize Ulang tahun perkawinan ke-40 \bapak - \ibu}]{}
\rfoot[]{\emph{\scriptsize Pemberkatan rumah Keluarga \keluarga}}
\cfoot{}

\makeatletter
\newcommand{\judul}[1]{%
  {\parindent \z@ \centering \normalfont
    \interlinepenalty\@M \large \bfseries #1\par\nobreak \vskip 20\p@ }}
\newcommand{\subjudul}[1]{%
  {\parindent \z@ \normalfont
    \interlinepenalty\@M \bfseries #1\par\nobreak \vskip 20\p@ }}
\newcommand{\lagu}[1]{%
  {\parindent \z@ \normalfont
    \interlinepenalty\@M \bfseries \emph{#1}\par\nobreak \vskip 20\p@ }}

\renewenvironment{description}
               {\list{}{\labelwidth\z@ \itemindent-\leftmargin
                        \let\makelabel\descriptionlabel}}
               {\endlist}
\renewcommand*\descriptionlabel[1]{\hspace\labelsep 
                                \normalfont\bfseries #1 }
    
\newcommand{\doa}[2]{%
  \begin{description}
  \item[Doa untuk #1] #2
   
   Kami mohon : Kabulkanlah doa kami ya Tuhan.
  \end{description}
}

\newcommand{\bait}[1]{%
  \begin{enumerate}
  \slshape
  \setcounter{enumi}{\value{urut}}
  \item #1
  \setcounter{urut}{\value{enumi}}
  \end{enumerate}	
}
  

  
\makeatother

\newcommand{\BU}[1]{\begin{itemize} \item[U:] #1 \end{itemize}}
\newcommand{\BI}[1]{\begin{itemize} \item[I:] #1 \end{itemize}}
\newcommand{\BP}[1]{\begin{itemize} \item[P:] #1 \end{itemize}}
\newcommand{\Bgen}[2]{\begin{itemize} \item[#1:] #2 \end{itemize}}

\newcommand{\keluarga}{FX Arie Wibowo Sudaryanto dan R Zeli Puspitasari\xspace}
\newcommand{\kelbpk}{FX Arie Wibowo Sudaryanto\xspace}
\newcommand{\kelibu}{R Zeli Puspitasari\xspace}
\newcommand{\ibu}{Anastasia Hedwig Djuwarni\xspace}
\newcommand{\bapak}{Yohanes Sudarmadi\xspace}
\newcommand{\ibup}{Djuwarni\xspace}
\newcommand{\bapakp}{Sudarmadi\xspace}
\newcommand{\romo}{Cletus Nenda, SVD\xspace}


\DeclareFixedFont{\PT}{T1}{ppl}{b}{it}{0.5in}
\DeclareFixedFont{\PTsmall}{T1}{ppl}{b}{it}{0.225in}
\DeclareFixedFont{\PTsmallest}{T1}{ppl}{b}{it}{0.2in}
\DeclareFixedFont{\PTtext}{T1}{ppl}{b}{it}{11pt}
\DeclareFixedFont{\Logo}{T1}{pbk}{m}{n}{0.3in}
\hyphenation{a-kan}
\hyphenation{ba-gi-mu}
\hyphenation{ber-a-da}
\hyphenation{ber-du-a}
\hyphenation{be-ri-kan}
\hyphenation{ber-ka-ta}
\hyphenation{ber-nya-nyi}
\hyphenation{ber-sa-ma}

\hyphenation{dah-syat}
\hyphenation{DA-RAH-KU}
\hyphenation{da-tang}
\hyphenation{di-ka-ta-kan}
\hyphenation{di-pim-pin}
\hyphenation{di-se-rah-kan}
\hyphenation{di-tum-pah-kan}

\hyphenation{Eng-kau}
\hyphenation{ha-dap-an}
\hyphenation{han-tar-kan-lah}
\hyphenation{ha-rap-an}

\hyphenation{ja-lan}
\hyphenation{ja-ngan-lah}

\hyphenation{ka-nak}
\hyphenation{ka-re-na}
\hyphenation{kau-lim-pah-kan}
\hyphenation{Kau-cip-ta-kan}
\hyphenation{ke-bang-kit-an-Nya}
\hyphenation{ke-da-tang-an}
\hyphenation{ke-da-tang-an-Nya}
\hyphenation{ke-dua}
\hyphenation{ke-na-ik-kan-nya}
\hyphenation{ke-pa-daMu}
\hyphenation{ke-ra-him-an}
\hyphenation{ke-se-jah-te-ra-an-mu}
\hyphenation{ko-men-tar}

\hyphenation{la-ma-nya}
\hyphenation{lim-pah-kan}

\hyphenation{ma-nu-sia}
\hyphenation{me-nga-da-kan}
\hyphenation{me-ngan-dung-lah}
\hyphenation{me-ngu-kuh-kan}
\hyphenation{me-la-lui}
\hyphenation{me-lim-pah-kan}
\hyphenation{me-lu-hur-kan}
\hyphenation{me-me-cah-me-cah-kan}
\hyphenation{mem-per-sem-bah-kan}
\hyphenation{me-nan-da-ta-ngan-i}
\hyphenation{men-cin-tai}
\hyphenation{meng-a-lir-kan}
\hyphenation{me-nga-sihi}
\hyphenation{me-nge-lu-ar-kan}
\hyphenation{meng-u-cap-kan}
\hyphenation{meng-ung-kap-kan}
\hyphenation{me-num-buh-kan}
\hyphenation{me-nya-ta-kan}
\hyphenation{me-nye-la-mat-kan}
\hyphenation{me-nye-rah-kan}
\hyphenation{me-nye-rah-kanNya}
\hyphenation{me-ra-ya-kan}

\hyphenation{o-rang}
\hyphenation{o-rang-o-rang}

\hyphenation{pa-sang-kan-lah}
\hyphenation{pa-tut}
\hyphenation{pe-ne-ri-ma-an}
\hyphenation{pe-ngam-pun-an}
\hyphenation{Pe-ngan-ta-ra}
\hyphenation{peng-hi-bur-an}
\hyphenation{per-bu-at-an-nya}
\hyphenation{per-ka-ta-an}
\hyphenation{per-ka-win-an}
\hyphenation{per-ni-kah-an}
\hyphenation{per-se-ku-tu-an}
\hyphenation{per-sem-bah-an}
\hyphenation{rom-bong-an}

\hyphenation{se-la-ma}
\hyphenation{se-ka-li-an}
\hyphenation{se-pan-jang}
\hyphenation{se-ra-ya}
\hyphenation{Su-dar-yan-to}

\hyphenation{te-ta-pi}
\hyphenation{ta-ngan-Mu}
\hyphenation{Tu-han}
\hyphenation{tu-lang}
\hyphenation{tu-lang-tu-lang}

\hyphenation{u-mat-Mu}
\hyphenation{wa-kil}

\hyphenation{ba-gi-mu}
\hyphenation{di-se-rah-kan}
\hyphenation{me-la-lui}
\hyphenation{ka-nak}
\hyphenation{ka-re-na}
\hyphenation{ber-ka-ta}
\hyphenation{te-ta-pi}
\hyphenation{per-ka-win-an}
\hyphenation{pa-tut}
\hyphenation{me-lu-hur-kan}
\hyphenation{ber-nya-nyi}
\hyphenation{di-tum-pah-kan}
\hyphenation{pe-ngam-pun-an}
\hyphenation{ber-a-da}
\hyphenation{kau-lim-pah-kan}
\hyphenation{ke-bang-kit-an-Nya}
\hyphenation{per-ka-ta-an}
\hyphenation{pa-sang-kan-lah}
\hyphenation{DA-RAH-KU}
\hyphenation{ke-na-ik-kan-nya}
\hyphenation{per-sem-bah-an}
\hyphenation{per-se-ku-tu-an}



\begin{document}
\thispagestyle{empty}
\begin{center}
{\PT Ekaristi} 
\vspace{1cm}\\
{\PTsmall ULANG TAHUN PERKAWINAN ke-40}
\vspace{0.5cm}\\
{\PTsmallest \bapak \\dan\\ \ibu}\\
 
\vspace{1.5cm}
{\PTsmallest serta}\\
\vspace{1.5cm}
{\PTsmall PEMBERKATAN RUMAH}
\vspace{0.5cm}\\
{\PTsmallest Keluarga \\ \kelbpk \\dan\\ \kelibu}
 
{~}\vspace{2.5cm}

oleh Romo\\ \romo

{~}\\

9 September 2012

\end{center}

\newpage
\judul{RITUS PEMBUKA}
\lagu{Lagu Pembuka}
\small
\begin{center}
\itshape{Sungai Mengalir}
\end{center}

\begin{verse}
\itshape{
Sungai mengalir tiada henti-hentinya,\\ 
memberi hidup di sekitarnya.\\
Tuhan melimpahkan RahmatNya \\
bagi yang percaya kepadaNya.\\
{~}\\
Bunga-bunga tiada akan mekar mewangi,\\ 
jika tanpa disegarkan air.\\
Hidup akan menjadi hampa, \\
jika tanpa Cinta Kasih Tuhan.\\
{~}\\
Ya Tuhan Allah limpahkan Kasih SayangMu,\\ 
bagaikan air sungai abadi,\\
agar segarlah hidup kami, \\
tiada akan layu selamanya.\\
}
\end{verse}
\normalsize


\subjudul{Salam pembuka}

\BI{Demi Nama Bapa dan Putera dan Roh Kudus}
\BU{Amin.}
\BI{Semoga Allah Bapa serta PuteraNya, Tuhan kita Yesus Kristus, memberikan Kurnia dan Kesejahteraan kepada kita.}
\BU{Sekarang dan selama-lamanya.}

\subjudul{Pengantar}

\BI{Saudara-saudara terkasih, 

}

\subjudul{Tobat}
\BI{Marilah kita hening sejenak untuk mempersiapkan diri dalam perayaan syukur ini sambil menyadari bahwa kita sering melupakan kebaikan Tuhan dan enggan mewartakan dan mewujudkan kebaikan tersebut melalui pikiran, perkataan, dan perbuatan kita.}

\BI{Saya mengaku}

\BU{Kepada Allah yang Maha Kuasa dan kepada saudara sekalian bahwa saya telah berdosa dengan pikiran dan perkataan, dengan perbuatan dan kelalaian. Saya berdosa, saya berdosa, saya sungguh berdosa. Oleh sebab itu saya mohon kepada Santa Perawan Maria, kepada Para Malaikat dan orang kudus dan kepada saudara sekalian, supaya mendoakan saya kepada Allah Tuhan kita.}

\BI{Semoga Allah Yang Maha Kuasa mengasihi kita, mengampuni dosa kita dan menghantar kita ke hidup yang kekal.}

\BU{Amin}

\lagu{Tuhan Kasihanilah Kami}

\subjudul{Doa Pembuka}

\BI{Marilah berdoa

Allah Bapa yang mahakuasa, pandanglah dengan rela suami isteri \bapak
dan \ibu ini. Engkau telah meneguhkan permulaan cinta mereka 40 tahun yang lalu dengan sakramen pernikahan yang mengagumkan. Berkatilah hari-hari mereka selanjutnya demi jasa yang mereka sumbangkan sejak masa muda mereka. 

Berkatilah pula puteri-putera
mereka agar tetap rukun dan bersatu dalam
suka-duka menghadapi perjalanan hidup mereka,
dan berkatilah pula rumah tempat tinggal untuk keluarga \keluarga ,

Dengan pengantaraan Yesus Kristus Putra-Mu,
Tuhan kami,
yang bersama dengan Dikau,
dalam persatuan Roh Kudus
hidup dan berkuasa,
Allah, kini dan sepanjang masa.
}

\judul{LITURGI SABDA}

\subjudul{Bacaan Kitab Suci 1Kor 12:31, 13:1-8a, 13}

Bacaan dari Pembacaan dari surat Paulus yang pertama kepada
jemaat di Korintus

\BP{
\begin{center}\emph{
”Jika aku tidak mempunyai kasih, sedikitpun tidak ada
faedahnya bagiku”}
\end{center}
 
Saudara-Saudari, jadi berusahalah untuk memperoleh
karunia-karunia yang paling utama. Dan aku menunjukkan
kepadamu jalan yang lebih utama lagi.

Sekalipun aku dapat berkata-kata dengan semua
bahasa manusia dan bahasa malaikat, tetapi jika aku tidak
mempunyai kasih, aku sama dengan gong yang
berkumandang dan canang yang bergemerincing.
Sekalipun aku mempunyai karunia untuk bernubuat dan
aku mengetahui segala rahasia dan memiliki seluruh
pengetahuan, dan sekalipun aku memiliki iman yang
sempurna untuk memindahkan gunung, tetapi jika aku
tidak memiliki kasih aku sama sekali tidak berguna.
Dan sekalipun aku membagi-bagikan segala sesuatu
yang ada padaku, bahkan menyerahkan tubuhku untuk
dibakar, tetapi jika aku tidak mempunyai kasih, sedikitpun
tak ada faedahnya bagiku.

Kasih itu sabar, kasih itu murah hati, ia tidak cemburu.
Ia tidak memegahkan diri dan tidak sombong.
Ia tidak melakukan yang tidak sopan dan tidak mencari
keuntungan diri sendiri. Ia tidak pemarah dan tidak
menyimpan kesalahan orang lain.

Ia tidak bersuka cita karena ketidak adilan, tetapi
karena kebenaran.
Ia menutupi segala sesuatu, percaya segala sesuatu,
mengharapkan segala sesuatu, sabar menanggung segala
sesuatu. Kasih tidak berkesudahan;
Demikianlah tinggal ketiga hal ini, yaitu iman,
pengharapan, dan kasih, dimana yang paling besar
diantara ialah Kasih.

Demikian Sabda Tuhan}

\BU{Syukur kepada Allah}

\lagu{Lagu Pengantar Bacaan}
\small
\begin{center}
\itshape{Kasih}
\end{center}
\begin{verse}
\itshape{
Kasih pasti lemah lembut.\\ 
Kasih pasti memaafkan.\\
Kasih pasti murah hati.\\
KasihMu, KasihMu Tuhan\\
{~}\\
Ajarilah kami saling mengasihi.\\
Ajarilah kami saling mengampuni.\\
Ajarilah kami kasihMu ya Tuhan.\\
KasihMu kudus tiada batasnya.
}
\end{verse}
\normalsize





\small
\lagu{BAIT PENGANTAR INJIL}
\BI{Alleluya, Alleluya, Alleluya,}
  
\BI{Jikalau kamu menuruti perintah-Ku, kamu akan tinggal di dalam kasih-Ku, seperti Aku menuruti perintah Bapa-Ku dan tinggal di dalam kasih-Nya.}
\normalsize

\subjudul{Injil}
\BI{Tuhan sertamu,}
\BU{Dan sertamu juga.}
\BI{Inilah Injil Yesus Kristus menurut Yohanes - (5:31--47)}
\BU{Terpujilah Kristus}

\BI{Yesus bersabda kepada murid-muridnya: 
				”Seperti Bapa telah mengasihi Aku, demikianlah Aku telah mengasihi kamu, tinggallah dalam kasihKu itu. Jikalau kamu menuruti perintahKu, kamu akan tinggal di dalam kasihKu, seperti Aku menuruti perintah BapaKu dan tinggal di dalam kasihNya.
				 
				Semuanya ini Kukatakan kepadamu, supaya sukacitaKu ada di dalam kamu dan sukacitamu menjadi penuh. Inilah perintahKu, yaitu supaya kamu saling mengasihi, seperti Aku telah mengasihi kamu. Tidak ada kasih yang lebih besar daripada kasih seorang yang memberikan nyawanya untuk sahabat-sahabatnya. Kamu adalah sahabatKu, jikalau kamu berbuat apa yang Kuperintahkan kepadamu.

				Aku tidak menyebut kamu lagi hamba, sebab hamba tidak tahu apa yang diperbuat oleh tuannya, tetapi aku menyebut kamu sahabat, karena Aku telah memberitahukan kepada kamu segala sesuatu yang telah Aku dengar dari BapaKu. Bukan kamu yang memilih Aku, tetapi Akulah yang memilih kamu. Dan Aku telah menetapkan kamu, supaya kamu pergi dan menghasilkan buah dan buah itu tetap, maka yang aku minta pada Bapa dalam namaKu, diberikannya kepadamu. 
				Inilah perintahKu kepadamu: ” Kasihilah seorang akan yang lain.”

Berbahagialah orang yang mendengarkan sabda Tuhan dan tekun melaksanakannya. }

\BU{Sabda-Mu adalah jalan, kebenaran, dan hidup kami.}


\subjudul{Homili}

\subjudul{PEMBERKATAN \& PEMASANGAN CINCIN}

\BI{Ya Allah, sumber kesetiaan, sudilah memberkati sepasang cincin ini supaya menjadi lambang kesetiaan bagi suami istri ini dan lambang cinta kasihMu yang tak berkesudahan dan kesetiaan yang tiada henti. Demi Kristus, Tuhan dan Pengantara kami}
\BU{Amin.}

\noindent{\emph{(Imam memerciki kedua cincin dengan air suci, kemudian menyerahkan kedua cincin tersebut kepada Kepala Keluarga)}}

\BI{\bapak, kenakanlah cincin ini pada jari isterimu sebagai lambang cinta dan kesetiaan.}

\noindent{\emph{(\bapak mengenakan cincin kepada isterinya)}}

\Bgen{\bapakp}{\ibu, terimalah cincin ini sebagai lambang cinta dari kesetiaanku padamu.}

\noindent{\emph{(Imam menyerahkan cincin kepada \ibu)}}

\BI{\ibu, kenakanlah cincin ini pada jari suamimu sebagai lambang cinta dan kesetiaan.}

\noindent{\emph{(\ibu mengenakan cincin kepada suaminya)}}

\Bgen{\ibup}{\bapak, terimalah cincin ini sebagai lambang cinta dan kesetiaanku padamu.}

\BI{\ibu dan \bapak, semoga kalian senantiasa saling memandang dengan wajah berseri dan penuh cinta kasih. Semoga ikatan kasih ini menjadi sumber kebahagiaan sejati. Demi Kristus, Tuhan, dan Pengantara kami.}

\BU{Amin.}

\subjudul{PEMBAHARUAN JANJI PERKAWINAN}

\BI{Sekarang tiba saatnya Bapak-Ibu \bapakp akan menyatakan doa janji hidup berkeluarga.}

\Bgen{\bapakp}{Allah Bapa sumber segala cinta, pada hari yang bahagia ini, saya ingin membaharui janji perkawinan saya di hadapanMu, di hadapan Imam serta hadirin hadirin sekalian. Saya ingin menegaskan kembali janji yang saya nyatakan 40 tahun yang lalu. Saya akan tetap setia kepada istri saya, \ibu, dalam suka dan duka, dalam untung dan malang, dalam segala keadaan sehat maupun sakit, dan dengan segala kelebihan maupun kekurangannya. Tuhan, berkatilah niat saya ini dan jadikanlah agar seluruh hidup saya menjadi berkat bagi istri saya sekarang sampai selama-lamanya. Amin.}

\Bgen{\ibup}{Allah Bapa sumber segala cinta, pada hari yang bahagia ini, saya ingin memperbaharui janji perkawinan saya di hadapanMu, di hadapan Imam serta hadirin sekalian. Saya ingin menegaskan kembali janji yang saya nyatakan 40 tahun yang lalu. Saya akan tetap setia kepada suami saya, \bapak, dalam suka dan duka, dalam untung dan malang, dalam keadaan sehat maupun sakit, dan dengan segala kelebihan maupun kekurangannya. Tuhan, berkatilah niat saya ini dan jadikanlah seluruh hidup saya menjadi berkat bagi suami saya sekarang sampai selama-lamanya. Amin.}

\Bgen{Pasutri}{Allah Bapa, sumber segala cinta, kami bersyukur karena Engkau telah menganugerahi kami anak-anak. Sebagai tanda syukur, di hadapan Allah, di hadapan Imam dan hadirin sekalian, kami berjanji akan menjadi ayah dan ibu yang baik-baik bagi anak-anak kami semua, menyadari dan menghormati kedewasaan dan kemandirian, dalam ikatan batin orang tua dan anak-anak. Terimalah niat kami dengan cintaMu.  Amin.}

\Bgen{Anak}{Allah Bapa yang Maha Baik, terimakasih karena Engkau telah memberikan kami bapak dan ibu yang sangat memperhatikan dan mencintai kami. Berkatilah bapak dan ibu, berikanlah kelimpahan berkat, rahmat, kasih, kesehatan, panjang umur dan rejeki yang melimpah. Ibu Maria doakanlah kedua orang tua kami selalu.	Amin.}

\subjudul{PEMBERIAN DOA RESTU}

\BI{Bapak Ibu dan Saudara-saudara, sekarang saatnya bapak dan ibu memberikan restu untuk anak cucunya. Semoga, oleh doa restu mereka, para anak dan cucu mendapatkan rahmat seperti yang telah mereka terima selama ini.}

\noindent{\emph{Kemudian, anak-anak beserta cucu-cucu satu per satu datang kepada bapak dan ibu untuk menerima doa restu. Untuk anak dan cucu, doa restu bisa diberikan dengan menandai masing-masing dengan tanda salib di dahinya.}}

\judul{PEMBERKATAN RUMAH}

\BI{Allah Pencipta, Pemelihara, dan Bapa keluarga kami. Pada hari ini kami menyerahkan diri, seluruh keluarga kami serta seisi rumah kami kepada kekuasaanMu yang manis. Sudilah memberkati rumah kami dan semua penghuninya \Cross \xspace 

Perkuatlah kuasa Yesus Kristus PuteraMu atas rumah kami, agar Dia kami-cintai dan taati, supaya di bawah perlindunganNya, kami aman, selamat dan sejahtera. Supaya Kristus menjadi Kepala keluarga kami, menolong kami mengendalikan pikiran, hati dan tingkah-laku kami. Agar kami selalu sadar akan status kami sebagai umat yang telah Kau tebus. Agar kami hidup seperti PuteraMu itu : baik hati, suci, bijaksana, sederhana, rukun, saling sayang-menyayangi, hormat-menghormati, dan tolong-menolong. Berkatilah ya Bapa agar jangan seorang pun dari kami menjauh dari padaMu karena dosa atau kekecewaan. Sebab Engkaulah satu-satunya sumber kehidupan dan kebahagiaan kami, sekarang dan selama-lamanya.
Amin.}


{\itshape Rumah diberkati. Sementara itu umat menyanyikan lagu \emph{DATANGLAH ROH MAHAKUDUS} dan setiap bait diselingi dengan doa.}


\newcounter{urut}
\bait{Datanglah Roh Mahakudus. \\Masuki hati umatMu. \\Sirami jiwa yang layu, \\dengan embun KurniaMu.}

\doa{kamar tamu}{%
Tuhan, berkatilah kamar tamu. Penuhilah ruangan ini dengan kesejukan sejati. Semoga para tamu yang memasuki rumah ini, membawa Berkah, damai dan cinta. Dan semoga mereka pulang dengan senang hati.}

\bait{Roh Cinta Bapa dan Putera.\\ Taburkanlah cinta mesra,\\ dalam hati manusia,\\ Cinta anak pada Bapa.}

\doa{kamar tidur}{
Tuhan, berkatilah kamar tidur. Semoga kamar ini dapat menjadi tempat istirahat yang nyaman, sehingga pulihlah segala kelelahan. Lindungilah penghuninya dari dosa dan bahaya yang mengancam mereka pada waktu mereka tidur.}

\bait{Datanglah Roh Mahakudus.\\ Bentara Cinta Sang Kristus.\\ Tolong kami jadi saksi,\\ membawa Cinta Ilahi.}

\doa{kamar belajar}{
Tuhan, berkatilah kamar belajar dan yang memakainya. Semoga dalam kamar yang Kau berkati ini, anak-anak tekun belajar, guna mempersiapkan masa depan yang cerah.}

\bait{Lidah, api, angin taufan,\\ lambang Roh Kudus yang datang.\\ Muka bumi dibarui \\oleh pembaru yang suci.}

\doa{kamar kerja}{
Tuhan, berkatilah kamar kerja. Semoga segala yang direncanakan dan dikerjakan di tempat ini berkenan kepadaMu dan mendatangkan kesejahteraan bagi seluruh keluarga dan masyarakat.}

\bait{Roh Kristus, ajari kami\\ bahasa cinta ilahi. \\Satulah bangsa semua,\\ karena bahasa cinta.}

\doa{dapur}{
Tuhan, berkatilah dapur. Jauhkanlah dari bahaya kebakaran. Semoga rejeki yang disiapkan di tempat ini sungguh bermanfaat bagi perkembangan jasmani-rohani keluarga ini. Jauhkanlah keluarga ini dari kekurangan dan kelaparan, supaya mereka tetap hidup dengan tenteram dan aman sentausa.}

\bait{Cinta yang laksana api,\\ kobarkan semangat kami,\\ agar musnahlah terbasmi\\ jiwa angkuh, hati dengki.}

\doa{sumur dan kamar mandi}{
Tuhan, berkatilah sumur dan kamar mandi. Bukalah selalu sumber-sumber air, agar bagi keluarga ini selalu tersedia air bersih secukupnya. Semoga setiap mengalami kesejukan dan kesegaran daripadanya, mereka selalu ingat akan Dikau, Sumber Kehidupan yang tidak pernah surut.}

\bait{Sang Penghibur umat Allah, \\kuatkan iman yang lemah,\\ agar hati bergembira \\walau dilanda derita.}

\doa{seluruh dan sekeliling rumah}{
Tuhan, berkatilah seluruh rumah ini, karena di dalamnya berdiam putera-puteri kesayanganMu. Jauhkanlah keluarga ini dari segala yang dapat merusak dan mengganggu ketenangan dan ketenteraman. Berilah mereka rejeki melimpah. Jagalah agar rumah ini tetap kokoh berdiri, sehingga dapat menjadi naungan yang aman bagi seluruh penghuninya.}

\bait{Penggerak para rasulMu,\\ lepaskan lidah yang kelu,\\ supaya kami wartakan\\ Karya Keslamatan Tuhan.\\ AMIN.}

\subjudul{Pemberkatan Salib}
\noindent{\emph{(Salib adalah gambaran Iman kita. Kayu vertikal (dari atas ke bawah) melambangkan hubungan mesra Allah-manusia. Kayu horisontal (dari kiri ke kanan): hubungan mesra kita dengan sesama. Tubuh Kristus adalah kurban Diri-Allah demi keselamatan umatNya, dan kesetiaan manusia sampai mati kepada kehendak Allah. Dengan pemasangan salib ini di kamar tamu, kita memaklumkan kepada seluruh dunia, bahwa rumah ini percaya kepada kekuasaan Allah yang tidak dapat dibatalkan, dan bahwa rumah ini telah menyerahkan diri untuk dikuasai dan diatur oleh Allah. Seringlah memandangNya, memberi hormat dan berdoa dalam hati, agar kita dikuatkan, diberi ketenteraman dan semangat.)}}


\BI{Marilah berdoa.

Tuhan, berkatilah salib ini \Cross {~}curahkanlah Roh KudusMu padanya, dan usirlah Kuasa Kegelapan dari padanya, agar salib ini menjadi tanda kehadiranMu di tengah keluarga \keluarga , supaya barangsiapa memandangnya dan diberkati dengannya, memperoleh kekuatan Iman-Pengharapan-dan-Cinta-Kasih dari padaMu dan Kau selamatkan, demi Kristus pengantara kami.}
\BU{Amin}
\BI{(\textit{Kepada Kepala Keluarga})\\
Terimalah salib ini dan pasanglah di rumahmu sebagai tanda Imanmu.}
\BU{Syukur kepada Allah}

(\textit{Lalu Kepala Keluarga memasang salib tersebut di ruang tamu}).

\subjudul{Doa Umat}
\BI{Allah Bapa yang Mahakuasa, berkat dan rahmat-Mu senantiasa Kau limpahkan kepada kami, khususnya keluarga ini. Engkaulah tumpuan dan harapan kami. Untuk itu perkenankanlah kami berdoa:}

\BP{Semoga mereka menghayati hidup perkawinan dalam cinta kasih dan damai, sehingga rahmat dan kebaikan Kristus bersinar dari rumah tangga mereka. 

Kami mohon...}

\BU{Kabulkanlah doa kami, ya Tuhan}

\BP{Semoga cinta kasih mereka diberkati oleh Tuhan dengan kurnia yang berlimpah, sehingga anak, cucu, dan buyut yang dianugerahkan kepada mereka sungguh-sungguh menggembirakan hati orang tuanya.

Kami mohon...}

\BU{Kabulkanlah doa kami, ya Tuhan.}

\BP{Semoga mereka tetap sehat walafiat dan sanggup menjalankan tugasnya dalam masyarakat, sehingga mereka berjasa bagi sesama, dan keluarga mereka aman sentosa. 

Kami mohon...}

\BU{Kabulkanlah doa kami, ya Tuhan.}

\BP{Ya Tuhan, pelindung dan penyelamat kami, tunjukkanlah belas kasihMu kepada kami semua dan limpahilah keluarga-keluarga kami dengan kurniaMu.

Kami mohon...}

\BU{Kabulkanlah doa kami, ya Tuhan.}

\BP{Ya Tuhan, gembala dan penghibur umat, bahagiakanlah arwah nenek moyang keluarga ini dan terimalah mereka dalam perjamuan nikah PuteraMu.

Kami mohon...}

\BU{Kabulkanlah doa kami, ya Tuhan.}

\BP{Ya Bapa, keluarga \keluarga mengucapkan syukur atas tempat tinggal ini terlebih atas diterimanya sebagai anggota dalam keluarga besar lingkungan St. Petrus. Sucikan rumah ini ya Tuhan agar bersih dan bebas dari segala marabahaya, serta gangguan, baik yang kelihatan maupun tidak kelihatan, sehingga di tempat tinggal ini hanya Engkaulah yang meraja yang senantiasa memberi ketentraman dan kedamaian bagi penghuninya.  

Marilah kita mohon \ldots}

\BU{Kabulkanlah doa kami, ya Tuhan}

\BP{Ya Bapa, kami berdoa bagi umat lingkungan St. Petrus, serta bagi sanak saudara yang hadir dalam perayaan Ekaristi untuk memberikan dukungan pada upacara mitoni serta pemberkatan rumah ini. Berkatilah mereka ya Tuhan berserta keluarganya dengan karunia kesehatan, kedamaian dan kesejahteraan.

Marilah kita mohon \ldots}

\BU{Kabulkanlah doa kami, ya Tuhan}

\BI{Bapa Yang Maha Pemurah, itulah doa-doa yang kami panjatkan ke hadirat-Mu. Semoga rahmat dan karunia yang telah kami terima dan senantiasa akan Engkau alirkan semakin membesarkan Nama-Mu dan semakin menjadi kesaksian iman dalam hidup kami. Ya Bapa, sudilah kiranya Engkau mengabulkan doa-doa kami, demi Kristus, Tuhan dan pengantara kami.}

\BU{Amin.}

\judul{LITURGI EKARISTI}

\lagu{Lagu persembahan}
\begin{center}
\itshape{Ku Persembahkan}
\end{center}

\small
\begin{verse}
\itshape{
Ku persembahkan bagi-Mu Yesus\\
yang terbaik dan terindah.\\
Segala apa yang kumiliki\\
itu semua anugrah-Mu.\\
{~}\\
Ref:\\
Terimalah Tuhan persembahan ini \\
dengan tulus hati dan dengan hati bersyukur\\ 
terimalah Tuhan persembahan ini\\ 
segala puji hormat bagimu.\\
}
\end{verse}
\normalsize



\subjudul{Doa Syukur Agung}

\subjudul{Bapa Kami}

\lagu{Lagu Komuni}

\judul{RITUS PENUTUP}

\BI{Marilah berdoa,\\
Allah, Bapa kami. Engkau menghidupkan kami dan menghendaki agar kami hidup berbahagia untuk {sela-ma}-{lamanya}. Kami mengucap syukur atas segala berkat yang kami terima sekeluarga. Terlebih kami bersyukur atas PuteraMu Yesus Kristus yang Kau berikan kepada kami sebagai kekuatan untuk hidup sekarang, dan jaminan untuk hidup surgawi. Semoga dengan berlindung di bawah pimpinanNya dan meneladan jalan hidupNya, kami sampai ke rumahMu dan berbahagia abadi bersama dengan Dikau, dalam persatuan dengan PuteraMu dan Roh Kudus, sepanjang segala masa,}
\BU{Amin.}

\subjudul{Berkat}

\BI{Tuhan sertamu}
\BU{Dan sertamu juga}
\BI{Semoga saudara sekalian diberkati oleh Allah yang mahakuasa :
\Cross {~}Bapa, dan Putera, dan Roh Kudus,}
\BU{Amin.}
\BI{Saudara sekalian, ibadat ulang tahun perkawinan dan pemberkatan rumah sudah selesai.
Marilah kita memuji Tuhan.}
\BU{Syukur kepada Allah.}
\BI{Marilah pergi, kita diutus}
\BU{Amin.}

\lagu{Penutup}

\small
\itshape{
\begin{center}
NDHEREK DEWI MARIYAH\end{center} 
\begin{verse}
Ndherek Dewi Mariyah temtu geng kang manah \\
Mboten yen kuwatosa Ibu njangkung tansah \\
Kanjeng Ratu ing swarga amba sumarah samya \\
Sang Dewi Sang Dewi Mangestonana \\
Sang Dewi Sang Dewi Mangestonana 
\end{verse}
\begin{verse}
Nadyan manah getera dipun godha setan\\ 
Nanging batos engetna wonten pitulungan\\ 
Wit Sang Putri Mariyah mangsa tega anilar\\ 
Sang Dewi Sang Dewi Mangestonana \\
Sang Dewi Sang Dewi Mangestonana 
\end{verse}
}
\normalsize

%\newpage
\hrule
{~}\\[-0.45cm]\hrule

\begin{flushright}
{\Large Ucapan terima kasih}

\noindent Dengan penuh syukur dalam kasih Tuhan, kami mengucapkan banyak
terima kasih kepada:

\textbf{Romo \romo}\\
yang telah berkenan mempimpin perayaan ekaristi pada malam hari ini.

\textbf{Umat lingkungan Santo Petrus Maguwo\\ dan tamu undangan}\\
yang telah mendukung perayaan ekaristi ini.

\textbf{Segenap keluarga dan orang-orang terkasih}\\
yang telah berkenan hadir memberikan cinta dan doa dalam perayaan
ekaristi ini.

Semoga Tuhan memberkati dan memelihara ikatan kasih\\ di antara kita semua.

Amin.

\bigskip 

\ibu dan \bapak\\
\keluarga\\
dan segenap keluarga
\end{flushright}


\end{document}