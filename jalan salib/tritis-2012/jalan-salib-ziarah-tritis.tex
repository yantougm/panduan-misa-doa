\documentclass[a5paper,titlepage,11pt,openany]{scrbook}
\usepackage[a5paper,backref]{hyperref}
\usepackage[papersize={148.5mm,215mm},twoside,bindingoffset=0.5cm,hmargin={1cm,1cm},
vmargin={2cm,2cm},footskip=1.1cm,driver=dvipdfm]{geometry}
\usepackage{palatino}
\usepackage[utf8]{inputenc}

\usepackage{pstricks}
\usepackage{graphicx}
\usepackage[bahasa]{babel} 
\usepackage{lettrine}
\usepackage{pifont}
\usepackage{enumitem}
\usepackage{wrapfig}
\usepackage{indentfirst}
\usepackage{parcolumns}
\usepackage[titles]{tocloft}
\usepackage{longtable}
\usepackage{microtype}
\usepackage{hyphenat}


\renewcommand{\cftchapfont}{%
  \fontsize{9}{8}\selectfont
}

\makeatletter
\renewcommand{\@pnumwidth}{1em} 
\renewcommand{\@tocrmarg}{1em}
\makeatother

\author{Lingkungan St. Petrus Maguwo}
\title{Warta Iman}
\setlength{\parindent}{1cm}
\psset{unit=1mm}
 
\newcommand{\BU}[1]{\begin{itemize} \item[U:] #1 \end{itemize}}
\newcommand{\BI}[1]{\begin{itemize} \item[I:] #1 \end{itemize}}
\newcommand{\BP}[1]{\begin{itemize} \item[P:] #1 \end{itemize}}
\newcommand{\kamiMenyembah}{\BP{ Kami menyembah Dikau ya Tuhan dan bersyukur kepadaMu.}
\BU{Sebab dengan salib suciMu, Engkau telah menebus dunia.}
}
\newcommand{\kasihanilahKami}{\BP{Kasihanilah kami ya Tuhan, kasihanilah kami.}
\BU{Allah ampunilah kami orang berdosa.}}

\newcommand{\BPi}[2]
{\begin{itemize} \item[P1:] \textbf{\emph{#1}}\\#2 \end{itemize}}

\newcommand{\BPii}[1]
{\begin{itemize} \item[P2:] \textbf{\emph{MEDITASI}}\\#1 \end{itemize}}

\newcommand{\lagu}[2]{%
\begin{itemize}
\item[#1.] \it{#2}
\end{itemize}}

\newcommand{\henti}[2]{%
\subsection*{Perhentian #1\\#2 } 
\kamiMenyembah
}

\begin{document}
\begin{center}
\textbf{Doa jalan salib pada\\
Ziarah ke Goa Maria Tritis 2012\\
Lingkungan St Petrus Stasi Maguwo}
\end{center}



\textbf{DOA PEMBUKAAN}

\BP{Dalam nama Bapa, dan Putra, dan Roh Kudus. }
\BU{Amin }



\BP{Tuhan Yesus SengsaraMu adalah keseluruhan dari sejarah manusia: sebuah sejarah di mana yang baik dihina, yang lembut diserang, yang jujur dihancurkan, dan yang murni hati diejek-ejek. Siapa yang akan menjadi pemenang? Siapa yang memiliki kata terakhir? Tuhan Yesus. Kami percaya bahwa Engkau adalah kata terakhir: Dalam Engkau yang baik telah menang, dalam Engkau yang lembut telah berjaya, dalam Engkau yang jujur telah menerima mahkota mereka, dan yang murni hati bersinar seperti bintang-bintang di malam hari. Tuhan Yesus, hari ini kami menjalankan sekali lagi perjalanan dari salibMu, Menyadari bahwa itu adalah jalan kami juga. Satu kepastian menyinari jalan kami: jalan itu tidak berakhir pada salib namun berlanjut terus, menuju Kerajaan Kehidupan, menuju sebuah aliran deras akan kegembiraan, kegembiraan yang tiada seorang pun mampu merampasnya dari kami!}

Yoh 16:22: Mat 5:12.

\BU{Oh Yesus, kami berdiri dalam penderitaan di kaki salibMu: kami sendiri telah membantu menegakkannya dengan dosa-dosa kami! Kebaikanmu yang tidak menawarkan perlawanan, dan membiarkannya sendiri untuk disalibkan, adalah sebuah misteri di luar genggaman kami; ia menyisakan kegelisahan yang amat mendalam. Tuhan, Engkau datang ke dunia bagi kami, untuk mencari dan menunjukkan kepada kami rangkulan penuh kasih dari Bapa:}

Luk 15:20  rangkulan yang amat kami nantikan! Engkau adalah Wajah yang sejati dari keindahan dan dari belas kasih: itulah mengapa Engkau ingin menyelamatkan kami! Dalam diri kami ada penuh keegoisan: datanglah kepada kami dengan kasihMu yang membebaskan! Dalam diri kami ada keseombongan dan kebencian: datanglah kepada kami dengan kelembutan dan kerendahan hatiMu! Tuhan, kami adalah pendosa yang perlu diselamatkan: kami adalah anak pemboros yang perlu kembali! Tuhan, berikanlah kepada kami hadiah air mata! Sehingga kami boleh menemukan sekali lagi kebebasan dan kehidupan, kedamaian denganMu, dan kegembiraan di dalam Engkau, Amin. 





\lagu{1}{Marilah kita renungkan,\\
Yesus yang menjadi kurban,\\
kar'na cinta kasihNya}

\henti{I}{Yesus dihukum mati}

\BPi{Injil Matius 27:22-23,26 }{
Kata Pilatus kepada mereka: Jika begitu, apakah yang harus kuperbuat dengan Yesus, yang disebut Kristus?'' Mereka semua berseru: ``Ia harus disalibkan!'' Katanya: ``Tetapi kejahatan apakah yang telah dilakukan-Nya?'' Namun mereka makin keras berteriak: `` Ia harus disalibkan!'' Lalu ia membebaskan Barabas bagi mereka, tetapi Yesus disesahnya lalu diserahkannya untuk disalibkan. }

\BPii{
Kami mengetahui adegan dari penghukuman ini terlalu baik: kami melihatnya dipermainkan setiap hari! Namun satu pertanyaan menggelisahkan hati kami: mengapa Allah membiarkan diriNya sendiri untuk dihukum? Mengapa Allah, Maha Kuasa, menunjukkan diriNya ditutupi dengan kelemahan? Mengapa Allah membiarkan diriNya diserang oleh kesombongan, angkara dan keangkuhan manusia? Mengapa Allah tetap diam? Sikap diam Allah menyakiti kita? itu adalah percobaan dan godaan kita! Namun itu juga yang memurnikan penilaian-penilaian kita yang tergesa-gesa, dan menyembuhkan kehausan kita untuk balas dendam. Sikap diam Allah adalah tanah di mana kesombongan kita mati dan iman sejati bersemi, sebuah iman yang rendah hati, sebuah iman yang tidak menantang Allah, namun yang berserah kepadaNya dengan kepercayaan seorang anak. }


\textbf{DOA }


\BU{Tuhan, betapa mudahnya untuk menghukum! betapa mudahnya untuk melemparkan batu: batu dari penilaian dan fitnah, batu dari ketidakacuhan dan pengabaian! Tuhan, Engkau memilih untuk berdiri di sisi yang kalah, di sisi yang tercela dan dihukum!

Bantulah kami untuk tidak pernah menyebabkan perasaan sakit untuk saudara-saudari kami yang tak berdaya. Bantulah kami untuk mengambil keputusan yang berani dalam membela yang lemah. Bantulah kami untuk menolak air Pilatus, yang tidak membersihkan tangan kami namun yang menodainya dengan darah yang tak berdosa,Amin.}


\kasihanilahKami

\lagu{2}{Sri Yesus Penebus kami,\\
dijatuhi hukum mati,\\
agar umatNya hidup}





\henti{II}{Yesus memanggul SalibNya}

\BPi{Injil Matius 27:27-31}{%
Kemudian serdadu-serdadu wali negeri membawa Yesus ke gedung pengadilan, lalu memanggil seluruh pasukan berkumpul sekeliling Yesus. Mereka menanggalkan pakaian-Nya dan mengenakan jubah ungu kepada-Nya. Mereka menganyam sebuah mahkota duri dan menaruhnya di atas kepala-Nya, lalu memberikan Dia sebatang buluh di tangan kanan-Nya. Kemudian mereka berlutut di hadapan-Nya dan mengolok-olokkan Dia, katanya: ``Salam, hai Raja orang Yahudi!'' Mereka meludahi-Nya dan mengambil buluh itu dan memukulkannya ke kepala-Nya. Sesudah mengolok-olokkan Dia mereka menanggalkan jubah itu dari pada-Nya dan mengenakan pula pakaian-Nya kepada-Nya. Kemudian mereka membawa Dia ke luar untuk disalibkan.}

\BPii{
Di dalam Sengsara Kristus, kebencian dilepaskan: kebencian kita sendiri, dan kebencian semua manusia. Di dalam Sengsara Kristus, kejahatan kita mengecut dihadapan kebaikan, kesombongan kita meletus dengan kesakitan di dalam wajah kerendahan hati, kebejatan moral kita dihina dengan kemurnian Allah yang bersinar-sinar. Dan oleh karenanya kita menjadi ... salib Allah! Kita, dalam pemberontakkan kita yang bodoh, dengan kebodohan dosa-dosa kita, telah membuat sebuah salib dari kegelisahan kita sendiri dan ketidakbahagiaan kita sendiri: kita merancang hukuman kita sendiri. Namun Allah memanggul salib dibahuNya, salib kita, dan Ia menghadapi kita dengan kekuatan KasihNya. Allah memanggul salib! Misteri kebaikan yang tak terduga! Misteri kerendahan hati, yang membuat kita malu pada kekerasan kesombongan kita.}


\textbf{DOA }


\BU{Tuhan Yesus, Engkau memasuki sejarah manusia dan menemukannya memusuhiMu,  menantang Allah, yang tak waras oleh keseombongan yang membuat kami berpikir bahwa kami berdiri setinggi bayangan kami! Tuhan Yesus, Engkau tidak melawan kami, namun membiarkan diriMu diserang oleh manusia, oleh kami, oleh setiap orang! Tuhan Yesus Sembuhkanlah kami dengan kesabaranMu, sembuhkanlah kami dengan kerendahan hatiMu, turunkanlah kami seukuran yang benar, ukuran seorang makhluk, seorang makhluk kecil ... namun yang menjadi obyek dari KasihMu yang abadi,Amin }

\kasihanilahKami

\lagu{3}{Salib berat dipanggulNya,\\
agar kita ikutiNya,\\
memikul salib kita}

\henti{III}{Yesus jatuh pertama kali}

\BPi{Kitab Nabi Yesaya 53:4-6 }{Tetapi sesungguhnya, penyakit kitalah yang ditanggungnya dan kesengsaraan kita yang dipikulnya; padahal kita mengira dia kena tulah, dipukul dan ditindas Allah. Tetapi dia tertikam oleh karena pemberontakan kita, diremukkan oleh karena kejahatan kita; dia ganjaran yang mendatangkan keselamatan bagi kita ditimpakan kepadanya, dan oleh bilur-bilurnya kita menjadi sembuh. Kita sekalian sesat seperti domba; masing-masing kita mengambil jalannya sendiri, tetapi TUHAN telah menimpakan kepadanya kejahatan kita sekalian.}





\BPii{
Dalam cara pikiran manusiawi kita, Allah tidak dapat jatuh, ... namun Ia jatuh. Mengapa? Itu bukan menjadi sebuah tanda kelemahan, melainkan hanya sebuah tanda Kasih: sebuah pesan akan Kasih bagi kita. Jatuh di bawah salib yang berat, Yesus mengingatkan kita bahwa dosa adalah sebuah beban yang berat, dosa merendahkan kita dan menghancurkan kita, dosa menghukum kita dan membawa kepada kita yang jahat: dengan kata lain, dosa adalah jahat!  Namun demikian Allah masih mencintai kita dan menginginkan yang baik bagi kita; cintaNya membuatNya berteriak kepada yang tuli, kepada kita, yang tidak bersedia mendengarkan: ``Tinggalkan dosa, sebab ia menyakitimu. Ia mengambil kedamaianmu, kegembiraanmu daripadamu; ia mematikanmu dari kehidupan, dan mengeringkan di dalam dirimu sumber sejati dari kebebasan dan martabatmu''. Tinggalkanlah! Tinggalkanlah!}

\textbf{DOA}

\BU{Tuhan, kami telah kehilangan perasaan berdosa kami! Saat ini sebuah kampanye propaganda yang licik sedang menyebarkan sebuah pembenaran bodoh akan kejahatan, sebuah pemujaan kepada Setan yang sia-sia, sebuah keinginan melanggar hukum yang bodoh, sebuah kebebasan yang tidak jujur dan sembrono, mengagungkan tindakan seenak diri, tak bermoral dan egois layaknya semua itu adalah puncak baru dari gaya hidup. Tuhan Yesus, bukalah mata kami: biarkan kami melihat kekotoran di sekitar kami dan mengenalinya apa adanya, sehingga sebuah air mata kepedihan dapat mengembalikan kami kepada kemurnian hati dan kebebasan sejati yang meluas. Bukalah mata kami, Tuhan Yesus, Amin}

\kasihanilahKami

\lagu{4}{Sri Yesus tolonglah kami,\\
bila kami jatuh lagi,\\
tertindih salib berat.}





\henti{IV}{Yesus berjumpa dengan IbuNya}

\BPi{Injil Lukas 2:34-35,51}{%
Lalu Simeon memberkati mereka dan berkata kepada Maria, ibu Anak itu: \\``Sesungguhnya Anak ini ditentukan untuk menjatuhkan atau membangkitkan banyak orang di Israel dan untuk menjadi suatu tanda yang menimbulkan perbantahan dan suatu pedang akan menembus jiwamu sendiri, supaya menjadi nyata pikiran hati banyak orang.'' Lalu Ia pulang bersama-sama mereka ke Nazaret; dan Ia tetap hidup dalam asuhan mereka. Dan ibu-Nya menyimpan semua perkara itu di dalam hatinya. }

\BPii{
Setiap ibu adalah cinta yang dibuat nyata, sebuah harapan akan kasih sayang yang lembut dan kesetiaan yang tak akan mati. Karena seorang ibu sejati mencintai, bahkan saat ia tidak dibalas dicintai. Maria adalah sang Ibu! Dalam dia, kewanitaan murni, dan cinta tidak diracuni oleh gelombang keegoisan yang menyesakkan dan menekan hati manusia. Maria adalah sang Ibu! Hatinya dengan setia mendampingi hati Putranya, merasakan penderitaanNya, membawa salibNya, dan dirinya merasakan rasa sakit dari setiap luka yang diderita pada tubuh Putranya. Maria adalah sang Ibu! Ia tetap menjadi seorang Ibu, bagi kita, selama-lamanya! }

\textbf{DOA}

\BU{Tuhan Yesus, kami semua membutuhkan seorang Ibu! Kami membutuhkan sebuah kasih yang setia dan sejati. Kami membutuhkan sebuah kasih yang tak pernah bimbang, sebuah kasih yang menjadi tempat berlindung yang pasti pada saat-saat ketakutkan, pada saat-saat penderitaan dan percobaan. Tuhan Yesus, kami membutuhkan para wanita: istri-istri dan ibu-ibu yang dapat memulihkan dunia kami wajah bijaksana dari kemanusiaan. Tuhan Yesus, kami membutuhkan Maria: seorang wanita, istri dan ibu, yang tak pernah merendahkan atau menolak cinta! Tuhan Yesus kami berdoa kepadaMu untuk semua wanita di dunia ini, Amin.}


\kasihanilahKami

\lagu{5}{Maria selalu setia,\\
pada Sang Kristus Putranya,\\
dalam suka dan duka.}

\henti{V}{Yesus ditolong oleh Simon dari Kirene memanggul SalibNya}

\BPi{Injil Matius 27:32; 16:26 }{
	Ketika mereka berjalan ke luar kota, mereka berjumpa dengan seorang dari Kirene yang bernama Simon. Orang itu mereka paksa untuk memikul salib Yesus. Yesus berkata kepada murid-murid-Nya: ``Setiap orang yang mau mengikuti Aku, ia harus menyangkal dirinya, memikul salibnya dan mengikut Aku. }





\BPii{
	Simon dari Kirene, engkau adalah salah satu dari yang kecil, yang miskin, seorang tanpa nama dari pedesaan, seseorang yang diabaikan oleh buku-buku sejarah. Namun demikian engkau membuat sejarah! Engkau menulis salah satu bab-bab terindah dalam sejarah manusia: engkau membawa salib dari Orang Lain, engkau memanggul salib itu, dan mencegahnya merusak si Korban. Engkau memulihkan martabat kami semua, dengan mengingatkan kami bahwa kami sungguh-sungguh menjadi diri kami sendiri apabila kami hanyai memikirkan tentang diri kami sendiri saja.  Engkau mengingatkan kami bahwa Kristus sedang menunggu kami di jalanan, di daratan, di rumah-rumah sakit, di penjara, di kolong jembatan dan di pinggiran kota-kota kami. Kristus menunggu kami!  Akankah kami mengenaliNya? Akankah kami menolongNya? Atau akankah kami mati dalam keegoisan kami? }

\textbf{DOA}

\BU{Tuhan Yesus, kasih sedang memudar, dan dunia kami menjadi dingin, tidak ramah, tak tertahankan. Remukkanlah rantai-rantai yang menahan kami dalam menjangkau orang lain. Tolonglah kami, melalui kasih, untuk menemukan diri kami sendiri. Tuhan Yesus, kemakmuran kami membuat kami menjadi kurang manusiawi, hiburan kami telah menjadi sebuah obat bius, sebuah sumber pengasingan, dan pesan dari masyarakat kami yang tak henti dan membosankan adalah sebuah undangan untuk mati dalam keegoisan. Tuhan Yesus, perbaharuliah dalam diri kami percikan kemanusiaan yang Allah tempatkan dalam hati kami pada awal penciptaan. Bebaskanlah kami dari cinta pada diri kami sendiri yang merosot, dan kami akan menemukan kegembiraan baru dalam hidup dan masuk ke dalam nyanyian riang gembira, Amin.}

\kasihanilahKami

\lagu{6}{Cinta bakti pada Tuhan,\\
hanya dapat dibuktikan,\\
dengan saling mengabdi.}

\henti{VI}{Veronika mengusap wajah Yesus}

\BPi{Kitab Yesaya 53:2-3 }{%
Ia tidak tampan dan semaraknyapun tidak ada sehingga kita tidak memandang dia, dan rupapun tidak, sehingga kita tidak menginginkannya. Ia dihina dan dihindari orang, seorang yang penuh kesengsaraan dan yang biasa menderita kesakitan; ia sangat dihina, sehingga orang menutup mukanya terhadap dia dan bagi kitapun dia tidak masuk hitungan. Seperti rusa yang merindukan sungai yang berair, demikianlah jiwaku merindukan Engkau, ya Allah. Jiwaku haus kepada Allah, kepada Allah yang hidup.}

\BPii{
Wajah Yesus bermandikan peluh, mengalirkan darah, ditutupi dengan air liur penghinaan. Siapa yang berani mendekatiNya? Seorang wanita! Seorang wanita keluar dari kerumunan, memelihara dengan ceria lampu kemanusiaan kita, ... dan mengusap WajahNya dan menemukan Wajahnya! Betapa banyak orang saat ini tidak memiliki wajah! Betapa banyak orang terbuang ke dalam batas kehidupan, terasingkan, terabaikan, oleh sebuah kelesuan yang embunuh yang tak acuh. Hanya mereka yang terbakar oleh kasih yang sungguh-sungguh hidup, mereka yang membungkuk rendah di hadapan Kristus yang menderita dan yang menanti kita dalam diri mereka yang menderita: hari ini! Hari ini! Karena esok akan terlambat!}

\textbf{DOA}

\BU{Tuhan Yesus, sebuah langkah tunggal dan dunia dapat berubah! Sebuah langkah tunggal, dan kedamaian dapat kembali kepada keluarga-keluarga, sebuah langkah tunggal, dan yang miskin tidak akan lagi sendirian; sebuah langkah tunggal, dan penderitaan dapat mersakan sebuah tangan yang menggapai tangan mereka ... dan membawa kesembuhan bagi keduanya. Sebuah langkah tunggal, dan yang miskin dapat menemukan sebuah tempat di meja makan mengangkat kesedihan yang menghantui meja-meja makan dari yang egois, yang tak menemukan kegembiraan dalam berpesta sendiri. Tuhan Yesus, sebuah langkah tunggal adalah semua yang diperlukan! Bantulah kami untuk mengambil langkah itu, karena dunia kami perlahan-lahan mengosongkan semua persediaan kegembiraannya. Bantulah kami, Tuhan, Amin.}

\kasihanilahKami

\lagu{7}{Lipuran yang meringankan,\\
duka orang yang tertekan,\\
menghibur Kristus juga.}

\henti{VII}{Yesus jatuh kedua kalinya}

\BPi{	Kitab Nabi Yeremia 12,1 }{
	Engkau memang benar, ya TUHAN, bilamana aku berbantah dengan Engkau! Tetapi aku mau berbicara dengan Engkau tentang keadilan: Mengapakah mujur hidup orang-orang fasik? Mengapakah sentosa semua orang yang berlaku tidak setia? \\
\textbf{\emph{Kitab Mazmur 37:1-2, 10-11 }}\\
	Jangan marah karena orang yang berbuat jahat, jangan iri hati kepada orang yang berbuat curang! sebab mereka segera lisut seperti rumput dan layu seperti tumbuh-tumbuhan hijau. Karena sedikit waktu lagi, maka lenyaplah orang fasik; jika engkau memperhatikan tempatnya, maka ia sudah tidak ada lagi, Tetapi orang-orang yang rendah hati akan mewarisi negeri, dan bergembira karena kesejahteraan yang berlimpah-limpah. }

\BPii{
	Kesombongan kita, kekerasan kita, ketidakadilan kita semua membebani tubuh Kristus. Mereka memberatiNya dan Ia jatuh kedua kalinya, untuk menunjukkan kepada kita beban yang tak tertanggung dari dosa-dosa kita. Namun apakah yang saat ini, khususnya, memukul tubuh suci Kristus? Tentunya Allah sangat tersiksa oleh serangan terhadap keluarga. Saat ini kita seperti menyaksikan sejenis anti-Kejadian, sebuah rencana-balasan, sebuah kesombongan diabolik yang ditujukan untuk menghapuskan keluarga. Terdapat sebuah gerakan untuk menciptakan ulang manusia, untuk memodifikasi tata bahasa sejati dari kehidupan seperti yang direncanakan dan dikehendaki oleh Allah. \\
	Namun, untuk mengambil posisi allah, tanpa menjadi Allah, adalah sebuah kesombongan gila, sebuah spekulasi yang beresiko dan berbahaya. Semoga jatuhnya Kristus membuka mata kita untuk melihat sekali lagi wajah indah, wajah sejati, wajah suci dari keluarga. Wajah dari keluarga yang kita semua butuhkan.}

\textbf{DOA}

\BU{Tuhan Yesus, keluarga adalah salah satu impian Allah yang dipercayakan kepada manusia; keluarga adalah sebuah percikan dari Surga yang dibagikan kepada semua manusia: keluarga adalah buaian di mana kami dilahirkan dan yang kami terus-menerus dilahirkan kembali dalam cinta. Tuhan Yesus, masuklah ke dalam rumah-rumah kami dan pimpinlah kami dalam nyanyian kehidupan. Perbaharuilah cahaya cinta dan buatlah kami merasakan keindahan menjadi terikat satu dengan yang lainnya dalam sebuah rangkulan kehidupan: sebuah kehidupan yang dihangatkan oleh nafas Allah sendiri, nafas dari Allah yang adalah Cinta. Tuhan Yesus, selamatkanlah keluarga, dan selamatkanlah hidup itu sendiri! Tuhan Yesus, selamatkanlah keluargaku sendiri, selamatkanlah keluarga-keluarga kami, Amin.}


\kasihanilahKami

\lagu{8}{Bilamana kami lemah,\\
jatuh tercampak di tanah, \\
tegakkan kami lagi.}

\henti{VIII}{Yesus menasehati para wanita yang menangisinya}

\BPi{	Injil Lukas 23:27-29, 31 }{

	Sejumlah besar orang mengikuti Dia; di antaranya banyak perempuan yang menangisi dan meratapi Dia. Yesus berpaling kepada mereka dan berkata: ''Hai puteri-puteri Yerusalem, janganlah kamu menangisi Aku, melainkan tangisilah dirimu sendiri dan anak-anakmu! Sebab lihat, akan tiba masanya orang berkata: Berbahagialah perempuan mandul dan yang rahimnya tidak pernah melahirkan, dan yang susunya tidak pernah menyusui. Sebab jikalau orang berbuat demikian dengan kayu hidup, apakah yang akan terjadi dengan kayu kering?`` }

\BPii{
	Air mata dari para ibu di Yerusalem membanjiri dengan rasa kasihan jalan yang ditempuh oleh Terhukum, melembutkan keganasan dari sebuah eksekusi dan mengingatkan kita bahwa kita semua adalah anak-anak: anak-anak yang keluar dari rangkulan seorang ibu. Namun air mata dari para ibu di Yerusalem adalah hanya sebuah tetesan kecil di sungai air mata yang diteteskan oleh para ibu: para ibu dari yang tersalib, para ibu dari para pembunuh, para ibu dari pecandu obat-obatan, para ibu dari para teroris, para ibu dari para pemerkosa, para ibu dari para penderita sakit jiwa: tetapi ibu-ibu semuanya sama! Namun air mata tidaklah cukup. Air mata harus membanjir menjadi cinta yang mengasuh, kekuatan yang memberikan arah, ketegasan yang mengoreksi, dialog yang membangun, sebuah kehadiran yang memercik! Air mata harus mencegah air mata lainnya! }

\textbf{DOA}


\BU{Tuhan Yesus, Engkau tahu benar air mata setiap ibu, Engkau melihat di setiap sudut rumah suara kesakitan, Engkau mendengar tangisan sunyi dari banyak ibu yang disakiti oleh anak-anak mereka: menahan luka-luka yang mematikan ... namun yang masih hidup! Tuhan Yesus, larutkanlah kebekuan dari tak berperasaan yang mencegah cinta untuk bersikulasi dalam pembuluh-pembuluh nadi keluarga-keluarga kami. Buatlah kami, sekali lagi, sadar menjadi anak-anak, sehingga kami dapat memberikan para ibu kami - di bumi dan di surga - kebanggaan untuk telah melahirkan kami, dan kegembiraan dalam memberkati hari kelahiran kami. Tuhan Yesus, usaplah air mata dari semua ibu, sehingga sebuah senyuman dapat kembali kepada wajah-wajah anak-anak mereka, kepada wajah semuanya, Amin. }


\kasihanilahKami

\lagu{9}{Tobatkanlah jiwa kami,\\
arahkanlah sikap hati,\\
pada cinta sejati.}

\henti{IX}{Yesus jatuh ketiga kalinya}

\BPi{ Kitab Nabi Habakuk 1:12-13; 2:2-3 }{
	Bukankah Engkau, ya TUHAN, dari dahulu Allahku, Yang Mahakudus? Mata-Mu terlalu suci untuk melihat kejahatan, dan Engkau tidak dapat memandang kelaliman. Mengapa Engkau memandangi orang-orang yang berbuat khianat itu, dan Engkau berdiam diri, apabila orang fasik menelan orang yang lebih benar dari dia? ''Tuliskanlah penglihatan itu, dan ukirkanlah itu pada loh-loh, supaya orang sambil lalu dapat membacanya. Sebab penglihatan itu masih menanti saatnya, tetapi ia bersegera menuju kesudahannya dengan tidak menipu; apabila berlambat-lambat, nantikanlah itu, sebab itu sungguh-sungguh akan datang dan tidak akan bertangguh. }

\BPii{
	Saat Paskah diperingati: ``Yesus akan berada dalam penderitaan sampai akhir jaman selama manusia tetap jatuh dalam dosa; dan kita tidak dapat tidur selama waktu ini''.
	Di manakah Yesus dalam penderitaan pada masa kita? Dalam pemecahan dunia kita ke dalam sabuk kemakmuran dan sabuk kemiskinan ... inilah penderitaan Kristus saat ini. Dunia kita dibagi dalam dua ruangan: dalam satu ruangan, semua hal diboroskan, dalam ruangan yang lainnya, orang-orang terbuang, dalam satu ruangan, orang-orang mati akibat kekenyangan, dalam ruangan lainnya, mereka mati akibat kemiskinan; dalam satu ruangan, mereka prihatin akan kegemukan, dalam ruangan lainnya, mereka meminta-minta derma. Mengapa kita tidak membuka sebuah pintu? Mengapa kita tidak duduk di satu meja? Mengapa kita tidak menyadari bahwa yang miskin dapat menolong yang kaya? Mengapa? Mengapa? Mengapa kita begitu buta? }

\textbf{DOA}

\BU{Tuhan Yesus, mereka yang hidup dalam timbunan kekayaan adalah mereka yang Kau katakan sebagai yang bodoh!  Ya, mereka yang berpikir mereka memiliki semuanya sungguh-sungguh bodoh, karena hanya ada satu Pemilik dunia ini. Tuhan Yesus, dunia adalah milikMu dan hanya milikMu saja. Namun demikian Engkau telah memberikannya kepada setiap orang sehingga bumi dapat menjadi sebuah rumah di mana semua menemukan makanan dan perlindungan. Jadi timbunan kekayaan adalah pencurian, jika penimbunannya yang sia-sia mencegah orang lain untuk hidup. Tuhan Yesus, tempatkanlah sebuah akhir dari skandal yang memecah belah dunia ke dalam benteng-benteng dan perkampungan miskin. Tuhan, ajarkanlah kami sekali lagi arti dari persaudaraan, Amin. }

\kasihanilahKami

\lagu{10}{Bila hatiku gelisah\\
Kar'na dosa atau susah,\\
ulurkanlah tangan-Mu.}

\henti{X}{	Pakaian Yesus ditanggalkan}

\BPi{	Injil Yohanes 19:23-24 }{
	Sesudah prajurit-prajurit itu menyalibkan Yesus, mereka mengambil pakaian-Nya lalu membaginya menjadi empat bagian untuk tiap-tiap prajurit satu bagian--dan jubah-Nya juga mereka ambil. Jubah itu tidak berjahit, dari atas ke bawah hanya satu tenunan saja. Karena itu mereka berkata seorang kepada yang lain: ``Janganlah kita membaginya menjadi beberapa potong, tetapi baiklah kita membuang undi untuk menentukan siapa yang mendapatnya.'' Demikianlah hendaknya supaya genaplah yang ada tertulis dalam Kitab Suci: ``Mereka membagi-bagi pakaian-Ku di antara mereka dan mereka membuang undi atas jubah-Ku.'' }

\BPii{
	Para parajurit mengambil jubah Yesus daripadaNya dengan brutalitas para pencuri; mereka juga mencoba merampokNya dari kesederhanaan dan martabatnya. Namun Yesus adalah kesederhanaan, Yesus adalah sang martabat yang menjaid milik manusia dan tubuh manusia. Dan tubuh Kristus yang dicemooh menjadi tuduhan dari semua cemooh yang pernah ditujukan kepada tubuh manusia, yang Allah ciptakan sebagai cermin dari jiwa dan bahasa untuk berbicara tentang cinta. Hari ini tubuh-tubuh terus menerus dijual-belikan di jalan-jalan dari kota-kota kami, di jalan-jalan dari televisi kami, di rumah-rumah yang telah menjadi seperti jalan-jalan. Kapankah kita akan sadar bahwa kita membunuh cinta? Kapankah kita akan sadar bahwa tanpa kesucian, tubuh tidak akan pernah menjadi hidup atau pemberi hidup? }

\textbf{DOA}

\BU{Tuhan Yesus, kesucian di mana-mana telah menjadi korban dari sebuah konspirasi yang sudah diperhitungkan dari kesunyian: sebuah kesunyian tak murni! Orang-orang bahkan telah menjadi percaya sebuah kebohongan lengkap: bahwa kesucian bagaimanapun adalah musuh dari cinta. Namun kebalikannya adalah benar, Oh Tuhan! Kesucian adalah perlu sebagai sebuah syarat bagi cinta: sebuah cinta yang sejati, sebuah cinta yang setia. Dalam peristiwa apapun, Tuhan, jika kami tidak mampu menjadi tuan bagi diri kami sendiri? Bagaimana kami dapat memberikan diri kami bagi orang lain? Hanya yang murni yang mampu mencintai; hanya yang murni dapat mencintai tanpa merendahkan cinta. Tuhan Yesus, dengan kekuatan dari darahMu yang tertumpah dalam cinta, berikanlah kami hati yang murni, sehingga dunia kami dapat melihat sebuah kelahiran kembali dari cinta, sebuah cinta yang mana hati kami sangat rindukan, Amin.}



\kasihanilahKami

\lagu{11}{PakaianMu dibagikan,\\
jubah utuh diundikan,\\
martabatMu dihina.}

\henti{XI}{	Yesus dipaku di kayu salib}

\BPi{	Injil Matius 27:35-42 }{
	Sesudah menyalibkan Dia mereka membagi-bagi pakaian-Nya dengan membuang undi. Lalu mereka duduk di situ menjaga Dia. Dan di atas kepala-Nya terpasang tulisan yang menyebut alasan mengapa Ia dihukum: ``Inilah Yesus Raja orang Yahudi.'' Bersama dengan Dia disalibkan dua orang penyamun, seorang di sebelah kanan dan seorang di sebelah kiri-Nya. Orang-orang yang lewat di sana menghujat Dia dan sambil menggelengkan kepala, mereka berkata: ``Hai Engkau yang mau merubuhkan Bait Suci dan mau membangunnya kembali dalam tiga hari, selamatkanlah diri-Mu jikalau Engkau Anak Allah, turunlah dari salib itu!'' Demikian juga imam-imam kepala bersama-sama ahli-ahli Taurat dan tua-tua mengolok-olokkan Dia dan mereka berkata: ``Orang lain Ia selamatkan, tetapi diri-Nya sendiri tidak dapat Ia selamatkan! Ia Raja Israel? Baiklah Ia turun dari salib itu dan kami akan percaya kepada-Nya. }

\BPii{
	Tangan yang memberkati setiap orang sekarang dipaku pada salib; kaki yang berjalan ke mana saja, membawa harapan dan cinta, sekarang terikat pada kayu. Mengapa, Oh Tuhan? Karena cinta!  Mengapa SengsaraMu? Karena cinta! Mengapa salibMu? Karena cinta! Tuhan, mengapa Engkau tidak turun dari salib, untuk menjawab ejekan kami? \textbf{\emph{Aku tidak turun dari salib, karena jika ya Aku akan membuat kekuasaan tuan dari dunia, dimana cinta saja adalah kuasa yang dapat merubah dunia.}} Mengapa Tuhan? Engkau membayar harga yang mengerikan ini? \textbf{\emph{Untuk mengatakan kepada mu bahwa Allah adalah Cinta, Cinta abadi, semua yang sempurna. Apakah kamu mempercayai Aku?}}}

\textbf{DOA}

\BU{Yesus, Tuhan yang disalibkan, semua orang telah menipu Engkau, bebas tanpa kendali telah memperdayai MU, Engkau sendiri tidak pernah akan memperdaya kami, Engkau membiarkan paku menancap ditangan Mu dan membiarkan kami dengan brutal memakunya kepada kayu salib, sebagai caraMu untuk menyampaikan kepada kami bahwa cinta Mu benar, tulus hati, tidak dapat dibatalkan dan setia. 

Yesus, Tuhan yang disalibkan, mata kami menatap tangan Mu yang tetembus dengan paku, namun kami tidak mampu berbuat apa-apa untuk menghalangi mereka, kami justru menambahkan jumlah paku itu dengan berbuat dosa dan kesalahan yang sama, mereka menatap kaki Mu, memakunya kepada salib, Yesus, Tuhan yang disalibkan, gambaran suatu kebahagiaan terlepas dari harapan, Tuhan sudah mati. 

Kami ingin kembali kepada Mu, merasakan kebebasan dan harapan melalui jalan salib kami masing-masing, merasakan kebenaran dan kegembiraan pada setiap jalan salib yang harus kami jalani: Yesus, Tuhan yang disalibkan, dampingilah kami agar mampu menjalani jalan salib kami didunia ini, Amin.}


\kasihanilahKami

\lagu{12}{Dari salibMu Kaulihat,\\
tak terbilang yang menghujat,\\
berapakah yang setia?}
\henti{XII}{	Yesus wafat di Salib}

\BPi{	Injil Yohanes 19:25-27 }{
	Dan dekat salib Yesus berdiri ibu-Nya dan saudara ibu-Nya, Maria, isteri Klopas dan Maria Magdalena. Ketika Yesus melihat ibu-Nya dan murid yang dikasihi-Nya di sampingnya, berkatalah Ia kepada ibu-Nya: ``Ibu, inilah, anakmu!'' Kemudian kata-Nya kepada murid-murid-Nya: ``Inilah ibumu!''. Dan sejak saat itu murid itu menerima dia di dalam rumahnya. 

\textbf{\emph{	Injil Matius 27:45-46, 50}}

	Mulai dari jam dua belas kegelapan meliputi seluruh daerah itu sampai jam tiga. Kira-kira jam tiga berserulah Yesus dengan suara nyaring: ``Eli, Eli, lama sabakhtani?'', Artinya: ``Allah-Ku, Allah-Ku, mengapa Engkau meninggalkan Aku?'' Yesus berseru pula dengan suara nyaring lalu menyerahkan nyawa-Nya. }\textbf{(hening)}

\BPii{
	Orang-orang berpikir: Allah mati! Tetapi jika Allah mati, siapa yang masih memberikan kita hidup? Jika Allah mati, apakah hidup itu sendiri? Hidup adalah Kasih! Jadi salib bukanlah kematian Kristus, namun saat ketika kulit rapuh dari kemanusiaan yang diambil oleh Allah dipecahkan dan sebuah darah kasih mengucur segera untuk membaharui semua kemanusiaan. 
\\
	Dari salib dilahirkan hidup baru Saul, dari salib dilahirkan pertobatan Agustinus, dari salib dilahirkan kemiskinan yang menggembirakan dari Fransiskus Asisi, dari salib dilahirkan kebaikan yang bersinar dari Vincent de Paul; dari salib dilahirkan kepahlawanan dari Maximilianus Kolbe, dari salib dilahirkan amal menakjubkan dari Bunda Teresa dari Kalkuta, dari salib dilahirkan keberanian dari Yohanes Paulus II, dari salib dilahirkan revolusi kasih: Jadi salib bukanlah kematian Allah, melainkan kelahiran dari KasihNya di dunia kita. Terberkatilah Salib Kristus! }

\textbf{DOA}

\BU{Tuhan Yesus, dalam keheningan ruangan Gereja Santa Maria pagi ini, suaraMu terdengar: ``Aku haus! Aku haus akan kasihmu!''  Dalam keheningan pagi ini juga, doamu terdengar: ``Ya Bapa, ampunilah mereka! Ya Bapa, ampunilah mereka!''  Dalam keheningan sejarah, tangisanMu terdengar: ``Sudah selesai''. Apa yang selesai? ``Aku telah memberikanmu segalanya, Aku telah mengatakan segalanya kepadamu, Aku membawa kepadamu pesan yang terindah dari semua: Allah adalah Cinta! Allah mencintaimu!'' Dalam keheningan hati, kami dapat merasakan sentuhan dari hadiahMu yang terakhir: ``Inilah ibumu: ibuKu!''  Terima kasih Yesus, karena memberikan Bunda Maria misi yang mengingatkan kami setiap hari bahwa arti dari semuanya adalah untuk ditemukan dalam cinta: Cinta Allah yang ditanamkan di dunia seperti sebuah salib! Terima kasih, Yesus, Amin. }

\kasihanilahKami

\lagu{13}{Benih yang mati hasilkan,\\
buah yang berkelimpahan,\\
wafatMu: sumber hidup.}

\henti{XIII}{	Jenazah Yesus diturunkan dan diberikan kepada IbuNya}

\BPi{	Injil Matius 27:55, 57-58; 17:22-23 }{
	Dan ada di situ banyak perempuan yang melihat dari jauh, yaitu perempuan-perempuan yang mengikuti Yesus dari Galilea untuk melayani Dia.Menjelang malam datanglah seorang kaya, orang Arimatea, yang bernama Yusuf dan yang telah menjadi murid Yesus juga. Ia pergi menghadap Pilatus dan meminta mayat Yesus. Pilatus memerintahkan untuk menyerahkannya kepadanya. Pada waktu Yesus dan murid-murid-Nya bersama-sama di Galilea, Ia berkata kepada mereka: ''Anak Manusia akan diserahkan ke dalam tangan manusia, dan mereka akan membunuh Dia dan pada hari ketiga Ia akan dibangkitkan.`` Maka hati murid-murid-Nya itupun sedih sekali. }

\BPii{Perbuatan telah dilakukan: kita telah membunuh Yesus!  Dan luka-luka Kristus terus menyengat di hati Maria, sebagai satu penderitaan yang membungkus baik Ibu dan Anak. Penderitaan Ibu dan Anaknya! Adegan itu berteriak kepada kita, ia membawa kepedihan dan rasa sakit bahkan kepada mereka yang biasa menimbulkan kepedihan kepada orang lain. Kita hampir kelihatan merasa kasihan untuk Allah namun -- sekali lagi - adalah Allah yang merasa kasihan untuk kita. Rasa sakit kita tidak lagi tanpa harapan atau tidak akan pernah tanpa harapan lagi, karena Allah telah datang untuk menderita bersama kita. Dan bersama Allah, dapatkah kita menjadi tanpa harapan? }

\textbf{	DOA}

\BU{Oh, Maria, dalam Putramu engkau merangkul setiap putra dan putri, dan berbagi kesedihan mendalam dari setiap ibu di seluruh dunia. Oh Maria, air matamu terus menetes di setiap usia; mereka memandikan wajah-wajah dan cermin kedukaan dari setiap laki-laki dan wanita. Oh Maria, engkau telah mengenal kepedihan ... namun engkau tetap percaya! Engkau percaya bahwa awan tidak menggelapkan matahari, engkau percaya bahwa malam memberikan jalan menuju pagi. Oh Maria, engkau yang menyanyikan Magnificat, memimpin kami dalam lagu yang mengalahkan kepedihan seperti kelahiran yang datang tiba-tiba yang membawa segera kehidupan baru. Oh Maria, doakanlah kami! Doakanlah agar kami juga boleh mengalami \\kekuatan yang menularkan dari harapan sejati, Amin. }

\kasihanilahKami

\lagu{14}{Salib tanda penghinaan,\\
jadi lambang kemenangan,\\
lantaran wafat Yesus.}

\henti{XIV}{	Jenazah Yesus dimakamkan}

\BPi{Injil Matius 27:59-61 }{
	Dan Yusufpun mengambil mayat itu, mengapaninya dengan kain lenan yang putih bersih, lalu membaringkannya di dalam kuburnya yang baru, yang digalinya di dalam bukit batu, dan sesudah menggulingkan sebuah batu besar ke pintu kubur itu, pergilah ia. Tetapi Maria Magdalena dan Maria yang lain tinggal di situ duduk di depan kubur itu. Dari kitab Mazmur 16:9-11 Sebab itu hatiku bersukacita dan jiwaku bersorak-sorak, bahkan tubuhku akan diam dengan tenteram; sebab Engkau tidak menyerahkan aku ke dunia orang mati, dan tidak membiarkan Orang Kudus-Mu melihat kebinasaan. Engkau memberitahukan kepadaku jalan kehidupan; di hadapan-Mu ada sukacita berlimpah-limpah, di tangan kanan-Mu ada nikmat senantiasa. }

\BPii{
	Ada saat di mana hidup tampak seperti sebuah Sabtu Suci yang panjang dan suram. Semuanya tampak berlalu, yang culas kelihatan menang, dan yang jahat tampak lebih kuat daripada yang baik. Tetapi iman membuat kita menatap ke depan, iman membuat kita memandang datangnya hari baru pada sisi lain dari hari ini. Iman menjanjikan kita kata terakhir yang adalah milik Allah: milik Allah saja! Iman adalah sungguh-sungguh sebuah lampu kecil, namun itu satu-satunya lampu yang dapat menyinari malamnya dunia: dan cahayanya yang kecil menyatu Dengan cahaya dari sebuah hari baru: hari dari Kebangkitan Kristus. Jadi kisahnya tidak berakhir dengan makam, melainkan menyembul keluar dari makam: seperti yang dijanjikan Yesus kepada kita,  itu terjadi, dan akan terjadi lagi}

\textbf{DOA}

\BU{Tuhan Yesus, Jumat Agung adalah hari kegelapan, hari dari kebencian membuta, hari ketika Yang Adil dimatikan! Tetapi Jumat Agung bukanlah kata terakhir: kata terakhir adalah Paskah, kejayaan dari Hidup, kemenangan dari Kebaikan atas Kejahatan. Tuhan Yesus, Sabtu Suci adalah hari yang kosong, hari dari kebingungan dan ketakutan, hari saat segalanya tampak berakhir! Namun Sabtu Suci bukanlah hari terakhir, hari terakhir adalah Paskah, Cahaya yang diperbaharui, Cinta yang menguasai segala kebencian. Tuhan Yesus, setiap kali kami mengalami Jumat Agung kami sendiri, dan kami merasa kegelisahan Sabtu Suci, berikanlah kami iman yang tak redup dari Maria, sehingga kami dapat percaya dalam kenyataan Paskah; berikanlah kami pandangannya matanya yang jernih sehingga kami dapat menatap pagi yang cemerlang yang mengumumkan hari terakhir dari sejarah: ``Surga yang baru dan sebuah bumi yang baru'' yang sudah ada di dalam diriMu, Yesus, yang Disalibkan dan Bangkit. Amin! }

\kasihanilahKami

\lagu{15}{Nama Yesus dimuliakan,\\
kurban salib dikenangkan,\\
untuk s'lama lamanya.}

\BP{\textbf{Penutup}\\Walaupun dalam rupa Allah, Kristus Yesus tidak menganggap kesetaraan dengan Allah itu sebagai milik yang harus dipertahankan. Sebaliknya, Ia mengosongkan diri-Nya, mengambil rupa hamba dan menjadi sama dengan manusia. Dan dalam keadaan sebagai manusia, Ia merendahkan diri dan taat sampai mati, bahkan sampai mati di kayu salib. Itulah sebabnya Allah amat meninggikan Dia dan mengaruniakan kepada-Nya nama diatas segala nama, supaya dalam nama Yesus bertekuklah segala yang ada di langit, yang ada diatas bumi dan yang ada di bawah bumi, dan segala lidah mengaku; Yesus Kristus adalah Tuhan, untuk kemuliaan Allah Bapa.

	Marilah berdoa,

	Allah Mahapengasih, kami bersyukur karena dapat mengenangkan Yesus yang sengsara dan wafat demi keselamatan kami. Limpahkanlah berkat-Mu atas kami yang mengharapkan kebangkitan bersama Dia. Semoga karena berkat-Mu, kami bertumbuh dalam iman dan keyakinan akan kebahagiaan abadi. Demi Kristus Tuhan kami. Amin}


Salam Maria \ldots{} 3$\times$


Bapa Kami\ldots{}


Kemulian\ldots{}


Terpujilah\ldots{}


Dalam Nama Bapa, Putera dan Roh Kudus, Amin.

\end{document}
