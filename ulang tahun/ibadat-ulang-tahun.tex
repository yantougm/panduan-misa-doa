% This file was converted to LaTeX by Writer2LaTeX ver. 1.4
% see http://writer2latex.sourceforge.net for more info
\documentclass{article}
\usepackage[ascii]{inputenc}
\usepackage[T1]{fontenc}
\usepackage{amsmath}
\usepackage{amssymb,amsfonts,textcomp}
\usepackage{array}
\usepackage{hhline}
\title{}
\author{}
\date{2018-05-16}
\begin{document}
Wednesday, January 30, 2008

\section[Ibadat Ulang Tahun ]{Ibadat Ulang Tahun }
DOA SYUKURAN Lingkungan Antonius 4 Rabu, 27 Juni 2007 SUNATAN Sdr James Chandra Putra Baptisan Kel Bp Chandra Doa
Dipimpin oleh : Ibu GREDYANA RUMLUS  

\section{PEMBUKA}
 Nyanyian : 

Salam dan Berkat : 

P : {\dag} Dalam nama Bapa dan Putera dan Roh Kudus, Amien{\dots}{\dots}.. Tuhan beserta kita  

U : Sekarang dan selama lamanya 

P : Bapak / Ibu dan Saudara (i) yang terkasih dalam Kristus. Bagi otrang beriman, peristiwa syukuran hendaknya dilihat
sebagai momen rasa syukur atas penyelenggaraan Allah. Bahwa Allah masih memberikan kesempatan kepada kita untuk selalu
memperbaiki dan menyempurnaan hidup. Soal panjang usia, keberhasilan hidup memang menjadi harapan kita semua, namun
kita serahkan saja kepada-Nya. Kebijaksanaan Allah sendirilah yang akan menentukan. Dan kita percaya bahwa Allah selalu
memberikan yang terbaik bagi umat-Nya 

P : Sebelum kita melanjutkan doa pada malam ini, marilah kita memeriksa hati dan menyesali kesalahan serta dosa kita
serta sejenak merenungkan perjalanan hidup kita selama ini. Kita sesali segala dosa yang menghambat perjalanan kita,
agar pantas merayakan ibadat syukur ini di hadapan-Nya ( {\dots} hening sejenak ) Saya mengaku kepada Allah yang
mahakuasa dan kepada Saudara sekalian
{\dots}{\dots}{\dots}{\dots}{\dots}{\dots}{\dots}{\dots}{\dots}{\dots}{\dots}{\dots}{\dots}{\dots}{\dots}.. 

\subsection[DOA PEMBUKA]{DOA PEMBUKA }
P: Marilah kita berioa ( hening sejenak ) 

Ya, Allah Bapa yang mahabaik, saat-saat ini Saudara kami {\dots}{\dots}{\dots}{\dots}.{\dots} masih dalam suasana
merayakan hari ulang tahunnya. Kami bersyukur dan berterima kasih karena Engkau telah menganugerahkan keselamatan
kepada saudara -- saudara kami itu. Dengan bertambahnya usia, semoga saudara kami semakin bertambah pandai, setia dan
taat kepada orang tua dan gurunya, tekun berdoa, jujur dan rendah hati. Dengan iman kami yakin bahwa Engkau berkenan
terhadap saudara kami yang hatinya masih suci, ceria, namun bersahaja. Curahkan Roh-Mu atas saudara kami, jagalah pula
kesehatan jiwa dan raganya, agar senantiasa dapat memenuhi panggilan hidupnya , agar semakin dewasa dalam setiap
langkah yang ditempuhnya, lebih-lebih kehidupan imannya. Berkatilah saudara-saudara kami dan seluruh anggota keluarga
lainnya, sanak saudara yang ada di sekitarnya, agar tercipta suasana rukun dan damai Demi Kristus, PutraMu, Tuhan dan
Pengantara kami 

U: Amin  

\section[SABDA]{SABDA }
 Nyanyian : Bacaan Injil :  

\subsection[GAGASAN HOMILI]{GAGASAN HOMILI }
 1 Ada kalanya, orang hanya memandang sebelah mata keberadaan seorang anak kecil. Orang merasa cukup hanya dengan
memenuhi kebutuhan -- kebutuhan hidupnya, itu pun menurut seleranya. Selebihnya, seperti tidak ada tempat bagi
anak-anak kecil. Namun bagi Yesus, anak kecil justru mendapatkan peranan penting dalam ajaran-Nya. Keterbukaan,
kepolosan, kesederhaanan dan kerendahan hati seorang anak kecil menjadi suri teladan kita. Yesus menyuruh para
murid-Nya untuk bersikap seperti anak kecil, kalau ingin menjadi yang terbesar  2 Saat -- saat ini dalam suasana Ulang
Tahun, hendaknya dijadikan kesempatan bagi kita, khususnya para orang tua, untuk merenungkan bagaimana seharusnya
membimbing dan membina setiap anak kita agar selalu taat kepada Allah, setia dan patuh pada orang tua dan bersahabat
dengan sesamanya. Sekaligus kita mengambil hikmah dari keberadaannya yang masih suci, lugu dan selalu gembira untuk
menyikapi segala peristiwa kehidupan ini  3 Nasihat Santo Paulus kepada umat di Efesus juga menekankan agar mereka
sungguh -- sungguh memperhatikan anak-anaknya dan mendidik mereka menurut ajaran Tuhan. Hubungan antara anak dan
orangtua, orang tua anak, hubungan antara hamba dan tuannya maupun sebaliknya, juga mendapatkan perhatian Allah. Antara
anak dan orang tua, antara hamba dan tuannya, meskipun berbeda, tetapi kedua-duanya adalah hamba-hamba Allah yang
tunduk di bawah cinta kasih, yang saling menerima dan memberi serta saling menghargai. Dalam hal ini Allah tidak
memandang muka  4 Kita menyadari betapa besar cinta kasih Tuhan kepada umat-Nya. Maka pada kesempatan ini, kita patut
bersyukur dan memuji Allah yang telah menganugrahkan segala kebutuhan hidup ini kepada kita terutama kepada saudara
kecil kita .. yang masih dalam suasana peringatan hari kelahirannya. Semoga seluruh hidup kita menjadi pancaran cinta
kasih Allah 

\subsection{SYAHADAT}
 P : Marilah kita menguatkan dan memperbaharui iman kita dengan mengucapkan Doa Syahadat .. yang singkat U : .. Aku
percaya akan Allah, Bapa yang mahakuasa, Pencipta langit dan bumi {\dots}{\dots}{\dots}{\dots}{\dots}{\dots}{\dots}.  

\subsection{DOA UMAT}
P : Bapak / ibu, Saudar ( i ), marilah kita panjatkan puji syukur dan permohonan kita kepada Allah yang mahabaik karena
telah mencurahkan berkatnya bagi kita semua, terutama atas saudara -- saudara kami yang pada saat ini masih dalam
suasana perayaan ulang tahunnya  ( dipersilahkan memanjatkan Doa masing masing dimulai dari Saudara yang berulangtahun,
kemudian orang tua .. selanjutnya Umat yang lain )  

P : Bagi terselenggaranya ibadat syukur ini Kita bersyukur kepada Tuhan atas rahmat-Nya pada hari yang baik ini,
sehingga kita boleh berkumpul dan bergembira di sini bersama saudara kami yang berbahagia. Semoga hari-hari hidup kita
selanjutnya dipenuhi oleh sukacita yang sama dari Tuhan sendiri Kami mohon :
{\dots}{\dots}{\dots}{\dots}{\dots}{\dots}..  

P : Bagi kebagiaan dan kerukunan dalam hidup berkeluarga Semoga Tuhan senantiasa menyertai keluarga ini dan memenuhi
mereka dengan kebahagiaan dan kesuksesan dalam setiap gerak langkah perjuangan hidup mereka sehari -- hari. Semoga
tercapai suasana saling menghormati dan menghargai dalam keluarga kita masing-masing, agar kita lebih peka mendengarkan
suara Tuhan yang menyapa kita lewat kata-kata dan perbuatan anggota keluarga kita Kami mohon :
{\dots}{\dots}{\dots}{\dots}{\dots}{\dots}.  

P : Untuk Perdamaian dunia Ya Bapa .. Sudilah berkati seluruh pemimpin dunia bersama masyarakat di dunia, curahkanlah
roh-Mu bagi para pemimpin bangsa agar didalam memimpin bangsa -- bangsa di dunia mereka selalu mengusahakan perdamaian
antar bangsa. Kami Mohon {\dots}{\dots}{\dots}{\dots}{\dots}{\dots}{\dots}.  

P : Untuk orang yang merindukan keselamatan melalui Yesus Ya Bapa {\dots} bukalah jalan bagi setiap orang yang
merindukan kasih Yesus namun menghadapi berbagai tantangan sudilah kiranya tuntun mereka untuk menemukan jalan
mengikuti Yesus. Kami mohon {\dots}{\dots}{\dots}{\dots}.{\dots}..  

P : Bapa yang mahabaik, hanya Engkaulah yang pantas dipuji dan dimuliakan bersama Putra dan Roh Kudus, hidup dan
bertakhta, sepanjang segala masa.  

U : Amien  

P : Marilah kita sempurnakan semua doa kita tadi dalam doa yang diajarkan oleh Yesus sendiri Bapa kami
{\dots}{\dots}{\dots}{\dots}{\dots}{\dots}{\dots}..   

\subsection{UCAPAN SYUKUR}
  Nyanyian Syukur atau nyanyian Selamat Ulang Tahun, Penyalaan Lilin, Potong Kue dll{\dots}.  

\subsection{DOA PENUTUP}
  P : Marilah berdoa Ya Allah, Bapa kami yang mahakasih. Semoga peristiwa perayaan ulang tahun selalu mengingatkan kami
untuk semakin menyadari arti hidup ini dan setiap kali berusaha menghayati dengan lebih baik dari hari sebelumnya. Kami
percaya, walaupun banyak rintangan yang menghadang, Engkau tidak akan merelakan hamba-Mu ini jatuh ke dalam lembah
nista. Oleh karena itu, berkatilah kami, terutama saudara kami {\dots}{\dots}. Agar tetap tegar dalam membela kebenaran
sebagai saksi Putra-Mu, Yesus Kritus, Tuhan dan Pengantara Kami  

U : Amien  

P : Bapak / Ibu dan Saudara ( i ) yang terkasih dalam Kristus, marilah kita mohon berkat sebelum menutup Ibadat syukur
kali ini  Semoga kita semua diberkati {\dag} Dalam nama Bapa dan Putera dan Roh Kudus, Amien{\dots}{\dots}..   

\subsection{Nyanyian Penutup }
\end{document}
