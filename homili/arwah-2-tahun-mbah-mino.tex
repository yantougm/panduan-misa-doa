\documentclass[12pt,a4paper]{article}
\usepackage[latin1]{inputenc}
\usepackage{amsmath}
\usepackage{amsfonts}
\usepackage{amssymb}
\usepackage[width=142.50mm, height=210.00mm, left=10.00mm, right=10.00mm, top=10.00mm, bottom=10.00mm]{geometry}
\author{Yohanes Suyanto}
\title{Homili 2 tahun mBah Mino}
\date{2 Maret 2015}
\begin{document}
\maketitle
\section*{Pambuka}
Bapak, Ibu, lan sadherek sadaya, dalu punika kita ngempal wonten mriki saperlu nyenyuwunaken sukmanipun Bapak Petrus Samino ingkang sampun 2 tahun kepengker katimbalan dening Gusti. Kita boten badhe nambah-nambah kasisahan ananging ing kalodhangan punika saged kita angge sarana nambah semangat kita amrih kapitadosan kita tansah mindhak-mindhak. 

Kita pidatos bilih Gusti sampun prajanji badhe maringi papan ingkang mulya ing kalanggengan, 
Kita ingkang taksih gesang tansah nyenyuwun amrih para sedherek ingkang sampun tinimbalan saged ngaso kanthi tentrem lan mulya ing kalanggengan. 

Kita tansah dedonga amrih semangat kita lan ugi semangat kulawarga, pangajeng-ajeng kita lan ugi pengajeng-ajeng kulawarga tetep teguh tumuju dhumateng Gusti.

\section*{Renungan}
Dereng dangu televisi tansah nyiaraken kacilakan montor muluk Air Asia. Kathah kurban ingkang pejah lan malah ngantos sapriki wonten ingkang dereng kepanggih kuwandhanipun.

Kados pundi raos kita menawi sadherek kita wonten ingkang dados kurban? Sedhih temtu kemawon.


Babagan lair lan pejah kaserat wonten kitab Juru Khotbah bilih samubarang wonten wayahipun.
Tangis lan gumujeng ugi wonten wayahipun.
Menawi dipunpasangaken tetembungan kalawau
lair-tangis, pejah-gumujeng.
Wekdal bayi lair, jabang bayi nangis lan tiyang-tiyang sanes gumujeng.
Wekdal tiyang pejah, tiyang-tiyang sami nangis nanging tetiyang ingkang pejah malah gumujeng.

Pepejah ateges wiwitan gesang langgeng ingkang kebak ing sukagambira. Pramila ingkang katilar dipun suwun boten sanget-sanget anggenipun sedhih, ananging sageda nglajengaken gesang miturut karsa Dalem. Gesang kita ing donya namung ngisi wekdal saking lair dumugi pejah, nanging gesang langgeng boten wonten watesipun.
Kados pundi kita sedaya ngisi wekdal wonten ing ndonya? 

Gusti sampun dhawuh "... dipadha tresna tinresnan ...". Gusti sampun prajanji bilih kabingahan sejati bdhe kita tampi menawi kita pinanggih lan manunggil ing Gusti. Kita saged pinanggih Gusti kanthi mirunggan menawi kita sembahyang, ngibadah. Manunggil kita kaliyang Sang Kristus boten namung ing salebeting keyakinan iman kita ing Panjenenganipun, utawi margi kita sregep sembahyang dhateng Sang Kristus, ananging ugi amargi tindak katresnan kita ingkang kita tindakaken dhateng sesami. 

\section*{Panutup}
Bapak Petrus Samino sampun kawilujengaken Gusti, lan temtu panjenenganipun langkung rena menawi kulawarga ingkar tinilar tetep semangat mandhep mantep anggayuh kasaenan dhateng sesami.

Gusti Yesus, ingkang wungu mulya kapisanan sampun prajanji boten namung kangge mBah Mino ananging kita sadaya badhe kaparingan swarga langgeng. Pramila sampun samesthinipun kita nuhoni dhawuh Dalem kanthi tansah tresna tinresnan dhateg sesami. Mugi-mugi Gusti tansah maringi semangat pangajeng-ajeng lan kapitadosan. 
Gusti ngandika "Iki dhawuhku, dipadha tresna tinresnan".
Amin.

\end{document}

Samubarang kabeh iku ana wayahe
3:1-15
1 Samubarang kabeh iku ana wayahe, apa
bae ing sangisore langit iki ana wayahe.
2 Ana wayahe lair, ana wayahe mati, ana
wayahe nenandur, ana wayahe mbedholi kang
ditandur. 3 Ana wayahe mateni, ana wayahe
marasake; ana wayahe mbubrah, ana wayahe
mbangun. 4 Ana wayahe nangis, ana wayahe
ngguyu; ana wayahe sesambat, ana wayahe
jejogedan. 5 Ana wayahe mbuwang watu,
ana wayahe nglumpukake watu; ana wayahe
ngrangkul, ana wayahe cegah ngrangkul. 6 Ana
wayahe nggoleki, ana wayahe ngeklasake
ilanging barang; ana wayahe nyimpen, ana
wayahe mbuwang. 7 Ana wayahe nyuwek, ana
wayahe ndondomi; ana wayahe meneng, ana
wayahe caturan. 8 Ana wayahe nresnani, ana
wayahe nyengiti; ana wayahe perang, ana
wayahe rukun. 9 Apa ta pituwase wong kang
nindakake pagawean kang dilakoni kanthi
kangelan? 10 Aku wus ndeleng pagawean
kang diparingake dening Gusti Allah marang
para anaking manungsa, supaya padha
gawea keseling awake. 11 Samubarang kabeh
katitahake endah ing wayah kang wus
kapesthekake, malah atine padha kaparingan
kalanggengan, nanging manungsa ora bisa
nyumurupi pakaryaning Allah wiwit wiwitan
nganti wekasan. 12 Aku sumurup manawa
tumraping para manungsa iku ora ana
kang becike ngluwihi seneng-seneng lan
ngematake kasenengan ana ing sajroning
uripe. 13 Dene saben wong bisa mangan,
ngombe lan ngrasakake kasenengan ing
sajroning kangelane kabeh, iku uga peparinge
Gusti Allah.

14 Aku sumurup manawa samubarang kabeh
kang katindakake dening Gusti Allah iku bakal
tetep anane ing salawas-lawase; iku ora
bisa diwuwuhi utawa dikurangi; Gusti Allah
tumindak mangkono, supaya manungsa wedia
marang Panjenengane. 15 Kang saiki ana, iku
biyen wus ana, lan kang bakal ana iku wus
lawas anane; sarta Gusti Allah ngupadosi apa
kang wus kapungkur.
